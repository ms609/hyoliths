\documentclass[]{book}
\usepackage{lmodern}
\usepackage{amssymb,amsmath}
\usepackage{ifxetex,ifluatex}
\usepackage{fixltx2e} % provides \textsubscript
\ifnum 0\ifxetex 1\fi\ifluatex 1\fi=0 % if pdftex
  \usepackage[T1]{fontenc}
  \usepackage[utf8]{inputenc}
\else % if luatex or xelatex
  \ifxetex
    \usepackage{mathspec}
  \else
    \usepackage{fontspec}
  \fi
  \defaultfontfeatures{Ligatures=TeX,Scale=MatchLowercase}
\fi
% use upquote if available, for straight quotes in verbatim environments
\IfFileExists{upquote.sty}{\usepackage{upquote}}{}
% use microtype if available
\IfFileExists{microtype.sty}{%
\usepackage{microtype}
\UseMicrotypeSet[protrusion]{basicmath} % disable protrusion for tt fonts
}{}
\usepackage[margin=1in]{geometry}
\usepackage{hyperref}
\hypersetup{unicode=true,
            pdftitle={Brachiopod origins - Supplementary material - Phylogenetic analysis},
            pdfauthor={Sun, Hai-Jing; Smith, Martin Ross; Zhu, Mao-Yan; Zeng, Han; Zhao, Fang-Chen},
            pdfborder={0 0 0},
            breaklinks=true}
\urlstyle{same}  % don't use monospace font for urls
\usepackage{natbib}
\bibliographystyle{plainnat}
\usepackage{color}
\usepackage{fancyvrb}
\newcommand{\VerbBar}{|}
\newcommand{\VERB}{\Verb[commandchars=\\\{\}]}
\DefineVerbatimEnvironment{Highlighting}{Verbatim}{commandchars=\\\{\}}
% Add ',fontsize=\small' for more characters per line
\usepackage{framed}
\definecolor{shadecolor}{RGB}{248,248,248}
\newenvironment{Shaded}{\begin{snugshade}}{\end{snugshade}}
\newcommand{\KeywordTok}[1]{\textcolor[rgb]{0.13,0.29,0.53}{\textbf{#1}}}
\newcommand{\DataTypeTok}[1]{\textcolor[rgb]{0.13,0.29,0.53}{#1}}
\newcommand{\DecValTok}[1]{\textcolor[rgb]{0.00,0.00,0.81}{#1}}
\newcommand{\BaseNTok}[1]{\textcolor[rgb]{0.00,0.00,0.81}{#1}}
\newcommand{\FloatTok}[1]{\textcolor[rgb]{0.00,0.00,0.81}{#1}}
\newcommand{\ConstantTok}[1]{\textcolor[rgb]{0.00,0.00,0.00}{#1}}
\newcommand{\CharTok}[1]{\textcolor[rgb]{0.31,0.60,0.02}{#1}}
\newcommand{\SpecialCharTok}[1]{\textcolor[rgb]{0.00,0.00,0.00}{#1}}
\newcommand{\StringTok}[1]{\textcolor[rgb]{0.31,0.60,0.02}{#1}}
\newcommand{\VerbatimStringTok}[1]{\textcolor[rgb]{0.31,0.60,0.02}{#1}}
\newcommand{\SpecialStringTok}[1]{\textcolor[rgb]{0.31,0.60,0.02}{#1}}
\newcommand{\ImportTok}[1]{#1}
\newcommand{\CommentTok}[1]{\textcolor[rgb]{0.56,0.35,0.01}{\textit{#1}}}
\newcommand{\DocumentationTok}[1]{\textcolor[rgb]{0.56,0.35,0.01}{\textbf{\textit{#1}}}}
\newcommand{\AnnotationTok}[1]{\textcolor[rgb]{0.56,0.35,0.01}{\textbf{\textit{#1}}}}
\newcommand{\CommentVarTok}[1]{\textcolor[rgb]{0.56,0.35,0.01}{\textbf{\textit{#1}}}}
\newcommand{\OtherTok}[1]{\textcolor[rgb]{0.56,0.35,0.01}{#1}}
\newcommand{\FunctionTok}[1]{\textcolor[rgb]{0.00,0.00,0.00}{#1}}
\newcommand{\VariableTok}[1]{\textcolor[rgb]{0.00,0.00,0.00}{#1}}
\newcommand{\ControlFlowTok}[1]{\textcolor[rgb]{0.13,0.29,0.53}{\textbf{#1}}}
\newcommand{\OperatorTok}[1]{\textcolor[rgb]{0.81,0.36,0.00}{\textbf{#1}}}
\newcommand{\BuiltInTok}[1]{#1}
\newcommand{\ExtensionTok}[1]{#1}
\newcommand{\PreprocessorTok}[1]{\textcolor[rgb]{0.56,0.35,0.01}{\textit{#1}}}
\newcommand{\AttributeTok}[1]{\textcolor[rgb]{0.77,0.63,0.00}{#1}}
\newcommand{\RegionMarkerTok}[1]{#1}
\newcommand{\InformationTok}[1]{\textcolor[rgb]{0.56,0.35,0.01}{\textbf{\textit{#1}}}}
\newcommand{\WarningTok}[1]{\textcolor[rgb]{0.56,0.35,0.01}{\textbf{\textit{#1}}}}
\newcommand{\AlertTok}[1]{\textcolor[rgb]{0.94,0.16,0.16}{#1}}
\newcommand{\ErrorTok}[1]{\textcolor[rgb]{0.64,0.00,0.00}{\textbf{#1}}}
\newcommand{\NormalTok}[1]{#1}
\usepackage{longtable,booktabs}
\usepackage{graphicx,grffile}
\makeatletter
\def\maxwidth{\ifdim\Gin@nat@width>\linewidth\linewidth\else\Gin@nat@width\fi}
\def\maxheight{\ifdim\Gin@nat@height>\textheight\textheight\else\Gin@nat@height\fi}
\makeatother
% Scale images if necessary, so that they will not overflow the page
% margins by default, and it is still possible to overwrite the defaults
% using explicit options in \includegraphics[width, height, ...]{}
\setkeys{Gin}{width=\maxwidth,height=\maxheight,keepaspectratio}
\IfFileExists{parskip.sty}{%
\usepackage{parskip}
}{% else
\setlength{\parindent}{0pt}
\setlength{\parskip}{6pt plus 2pt minus 1pt}
}
\setlength{\emergencystretch}{3em}  % prevent overfull lines
\providecommand{\tightlist}{%
  \setlength{\itemsep}{0pt}\setlength{\parskip}{0pt}}
\setcounter{secnumdepth}{5}
% Redefines (sub)paragraphs to behave more like sections
\ifx\paragraph\undefined\else
\let\oldparagraph\paragraph
\renewcommand{\paragraph}[1]{\oldparagraph{#1}\mbox{}}
\fi
\ifx\subparagraph\undefined\else
\let\oldsubparagraph\subparagraph
\renewcommand{\subparagraph}[1]{\oldsubparagraph{#1}\mbox{}}
\fi

%%% Use protect on footnotes to avoid problems with footnotes in titles
\let\rmarkdownfootnote\footnote%
\def\footnote{\protect\rmarkdownfootnote}

%%% Change title format to be more compact
\usepackage{titling}

% Create subtitle command for use in maketitle
\newcommand{\subtitle}[1]{
  \posttitle{
    \begin{center}\large#1\end{center}
    }
}

\setlength{\droptitle}{-2em}
  \title{Brachiopod origins - Supplementary material - Phylogenetic analysis}
  \pretitle{\vspace{\droptitle}\centering\huge}
  \posttitle{\par}
  \author{Sun, Hai-Jing; Smith, Martin Ross; Zhu, Mao-Yan; Zeng, Han; Zhao,
Fang-Chen}
  \preauthor{\centering\large\emph}
  \postauthor{\par}
  \predate{\centering\large\emph}
  \postdate{\par}
  \date{2018-03-12}

\usepackage{booktabs}

\usepackage{amsthm}
\newtheorem{theorem}{Theorem}[chapter]
\newtheorem{lemma}{Lemma}[chapter]
\theoremstyle{definition}
\newtheorem{definition}{Definition}[chapter]
\newtheorem{corollary}{Corollary}[chapter]
\newtheorem{proposition}{Proposition}[chapter]
\theoremstyle{definition}
\newtheorem{example}{Example}[chapter]
\theoremstyle{definition}
\newtheorem{exercise}{Exercise}[chapter]
\theoremstyle{remark}
\newtheorem*{remark}{Remark}
\newtheorem*{solution}{Solution}
\begin{document}
\maketitle

{
\setcounter{tocdepth}{1}
\tableofcontents
}
\hypertarget{brachiopod-origins}{%
\chapter*{Brachiopod origins}\label{brachiopod-origins}}
\addcontentsline{toc}{chapter}{Brachiopod origins}

This document provides a detailed discussion of analyses of the
\protect\hyperlink{dataset}{morphological dataset} constructed to
accompany Sun \emph{et al.} \citeyearpar{Sun2018}, and their results.

We first discuss the results presented in the main paper, which employ
the algorithm described by Brazeau, Guillerme and Smith
\citeyearpar{Brazeau2018} for correct handling of inapplicable data in a
parsimony setting, and explore how each character is
\protect\hyperlink{reconstructions}{reconstructed} on an optimal tree.

For completeness, we also document the results of
\protect\hyperlink{tnt}{standard Fitch parsimony} analysis, and the
results of \protect\hyperlink{MrBayes}{Bayesian analysis}, neither of
which treat inapplicable data in a logically consistent fashion.

\hypertarget{the-dataset}{%
\chapter{The dataset}\label{the-dataset}}

Analysis was performed on a new matrix of 35 early brachiozoan taxa,
including hyoliths, tommotiids and mickwitziids, which were coded for 95
morphological characters (42 neomorphic, 53 transformational).

The dataset can be viewed and downloaded at Morphobank
(\href{https://morphobank.org/permalink/?P2800}{project 2800}), where
each character is defined and its coding for each taxon discussed.

Characters are coded following the recommendations of Brazeau, Guillerme
and Smith \citep{Brazeau2018}. In brief, we have employed reductive
coding, using a distinct state to mark character inapplicability.
Character specifications follow the model of Sereno
\citeyearpar{Sereno2007}.

We have distinguished between neomorphic and transformational characters
\citep[sensu][]{Sereno2007} by reserving the token \texttt{0} to refer
to the absence of a neomorphic character. The states of transformational
characters are represented by the tokens \texttt{1}, \texttt{2},
\texttt{3}, \ldots{}

Following the recommendations of Brazeau, Guillerme and Smith
\citep[supplementary discussion]{Brazeau2018}, we code the absence of
neomorphic ontologically dependent characters \citep[sensu][]{Vogt2017}
as absence, rather than inapplicability.

\hypertarget{treesearch}{%
\chapter{Parsimony analysis}\label{treesearch}}

The phylogenetic dataset contains a high proportion of inapplicable
codings (404/ 3325 = 12\% of tokens), which are known to introduce error
and bias to phylogenetic reconstruction
(\citep{Maddison1993, Brazeau2018}). As such, phylogenetic search
employed a new algorithm that correctly handles inapplicable data
\citep{Brazeau2018}. This algorithm is implemented in the
\emph{MorphyLib} C library \citep{Brazeau2017Morphylib}, and
phylogenetic search was conducted using the \emph{R} package
\emph{TreeSearch} v0.0.8 \citep{Smith2018TreeSearch}.

\emph{Namacalathus} is included in the matrix but has been excluded from
analysis due to its potentially long branch, which is likely to mislead
analysis.

\hypertarget{search-parameters}{%
\section{Search parameters}\label{search-parameters}}

Heuristic searches were conducted using the parsimony ratchet
\citep{Nixon1999} under equal and implied weights \citep{Goloboff1997}
with a variety of concavity constants. The consensus tree presented in
the main manuscript represents a strict consensus of all trees that are
most parsimonious under one or more of the concavity constants
(\emph{k}) 2, 3, 4.5, 7, 10.5, 16 and 24, an approach that is known to
produce higher accuracy than equal weights at any fixed level of
precision \citep{Smith2017}.

\hypertarget{analysis}{%
\section{Analysis}\label{analysis}}

\hypertarget{load-data}{%
\subsection{Load data}\label{load-data}}

\hypertarget{generate-starting-tree}{%
\subsection{Generate starting tree}\label{generate-starting-tree}}

\emph{Dailyatia} is used as an outgroup.

\begin{Shaded}
\begin{Highlighting}[]
\NormalTok{nj.tree <-}\StringTok{ }\KeywordTok{NJTree}\NormalTok{(my_data)}
\NormalTok{rooted.tree <-}\StringTok{ }\KeywordTok{EnforceOutgroup}\NormalTok{(nj.tree, }\StringTok{'Dailyatia'}\NormalTok{)}
\NormalTok{start.tree <-}\StringTok{ }\KeywordTok{TreeSearch}\NormalTok{(}\DataTypeTok{tree=}\NormalTok{rooted.tree, }\DataTypeTok{dataset=}\NormalTok{my_data, }\DataTypeTok{maxIter=}\DecValTok{3000}\NormalTok{,}
                         \DataTypeTok{EdgeSwapper=}\NormalTok{RootedNNISwap, }\DataTypeTok{verbosity=}\DecValTok{0}\NormalTok{)}
\end{Highlighting}
\end{Shaded}

\hypertarget{implied-weights-analysis}{%
\subsection{Implied weights analysis}\label{implied-weights-analysis}}

\begin{Shaded}
\begin{Highlighting}[]
\ControlFlowTok{for}\NormalTok{ (k }\ControlFlowTok{in}\NormalTok{ kValues) \{}
\NormalTok{  iw.tree <-}\StringTok{ }\KeywordTok{IWRatchet}\NormalTok{(start.tree, iw_data, }\DataTypeTok{concavity=}\NormalTok{k,}
                       \DataTypeTok{ratchHits =} \DecValTok{60}\NormalTok{, }\DataTypeTok{searchHits=}\DecValTok{55}\NormalTok{,}
                       \DataTypeTok{swappers=}\KeywordTok{list}\NormalTok{(RootedTBRSwap, RootedSPRSwap, RootedNNISwap),}
                       \DataTypeTok{verbosity=}\DecValTok{0}\NormalTok{)}
\NormalTok{  score <-}\StringTok{ }\KeywordTok{IWScore}\NormalTok{(iw.tree, iw_data, }\DataTypeTok{concavity=}\NormalTok{k)}
  \CommentTok{# Write single best tree}
  \KeywordTok{write.nexus}\NormalTok{(iw.tree, }\DataTypeTok{file=}\KeywordTok{paste0}\NormalTok{(}\StringTok{"TreeSearch/hy_iw_k"}\NormalTok{, k, }\StringTok{"_"}\NormalTok{, }\KeywordTok{signif}\NormalTok{(score, }\DecValTok{5}\NormalTok{), }\StringTok{".nex"}\NormalTok{, }\DataTypeTok{collapse=}\StringTok{''}\NormalTok{))}

\NormalTok{  suboptFraction =}\StringTok{ }\FloatTok{0.02}
\NormalTok{  iw.consensus <-}\StringTok{ }\KeywordTok{IWRatchetConsensus}\NormalTok{(iw.tree, iw_data, }\DataTypeTok{concavity=}\NormalTok{k,}
                  \DataTypeTok{swappers=}\KeywordTok{list}\NormalTok{(RootedTBRSwap, RootedNNISwap),}
                  \DataTypeTok{searchHits=}\DecValTok{4}\NormalTok{,}
                  \DataTypeTok{suboptimal=}\NormalTok{score }\OperatorTok{*}\StringTok{ }\NormalTok{suboptFraction,}
                  \DataTypeTok{nSearch=}\DecValTok{150}\NormalTok{, }\DataTypeTok{verbosity=}\NormalTok{0L)}
  \KeywordTok{write.nexus}\NormalTok{(iw.consensus, }\DataTypeTok{file=}\KeywordTok{paste0}\NormalTok{(}\StringTok{"TreeSearch/hy_iw_k"}\NormalTok{, k, }\StringTok{"_"}\NormalTok{, }\KeywordTok{signif}\NormalTok{(}\KeywordTok{IWScore}\NormalTok{(iw.tree, iw_data, }\DataTypeTok{concavity=}\NormalTok{k), }\DecValTok{3}\NormalTok{), }\StringTok{".all.nex"}\NormalTok{, }\DataTypeTok{collapse=}\StringTok{''}\NormalTok{))}
\NormalTok{\}}
\end{Highlighting}
\end{Shaded}

\hypertarget{equal-weights-analysis}{%
\subsection{Equal weights analysis}\label{equal-weights-analysis}}

\begin{Shaded}
\begin{Highlighting}[]
\NormalTok{ew.tree <-}\StringTok{ }\KeywordTok{Ratchet}\NormalTok{(start.tree, my_data, }\DataTypeTok{verbosity=}\NormalTok{0L,}
                   \DataTypeTok{ratchHits =} \DecValTok{10}\NormalTok{, }\DataTypeTok{searchHits=}\DecValTok{55}\NormalTok{,}
                   \DataTypeTok{swappers=}\KeywordTok{list}\NormalTok{(RootedTBRSwap, RootedSPRSwap, RootedNNISwap))}
\KeywordTok{write.nexus}\NormalTok{(best.tree, }\DataTypeTok{file=}\KeywordTok{paste0}\NormalTok{(}\StringTok{"TreeSearch/hy_ew_"}\NormalTok{, }\KeywordTok{Fitch}\NormalTok{(ew.tree, my_data), }\StringTok{".nex"}\NormalTok{, }\DataTypeTok{collapse=}\StringTok{''}\NormalTok{))}

\NormalTok{ew.consensus <-}\StringTok{ }\KeywordTok{RatchetConsensus}\NormalTok{(ew.tree, my_data, }\DataTypeTok{nSearch=}\DecValTok{150}\NormalTok{,}
                                 \DataTypeTok{swappers=}\KeywordTok{list}\NormalTok{(RootedTBRSwap, RootedNNISwap),}
                                 \DataTypeTok{verbosity=}\NormalTok{0L)}
\KeywordTok{write.nexus}\NormalTok{(ew.consensus, }\DataTypeTok{file=}\KeywordTok{paste0}\NormalTok{(}\StringTok{"TreeSearch/hy_ew_"}\NormalTok{, }\KeywordTok{Fitch}\NormalTok{(ew.tree, my_data), }\StringTok{".nex"}\NormalTok{, }\DataTypeTok{collapse=}\StringTok{''}\NormalTok{))}
\end{Highlighting}
\end{Shaded}

\hypertarget{results}{%
\section{Results}\label{results}}

\hypertarget{implied-weights-results}{%
\subsection{Implied weights results}\label{implied-weights-results}}

\begin{Shaded}
\begin{Highlighting}[]
\CommentTok{# Read results from files}
\NormalTok{iw.trees <-}\StringTok{ }\KeywordTok{lapply}\NormalTok{(kValues, }\ControlFlowTok{function}\NormalTok{ (k) \{}
\NormalTok{  iw.best <-}\StringTok{ }\KeywordTok{list.files}\NormalTok{(}\StringTok{'TreeSearch'}\NormalTok{, }
                      \DataTypeTok{pattern=}\KeywordTok{paste0}\NormalTok{(}\StringTok{'hy_iw_k'}\NormalTok{,}
                                     \KeywordTok{gsub}\NormalTok{(}\StringTok{'}\CharTok{\textbackslash{}\textbackslash{}}\StringTok{.'}\NormalTok{, }\StringTok{'}\CharTok{\textbackslash{}\textbackslash{}\textbackslash{}\textbackslash{}}\StringTok{.'}\NormalTok{, k),}
                                     \StringTok{'_}\CharTok{\textbackslash{}\textbackslash{}}\StringTok{d+}\CharTok{\textbackslash{}\textbackslash{}}\StringTok{.?}\CharTok{\textbackslash{}\textbackslash{}}\StringTok{d*}\CharTok{\textbackslash{}\textbackslash{}}\StringTok{.all}\CharTok{\textbackslash{}\textbackslash{}}\StringTok{.nex'}\NormalTok{),}
                      \DataTypeTok{full.names=}\OtherTok{TRUE}\NormalTok{)}
  \CommentTok{# Return:}
  \ControlFlowTok{if}\NormalTok{ (}\KeywordTok{length}\NormalTok{(iw.best) }\OperatorTok{==}\StringTok{ }\DecValTok{0}\NormalTok{) \{}
    \KeywordTok{list}\NormalTok{()}
\NormalTok{  \} }\ControlFlowTok{else}\NormalTok{ \{}
    \KeywordTok{read.nexus}\NormalTok{(iw.best[}\KeywordTok{which.max}\NormalTok{(}\KeywordTok{file.mtime}\NormalTok{(iw.best))])}
\NormalTok{  \}}
\NormalTok{\})}
\end{Highlighting}
\end{Shaded}

\begin{Shaded}
\begin{Highlighting}[]
\NormalTok{omit=}\KeywordTok{c}\NormalTok{(lon, }\StringTok{'Clupeafumosus_socialis'}\NormalTok{,}
       \StringTok{'Heliomedusa_orienta'}\NormalTok{, }\StringTok{'Haplophrentis_carinatus'}\NormalTok{)}
\KeywordTok{ColPlot}\NormalTok{(}\KeywordTok{ConsensusWithout}\NormalTok{(}\KeywordTok{lapply}\NormalTok{(iw.trees, consensus), omit))}
\KeywordTok{ColMissing}\NormalTok{(omit)}
\end{Highlighting}
\end{Shaded}

\begin{figure}
\centering
\includegraphics{Brachiopod_phylogeny_files/figure-latex/unnamed-chunk-1-1.pdf}
\caption{\label{fig:unnamed-chunk-1}Consensus of implied weights analyses at
all values of k}
\end{figure}

\begin{Shaded}
\begin{Highlighting}[]
\CommentTok{# Plot consensus results}
\KeywordTok{par}\NormalTok{(}\DataTypeTok{mfrow=}\KeywordTok{c}\NormalTok{(}\DecValTok{4}\NormalTok{, }\DecValTok{2}\NormalTok{), }\DataTypeTok{mar=}\KeywordTok{rep}\NormalTok{(}\FloatTok{0.2}\NormalTok{, }\DecValTok{4}\NormalTok{))}

\KeywordTok{ColPlot}\NormalTok{(}\KeywordTok{consensus}\NormalTok{(}\KeywordTok{lapply}\NormalTok{(iw.trees, consensus)))}
\KeywordTok{text}\NormalTok{(}\OperatorTok{-}\FloatTok{0.5}\NormalTok{, }\DecValTok{1}\NormalTok{, }\DataTypeTok{pos=}\DecValTok{4}\NormalTok{, }\StringTok{"Consensus of all k values"}\NormalTok{, }\DataTypeTok{cex=}\FloatTok{0.8}\NormalTok{)}

\CommentTok{# Plot results for each value of k}
\ControlFlowTok{for}\NormalTok{ (i }\ControlFlowTok{in} \KeywordTok{seq_along}\NormalTok{(iw.trees)) \{}
  \KeywordTok{ColPlot}\NormalTok{(}\KeywordTok{consensus}\NormalTok{(iw.trees[[i]]))}
  \KeywordTok{text}\NormalTok{(}\DecValTok{1}\NormalTok{, }\DecValTok{1}\NormalTok{, }\KeywordTok{paste0}\NormalTok{(}\StringTok{'k = '}\NormalTok{, kValues[i]), }\DataTypeTok{pos=}\DecValTok{4}\NormalTok{)}
\NormalTok{\}}
\end{Highlighting}
\end{Shaded}

\begin{figure}
\centering
\includegraphics{Brachiopod_phylogeny_files/figure-latex/unnamed-chunk-2-1.pdf}
\caption{\label{fig:unnamed-chunk-2}Implied weights results}
\end{figure}

\hypertarget{equal-weights-results}{%
\subsection{Equal weights results}\label{equal-weights-results}}

\begin{Shaded}
\begin{Highlighting}[]
\NormalTok{ew.best <-}\StringTok{ }\KeywordTok{list.files}\NormalTok{(}\StringTok{'TreeSearch'}\NormalTok{, }\DataTypeTok{pattern=}\StringTok{'hy_ew_}\CharTok{\textbackslash{}\textbackslash{}}\StringTok{d*}\CharTok{\textbackslash{}\textbackslash{}}\StringTok{.nex'}\NormalTok{, }\DataTypeTok{full.names=}\OtherTok{TRUE}\NormalTok{)}
\NormalTok{ew.tree <-}\StringTok{ }\KeywordTok{read.nexus}\NormalTok{(}\DataTypeTok{file=}\NormalTok{ew.best[}\KeywordTok{which.max}\NormalTok{(}\KeywordTok{file.mtime}\NormalTok{(ew.best))])}
\KeywordTok{ColPlot}\NormalTok{(}\KeywordTok{consensus}\NormalTok{(ew.tree))}
\end{Highlighting}
\end{Shaded}

\begin{figure}
\centering
\includegraphics{Brachiopod_phylogeny_files/figure-latex/equal weights results in TreeSearch-1.pdf}
\caption{(\#fig:equal weights results in TreeSearch)Strict consensus of
equal weights results}
\end{figure}

\begin{Shaded}
\begin{Highlighting}[]
\NormalTok{omit <-}\StringTok{ }\KeywordTok{c}\NormalTok{(lon)}
\KeywordTok{ColPlot}\NormalTok{(}\KeywordTok{ConsensusWithout}\NormalTok{(ew.tree, omit))}
\KeywordTok{ColMissing}\NormalTok{(omit)}
\end{Highlighting}
\end{Shaded}

\begin{figure}
\centering
\includegraphics{Brachiopod_phylogeny_files/figure-latex/unnamed-chunk-3-1.pdf}
\caption{\label{fig:unnamed-chunk-3}Strict consensus of equal weights
results, taxa excluded}
\end{figure}

\hypertarget{reconstructions}{%
\chapter{Character reconstructions}\label{reconstructions}}

Here's how each character maps onto one of the most parsimonious trees
(obtained under implied weighting, \(k = 4.5\)):

\includegraphics{Brachiopod_phylogeny_files/figure-latex/unnamed-chunk-4-1.pdf}
\includegraphics{Brachiopod_phylogeny_files/figure-latex/unnamed-chunk-4-2.pdf}
\includegraphics{Brachiopod_phylogeny_files/figure-latex/unnamed-chunk-4-3.pdf}
\includegraphics{Brachiopod_phylogeny_files/figure-latex/unnamed-chunk-4-4.pdf}
\includegraphics{Brachiopod_phylogeny_files/figure-latex/unnamed-chunk-4-5.pdf}
\includegraphics{Brachiopod_phylogeny_files/figure-latex/unnamed-chunk-4-6.pdf}
\includegraphics{Brachiopod_phylogeny_files/figure-latex/unnamed-chunk-4-7.pdf}
\includegraphics{Brachiopod_phylogeny_files/figure-latex/unnamed-chunk-4-8.pdf}
\includegraphics{Brachiopod_phylogeny_files/figure-latex/unnamed-chunk-4-9.pdf}
\includegraphics{Brachiopod_phylogeny_files/figure-latex/unnamed-chunk-4-10.pdf}
\includegraphics{Brachiopod_phylogeny_files/figure-latex/unnamed-chunk-4-11.pdf}
\includegraphics{Brachiopod_phylogeny_files/figure-latex/unnamed-chunk-4-12.pdf}
\includegraphics{Brachiopod_phylogeny_files/figure-latex/unnamed-chunk-4-13.pdf}
\includegraphics{Brachiopod_phylogeny_files/figure-latex/unnamed-chunk-4-14.pdf}
\includegraphics{Brachiopod_phylogeny_files/figure-latex/unnamed-chunk-4-15.pdf}
\includegraphics{Brachiopod_phylogeny_files/figure-latex/unnamed-chunk-4-16.pdf}
\includegraphics{Brachiopod_phylogeny_files/figure-latex/unnamed-chunk-4-17.pdf}
\includegraphics{Brachiopod_phylogeny_files/figure-latex/unnamed-chunk-4-18.pdf}
\includegraphics{Brachiopod_phylogeny_files/figure-latex/unnamed-chunk-4-19.pdf}
\includegraphics{Brachiopod_phylogeny_files/figure-latex/unnamed-chunk-4-20.pdf}
\includegraphics{Brachiopod_phylogeny_files/figure-latex/unnamed-chunk-4-21.pdf}
\includegraphics{Brachiopod_phylogeny_files/figure-latex/unnamed-chunk-4-22.pdf}
\includegraphics{Brachiopod_phylogeny_files/figure-latex/unnamed-chunk-4-23.pdf}
\includegraphics{Brachiopod_phylogeny_files/figure-latex/unnamed-chunk-4-24.pdf}
\includegraphics{Brachiopod_phylogeny_files/figure-latex/unnamed-chunk-4-25.pdf}
\includegraphics{Brachiopod_phylogeny_files/figure-latex/unnamed-chunk-4-26.pdf}
\includegraphics{Brachiopod_phylogeny_files/figure-latex/unnamed-chunk-4-27.pdf}
\includegraphics{Brachiopod_phylogeny_files/figure-latex/unnamed-chunk-4-28.pdf}
\includegraphics{Brachiopod_phylogeny_files/figure-latex/unnamed-chunk-4-29.pdf}
\includegraphics{Brachiopod_phylogeny_files/figure-latex/unnamed-chunk-4-30.pdf}
\includegraphics{Brachiopod_phylogeny_files/figure-latex/unnamed-chunk-4-31.pdf}
\includegraphics{Brachiopod_phylogeny_files/figure-latex/unnamed-chunk-4-32.pdf}
\includegraphics{Brachiopod_phylogeny_files/figure-latex/unnamed-chunk-4-33.pdf}
\includegraphics{Brachiopod_phylogeny_files/figure-latex/unnamed-chunk-4-34.pdf}
\includegraphics{Brachiopod_phylogeny_files/figure-latex/unnamed-chunk-4-35.pdf}
\includegraphics{Brachiopod_phylogeny_files/figure-latex/unnamed-chunk-4-36.pdf}
\includegraphics{Brachiopod_phylogeny_files/figure-latex/unnamed-chunk-4-37.pdf}
\includegraphics{Brachiopod_phylogeny_files/figure-latex/unnamed-chunk-4-38.pdf}
\includegraphics{Brachiopod_phylogeny_files/figure-latex/unnamed-chunk-4-39.pdf}
\includegraphics{Brachiopod_phylogeny_files/figure-latex/unnamed-chunk-4-40.pdf}
\includegraphics{Brachiopod_phylogeny_files/figure-latex/unnamed-chunk-4-41.pdf}
\includegraphics{Brachiopod_phylogeny_files/figure-latex/unnamed-chunk-4-42.pdf}
\includegraphics{Brachiopod_phylogeny_files/figure-latex/unnamed-chunk-4-43.pdf}
\includegraphics{Brachiopod_phylogeny_files/figure-latex/unnamed-chunk-4-44.pdf}
\includegraphics{Brachiopod_phylogeny_files/figure-latex/unnamed-chunk-4-45.pdf}
\includegraphics{Brachiopod_phylogeny_files/figure-latex/unnamed-chunk-4-46.pdf}
\includegraphics{Brachiopod_phylogeny_files/figure-latex/unnamed-chunk-4-47.pdf}
\includegraphics{Brachiopod_phylogeny_files/figure-latex/unnamed-chunk-4-48.pdf}
\includegraphics{Brachiopod_phylogeny_files/figure-latex/unnamed-chunk-4-49.pdf}
\includegraphics{Brachiopod_phylogeny_files/figure-latex/unnamed-chunk-4-50.pdf}
\includegraphics{Brachiopod_phylogeny_files/figure-latex/unnamed-chunk-4-51.pdf}
\includegraphics{Brachiopod_phylogeny_files/figure-latex/unnamed-chunk-4-52.pdf}
\includegraphics{Brachiopod_phylogeny_files/figure-latex/unnamed-chunk-4-53.pdf}
\includegraphics{Brachiopod_phylogeny_files/figure-latex/unnamed-chunk-4-54.pdf}
\includegraphics{Brachiopod_phylogeny_files/figure-latex/unnamed-chunk-4-55.pdf}
\includegraphics{Brachiopod_phylogeny_files/figure-latex/unnamed-chunk-4-56.pdf}
\includegraphics{Brachiopod_phylogeny_files/figure-latex/unnamed-chunk-4-57.pdf}
\includegraphics{Brachiopod_phylogeny_files/figure-latex/unnamed-chunk-4-58.pdf}
\includegraphics{Brachiopod_phylogeny_files/figure-latex/unnamed-chunk-4-59.pdf}
\includegraphics{Brachiopod_phylogeny_files/figure-latex/unnamed-chunk-4-60.pdf}
\includegraphics{Brachiopod_phylogeny_files/figure-latex/unnamed-chunk-4-61.pdf}
\includegraphics{Brachiopod_phylogeny_files/figure-latex/unnamed-chunk-4-62.pdf}
\includegraphics{Brachiopod_phylogeny_files/figure-latex/unnamed-chunk-4-63.pdf}
\includegraphics{Brachiopod_phylogeny_files/figure-latex/unnamed-chunk-4-64.pdf}
\includegraphics{Brachiopod_phylogeny_files/figure-latex/unnamed-chunk-4-65.pdf}
\includegraphics{Brachiopod_phylogeny_files/figure-latex/unnamed-chunk-4-66.pdf}
\includegraphics{Brachiopod_phylogeny_files/figure-latex/unnamed-chunk-4-67.pdf}
\includegraphics{Brachiopod_phylogeny_files/figure-latex/unnamed-chunk-4-68.pdf}
\includegraphics{Brachiopod_phylogeny_files/figure-latex/unnamed-chunk-4-69.pdf}
\includegraphics{Brachiopod_phylogeny_files/figure-latex/unnamed-chunk-4-70.pdf}
\includegraphics{Brachiopod_phylogeny_files/figure-latex/unnamed-chunk-4-71.pdf}
\includegraphics{Brachiopod_phylogeny_files/figure-latex/unnamed-chunk-4-72.pdf}
\includegraphics{Brachiopod_phylogeny_files/figure-latex/unnamed-chunk-4-73.pdf}
\includegraphics{Brachiopod_phylogeny_files/figure-latex/unnamed-chunk-4-74.pdf}
\includegraphics{Brachiopod_phylogeny_files/figure-latex/unnamed-chunk-4-75.pdf}
\includegraphics{Brachiopod_phylogeny_files/figure-latex/unnamed-chunk-4-76.pdf}
\includegraphics{Brachiopod_phylogeny_files/figure-latex/unnamed-chunk-4-77.pdf}
\includegraphics{Brachiopod_phylogeny_files/figure-latex/unnamed-chunk-4-78.pdf}
\includegraphics{Brachiopod_phylogeny_files/figure-latex/unnamed-chunk-4-79.pdf}
\includegraphics{Brachiopod_phylogeny_files/figure-latex/unnamed-chunk-4-80.pdf}
\includegraphics{Brachiopod_phylogeny_files/figure-latex/unnamed-chunk-4-81.pdf}
\includegraphics{Brachiopod_phylogeny_files/figure-latex/unnamed-chunk-4-82.pdf}
\includegraphics{Brachiopod_phylogeny_files/figure-latex/unnamed-chunk-4-83.pdf}
\includegraphics{Brachiopod_phylogeny_files/figure-latex/unnamed-chunk-4-84.pdf}
\includegraphics{Brachiopod_phylogeny_files/figure-latex/unnamed-chunk-4-85.pdf}
\includegraphics{Brachiopod_phylogeny_files/figure-latex/unnamed-chunk-4-86.pdf}
\includegraphics{Brachiopod_phylogeny_files/figure-latex/unnamed-chunk-4-87.pdf}
\includegraphics{Brachiopod_phylogeny_files/figure-latex/unnamed-chunk-4-88.pdf}
\includegraphics{Brachiopod_phylogeny_files/figure-latex/unnamed-chunk-4-89.pdf}
\includegraphics{Brachiopod_phylogeny_files/figure-latex/unnamed-chunk-4-90.pdf}
\includegraphics{Brachiopod_phylogeny_files/figure-latex/unnamed-chunk-4-91.pdf}
\includegraphics{Brachiopod_phylogeny_files/figure-latex/unnamed-chunk-4-92.pdf}
\includegraphics{Brachiopod_phylogeny_files/figure-latex/unnamed-chunk-4-93.pdf}
\includegraphics{Brachiopod_phylogeny_files/figure-latex/unnamed-chunk-4-94.pdf}
\includegraphics{Brachiopod_phylogeny_files/figure-latex/unnamed-chunk-4-95.pdf}

These reconstructions were created using the \emph{Inapp} \emph{R}
package \citep{Brazeau2018}.

Full character definitions can be found by browsing the
\protect\hyperlink{dataset}{morphological dataset} on MorphoBank
(\href{https://morphobank.org/permalink/?P2800}{project 2800}).

\hypertarget{tnt}{%
\chapter{Fitch parsimony}\label{tnt}}

Parsimony search was conducted in TNT v1.5 \citep{Goloboff2016} using
sectorial and ratchet heuristics under equal and implied weights. We
acknowledge the Willi Hennig Society for their sponsorship of the TNT
software.

\hypertarget{implied-weights}{%
\section{Implied weights}\label{implied-weights}}

The consensus of all implied weights runs is not very well resolved:

\begin{figure}
\centering
\includegraphics{Brachiopod_phylogeny_files/figure-latex/unnamed-chunk-6-1.pdf}
\caption{\label{fig:unnamed-chunk-6}TNT implied weights consensus}
\end{figure}

This lack of resolution is largely a product of a few wildcard taxa,
which obscure relationships that are nevertheless present in all most
parsimonious trees:

\hypertarget{paterinids-included}{%
\subsection{Paterinids included}\label{paterinids-included}}

\begin{figure}
\centering
\includegraphics{Brachiopod_phylogeny_files/figure-latex/unnamed-chunk-7-1.pdf}
\caption{\label{fig:unnamed-chunk-7}TNT implied weights consensus}
\end{figure}

\hypertarget{paterinids-excluded}{%
\subsection{Paterinids excluded}\label{paterinids-excluded}}

\begin{figure}
\centering
\includegraphics{Brachiopod_phylogeny_files/figure-latex/unnamed-chunk-8-1.pdf}
\caption{\label{fig:unnamed-chunk-8}TNT implied weights consensus}
\end{figure}

\hypertarget{equal-weights}{%
\section{Equal weights}\label{equal-weights}}

\begin{figure}
\centering
\includegraphics{Brachiopod_phylogeny_files/figure-latex/unnamed-chunk-9-1.pdf}
\caption{\label{fig:unnamed-chunk-9}TNT Equal weights consensus}
\end{figure}

\hypertarget{bayesian-analysis}{%
\chapter{Bayesian analysis}\label{bayesian-analysis}}

Bayesian search was conducted in MrBayes v3.2.6 \citep{Ronquist2012}
using the Mk model \citep{Lewis2001} with a gamma parameter:

\begin{quote}
lset coding=variable rates=gamma;
\end{quote}

Branch length was drawn from a dirichlet prior distribution, which is
less informative than an exponential model \citep{Rannala2012}, but
requires a prior mean tree length within about two orders of magnitude
of the true value \citep{Zhang2012}. To satisfy this latter criterion,
we specified the prior mean tree length to be equal to the length of the
most parsimonious tree under equal weights, usinga Dirichlet prior with
\(α_T = 1\), \(β_T = 1/\)(\emph{equal weights tree length} /
\emph{number of characters}), \(α = c = 1\):

\begin{quote}
prset brlenspr = unconstrained: gammadir(1, 0.33, 1, 1);
\end{quote}

Neomorphic and transformational characters \citep[sensu][]{Sereno2007}
were allocated to two separate partitions whose proportion of invariant
characters and gamma shape parameters were allowed to vary
independently:

\begin{quote}
charset Neomorphic = 2 3 5 10 13 14 18 19 24 25 29 30 32 34 36 37 38 39
40 43 50 51 53 54 55 60 63 64 67 70 72 73 76 77 78 79 84 88 89 92 94 95;

charset Transformational = 1 4 6 7 8 9 11 12 15 16 17 20 21 22 23 26 27
28 31 33 35 41 42 44 45 46 47 48 49 52 56 57 58 59 61 62 65 66 68 69 71
74 75 80 81 82 83 85 86 87 90 91 93;

partition chartype = 2: Neomorphic, Transformational;

set partition = chartype;

unlink shape=(all) pinvar=(all);
\end{quote}

Neomorphic characters were not assumed to have a symmetrical transition
rate -- that is, the probability of the absent → present transition was
allowed to differ from that of the present → absent transition, being
drawn from a uniform prior:

\begin{quote}
prset applyto=(1) symdirihyperpr=fixed(1.0);
\end{quote}

Four MrBayes runs were executed, each sampling eight chains for 1 000
000 generations, with samples taken every 500 generations:

\begin{quote}
mcmcp ngen=1000000 samplefreq=500 nruns=2 nchains=8;
\end{quote}

The first 10\% of samples were discarded as burn-in
(\texttt{burninfrac=0.1}), and a posterior tree topology was derived
from the combined posterior sample of both runs.

\begin{figure}
\centering
\includegraphics{Brachiopod_phylogeny_files/figure-latex/unnamed-chunk-11-1.pdf}
\caption{\label{fig:unnamed-chunk-11}Bayesian analysis, posterior
probability \textgreater{} 50\%}
\end{figure}

\begin{figure}
\centering
\includegraphics{Brachiopod_phylogeny_files/figure-latex/unnamed-chunk-12-1.pdf}
\caption{\label{fig:unnamed-chunk-12}Bayesian analysis, posterior
probability \textgreater{} 50\%}
\end{figure}

Convergence was indicated by PSRF = 1.00 and an estimated sample size of
\textgreater{} 500 for each parameter:

\begin{table}

\caption{(\#tab:MrBayes parameter summary)MrBayes parameter estimates (.pstat file)}
\centering
\begin{tabular}[t]{l|r|r|r|r|r|r|r|r}
\hline
Parameter & Mean & Variance & Lower & Upper & Median & minESS & avgESS & PSRF\\
\hline
TL\{all\} & 6.708892 & 1.624940 & 4.8329690 & 9.249889 & 6.781895 & 9.742945 & 877.9326 & 1.022541\\
\hline
alpha\{1\} & 2.403829 & 1.536554 & 0.0001827 & 4.702963 & 2.194241 & 23.533840 & 1038.9600 & 1.005475\\
\hline
alpha\{2\} & 2.950148 & 2.036612 & 0.0022876 & 5.567708 & 2.770201 & 22.238850 & 1073.5860 & 1.004902\\
\hline
\end{tabular}
\end{table}

\bibliography{References.bib}


\end{document}
