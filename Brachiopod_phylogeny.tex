\documentclass[openany]{book}
\usepackage{lmodern}
\usepackage{amssymb,amsmath}
\usepackage{ifxetex,ifluatex}
\usepackage{fixltx2e} % provides \textsubscript
\ifnum 0\ifxetex 1\fi\ifluatex 1\fi=0 % if pdftex
  \usepackage[T1]{fontenc}
  \usepackage[utf8]{inputenc}
\else % if luatex or xelatex
  \ifxetex
    \usepackage{mathspec}
  \else
    \usepackage{fontspec}
  \fi
  \defaultfontfeatures{Ligatures=TeX,Scale=MatchLowercase}
\fi
% use upquote if available, for straight quotes in verbatim environments
\IfFileExists{upquote.sty}{\usepackage{upquote}}{}
% use microtype if available
\IfFileExists{microtype.sty}{%
\usepackage{microtype}
\UseMicrotypeSet[protrusion]{basicmath} % disable protrusion for tt fonts
}{}
\usepackage[margin=1in]{geometry}
\usepackage{hyperref}
\hypersetup{unicode=true,
            pdftitle={Supplementary Information for: Hyoliths with pedicles constrain the origin of the brachiopod body plan},
            pdfauthor={Haijing Sun, Martin R. Smith, Han Zeng, Fangchen Zhao, Guoxiang Li and Maoyan Zhu},
            pdfborder={0 0 0},
            breaklinks=true}
\urlstyle{same}  % don't use monospace font for urls
\usepackage{natbib}
\bibliographystyle{apalike-doi}
\usepackage{color}
\usepackage{fancyvrb}
\newcommand{\VerbBar}{|}
\newcommand{\VERB}{\Verb[commandchars=\\\{\}]}
\DefineVerbatimEnvironment{Highlighting}{Verbatim}{commandchars=\\\{\}}
% Add ',fontsize=\small' for more characters per line
\usepackage{framed}
\definecolor{shadecolor}{RGB}{248,248,248}
\newenvironment{Shaded}{\begin{snugshade}}{\end{snugshade}}
\newcommand{\KeywordTok}[1]{\textcolor[rgb]{0.13,0.29,0.53}{\textbf{#1}}}
\newcommand{\DataTypeTok}[1]{\textcolor[rgb]{0.13,0.29,0.53}{#1}}
\newcommand{\DecValTok}[1]{\textcolor[rgb]{0.00,0.00,0.81}{#1}}
\newcommand{\BaseNTok}[1]{\textcolor[rgb]{0.00,0.00,0.81}{#1}}
\newcommand{\FloatTok}[1]{\textcolor[rgb]{0.00,0.00,0.81}{#1}}
\newcommand{\ConstantTok}[1]{\textcolor[rgb]{0.00,0.00,0.00}{#1}}
\newcommand{\CharTok}[1]{\textcolor[rgb]{0.31,0.60,0.02}{#1}}
\newcommand{\SpecialCharTok}[1]{\textcolor[rgb]{0.00,0.00,0.00}{#1}}
\newcommand{\StringTok}[1]{\textcolor[rgb]{0.31,0.60,0.02}{#1}}
\newcommand{\VerbatimStringTok}[1]{\textcolor[rgb]{0.31,0.60,0.02}{#1}}
\newcommand{\SpecialStringTok}[1]{\textcolor[rgb]{0.31,0.60,0.02}{#1}}
\newcommand{\ImportTok}[1]{#1}
\newcommand{\CommentTok}[1]{\textcolor[rgb]{0.56,0.35,0.01}{\textit{#1}}}
\newcommand{\DocumentationTok}[1]{\textcolor[rgb]{0.56,0.35,0.01}{\textbf{\textit{#1}}}}
\newcommand{\AnnotationTok}[1]{\textcolor[rgb]{0.56,0.35,0.01}{\textbf{\textit{#1}}}}
\newcommand{\CommentVarTok}[1]{\textcolor[rgb]{0.56,0.35,0.01}{\textbf{\textit{#1}}}}
\newcommand{\OtherTok}[1]{\textcolor[rgb]{0.56,0.35,0.01}{#1}}
\newcommand{\FunctionTok}[1]{\textcolor[rgb]{0.00,0.00,0.00}{#1}}
\newcommand{\VariableTok}[1]{\textcolor[rgb]{0.00,0.00,0.00}{#1}}
\newcommand{\ControlFlowTok}[1]{\textcolor[rgb]{0.13,0.29,0.53}{\textbf{#1}}}
\newcommand{\OperatorTok}[1]{\textcolor[rgb]{0.81,0.36,0.00}{\textbf{#1}}}
\newcommand{\BuiltInTok}[1]{#1}
\newcommand{\ExtensionTok}[1]{#1}
\newcommand{\PreprocessorTok}[1]{\textcolor[rgb]{0.56,0.35,0.01}{\textit{#1}}}
\newcommand{\AttributeTok}[1]{\textcolor[rgb]{0.77,0.63,0.00}{#1}}
\newcommand{\RegionMarkerTok}[1]{#1}
\newcommand{\InformationTok}[1]{\textcolor[rgb]{0.56,0.35,0.01}{\textbf{\textit{#1}}}}
\newcommand{\WarningTok}[1]{\textcolor[rgb]{0.56,0.35,0.01}{\textbf{\textit{#1}}}}
\newcommand{\AlertTok}[1]{\textcolor[rgb]{0.94,0.16,0.16}{#1}}
\newcommand{\ErrorTok}[1]{\textcolor[rgb]{0.64,0.00,0.00}{\textbf{#1}}}
\newcommand{\NormalTok}[1]{#1}
\usepackage{longtable,booktabs}
\usepackage{graphicx,grffile}
\makeatletter
\def\maxwidth{\ifdim\Gin@nat@width>\linewidth\linewidth\else\Gin@nat@width\fi}
\def\maxheight{\ifdim\Gin@nat@height>\textheight\textheight\else\Gin@nat@height\fi}
\makeatother
% Scale images if necessary, so that they will not overflow the page
% margins by default, and it is still possible to overwrite the defaults
% using explicit options in \includegraphics[width, height, ...]{}
\setkeys{Gin}{width=\maxwidth,height=\maxheight,keepaspectratio}
\IfFileExists{parskip.sty}{%
\usepackage{parskip}
}{% else
\setlength{\parindent}{0pt}
\setlength{\parskip}{6pt plus 2pt minus 1pt}
}
\setlength{\emergencystretch}{3em}  % prevent overfull lines
\providecommand{\tightlist}{%
  \setlength{\itemsep}{0pt}\setlength{\parskip}{0pt}}
\setcounter{secnumdepth}{5}
% Redefines (sub)paragraphs to behave more like sections
\ifx\paragraph\undefined\else
\let\oldparagraph\paragraph
\renewcommand{\paragraph}[1]{\oldparagraph{#1}\mbox{}}
\fi
\ifx\subparagraph\undefined\else
\let\oldsubparagraph\subparagraph
\renewcommand{\subparagraph}[1]{\oldsubparagraph{#1}\mbox{}}
\fi

%%% Use protect on footnotes to avoid problems with footnotes in titles
\let\rmarkdownfootnote\footnote%
\def\footnote{\protect\rmarkdownfootnote}

%%% Change title format to be more compact
\usepackage{titling}

% Create subtitle command for use in maketitle
\newcommand{\subtitle}[1]{
  \posttitle{
    \begin{center}\large#1\end{center}
    }
}

\setlength{\droptitle}{-2em}

  \title{Supplementary Information for: \newline\newline Hyoliths with pedicles
constrain the origin of the brachiopod body plan}
    \pretitle{\vspace{\droptitle}\centering\huge}
  \posttitle{\par}
    \author{Haijing Sun, Martin R. Smith, Han Zeng, Fangchen Zhao, Guoxiang Li and
Maoyan Zhu}
    \preauthor{\centering\large\emph}
  \postauthor{\par}
      \predate{\centering\large\emph}
  \postdate{\par}
    \date{2018-06-20}

\usepackage{doi} % Adds hyperlinks to dois
\setcitestyle{round}
\usepackage{hyperref}
\usepackage[nottoc]{tocbibind} % list references in TOC
\raggedbottom % already in pnas-new
\usepackage[section]{placeins}

\begin{document}
\maketitle

{
\setcounter{tocdepth}{1}
\tableofcontents
}
\chapter*{Supplementary Text}\label{supplementary-text}
\addcontentsline{toc}{chapter}{Supplementary Text}

This document comtains supplementary material to
\citet{Sun2018Hyolithswith}. It is best viewed in HTML format at
\href{https://ms609.github.io/hyoliths/}{ms609.github.io/hyoliths}.

It describes the \protect\hyperlink{dataset}{morphological dataset} and
the results of tree searches using \protect\hyperlink{fitch}{Fitch
parsimony} and a \protect\hyperlink{bayesian}{Bayesian method}:
approaches that are subject to errors resulting from logically
incoherent treatment of inapplicable data \citep{Maddison1993}. We also
present the \protect\hyperlink{treesearch}{results} of tree searches
with the algorithm described by \citet{Brazeau2018}, which reduces error
due to inapplicable data in a parsimony setting. Finally, we document
how each character is parsimoniously
\protect\hyperlink{reconstructions}{reconstructed} on an optimal tree.

Supplementary \protect\hyperlink{figures}{figures} and
\protect\hyperlink{table}{tables} appear after the text.







\hypertarget{dataset}{\chapter{Phylogenetic dataset}\label{dataset}}

Analysis was performed on a new matrix of 54 lophotrochozoan taxa, coded
for 225 morphological characters (129 neomorphic, 96 transformational).
The matrix can be viewed interactively at Morphobank
(\href{https://morphobank.org/permalink/?P2800}{project 2800}) {[}This
dataset will be released on publication of the paper. Referee access is
availble by
\href{https://morphobank.org/index.php/LoginReg/form}{logging in to
MorphoBank} using the e-mail address and password given in the
manuscript.{]}; a static version can be downloaded directly :

\begin{itemize}
\item
  \href{https://raw.githubusercontent.com/ms609/hyoliths/master/mbank_X24932_6-19-2018_744.nex}{raw.githubusercontent.com/ms609/hyoliths/master/mbank\_X24932\_6-19-2018\_744.nex}
  (Nexus format)
\item
  \href{https://raw.githubusercontent.com/ms609/hyoliths/master/mbank_X24932_6-15-2018_519.tnt}{raw.githubusercontent.com/ms609/hyoliths/master/mbank\_X24932\_6-15-2018\_519.tnt}
  (TNT format).
\end{itemize}

Taxa include sipunculans and molluscs, which have previously been
interpreted as having affinities with hyoliths. Other lophotrochozoan
groups help to constrain the outgroup topology, and a diversity of
brachiozoans helps to resolve the position of hyoliths within this
group.

Characters are coded following the recommendations of
\citet{Brazeau2018}:

\begin{itemize}
\item
  We have employed reductive coding, using a distinct state to mark
  character inapplicability. Character specifications follow the
  structural syntax of \citet{Sereno2007} in order to highlight
  ontological dependence between characters and emphasize the structure
  of the dataset.
\item
  We have distinguished between neomorphic and transformational
  characters \citep[\emph{sensu}][]{Sereno2007} by reserving the token
  \texttt{0} to refer to the absence of a neomorphic
  (i.e.~presence/absence) character. The states of transformational
  characters (i.e.~characters that describe a property of a feature) are
  represented by the tokens \texttt{1}, \texttt{2}, \texttt{3}, \ldots{}
\item
  We code the absence of neomorphic ontologically dependent characters
  \citep[\emph{sensu}][]{Vogt2017} as absence, rather than
  inapplicability.
\end{itemize}

The complete dataset comprises 4975 character codings, plus 1133
inapplicable codings. (The amount and quality of data that \emph{is}
coded is more instructive than the number of cells that are ambiguous
\citep{Wiens1998, Wiens2003}, which, for completeness, is 6042). Of the
225 characters, the number that were coded with an applicable token for
each taxon is:

\begin{tabular}{l|l|l|l|l|l}
\hline
 &  &  &  &  & \\
\hline
\_Acanthotretella spinosa\_ & 70 & \_Gasconsia\_ & 70 & \_Novocrania\_ & 186\\
\hline
\_Alisina\_ & 84 & \_Glyptoria\_ & 73 & \_Orthis\_ & 70\\
\hline
\_Amathia\_ & 157 & \_Halkieria evangelista\_ & 65 & \_Paramicrocornus\_ & 57\\
\hline
\_Antigonambonites planus\_ & 85 & \_Haplophrentis carinatus\_ & 82 & \_Paterimitra\_ & 67\\
\hline
\_Askepasma toddense\_ & 78 & \_Heliomedusa orienta\_ & 67 & \_Pauxillites\_ & 56\\
\hline
\_Bactrotheca\_ & 53 & \_Kutorgina chengjiangensis\_ & 84 & \_Pedunculotheca diania\_ & 63\\
\hline
\_Botsfordia\_ & 75 & \_Lingula\_ & 205 & \_Pelagodiscus atlanticus\_ & 166\\
\hline
\_Clupeafumosus socialis\_ & 76 & \_Lingulellotreta malongensis\_ & 87 & \_Phoronis\_ & 169\\
\hline
\_Conotheca\_ & 60 & \_Lingulosacculus\_ & 60 & \_Salanygolina\_ & 79\\
\hline
\_Coolinia pecten\_ & 80 & \_Longtancunella chengjiangensis\_ & 61 & \_Serpula\_ & 170\\
\hline
\_Cotyledion tylodes\_ & 65 & \_Loxosomella\_ & 164 & \_Siphonobolus priscus\_ & 74\\
\hline
\_Craniops\_ & 66 & \_Maxilites\_ & 61 & \_Sipunculus\_ & 168\\
\hline
\_Cupitheca holocyclata\_ & 63 & \_Mickwitzia muralensis\_ & 72 & \_Terebratulina\_ & 184\\
\hline
\_Dailyatia\_ & 55 & \_Micrina\_ & 71 & \_Tomteluva perturbata\_ & 58\\
\hline
\_Dentalium\_ & 169 & \_Micromitra\_ & 81 & \_Tonicella\_ & 188\\
\hline
\_Eccentrotheca\_ & 54 & \_Mummpikia nuda\_ & 55 & \_Ussunia\_ & 53\\
\hline
\_Eoobolus\_ & 81 & \_Namacalathus\_ & 59 & \_Wiwaxia corrugata\_ & 75\\
\hline
\_Flustra\_ & 168 & \_Nisusia sulcata\_ & 80 & \_Yuganotheca elegans\_ & 56\\
\hline
\end{tabular}

\hypertarget{fitch}{\chapter{Fitch parsimony}\label{fitch}}

Parsimony search with the \citet{Fitch1971} algorithm was conducted in
TNT v1.5 \citep{Goloboff2016} using Ratchet and tree drifting heuristics
\citep{Goloboff1999, Nixon1999}, repeating the search until the optimal
score had been hit by 1500 independent searches:

\begin{quote}
xmult:rat25 drift25 hits 1500 level 4 chklevel 5;
\end{quote}

Searches were conducted under equal weights and results saved to file:

\begin{quote}
piwe-; xmult; {/* Conduct search with equal weighting */}

tsav *TNT/ew.tre;sav;tsav/; {/* Save results to file */}
\end{quote}

Node support was estimated by calculating the proportion of jackknife
replicates in which each group occurred, using 5000 symmetric resampling
iterations, following the recommendations of \citet{Kopuchian2010} and
\citet{Simmons2011}.

\begin{quote}
var: nt; {/* Define a variable to track tree address */}

nelsen *; {/* Generate strict consensus tree */}

set nt ntrees; ttag=; {/* Prepare for resampling */}

resample=sym 5000 frequency from `nt'; {/* Symmetric resampling,
counting frequencies */}

log TNT/ew.sym; ttag/; log/; {/* Write results to log */}

keep 0; ttag-; hold 10000; {/* Clear memory */}
\end{quote}

Further searches were conducted under extended implied weighting
\citep{Goloboff1997, Goloboff2014}, under the concavity constants 3,
4.5, 7, 10.5, 16 and 24:

\begin{quote}
xpiwe=; {/* Enable extended implied weighting */}

piwe=3; xmult; {/* Conduct analysis at k = 3 */}

tsav *TNT/xpiwe3.tre; sav; tsav/; {/* Save results to file */}

nelsen *; set nt ntrees; ttag=; {/* Prepare for resampling */}

resample=frequency from `nt'; {/* Symmetric resampling */}

log TNT/xpiwe3.sym; ttag/; log/; {/* Write results to log */}

keep 0; ttag-; hold 10000; {/* Clear memory */}

{/* Repeat this block for each value of k */}
\end{quote}

Results can be replicated by:

\begin{itemize}
\tightlist
\item
  Downloading
  \href{https://raw.githubusercontent.com/ms609/hyoliths/master/mbank_X24932_6-15-2018_519.tnt}{the
  data in TNT format}:
  \href{https://raw.githubusercontent.com/ms609/hyoliths/master/mbank_X24932_6-15-2018_519.tnt}{raw.githubusercontent.com/ms609/hyoliths/master/mbank\_X24932\_6-15-2018\_519.tnt}.
\item
  Saving
  \href{https://raw.githubusercontent.com/ms609/hyoliths/master/tnt.run}{the
  script above} to the same directory, with the filename
  \texttt{tnt.run}.
\item
  Opening TNT, typing \texttt{piwe=} before you load the downloaded
  dataset (to enable extended implied weighting), then typing
  \texttt{tnt} into the command box to run the script.
\end{itemize}

We acknowledge the Willi Hennig Society for their sponsorship of the TNT
software.

\section{Results}\label{results}









\begin{figure}
\centering
\includegraphics{Brachiopod_phylogeny_files/figure-latex/tnt-iw-overview-1.pdf}
\caption{\label{fig:tnt-iw-overview}Majority-rule consensus of all trees that are optimal
under implied weights. Node labels denote, where less than 100\%, the
frequency of each node in the set of all optimal trees.}
\end{figure}




\begin{figure}
\centering
\includegraphics{Brachiopod_phylogeny_files/figure-latex/tnt-iw-indiv-1-1.pdf}
\caption{\label{fig:tnt-iw-indiv-1}Strict consensus of all optimal trees under Fitch
parsimony with implied weighting at \emph{k} = 3 and 4.5.
Nodes labelled with jackknife frequencies (\%).}
\end{figure}

\includegraphics{Brachiopod_phylogeny_files/figure-latex/tnt-iw-indiv-2-1.pdf}
\includegraphics{Brachiopod_phylogeny_files/figure-latex/tnt-iw-indiv-3-1.pdf}

\newpage




\begin{figure}
\centering
\includegraphics{Brachiopod_phylogeny_files/figure-latex/tnt-ew-consensus-1.pdf}
\caption{\label{fig:tnt-ew-consensus}Consensus of all trees that are optimal under equally
weighted Fitch parsimony. Nodes labelled with jackknife frequencies (\%).}
\end{figure}

\hypertarget{treesearch}{\chapter{Corrected
parsimony}\label{treesearch}}

The phylogenetic dataset contains a considerable proportion of
inapplicable codings (1133, = 18.5\% of 6108 non-ambiguous tokens; 9.3\%
of r 12150 total cells), which are known to introduce error and bias to
phylogenetic reconstruction when the Fitch algorithm is employed
\citep{Maddison1993, Brazeau2018}. As such, we used the R package
\emph{TreeSearch} v0.1.2 \citep{Smith2018TreeSearch} to conduct
phylogenetic tree search with a tree-scoring algorithm that avoids
logically impossible character transformations when handling
inapplicable data \citep{Brazeau2018}, implemented in the
\emph{MorphyLib} C library \citep{Brazeau2017Morphylib}.

\section{Search parameters}\label{search-parameters}

Heuristic searches were conducted using the parsimony ratchet
\citep{Nixon1999} under equal and implied weights \citep{Goloboff1997}.
The consensus tree presented in the main manuscript represents a strict
consensus of all trees that are most parsimonious under one or more of
the concavity constants (\emph{k}) 3, 4.5, 7, 10.5, 16 and 24, an
approach that has been shown to produce higher accuracy (i.e.~more nodes
and quartets resolved correctly) than equal weights at any given level
of precision \citep{Smith2017}.

\section{Analysis}\label{analysis}

The R commands used to conduct the analysis are reproduced below. The
results can most readily be replicated using the
\href{https://github.com/ms609/hyoliths/}{R markdown files} used to
generate these pages: in Rstudio, run
\href{https://raw.githubusercontent.com/ms609/hyoliths/master/index.Rmd}{raw.githubusercontent.com/ms609/hyoliths/master/index.Rmd},
then run each block in
\href{https://raw.githubusercontent.com/ms609/hyoliths/master/13_TreeSearch.Rmd}{raw.githubusercontent.com/ms609/hyoliths/master/13\_TreeSearch.Rmd}.
The complete analysis will take several hours.

\subsection{Initialize and load data}\label{initialize-and-load-data}

\begin{Shaded}
\begin{Highlighting}[]
\CommentTok{# Load data from locally downloaded copy of MorphoBank matrix}
\NormalTok{my_data <-}\StringTok{ }\KeywordTok{ReadAsPhyDat}\NormalTok{(nexusFile)}
\NormalTok{my_data[ignored_taxa] <-}\StringTok{ }\OtherTok{NULL}
\NormalTok{iw_data <-}\StringTok{ }\KeywordTok{PrepareDataIW}\NormalTok{(my_data)}
\end{Highlighting}
\end{Shaded}

\subsection{Generate starting tree}\label{generate-starting-tree}

Start by quickly rearranging a neighbour-joining tree, rooted on the
outgroup.

\begin{Shaded}
\begin{Highlighting}[]
\NormalTok{nj.tree <-}\StringTok{ }\KeywordTok{NJTree}\NormalTok{(my_data)}
\NormalTok{rooted.tree <-}\StringTok{ }\KeywordTok{EnforceOutgroup}\NormalTok{(nj.tree, outgroup)}
\NormalTok{start.tree <-}\StringTok{ }\KeywordTok{TreeSearch}\NormalTok{(}\DataTypeTok{tree=}\NormalTok{rooted.tree, }\DataTypeTok{dataset=}\NormalTok{my_data, }\DataTypeTok{maxIter=}\DecValTok{3000}\NormalTok{,}
                         \DataTypeTok{EdgeSwapper=}\NormalTok{RootedNNISwap, }\DataTypeTok{verbosity=}\DecValTok{0}\NormalTok{)}
\end{Highlighting}
\end{Shaded}

\subsection{Implied weights analysis}\label{implied-weights-analysis}

The position of the root does not affect tree score, so we keep it fixed
(using \texttt{RootedXXXSwap} functions) to avoid unnecessary swaps.

\begin{Shaded}
\begin{Highlighting}[]
\ControlFlowTok{for}\NormalTok{ (k }\ControlFlowTok{in}\NormalTok{ kValues) \{}
\NormalTok{  iw.tree <-}\StringTok{ }\KeywordTok{IWRatchet}\NormalTok{(start.tree, iw_data, }\DataTypeTok{concavity=}\NormalTok{k,}
                       \DataTypeTok{ratchHits =} \DecValTok{20}\NormalTok{, }\DataTypeTok{ratchIter=}\DecValTok{4000}\NormalTok{, }\DataTypeTok{searchHits=}\DecValTok{56}\NormalTok{,}
                       \DataTypeTok{swappers=}\KeywordTok{list}\NormalTok{(RootedTBRSwap, RootedSPRSwap, RootedNNISwap),}
                       \DataTypeTok{verbosity=}\NormalTok{0L)}
\NormalTok{  score <-}\StringTok{ }\KeywordTok{IWScore}\NormalTok{(iw.tree, iw_data, }\DataTypeTok{concavity=}\NormalTok{k)}
  \CommentTok{# Write a single best tree}
  \KeywordTok{write.nexus}\NormalTok{(iw.tree,}
              \DataTypeTok{file=}\KeywordTok{paste0}\NormalTok{(}\StringTok{"TreeSearch/hy_iw_k"}\NormalTok{, k, }\StringTok{"_"}\NormalTok{, }
                          \KeywordTok{signif}\NormalTok{(score, }\DecValTok{5}\NormalTok{), }\StringTok{".nex"}\NormalTok{, }\DataTypeTok{collapse=}\StringTok{''}\NormalTok{))}

\NormalTok{  iw.consensus <-}\StringTok{ }\KeywordTok{IWRatchetConsensus}\NormalTok{(iw.tree, iw_data, }\DataTypeTok{concavity=}\NormalTok{k,}
                  \DataTypeTok{swappers=}\KeywordTok{list}\NormalTok{(RootedTBRSwap, RootedNNISwap),}
                  \DataTypeTok{searchHits=}\DecValTok{55}\NormalTok{, }\DataTypeTok{searchIter=}\DecValTok{4000}\NormalTok{, }\DataTypeTok{nSearch=}\DecValTok{250}\NormalTok{, }\DataTypeTok{verbosity=}\NormalTok{0L)}
  \KeywordTok{write.nexus}\NormalTok{(iw.consensus, }
              \DataTypeTok{file=}\KeywordTok{paste0}\NormalTok{(}\StringTok{"TreeSearch/hy_iw_k"}\NormalTok{, k, }\StringTok{"_"}\NormalTok{, }
                          \KeywordTok{signif}\NormalTok{(}\KeywordTok{IWScore}\NormalTok{(iw.consensus[[}\DecValTok{1}\NormalTok{]], iw_data, }\DataTypeTok{concavity=}\NormalTok{k), }\DecValTok{5}\NormalTok{),}
                          \StringTok{".all.nex"}\NormalTok{, }\DataTypeTok{collapse=}\StringTok{''}\NormalTok{))}
\NormalTok{\}}
\end{Highlighting}
\end{Shaded}

\subsection{Equal weights analysis}\label{equal-weights-analysis}

\begin{Shaded}
\begin{Highlighting}[]
\NormalTok{ew.tree <-}\StringTok{ }\KeywordTok{Ratchet}\NormalTok{(start.tree, my_data, }\DataTypeTok{verbosity=}\NormalTok{0L,}
                   \DataTypeTok{ratchHits =} \DecValTok{20}\NormalTok{, }\DataTypeTok{ratchIter=}\DecValTok{4000}\NormalTok{, }\DataTypeTok{searchHits=}\DecValTok{55}\NormalTok{,}
                   \DataTypeTok{swappers=}\KeywordTok{list}\NormalTok{(RootedTBRSwap, RootedSPRSwap, RootedNNISwap))}
\NormalTok{ew.consensus <-}\StringTok{ }\KeywordTok{RatchetConsensus}\NormalTok{(ew.tree, my_data, }\DataTypeTok{nSearch=}\DecValTok{250}\NormalTok{, }\DataTypeTok{searchHits =} \DecValTok{85}\NormalTok{,}
                                 \DataTypeTok{swappers=}\KeywordTok{list}\NormalTok{(RootedTBRSwap, RootedNNISwap),}
                                 \DataTypeTok{verbosity=}\NormalTok{0L)}
\KeywordTok{write.nexus}\NormalTok{(ew.consensus, }\DataTypeTok{file=}\KeywordTok{paste0}\NormalTok{(}\DataTypeTok{collapse=}\StringTok{''}\NormalTok{, }\StringTok{"TreeSearch/hy_ew_"}\NormalTok{,}
                                      \KeywordTok{Fitch}\NormalTok{(ew.tree, my_data), }\StringTok{".nex"}\NormalTok{))}
\end{Highlighting}
\end{Shaded}

\section{Results}\label{results-1}







\begin{figure}
\centering
\includegraphics{Brachiopod_phylogeny_files/figure-latex/treesearch-maj-consensus-1.pdf}
\caption{\label{fig:treesearch-maj-consensus}Consensus of all parsimony trees, under equal and
implied weights.
Node labels denote the frequency of each clade in
most parsimonious trees under all analytical conditions.}
\end{figure}




\begin{figure}
\centering
\includegraphics{Brachiopod_phylogeny_files/figure-latex/treesearch-maj-consensus-pruned-1.pdf}
\caption{\label{fig:treesearch-maj-consensus-pruned}Consensus of same trees, with taxa pruned before
constructing consensus to give context to clade support.
Node labels denote the frequency of each clade in
most parsimonious trees under all analytical conditions.}
\end{figure}



\clearpage 

\begin{figure}
\centering
\includegraphics{Brachiopod_phylogeny_files/figure-latex/treesearch-iw-indiv-1-1.pdf}
\caption{\label{fig:treesearch-iw-indiv-1}Strict consensus trees of implied weights analyses
at \emph{k} = 3 and 4.5.}
\end{figure}

\clearpage 

\begin{figure}
\centering
\includegraphics{Brachiopod_phylogeny_files/figure-latex/treesearch-iw-indiv-2-1.pdf}
\caption{\label{fig:treesearch-iw-indiv-2}Strict consensus trees of implied weights analyses
at \emph{k} = 7 and 10.5.}
\end{figure}

\clearpage 

\begin{figure}
\centering
\includegraphics{Brachiopod_phylogeny_files/figure-latex/treesearch-iw-indiv-3-1.pdf}
\caption{\label{fig:treesearch-iw-indiv-3}Strict consensus trees of implied weights analyses
at \emph{k} = 16 and 24.}
\end{figure}

\clearpage

\begin{figure}
\centering
\includegraphics{Brachiopod_phylogeny_files/figure-latex/treesearch-equal-weights-results-1.pdf}
\caption{\label{fig:treesearch-equal-weights-results}Strict consensus of
most parsimonious trees under equally weighted parsimony}
\end{figure}

\clearpage

\hypertarget{bayesian}{\chapter{Bayesian analysis}\label{bayesian}}

Bayesian search was conducted in MrBayes v3.2.6 \citep{Ronquist2012}
using the Mk model \citep{Lewis2001} with gamma-distributed rate
variation across characters:

\begin{quote}
lset coding=variable rates=gamma;
\end{quote}

Branch length was drawn from a dirichlet prior distribution, which is
less informative than an exponential model \citep{Rannala2012}, but
requires a prior mean tree length within about two orders of magnitude
of the true value \citep{Zhang2012}. To satisfy this latter criterion,
we specified the prior mean tree length to be equal to the length of the
most parsimonious tree under equal weights, using a Dirichlet prior with
\(\alpha_T = 1\), \(\beta_T = 1/(\)\emph{equal weights tree
length}\(/\)\emph{number of characters}\()\), \(\alpha = c = 1\):

\begin{quote}
prset brlenspr = unconstrained: gammadir(1, 0.35, 1, 1);
\end{quote}

Neomorphic and transformational characters
\citep[\emph{sensu}][]{Sereno2007} were allocated to two separate
partitions whose proportion of invariant characters and gamma shape
parameters were allowed to vary independently:

\begin{quote}
charset Neomorphic = 1 6 7 9 10 15 18 19 20 21 22 24 26 29 30 34 35 37
38 40 43 44 45 48 49 53 54 55 56 57 58 59 60 61 65 66 67 69 70 72 77 79
80 81 82 84 87 88 89 92 93 94 95 98 99 100 101 102 103 107 109 110 112
119 120 122 123 126 127 130 137 138 139 140 141 142 143 144 148 150 151
153 154 155 156 157 158 160 164 168 170 174 175 176 177 179 180 181 182
183 184 185 186 187 188 190 192 193 196 199 202 203 204 205 206 207 209
210 211 212 214 215 216 217 218 219 220 222 225;

charset Transformational = 2 3 4 5 8 11 12 13 14 16 17 23 25 27 28 31 32
33 36 39 41 42 46 47 50 51 52 62 63 64 68 71 73 74 75 76 78 83 85 86 90
91 96 97 104 105 106 108 111 113 114 115 116 117 118 121 124 125 128 129
131 132 133 134 135 136 145 146 147 149 152 159 161 162 163 165 166 167
169 171 172 173 178 189 191 194 195 197 198 200 201 208 213 221 223 224;

partition chartype = 2: Neomorphic, Transformational;

set partition = chartype;

unlink shape=(all) pinvar=(all);
\end{quote}

Neomorphic characters were not assumed to have a symmetrical transition
rate -- that is, the probability of the absent → present transition was
allowed to differ from that of the present → absent transition, being
drawn from a uniform prior:

\begin{quote}
prset applyto=(1) symdirihyperpr=fixed(1.0);
\end{quote}

The rate of variation in neomorphic characters was also allowed to vary
from that of transformational characters:

\begin{quote}
prset applyto=(1) ratepr=variable;
\end{quote}

\emph{Loxosomella} was selected as an outgroup:

\begin{quote}
outgroup Loxosomella;
\end{quote}

Four MrBayes runs were executed, each sampling eight chains for
5~000~000 generations, with samples taken every 500 generations. The
first 10\% of samples were discarded as burn-in.

\begin{quote}
mcmcp ngen=5000000 samplefreq=500 nruns=4 nchains=8 burninfrac=0.1;
\end{quote}

A posterior tree topology was derived from the combined posterior sample
of all runs. Convergence was indicated by PSRF = 1.00 and an estimated
sample size of \textgreater{} 200 for each parameter. Nodes are labelled
with posterior probabilities; recall that caution must be applied when
interpreting these values \citep{Yang2018}.

The Nexus file used to generate these results in MrBayes can be
\href{https://raw.githubusercontent.com/ms609/hyoliths/master/MrBayes/hyo.nex}{raw.githubusercontent.com/ms609/hyoliths/master/MrBayes/hyo.nex},
and run in \href{http://mrbayes.sourceforge.net/download.php}{MrBayes}
by typing \texttt{exe\ path/to/download}.

\section{Parameter estimates}\label{parameter-estimates}

\begin{tabular}{l|r|r|r|r|r}
\hline
Parameter & Mean & Variance & minESS & avgESS & PSRF\\
\hline
TL\{all\} & 9.980 & 0.50100 & 5440 & 5970 & 0.99999\\
\hline
m\{1\} & 0.462 & 0.00265 & 3100 & 3720 & 0.99999\\
\hline
\end{tabular}

\section{Results}\label{results-2}

\begin{figure}
\centering
\includegraphics{Brachiopod_phylogeny_files/figure-latex/mrbayes-full-consensus-1.pdf}
\caption{\label{fig:mrbayes-full-consensus}Results of Bayesian analysis,
posterior probability \textgreater{} 50\%, all taxa}
\end{figure}

\begin{figure}
\centering
\includegraphics{Brachiopod_phylogeny_files/figure-latex/mrbayes-pruned-consensus-1.pdf}
\caption{\label{fig:mrbayes-pruned-consensus}Results of Bayesian analysis,
posterior probability \textgreater{} 50\%, wildcard taxa pruned}
\end{figure}

\hypertarget{reconstructions}{\chapter{Character
reconstructions}\label{reconstructions}}

This page provides definitions for each of the characters in our matrix,
and justifies codings in particular taxa where relevant. Further
citations for codings that are not discussed in the text can be viewed
by browsing the \protect\hyperlink{dataset}{morphological dataset} on
MorphoBank (\href{https://morphobank.org/permalink/?P2800}{project
2800}). {[}This dataset will be released on publication of the paper.
Referee access is availble by
\href{https://morphobank.org/index.php/LoginReg/form}{logging in to
MorphoBank} using the e-mail address and password given in the
manuscript.{]}

Alongside its definition, each character has been mapped onto a tree.
Here, we have arbitrarily selected one most parsimonious tree obtained
under implied weighting. Other trees can be viewed in the HTML version
of this document at
\href{https://ms609.github.io/hyoliths/reconstructions.html}{ms609.github.io/hyoliths}.
Each tip is labelled according to its coding in the matrix. These states
have been used to reconstruct the condition of each internal node, using
the parsimony method of \citet{Brazeau2018} as implemented in the
\emph{R} package \emph{Inapp}.

We emphasize that different trees give different reconstructions. The
character mappings are not intended to definitively establish how each
character evolved, but to help the reader quickly establish how each
character has been coded, and to visualize at a glance how each
character fits onto a given tree.







\includegraphics{Brachiopod_phylogeny_files/figure-latex/all-changes-1.pdf}

\section{Brephic shell}\label{brephic-shell}

\subsection*{{[}1{]} Embryonic shell}\label{embryonic-shell}
\addcontentsline{toc}{subsection}{{[}1{]} Embryonic shell}

\includegraphics{Brachiopod_phylogeny_files/figure-latex/character-mapping-1.pdf}

\begin{quote}
\textbf{Character 1: Brephic shell: Embryonic shell}

0: Absent\\
1: Present\\
Neomorphic character.
\end{quote}

The embryonic shell or protegulum is secreted by the embryo immediately
before hatching.

\hypertarget{Amathia-coding-1}{}
\emph{Amathia}: \citet{Reed1982}.

\hypertarget{Clupeafumosus_socialis-coding-1}{}
\emph{Clupeafumosus socialis}: Described by Topper \emph{et al}.
\citeyearpar{Topper2013Reappraisalof}.

\hypertarget{Conotheca-coding-1}{}
\emph{Conotheca}: \citep{Wrona2003}.

\hypertarget{Dentalium-coding-1}{}
\emph{Dentalium}: The shell does not form until the trochophore larval
stage, which has been exquisitely described in Antalis
\citep{Wanninger2001}.\\
This shell field is initially disc-like, subsequently expanding to fuse
ventrally and produce the cylindrical protoconch. The prototroch is
clearly delineated fro the telotroch in post-metamorphic juveniles
\citep{Wanninger2001}.

\hypertarget{Gasconsia-coding-1}{}
\emph{Gasconsia}: The earliest shell is not described by
\citet{Hanken1985Thetaxonomy} or \citet{Watkins2002Newrecord}.

\hypertarget{Loxosomella-coding-1}{}
\emph{Loxosomella}: Absent, with no possible equivalent
\citep{Nielsen1966}.

\hypertarget{Namacalathus-coding-1}{}
\emph{Namacalathus}: Inapplicable insofar as reproduction occurs by
budding; there is no evidence for a free-living larval stage.
Nevertheless, the presence of a sexual reproductive phase in addition to
asexual reproduction cannot be discounted.

\hypertarget{Novocrania-coding-1}{}
\emph{Novocrania}: Shell not secreted until after metamorphosis
\citep{Popov2010Earliestontogeny}. Freeman \& Lundelius
\citeyearpar{Freeman1999Changesin} report a \emph{Craniops}-like larval
shell in fossil ``Crania'', but observe that Quaternary {[}Novo{]}crania
no longer exhibit a larval shell.

\hypertarget{Paramicrocornus-coding-1}{}
\emph{Paramicrocornus}: ``The initial part of the conch appears to be a
simple apex without clearly delineated protoconch''
\citep{Zhang2018Ahyolithid}, though it is not clear from illustrated
figures whether an embryonic shell contiguous with the adult shell was
present.

\hypertarget{Pauxillites-coding-1}{}
\emph{Pauxillites}: Coded following \emph{Recilites} \citep{Dzik1978}, a
fellow member of Pauxillitidae \citep{Marek1967}.

\hypertarget{Tonicella-coding-1}{}
\emph{Tonicella}: On hatching, the polyplacophoran larva lacks a shell
field.

Shell fields develop during the trochophore larva stage. The larva of
the chiton \emph{Mopalia} has two distinct shell fields: that anterior
to the prototroch will develop into the first shell plate; the one
posterior to the prototroch becomes the subsequent plates
\citep{Wanninger2002C}.

This disc-shaped posterior plate, whose position corresponds to the
conchiferan shell field, bears a polygonal ornament and is subdivided by
a series of grooves that prefigure the adult shell plates
\citep{Wanninger2002C}.

\subsection*{{[}2{]} Morphology}\label{morphology}
\addcontentsline{toc}{subsection}{{[}2{]} Morphology}

\includegraphics{Brachiopod_phylogeny_files/figure-latex/character-mapping-2.pdf}

\begin{quote}
\textbf{Character 2: Brephic shell: Morphology}

1: Flat, disc-like (cf. \emph{Micrina})\\
2: Three prominent lobes forming a Y (cf. \emph{Paterimitra})\\
3: Spherical\\
4: Fusiform\\
Transformational character.
\end{quote}

The brephic shell is the shell possessed by the young organism
\citep[see][ and references therein for discussion of
terminology]{Ushatinskaya2016Revisionof}.

\emph{Micrina} resembles linguliforms \citep{Holmer2011Firstrecord}: in
both, the brephic mitral shell has one pair of setal sacs enclosed by
lateral lobes, whereas the brephic ventral shell has two lateral setal
tubes.

\emph{Paterimitra} and \emph{Salanygolina} have ``identical'' ventral
brephic shells \citep{Holmer2011Firstrecord}, resembling the shape of a
ship's propeller.

\emph{Haplophrentis} is coded following typical hyoliths, which have a
spherical brephic shell; \emph{Pedunculotheca}'s, in contrast, is
seemingly cap-shaped.

\hypertarget{Askepasma_toddense-coding-2}{}
\emph{Askepasma toddense}: Renoid -- see fig. 4B3 in
\citet{Topper2013Theoldest}.

\hypertarget{Clupeafumosus_socialis-coding-2}{}
\emph{Clupeafumosus socialis}: The flat larval shell of
\emph{Clupeafumosus} resembles that of \emph{Micrina} in outline
\citetext{\citealp{Topper2013Reappraisalof}; \citealp[cf.][]{Holmer2011Firstrecord}}.

\hypertarget{Conotheca-coding-2}{}
\emph{Conotheca}: \citep{Wrona2003}.

\hypertarget{Coolinia_pecten-coding-2}{}
\emph{Coolinia pecten}: See fig. 3 in
\citet{Bassett2017Earliestontogeny}.

\hypertarget{Craniops-coding-2}{}
\emph{Craniops}: The embryonic shell is more or less circular in outline
-- see \citet{Freeman1999Changesin}, fig. 6A.

\hypertarget{Cupitheca_holocyclata-coding-2}{}
\emph{Cupitheca holocyclata}: The impression of the larval shell on the
operculum is flat and disc-like \citep{Skovsted2016}. The fusiform
`protoconch' \citep{Sun2018} likely represents a larval (rather than
brephic) shell.

\hypertarget{Lingula-coding-2}{}
\emph{Lingula}: See fig. 159 in \citet{Williams1997Introduction}.

\hypertarget{Lingulellotreta_malongensis-coding-2}{}
\emph{Lingulellotreta malongensis}: Disc-like \citep{Li2004}.

\hypertarget{Mickwitzia_muralensis-coding-2}{}
\emph{Mickwitzia muralensis}: Trifoliate appearance results from
prominent attachment rudiment and bunching of setal sacs
\citep{Balthasar2009Thebrachiopod}.

\hypertarget{Micromitra-coding-2}{}
\emph{Micromitra}: Subtriangular -- essentially round.

\hypertarget{Pauxillites-coding-2}{}
\emph{Pauxillites}: Fusiform, following \emph{Recilites} (Pauxillitidae)
\citep{Dzik1978}.

\hypertarget{Pelagodiscus_atlanticus-coding-2}{}
\emph{Pelagodiscus atlanticus}: See e.g.~fig 169 in Williams \emph{et
al}. \citeyearpar{Williams1997Introduction}.

\hypertarget{Tonicella-coding-2}{}
\emph{Tonicella}: Disc-like, subdivided by transverse grooves
\citep{Wanninger2002C}.

\subsection*{{[}3{]} Embryonic shell extended in
larvae}\label{embryonic-shell-extended-in-larvae}
\addcontentsline{toc}{subsection}{{[}3{]} Embryonic shell extended in
larvae}

\includegraphics{Brachiopod_phylogeny_files/figure-latex/character-mapping-3.pdf}

\begin{quote}
\textbf{Character 3: Brephic shell: Embryonic shell extended in larvae}

1: Not extended; embryonic shell contiguous with adult shell\\
2: Extended into larval shell, separated from adult shell by prominent
nick\\
Transformational character.
\end{quote}

Many taxa add to their embryonic shell (the protegulum possessed by the
embryo upon hatching) during the larval phase of their life cycle. The
shell that exists at metamorphosis, marked by a halo or nick point, is
variously termed the ``first formed shell'', ``metamorphic shell'' or
``larval shell'' \citep{Bassett2017Earliestontogeny}.

\hypertarget{Bactrotheca-coding-3}{}
\emph{Bactrotheca}: There is a small ridge and a change in surface
ornament at the end of the larval shell \citep{Dzik1980Ontogenyof}.

\hypertarget{Clupeafumosus_socialis-coding-3}{}
\emph{Clupeafumosus socialis}: Described by Topper \emph{et al}.
\citeyearpar{Topper2013Reappraisalof}.

\hypertarget{Conotheca-coding-3}{}
\emph{Conotheca}: Prominent nick \citep[figs 5G-H, 6a1]{Wrona2003}.

\hypertarget{Craniops-coding-3}{}
\emph{Craniops}: Prominent nick; see \citet{Freeman1999Changesin}, fig.
6A.

\hypertarget{Cupitheca_holocyclata-coding-3}{}
\emph{Cupitheca holocyclata}: Prominent nick \citep{Skovsted2016}.

\hypertarget{Eoobolus-coding-3}{}
\emph{Eoobolus}: Nick point indicated by arrows in fig. 1 of Balthasar
\citeyearpar{Balthasar2009Thebrachiopod}.

\hypertarget{Lingulellotreta_malongensis-coding-3}{}
\emph{Lingulellotreta malongensis}: No prominent nick point
\citep{Holmer1997EarlyCambrian, Li2004}.

\hypertarget{Paramicrocornus-coding-3}{}
\emph{Paramicrocornus}: Not clearly delineated
\citep{Zhang2018Ahyolithid}, so either inapplicable or undifferentiated.

\hypertarget{Pauxillites-coding-3}{}
\emph{Pauxillites}: No prominent nick in \emph{Recilites}
(Pauxillitidae) \citep{Dzik1978}.

\hypertarget{Pedunculotheca_diania-coding-3}{}
\emph{Pedunculotheca diania}: The flattened region at the umbo of the
ventral valve in smaller specimens conceivably represents an embryonic
shell, though it may alternatively represent a cicatrix or
colleplax-like structure.

\hypertarget{Tonicella-coding-3}{}
\emph{Tonicella}: \citet{Wanninger2002C}.

\subsection*{{[}4{]} Surface ornament}\label{surface-ornament}
\addcontentsline{toc}{subsection}{{[}4{]} Surface ornament}

\includegraphics{Brachiopod_phylogeny_files/figure-latex/character-mapping-4.pdf}

\begin{quote}
\textbf{Character 4: Brephic shell: Surface ornament}

1: Smooth\\
2: Rounded pits\\
3: Polygonal impressions\\
4: Pustulose\\
Transformational character.
\end{quote}

Pitting of the larval shell characterises acrotretids and their
relatives. Pustules occur on Paterinidae. See Character 3 in Williams
\emph{et al}. \citeyearpar{Williams2000LinguliformeaCraniiformea} tables
5--6.

\hypertarget{Askepasma_toddense-coding-4}{}
\emph{Askepasma toddense}: Indented with hexagonal pits \citep[appendix
2]{Williams1998Thediversity}.

\hypertarget{Clupeafumosus_socialis-coding-4}{}
\emph{Clupeafumosus socialis}: ``Larval shells on both valves
{[}\ldots{}{]} are covered by fine, shallow pits'' --
\citet{Topper2013Reappraisalof}.

\hypertarget{Conotheca-coding-4}{}
\emph{Conotheca}: \citep{Wrona2003}.

\hypertarget{Cupitheca_holocyclata-coding-4}{}
\emph{Cupitheca holocyclata}: Perfectly smooth \citep{Skovsted2016}.

\hypertarget{Eoobolus-coding-4}{}
\emph{Eoobolus}: Pitted \citep[table
8]{Williams2000LinguliformeaCraniiformea}.

\hypertarget{Lingula-coding-4}{}
\emph{Pelagodiscus atlanticus}, \emph{Lingula}: Smooth, following
family-level codings of \citet{Williams2000LinguliformeaCraniiformea},
table 6.

\hypertarget{Lingulellotreta_malongensis-coding-4}{}
\emph{Lingulellotreta malongensis}: Smooth
\citep{Holmer1997EarlyCambrian, Li2004}.

\hypertarget{Micrina-coding-4}{}
\emph{Micrina}: Smooth \citep{Holmer2011Firstrecord}.

\hypertarget{Micromitra-coding-4}{}
\emph{Micromitra}: Pustolose in Paterinidae \citep[table
6]{Williams2000LinguliformeaCraniiformea}.

\hypertarget{Paterimitra-coding-4}{}
\emph{Paterimitra}: Polygonal texture present
\citep{Holmer2011Firstrecord}, as in the adult shell.

\hypertarget{Pauxillites-coding-4}{}
\emph{Pauxillites}: Essentially smooth in \emph{Recilites}
(Pauxillitidae) \citep{Dzik1978}.

\hypertarget{Salanygolina-coding-4}{}
\emph{Salanygolina}: Smooth \citep{Holmer2009Theenigmatic}.

\hypertarget{Siphonobolus_priscus-coding-4}{}
\emph{Siphonobolus priscus}: ``Smooth brephic shell'' --
\citet{Popov2009Earlyontogeny}.

\subsection*{{[}5{]} Larval attachment
structure}\label{larval-attachment-structure}
\addcontentsline{toc}{subsection}{{[}5{]} Larval attachment structure}

\includegraphics{Brachiopod_phylogeny_files/figure-latex/character-mapping-5.pdf}

\begin{quote}
\textbf{Character 5: Brephic shell: Larval attachment structure}

1: Without evidence of pedicle\\
2: With evidence of pedicle\\
Transformational character.
\end{quote}

Embryonic shells of \emph{Micrina} and certain linguliforms exhibit a
transversely folded posterior extension that speaks of the original
presence of a pedicle in the embryo.

This is independent of the presence of an adult pedicle, which may arise
after metamorphosis.

\hypertarget{Clupeafumosus_socialis-coding-5}{}
\emph{Clupeafumosus socialis}: The larval shell embraces the pedicle
foramen, suggesting a larval attachment. See fig. 4 of Topper \emph{et
al}. \citeyearpar{Topper2013Reappraisalof}.

\hypertarget{Conotheca-coding-5}{}
\emph{Conotheca}: \citep{Wrona2003}.

\hypertarget{Cupitheca_holocyclata-coding-5}{}
\emph{Cupitheca holocyclata}: A pedicle is not inferred by
\citet{Skovsted2016}, and is difficult to reconcile with the apical
morphology documented by \citet{Sun2018}.

\hypertarget{Eoobolus-coding-5}{}
\emph{Eoobolus}: Lobe related to the attachment rudiment \citep[fig.
2]{Balthasar2009Thebrachiopod}.

\hypertarget{Lingulellotreta_malongensis-coding-5}{}
\emph{Lingulellotreta malongensis}: The pedicle foramen intersects the
brephic shell \citep{Holmer1997EarlyCambrian, Li2004}, suggesting larval
attachment.

\hypertarget{Mickwitzia_muralensis-coding-5}{}
\emph{Mickwitzia muralensis}: Note the posterior lobe related to the
attachment rudiment in fig. 2 of \citet{Balthasar2009Thebrachiopod}.

\hypertarget{Pauxillites-coding-5}{}
\emph{Pauxillites}: Distal extension of \emph{Recilites} (Pauxillitidae)
has plausible similarity to pedicle \citep{Dzik1978}, but deemed
unlikely. Coded as ambiguous.

\hypertarget{Siphonobolus_priscus-coding-5}{}
\emph{Siphonobolus priscus}: Interpreted as having planktotrophic (and
thus non-attached) larvae \citep{Popov2009Earlyontogeny}.

\subsection*{{[}6{]} Setulose}\label{setulose}
\addcontentsline{toc}{subsection}{{[}6{]} Setulose}

\includegraphics{Brachiopod_phylogeny_files/figure-latex/character-mapping-6.pdf}

\begin{quote}
\textbf{Character 6: Brephic shell: Setulose}

0: No evidence of setae in embryonic shell\\
1: Setae present\\
Neomorphic character.
\end{quote}

The protegulum of \emph{Micrina} is penetrated with canals that were
originally associated with setae, a character that it has in common with
linguliforms \citep{Holmer2011Firstrecord}.

\hypertarget{Botsfordia-coding-6}{}
\emph{Botsfordia}: ``One specimen shows fine capillae running laterally
from the posterior tubercles on the dorsal valve (Pl. 3, fig. 5b). This
is possibly the imprints of setae.'' --
\citet{Ushatinskaya2016Revisionof}.

\hypertarget{Clupeafumosus_socialis-coding-6}{}
\emph{Clupeafumosus socialis}: Setal bundles interpreted as present in
acrotretids by Ushatinskaya \citeyearpar{Ushatinskaya2016Protegulumand}.

\hypertarget{Conotheca-coding-6}{}
\emph{Conotheca}: \citep{Wrona2003}.

\hypertarget{Cupitheca_holocyclata-coding-6}{}
\emph{Cupitheca holocyclata}: Absent \citep{Skovsted2016}.

\hypertarget{Lingulellotreta_malongensis-coding-6}{}
\emph{Lingulellotreta malongensis}: Possible suggestion of setal sacs
present on brephic shell \citep{Holmer1997EarlyCambrian, Li2004}, but
outline inadequately preserved to code with confidence; treated as
ambiguous.

\hypertarget{Mickwitzia_muralensis-coding-6}{}
\emph{Mickwitzia muralensis}: Four setal sacs.

\hypertarget{Pauxillites-coding-6}{}
\emph{Pauxillites}: Following \emph{Recilites} (Pauxillitidae)
\citep{Dzik1978}.

\subsection*{{[}7{]} Setal sacs}\label{setal-sacs}
\addcontentsline{toc}{subsection}{{[}7{]} Setal sacs}

\includegraphics{Brachiopod_phylogeny_files/figure-latex/character-mapping-7.pdf}

\begin{quote}
\textbf{Character 7: Brephic shell: Setal sacs}

0: Absent\\
1: Present\\
Neomorphic character.
\end{quote}

Setal sacs are recognizable as raised lumps on the juvenile shell
\citep[see][]{Bassett2017Earliestontogeny}.

\emph{Micrina} and linguliforms have setal sacs on their mitral/dorsal
embryonic shell, whereas these are absent in \emph{Paterimitra}
\citep{Holmer2011Firstrecord}.

\hypertarget{Bactrotheca-coding-7}{}
\emph{Bactrotheca}: Following \emph{Bactrotheca}
\citep{Dzik1980Ontogenyof}.

\hypertarget{Botsfordia-coding-7}{}
\emph{Botsfordia}: A single pair of low tubercles are \citep[ state
``may be'']{Ushatinskaya2016Revisionof} located in the middle region of
the dorsal and the ventral brephic valve; these are interpreted as a
single pair of setal sacs, with the identity of the (dorsally unpaired)
tubercles uncertain.

\hypertarget{Clupeafumosus_socialis-coding-7}{}
\emph{Clupeafumosus socialis}: Setal bundles interpreted as present in
acrotretids by Ushatinskaya \citeyearpar{Ushatinskaya2016Protegulumand}.

\hypertarget{Conotheca-coding-7}{}
\emph{Conotheca}: \citep{Wrona2003}.

\hypertarget{Lingula-coding-7}{}
\emph{Lingula}: Lingulids' larval setae are not arranged in bundles
\citep{Carlson1995Phylogeneticrelationships}.

\hypertarget{Lingulellotreta_malongensis-coding-7}{}
\emph{Lingulellotreta malongensis}: Possible suggestion of setal sacs
present on brephic shell \citep{Holmer1997EarlyCambrian, Li2004}, but
outline inadequately preserved to code with confidence; treated as
ambiguous.

\hypertarget{Novocrania-coding-7}{}
\emph{Novocrania}, \emph{Pelagodiscus atlanticus}: Three pairs
\citep{Carlson1995Phylogeneticrelationships}.

\hypertarget{Pauxillites-coding-7}{}
\emph{Pauxillites}: Following \emph{Recilites} (Pauxillitidae)
\citep{Dzik1978}.

\subsection*{{[}8{]} Number}\label{number}
\addcontentsline{toc}{subsection}{{[}8{]} Number}

\includegraphics{Brachiopod_phylogeny_files/figure-latex/character-mapping-8.pdf}

\begin{quote}
\textbf{Character 8: Brephic shell: Setal sacs: Number}

1: One pair\\
2: Two pairs\\
3: Three pairs\\
Transformational character.
\end{quote}

Two pairs on e.g.~Coolina; one on e.g. \emph{Micrina}.

\hypertarget{Botsfordia-coding-8}{}
\emph{Botsfordia}: ``larval shell with one to three apical tubercles in
ventral valve and two in dorsal valve''
\citep{Williams2000LinguliformeaCraniiformea} -- if these correspond to
setal sacs, then we interpret this as equivalent to one pair.

In \emph{B. minuta}, the ventral valve bears a single medial tubercle
(which in figured material seems to have two bilaterally symmetrical
fields), whereas the dorsal valve bears two apical tubercles
\citep{Li2004} -- supporting the interpretation of a single pair of
setal sacs.

\hypertarget{Clupeafumosus_socialis-coding-8}{}
\emph{Clupeafumosus socialis}: Two pairs identified in acrotretids by
Ushatinskaya \citeyearpar{Ushatinskaya2016Protegulumand}.

\hypertarget{Mickwitzia_muralensis-coding-8}{}
\emph{Mickwitzia muralensis}: See fig. 2 in
\citet{Balthasar2009Thebrachiopod}.

\hypertarget{Novocrania-coding-8}{}
\emph{Novocrania}, \emph{Pelagodiscus atlanticus}: Three pairs
\citep{Carlson1995Phylogeneticrelationships}.

\hypertarget{Siphonobolus_priscus-coding-8}{}
\emph{Siphonobolus priscus}: Two pairs of setal sacs
\citep{Popov2009Earlyontogeny}.

\section{Larval chaetae: Paired bundles
{[}9{]}}\label{larval-chaetae-paired-bundles-9}

\includegraphics{Brachiopod_phylogeny_files/figure-latex/character-mapping-9.pdf}

\begin{quote}
\textbf{Character 9: Larval chaetae: Paired bundles}

0: Absent\\
1: Present\\
Neomorphic character.
\end{quote}

Annelid chaetae are equivalent to the bundled setae expressed in certain
brachiopod larvae. See character 12 in \citet{Vinther2008}.

\hypertarget{Amathia-coding-9}{}
\emph{Amathia}: \citep{Reed1982}.

\hypertarget{Flustra-coding-9}{}
\emph{Flustra}: Absent \citep{Zimmer2013}.

\hypertarget{Pauxillites-coding-9}{}
\emph{Pauxillites}: Following \emph{Recilites} (Pauxillitidae)
\citep{Dzik1978}.

\hypertarget{Serpula-coding-9}{}
\emph{Serpula}: \citet{Luter2000} contends that annelid larvae lack
setae, but chaetae are present on day 16 of the larval development of
\emph{Serpula} \citep{Keay2007}.

\hypertarget{Terebratulina-coding-9}{}
\emph{Terebratulina}: \citet{Williams1997Introduction}.

\section{Adult setae {[}10{]}}\label{adult-setae-10}

\includegraphics{Brachiopod_phylogeny_files/figure-latex/character-mapping-10.pdf}

\begin{quote}
\textbf{Character 10: Adult setae}

0: Absent\\
1: Present\\
Neomorphic character.
\end{quote}

\citet{Luter2000} demonstrates that the setae of larval and adult
brachiopods exhibit fundamental structural differences and are
conceivably not homologous structures. Larval setae are thus described
separately.

Although preservation of setae in fossil brachiopods is exceptional,
their presence can be inferred from shelly material
\citep[see][]{Holmer2006Aspinose}.

\hypertarget{Acanthotretella_spinosa-coding-10}{}
\emph{Acanthotretella spinosa}: Note that the setae do not obviously
emerge from tubes, leading Holmer and Caron to question their homology
with the setae of other taxa (\emph{Heliomedusa}, \emph{Mickwitzia}).

Both valves of \emph{Acanthotretella} were covered by long spine-like
and shell penetrating setae. The setae of \emph{A. decaius} are usually
preserved along anterior and anterolateral margins
\citep{Hu2010Softpart}.

\hypertarget{Amathia-coding-10}{}
\emph{Amathia}: The teeth of the Bryozoan gizzard have been homologized
with annelid setae \citep{Gordon1975}.

\hypertarget{Clupeafumosus_socialis-coding-10}{}
\emph{Clupeafumosus socialis}: Setal bundles interpreted as present in
acrotretids by Ushatinskaya \citeyearpar{Ushatinskaya2016Protegulumand}.

\hypertarget{Flustra-coding-10}{}
\emph{Flustra}: A gizzard is not present in all bryozoans, and has not
been reported in \emph{Flustra}.

\hypertarget{Haplophrentis_carinatus-coding-10}{}
\emph{Haplophrentis carinatus}: Not observed
\citep{Moysiuk2017Hyolithsare}, despite suitable preservation.

\hypertarget{Lingulellotreta_malongensis-coding-10}{}
\emph{Lingulellotreta malongensis}: ``Setae appear short, delicate, and
are closely fringed with the entire\\
mantle margin, hardly extending beyond the edge of shell'' --
\citet{Zhang2005}.

\hypertarget{Novocrania-coding-10}{}
\emph{Novocrania}: ``Adult craniids are without setae (a feature shared
with the thecideides, the\\
shells of which are also cemented).'' -- \citet{Williams2007Supplement}.

\hypertarget{Siphonobolus_priscus-coding-10}{}
\emph{Siphonobolus priscus}: Phosphatised setae emerge from hollow
spines \citep{Popov2009Earlyontogeny}.

\hypertarget{Sipunculus-coding-10}{}
\emph{Sipunculus}: The absence of chitin or microvillar lineations in
sipunculan hooks argues against their interpretation as setae, but they
are coded as conceivable homologues, with these characteristics treated
separately.

\hypertarget{Tonicella-coding-10}{}
\emph{Tonicella}: The girdle elements of certain polyplacophorans are
chitinous and secreted by microvilli
\citep{Fischer1980, Leise1982, Leise1988}; it is therefore likely that
they are homologous with the setae of other lophotrochozoans.

\hypertarget{Wiwaxia_corrugata-coding-10}{}
\emph{Wiwaxia corrugata}: Sclerites likely correspond with
lophotrochozoan setae \citep{Butterfield1990, Smith2014, Zhang2015}.

\subsection*{{[}11{]} Secretion}\label{secretion}
\addcontentsline{toc}{subsection}{{[}11{]} Secretion}

\includegraphics{Brachiopod_phylogeny_files/figure-latex/character-mapping-11.pdf}

\begin{quote}
\textbf{Character 11: Adult setae: Secretion}

1: By basal microvilli\\
2: Epicuticular\\
Transformational character.
\end{quote}

The majority of lophotrochozoan sclerites bear a characteristic striated
texture that denotes their secretion by basal microvilli
\citep{Butterfield1990}. The seta-like hooks of sipunculans lack this
texture, suggesting that they may not be homologous with other setae.

\hypertarget{Sipunculus-coding-11}{}
\emph{Sipunculus}: No evidence of microvillar secretion
\citep[e.g.][]{Schulze2005}.

\subsection*{{[}12{]} Microvillar diameter}\label{microvillar-diameter}
\addcontentsline{toc}{subsection}{{[}12{]} Microvillar diameter}

\includegraphics{Brachiopod_phylogeny_files/figure-latex/character-mapping-12.pdf}

\begin{quote}
\textbf{Character 12: Adult setae: Secretion: Microvillar diameter}

1: Uniform\\
2: Decreasing towards seta margin\\
Transformational character.
\end{quote}

The diameter of secretory microvilli may vary across the diameter of a
seta \citep{Smith2014}.

\hypertarget{Amathia-coding-12}{}
\emph{Amathia}: No trend in microvillar size \citep{Gordon1975}.

\hypertarget{Lingula-coding-12}{}
\emph{Lingula}: Widest in centre \citep{Luter2000}.

\hypertarget{Pelagodiscus_atlanticus-coding-12}{}
\emph{Pelagodiscus atlanticus}: Slight decrease towards margin in
\emph{Discinisca} \citep{Luter2003}.

\hypertarget{Serpula-coding-12}{}
\emph{Serpula}: Following \emph{Scolelepis} \citep{Hausen2005}; Diasoma
\citep{Orrhage1971}.

\hypertarget{Terebratulina-coding-12}{}
\emph{Terebratulina}: Decreases towards margin in \emph{Terebratalia}
larvae \citep{Gustus1972}.

\hypertarget{Tonicella-coding-12}{}
\emph{Tonicella}: Uniform \citep{Fischer1980, Leise1982}.

\subsection*{{[}13{]} Microvillar canal
aspect}\label{microvillar-canal-aspect}
\addcontentsline{toc}{subsection}{{[}13{]} Microvillar canal aspect}

\includegraphics{Brachiopod_phylogeny_files/figure-latex/character-mapping-13.pdf}

\begin{quote}
\textbf{Character 13: Adult setae: Secretion: Microvillar canal aspect}

1: Round\\
2: Polygonal; close-packed\\
Transformational character.
\end{quote}

\citet{Luter2000} distinguishes between the polygonal outline of
microvillar canals in adult brachiopod setae and the oval outline of
larval setae.

\hypertarget{Amathia-coding-13}{}
\emph{Amathia}: Polygonal \citep{Gordon1975}.

\hypertarget{Lingula-coding-13}{}
\emph{Lingula}: Polygonal \citep{Luter2000}.

\hypertarget{Pelagodiscus_atlanticus-coding-13}{}
\emph{Pelagodiscus atlanticus}: Reported as hexagonal in Discina
\citep{Luter2000}.

\hypertarget{Serpula-coding-13}{}
\emph{Serpula}: Annelid setae are prominently round, with gaps between
each microvillar chamber \citep{Orrhage1971}.

\hypertarget{Terebratulina-coding-13}{}
\emph{Terebratulina}: Polygonal in \emph{Calloria} \citep{Luter2000}.

\hypertarget{Tonicella-coding-13}{}
\emph{Tonicella}: Round \citep{Fischer1980}.

\hypertarget{Wiwaxia_corrugata-coding-13}{}
\emph{Wiwaxia corrugata}: The loose spacing of pyrite infills of
microvillar canals in \emph{Wiwaxia} sclerites \citep{Smith2014} argues
against a close-packed arrangement.

\subsection*{{[}14{]} Composition}\label{composition}
\addcontentsline{toc}{subsection}{{[}14{]} Composition}

\includegraphics{Brachiopod_phylogeny_files/figure-latex/character-mapping-14.pdf}

\begin{quote}
\textbf{Character 14: Adult setae: Composition}

1: Chitin\\
2: Horny protein\\
Transformational character.
\end{quote}

The majority of lophotrochozoan sclerites are chitinous, occasionally
hosting secondary biominerals.

\hypertarget{Sipunculus-coding-14}{}
\emph{Sipunculus}: Enzymatic test for chitin proved negative
\citep{Rice1993}.

\subsection*{{[}15{]} Enamel}\label{enamel}
\addcontentsline{toc}{subsection}{{[}15{]} Enamel}

\includegraphics{Brachiopod_phylogeny_files/figure-latex/character-mapping-15.pdf}

\begin{quote}
\textbf{Character 15: Adult setae: Enamel}

0: Absent\\
1: Present\\
Neomorphic character.
\end{quote}

Certain setae are encapsulated in a 20 nm wide electron dense layer,
termed ``enamel'' by \citet{Gustus1973}. Enamel may be absent in larval
setae \citep{Luter2003}; this character refers to the condition in adult
setae.

\hypertarget{Amathia-coding-15}{}
\emph{Amathia}: Not evident \citep{Gordon1975}.

\hypertarget{Pelagodiscus_atlanticus-coding-15}{}
\emph{Pelagodiscus atlanticus}: Enamel layer apparent in Discina
\citep[fig. 47.1]{Williams1997Introduction}.

\hypertarget{Serpula-coding-15}{}
\emph{Serpula}: Present in \emph{Nereis} \citep{Gustus1973}.

\hypertarget{Terebratulina-coding-15}{}
\emph{Terebratulina}: Present in the terebratulid \emph{Calloria} (and
the Rhynchonellid \emph{Notosaria}) \citep{Luter2000}.

\hypertarget{Tonicella-coding-15}{}
\emph{Tonicella}: Not evident \citep{Leise1988, Fischer1980}.

\subsection*{{[}16{]} Distribution}\label{distribution}
\addcontentsline{toc}{subsection}{{[}16{]} Distribution}

\includegraphics{Brachiopod_phylogeny_files/figure-latex/character-mapping-16.pdf}

\begin{quote}
\textbf{Character 16: Adult setae: Distribution}

1: Uniform\\
2: Only present at margins of shell\\
3: In bundles, repeated on each metamere if serial repetition present\\
4: In digestive tract\\
Transformational character.
\end{quote}

Setae penetrate the valves of many brachiopods. In certain taxa, they
are apparent only at the margins of the valves, in association with the
commissure, being reduced or lost over the surface of the shell.

\hypertarget{Eccentrotheca-coding-16}{}
\emph{Eccentrotheca}: Skovsted \emph{et al}.
\citeyearpar{Skovsted2011Scleritomeconstruction} assumed the setae may
have been present along the margin of the adapical opening, but there is
no fossil evidence.

\hypertarget{Heliomedusa_orienta-coding-16}{}
\emph{Heliomedusa orienta}: Throughout the shell -- see
\citet{Williams2007Supplement} -- causing the pustulose appearance
remarked upon by \citet{Chen2007Reinterpretationof}.

\hypertarget{Lingulellotreta_malongensis-coding-16}{}
\emph{Lingulellotreta malongensis}: At margin of shell
\citep{Zhang2005}.

\hypertarget{Tonicella-coding-16}{}
\emph{Tonicella}: Uniformly distributed around girdle (though not within
shell) with no serial repetition \citep{Vinther2005, Leise1988}.

\subsection*{{[}17{]} Constitution}\label{constitution}
\addcontentsline{toc}{subsection}{{[}17{]} Constitution}

\includegraphics{Brachiopod_phylogeny_files/figure-latex/character-mapping-17.pdf}

\begin{quote}
\textbf{Character 17: Adult setae: Constitution}

1: Solid, blade-like\\
2: Basal invagination\\
Transformational character.
\end{quote}

Sipunculan ``setae'' are basally invaginated, suggesting that they may
not be homologous with annelid chaetae.

\hypertarget{Amathia-coding-17}{}
\emph{Amathia}: Cytoplasmic intrusion into a central cavity
\citep{Gordon1975}.

\subsection*{{[}18{]} Projecting knobs}\label{projecting-knobs}
\addcontentsline{toc}{subsection}{{[}18{]} Projecting knobs}

\includegraphics{Brachiopod_phylogeny_files/figure-latex/character-mapping-18.pdf}

\begin{quote}
\textbf{Character 18: Adult setae: Projecting knobs}

0: Absent\\
1: Individual peripheral projections present\\
Neomorphic character.
\end{quote}

Terebratulids and discinids instead exhibit knob-like individual spines
These are distinct from the rings of spines that fringe lingulid setae.

\hypertarget{Lingula-coding-18}{}
\emph{Lingula}: \citet{Luter2000}.

\hypertarget{Pelagodiscus_atlanticus-coding-18}{}
\emph{Pelagodiscus atlanticus}: \emph{Discinisca} sports individual
peripheral spines \citep{Williams1997Introduction, Luter2003}.\\
Note that the ``embryonic'' setae of Discinids correspond to the
``larval setae'' of other brachiopods, and the ``larval setae'' of
juvenile discinids correspond to adult setae \citep{Luter2003}.

\hypertarget{Serpula-coding-18}{}
\emph{Serpula}: Surface smooth \citep{Sun2012}.

\hypertarget{Terebratulina-coding-18}{}
\emph{Terebratulina}: Individual peripheral spines \citep[in
\emph{Calloria};][]{Luter2000}.

\subsection*{{[}19{]} Circling coronae}\label{circling-coronae}
\addcontentsline{toc}{subsection}{{[}19{]} Circling coronae}

\includegraphics{Brachiopod_phylogeny_files/figure-latex/character-mapping-19.pdf}

\begin{quote}
\textbf{Character 19: Adult setae: Circling coronae}

0: Absent\\
1: Present\\
Neomorphic character.
\end{quote}

Lingulid setae bear crown-like rings of fine spines delimiting vertical
sections, recalling the nodes of \emph{Equisetum} stems. These arise by
the addition of an additional circlet of microvilli \citep[see][fig.
1e]{Luter2000}.

\section{Body organization}\label{body-organization}

\subsection*{{[}20{]} Serial repetition}\label{serial-repetition}
\addcontentsline{toc}{subsection}{{[}20{]} Serial repetition}

\includegraphics{Brachiopod_phylogeny_files/figure-latex/character-mapping-20.pdf}

\begin{quote}
\textbf{Character 20: Body organization: Serial repetition}

0: Absent\\
1: Present\\
Neomorphic character.
\end{quote}

Serial repetition in adult, whether expressed in valves, soft tissues or
exoskeletal elements. See character 13 in \citet{Rouse1999}; 19 in
\citet{Vinther2008}; 38 in \citet{Haszprunar1996}; 40--41 in
\citet{Sutton2012}; \citet{Wanninger2009}.

\hypertarget{Bactrotheca-coding-20}{}
\emph{Bactrotheca}, \emph{Conotheca}, \emph{Maxilites},
\emph{Pauxillites}: The soft anatomy of this taxon is unknown, so it is
impossible to rule out the presence of serial repetition.

\hypertarget{Dailyatia-coding-20}{}
\emph{Dailyatia}: Unknown whether sclerites are serially repeated, or
whether metameres were present in underlying soft anatomy.

\hypertarget{Halkieria_evangelista-coding-20}{}
\emph{Halkieria evangelista}: Elements of the \emph{Halkieria}
scleritome adhere to a quincunx arrangement, with different spacing of
elements in each zone; there is no evidence of a metameric arrangement.

\hypertarget{Namacalathus-coding-20}{}
\emph{Namacalathus}: Not evident.

\subsection*{{[}21{]} Foot}\label{foot}
\addcontentsline{toc}{subsection}{{[}21{]} Foot}

\includegraphics{Brachiopod_phylogeny_files/figure-latex/character-mapping-21.pdf}

\begin{quote}
\textbf{Character 21: Body organization: Foot}

0: Absent\\
1: Present\\
Neomorphic character.
\end{quote}

See characters 8 in \citet{Haszprunar1996}; 4 in \citet{Vinther2008};
137 in \citet{Rouse1999}; 21 in \citet{BucklandNicks2008}; 37 in
\citet{Sutton2012}; 1, 3 and 4 in \citet{Haszprunar2008}.\\
It is assumed that the adult foot is homologous with (and thus
contingent on) the larval foot.

\hypertarget{Cotyledion_tylodes-coding-21}{}
\emph{Cotyledion tylodes}: The stalk may conceivably be homologous with
the entoproct foot, but the evidence for homology is weak.

\hypertarget{Halkieria_evangelista-coding-21}{}
\emph{Halkieria evangelista}: The ventral surface of \emph{Halkieria} is
unarmoured, but its soft anatomy is unknown.

\hypertarget{Loxosomella-coding-21}{}
\emph{Loxosomella}: See \citet{Haszprunar2008}.

\hypertarget{Sipunculus-coding-21}{}
\emph{Sipunculus}: LISTED AS PRESENT IN \citet{Smith2012}: WHY?.

\subsection*{{[}22{]} Coelom}\label{coelom}
\addcontentsline{toc}{subsection}{{[}22{]} Coelom}

\includegraphics{Brachiopod_phylogeny_files/figure-latex/character-mapping-22.pdf}

\begin{quote}
\textbf{Character 22: Body organization: Coelom}

0: Absent: adults acoelomate\\
1: Present: true coelomic cavities differentiated\\
Neomorphic character.
\end{quote}

\hypertarget{Bactrotheca-coding-22}{}
\emph{Bactrotheca}, \emph{Conotheca}, \emph{Maxilites},
\emph{Pauxillites}: The soft anatomy of this taxon is unknown, so it is
impossible to determine the presence of a coelom.

\hypertarget{Flustra-coding-22}{}
\emph{Flustra}: ``Adult ectoprocts differentiate true coelomic
cavities'' -- \citet{Fuchs2008}.

\hypertarget{Haplophrentis_carinatus-coding-22}{}
\emph{Haplophrentis carinatus}: Internal cavities indicated by
differentiation of internal organs
\citep[see][]{Moysiuk2017Hyolithsare}.

\hypertarget{Loxosomella-coding-22}{}
\emph{Loxosomella}: ``Adult entoprocts are acoelomate'' --
\citet{Fuchs2008}.

\hypertarget{Phoronis-coding-22}{}
\emph{Phoronis}: \citet{Temereva2017Innervationof}.

\subsection*{{[}23{]} Number}\label{number-1}
\addcontentsline{toc}{subsection}{{[}23{]} Number}

\includegraphics{Brachiopod_phylogeny_files/figure-latex/character-mapping-23.pdf}

\begin{quote}
\textbf{Character 23: Body organization: Coelomoducts: Number}

1: Single\\
2: Multiple\\
Transformational character.
\end{quote}

Character 27 in \citet{Haszprunar2000}. Coelomoducts are excretory
organs derived from the coelom, also in some cases serving as genital
ducts (gonoducts); they replace (and may resemble) nephridia
\citep{Goodrich1945}.

\hypertarget{Flustra-coding-23}{}
\emph{Flustra}: Multiple ciliated ducts leading to a common gonopore
\citep{Goodrich1945}.

\hypertarget{Loxosomella-coding-23}{}
\emph{Loxosomella}: Two coelomoducts pass outwards, meet, and open by a
common pore \citep{Goodrich1945}.

\hypertarget{Phoronis-coding-23}{}
\emph{Phoronis}: ``large coelomic funnels serving as genital ducts''
\citep{Goodrich1945}.

\subsection*{{[}24{]} Gills}\label{gills}
\addcontentsline{toc}{subsection}{{[}24{]} Gills}

\includegraphics{Brachiopod_phylogeny_files/figure-latex/character-mapping-24.pdf}

\begin{quote}
\textbf{Character 24: Body organization: Gills}

0: Absent\\
1: Present\\
Neomorphic character.
\end{quote}

Gills (or ctenidia) surround the molluscan foot.\\
Character 1.59--60, 2.09, 4.49 in \citet{SPS1996}; 10--11 in
\citet{Haszprunar2000}; 45 in \citet{Sutton2012}.

\subsection*{{[}25{]} Circulatory system}\label{circulatory-system}
\addcontentsline{toc}{subsection}{{[}25{]} Circulatory system}

\includegraphics{Brachiopod_phylogeny_files/figure-latex/character-mapping-25.pdf}

\begin{quote}
\textbf{Character 25: Body organization: Circulatory system}

1: Epithelially lined\\
2: Poorly defined, involving sinuses and lacunae\\
3: Closed circulatory system\\
Transformational character.
\end{quote}

After character 23 in \citet{Haszprunar1996}; 24 in
\citet{Haszprunar2000}; 41 in \citet{Rouse1999}; 16 in
\citet{Scheltema1993}; 16 in \citet{Vinther2008}; 5 in
\citet{Haszprunar2008}.

\hypertarget{Amathia-coding-25}{}
\emph{Flustra}, \emph{Amathia}: As Brachiopods, sipunculans and
relatives \citep{Ruppert1983}.

\hypertarget{Dentalium-coding-25}{}
\emph{Loxosomella}, \emph{Tonicella}, \emph{Dentalium}: See
\citet{Haszprunar2008}.

\hypertarget{Sipunculus-coding-25}{}
\emph{Sipunculus}: Open circulatory system.

\section{Pedicle}\label{pedicle}

\subsection*{{[}26{]} Presence}\label{presence}
\addcontentsline{toc}{subsection}{{[}26{]} Presence}

\includegraphics{Brachiopod_phylogeny_files/figure-latex/character-mapping-26.pdf}

\begin{quote}
\textbf{Character 26: Pedicle: Presence}

0: Absent\\
1: Present\\
Neomorphic character.
\end{quote}

The brachiopod pedicle is a fleshy protuberance that emerges from the
posterior part of the body wall -- as denoted in fossil taxa by its
occurrence between the dorsal and ventral valves.

It is important to distinguish the pedicle from the ``pedicle sheath'',
a tubular extension of the umbo that grows by accretion from an isolated
portion of the ventral mantle. For discussion see
\citet{Holmer2018Theattachment} and \citet{Bassett2017Earliestontogeny}.

\hypertarget{Acanthotretella_spinosa-coding-26}{}
\emph{Acanthotretella spinosa}: The attachment structure of
\emph{Acanthotretella} originates at the margin of the dorsal and
ventral valves; although it emerges from the umbo of the ventral valve,
the presence of an internal pedicle tube betrays its identity as a
pedicle, rather than a pedicle sheath.

The pedicle of \emph{Acanthotretella} emerges from a short extension of
the umbo of the ventral valve. This extension is contiguous with the
valve and presumably grew by accretion; its position and continuity with
the valve suggest its interpretation as a pedicle sheath that is
superseded as an attachment structure. On the other hand, its continuity
with the internal pedicle tube suggests that is may represent an
independent organ.

\hypertarget{Bactrotheca-coding-26}{}
\emph{Bactrotheca}: The apex of \emph{Bactrotheca} \emph{deleta} is
pointed (Novak, 1891).

\hypertarget{Botsfordia-coding-26}{}
\emph{Botsfordia}: Pedicle foramen was not necessarily occupied by a
pedicle (though it presumably was).

\hypertarget{Clupeafumosus_socialis-coding-26}{}
\emph{Clupeafumosus socialis}: A pedicle was presumably present, but
only the foramen is preserved.

\hypertarget{Cotyledion_tylodes-coding-26}{}
\emph{Cotyledion tylodes}: The stalk is conceivably homologous with the
brachiopod pedicle, but this possibility is impossible to test.

\hypertarget{Craniops-coding-26}{}
\emph{Craniops}: Attached apically by cementation.

\hypertarget{Cupitheca_holocyclata-coding-26}{}
\emph{Cupitheca holocyclata}: Not possible to reconcile with
decollation.

\hypertarget{Flustra-coding-26}{}
\emph{Flustra}: Grows directly onto the substrate.

\hypertarget{Heliomedusa_orienta-coding-26}{}
\emph{Heliomedusa orienta}: ``It seems unlikely that \emph{H. orienta}
possessed a pedicle that attached it to\\
the soft seafloor, like most other Chengjiang brachiopods.'' \ldots{}\\
``The putative pedicle illustrated by Chen \emph{et al}.
\citeyearpar[Figs 4, 6, 7]{Chen2007Reinterpretationof} in fact is the
mold of a three-dimensionally preserved visceral cavity'' --
\citet{Zhang2009Architectureand}.

\hypertarget{Lingulosacculus-coding-26}{}
\emph{Lingulosacculus}: The absence of a pedicle is inferred from the
absence of an internal pedicle tube, and the absence of a pedicle at the
hinge.

\hypertarget{Loxosomella-coding-26}{}
\emph{Loxosomella}: The stalk corresponds to the molluscan foot, rather
than a pedicle.

\hypertarget{Mickwitzia_muralensis-coding-26}{}
\emph{Mickwitzia muralensis}: An attachment structure is inferred based
on the presence of an opening \citep{Balthasar2004Shellstructure}; this
is assumed to have been homologous with the brachiopod pedicle.

\hypertarget{Micrina-coding-26}{}
\emph{Micrina}: The prominent foramen between artificially articulated
valves is taken to imply the presence of a pedicle
\citep{Holmer2008TheEarly}.

\hypertarget{Micromitra-coding-26}{}
\emph{Micromitra}: The presence of a pedicle is indicated by the
propensity of \emph{Micromitra} to attach to hard substrates, such as
sponge spicules \citep{Holmer2006Aspinose}.

\hypertarget{Namacalathus-coding-26}{}
\emph{Namacalathus}: There is no obvious way to homologise the
attachment structure with the ventral pedicle of brachiopods.

\hypertarget{Nisusia_sulcata-coding-26}{}
\emph{Nisusia sulcata}: Has a pedicle, rather than a pedicle sheath as
in \emph{Kutorgina}
\citep{Holmer2018Evolutionarysignificance, Holmer2018Theattachment}.

\hypertarget{Paterimitra-coding-26}{}
\emph{Paterimitra}: ``\emph{Paterimitra} is interpreted to have attached
to hard substrates via a pedicle that emerged through the small
posterior opening'' -- \citet{Skovsted2009Thescleritome}.

\hypertarget{Pauxillites-coding-26}{}
\emph{Pauxillites}: Uncertain: specimens in \citet{Valent2015} do not
unambiguously show apical termination.

\hypertarget{Phoronis-coding-26}{}
\emph{Phoronis}: The tube-bearing stalk of phoronids arises as an
eversion of the metastomal sac, a markedly different origin from the
brachiopod pedicle, which arises from a terminal attachment disc
\citep{Young2002}; the structures are of dubious homology.

\hypertarget{Siphonobolus_priscus-coding-26}{}
\emph{Siphonobolus priscus}: Presumed present, based on ventral foramen
with colleplax.

\hypertarget{Sipunculus-coding-26}{}
\emph{Sipunculus}: Absent; there is no clear basis to homologise the
larval attachment structure of certain sipunculans with a pedicle.

\subsection*{{[}27{]} Constitution}\label{constitution-1}
\addcontentsline{toc}{subsection}{{[}27{]} Constitution}

\includegraphics{Brachiopod_phylogeny_files/figure-latex/character-mapping-27.pdf}

\begin{quote}
\textbf{Character 27: Pedicle: Constitution}

1: Massive or uniform\\
2: Densely stacked tabular discs\\
Transformational character.
\end{quote}

The pedicle of certain chengjiang rhynchonelliforms comprises ``densely
stacked, three dimensionally preserved, tabular discs''
\citep{Holmer2018Evolutionarysignificance}.\\
This contrasts with the uniform (`massive') pedicles of living taxa.

\hypertarget{Antigonambonites_planus-coding-27}{}
\emph{Antigonambonites planus}: Biomineralized
\citep{Holmer2018Evolutionarysignificance}.

\hypertarget{Terebratulina-coding-27}{}
\emph{Terebratulina}: Extant rhynconellid pedicles are massive,
consisting of a thick outer chitinous cuticle, a pedicle epithelium, and
a core composed of collagen fibres and cartilage-like connective tissue
\citep{Holmer2018Evolutionarysignificance}.

\subsection*{{[}28{]} Biomineralization}\label{biomineralization}
\addcontentsline{toc}{subsection}{{[}28{]} Biomineralization}

\includegraphics{Brachiopod_phylogeny_files/figure-latex/character-mapping-28.pdf}

\begin{quote}
\textbf{Character 28: Pedicle: Biomineralization}

1: Absent\\
2: Present\\
Transformational character.
\end{quote}

\hypertarget{Kutorgina_chengjiangensis-coding-28}{}
\emph{Kutorgina chengjiangensis}: The tabular discs that make up the
pedicle ``clearly have a pronounced three-dimensional preservation and
may have been partly mineralized.'' --\citet{Holmer2018Theattachment}.

\hypertarget{Micromitra-coding-28}{}
\emph{Micromitra}: A pedicle has not been observed in biomineralized
material \citep{Williams1998Thediversity}, indicating an originally
non-mineralized constitution.

\subsection*{{[}29{]} Bulb}\label{bulb}
\addcontentsline{toc}{subsection}{{[}29{]} Bulb}

\includegraphics{Brachiopod_phylogeny_files/figure-latex/character-mapping-29.pdf}

\begin{quote}
\textbf{Character 29: Pedicle: Bulb}

0: Absent\\
1: Present\\
Neomorphic character.
\end{quote}

A bulb is an expanded region of the distal pedicle, often embedded into
the sediment to improve anchorage.

\hypertarget{Acanthotretella_spinosa-coding-29}{}
\emph{Acanthotretella spinosa}: Holmer and Caron
\citeyearpar{Holmer2006Aspinose} interpret the presence of a bulb as
tentative; we score it as ambiguous.

\subsection*{{[}30{]} Distal rootlets}\label{distal-rootlets}
\addcontentsline{toc}{subsection}{{[}30{]} Distal rootlets}

\includegraphics{Brachiopod_phylogeny_files/figure-latex/character-mapping-30.pdf}

\begin{quote}
\textbf{Character 30: Pedicle: Distal rootlets}

0: Absent\\
1: Present\\
Neomorphic character.
\end{quote}

Observed in \emph{Pedunculotheca} and \emph{Bethia}
\citep{Sutton2005Silurianbrachiopods}.

\subsection*{{[}31{]} Tapering}\label{tapering}
\addcontentsline{toc}{subsection}{{[}31{]} Tapering}

\includegraphics{Brachiopod_phylogeny_files/figure-latex/character-mapping-31.pdf}

\begin{quote}
\textbf{Character 31: Pedicle: Tapering}

1: Uniform thickness\\
2: Tapering\\
Transformational character.
\end{quote}

Holmer \emph{et al}. \citeyearpar{Holmer2018Theattachment} remark that
the tapering aspect of the \emph{Nisusia} pedicle recalls that of
certain Chengjiang taxa (\emph{Alisina}, \emph{Longtancunella}) whilst
distinguishing it from many other taxa (\emph{Eichwaldia},
\emph{Bethia}) in which the pedicle is a constant thickness.

\hypertarget{Antigonambonites_planus-coding-31}{}
\emph{Antigonambonites planus}: Tapered pedicle sheath with holdfast.

\hypertarget{Pedunculotheca_diania-coding-31}{}
\emph{Pedunculotheca diania}: The pedicle thickness gradually typering
from the apex of the shell to the holdfast.

\subsection*{{[}32{]} Coelomic region}\label{coelomic-region}
\addcontentsline{toc}{subsection}{{[}32{]} Coelomic region}

\includegraphics{Brachiopod_phylogeny_files/figure-latex/character-mapping-32.pdf}

\begin{quote}
\textbf{Character 32: Pedicle: Coelomic region}

1: Absent\\
2: Present\\
Transformational character.
\end{quote}

Certain brachiopods, such as \emph{Acanthotretella}, exhibit a coelomic
cavity within the pedicle or pedicle sheath.

Treated as transformational as it is not clear that either state is
necessarily ancestral.

\hypertarget{Nisusia_sulcata-coding-32}{}
\emph{Nisusia sulcata}: A coleomic canal is inferred based on the ease
with which the pedicle is deformed
\citep{Holmer2018Evolutionarysignificance}, but its presence is not
known for certain so is coded ambiguous.

\subsection*{{[}33{]} Surface ornament}\label{surface-ornament-1}
\addcontentsline{toc}{subsection}{{[}33{]} Surface ornament}

\includegraphics{Brachiopod_phylogeny_files/figure-latex/character-mapping-33.pdf}

\begin{quote}
\textbf{Character 33: Pedicle: Surface ornament}

1: Smooth\\
2: Regular annulations\\
3: Irregular wrinkles\\
Transformational character.
\end{quote}

Annulations are regular rings that surround the pedicle, and are
distinguished from wrinkles, which are irregular in magnitude and
spacing, and may branch or fail to entirely encircle the pedicle.

\hypertarget{Acanthotretella_spinosa-coding-33}{}
\emph{Acanthotretella spinosa}: ``The pedicle surface is ornamented with
pronounced annulated rings, disposed at intervals of about 0.2 mm''.

\hypertarget{Alisina-coding-33}{}
\emph{Alisina}: ``It appears that the pedicle lacks a coelomic space and
is distinctly annulated, with densely stacked tabular bodies'' --
\citet{Zhang2011Anobolellate}.

\hypertarget{Antigonambonites_planus-coding-33}{}
\emph{Antigonambonites planus}: ``The emerging pedicle has a consistent
shape in all the available specimens and is strongly annulated and
distally tapering'' -- \citet{Holmer2018Evolutionarysignificance}.

\hypertarget{Kutorgina_chengjiangensis-coding-33}{}
\emph{Kutorgina chengjiangensis}: ``Pronounced concentric annular discs
disposed at intervals of 0.6--1.0 mm'' --
\citet{Zhang2007Rhynchonelliformeanbrachiopods}.

\hypertarget{Lingulellotreta_malongensis-coding-33}{}
\emph{Lingulellotreta malongensis}: Regularly annotated \citep[see fig.
14.9 in][]{Hou2017Brachiopoda}.

\hypertarget{Longtancunella_chengjiangensis-coding-33}{}
\emph{Longtancunella chengjiangensis}: ``The preserved pedicle has
condensed annulations'' -- \citet{Zhang2011Theexceptionally}.

\hypertarget{Nisusia_sulcata-coding-33}{}
\emph{Nisusia sulcata}: The ``strong annulations'' vary significantly in
transverse thickness \citep{Holmer2018Evolutionarysignificance}, so it
is not clear whether these represent true annulations or wrinkles.

\hypertarget{Yuganotheca_elegans-coding-33}{}
\emph{Yuganotheca elegans}: Annulations present in median collar.

\subsection*{{[}34{]} Nerve impression}\label{nerve-impression}
\addcontentsline{toc}{subsection}{{[}34{]} Nerve impression}

\includegraphics{Brachiopod_phylogeny_files/figure-latex/character-mapping-34.pdf}

\begin{quote}
\textbf{Character 34: Pedicle: Nerve impression}

0: Absent\\
1: Present\\
Neomorphic character.
\end{quote}

In certain taxa the impression of the pedicle nerve is evident in the
shell. See character 28 in Williams \emph{et al}.
\citeyearpar{Williams1998Thediversity} appendix 1. Care must be taken
not to code an impression as absent when the preservational quality is
insufficient to safely infer a genuine absence. Treated as neomorphic as
the presence of an innervation is considered a derived state.

\hypertarget{Alisina-coding-34}{}
\emph{Alisina}: Not described by
\citet{Williams2000LinguliformeaCraniiformea}.

\hypertarget{Askepasma_toddense-coding-34}{}
\emph{Askepasma toddense}, \emph{Micromitra}, \emph{Glyptoria},
\emph{Kutorgina chengjiangensis}, \emph{Salanygolina}: Following
\citet{Williams1998Thediversity}, appendix 2.

\hypertarget{Botsfordia-coding-34}{}
\emph{Botsfordia}: Documented by \citet{Skovsted2017Depthrelated}.

\hypertarget{Clupeafumosus_socialis-coding-34}{}
\emph{Clupeafumosus socialis}: Coded as absent in Acrotretidae
\citep[table 6]{Williams2000LinguliformeaCraniiformea}.

\hypertarget{Lingula-coding-34}{}
\emph{Lingula}: Present in many lingulids
\citep{Williams2000LinguliformeaCraniiformea}, and coded as present in
Lingulidae \citep[table 6]{Williams2000LinguliformeaCraniiformea}.

\hypertarget{Lingulellotreta_malongensis-coding-34}{}
\emph{Lingulellotreta malongensis}: Coded as present in
Lingulellotretidae \citep[table
6]{Williams2000LinguliformeaCraniiformea}.

\hypertarget{Mummpikia_nuda-coding-34}{}
\emph{Mummpikia nuda}: Balthasar
\citeyearpar[p.~274]{Balthasar2008iMummpikia} identifies a canal as a
probable impression of a pedicle nerve.

\hypertarget{Nisusia_sulcata-coding-34}{}
\emph{Nisusia sulcata}, \emph{Orthis}: Not reported in
\citet{Williams2000LinguliformeaCraniiformea}.

\hypertarget{Pelagodiscus_atlanticus-coding-34}{}
\emph{Pelagodiscus atlanticus}: Coded as present in Discinidae
\citep[table 6]{Williams2000LinguliformeaCraniiformea}.

\hypertarget{Siphonobolus_priscus-coding-34}{}
\emph{Siphonobolus priscus}: Coded as absent in Siphonotretidae
\citep[table 6]{Williams2000LinguliformeaCraniiformea}.

\section{Mantle canals}\label{mantle-canals}

\subsection*{{[}35{]} Presence}\label{presence-1}
\addcontentsline{toc}{subsection}{{[}35{]} Presence}

\includegraphics{Brachiopod_phylogeny_files/figure-latex/character-mapping-35.pdf}

\begin{quote}
\textbf{Character 35: Mantle canals: Presence}

0: Absent\\
1: Present\\
Neomorphic character.
\end{quote}

Whether impressed on a shell or expressed solely in soft tissue.

\hypertarget{Cupitheca_holocyclata-coding-35}{}
\emph{Cupitheca holocyclata}: Not observed despite excellent
preservation.

\hypertarget{Paramicrocornus-coding-35}{}
\emph{Paramicrocornus}: Not impressed on valves, despite fine
preservation of muscular attachment \citep{Zhang2018Ahyolithid}.

\hypertarget{Paterimitra-coding-35}{}
\emph{Paterimitra}: Not evident.

\hypertarget{Ussunia-coding-35}{}
\emph{Ussunia}: Not preserved along muscle scars \citep{Nikitin1984},
presumably owing to quality of preservation rather than genuine absence.

\subsection*{{[}36{]} Morphology}\label{morphology-1}
\addcontentsline{toc}{subsection}{{[}36{]} Morphology}

\includegraphics{Brachiopod_phylogeny_files/figure-latex/character-mapping-36.pdf}

\begin{quote}
\textbf{Character 36: Mantle canals: Morphology}

1: Pinnate (=lemniscate)\\
2: Bifurcate\\
3: Baculate\\
4: Saccate\\
Transformational character.
\end{quote}

The morphology of dorsal and ventral canals is identical in all included
taxa, so is assumed not to be independent -- hence the use of a single
character \citep[contra][]{Williams2000LinguliformeaCraniiformea}.

For a description of terms see Williams \emph{et al}.
\citeyearpar[2000]{Williams1997Introduction}.

Pinnate = ``rapidly branch into a number of subequal, radially disposed
canals''\\
Bifurcate = ``\emph{vascula} \emph{lateralia} in both valves divide
immediately after leaving the body cavity''\\
Baculate = ``extend forward without any major dichotomy or bifurcation''
\citep[ p.~418]{Williams1997Introduction}\\
Saccate = ``pouchlike sinuses lying wholly posterior to the arcuate
\emph{vascula} \emph{media}'' (ibid., p412).

\hypertarget{Acanthotretella_spinosa-coding-36}{}
\emph{Acanthotretella spinosa}: Following Table 6, for Siphonotretidae,
in Williams \emph{et al}.
\citeyearpar{Williams2000LinguliformeaCraniiformea}.

\hypertarget{Alisina-coding-36}{}
\emph{Alisina}, \emph{Nisusia sulcata}: Following Table 15 in Williams
\emph{et al}. \citeyearpar{Williams2000LinguliformeaCraniiformea}.

\hypertarget{Antigonambonites_planus-coding-36}{}
\emph{Antigonambonites planus}: Not reported in Treatise
\citep{Williams2000LinguliformeaCraniiformea}.

\hypertarget{Askepasma_toddense-coding-36}{}
\emph{Askepasma toddense}: Described as pinnate (at least in ventral
valve) by Williams \emph{et al}.
\citeyearpar[p.~250]{Williams1998Thediversity}.

\hypertarget{Botsfordia-coding-36}{}
\emph{Botsfordia}, \emph{Eoobolus}: Following
\citet{Williams1998Thediversity}, appendix 2, and Williams \emph{et al}.
\citeyearpar{Williams2000LinguliformeaCraniiformea}, table 8.

\hypertarget{Clupeafumosus_socialis-coding-36}{}
\emph{Clupeafumosus socialis}: Following Table 8 (for Acrotreta) in
Williams \emph{et al}.
\citeyearpar{Williams2000LinguliformeaCraniiformea}, and the general
pinnate condition for acrotretoids stated in Williams \emph{et al}.
\citeyearpar{Williams1997Introduction}, p.~420.

\hypertarget{Coolinia_pecten-coding-36}{}
\emph{Coolinia pecten}: Not reported in
\citet{Williams2000LinguliformeaCraniiformea}.

\hypertarget{Craniops-coding-36}{}
\emph{Craniops}: Not reported from fossil material.

\hypertarget{Gasconsia-coding-36}{}
\emph{Gasconsia}: Williams \emph{et al}. \citeyearpar[table
15]{Williams2000LinguliformeaCraniiformea} appear to use
Palaeotrimerella \citep[as drawn in][]{Williams1997Introduction} as a
model for \emph{Gasconsia}, which pre-supposes a close relationship. We
are not aware of any report of mantle canals from \emph{Gasconsia}
itself.

\hypertarget{Glyptoria-coding-36}{}
\emph{Glyptoria}: Following appendix 2 (char. 21) in Williams \emph{et
al}. \citeyearpar{Williams1998Thediversity}.

\hypertarget{Heliomedusa_orienta-coding-36}{}
\emph{Heliomedusa orienta}: Described as pinnate by Jin \& Wang
\citeyearpar{Jin1992Revisionof}.

\hypertarget{Kutorgina_chengjiangensis-coding-36}{}
\emph{Novocrania}, \emph{Kutorgina chengjiangensis}: Following table 15
in Williams \emph{et al}.
\citeyearpar{Williams2000LinguliformeaCraniiformea} (for
\emph{Neocrania}).

\hypertarget{Lingula-coding-36}{}
\emph{Lingula}, \emph{Lingulellotreta malongensis}: Following table 6 in
Williams \emph{et al}.
\citeyearpar{Williams2000LinguliformeaCraniiformea}.

\hypertarget{Lingulosacculus-coding-36}{}
\emph{Lingulosacculus}: Baculate \emph{vascula} \emph{media} --
Balthasar \& Butterfield \citeyearpar{Balthasar2009EarlyCambrian}.

\hypertarget{Longtancunella_chengjiangensis-coding-36}{}
\emph{Longtancunella chengjiangensis}: Reported by Zhang \emph{et al}.
\citeyearpar[2011T]{Zhang2007Agregarious} though the interpretation is
tentative.

\hypertarget{Micromitra-coding-36}{}
\emph{Micromitra}: Described as saccate by Williams \emph{et al}.
\citeyearpar{Williams1998Thediversity}.

\hypertarget{Mummpikia_nuda-coding-36}{}
\emph{Mummpikia nuda}: ``Poorly resolved'' --
\citet{Balthasar2008iMummpikia}.

\hypertarget{Orthis-coding-36}{}
\emph{Orthis}: Sacculate (sometimes digitate in dorsal valve)
\citep[p716]{Williams2000LinguliformeaCraniiformea}.

\hypertarget{Pelagodiscus_atlanticus-coding-36}{}
\emph{Pelagodiscus atlanticus}: Following table 6, for Discinidae, in
Williams \emph{et al}.
\citeyearpar{Williams2000LinguliformeaCraniiformea}.

\hypertarget{Salanygolina-coding-36}{}
\emph{Salanygolina}: Coded uncertain in appendix 2 in Williams \emph{et
al}. \citeyearpar{Williams1998Thediversity}.

\hypertarget{Siphonobolus_priscus-coding-36}{}
\emph{Siphonobolus priscus}: Interpreted as baculate, following
\citet{Havlicek1982LingulaceaPaterinacea}.

\hypertarget{Terebratulina-coding-36}{}
\emph{Terebratulina}: ``In modern terebratulides, the \emph{vascula}
\emph{media} are subordinate to the lemniscate or pinnate \emph{vascula}
\emph{genitalia}'' -- \citet{Williams1997Introduction}.

\hypertarget{Tomteluva_perturbata-coding-36}{}
\emph{Tomteluva perturbata}: Preservation not adequate to evaluate
\citep{Streng2016Anew}.

\subsection*{\texorpdfstring{{[}37{]} \emph{vascula}
\emph{lateralia}}{{[}37{]} vascula lateralia}}\label{vascula-lateralia}
\addcontentsline{toc}{subsection}{{[}37{]} \emph{vascula}
\emph{lateralia}}

\includegraphics{Brachiopod_phylogeny_files/figure-latex/character-mapping-37.pdf}

\begin{quote}
\textbf{Character 37: Mantle canals: \emph{vascula} \emph{lateralia}}

0: Absent\\
1: Present\\
Neomorphic character.
\end{quote}

We treat the \emph{vascula} \emph{lateralia} as equivalent to the
\emph{vascula} \emph{genitalia} of articulated brachiopods, allowing
phylogenetic analysis to test their proposed homology.

Williams \emph{et al}. \citeyearpar{Williams1997Introduction} write:
``The mantle canal system of most of the organophosphate-shelled species
consists of a single pair of main trunks in the ventral mantle
(\emph{vascula} \emph{lateralia}) and two pairs in the dorsal mantle,
one pair (\emph{vascula} \emph{lateralia}) occupying a similar position
to the single pair in the ventral mantle and a second pair projecting
from the body cavity near the midline of the valve. This latter pair may
be termed the \emph{vascula} \emph{media}, but whether they are strictly
homologous with the \emph{vascula} \emph{media} of articulated
brachiopods is a matter of opinion. It is also impossible to assert that
the \emph{vascula} \emph{lateralia} are the homologues of the
\emph{vascula} \emph{myaria} or \emph{genitalia} of articulated species,
although they are likely to be so as they arise in a comparable
position.''

``In inarticulated brachiopods, two main mantle canals (\emph{vascula}
\emph{lateralia}) emerge from the main body cavity through muscular
valves and bifurcate distally to produce an increasingly dense array of
blindly ending branches near the periphery of the mantle (fig.
71.1--71.2).''

\hypertarget{Acanthotretella_spinosa-coding-37}{}
\emph{Acanthotretella spinosa}: Following table 8 (which records
presence in Siphonotreta) in Williams \emph{et al}.
\citeyearpar{Williams2000LinguliformeaCraniiformea}.

\hypertarget{Alisina-coding-37}{}
\emph{Alisina}, \emph{Kutorgina chengjiangensis}, \emph{Nisusia
sulcata}: Following table 15 in Williams \emph{et al}.
\citeyearpar{Williams2000LinguliformeaCraniiformea}.

\hypertarget{Askepasma_toddense-coding-37}{}
\emph{Askepasma toddense}, \emph{Micromitra}: ``Laurie
\citeyearpar{Laurie1987Themusculature} has shown that arcuate
\emph{vascula} \emph{media} were present in the mantles of both valves
as were pouchlike \emph{vascula} \emph{genitalia}, especially in the
ventral valve'' -- \citet{Williams1997Introduction}.

\hypertarget{Botsfordia-coding-37}{}
\emph{Botsfordia}: Following Popov \citeyearpar{Popov1992TheCambrian}.

\hypertarget{Clupeafumosus_socialis-coding-37}{}
\emph{Clupeafumosus socialis}: Presence indicated in Table 8 (for
Acrotreta) in Williams \emph{et al}.
\citeyearpar{Williams2000LinguliformeaCraniiformea}.

\hypertarget{Gasconsia-coding-37}{}
\emph{Gasconsia}: Williams \emph{et al}. \citeyearpar[table
15]{Williams2000LinguliformeaCraniiformea} appear to use
Palaeotrimerella \citep[as drawn in][]{Williams1997Introduction} as a
model for \emph{Gasconsia}, which pre-supposes a close relationship. We
are not aware of any report of mantle canals from \emph{Gasconsia}
itself.

\hypertarget{Heliomedusa_orienta-coding-37}{}
\emph{Heliomedusa orienta}: Present: Williams \emph{et al}.
\citeyearpar{Williams2000LinguliformeaCraniiformea}; Jin \& Wang
\citeyearpar{Jin1992Revisionof}.

\hypertarget{Lingulellotreta_malongensis-coding-37}{}
\emph{Lingulellotreta malongensis}: Present
\citep{Williams2000LinguliformeaCraniiformea}.

\hypertarget{Longtancunella_chengjiangensis-coding-37}{}
\emph{Longtancunella chengjiangensis}: Presence is possible but requires
interpretation that is not unambiguous:

``In the dorsal valve, there can be seen two baculate grooves that arise
from the\\
anterior body wall at an antero-lateral position. These two grooves
(Figs 4H, 5D) could be taken to represent the \emph{vascula}
\emph{lateralia}'' -- \citet{Zhang2007Agregarious}.

\hypertarget{Novocrania-coding-37}{}
\emph{Novocrania}: Following table 15 in Williams \emph{et al}.
\citeyearpar{Williams2000LinguliformeaCraniiformea} (for
\emph{Neocrania}), who write that ``Holocene craniides have only a
single pair of main trunks in both valves, corresponding to the
\emph{vascula} \emph{lateralia}''. Williams \emph{et al}.
\citeyearpar{Williams2007Supplement} reiterate this position (p.~2875),
at least for the ventral valve.

\hypertarget{Orthis-coding-37}{}
\emph{Terebratulina}, \emph{Orthis}: = \emph{vascula} \emph{genitalia}.

\hypertarget{Pelagodiscus_atlanticus-coding-37}{}
\emph{Pelagodiscus atlanticus}: Following \emph{Lochkothele}
(Discinidae), Fig. 43.4a in Williams \emph{et al}.
\citeyearpar{Williams2000LinguliformeaCraniiformea}.

\hypertarget{Siphonobolus_priscus-coding-37}{}
\emph{Siphonobolus priscus}: Noted in \emph{Siphonobolus} by Williams
\emph{et al}. \citeyearpar{Williams2000LinguliformeaCraniiformea}, with
reference to Havlicek \citeyearpar{Havlicek1982LingulaceaPaterinacea}.

\hypertarget{Tomteluva_perturbata-coding-37}{}
\emph{Tomteluva perturbata}: Preservation not adequate to evaluate
\citep{Streng2016Anew}.

\hypertarget{Yuganotheca_elegans-coding-37}{}
\emph{Yuganotheca elegans}: Based on the figures and sketches in
\citet{Zhang2014Anearly} (and supplementary material), the mantle canals
are interpreted as lateral, with no clear \emph{vascula} \emph{media}
present.

\subsection*{\texorpdfstring{{[}38{]} \emph{vascula}
\emph{media}}{{[}38{]} vascula media}}\label{vascula-media}
\addcontentsline{toc}{subsection}{{[}38{]} \emph{vascula} \emph{media}}

\includegraphics{Brachiopod_phylogeny_files/figure-latex/character-mapping-38.pdf}

\begin{quote}
\textbf{Character 38: Mantle canals: \emph{vascula} \emph{media}}

0: Absent\\
1: Present (in dorsal valve)\\
Neomorphic character.
\end{quote}

Williams \emph{et al}. \citeyearpar{Williams1997Introduction} note that
in addition to the \emph{vascula} \emph{lateralia}, ``\emph{Discinisca}
has two additional mantle canals emanating from the body cavity into the
dorsal mantle (\emph{vascula} \emph{media}).''

These structures are only evident in the dorsal valve for the included
taxa, so only a single character is necessary.

\hypertarget{Acanthotretella_spinosa-coding-38}{}
\emph{Acanthotretella spinosa}: Following table 6 (for Siphonotretidae)
in Williams \emph{et al}.
\citeyearpar{Williams2000LinguliformeaCraniiformea}.

\hypertarget{Alisina-coding-38}{}
\emph{Alisina}, \emph{Kutorgina chengjiangensis}, \emph{Nisusia
sulcata}: Following table 15 in Williams \emph{et al}.
\citeyearpar{Williams2000LinguliformeaCraniiformea}.

\hypertarget{Askepasma_toddense-coding-38}{}
\emph{Askepasma toddense}: Following table 6 (for Paterinidae) in
Williams \emph{et al}.
\citeyearpar{Williams2000LinguliformeaCraniiformea}.

\hypertarget{Botsfordia-coding-38}{}
\emph{Botsfordia}: Following Popov \citeyearpar[fig.
2]{Popov1992TheCambrian}.

\hypertarget{Clupeafumosus_socialis-coding-38}{}
\emph{Clupeafumosus socialis}: Following \emph{Hadrotreta} schematic in
Williams \emph{et al}.
\citeyearpar{Williams2000LinguliformeaCraniiformea}.

\hypertarget{Eoobolus-coding-38}{}
\emph{Eoobolus}: Fig. 5 in \citet{Balthasar2009Thebrachiopod}.

\hypertarget{Gasconsia-coding-38}{}
\emph{Gasconsia}: Williams \emph{et al}. \citeyearpar[table
15]{Williams2000LinguliformeaCraniiformea} appear to use
Palaeotrimerella \citep[as drawn in][]{Williams1997Introduction} as a
model for \emph{Gasconsia}, which pre-supposes a close relationship. We
are not aware of any report of mantle canals from \emph{Gasconsia}
itself.

\hypertarget{Glyptoria-coding-38}{}
\emph{Glyptoria}: Present and divergent
\citep{Williams2000LinguliformeaCraniiformea}.

\hypertarget{Heliomedusa_orienta-coding-38}{}
\emph{Heliomedusa orienta}: Present: Williams \emph{et al}.
\citeyearpar{Williams2000LinguliformeaCraniiformea} p162, Jin \& Wang
\citeyearpar{Jin1992Revisionof}.

\hypertarget{Lingula-coding-38}{}
\emph{Lingula}, \emph{Lingulellotreta malongensis}: Following table 6 in
Williams \emph{et al}.
\citeyearpar{Williams2000LinguliformeaCraniiformea}.

\hypertarget{Longtancunella_chengjiangensis-coding-38}{}
\emph{Longtancunella chengjiangensis}: Reported by Zhang \emph{et al}.
\citeyearpar{Zhang2007Agregarious} though the interpretation is
tentative.

\hypertarget{Micromitra-coding-38}{}
\emph{Micromitra}: Reported by Williams \emph{et al}.
\citeyearpar{Williams1998Thediversity}.

\hypertarget{Novocrania-coding-38}{}
\emph{Novocrania}: Williams \emph{et al}.
\citeyearpar{Williams2000LinguliformeaCraniiformea} write ``Holocene
craniides have only a single pair of main trunks in both valves,
corresponding to the \emph{vascula} \emph{lateralia}'' -- an observation
reflected in their table 15 (for \emph{Neocrania}).\\
But in contrast, \citet{Williams2007Supplement}, p.~2875, identify the
dorsal valve's canals as a \emph{vascula} \emph{media} in living
cranidds (though both are \emph{lateralia} in Ordoviian craniides). This
character is therefore coded as ambiguous.

\hypertarget{Orthis-coding-38}{}
\emph{Orthis}: From idealised morphology in Williams \emph{et al}.
\citeyearpar{Williams2000LinguliformeaCraniiformea}.

\hypertarget{Pelagodiscus_atlanticus-coding-38}{}
\emph{Pelagodiscus atlanticus}: Following table 6 (for Discinidae) in
Williams \emph{et al}.
\citeyearpar{Williams2000LinguliformeaCraniiformea}.

\hypertarget{Siphonobolus_priscus-coding-38}{}
\emph{Siphonobolus priscus}: Noted in \emph{Siphonobolus} by Havlicek
\citeyearpar{Havlicek1982LingulaceaPaterinacea}.

\hypertarget{Terebratulina-coding-38}{}
\emph{Terebratulina}: ``In modern terebratulides, the \emph{vascula}
\emph{media} are subordinate to the lemniscate or pinnate \emph{vascula}
\emph{genitalia}'' -- \citet{Williams1997Introduction} p417.

\hypertarget{Tomteluva_perturbata-coding-38}{}
\emph{Tomteluva perturbata}: Preservation not adequate to evaluate
\citep{Streng2016Anew}.

\hypertarget{Yuganotheca_elegans-coding-38}{}
\emph{Yuganotheca elegans}: Based on the figures and sketches in
\citet{Zhang2014Anearly} (and supplementary material), the mantle canals
are interpreted as lateral, with no clear \emph{vascula} \emph{media}
present.

\subsection*{\texorpdfstring{{[}39{]} \emph{vascula}
\emph{terminalia}}{{[}39{]} vascula terminalia}}\label{vascula-terminalia}
\addcontentsline{toc}{subsection}{{[}39{]} \emph{vascula}
\emph{terminalia}}

\includegraphics{Brachiopod_phylogeny_files/figure-latex/character-mapping-39.pdf}

\begin{quote}
\textbf{Character 39: Mantle canals: \emph{vascula} \emph{terminalia}}

1: Exclusively marginal (peripheral)\\
2: Directed peripherally and (intero)medially\\
Transformational character.
\end{quote}

Presumed to be connected with setal follicles in life
\citep{Williams1998Thediversity}. See Williams \emph{et al}.
\citeyearpar{Williams2000LinguliformeaCraniiformea} for discussion.

\hypertarget{Acanthotretella_spinosa-coding-39}{}
\emph{Acanthotretella spinosa}: Preservation not clear enough to score
with certainty \citep{Holmer2006Aspinose}.

\hypertarget{Alisina-coding-39}{}
\emph{Alisina}: Interomedial \emph{vascula} \emph{terminalia} not
reported by Williams \emph{et al}.
\citeyearpar{Williams2000LinguliformeaCraniiformea}.

\hypertarget{Askepasma_toddense-coding-39}{}
\emph{Askepasma toddense}, \emph{Micromitra}: Peripheral only
\citep{Williams1998Thediversity, Williams2000LinguliformeaCraniiformea}.

\hypertarget{Botsfordia-coding-39}{}
\emph{Botsfordia}, \emph{Eoobolus}: Following
\citet{Williams1998Thediversity}, appendix 2.

\hypertarget{Glyptoria-coding-39}{}
\emph{Glyptoria}: Following appendix 2 in Williams \emph{et al}.
\citeyearpar{Williams1998Thediversity}.

\hypertarget{Heliomedusa_orienta-coding-39}{}
\emph{Heliomedusa orienta}: Inferred from Jin \& Wang
\citeyearpar{Jin1992Revisionof}.

\hypertarget{Kutorgina_chengjiangensis-coding-39}{}
\emph{Kutorgina chengjiangensis}, \emph{Salanygolina}: Coded uncertain
in appendix 2 in Williams \emph{et al}.
\citeyearpar{Williams1998Thediversity}.

\hypertarget{Lingula-coding-39}{}
\emph{Lingula}: Peripheral and medial for all Lingulata
\citep{Williams2000LinguliformeaCraniiformea}.

\hypertarget{Lingulellotreta_malongensis-coding-39}{}
\emph{Lingulellotreta malongensis}: Not described in Williams \emph{et
al}. \citeyearpar{Williams2000LinguliformeaCraniiformea}.

\hypertarget{Lingulosacculus-coding-39}{}
\emph{Lingulosacculus}: Strong indication of medially directed
\emph{vascula} \emph{terminalia} from \emph{vascula} \emph{lateralia};
see fig. 1.A1 in \citet{Balthasar2009EarlyCambrian}.

\hypertarget{Novocrania-coding-39}{}
\emph{Novocrania}: Peripheral only
\citep[p.158]{Williams2000LinguliformeaCraniiformea}.

\hypertarget{Orthis-coding-39}{}
\emph{Orthis}: See schematics in Williams \emph{et al}.
\citeyearpar{Williams2000LinguliformeaCraniiformea}.

\hypertarget{Pelagodiscus_atlanticus-coding-39}{}
\emph{Pelagodiscus atlanticus}: Following \emph{Lochkothele}
(Discinidae), fig. 43.4a in Williams \emph{et al}.
\citeyearpar{Williams2000LinguliformeaCraniiformea}.

\hypertarget{Siphonobolus_priscus-coding-39}{}
\emph{Siphonobolus priscus}: Not reported in
\citet{Havlicek1982LingulaceaPaterinacea} or
\citet{Williams2000LinguliformeaCraniiformea}.

\hypertarget{Terebratulina-coding-39}{}
\emph{Terebratulina}: Following idealised plectolophous terebratulid of
Emig \citeyearpar{Emig1992Functionaldisposition}.

\section{Perioral tentacular
apparatus}\label{perioral-tentacular-apparatus}

\subsection*{{[}40{]} Presence}\label{presence-2}
\addcontentsline{toc}{subsection}{{[}40{]} Presence}

\includegraphics{Brachiopod_phylogeny_files/figure-latex/character-mapping-40.pdf}

\begin{quote}
\textbf{Character 40: Perioral tentacular apparatus: Presence}

0: Absent\\
1: Present\\
Neomorphic character.
\end{quote}

The lophophore is a ring of tentacles that surrounds the mouth.
\citet{Temereva2017Innervationof} suggests that true lophophores must
also encompass the anus, which excludes the tentacular apparatus of
entoprocts from the definition; as homology between the tentacular
apparatuses of entoprocts and other lophophorates has often been
assumed, we prefer to take a more inclusive stance and code the
structures as potentially homologous.

It is unlikely that the tentacles of annelids and sipunculans correspond
to the lophophore, yet homology is not inconceivable. In order that the
tentacular apparatus of \emph{Haplophrentis} can be compared with both
organs without prejudice, we capture the presence of a tentacular
apparatus in this very broad character, with arguments against homology
reflected in separate transformation series.

\hypertarget{Cotyledion_tylodes-coding-40}{}
\emph{Cotyledion tylodes}: The tentacular crown \citep{Zhang2013} is
interpreted as a lophophore.

\hypertarget{Dentalium-coding-40}{}
\emph{Dentalium}: The scaphopod captacula is conceivably equivalent to
the tentacular apparatus of other lophotrochozoans. It is
developmentally pre-oral, and has tentatively been homologised with the
pre-oral tentacles of Monoplacophora and Gastropoda \citep{Steiner1992},
though their musculature and late development suggests instead that they
may derive from the molluscan foot, as do the arms of cephalopods
\citep{Wanninger2002M}.

\hypertarget{Haplophrentis_carinatus-coding-40}{}
\emph{Haplophrentis carinatus}: \citet{Moysiuk2017Hyolithsare}.

\subsection*{{[}41{]} Origin}\label{origin}
\addcontentsline{toc}{subsection}{{[}41{]} Origin}

\includegraphics{Brachiopod_phylogeny_files/figure-latex/character-mapping-41.pdf}

\begin{quote}
\textbf{Character 41: Perioral tentacular apparatus: Origin}

1: Prostomium (i.e.~anterior of larval prototroch)\\
2: Second pair of coelomic sacs, at metamorphosis\\
3: Mid-trunk, prior to metamorphosis\\
Transformational character.
\end{quote}

The tentacles of annelids and sipunculans originate from a dorsal pair
of buds on the prostomium \citep{Adrianov2006}, whereas the brachiopod
lophophore arises from the second pair of coelomic sacs
\citep{Nielsen1991}.

\hypertarget{Dentalium-coding-41}{}
\emph{Dentalium}: The captacula arise close to the mouth after
metamorphosis \citep{Wanninger2002M}, in a position not dissimilar from
that of the phoronid tentacles \citep{Santagata2004}.

\hypertarget{Flustra-coding-41}{}
\emph{Flustra}: The tentacles appear at metamorphosis, seemingly from
below the corona (=prototroch) \citep{Young2002}.

\hypertarget{Loxosomella-coding-41}{}
\emph{Loxosomella}: Arising after metamorphosis \citep{Nielsen1971}.

\hypertarget{Novocrania-coding-41}{}
\emph{Novocrania}: ``At metamorphosis {[}\ldots{}.{]} the second pair of
coelomic sacs develop small attachment areas at the edge of the dorsal
valve and become the lophophore coelom'' \citep{Nielsen1991}

``The larval lobes are retained during the first steps of metamorphosis
and are\\
subsequently remodeled to form the lophophore and other adult organs''
-- \citet{Altenburger2013}.

\hypertarget{Phoronis-coding-41}{}
\emph{Phoronis}: At the posterior of the head, at the late larval stage
\citep{Santagata2004}.

\hypertarget{Sipunculus-coding-41}{}
\emph{Sipunculus}: \citep{Adrianov2006}.

\hypertarget{Terebratulina-coding-41}{}
\emph{Terebratulina}: Lophophore of \emph{Terebratalia} arises post
metamorphosis \citep{Young2002}; lophophore conceivably arising from
vesicular bodies at base of apical lobe?.

\subsection*{{[}42{]} Tentacle disposition}\label{tentacle-disposition}
\addcontentsline{toc}{subsection}{{[}42{]} Tentacle disposition}

\includegraphics{Brachiopod_phylogeny_files/figure-latex/character-mapping-42.pdf}

\begin{quote}
\textbf{Character 42: Perioral tentacular apparatus: Tentacle
disposition}

1: Single side\\
2: Both sides\\
Transformational character.
\end{quote}

Tentacles may occur along one or both sides of the axis of the
lophophore arm \citep{Carlson1995Phylogeneticrelationships}.

\hypertarget{Acanthotretella_spinosa-coding-42}{}
\emph{Acanthotretella spinosa}: Preservation insufficient to evaluate
\citep{Holmer2006Aspinose}.

\hypertarget{Alisina-coding-42}{}
\emph{Alisina}, \emph{Lingulosacculus}, \emph{Longtancunella
chengjiangensis}: Preservation inadequate.

\hypertarget{Amathia-coding-42}{}
\emph{Amathia}: Single side \citep{Temereva2016Thenervous}.

\hypertarget{Cotyledion_tylodes-coding-42}{}
\emph{Cotyledion tylodes}: Tentacles seemingly occupy a single side of
the lophophore \citep{Zhang2013}.

\hypertarget{Dentalium-coding-42}{}
\emph{Dentalium}: On rim of basal lobe only \citep{Morton1959}.

\hypertarget{Flustra-coding-42}{}
\emph{Flustra}: Both sides \citep{Schwaha2015, Shunkina2015}.

\hypertarget{Heliomedusa_orienta-coding-42}{}
\emph{Heliomedusa orienta}: ``Each lophophoral arm bears a row of long,
slender flexible tentacles'' -- \citet{Zhang2009Architectureand}.

\hypertarget{Kutorgina_chengjiangensis-coding-42}{}
\emph{Kutorgina chengjiangensis}: Tentacles ``cannot be confidently
demonstrated in the available specimens.'' --
\citet{Zhang2007Rhynchonelliformeanbrachiopods}.

\hypertarget{Lingula-coding-42}{}
\emph{Novocrania}, \emph{Pelagodiscus atlanticus}, \emph{Lingula},
\emph{Terebratulina}, \emph{Phoronis}: Following coding for higher group
in \citet{Carlson1995Phylogeneticrelationships}, appendix 1, character
36.

\hypertarget{Lingulellotreta_malongensis-coding-42}{}
\emph{Lingulellotreta malongensis}: ``The tentacles are clearly visible,
and closely arranged in a single palisade'' -- \citet{Zhang2004Newdata}.

\hypertarget{Loxosomella-coding-42}{}
\emph{Loxosomella}: Single side \citep{Nielsen1966}.

\hypertarget{Sipunculus-coding-42}{}
\emph{Sipunculus}: Both sides in tentacle-breathers such as
\emph{Themiste} \citep{Ruppert1995, Adrianov2006}; only one side in
\emph{Sipunculus} \citep{Ruppert1995, Adrianov2006}.

\subsection*{{[}43{]} Tentacle rows per side in trocholophe
stage}\label{tentacle-rows-per-side-in-trocholophe-stage}
\addcontentsline{toc}{subsection}{{[}43{]} Tentacle rows per side in
trocholophe stage}

\includegraphics{Brachiopod_phylogeny_files/figure-latex/character-mapping-43.pdf}

\begin{quote}
\textbf{Character 43: Perioral tentacular apparatus: Tentacle rows per
side in trocholophe stage}

0: No additional ablabial row\\
1: Adlabial and ablabial row\\
Neomorphic character.
\end{quote}

After Carlson \citeyearpar{Carlson1995Phylogeneticrelationships},
character 37. Lophophore tentacles are commonly arranged into an
ablabial and adlabial row, with ablabial tentacles sometimes added later
in development.

\hypertarget{Amathia-coding-43}{}
\emph{Amathia}: \citep{Temereva2016Thenervous}.

\hypertarget{Flustra-coding-43}{}
\emph{Loxosomella}, \emph{Flustra}: Inapplicable.

\hypertarget{Lingula-coding-43}{}
\emph{Pelagodiscus atlanticus}, \emph{Lingula}, \emph{Terebratulina},
\emph{Phoronis}: Following coding for higher taxon in Carlson
\citeyearpar{Carlson1995Phylogeneticrelationships}, appendix 1,
character 37.

\hypertarget{Novocrania-coding-43}{}
\emph{Novocrania}: Following coding for higher taxon in Carlson
\citeyearpar{Carlson1995Phylogeneticrelationships}, appendix 1,
character 37. Also states in
\citet{Williams2000LinguliformeaCraniiformea}, p.~158.

\subsection*{{[}44{]} Tentacle rows per side in post-trocholophe
stage}\label{tentacle-rows-per-side-in-post-trocholophe-stage}
\addcontentsline{toc}{subsection}{{[}44{]} Tentacle rows per side in
post-trocholophe stage}

\includegraphics{Brachiopod_phylogeny_files/figure-latex/character-mapping-44.pdf}

\begin{quote}
\textbf{Character 44: Perioral tentacular apparatus: Tentacle rows per
side in post-trocholophe stage}

0: No additional ablabial row\\
1: Adlabial and ablabial row\\
Neomorphic character.
\end{quote}

After Carlson \citeyearpar{Carlson1995Phylogeneticrelationships},
character 37. Lophophore tentacles are commonly arranged into an
ablabial and adlabial row, with ablabial tentacles sometimes added later
in development (and thus interpreted as a neomorphic addition).

\hypertarget{Acanthotretella_spinosa-coding-44}{}
\emph{Acanthotretella spinosa}: Preservation insufficient to evaluate
\citep{Holmer2006Aspinose}.

\hypertarget{Amathia-coding-44}{}
\emph{Amathia}: \citep{Temereva2016Thenervous}.

\hypertarget{Cotyledion_tylodes-coding-44}{}
\emph{Cotyledion tylodes}: Additional row not evident \citep{Zhang2013}.

\hypertarget{Heliomedusa_orienta-coding-44}{}
\emph{Heliomedusa orienta}: ``The lophophoral arms bear laterofrontal
tentacles with a double row of cilia along their lateral edge, as in
extant lingulid brachiopods'' -- \citet{Zhang2009Architectureand}.

\hypertarget{Kutorgina_chengjiangensis-coding-44}{}
\emph{Kutorgina chengjiangensis}: Tentacles ``cannot be confidently
demonstrated in the available specimens.'' --
\citet{Zhang2007Rhynchonelliformeanbrachiopods}.

\hypertarget{Lingula-coding-44}{}
\emph{Novocrania}, \emph{Pelagodiscus atlanticus}, \emph{Lingula},
\emph{Terebratulina}, \emph{Phoronis}: Following coding for higher taxon
in Carlson \citeyearpar{Carlson1995Phylogeneticrelationships}, appendix
1, character 37.

\hypertarget{Lingulellotreta_malongensis-coding-44}{}
\emph{Lingulellotreta malongensis}: Single palisade
\citep{Zhang2004Newdata}.

\hypertarget{Lingulosacculus-coding-44}{}
\emph{Lingulosacculus}: Preservation insufficient to evaluate.

\hypertarget{Loxosomella-coding-44}{}
\emph{Loxosomella}: \citet{Nielsen1966}.

\hypertarget{Yuganotheca_elegans-coding-44}{}
\emph{Yuganotheca elegans}: ``helical lophophore fringed with a single
row of thick, widely spaced, parallel-sided and hollow tentacles'' --
\citet{Zhang2014Anearly}.

\subsection*{{[}45{]} Median tentacle in early
development}\label{median-tentacle-in-early-development}
\addcontentsline{toc}{subsection}{{[}45{]} Median tentacle in early
development}

\includegraphics{Brachiopod_phylogeny_files/figure-latex/character-mapping-45.pdf}

\begin{quote}
\textbf{Character 45: Perioral tentacular apparatus: Median tentacle in
early development}

0: Absent\\
1: Present\\
Neomorphic character.
\end{quote}

Following character 28 in \citet{Carlson1995Phylogeneticrelationships}.
Certain taxa exhibit a median tentacle early in development that is lost
during ontogeny.

\hypertarget{Acanthotretella_spinosa-coding-45}{}
\emph{Namacalathus}, \emph{Haplophrentis carinatus},
\emph{Pedunculotheca diania}, \emph{Dailyatia}, \emph{Acanthotretella
spinosa}, \emph{Alisina}, \emph{Askepasma toddense}, \emph{Micromitra},
\emph{Antigonambonites planus}, \emph{Clupeafumosus socialis},
\emph{Coolinia pecten}, \emph{Eccentrotheca}, \emph{Gasconsia},
\emph{Glyptoria}, \emph{Heliomedusa orienta}, \emph{Kutorgina
chengjiangensis}, \emph{Lingulosacculus}, \emph{Lingulellotreta
malongensis}, \emph{Longtancunella chengjiangensis}, \emph{Micrina},
\emph{Mummpikia nuda}, \emph{Nisusia sulcata}, \emph{Orthis},
\emph{Paterimitra}, \emph{Salanygolina}, \emph{Tomteluva perturbata},
\emph{Yuganotheca elegans}: Lophophore ontogeny presently unknown.

\hypertarget{Loxosomella-coding-45}{}
\emph{Loxosomella}: \citet{Nielsen1966}.

\subsection*{{[}46{]} Site of tentacle
addition}\label{site-of-tentacle-addition}
\addcontentsline{toc}{subsection}{{[}46{]} Site of tentacle addition}

\includegraphics{Brachiopod_phylogeny_files/figure-latex/character-mapping-46.pdf}

\begin{quote}
\textbf{Character 46: Perioral tentacular apparatus: Site of tentacle
addition}

1: At two points located behind the mouth\\
2: At the tip of each lophophore arm\\
Transformational character.
\end{quote}

Following \citet{Temereva2017Innervationof}.

\hypertarget{Amathia-coding-46}{}
\emph{Flustra}, \emph{Amathia}, \emph{Novocrania}, \emph{Pelagodiscus
atlanticus}, \emph{Lingula}, \emph{Terebratulina}: Following
\citet{Temereva2017Innervationof}.

\hypertarget{Phoronis-coding-46}{}
\emph{Phoronis}: Following \citet{Temereva2017Innervationof} -- though
in larvae, tentacles are added at the tips of the developing lophophore.

\hypertarget{Sipunculus-coding-46}{}
\emph{Sipunculus}: New branches added at each lateral extreme, behind
mouth \citep{Adrianov2006}.

\subsection*{{[}47{]} Innervation}\label{innervation}
\addcontentsline{toc}{subsection}{{[}47{]} Innervation}

\includegraphics{Brachiopod_phylogeny_files/figure-latex/character-mapping-47.pdf}

\begin{quote}
\textbf{Character 47: Perioral tentacular apparatus: Innervation}

1: Extensions of a circumoral nerve ring\\
2: Nerve rings within the tentacle ring itself\\
Transformational character.
\end{quote}

Annelid tentacles are innervated by palp nerves \citep{Orrhage2005};
lophophores ancestrally contained a pair of nerve rings
\citep{Temereva2017Innervationof}.

\hypertarget{Amathia-coding-47}{}
\emph{Flustra}, \emph{Amathia}: Following
\citet{Temereva2017Innervationof}.

\hypertarget{Dentalium-coding-47}{}
\emph{Dentalium}: The captacula each bear an individual nerve fibre
emanating from the cerebral ganglia, which is also associated with the
circumoesophageal nerve ring \citep{SumnerRooney2015}, recalling the
situation in annelids and sipunculans.

\hypertarget{Loxosomella-coding-47}{}
\emph{Loxosomella}: Tentacle nerves originate laterally from the
cerebral ganglion, branching three times and leading to a single nerve
within each tentacle \citep{Fuchs2006}.

\hypertarget{Serpula-coding-47}{}
\emph{Serpula}: \citet{Orrhage2005}.

\hypertarget{Sipunculus-coding-47}{}
\emph{Sipunculus}: \citet{Rice1993}.

\subsection*{{[}48{]} Inner nerve ring}\label{inner-nerve-ring}
\addcontentsline{toc}{subsection}{{[}48{]} Inner nerve ring}

\includegraphics{Brachiopod_phylogeny_files/figure-latex/character-mapping-48.pdf}

\begin{quote}
\textbf{Character 48: Perioral tentacular apparatus: Inner nerve ring}

0: Not reduced (whether present or absent due to absence of lophophore
nerve rings)\\
1: Reduced, weakly developed or absent in adults\\
Neomorphic character.
\end{quote}

Juvenile lophophorates exhibit two nerve rings in the tentacles; one of
these rings is often reduced or lost at adulthood
\citep{Temereva2017Innervationof}.

\hypertarget{Amathia-coding-48}{}
\emph{Flustra}, \emph{Amathia}: Following
\citet{Temereva2017Innervationof}.

\hypertarget{Lingula-coding-48}{}
\emph{Lingula}: \citet{Temereva2017Thefirst}.

\hypertarget{Loxosomella-coding-48}{}
\emph{Loxosomella}: Nerves present in tentacles, but not forming rings
\citep{Fuchs2006}.

\hypertarget{Novocrania-coding-48}{}
\emph{Novocrania}: Probably only a single ring is present, but only
available illustrations are 19th century sketches \citep{Luter2016}.

\hypertarget{Phoronis-coding-48}{}
\emph{Phoronis}: \citep{Temereva2017Innervationof}.

\hypertarget{Terebratulina-coding-48}{}
\emph{Terebratulina}: In \emph{Gryphus} \citep{Temereva2017Thefirst}.

\subsection*{{[}49{]} Outer nerve ring}\label{outer-nerve-ring}
\addcontentsline{toc}{subsection}{{[}49{]} Outer nerve ring}

\includegraphics{Brachiopod_phylogeny_files/figure-latex/character-mapping-49.pdf}

\begin{quote}
\textbf{Character 49: Perioral tentacular apparatus: Outer nerve ring}

0: Not reduced (whether present or absent due to absence of lophophore
nerve rings)\\
1: Reduced, weakly developed or absent in adults\\
Neomorphic character.
\end{quote}

Juvenile lophophorates exhibit two nerve rings in the tentacles; one of
these rings is often reduced or lost at adulthood
\citep{Temereva2017Innervationof}.

\hypertarget{Amathia-coding-49}{}
\emph{Amathia}: Following \citet{Temereva2017Innervationof}; only one
tentacle nerve ring evident in \citet{Temereva2016Thenervous}.

\hypertarget{Flustra-coding-49}{}
\emph{Flustra}: ``Most species of bryozoans have only the inner'' nerve
ring -- \citet{Temereva2017Innervationof}.

\hypertarget{Lingula-coding-49}{}
\emph{Lingula}, \emph{Terebratulina}: \citet{Temereva2017Innervationof}.

\hypertarget{Loxosomella-coding-49}{}
\emph{Loxosomella}: Nerves present in tentacles, but not forming rings
\citep{Fuchs2006}.

\hypertarget{Novocrania-coding-49}{}
\emph{Novocrania}: Probably only a single ring is present, but only
available illustrations are 19th century sketches \citep{Luter2016}.

\hypertarget{Phoronis-coding-49}{}
\emph{Phoronis}: \citet{Temereva2017Thefirst}.

\subsection*{{[}50{]} Musculature}\label{musculature}
\addcontentsline{toc}{subsection}{{[}50{]} Musculature}

\includegraphics{Brachiopod_phylogeny_files/figure-latex/character-mapping-50.pdf}

\begin{quote}
\textbf{Character 50: Perioral tentacular apparatus: Musculature}

1: Outer main tentacle muscle; two pairs of inner longitudinal muscles\\
2: Peripheral muscle fibres\\
3: Core of longitudinal muscle fibres surrounded by circular muscles\\
Transformational character.
\end{quote}

\hypertarget{Dentalium-coding-50}{}
\emph{Dentalium}: Six to eight elongate muscle cells in core
\citep{Shimek1988}, surrounded by circular muscles \citep{Byrum1994}.

\hypertarget{Lingula-coding-50}{}
\emph{Novocrania}, \emph{Pelagodiscus atlanticus}, \emph{Lingula},
\emph{Terebratulina}: ``Inner coelomic epithelium underlain by muscle
fibers, or in the tentacles, myoepithelial cells.'' --
\citet{Williams1997Introduction}.

\hypertarget{Loxosomella-coding-50}{}
\emph{Loxosomella}: Outer main tentacle muscle; two pairs of inner
longitudinal muscles \citep{Fuchs2006}.

\hypertarget{Phoronis-coding-50}{}
\emph{Phoronis}: \citep{Pardos1991}.

\hypertarget{Serpula-coding-50}{}
\emph{Serpula}: Peripheral muscle fibres \citep{Hanson1949}.

\hypertarget{Sipunculus-coding-50}{}
\emph{Sipunculus}: Peripheral to main tentacle cavity
\citep{Pilger1982}.

\subsection*{{[}51{]} Forms closed loop}\label{forms-closed-loop}
\addcontentsline{toc}{subsection}{{[}51{]} Forms closed loop}

\includegraphics{Brachiopod_phylogeny_files/figure-latex/character-mapping-51.pdf}

\begin{quote}
\textbf{Character 51: Perioral tentacular apparatus: Forms closed loop}

1: Diverging laterally\\
2: Closed loop\\
Transformational character.
\end{quote}

Whereas the lophophore of crown-group brachiopods typically forms a
closed loop, those of \emph{Haplophrentis} and \emph{Heliomedusa}
diverge laterally \citep{Moysiuk2017Hyolithsare}.

\hypertarget{Amathia-coding-51}{}
\emph{Amathia}: Ends of arms meet to form closed loop
\citep{Temereva2016Thenervous}.

\hypertarget{Cotyledion_tylodes-coding-51}{}
\emph{Cotyledion tylodes}: Tentacles form almost complete circular
crown.

\hypertarget{Lingulosacculus-coding-51}{}
\emph{Lingulosacculus}: Two diverging arms of the lophophore are
preserved \citep{Balthasar2009EarlyCambrian}.

\hypertarget{Longtancunella_chengjiangensis-coding-51}{}
\emph{Longtancunella chengjiangensis}: Two distinct, diverging arms
reconstructed by \citet{Zhang2007Agregarious}.

\hypertarget{Loxosomella-coding-51}{}
\emph{Loxosomella}: \citet{Nielsen1966}.

\hypertarget{Namacalathus-coding-51}{}
\emph{Namacalathus}: The existence of a lophophore is speculative.

\hypertarget{Nisusia_sulcata-coding-51}{}
\emph{Nisusia sulcata}: No specimens of \emph{Nisusia} preserve the
lophophore.

\hypertarget{Phoronis-coding-51}{}
\emph{Phoronis}: Two lophophore arms rather than a single continuous
loop, but with tips close together to form functional loop
\citep{Torrey1901}.

\hypertarget{Serpula-coding-51}{}
\emph{Sipunculus}, \emph{Serpula}: Growing to encircle mouth in adults.

\subsection*{{[}52{]} Coiling direction}\label{coiling-direction}
\addcontentsline{toc}{subsection}{{[}52{]} Coiling direction}

\includegraphics{Brachiopod_phylogeny_files/figure-latex/character-mapping-52.pdf}

\begin{quote}
\textbf{Character 52: Perioral tentacular apparatus: Coiling direction}

1: Anteriad\\
2: Posteriad\\
Transformational character.
\end{quote}

The lophophore arms of \emph{Heliomedusa} and \emph{Haplophrentis} arch
posteriad, rather than anteriad as in lingulids. See
\citet{Zhang2009Architectureand}; \citet{Moysiuk2017Hyolithsare}.

\hypertarget{Acanthotretella_spinosa-coding-52}{}
\emph{Acanthotretella spinosa}, \emph{Lingulellotreta malongensis}: Arms
proceed anteriad before recurving.

\hypertarget{Amathia-coding-52}{}
\emph{Flustra}, \emph{Amathia}: Bryozoan arms reach in anal
(i.e.~posterior) direction \citep{Shunkina2015}.

\hypertarget{Cotyledion_tylodes-coding-52}{}
\emph{Cotyledion tylodes}: Cannot establish without distinguishing gut
from anus.

\hypertarget{Loxosomella-coding-52}{}
\emph{Loxosomella}: Posterior (anal side) of lophophore has short
stretch lacking tentacles.

\hypertarget{Pelagodiscus_atlanticus-coding-52}{}
\emph{Pelagodiscus atlanticus}: ``converging anteriorly and coiling
anterior to the body cavity'' -- \citet{Zhang2009Architectureand}.

\hypertarget{Phoronis-coding-52}{}
\emph{Phoronis}: Coiling in direction of anus (i.e.~posteriad).

\subsection*{{[}53{]} Adjustor muscle}\label{adjustor-muscle}
\addcontentsline{toc}{subsection}{{[}53{]} Adjustor muscle}

\includegraphics{Brachiopod_phylogeny_files/figure-latex/character-mapping-53.pdf}

\begin{quote}
\textbf{Character 53: Perioral tentacular apparatus: Adjustor muscle}

0: Absent\\
1: Present\\
Neomorphic character.
\end{quote}

Following character 55 in Carlson
\citeyearpar{Carlson1995Phylogeneticrelationships}. Not possible to code
in most fossil taxa.

\hypertarget{Acanthotretella_spinosa-coding-53}{}
\emph{Namacalathus}, \emph{Haplophrentis carinatus},
\emph{Pedunculotheca diania}, \emph{Dailyatia}, \emph{Acanthotretella
spinosa}, \emph{Alisina}, \emph{Askepasma toddense}, \emph{Micromitra},
\emph{Antigonambonites planus}, \emph{Clupeafumosus socialis},
\emph{Coolinia pecten}, \emph{Eccentrotheca}, \emph{Gasconsia},
\emph{Glyptoria}, \emph{Heliomedusa orienta}, \emph{Kutorgina
chengjiangensis}, \emph{Lingulosacculus}, \emph{Lingulellotreta
malongensis}, \emph{Longtancunella chengjiangensis}, \emph{Micrina},
\emph{Mummpikia nuda}, \emph{Nisusia sulcata}, \emph{Orthis},
\emph{Paterimitra}, \emph{Salanygolina}, \emph{Tomteluva perturbata},
\emph{Yuganotheca elegans}: Preservation not adequate to evaluate
presence or absence of this muscle.

\hypertarget{Lingula-coding-53}{}
\emph{Novocrania}, \emph{Pelagodiscus atlanticus}, \emph{Lingula},
\emph{Terebratulina}, \emph{Phoronis}: Following coding for higher taxon
in Carlson \citeyearpar{Carlson1995Phylogeneticrelationships}, appendix
1, character 55.

\section{Digestive tract}\label{digestive-tract}

\subsection*{{[}54{]} Prominent pharynx}\label{prominent-pharynx}
\addcontentsline{toc}{subsection}{{[}54{]} Prominent pharynx}

\includegraphics{Brachiopod_phylogeny_files/figure-latex/character-mapping-54.pdf}

\begin{quote}
\textbf{Character 54: Digestive tract: Prominent pharynx}

0: Absent\\
1: Present\\
Neomorphic character.
\end{quote}

Hyoliths exhibit a prominent protrusible muscular pharynx at the base of
the lophophore \citep{Moysiuk2017Hyolithsare}. This is considered as
potentially equivalent to the anterior projection of the visceral cavity
in \emph{Heliomedusa}, and, by extension, in \emph{Lingulosacculus} and
Lingulotreta.

\hypertarget{Eoobolus-coding-54}{}
\emph{Eoobolus}: Prominent extension of dorsal visceral platform
\citep{Balthasar2009Thebrachiopod}.

\hypertarget{Heliomedusa_orienta-coding-54}{}
\emph{Heliomedusa orienta}: Corresponding to the ``neck'' of the
vase-shaped visceral cavity reported by
\citet{Zhang2009Architectureand}.

\hypertarget{Lingulellotreta_malongensis-coding-54}{}
\emph{Lingulellotreta malongensis}: An anterior projection of the
visceral area is noted by Williams \emph{et al}.
\citeyearpar{Williams2000LinguliformeaCraniiformea} and considered
equivalent to that observed in \emph{Lingulosacculus}
\citep{Balthasar2009EarlyCambrian}.

\hypertarget{Lingulosacculus-coding-54}{}
\emph{Lingulosacculus}: The prominent anterior extension of the visceral
area noted by Balthasar \& Butterfield
\citeyearpar{Balthasar2009EarlyCambrian} is considered as potentially
homologous with that of \emph{Heliomedusa}
\citep{Zhang2009Architectureand} and, by extension, \emph{Haplophrentis}
\citep{Moysiuk2017Hyolithsare}.

\hypertarget{Sipunculus-coding-54}{}
\emph{Sipunculus}: Eversible pharynx (introvert).

\hypertarget{Yuganotheca_elegans-coding-54}{}
\emph{Yuganotheca elegans}: Possibly present, following interpretation
of mouth \citep[see fig. 2c, d in][]{Zhang2014Anearly}.

\subsection*{{[}55{]} Radula}\label{radula}
\addcontentsline{toc}{subsection}{{[}55{]} Radula}

\includegraphics{Brachiopod_phylogeny_files/figure-latex/character-mapping-55.pdf}

\begin{quote}
\textbf{Character 55: Digestive tract: Radula}

0: Absent\\
1: Present\\
Neomorphic character.
\end{quote}

Any apparatus comprising multiple denticulate rows arranged serially in
the sagittal plane is treated as potentially homologous with the
molluscan radula.

\hypertarget{Acanthotretella_spinosa-coding-55}{}
\emph{Haplophrentis carinatus}, \emph{Acanthotretella spinosa},
\emph{Heliomedusa orienta}, \emph{Kutorgina chengjiangensis},
\emph{Lingulosacculus}, \emph{Lingulellotreta malongensis},
\emph{Longtancunella chengjiangensis}, \emph{Yuganotheca elegans}: No
candidate observed despite exceptional preservation.

\hypertarget{Wiwaxia_corrugata-coding-55}{}
\emph{Wiwaxia corrugata}: \citet{Smith2012M}.

\subsection*{{[}56{]} Oesophageal folds}\label{oesophageal-folds}
\addcontentsline{toc}{subsection}{{[}56{]} Oesophageal folds}

\includegraphics{Brachiopod_phylogeny_files/figure-latex/character-mapping-56.pdf}

\begin{quote}
\textbf{Character 56: Digestive tract: Oesophageal folds}

0: Absent\\
1: Present\\
Neomorphic character.
\end{quote}

Following character 86 in \citet{Giribet2002}.

\hypertarget{Phoronis-coding-56}{}
\emph{Phoronis}: Ciliated ridge in oesophagus \citep{Torrey1901}.

\subsection*{{[}57{]} Oral sphincter}\label{oral-sphincter}
\addcontentsline{toc}{subsection}{{[}57{]} Oral sphincter}

\includegraphics{Brachiopod_phylogeny_files/figure-latex/character-mapping-57.pdf}

\begin{quote}
\textbf{Character 57: Digestive tract: Oral sphincter}

0: Absent\\
1: Present\\
Neomorphic character.
\end{quote}

Character 133 in \citet{Grobe2007}.

\hypertarget{Dentalium-coding-57}{}
\emph{Dentalium}: Present, but secondarily reduced.

\subsection*{{[}58{]} Locomotory cilia}\label{locomotory-cilia}
\addcontentsline{toc}{subsection}{{[}58{]} Locomotory cilia}

\includegraphics{Brachiopod_phylogeny_files/figure-latex/character-mapping-58.pdf}

\begin{quote}
\textbf{Character 58: Digestive tract: Foregut: Locomotory cilia}

0: Absent\\
1: Present\\
Neomorphic character.
\end{quote}

Character 66 in \citet{Haszprunar2000}.

\subsection*{{[}59{]} Subdivisions}\label{subdivisions}
\addcontentsline{toc}{subsection}{{[}59{]} Subdivisions}

\includegraphics{Brachiopod_phylogeny_files/figure-latex/character-mapping-59.pdf}

\begin{quote}
\textbf{Character 59: Digestive tract: Midgut: Subdivisions}

0: Not subdivided\\
1: Functional subdivisions\\
Neomorphic character.
\end{quote}

The molluscan midgut is functionally subdivided into a sorting area
(stomach), digestion area (midgut sac or gland), and transport tube
(intestine). Characters 42 in \citet{Haszprunar2000}, 1.38 in
\citet{SPS1996}.

\hypertarget{Conotheca-coding-59}{}
\emph{Conotheca}: \citep{Devaere2014}.

\hypertarget{Wiwaxia_corrugata-coding-59}{}
\emph{Wiwaxia corrugata}: Subdivided, presumably functionally, but with
some ambiguity {[}\citet{Smith2012M};Smith2014{]}.

\subsection*{{[}60{]} Glands}\label{glands}
\addcontentsline{toc}{subsection}{{[}60{]} Glands}

\includegraphics{Brachiopod_phylogeny_files/figure-latex/character-mapping-60.pdf}

\begin{quote}
\textbf{Character 60: Digestive tract: Midgut: Glands}

0: Absent\\
1: Present\\
Neomorphic character.
\end{quote}

Characters 1.40, 2.30 and 4.59 in \citet{SPS1996}; 42 in
\citet{Haszprunar2000}.

\hypertarget{Wiwaxia_corrugata-coding-60}{}
\emph{Wiwaxia corrugata}: Annex to midgut interpreted as a gland
\citep{Smith2012M}.

\section{Digestive tract: Anus}\label{digestive-tract-anus}

\subsection*{{[}61{]} Presence}\label{presence-3}
\addcontentsline{toc}{subsection}{{[}61{]} Presence}

\includegraphics{Brachiopod_phylogeny_files/figure-latex/character-mapping-61.pdf}

\begin{quote}
\textbf{Character 61: Digestive tract: Anus: Presence}

0: Anus present: through-gut\\
1: Anus lost: digestive tract is blind sac\\
Neomorphic character.
\end{quote}

The digestive tract may either constitute a blind sac, or a through gut
with anus. The loss of an anus is known to be derived within spiralia,
so this character is treated as neomorphic.

\hypertarget{Conotheca-coding-61}{}
\emph{Conotheca}: \citep{Devaere2014}.

\hypertarget{Glyptoria-coding-61}{}
\emph{Glyptoria}: Scored according to familial level feature.

\hypertarget{Kutorgina_chengjiangensis-coding-61}{}
\emph{Kutorgina chengjiangensis}: Although ``the possibility of a blind
ending may not be completely eliminated {[}\ldots{}{]} the weight of
evidence {[}\ldots{}{]} leads us to reject the possibility of a
blind-ending intestine'' --
\citet{Zhang2007Rhynchonelliformeanbrachiopods}, p.~1399.

\subsection*{{[}62{]} Location}\label{location}
\addcontentsline{toc}{subsection}{{[}62{]} Location}

\includegraphics{Brachiopod_phylogeny_files/figure-latex/character-mapping-62.pdf}

\begin{quote}
\textbf{Character 62: Digestive tract: Anus: Location}

1: Straight gut with posterior anus\\
2: Anus migrated posteriad to create U-shaped gut\\
3: Anus opening to rear of pedal sole, causing slight U-shape to gut\\
Transformational character.
\end{quote}

``The relative position of the mouth and anus in the larvae of
brachiopods and phoronids is similar: posterior anus and anterior
mouth'' -- \citet{Williams2007Supplement}, p.~2884\\
See also character 6 in \citet{Haszprunar2008}.

\hypertarget{Conotheca-coding-62}{}
\emph{Conotheca}: \citep{Devaere2014}.

\hypertarget{Dentalium-coding-62}{}
\emph{Dentalium}: The U-shaped gut of scaphopods arises by exaggeration
of the dorsal surface, rather than migration of the anus
\citep{Steiner1992}.

\hypertarget{Kutorgina_chengjiangensis-coding-62}{}
\emph{Kutorgina chengjiangensis}: ``Five specimens have an exceptionally
preserved digestive tract, dorsally curved, with a putative
dorso-terminal anus located near the proximal end of a pedicle'' --
\citet{Zhang2007Rhynchonelliformeanbrachiopods}.

\hypertarget{Terebratulina-coding-62}{}
\emph{Terebratulina}: ``In rhynchonelliforms, the gut curves somewhat
into a C-shape and the (blind) anus becomes posteroventral in
position.'' -- \citet{Williams2007Supplement}, p.\\
2884.

\subsection*{{[}63{]} Migration: Within ring of
tentacles}\label{migration-within-ring-of-tentacles}
\addcontentsline{toc}{subsection}{{[}63{]} Migration: Within ring of
tentacles}

\includegraphics{Brachiopod_phylogeny_files/figure-latex/character-mapping-63.pdf}

\begin{quote}
\textbf{Character 63: Digestive tract: Anus: Migration: Within ring of
tentacles}

1: Not within ring of tentacles\\
2: Anterior - within ring of feeding tentacles\\
Transformational character.
\end{quote}

A migrated anus may be located laterally or within the lophophore ring
(as in entoprocts).

\hypertarget{Kutorgina_chengjiangensis-coding-63}{}
\emph{Kutorgina chengjiangensis}: ``Presumed to terminate in a
functional anus located near the proximal end of the pedicle.'' --
\citet{Zhang2007Rhynchonelliformeanbrachiopods}.

\subsection*{{[}64{]} Migration: Position}\label{migration-position}
\addcontentsline{toc}{subsection}{{[}64{]} Migration: Position}

\includegraphics{Brachiopod_phylogeny_files/figure-latex/character-mapping-64.pdf}

\begin{quote}
\textbf{Character 64: Digestive tract: Anus: Migration: Position}

1: Left\\
2: Right\\
3: Dorsally\\
4: Ventrally\\
Transformational character.
\end{quote}

If the anus is not within the ring of tentacles, in which direction is
it oriented?.

\hypertarget{Amathia-coding-64}{}
\emph{Flustra}, \emph{Amathia}: Anus remains on ventral surface.
Arguably, rather than the anus migrating, the dorsal surface of the
animal has become extended.

\hypertarget{Conotheca-coding-64}{}
\emph{Conotheca}: Opening to left in \citet{Devaere2014}, fig. 5A.

\hypertarget{Dentalium-coding-64}{}
\emph{Dentalium}: An alternative interpretation would be that the
posterior of the scaphopod has been extended to generate the relatively
anterior position of the originally ventral anus.

\hypertarget{Haplophrentis_carinatus-coding-64}{}
\emph{Haplophrentis carinatus}: Opening to the right -- see figures 1,
3, and extended data 5 in Moysiuk \emph{et al}.
\citeyearpar{Moysiuk2017Hyolithsare}. The text states in error that the
anus is to the left of the midline.

\hypertarget{Kutorgina_chengjiangensis-coding-64}{}
\emph{Kutorgina chengjiangensis}: ``Five specimens have an exceptionally
preserved digestive tract, dorsally curved, with a putative
dorso-terminal anus located near the proximal end of a pedicle'' --
\citet{Zhang2007Rhynchonelliformeanbrachiopods}.

\hypertarget{Lingula-coding-64}{}
\emph{Lingula}: ``In the lingulids, the {[}intestine{]} follows an
oblique course anteriorly to open at the anus on the right body wall.''
-- \citet{Williams1997Introduction}, p.~89.

\hypertarget{Lingulellotreta_malongensis-coding-64}{}
\emph{Lingulellotreta malongensis}: ``finally terminating in an anal
opening on the right anterior body wall'' \citep[p.66]{Zhang2007Noteon}.

\hypertarget{Lingulosacculus-coding-64}{}
\emph{Lingulosacculus}: ``This same arrangement occurs in \emph{L.
nuda}, with the looped dark line tracking the same course as the
exceptionally preserved guts of Chengjiang lingulellotretids, including
the median position of its posterior loop and the sharp right turn as it
exits the posterior extension of the ventral valve''
\citep[p.310]{Balthasar2009EarlyCambrian}.

\hypertarget{Longtancunella_chengjiangensis-coding-64}{}
\emph{Longtancunella chengjiangensis}: ``The intestine extends
posteriorly, and then turns right to continue as a tortuous strand,
finally terminating at the latero-median position of the anterior body
wall'' -- \citet{Zhang2007Agregarious}.

\hypertarget{Terebratulina-coding-64}{}
\emph{Terebratulina}: ``In rhynchonelliforms, the gut curves somewhat
into a C-shape and the (blind) anus becomes posteroventral in
position.'' -- \citet{Williams2007Supplement}, p.\\
2884.

\hypertarget{Yuganotheca_elegans-coding-64}{}
\emph{Yuganotheca elegans}: The identification of the ``very poorly
impressed possible anus at the lateral side of the anterior body wall''
is not yet confident, so this character is coded as not presently
available.

\section{Sclerites}\label{sclerites}

\subsection*{{[}65{]} Present in adult}\label{present-in-adult}
\addcontentsline{toc}{subsection}{{[}65{]} Present in adult}

\includegraphics{Brachiopod_phylogeny_files/figure-latex/character-mapping-65.pdf}

\begin{quote}
\textbf{Character 65: Sclerites: Present in adult}

0: Absent\\
1: Present\\
Neomorphic character.
\end{quote}

Plate-like (wider than tall) skeletal elements, whether mineralized or
non-mineralized.\\
The definition deliberately excludes setae (which are taller than wide).

\hypertarget{Dentalium-coding-65}{}
\emph{Tonicella}, \emph{Dentalium}: Molluscan valves are treated as
potential homologues of brachiopod valves.

\hypertarget{Halkieria_evangelista-coding-65}{}
\emph{Halkieria evangelista}: Halkieriid sclerites are interpreted as
potentially homologous with those of \emph{Dailyatia} and hence the
brachiopods \citep{Zhao2017}.

\hypertarget{Namacalathus-coding-65}{}
\emph{Namacalathus}: The mineralized endoskeleton of \emph{Namacalathus}
is not interpreted as a sclerite.

\hypertarget{Serpula-coding-65}{}
\emph{Serpula}: Annelid setae are not considered to represent potential
homologues with the brachiopod shell.

\hypertarget{Sipunculus-coding-65}{}
\emph{Sipunculus}: Hooks are present, though the absence of chitin or
microvillar impressions indicates that they are not homologous with
those of other lophotrochozoans.

\hypertarget{Wiwaxia_corrugata-coding-65}{}
\emph{Wiwaxia corrugata}: The scales of \emph{Wiwaxia} are treated as
homologous with the chaetae of annelids and brachiopods
\citep{Butterfield1990, Smith2014, Zhang2015}, rather than brachiopod
shell.

\subsection*{{[}66{]} Periodically shed and
replaced}\label{periodically-shed-and-replaced}
\addcontentsline{toc}{subsection}{{[}66{]} Periodically shed and
replaced}

\includegraphics{Brachiopod_phylogeny_files/figure-latex/character-mapping-66.pdf}

\begin{quote}
\textbf{Character 66: Sclerites: Periodically shed and replaced}

0: Absent\\
1: Present\\
Neomorphic character.
\end{quote}

Certain taxa periodically slough and replace some of their individual
sclerites during growth.

\section{Sclerites: Bivalved {[}67{]}}\label{sclerites-bivalved-67}

\includegraphics{Brachiopod_phylogeny_files/figure-latex/character-mapping-67.pdf}

\begin{quote}
\textbf{Character 67: Sclerites: Bivalved}

0: Scleritomous: without differentiated dorsal and ventral valves\\
1: Bivalved: scleritome dominated by prominent dorsal and ventral
valve\\
Neomorphic character.
\end{quote}

Scleritome dominated by prominent differentiated dorsal and ventral
valves.

\hypertarget{Tonicella-coding-67}{}
\emph{Tonicella}: As larvae, polyplacophorans exhibit an anterior and a
posterior shell field \citep{Wanninger2002C}; subsequent subdivision of
the posterior field gives rise to the posterior seven valves.
\emph{Tonicella} is thus tentatively coded as `bivalved' to reflect the
potential (if perhaps unlikely) homology with the paired elements of
brachiopods.

\subsection*{{[}68{]} Reduced}\label{reduced}
\addcontentsline{toc}{subsection}{{[}68{]} Reduced}

\includegraphics{Brachiopod_phylogeny_files/figure-latex/character-mapping-68.pdf}

\begin{quote}
\textbf{Character 68: Sclerites: Accessory sclerites: Reduced}

1: Accessory sclerites not reduced\\
2: Accessory sclerites absent: two valves only\\
Transformational character.
\end{quote}

Taxa in the bivalved condition may retain sclerites as small additional
elements, such as the L-elements of \emph{Paterimitra}
\citep{Skovsted2015Theearly}. Hyolithid helens are coded as potentially
homologous to these elements
\citep[following][]{Moysiuk2017Hyolithsare}.

This character is treated as neomorphic, with accessory sclerites
ancestrally present, recognizing the likely origin of brachiozoans (and
Lophotrochozoans more generally) from a scleritomous organism.

\hypertarget{Conotheca-coding-68}{}
\emph{Conotheca}: {[}Reference needed: not found, but does this
necessarily mean that they were absent?{]}.

\hypertarget{Cupitheca_holocyclata-coding-68}{}
\emph{Cupitheca holocyclata}: Helens never observed and considered
absent \citep{Skovsted2016}.

\hypertarget{Dentalium-coding-68}{}
\emph{Dentalium}: The scaphopod valve arises posterior of the prototroch
and is thus homologous with the posterior valves of Chiton, assuming
that molluscan shell fields are homologous features.

\hypertarget{Paramicrocornus-coding-68}{}
\emph{Paramicrocornus}: Helens absent, with no possible insertion point
\citep{Zhang2018Ahyolithid}.

\hypertarget{Paterimitra-coding-68}{}
\emph{Paterimitra}: L-sclerites \citep{Skovsted2009Thescleritome}.

\hypertarget{Pauxillites-coding-68}{}
\emph{Pauxillites}: Not found by \citet{Valent2015} or
\citet{Marek1966}, but reconstructed as present by \citet{Marek1976}.

\hypertarget{Tonicella-coding-68}{}
\emph{Tonicella}: The intermediate shell plates arise by subdivision of
the posterior shell field \citep{Wanninger2002C}, and are thus treated
as equivalent to the posterior valve rather than as distinct elements.\\
The girdle elements are homologous with annelid chaetae / brachiopod
setae \citep{Leise1982}, rather than sclerites.

\subsection*{{[}69{]} Arrangement}\label{arrangement}
\addcontentsline{toc}{subsection}{{[}69{]} Arrangement}

\includegraphics{Brachiopod_phylogeny_files/figure-latex/character-mapping-69.pdf}

\begin{quote}
\textbf{Character 69: Sclerites: Accessory sclerites: Arrangement}

0: Single field\\
1: Peripheral and medial fields with distinct sclerite arrangements\\
Neomorphic character.
\end{quote}

Following \citet{Zhao2017}.

\hypertarget{Dailyatia-coding-69}{}
\emph{Dailyatia}: Following the reconstruction of
\citet{Skovsted2015Theearly}.

\subsection*{{[}70{]} Symmetry}\label{symmetry}
\addcontentsline{toc}{subsection}{{[}70{]} Symmetry}

\includegraphics{Brachiopod_phylogeny_files/figure-latex/character-mapping-70.pdf}

\begin{quote}
\textbf{Character 70: Sclerites: Accessory sclerites: Symmetry}

0: Dextral and sinistral forms absent\\
1: Occuring in dextral and sinistral forms\\
Neomorphic character.
\end{quote}

Following \citet{Zhao2017}.

\hypertarget{Eccentrotheca-coding-70}{}
\emph{Eccentrotheca}: \citet{Skovsted2008Thescleritome}.

\hypertarget{Haplophrentis_carinatus-coding-70}{}
\emph{Haplophrentis carinatus}, \emph{Maxilites}, \emph{Pauxillites}:
Helens are symmetrical.

\subsection*{{[}71{]} Hinge line shape}\label{hinge-line-shape}
\addcontentsline{toc}{subsection}{{[}71{]} Hinge line shape}

\includegraphics{Brachiopod_phylogeny_files/figure-latex/character-mapping-71.pdf}

\begin{quote}
\textbf{Character 71: Sclerites: Bivalved: Hinge line shape}

1: Astrophic\\
2: Strophic\\
Transformational character.
\end{quote}

\hypertarget{Askepasma_toddense-coding-71}{}
\emph{Askepasma toddense}, \emph{Micromitra}, \emph{Glyptoria}: Coded as
strophic in Williams \emph{et al}.
\citeyearpar{Williams1998Thediversity}.

\hypertarget{Botsfordia-coding-71}{}
\emph{Botsfordia}: Coded as dissociated in Williams \emph{et al}.
\citeyearpar{Williams1998Thediversity}, appendix 2.

\hypertarget{Craniops-coding-71}{}
\emph{Craniops}: Astrophic: rounded posterior margin \citep[see fig. 91
in][]{Williams2000LinguliformeaCraniiformea}.

\hypertarget{Cupitheca_holocyclata-coding-71}{}
\emph{Cupitheca holocyclata}: \citep{Skovsted2016}.

\hypertarget{Eoobolus-coding-71}{}
\emph{Eoobolus}: Coded as astrophic in Williams \emph{et al}.
\citeyearpar{Williams1998Thediversity}; see also
\citet{Balthasar2009Thebrachiopod}.

\hypertarget{Gasconsia-coding-71}{}
\emph{Gasconsia}: The straight posterior margin of \emph{Gasconsia}
contributes to an overall resemblance with the Chileids
\citep{Holmer2014OrdovicianSilurian}.

\hypertarget{Halkieria_evangelista-coding-71}{}
\emph{Halkieria evangelista}, \emph{Mickwitzia muralensis}:
Non-strophic.

\hypertarget{Kutorgina_chengjiangensis-coding-71}{}
\emph{Kutorgina chengjiangensis}: \citet{Williams1998Thediversity}
(appendix 2) and \citet{Williams2000LinguliformeaCraniiformea} (p.~208)
consider the hinge of \emph{Kutorgina} to be stropic, whereas Bassett
\emph{et al}. \citeyearpar{Bassett2001Functionalmorphology} argue for an
astropic interpretation -- whilst noting that the arrangement is
prominently different from other astrophic taxa. We therefore code this
taxon as ambiguous.

\hypertarget{Longtancunella_chengjiangensis-coding-71}{}
\emph{Longtancunella chengjiangensis}: ``\emph{Longtancunella} has an
oval to subcircular shell with a very short strophic hinge line'' --
\citet{Zhang2011Theexceptionally}.

\hypertarget{Micrina-coding-71}{}
\emph{Micrina}: Non-strophic: see \citet{Holmer2008TheEarly}.

\hypertarget{Nisusia_sulcata-coding-71}{}
\emph{Nisusia sulcata}: ``The strophic, articulated shells of the
Kutorginata rotated on simple hinge mechanisms that are different from
those of other rhynchonelliforms''
\citep[p.~208]{Williams2000LinguliformeaCraniiformea}.

\hypertarget{Novocrania-coding-71}{}
\emph{Novocrania}: Craniides have a strophic posterior valve edge
\citep[table 39 on p.~2853]{Williams2007Supplement}: \emph{Novocrania}'s
``dorsal posterior margin'' is ``straight''
\citep[p.~171]{Williams2000LinguliformeaCraniiformea}.

\hypertarget{Salanygolina-coding-71}{}
\emph{Salanygolina}: Coded as strophic in Williams \emph{et al}.
\citeyearpar{Williams1998Thediversity}; see
\citet{Holmer2009Theenigmatic}.

\hypertarget{Tomteluva_perturbata-coding-71}{}
\emph{Tomteluva perturbata}: ``Tomteluvid taxa all have a strongly
ventribiconvex, astrophic shell with a unisulcate commissure'' --
\citet{Streng2016Anew}, p5.

\hypertarget{Tonicella-coding-71}{}
\emph{Tonicella}: A linear hinge articulation does not exist between
valves 1 and 2; nor would it exist between valves 1 and 8 were these
adjacent \citep{Connors2012}.

\hypertarget{Yuganotheca_elegans-coding-71}{}
\emph{Yuganotheca elegans}: Not evident from fossil material; the
possibility of a short strophic hinge line (as in \emph{Longtancunella})
is difficult to discount.

\subsection*{{[}72{]} Enclosing filtration
chamber}\label{enclosing-filtration-chamber}
\addcontentsline{toc}{subsection}{{[}72{]} Enclosing filtration chamber}

\includegraphics{Brachiopod_phylogeny_files/figure-latex/character-mapping-72.pdf}

\begin{quote}
\textbf{Character 72: Sclerites: Bivalved: Enclosing filtration chamber}

0: No filtration chamber, or open chamber\\
1: Shells close to form enclosed filtration chamber\\
Neomorphic character.
\end{quote}

In crown-group brachiopods, the two primary shells close to form an
enclosed filtration chamber. Further down the stem, taxa such as
\emph{Micrina} do not.

\subsection*{{[}73{]} Commissure: Exact correspondence of valve
margins}\label{commissure-exact-correspondence-of-valve-margins}
\addcontentsline{toc}{subsection}{{[}73{]} Commissure: Exact
correspondence of valve margins}

\includegraphics{Brachiopod_phylogeny_files/figure-latex/character-mapping-73.pdf}

\begin{quote}
\textbf{Character 73: Sclerites: Bivalved: Commissure: Exact
correspondence of valve margins}

1: Margins align exactly when valves closed\\
2: Margins of different shape or size\\
Transformational character.
\end{quote}

Orthothecid hyoliths can retract their operculum into their conical
shell, in contrast to most other taxa, where the valves align exactly
when they are closed, save perhaps for a pedicle notch or, in the case
of hyolithids, depressions that allow the helens to protrude. Refers
only to two prominent valves, not to additional sclerites of
e.g.~Eccentrothca.

\hypertarget{Bactrotheca-coding-73}{}
\emph{Bactrotheca}: Operculum interpreted as sitting inside conical
shell \citep{Marek1976}.

\hypertarget{Conotheca-coding-73}{}
\emph{Conotheca}: ``The diameter of orthothecid opercula is generally
slightly smaller than the diameter of the conch aperture. This
observation is confirmed {[}in \emph{Conotheca}{]} and further supported
by the recurrent presence of opercula preserved withdrawn inside the
conch'' -- \citet{Devaere2014}.

\hypertarget{Cupitheca_holocyclata-coding-73}{}
\emph{Cupitheca holocyclata}: ``The width range of opercula matches that
of the apertures'' \citep{Sun2018}, but it is not possible to determine
whether this is an exact match or whether the operculum may have been
slightly smaller, and hence retractable, as anticipated for
orthothecids.

\hypertarget{Maxilites-coding-73}{}
\emph{Maxilites}: \citet{MartiMus2005}.

\hypertarget{Paramicrocornus-coding-73}{}
\emph{Paramicrocornus}: Articulated specimens unknown.

\hypertarget{Pauxillites-coding-73}{}
\emph{Pauxillites}: Following depiction in \citet{Marek1976}.

\subsection*{{[}74{]} Commissure: Sulcate}\label{commissure-sulcate}
\addcontentsline{toc}{subsection}{{[}74{]} Commissure: Sulcate}

\includegraphics{Brachiopod_phylogeny_files/figure-latex/character-mapping-74.pdf}

\begin{quote}
\textbf{Character 74: Sclerites: Bivalved: Commissure: Sulcate}

1: Rectimarginate\\
2: Uniplicate\\
3: Sulcate\\
Transformational character.
\end{quote}

The anterior commissure can be rectimarginate (i.e.~straight),
uniplicate (i.e.~median sulcus in ventral valve), or sulcate (with
median sulcus in dorsal valve).

\hypertarget{Askepasma_toddense-coding-74}{}
\emph{Askepasma toddense}: Coded as rectimarginate in
\citet{Williams1998Thediversity}, though note that the ``ventral valve
weakly to moderately sulcate'' \citep{Topper2013Theoldest}; a similar
description is provided by Williams \emph{et al}.
\citeyearpar{Williams2000LinguliformeaCraniiformea}. Coded as ambiguous
for these two states accordingly.

\hypertarget{Glyptoria-coding-74}{}
\emph{Micromitra}, \emph{Glyptoria}, \emph{Kutorgina chengjiangensis},
\emph{Salanygolina}: Following appendix 2 in Williams \emph{et al}.
\citeyearpar{Williams1998Thediversity}.

\hypertarget{Terebratulina-coding-74}{}
\emph{Terebratulina}: ``Anterior commissure rectimarginate to
uniplicate'' -- uniplicate in fig. 1425.1c of Williams \emph{et al}.
\citeyearpar{Williams2006Rhynchonelliformeapart}.

\subsection*{{[}75{]} Commissure: Circular}\label{commissure-circular}
\addcontentsline{toc}{subsection}{{[}75{]} Commissure: Circular}

\includegraphics{Brachiopod_phylogeny_files/figure-latex/character-mapping-75.pdf}

\begin{quote}
\textbf{Character 75: Sclerites: Bivalved: Commissure: Circular}

1: Continuous round outline with no corners (save at the hinge);\\
2: Lateral margins linear\\
Transformational character.
\end{quote}

Shape of the commissure in plan view, ignoring any deflection arising
due to articulation at the hinge (e.g.~delthyrium/notothyrium). This
character seeks to discriminate the essentially conical `conchs' of
orthothecid hyoliths from the polygonal `conchs' of hyolithids.
Triangular and oblong outlines are not distinguished, as this is not
entirely independent of the strophic/astrophic nature of the hinge.

\hypertarget{Acanthotretella_spinosa-coding-75}{}
\emph{Acanthotretella spinosa}: Round \citep{Holmer2006Aspinose}.

\hypertarget{Alisina-coding-75}{}
\emph{Alisina}: Sub-triangular \citep{Zhang2011Anobolellate}.

\hypertarget{Antigonambonites_planus-coding-75}{}
\emph{Askepasma toddense}, \emph{Antigonambonites planus},
\emph{Coolinia pecten}, \emph{Glyptoria}, \emph{Nisusia sulcata}:
Lateral margins subparallel, cf. \emph{Askepasma}
\citep{Williams2000LinguliformeaCraniiformea}.

\hypertarget{Bactrotheca-coding-75}{}
\emph{Bactrotheca}: Operculum essentially round, though becoming
flattened on functionally ventral surface \citep{Dzik1980Ontogenyof}.

\hypertarget{Botsfordia-coding-75}{}
\emph{Botsfordia}: Diverging lateral margins \citep{Li2004}.

\hypertarget{Clupeafumosus_socialis-coding-75}{}
\emph{Clupeafumosus socialis}: Round \citep{Topper2013Reappraisalof}.

\hypertarget{Craniops-coding-75}{}
\emph{Craniops}, \emph{Siphonobolus priscus}: Round
\citep{Williams2000LinguliformeaCraniiformea}.

\hypertarget{Dentalium-coding-75}{}
\emph{Dentalium}: Opening essentially polygonal (a rolled rectangle).

\hypertarget{Eoobolus-coding-75}{}
\emph{Eoobolus}: Essentially triangular
\citep{Williams2000LinguliformeaCraniiformea}.

\hypertarget{Gasconsia-coding-75}{}
\emph{Gasconsia}: Round, hinge notwithstanding
\citep{Hanken1985Thetaxonomy}.

\hypertarget{Halkieria_evangelista-coding-75}{}
\emph{Halkieria evangelista}: Anterior shell essentially triangular.

\hypertarget{Kutorgina_chengjiangensis-coding-75}{}
\emph{Kutorgina chengjiangensis}: Lateral margins subparallel
\citep[fig. 125]{Williams2000LinguliformeaCraniiformea}; clear angular
corners in \emph{K. chengjiangensis} \citep{Holmer2018Theattachment}.

\hypertarget{Lingulellotreta_malongensis-coding-75}{}
\emph{Lingulellotreta malongensis}: Linear, diverging lateral margins
\citep{Zhang2007Noteon}.

\hypertarget{Lingulosacculus-coding-75}{}
\emph{Lingulosacculus}: Round \citep{Balthasar2009EarlyCambrian}.

\hypertarget{Mickwitzia_muralensis-coding-75}{}
\emph{Mickwitzia muralensis}: Round \citep{Balthasar2004Shellstructure}.

\hypertarget{Micrina-coding-75}{}
\emph{Micrina}: Mitral valve aperture essentially round
\citep{Holmer2008TheEarly}.

\hypertarget{Micromitra-coding-75}{}
\emph{Micromitra}: Lateral margins subparallel, cf. \emph{Askepasma}
\citep{Robson2001Cambrianand}.

\hypertarget{Mummpikia_nuda-coding-75}{}
\emph{Mummpikia nuda}: Polygonal \citep{Balthasar2008iMummpikia}.

\hypertarget{Novocrania-coding-75}{}
\emph{Novocrania}, \emph{Pelagodiscus atlanticus}: Essentially round
\citep{Williams2000LinguliformeaCraniiformea}.

\hypertarget{Orthis-coding-75}{}
\emph{Orthis}: Essentially round, hinge notwithstanding \citep[fig.
523]{Williams2000LinguliformeaCraniiformea}.

\hypertarget{Paramicrocornus-coding-75}{}
\emph{Paramicrocornus}: Polygonal \citep{Zhang2018Ahyolithid}.

\hypertarget{Paterimitra-coding-75}{}
\emph{Paterimitra}: Broad posterior sinus: not directly comparable with
brachiopod condition.

\hypertarget{Pauxillites-coding-75}{}
\emph{Pauxillites}: Subtriangular \citep{Malinky1987}.

\hypertarget{Salanygolina-coding-75}{}
\emph{Salanygolina}: Short subparallel stretch of lateral sides of adult
shell \citep[fig. 1a]{Holmer2009Theenigmatic}, recalling D-shaped
profile of \emph{Askepasma}.

\hypertarget{Tomteluva_perturbata-coding-75}{}
\emph{Tomteluva perturbata}: Essentially round, sulcus notwithstanding
\citep{Streng2016Anew}.

\hypertarget{Ussunia-coding-75}{}
\emph{Ussunia}: Essentially round, hinge notwithstanding
\citep{Williams2000LinguliformeaCraniiformea}.

\hypertarget{Yuganotheca_elegans-coding-75}{}
\emph{Yuganotheca elegans}: Polygonal \citep{Zhang2014Anearly}.

\subsection*{{[}76{]} Commissure: Lateral
margins}\label{commissure-lateral-margins}
\addcontentsline{toc}{subsection}{{[}76{]} Commissure: Lateral margins}

\includegraphics{Brachiopod_phylogeny_files/figure-latex/character-mapping-76.pdf}

\begin{quote}
\textbf{Character 76: Sclerites: Bivalved: Commissure: Lateral margins}

1: Subparallel\\
2: Diverging\\
3: Initially diverging, becoming subparallel\\
Transformational character.
\end{quote}

If lateral margins are linear, are the subparallel (i.e.~commissure
profile oblong, with long hinge) or diverging (i.e.~commissure profile
triangular, with short hinge)?.

\hypertarget{Maxilites-coding-76}{}
\emph{Maxilites}: Initially diverging, becoming subparallel
\citep{MartiMus2005}.

\subsection*{{[}77{]} Apophyses}\label{apophyses}
\addcontentsline{toc}{subsection}{{[}77{]} Apophyses}

\includegraphics{Brachiopod_phylogeny_files/figure-latex/character-mapping-77.pdf}

\begin{quote}
\textbf{Character 77: Sclerites: Bivalved: Apophyses}

0: Absent\\
1: Present\\
Neomorphic character.
\end{quote}

\emph{Micrina}, like many brachiopods, bears tooth-like structures or
processes that articulate the two primary valves. Caution must be
applied before taxa are coded as ``absent'', as teeth can be subtle and
may be overlooked.

\hypertarget{Alisina-coding-77}{}
\emph{Alisina}: ``Strophic articulation with paired, ventral denticles,
composed of secondary shell'' -- definition of family Trematobolidae in
\citet{Williams2000LinguliformeaCraniiformea}.

\hypertarget{Clupeafumosus_socialis-coding-77}{}
\emph{Clupeafumosus socialis}: No articulating processes evident or
reported by Topper \emph{et al}. \citeyearpar{Topper2013Reappraisalof}.

\hypertarget{Gasconsia-coding-77}{}
\emph{Gasconsia}: ``Articulatory structure comprising ventral cardinal
socket and dorsal hinge plate {[}\ldots{}{]} The shape of the shell
probably correlates strongly with the unique type of articulation, which
consists of a dorsal hinge plate that fits tightly into a cardinal
socket in the ventral valve, with a concave homeodeltidium in the center
of the ventral interarea'' --
\citet{Williams2000LinguliformeaCraniiformea}, p.184, concerning order
Trimerellida.

\hypertarget{Kutorgina_chengjiangensis-coding-77}{}
\emph{Kutorgina chengjiangensis}, \emph{Nisusia sulcata}: Kutorginata
don't have teeth or dental sockets, but their shells are articulated by
``two triangular plates formed by dorsal interarea, bearing oblique
ridges on the inner sides''
\citep[p.~211]{Williams2000LinguliformeaCraniiformea}; this simple hinge
mechanism is different from other rhynchonelliforms
\citep[p.208]{Williams2000LinguliformeaCraniiformea}, but serves an
equivalent purpose and is thus potentially homologous. We thus code
kutorginids as present, using a subsequent character to capture
difference in tooth morphology.

\hypertarget{Mickwitzia_muralensis-coding-77}{}
\emph{Mickwitzia muralensis}: Not reported by or evident in Balthasar
\citeyearpar{Balthasar2004Shellstructure}.

\hypertarget{Mummpikia_nuda-coding-77}{}
\emph{Mummpikia nuda}: No articulation structures are evident; instead,
the propareas are rotated inwards \citep{Balthasar2008iMummpikia}. The
definition of Family Obolellidae in Williams \emph{et al}.
\citeyearpar{Williams2000LinguliformeaCraniiformea} notes that
articulation may be lacking or vestigial in the group.

\hypertarget{Tomteluva_perturbata-coding-77}{}
\emph{Tomteluva perturbata}: Tomteluvids {[}\ldots{}{]} lack
articulation structures such as teeth and sockets
\citep{Streng2016Anew}.

\hypertarget{Tonicella-coding-77}{}
\emph{Tonicella}: The sutural laminae correspond in function and
position to brachiopod apophyses \citep{Connors2012}, and so are coded
as potentially homologous.

\hypertarget{Ussunia-coding-77}{}
\emph{Ussunia}: ``articulatory structures poorly developed'' --
\citet{Williams2000LinguliformeaCraniiformea}, p.~192.

\subsection*{{[}78{]} Apophyses: Morphology}\label{apophyses-morphology}
\addcontentsline{toc}{subsection}{{[}78{]} Apophyses: Morphology}

\includegraphics{Brachiopod_phylogeny_files/figure-latex/character-mapping-78.pdf}

\begin{quote}
\textbf{Character 78: Sclerites: Bivalved: Apophyses: Morphology}

1: Deltidiodont\\
2: Cyrtomatodont\\
3: Pseudodont\\
Transformational character.
\end{quote}

Deltidiodont teeth are simple hinge teeth developed by the distal
accretion of secondary shell; Cyrtomatodont teeth are knoblike or
hook-shaped hinge teeth developed by differential secretion and
resorption of the secondary shell \citep[fig. 322
in][]{Williams1997Introduction}.

Kutorginata (here represented by \emph{Kutorgina} and \emph{Nisusia})
don't have teeth (apophyses) or dental sockets, but their shells are
articulated by ``two triangular plates formed by dorsal interarea,
bearing oblique ridges on the inner sides''
\citep[p.~211]{Williams2000LinguliformeaCraniiformea}; this simple hinge
mechanism is different from other rhynchonelliforms
{[}\citet{Williams2000LinguliformeaCraniiformea}, p.208; table 13
character 30{]}, and is described as a ``pseudodont articulation''
\citep{Holmer2018Evolutionarysignificance}.

\hypertarget{Antigonambonites_planus-coding-78}{}
\emph{Antigonambonites planus}, \emph{Glyptoria}: Coded as deltidiodont
in Benedetto \citeyearpar{Benedetto2009iChaniella}.

\hypertarget{Kutorgina_chengjiangensis-coding-78}{}
\emph{Kutorgina chengjiangensis}: ``Articulation characterized by two
triangular plates formed by dorsal interarea, bearing oblique ridges on
the inner sides'' -- \citet{Williams2000LinguliformeaCraniiformea},
p.~211.

\hypertarget{Micrina-coding-78}{}
\emph{Micrina}: The simple knob-like teeth of \emph{Micrina} show no
evidence of resprobtion or the hook-like shape that characterises
Cyrtomatodont teeth.

\hypertarget{Nisusia_sulcata-coding-78}{}
\emph{Nisusia sulcata}: The `teeth' are formed by the distal lateral
extensions from the ventral\\
pseudodeltidium fitting into the `sockets' on the inner side of the
dorsal interarea \citep{Holmer2018Evolutionarysignificance}. {[}Coded as
``deltidiodont teeth absent'' in Benedetto
\citeyearpar{Benedetto2009iChaniella}.{]}.

\hypertarget{Orthis-coding-78}{}
\emph{Orthis}: Coded as deltidiodont (in \emph{Eoorthis}) in Benedetto
\citeyearpar{Benedetto2009iChaniella}.

\hypertarget{Terebratulina-coding-78}{}
\emph{Terebratulina}: Cyrtomatodont -- see fig. 322 in Williams \emph{et
al}. \citeyearpar{Williams2000LinguliformeaCraniiformea}.

\hypertarget{Tonicella-coding-78}{}
\emph{Tonicella}: Chiton apophyses (sutural laminae) are accretions
deriving from the ventral shell layer of the intermediate and tail
valves \citep{Schwabe2010}, so correspond to the deltidiodont situation
in brachiopods.

\subsection*{{[}79{]} Apophyses: Dental
plates}\label{apophyses-dental-plates}
\addcontentsline{toc}{subsection}{{[}79{]} Apophyses: Dental plates}

\includegraphics{Brachiopod_phylogeny_files/figure-latex/character-mapping-79.pdf}

\begin{quote}
\textbf{Character 79: Sclerites: Bivalved: Apophyses: Dental plates}

0: Absent\\
1: Present\\
Neomorphic character.
\end{quote}

\citet{Williams1997Introduction} (p362) write: ``Teeth {[}\ldots{}{]}
are commonly supported by a pair of variably disposed plates also built
up exclusively of secondary shell and known as dental plates (Fig.
323.1, 323.3).''

Dewing \citeyearpar{Dewing2001Hingemodifications} elaborates: ``Dental
plates are near-vertical, narrow sheets of shell tissue between the
anteromedian edge of the teeth and floor of the ventral valve. They are
a composite structure, resulting from the growth of teeth over the ridge
that bounds the ventral-valve muscle field.''

\citet{Williams2000LinguliformeaCraniiformea} (p.201) write: ``The
denticles lack supporting structures in all Obolellida, but in Naukatida
they are supported by an arcuate plate below the\\
interarea, the anterise (Fig. 119.3a)''.

The anterise is conceivably homologous with the dental plates, thus the
presence of either is coded ``present'' for this character.

\hypertarget{Antigonambonites_planus-coding-79}{}
\emph{Antigonambonites planus}: Coded as present (well developed) in
Benedetto \citeyearpar{Benedetto2009iChaniella}.

\hypertarget{Coolinia_pecten-coding-79}{}
\emph{Coolinia pecten}: Coded as present following Dewing
\citeyearpar{Dewing2001Hingemodifications}, who seems to use the term
Strophomenoids to encompass \emph{Coolinia}, and attests to the presence
of dental plates.

\hypertarget{Gasconsia-coding-79}{}
\emph{Gasconsia}: Coded ambiguous to reflect the possibility that the
hinge plate in trimerellids is homologous to the dental plates of other
taxa, and has replaced the teeth themselves as the primary articulatory
mechanism \citep[see][p.~184, for details of the
articulation]{Williams2000LinguliformeaCraniiformea}.

\hypertarget{Glyptoria-coding-79}{}
\emph{Glyptoria}, \emph{Nisusia sulcata}: Coded as absent in Benedetto
\citeyearpar{Benedetto2009iChaniella}.

\hypertarget{Orthis-coding-79}{}
\emph{Orthis}: Coded as present (short and recessive, in
\emph{Eoorthis}) in Benedetto \citeyearpar{Benedetto2009iChaniella}.

\subsection*{{[}80{]} Sockets}\label{sockets}
\addcontentsline{toc}{subsection}{{[}80{]} Sockets}

\includegraphics{Brachiopod_phylogeny_files/figure-latex/character-mapping-80.pdf}

\begin{quote}
\textbf{Character 80: Sclerites: Bivalved: Sockets}

0: Absent\\
1: Present\\
Neomorphic character.
\end{quote}

Simplified from Bassett \emph{et al}.
\citeyearpar{Bassett2001Functionalmorphology} character 16.\\
This character is independent of apophyses, as several taxa bear sockets
without corresponding teeth; the function of these sockets is unknown.\\
See figs 323ff in Williams \emph{et al}.
\citeyearpar{Williams1997Introduction}.

\hypertarget{Alisina-coding-80}{}
\emph{Alisina}: ``bearing sockets, bounded by low ridges'' --
\citet{Williams2000LinguliformeaCraniiformea}.

\hypertarget{Antigonambonites_planus-coding-80}{}
\emph{Antigonambonites planus}: Coded as present in Benedetto
\citeyearpar{Benedetto2009iChaniella}.

\hypertarget{Gasconsia-coding-80}{}
\emph{Gasconsia}: ``Articulatory structure comprising ventral cardinal
socket and dorsal hinge plate'' --
\citet{Williams2000LinguliformeaCraniiformea}, p.~184.

\hypertarget{Glyptoria-coding-80}{}
\emph{Glyptoria}, \emph{Nisusia sulcata}: Coded as absent in Benedetto
\citeyearpar{Benedetto2009iChaniella}.

\hypertarget{Mickwitzia_muralensis-coding-80}{}
\emph{Mickwitzia muralensis}: Not reported by or evident in Balthasar
\citeyearpar{Balthasar2004Shellstructure}.

\hypertarget{Tomteluva_perturbata-coding-80}{}
\emph{Tomteluva perturbata}: Tomteluvids {[}\ldots{}{]} lack
articulation structures such as teeth and sockets
\citep{Streng2016Anew}.

\hypertarget{Ussunia-coding-80}{}
\emph{Ussunia}: Following table 15 in
\citet{Williams2000LinguliformeaCraniiformea}.

\subsection*{{[}81{]} Socket ridges}\label{socket-ridges}
\addcontentsline{toc}{subsection}{{[}81{]} Socket ridges}

\includegraphics{Brachiopod_phylogeny_files/figure-latex/character-mapping-81.pdf}

\begin{quote}
\textbf{Character 81: Sclerites: Bivalved: Socket ridges}

0: Absent\\
1: Present\\
Neomorphic character.
\end{quote}

After Bassett \emph{et al}.
\citeyearpar{Bassett2001Functionalmorphology} character 17. May be
difficult to distinguish from a brachiophore \citep[see Fig 323
in][]{Williams1997Introduction}, so the two structures are not
distinguished here.

\hypertarget{Alisina-coding-81}{}
\emph{Alisina}: ``bearing sockets, bounded by low ridges'' --
\citet{Williams2000LinguliformeaCraniiformea}.

\hypertarget{Antigonambonites_planus-coding-81}{}
\emph{Antigonambonites planus}: Coded as present in Benedetto
\citeyearpar{Benedetto2009iChaniella}.

\hypertarget{Glyptoria-coding-81}{}
\emph{Glyptoria}, \emph{Nisusia sulcata}: Coded as absent in Benedetto
\citeyearpar{Benedetto2009iChaniella}.

\hypertarget{Tomteluva_perturbata-coding-81}{}
\emph{Tomteluva perturbata}: Tomteluvids {[}\ldots{}{]} lack
articulation structures such as teeth and sockets
\citep{Streng2016Anew}.

\subsection*{{[}82{]} Muscle scars: Ventral}\label{muscle-scars-ventral}
\addcontentsline{toc}{subsection}{{[}82{]} Muscle scars: Ventral}

\includegraphics{Brachiopod_phylogeny_files/figure-latex/character-mapping-82.pdf}

\begin{quote}
\textbf{Character 82: Sclerites: Bivalved: Muscle scars: Ventral }

0: Absent\\
1: Present\\
Neomorphic character.
\end{quote}

After Bassett \emph{et al}.
\citeyearpar{Bassett2001Functionalmorphology} character 6.

\hypertarget{Alisina-coding-82}{}
\emph{Alisina}: Muscle scars scored based on \emph{Alisina}
\emph{comleyensis} \citep{Bassett2001Functionalmorphology}.

\hypertarget{Bactrotheca-coding-82}{}
\emph{Bactrotheca}: ``Muscle scars were not found'' --
\citet{Valent2012}.

\hypertarget{Conotheca-coding-82}{}
\emph{Conotheca}: {[}Reference needed{]}.

\hypertarget{Halkieria_evangelista-coding-82}{}
\emph{Halkieria evangelista}: Muscle scars are known from the Type A,
but not Type B, morphs of the halkieriid \emph{Oikozetetes}
\citep{Paterson2009, Jacquet2014}.

\hypertarget{Mickwitzia_muralensis-coding-82}{}
\emph{Mickwitzia muralensis}: Scars absent; instead, cones ornament
shell's internal surface.

\hypertarget{Micrina-coding-82}{}
\emph{Micrina}: Prominent ventral muscle scars -- see e.g.
\citet{Holmer2008TheEarly}, fig. 1f.

\hypertarget{Paramicrocornus-coding-82}{}
\emph{Paramicrocornus}: Not preserved in conchs, though likely present
(by analogy with \emph{Maxilites}) \citep{Zhang2018Ahyolithid}.

\hypertarget{Tonicella-coding-82}{}
\emph{Tonicella}: Absent \citep{Schwabe2010}.

\hypertarget{Ussunia-coding-82}{}
\emph{Ussunia}: Scars preserved on ventral valve \citep{Nikitin1984}.

\subsection*{{[}83{]} Muscle scars: Ventral:
Position}\label{muscle-scars-ventral-position}
\addcontentsline{toc}{subsection}{{[}83{]} Muscle scars: Ventral:
Position}

\includegraphics{Brachiopod_phylogeny_files/figure-latex/character-mapping-83.pdf}

\begin{quote}
\textbf{Character 83: Sclerites: Bivalved: Muscle scars: Ventral:
Position}

1: Posterolateral and medial attachments\\
2: Medial attachments only\\
Transformational character.
\end{quote}

Muscles can attach to the ventral valve posterolaterally to, as well as
between, the \emph{vascula} \emph{lateralia}
\citep{Popov1992TheCambrian}.

\hypertarget{Acanthotretella_spinosa-coding-83}{}
\emph{Acanthotretella spinosa}: ``Individual muscle scars cannot be
distinguished'' -- \citet{Holmer2006Aspinose}.

\hypertarget{Alisina-coding-83}{}
\emph{Alisina}: Following reconstruction of Gorjansky \& Popov
\citeyearpar{Gorjansky1986Onthe}.

\hypertarget{Askepasma_toddense-coding-83}{}
\emph{Askepasma toddense}: Restricted to medial field, following the
interpretation of the musculature presented by Williams \emph{et al}.
\citeyearpar{Williams2000LinguliformeaCraniiformea}, fig. 81.

\hypertarget{Clupeafumosus_socialis-coding-83}{}
\emph{Clupeafumosus socialis}: Coded following \emph{Hadrotreta}, as
illustrated in Popov \citeyearpar{Popov1992TheCambrian}.

\hypertarget{Craniops-coding-83}{}
\emph{Craniops}: See fig. 89 in Williams \emph{et al}.
\citeyearpar{Williams2000LinguliformeaCraniiformea}.

\hypertarget{Eoobolus-coding-83}{}
\emph{Eoobolus}: The `laterals' of Balthasar \citeyearpar[fig.
5]{Balthasar2009Thebrachiopod} are situated almost upon the
\emph{vascula} \emph{lateralia}; they are interpreted as sitting
posterolateral to them.

\hypertarget{Gasconsia-coding-83}{}
\emph{Gasconsia}: Musculature described in Hanken \& Harper
\citeyearpar{Hanken1985Thetaxonomy}, but location of mantle canals
unknown.

\hypertarget{Glyptoria-coding-83}{}
\emph{Glyptoria}: Posterolateral reflected by diductor attachments; see
fig. 18.3.2 in \citet{Bassett2001Functionalmorphology}.

\hypertarget{Kutorgina_chengjiangensis-coding-83}{}
\emph{Kutorgina chengjiangensis}: Following situation in \emph{Nisusia};
see fig. 18.2 in Bassett \emph{et al}.
\citeyearpar{Bassett2001Functionalmorphology}.

\hypertarget{Lingulellotreta_malongensis-coding-83}{}
\emph{Lingulellotreta malongensis}: See fig. 5 in
\citet{Holmer1997EarlyCambrian}.

\hypertarget{Micromitra-coding-83}{}
\emph{Micromitra}: Posteriomedial muscle field \citep[text-fig.
6]{Williams1998Thediversity} treated as equivalent to posterolateral
muscles.

\hypertarget{Nisusia_sulcata-coding-83}{}
\emph{Nisusia sulcata}: Posterolateral diductors \citep[fig. 18.2
in][]{Bassett2001Functionalmorphology}.

\hypertarget{Novocrania-coding-83}{}
\emph{Novocrania}: Posterior adductor muscles attach posterolaterally to
ventral mantle canal \citep{Robinson2014Themuscles}.

\hypertarget{Orthis-coding-83}{}
\emph{Orthis}: Not applicable: \emph{vascula} \emph{lateralia} not
comparable to those of other taxa.

\hypertarget{Pelagodiscus_atlanticus-coding-83}{}
\emph{Pelagodiscus atlanticus}: Inapplicable as vascular system not
directly equivalent to the canonical; see. fig 6b in Balthasar
\citeyearpar{Balthasar2009Thebrachiopod}.

\hypertarget{Salanygolina-coding-83}{}
\emph{Salanygolina}: Ventral musculature not clearly constrained
\citep{Holmer2009Theenigmatic}.

\hypertarget{Siphonobolus_priscus-coding-83}{}
\emph{Siphonobolus priscus}: Coded following general siphonotretid
condition described by Popov \citeyearpar[p.~407]{Popov1992TheCambrian}.

\hypertarget{Ussunia-coding-83}{}
\emph{Ussunia}: Internal anatomy not adequately preserved to evaluate
\citep{Nikitin1984}.

\subsection*{{[}84{]} Muscle scars:
Adjustor}\label{muscle-scars-adjustor}
\addcontentsline{toc}{subsection}{{[}84{]} Muscle scars: Adjustor}

\includegraphics{Brachiopod_phylogeny_files/figure-latex/character-mapping-84.pdf}

\begin{quote}
\textbf{Character 84: Sclerites: Bivalved: Muscle scars: Adjustor}

0: Absent\\
1: Present\\
Neomorphic character.
\end{quote}

After Bassett \emph{et al}.
\citeyearpar{Bassett2001Functionalmorphology} character 7.\\
This character is contingent on the presence of a pedicle. Extreme
caution must be used in inferring an absent state, as adjustor scars can
be extremely difficult to distinguish from the adductor scars.

\hypertarget{Alisina-coding-84}{}
\emph{Alisina}: Muscle scars scored based on \emph{Alisina}
\emph{comleyensis} \citep{Bassett2001Functionalmorphology}. The presence
of an adjustor is marked as not presently available, as it is not clear
that a scar, if present, could be distinguished from the diminutive
muscle scars present.

\hypertarget{Askepasma_toddense-coding-84}{}
\emph{Askepasma toddense}: Following the interpretation of the
musculature presented by Williams \emph{et al}.
\citeyearpar{Williams2000LinguliformeaCraniiformea}, fig. 81.

\hypertarget{Botsfordia-coding-84}{}
\emph{Botsfordia}: Not described in Popov
\citeyearpar{Popov1992TheCambrian}.

\hypertarget{Clupeafumosus_socialis-coding-84}{}
\emph{Clupeafumosus socialis}: Not known in any acrotretid
\citep{Williams2000LinguliformeaCraniiformea}; not evident in
\emph{Clupeafumosus} \citep{Topper2013Reappraisalof}.

\hypertarget{Gasconsia-coding-84}{}
\emph{Gasconsia}: No mention of an adjustor muscle in \emph{Gasconsia}
or Trimerellida more generally on pp.~184--185 of
\citet{Williams2000LinguliformeaCraniiformea}, nor in discussion in
\citet{Williams2007Supplement} (p.~2850). Coded as absent.

\hypertarget{Mickwitzia_muralensis-coding-84}{}
\emph{Mickwitzia muralensis}: Scars absent; instead, cones ornament
shell's internal surface.

\hypertarget{Siphonobolus_priscus-coding-84}{}
\emph{Siphonobolus priscus}: Ventral musculature poorly constrained
\citep{Williams2000LinguliformeaCraniiformea, Popov2009Earlyontogeny}.

\subsection*{{[}85{]} Muscle scars: Dorsal
adductors}\label{muscle-scars-dorsal-adductors}
\addcontentsline{toc}{subsection}{{[}85{]} Muscle scars: Dorsal
adductors}

\includegraphics{Brachiopod_phylogeny_files/figure-latex/character-mapping-85.pdf}

\begin{quote}
\textbf{Character 85: Sclerites: Bivalved: Muscle scars: Dorsal
adductors}

1: Dispersed\\
2: Radially arranged\\
3: Quadripartite\\
Transformational character.
\end{quote}

After Bassett \emph{et al}.
\citeyearpar{Bassett2001Functionalmorphology} character 8, and Williams
\emph{et al}. {[}\citet{Williams1996Asupra}, character 35; 2000, p.~160,
character 54{]}

In the dorsal valve, the anterior and posterior adductor scars of
articulated brachiopods form a single (quadripartite) muscle field
\citep[p.~201]{Williams2000LinguliformeaCraniiformea}

In contrast, the anterior and posterior scars of e.g.~trimerellids have
prominently separate attachment points, with anterior and posterior
muscle fields clearly distinct, and coded as ``dispersed''.

In e.g.~kutorginates, adductor muscles are separated into left and right
fields; the same is the case in lingulids, where there are more separate
muscle groups and the left and right fields conspire to produce a radial
arrangement; both of these configurations are scored as ``radially
arranged''.

\hypertarget{Alisina-coding-85}{}
\emph{Alisina}: Following Williams \emph{et al}.
\citeyearpar{Williams2000LinguliformeaCraniiformea} table 15 (their
character 54).

\hypertarget{Antigonambonites_planus-coding-85}{}
\emph{Antigonambonites planus}: Treatise.

\hypertarget{Askepasma_toddense-coding-85}{}
\emph{Askepasma toddense}: Separate left and right fields, so radially
arranged -- following the interpretation of the musculature presented by
Williams \emph{et al}.
\citeyearpar{Williams2000LinguliformeaCraniiformea}, fig. 81.

\hypertarget{Botsfordia-coding-85}{}
\emph{Botsfordia}: Following \citet{Williams1998Thediversity}, appendix
2.

\hypertarget{Clupeafumosus_socialis-coding-85}{}
\emph{Clupeafumosus socialis}: Following reconstruction of
\emph{Hadrotreta} by Williams
\citeyearpar{Williams2000LinguliformeaCraniiformea}, fig. 51, which
exhibits distinct left and right fields.

\hypertarget{Coolinia_pecten-coding-85}{}
\emph{Coolinia pecten}: ``radially arranged adductor scars'' --
\citet{Bassett2017Earliestontogeny}, p1.

\hypertarget{Gasconsia-coding-85}{}
\emph{Gasconsia}: Following the coding of Williams \emph{et al}.
\citeyearpar{Williams2000LinguliformeaCraniiformea}, table 15.

\hypertarget{Glyptoria-coding-85}{}
\emph{Glyptoria}: Scored as ``dispersed'' by Williams \emph{et al}.
\citeyearpar{Williams1998Thediversity} \ldots{} but then so is
\emph{Kutorgina}, which Bassett \emph{et al}.
\citeyearpar{Bassett2001Functionalmorphology} score as radial.

Williams \emph{et al}.
\citeyearpar{Williams2000LinguliformeaCraniiformea} state, for
superfamily Protorthida, ``dorsal adductor scars probably linear'',
which fits in the category of ``radial'' employed herein -- so that's
what we follow.

\hypertarget{Halkieria_evangelista-coding-85}{}
\emph{Halkieria evangelista}: It is unclear whether the paired muscle
scars of \emph{Oikozetetes} may be homologous to brachiopod adductors.

\hypertarget{Haplophrentis_carinatus-coding-85}{}
\emph{Haplophrentis carinatus}: Moysiuk \emph{et al}.
\citeyearpar{Moysiuk2017Hyolithsare} reconstruct distinct left and right
attachment scars, consistent with general situation in hyoliths
\citep[see][]{Dzik1980Ontogenyof}; it is unclear whether additional
smaller scars were present in a radial arrangement \citep[as in e.g.
\emph{Gompholites},][]{Marek1967} or whether unseen scars were
dispersed, hence the partially ambiguous coding.

\hypertarget{Heliomedusa_orienta-coding-85}{}
\emph{Heliomedusa orienta}: Distinct anterior and posterior fields
\citep{Chen2007Reinterpretationof}; coded as ``dispersed'' by Williams
\emph{et al}. \citeyearpar{Williams2000LinguliformeaCraniiformea} in
table 15.

\hypertarget{Maxilites-coding-85}{}
\emph{Maxilites}: A pair of large oval muscle scars on the conical
shield is complemented with smaller scars elsewhere on the operculum
\citep{Marek1972, MartiMus2005}.

\hypertarget{Mickwitzia_muralensis-coding-85}{}
\emph{Mickwitzia muralensis}: Scars absent; instead, cones ornament
shell's internal surface.

\hypertarget{Micromitra-coding-85}{}
\emph{Micromitra}: Williams \emph{et al}.
\citeyearpar{Williams1998Thediversity} code as ``dispersed'', but have a
less divided scheme of character states and disagree with other sources
in some codings \citep[e.g.][in
Kutorginates]{Bassett2001Functionalmorphology}. Williams \emph{et al}.
\citeyearpar{Williams2000LinguliformeaCraniiformea} do not describe
\emph{Micromitra} musculature and we were unable to find any reliable
description of the scars, so we code as ``not presently available''.

\hypertarget{Novocrania-coding-85}{}
\emph{Novocrania}: Craniids scored as ``open, quadripartite'' by
Williams \emph{et al}. \citeyearpar{Williams1996Asupra}.

\hypertarget{Pelagodiscus_atlanticus-coding-85}{}
\emph{Pelagodiscus atlanticus}: Discinids scored as ``open,
quadripartite'' by Williams \emph{et al}.
\citeyearpar{Williams1996Asupra}.

\hypertarget{Salanygolina-coding-85}{}
\emph{Salanygolina}: ``The dorsal valve of \emph{Salanygolina} has a
radial arrangement of adductor muscle scars and the scars of
posteromedially placed internal oblique muscles, which are also
characteristic of paterinates and chileates'' -- Holmer \emph{et al}.
\citeyearpar{Holmer2009Theenigmatic}.

\hypertarget{Siphonobolus_priscus-coding-85}{}
\emph{Siphonobolus priscus}: Ventral musculature poorly constrained
\citep{Williams2000LinguliformeaCraniiformea, Popov2009Earlyontogeny}.

\hypertarget{Terebratulina-coding-85}{}
\emph{Terebratulina}: Coded as ``grouped, quadripartite'' by Williams
\emph{et al}. \citeyearpar{Williams1996Asupra}.

\hypertarget{Ussunia-coding-85}{}
\emph{Ussunia}: Following coding with state 0 (dispersed) in table 15 in
\citet{Williams2000LinguliformeaCraniiformea}.

\subsection*{{[}86{]} Muscle scars: Adductors:
Position}\label{muscle-scars-adductors-position}
\addcontentsline{toc}{subsection}{{[}86{]} Muscle scars: Adductors:
Position}

\includegraphics{Brachiopod_phylogeny_files/figure-latex/character-mapping-86.pdf}

\begin{quote}
\textbf{Character 86: Sclerites: Bivalved: Muscle scars: Adductors:
Position}

1: Oblique\\
2: At high angle\\
Transformational character.
\end{quote}

Position of adductor muscles relative to commissural plane.\\
After Bassett \emph{et al}.
\citeyearpar{Bassett2001Functionalmorphology} character 11.

\hypertarget{Askepasma_toddense-coding-86}{}
\emph{Askepasma toddense}: Following the interpretation of the
musculature presented by Williams \emph{et al}.
\citeyearpar{Williams2000LinguliformeaCraniiformea}, fig. 81.

\hypertarget{Botsfordia-coding-86}{}
\emph{Botsfordia}: Following description of Popov
\citeyearpar{Popov1992TheCambrian}.

\hypertarget{Coolinia_pecten-coding-86}{}
\emph{Coolinia pecten}: Not reported by Williams \emph{et al}.
\citeyearpar{Williams2000LinguliformeaCraniiformea}, nor Bassett \&
Popov \citeyearpar{Bassett2017Earliestontogeny}, nor explicitly by
Dewing \citeyearpar{Dewing2001Hingemodifications}.

\hypertarget{Eoobolus-coding-86}{}
\emph{Eoobolus}: ``\emph{Eoobolus} should have anterior and posterior
adductors and a variety of oblique muscles which were probably arranged
in criss-crossing pairs'' -- \citet{Balthasar2009Thebrachiopod}.

\hypertarget{Gasconsia-coding-86}{}
\emph{Gasconsia}: See discussion under Trimerellida in Williams \emph{et
al}. \citeyearpar{Williams2000LinguliformeaCraniiformea}.

\hypertarget{Mickwitzia_muralensis-coding-86}{}
\emph{Mickwitzia muralensis}: Scars absent; instead, cones ornament
shell's internal surface.

\hypertarget{Pelagodiscus_atlanticus-coding-86}{}
\emph{Pelagodiscus atlanticus}: Musculature considered essentially
equivalent to \emph{Lingula} by
\citet{Williams2000LinguliformeaCraniiformea}, so \emph{Lingula} coding
followed here.

\hypertarget{Siphonobolus_priscus-coding-86}{}
\emph{Siphonobolus priscus}: Ventral musculature poorly constrained
\citep{Williams2000LinguliformeaCraniiformea, Popov2009Earlyontogeny}.

\subsection*{{[}87{]} Muscle scars: Dermal
muscles}\label{muscle-scars-dermal-muscles}
\addcontentsline{toc}{subsection}{{[}87{]} Muscle scars: Dermal muscles}

\includegraphics{Brachiopod_phylogeny_files/figure-latex/character-mapping-87.pdf}

\begin{quote}
\textbf{Character 87: Sclerites: Bivalved: Muscle scars: Dermal muscles}

0: Absent or weakly developed\\
1: Strongly developed\\
Neomorphic character.
\end{quote}

Based on character 11 in Zhang \emph{et al}.
\citeyearpar{Zhang2014Anearly}.\\
Well developed dermal muscles present in the body wall of recent
lingulates, which are absent in all calcareous-shelled brachiopods.
These muscles are responsible for the hydraulic shell-opening mechanism,
and possibly present in all organophosphatic-shelled brachiopods, with
the possible exception of the paterinates
\citep[p.~32]{Williams2000LinguliformeaCraniiformea}.

\hypertarget{Alisina-coding-87}{}
\emph{Alisina}, \emph{Antigonambonites planus}, \emph{Gasconsia},
\emph{Glyptoria}, \emph{Nisusia sulcata}, \emph{Orthis},
\emph{Salanygolina}: According to the statement of Williams \emph{et
al}. \citeyearpar[p.~32]{Williams2000LinguliformeaCraniiformea} that
these muscle are absent in all carbonate- shelled brachiopods.

\hypertarget{Askepasma_toddense-coding-87}{}
\emph{Askepasma toddense}: According to the statement of Williams
\emph{et al}. \citeyearpar[p.~32]{Williams2000LinguliformeaCraniiformea}
that the presence of these muscles in paterinates is uncertain.

\hypertarget{Botsfordia-coding-87}{}
\emph{Botsfordia}: Implicitly taken as present in Popov
\citeyearpar{Popov1992TheCambrian}, though not marked in diagrams --
suggesting not strongly developed.

\hypertarget{Clupeafumosus_socialis-coding-87}{}
\emph{Clupeafumosus socialis}: This character is coded based on the
score of Acrotreta in Zhang \emph{et al}.
\citeyearpar{Zhang2014Anearly}, and statement in Williams \emph{et al}.
\citeyearpar[P.32]{Williams2000LinguliformeaCraniiformea}.

\hypertarget{Coolinia_pecten-coding-87}{}
\emph{Coolinia pecten}: According to the statement of Williams \emph{et
al}. \citeyearpar[p.~32]{Williams2000LinguliformeaCraniiformea} that
these muscle are absent in all carbonate-shelled brachiopods.

\hypertarget{Eoobolus-coding-87}{}
\emph{Eoobolus}: Not remarked upon by Balthasar
\citeyearpar{Balthasar2009Thebrachiopod}.

\hypertarget{Kutorgina_chengjiangensis-coding-87}{}
\emph{Kutorgina chengjiangensis}: According to the statement of Williams
\emph{et al}. \citeyearpar[p.~32]{Williams2000LinguliformeaCraniiformea}
that these muscle are absent in all carbonate- shelled brachiopods, and
the coding for kutorginids in Zhang \emph{et al}.
\citeyearpar{Zhang2014Anearly}.

\hypertarget{Micromitra-coding-87}{}
\emph{Micromitra}: Williams \emph{et al}.
\citeyearpar[p.~32]{Williams2000LinguliformeaCraniiformea} are uncertain
about the presence of these muscles in the paterinates. Zhang \emph{et
al}. \citeyearpar{Zhang2014Anearly} code absence in Paterinida, but
without specifying evidence; we follow their coding here.

\hypertarget{Mummpikia_nuda-coding-87}{}
\emph{Mummpikia nuda}, \emph{Tomteluva perturbata}: Though Williams
\emph{et al}. \citeyearpar[p.~32]{Williams2000LinguliformeaCraniiformea}
state that these muscles are absent in all carbonate-shelled
brachiopods, their existence cannot be discounted with certainty in this
taxon, which is therefore coded not presently available.

\hypertarget{Novocrania-coding-87}{}
\emph{Novocrania}: Following Zhang \emph{et al}.
\citeyearpar{Zhang2014Anearly}, and the statement of Williams \emph{et
al}. \citeyearpar{Williams2000LinguliformeaCraniiformea} that such
muscles are absent in all calcite-shelled brachiopods.

\hypertarget{Pelagodiscus_atlanticus-coding-87}{}
\emph{Pelagodiscus atlanticus}: Musculature considered essentially
equivalent to \emph{Lingula} by
\citet{Williams2000LinguliformeaCraniiformea}, so \emph{Lingula} coding
followed here.

\hypertarget{Siphonobolus_priscus-coding-87}{}
\emph{Siphonobolus priscus}: Ventral musculature poorly constrained
\citep{Williams2000LinguliformeaCraniiformea, Popov2009Earlyontogeny}.

\hypertarget{Terebratulina-coding-87}{}
\emph{Terebratulina}: Williams \emph{et al}.
\citeyearpar[p.~32]{Williams2000LinguliformeaCraniiformea} state that
these muscles are absent in all carbonate-shelled brachiopods.

\subsection*{\texorpdfstring{{[}88{]} Muscle scars: Unpaired median
(\emph{levator
ani})}{{[}88{]} Muscle scars: Unpaired median (levator ani)}}\label{muscle-scars-unpaired-median-levator-ani}
\addcontentsline{toc}{subsection}{{[}88{]} Muscle scars: Unpaired median
(\emph{levator ani})}

\includegraphics{Brachiopod_phylogeny_files/figure-latex/character-mapping-88.pdf}

\begin{quote}
\textbf{Character 88: Sclerites: Bivalved: Muscle scars: Unpaired median
(\emph{levator ani})}

0: Absent\\
1: Present\\
Neomorphic character.
\end{quote}

The \emph{levator ani} is a diminutive unpaired medial muscle found in
certain calcitic brachiopods
{[}\citet{Williams2000LinguliformeaCraniiformea}; see fig. 89, character
34 in table 13{]}.

\hypertarget{Alisina-coding-88}{}
\emph{Alisina}, \emph{Kutorgina chengjiangensis}, \emph{Nisusia
sulcata}: Following table 13 in
\citet{Williams2000LinguliformeaCraniiformea}.

\hypertarget{Coolinia_pecten-coding-88}{}
\emph{Coolinia pecten}: Not reported in Dewing
\citeyearpar{Dewing2001Hingemodifications}.

\hypertarget{Craniops-coding-88}{}
\emph{Craniops}: See fig. 90 in
\citet{Williams2000LinguliformeaCraniiformea}.

\hypertarget{Gasconsia-coding-88}{}
\emph{Gasconsia}: \citet{Williams2000LinguliformeaCraniiformea} code an
unpaired medial muscle scar as present in their table 13, but give no
reference for this coding, which perhaps arises from their
interpretation of the taxon as a trimerellid. Hanken and Harper
\citeyearpar[p.~249 and text-fig. 2]{Hanken1985Thetaxonomy} explicitly
identify a pair of central muscles, so we code a \emph{levator ani} as
absent.

\hypertarget{Heliomedusa_orienta-coding-88}{}
\emph{Heliomedusa orienta}: Poor preservation of minor muscle scars
noted by Chen \emph{et al}. \citeyearpar{Chen2007Reinterpretationof}.

\hypertarget{Mickwitzia_muralensis-coding-88}{}
\emph{Mickwitzia muralensis}: Scars absent; instead, cones ornament
shell's internal surface.

\hypertarget{Novocrania-coding-88}{}
\emph{Novocrania}: Following table 13 in
\citet{Williams2000LinguliformeaCraniiformea} (for \emph{Novocrania}).

\hypertarget{Pelagodiscus_atlanticus-coding-88}{}
\emph{Pelagodiscus atlanticus}: Musculature considered essentially
equivalent to \emph{Lingula} by
\citet{Williams2000LinguliformeaCraniiformea}, so \emph{Lingula} coding
followed here.

\hypertarget{Siphonobolus_priscus-coding-88}{}
\emph{Siphonobolus priscus}: Ventral musculature poorly constrained
\citep{Williams2000LinguliformeaCraniiformea, Popov2009Earlyontogeny}.

\hypertarget{Ussunia-coding-88}{}
\emph{Ussunia}: Following table 15 in
\citet{Williams2000LinguliformeaCraniiformea}.

\subsection*{{[}89{]} Muscle scars: Dorsal
diductor}\label{muscle-scars-dorsal-diductor}
\addcontentsline{toc}{subsection}{{[}89{]} Muscle scars: Dorsal
diductor}

\includegraphics{Brachiopod_phylogeny_files/figure-latex/character-mapping-89.pdf}

\begin{quote}
\textbf{Character 89: Sclerites: Bivalved: Muscle scars: Dorsal
diductor}

0: Absent\\
1: Present\\
Neomorphic character.
\end{quote}

After Bassett \emph{et al}.
\citeyearpar{Bassett2001Functionalmorphology} character 9.

\hypertarget{Acanthotretella_spinosa-coding-89}{}
\emph{Acanthotretella spinosa}: Not observable in \emph{Acanthotretella}
itself, so coded as ambiguous -- though it is likely based on the
anticipated phylogenetic affinities of \emph{Acanthotretella} that the
muscles are absent.

\hypertarget{Askepasma_toddense-coding-89}{}
\emph{Askepasma toddense}: Possible diductor scar could instead
correspond to discinoid posterior adductors
\citep{Williams1998Thediversity}; coded as uncertain. Not reconstructed
in the the interpretation of the musculature presented by Williams
\emph{et al}. \citeyearpar{Williams2000LinguliformeaCraniiformea}, fig.
81.

\hypertarget{Clupeafumosus_socialis-coding-89}{}
\emph{Clupeafumosus socialis}: Not reported by Topper \emph{et al}.
\citeyearpar{Topper2013Reappraisalof}, nor reconstructed in generic
acrotretid by Williams \emph{et al}.
\citeyearpar{Williams2000LinguliformeaCraniiformea}.

\hypertarget{Gasconsia-coding-89}{}
\emph{Gasconsia}: Internal oblique muscles serve as diductors.

\hypertarget{Halkieria_evangelista-coding-89}{}
\emph{Halkieria evangelista}: It is unclear whether the paired muscle
scars of \emph{Oikozetetes} are homologous to brachiopod diductors.

\hypertarget{Micromitra-coding-89}{}
\emph{Micromitra}: Possible diductor scar could instead correspond to
discinoid posterior adductors \citep{Williams1998Thediversity}; coded as
uncertain.

\hypertarget{Siphonobolus_priscus-coding-89}{}
\emph{Siphonobolus priscus}: Ventral musculature poorly constrained
\citep{Williams2000LinguliformeaCraniiformea, Popov2009Earlyontogeny}.

\hypertarget{Ussunia-coding-89}{}
\emph{Ussunia}: Internal oblique muscles present \citep{Nikitin1984} and
taken to serve as diductors by analogy with \emph{Gasconsia}.

\subsection*{{[}90{]} Muscle scars: Dorsal diductor:
Position}\label{muscle-scars-dorsal-diductor-position}
\addcontentsline{toc}{subsection}{{[}90{]} Muscle scars: Dorsal
diductor: Position}

\includegraphics{Brachiopod_phylogeny_files/figure-latex/character-mapping-90.pdf}

\begin{quote}
\textbf{Character 90: Sclerites: Bivalved: Muscle scars: Dorsal
diductor: Position}

1: Close to commissural plane\\
2: Oblique to commissural plane\\
3: At high angle to commissural plane\\
Transformational character.
\end{quote}

After Bassett \emph{et al}.
\citeyearpar{Bassett2001Functionalmorphology} character 10.

\hypertarget{Siphonobolus_priscus-coding-90}{}
\emph{Siphonobolus priscus}: Ventral musculature poorly constrained
\citep{Williams2000LinguliformeaCraniiformea, Popov2009Earlyontogeny}.

\section{Sclerites: Dorsal valve}\label{sclerites-dorsal-valve}

\subsection*{{[}91{]} Growth direction}\label{growth-direction}
\addcontentsline{toc}{subsection}{{[}91{]} Growth direction}

\includegraphics{Brachiopod_phylogeny_files/figure-latex/character-mapping-91.pdf}

\begin{quote}
\textbf{Character 91: Sclerites: Dorsal valve: Growth direction}

1: Holoperipheral\\
2: Mixoperipheral\\
3: Hemiperipheral\\
Transformational character.
\end{quote}

See Fig. 284 in Williams \emph{et al}.
\citeyearpar{Williams1997Introduction}.\\
The growth direction dictates the attitude of the cardinal area relative
to the hinge, which does not therefore represent an independent
character.\\
Crudely put, if, viewed from a dorsal position, the umbo falls within
the outer margin of the shell, growth is holoperipheral; if it falls
outside the margin, it is mixoperipheral; if it falls exactly on the
margin, it is hemiperipheral.

\hypertarget{Clupeafumosus_socialis-coding-91}{}
\emph{Clupeafumosus socialis}: Appears hemiperipheral in fig. 3 in
Topper \emph{et al}. \citeyearpar{Topper2013Reappraisalof}, though
bordering on holoperipheral, so scored as ambiguous.

\hypertarget{Craniops-coding-91}{}
\emph{Craniops}: ``Both valves with growth holoperipheral'' --
\citet{Williams2000LinguliformeaCraniiformea}, p.~164.

\hypertarget{Heliomedusa_orienta-coding-91}{}
\emph{Heliomedusa orienta}: ``holoperipheral growth in dorsal valve'' --
\citet{Williams2007Supplement}.

Zhang \emph{et al}. \citeyearpar{Zhang2009Architectureand} conclude that
Chen \emph{et al}. \citeyearpar{Chen2007Reinterpretationof} misidentify
the dorsal valve as the ventral valve.

\hypertarget{Micrina-coding-91}{}
\emph{Micrina}: See Holmer \emph{et al}.
\citeyearpar{Holmer2008TheEarly}.

\hypertarget{Paterimitra-coding-91}{}
\emph{Paterimitra}: S2 and L sclerites are clearly holoperipheral. See
\citet{Larsson2014iPaterimitra}, fig. 2.

\hypertarget{Tonicella-coding-91}{}
\emph{Tonicella}: For the purposes of this analysis, we must treat
polyplacophoran and brachiopod valves as potentially homologous.

In brachiopods, the dorsal valve bears the lophophore, which arises from
the anterior lobe of the larva \citep{Altenburger2013} -- indicating
that the dorsal shell field is associated with the anterior lobe.

In polyplacophorans, the head valve arises from a shell field on the
anterior (pre-prototroch) lobe of the larva \citep{Wanninger2002C},
which we therefore treat as homologous with the brachiopod dorsal valve.

In support of this hypothesis, we note that the posterior (but not
anterior) valves of chitons bear apophyses
\citep{Schwabe2010, Connors2012}, which are most prominent in the
ventral (but not dorsal) valves of brachiopods \citep[fig.
322]{Williams1997Introduction}, and which occur in the morph A shell of
\emph{Oikozetetes}, which is interpreted as the posterior valve of a
halkieriid \citep{Paterson2009}.

As the single posterior shell field of polyplacophorans subdivides to
give rise to the six intermediate valves plus the tail valve
\citep{Wanninger2002C}, we prefer to consider the intermediate valves as
representing ``subdivisions'' of a single valve rather than additional
valves added to the body plan.

Growth is hemiperipheral in the anterior valve of polyplacophorans and
holoperipheral in the posterior valves \citep{Schwabe2010, Connors2012}.

\hypertarget{Ussunia-coding-91}{}
\emph{Ussunia}: Growth ``mixoperipheral in both valves'' in trimerellids
\citep{Williams2000LinguliformeaCraniiformea, Popov1997}.

\subsection*{{[}92{]} Posterior surface:
Differentiated}\label{posterior-surface-differentiated}
\addcontentsline{toc}{subsection}{{[}92{]} Posterior surface:
Differentiated}

\includegraphics{Brachiopod_phylogeny_files/figure-latex/character-mapping-92.pdf}

\begin{quote}
\textbf{Character 92: Sclerites: Dorsal valve: Posterior surface:
Differentiated}

0: Posterior face of dorsal valve not differentiated\\
1: Posterior face of dorsal valve forms distinct cardinal area or
pseudointerarea\\
Neomorphic character.
\end{quote}

In shells that grow by mixoperipheral growth, the triangular area
subtended between each apex and the posterior ends of the lateral
margins is termed the cardinal area. In shells with holoperipheral
growth, a flattened surface on the posterior margin of the valve is
termed a pseudointerarea
\citep[paraphrasing][]{Williams1997Introduction}.

In order for this character to be independent of a shell's growth
direction, we do not distinguish between a ``cardinal area'',
``interarea'' or ``pseudointerarea''.

\hypertarget{Acanthotretella_spinosa-coding-92}{}
\emph{Acanthotretella spinosa}: Pseudointerarea present, following
Siphonotretidae coding in Williams \emph{et al}.
\citeyearpar{Williams2000LinguliformeaCraniiformea}, table 6.

\hypertarget{Alisina-coding-92}{}
\emph{Alisina}, \emph{Antigonambonites planus}, \emph{Coolinia pecten},
\emph{Glyptoria}, \emph{Kutorgina chengjiangensis}, \emph{Orthis},
\emph{Salanygolina}, \emph{Tomteluva perturbata}: Cardinal area
(interarea) present.

\hypertarget{Askepasma_toddense-coding-92}{}
\emph{Askepasma toddense}: Well-defined pseudointerarea
\citep[p153]{Williams2000LinguliformeaCraniiformea}.

\hypertarget{Botsfordia-coding-92}{}
\emph{Botsfordia}: ``dorsal pseudointerarea vestigial, divided by median
groove'' -- \citet{Williams2000LinguliformeaCraniiformea}.

\hypertarget{Clupeafumosus_socialis-coding-92}{}
\emph{Clupeafumosus socialis}: Pseudointerarea present; figured by
Topper \emph{et al}. \citeyearpar{Topper2013Reappraisalof}, fig. 3j.

\hypertarget{Craniops-coding-92}{}
\emph{Craniops}: ``Only some craniopsids (Lingulapholis, Pseudopholidops
{[}not \emph{Craniops}{]}) have well-developed pseudointerareas.'' --
\citet{Williams2000LinguliformeaCraniiformea}.

\hypertarget{Gasconsia-coding-92}{}
\emph{Gasconsia}: Absent: the dorsal (branchial) pseudointerarea of
\emph{G. schucherti} is ``reduced or obsolete''; that of \emph{G.
worsleyi} ``short, virtually obsolete'' \citep{Hanken1985Thetaxonomy}.

\hypertarget{Haplophrentis_carinatus-coding-92}{}
\emph{Haplophrentis carinatus}: A very short pseudointerarea appears to
be present \citep{Moysiuk2017Hyolithsare}.

\hypertarget{Heliomedusa_orienta-coding-92}{}
\emph{Heliomedusa orienta}: Pseudointerea in ventral valve, but not
dorsal valve \citep[2007]{Williams2000LinguliformeaCraniiformea}.

\hypertarget{Lingula-coding-92}{}
\emph{Lingula}, \emph{Lingulellotreta malongensis}: Pseudointerarea
present, following Williams \emph{et al}.
\citeyearpar{Williams2000LinguliformeaCraniiformea}, table 6.

\hypertarget{Lingulosacculus-coding-92}{}
\emph{Lingulosacculus}: Unclear from fossil material.

\hypertarget{Longtancunella_chengjiangensis-coding-92}{}
\emph{Longtancunella chengjiangensis}: Zhang \emph{et al}.
\citeyearpar{Zhang2011Theexceptionally} note that ``all evidence of a
pseudointerarea is lacking'', but the two-dimensional preservation style
of Chengjiang material makes details of dorsal valve difficult to
distinguish, and the possibility of a diminutive pseudointerarea cannot
be excluded with total confidence.

\hypertarget{Mickwitzia_muralensis-coding-92}{}
\emph{Mickwitzia muralensis}: Shell flat.

\hypertarget{Micrina-coding-92}{}
\emph{Micrina}: = Sellate sclerite duplicature
\citep{Holmer2008TheEarly}.

\hypertarget{Micromitra-coding-92}{}
\emph{Micromitra}: ``Dorsal pseudointerarea usually well defined, low,
anacline to catacline'' --
\citet{Williams2000LinguliformeaCraniiformea}.

\hypertarget{Mummpikia_nuda-coding-92}{}
\emph{Mummpikia nuda}: ``Information on the dorsal interarea is
inconclusive {[}\ldots{}{]} no obvious\\
interarea is recognisable; whether or not this is the primary state or a
taphonomic artefact is difficult to assess'' --
\citet{Balthasar2008iMummpikia}, p.~276.

\hypertarget{Nisusia_sulcata-coding-92}{}
\emph{Nisusia sulcata}: Cardinal area (interarea) present -- with
reference to Holmer \emph{et al}.
\citeyearpar{Holmer2018Evolutionarysignificance}.

\hypertarget{Novocrania-coding-92}{}
\emph{Pedunculotheca diania}, \emph{Novocrania}, \emph{Paterimitra}:
Pseudointerarea.

\hypertarget{Pauxillites-coding-92}{}
\emph{Pauxillites}: \citet{Marek1966}.

\hypertarget{Pelagodiscus_atlanticus-coding-92}{}
\emph{Pelagodiscus atlanticus}: Absent, following entry for Discinidae
in Williams \emph{et al}.
\citeyearpar{Williams2000LinguliformeaCraniiformea}, table 6.

\hypertarget{Siphonobolus_priscus-coding-92}{}
\emph{Siphonobolus priscus}: ``Dorsal pseudointerarea weakly anacline,
undivided, elevated above the valve floor'' --
\citet{Popov2009Earlyontogeny}.

\hypertarget{Terebratulina-coding-92}{}
\emph{Terebratulina}: Interarea present.

\hypertarget{Tonicella-coding-92}{}
\emph{Tonicella}: V-shaped notch in anterior valve \citep{Schwabe2010}.

\hypertarget{Ussunia-coding-92}{}
\emph{Ussunia}: Following table 15 in
\citet{Williams2000LinguliformeaCraniiformea}.

\hypertarget{Yuganotheca_elegans-coding-92}{}
\emph{Yuganotheca elegans}: A differentiated region is not obvious in
fossil material or its reconstruction \citep{Zhang2014Anearly}, but the
two-dimensional preservation style of Chengjiang material makes details
of dorsal valve difficult to distinguish, and the possibility of a
diminutive pseudointerarea cannot be excluded with confidence.

\subsection*{{[}93{]} Differentiated posterior surface:
Morphology}\label{differentiated-posterior-surface-morphology}
\addcontentsline{toc}{subsection}{{[}93{]} Differentiated posterior
surface: Morphology}

\includegraphics{Brachiopod_phylogeny_files/figure-latex/character-mapping-93.pdf}

\begin{quote}
\textbf{Character 93: Sclerites: Dorsal valve: Differentiated posterior
surface: Morphology}

0: Curved lateral profile\\
1: Planar lateral profile\\
Neomorphic character.
\end{quote}

It is possible for a cardinal area or pseudointerarea to be distinct
from the anterior part of the shell, yet to remain curved in lateral
profile.

Taking an undifferentiated posterior margin as primitive, the primitive
condition is curved -- flattening of the posterior margin represents an
additional modification that can only occur once the posterior margin is
differentiated.

\hypertarget{Bactrotheca-coding-93}{}
\emph{Bactrotheca}: The short aspect of the cardinal interarea
\citep{Valent2012} makes it difficult to evaluate whether it is planar
or curved.

\hypertarget{Botsfordia-coding-93}{}
\emph{Botsfordia}: ``Curved pseudointerarea'' --
\citet{Skovsted2017Depthrelated}.

\hypertarget{Clupeafumosus_socialis-coding-93}{}
\emph{Clupeafumosus socialis}: Truncated but essentially planar surface;
see e.g.~p196 of \citet{Topper2013Reappraisalof}.

\hypertarget{Conotheca-coding-93}{}
\emph{Conotheca}: Difficult to establish from material figured in
\citet{Devaere2014} due to low elevation of operculum.

\hypertarget{Cupitheca_holocyclata-coding-93}{}
\emph{Cupitheca holocyclata}: Curved \citep{Sun2018}.

\hypertarget{Eoobolus-coding-93}{}
\emph{Eoobolus}: Essentially planar; see Balthasar
\citeyearpar{Balthasar2009Thebrachiopod}, fig. 4a.

\hypertarget{Gasconsia-coding-93}{}
\emph{Pelagodiscus atlanticus}, \emph{Gasconsia}, \emph{Heliomedusa
orienta}, \emph{Mickwitzia muralensis}, \emph{Ussunia}: Posterior
surface cannot be flat if it is not differentiated.

\hypertarget{Maxilites-coding-93}{}
\emph{Maxilites}: Approximately planar \citep{Marek1972}.

\hypertarget{Micromitra-coding-93}{}
\emph{Micromitra}: Essentially straight; see fig. 3.7 in
\citet{Ushatinskaya2016Protegulumand}.

\hypertarget{Paramicrocornus-coding-93}{}
\emph{Paramicrocornus}: \citet{Zhang2018Ahyolithid}.

\hypertarget{Pauxillites-coding-93}{}
\emph{Pauxillites}: Not clear from published material.

\hypertarget{Pedunculotheca_diania-coding-93}{}
\emph{Pedunculotheca diania}: Difficult to evaluate based on present
material, given low nature of valve and compressed preservation.

\hypertarget{Siphonobolus_priscus-coding-93}{}
\emph{Siphonobolus priscus}: The short interarea appears planar (see for
example Popov et a. 2009 fig. 6A), but its short length makes it
difficult to establish whether slight curvature is present.

\hypertarget{Tonicella-coding-93}{}
\emph{Tonicella}: Essentially planar, though open in aspect
\citep[following Chiton in][]{Schwabe2010}.

\subsection*{{[}94{]} Posterior surface: Medial
groove}\label{posterior-surface-medial-groove}
\addcontentsline{toc}{subsection}{{[}94{]} Posterior surface: Medial
groove}

\includegraphics{Brachiopod_phylogeny_files/figure-latex/character-mapping-94.pdf}

\begin{quote}
\textbf{Character 94: Sclerites: Dorsal valve: Posterior surface: Medial
groove}

0: Absent\\
1: Present\\
Neomorphic character.
\end{quote}

Following character 29 in Williams \emph{et al}.
\citeyearpar{Williams2000LinguliformeaCraniiformea}, table 9 (which
relates to pseudointerarea).

\hypertarget{Acanthotretella_spinosa-coding-94}{}
\emph{Acanthotretella spinosa}: The dorsal pseudointerarea is poorly
preserved, but appears to have a median groove
\citep{Holmer2006Aspinose}.

\hypertarget{Botsfordia-coding-94}{}
\emph{Botsfordia}: ``dorsal pseudointerarea vestigial, divided by median
groove'' -- \citet{Williams2000LinguliformeaCraniiformea}.

\hypertarget{Clupeafumosus_socialis-coding-94}{}
\emph{Clupeafumosus socialis}: Present; figured by Topper \emph{et al}.
\citeyearpar{Topper2013Reappraisalof}, fig. 3j.

\hypertarget{Eoobolus-coding-94}{}
\emph{Eoobolus}: Prominent medial groove
\citep{Balthasar2009Thebrachiopod}.

\hypertarget{Heliomedusa_orienta-coding-94}{}
\emph{Heliomedusa orienta}: ``A posteriorly protruding dorsal
pseudointerarea with no median groove and no flexure lines'' --
\citet{Chen2007Reinterpretationof}.

\hypertarget{Lingulellotreta_malongensis-coding-94}{}
\emph{Lingulellotreta malongensis}: Dorsal pseudointerarea with wide,
concave median groove and short propareas" --
\citet{Williams2000LinguliformeaCraniiformea}.

\hypertarget{Maxilites-coding-94}{}
\emph{Maxilites}: \citet{Marek1972}.

\hypertarget{Paramicrocornus-coding-94}{}
\emph{Paramicrocornus}: \citet{Zhang2018Ahyolithid}.

\hypertarget{Siphonobolus_priscus-coding-94}{}
\emph{Siphonobolus priscus}: The dorsal pseudointerarea of \emph{S.
priscus} is undivided \citep{Popov2009Earlyontogeny}, but in other
species it is divided by a ``wide, poorly defined median groove''
\citep{Williams2000LinguliformeaCraniiformea}. Coded, therefore, as
polymorphic.

\subsection*{{[}95{]} Posterior surface:
Notothyrium}\label{posterior-surface-notothyrium}
\addcontentsline{toc}{subsection}{{[}95{]} Posterior surface:
Notothyrium}

\includegraphics{Brachiopod_phylogeny_files/figure-latex/character-mapping-95.pdf}

\begin{quote}
\textbf{Character 95: Sclerites: Dorsal valve: Posterior surface:
Notothyrium}

0: Absent\\
1: Present\\
Neomorphic character.
\end{quote}

A notothyrium is an opening in an interarea that accommodates the
pedicle, and may be filled with plates.

\hypertarget{Botsfordia-coding-95}{}
\emph{Botsfordia}: Following \citet{Williams1998Thediversity}, appendix
2.

\hypertarget{Longtancunella_chengjiangensis-coding-95}{}
\emph{Longtancunella chengjiangensis}: No evidence or report of an
opening at the hinge line in fossil material in
\citet{Zhang2007Agregarious} or \citet{Zhang2011Theexceptionally}.

\hypertarget{Maxilites-coding-95}{}
\emph{Maxilites}: \citet{Marek1972}.

\hypertarget{Paramicrocornus-coding-95}{}
\emph{Paramicrocornus}: \citet{Zhang2018Ahyolithid}.

\hypertarget{Tonicella-coding-95}{}
\emph{Tonicella}: The deep V-shaped notch \citep[fig. 8]{Schwabe2010} is
positionally equivalent to the brachiopod notothyrium.

\subsection*{{[}96{]} Posterior surface: Notothyrium:
Shape}\label{posterior-surface-notothyrium-shape}
\addcontentsline{toc}{subsection}{{[}96{]} Posterior surface:
Notothyrium: Shape}

\includegraphics{Brachiopod_phylogeny_files/figure-latex/character-mapping-96.pdf}

\begin{quote}
\textbf{Character 96: Sclerites: Dorsal valve: Posterior surface:
Notothyrium: Shape}

1: Parallel-sided cleft\\
2: Triangular\\
Transformational character.
\end{quote}

A notothyrium is an opening in an interarea that accommodates the
pedicle, and may be filled with plates.

A simplification of character 5 in
\citet{Bassett2001Functionalmorphology}.

\subsection*{{[}97{]} Posterior surface: Notothyrium: Chilidial
plates}\label{posterior-surface-notothyrium-chilidial-plates}
\addcontentsline{toc}{subsection}{{[}97{]} Posterior surface:
Notothyrium: Chilidial plates}

\includegraphics{Brachiopod_phylogeny_files/figure-latex/character-mapping-97.pdf}

\begin{quote}
\textbf{Character 97: Sclerites: Dorsal valve: Posterior surface:
Notothyrium: Chilidial plates}

1: Open\\
2: Covered by chilidial plates\\
Transformational character.
\end{quote}

A notothyrium may be open or covered by a chilidium or two chilidial
plates.\\
No included taxa exhibit more than one chilidial plate.\\
Transformational as it is not self-evident whether the ancestral taxon
had an open or closed notothyrium.

\subsection*{{[}98{]} Notothyrial platform}\label{notothyrial-platform}
\addcontentsline{toc}{subsection}{{[}98{]} Notothyrial platform}

\includegraphics{Brachiopod_phylogeny_files/figure-latex/character-mapping-98.pdf}

\begin{quote}
\textbf{Character 98: Sclerites: Dorsal valve: Notothyrial platform}

0: Absent\\
1: Present\\
Neomorphic character.
\end{quote}

After Bassett \emph{et al}.
\citeyearpar{Bassett2001Functionalmorphology} character 12.\\
The presence or absence of a notothyrial platform, which often serves as
an attachment point for the diductors in a similar fashion to the
cardinal processes, is independent of the presence of a notothyrium.

\hypertarget{Alisina-coding-98}{}
\emph{Alisina}, \emph{Glyptoria}: Bassett \emph{et al}.
\citeyearpar{Bassett2001Functionalmorphology} score as present in Table
18.1.

\hypertarget{Askepasma_toddense-coding-98}{}
\emph{Askepasma toddense}: Raised notothyrial platform
\citep{Williams1998Thediversity}.

\hypertarget{Coolinia_pecten-coding-98}{}
\emph{Coolinia pecten}: Referred to as the ``posterior platform'' in
Dewing \citeyearpar{Dewing2001Hingemodifications}.

\hypertarget{Kutorgina_chengjiangensis-coding-98}{}
\emph{Kutorgina chengjiangensis}: ``Dorsal diductor scars impressed on
floor of notothyrial cavity'':
\citet{Williams2000LinguliformeaCraniiformea}, regarding Kutorginata.\\
Bassett \emph{et al}. \citeyearpar{Bassett2001Functionalmorphology}
score as absent in Table 18.1.

\hypertarget{Micromitra-coding-98}{}
\emph{Micromitra}: A low notothyrial plate
\citep{Williams1998Thediversity} conceivably correspond to the raised
notothyrial platform of \emph{Askepasma}; coded ambiguous accordingly.

\hypertarget{Nisusia_sulcata-coding-98}{}
\emph{Nisusia sulcata}: Bassett \emph{et al}.
\citeyearpar{Bassett2001Functionalmorphology} score as absent in Table
18.1.\\
``Dorsal diductor scars impressed on floor of notothyrial cavity'':
\citet{Williams2000LinguliformeaCraniiformea}, regarding Kutorginata.

\hypertarget{Ussunia-coding-98}{}
\emph{Ussunia}: ``Visceral platforms absent in both valves'' --
\citet{Williams2000LinguliformeaCraniiformea}, p.~192.

\subsection*{{[}99{]} Medial septum}\label{medial-septum}
\addcontentsline{toc}{subsection}{{[}99{]} Medial septum}

\includegraphics{Brachiopod_phylogeny_files/figure-latex/character-mapping-99.pdf}

\begin{quote}
\textbf{Character 99: Sclerites: Dorsal valve: Medial septum}

0: Absent\\
1: Present\\
Neomorphic character.
\end{quote}

The dorsal valve of many taxa is exhibits a septum or process (or
myophragm) along the medial line. See character 25 in Benedetto
\citeyearpar{Benedetto2009iChaniella}.

\hypertarget{Acanthotretella_spinosa-coding-99}{}
\emph{Acanthotretella spinosa}: Not described by Holmer \& Caron
\citeyearpar{Holmer2006Aspinose}, but an unannotated linear feature
corresponds to the position of a median septum. Without detailed study
of the specimen, we opt to score this as ambiguous.

\hypertarget{Antigonambonites_planus-coding-99}{}
\emph{Antigonambonites planus}: Weakly developed septum evident in
internal cast: \citet{Williams2000LinguliformeaCraniiformea}, fig.
508.2e.

\hypertarget{Botsfordia-coding-99}{}
\emph{Botsfordia}: ``dorsal interior with narrow anterior projection
extending to midvalve, bisected by median ridge'' --
\citet{Williams2000LinguliformeaCraniiformea}.

\hypertarget{Clupeafumosus_socialis-coding-99}{}
\emph{Clupeafumosus socialis}: Prominent process evident
\citep{Topper2013Reappraisalof}.

\hypertarget{Eoobolus-coding-99}{}
\emph{Eoobolus}: A ``median projection'' is present \citep[fig. 4g
in][]{Balthasar2009Thebrachiopod}.

\hypertarget{Glyptoria-coding-99}{}
\emph{Glyptoria}: Neither evident nor reported in Williams \emph{et al}.
\citeyearpar{Williams2000LinguliformeaCraniiformea}.

\hypertarget{Heliomedusa_orienta-coding-99}{}
\emph{Heliomedusa orienta}: Reported on `ventral' valve by Chen \emph{et
al}. \citeyearpar{Chen2007Reinterpretationof}; we consider their
`ventral' valve to be the dorsal valve.

The structure is unambiguously figured \citep[e.g.~fig. 5.1
in][]{Chen2007Reinterpretationof}, contra its coding as absent in
\citet{Williams2000LinguliformeaCraniiformea} and its lack of mention in
\citet{Williams2007Supplement} or \citet{Zhang2009Architectureand}.

\hypertarget{Kutorgina_chengjiangensis-coding-99}{}
\emph{Kutorgina chengjiangensis}: Absent -- fig. 129.1f in Williams
\emph{et al}. \citeyearpar{Williams2000LinguliformeaCraniiformea}.

\hypertarget{Lingulellotreta_malongensis-coding-99}{}
\emph{Lingulellotreta malongensis}: Very weakly developed but seemingly
present between muscle scars in \emph{Lingulellotreta}, more prominent
in Aboriginella (also Lingulellotretidae) \citep[fig.
34]{Williams2000LinguliformeaCraniiformea}.

\hypertarget{Lingulosacculus-coding-99}{}
\emph{Lingulosacculus}: It is not possible to determine, based on the
material presented in Balthasar \& Butterfield
\citeyearpar{Balthasar2009EarlyCambrian}, whether the anterior
projection of the visceral area in the dorsal valve corresponds to a
medial septum in the underlying shell.

\hypertarget{Mummpikia_nuda-coding-99}{}
\emph{Mummpikia nuda}: See pl. 2 panel 6 in Balthasar
\citeyearpar{Balthasar2008iMummpikia}.

\hypertarget{Nisusia_sulcata-coding-99}{}
\emph{Nisusia sulcata}: Fig. 125 in Williams \emph{et al}.
\citeyearpar{Williams2000LinguliformeaCraniiformea}.

\hypertarget{Novocrania-coding-99}{}
\emph{Novocrania}: Median process evident: Williams \emph{et al}.
\citeyearpar{Williams2000LinguliformeaCraniiformea} fig. 100.2a, d.

\hypertarget{Orthis-coding-99}{}
\emph{Orthis}: Short medial process (``low median ridge'', p.~724)
present in dorsal valve; see Fig. 523.3b in Williams \emph{et al}.
\citeyearpar{Williams2000LinguliformeaCraniiformea}.

\hypertarget{Paramicrocornus-coding-99}{}
\emph{Paramicrocornus}: \citet{Zhang2018Ahyolithid}.

\hypertarget{Pauxillites-coding-99}{}
\emph{Pauxillites}: \citet{Marek1966}.

\hypertarget{Siphonobolus_priscus-coding-99}{}
\emph{Siphonobolus priscus}: ``Dorsal interior {[}\ldots{}{]} bisected
by a short median ridge.'' -- \citet{Popov2009Earlyontogeny}.

\hypertarget{Ussunia-coding-99}{}
\emph{Ussunia}: Following char 42 in table 15 in
\citet{Williams2000LinguliformeaCraniiformea}.

\subsection*{{[}100{]} Cardinal shield}\label{cardinal-shield}
\addcontentsline{toc}{subsection}{{[}100{]} Cardinal shield}

\includegraphics{Brachiopod_phylogeny_files/figure-latex/character-mapping-100.pdf}

\begin{quote}
\textbf{Character 100: Sclerites: Dorsal valve: Cardinal shield}

0: Absent\\
1: Present\\
Neomorphic character.
\end{quote}

The hyolithid operculum is divided into a cardinal and conical shield
\citep{Zhang2018Ahyolithid}, separated by furrows corresponding to the
position of the helens. See \citet{Marek1976} (fig. 2) or
\citet{MartiMus2005} (fig. 1) for schematic.

With no obvious sites for muscle attachment, the shields are unlikely to
be homologous to the notothyrial platform.

\hypertarget{Bactrotheca-coding-100}{}
\emph{Bactrotheca}: No differentiation between the cardinal and conical
shields.

\hypertarget{Conotheca-coding-100}{}
\emph{Conotheca}: No differentiation between cardinal shield and conical
shield \citep{Wrona2003, Devaere2014}.

\hypertarget{Cupitheca_holocyclata-coding-100}{}
\emph{Cupitheca holocyclata}: Narrow cardinal shield
\citep{Skovsted2016}.

\hypertarget{Maxilites-coding-100}{}
\emph{Maxilites}: \citet{Marek1972}.

\hypertarget{Paramicrocornus-coding-100}{}
\emph{Paramicrocornus}: \citet{Zhang2018Ahyolithid}.

\subsection*{{[}101{]} Cardinal processes}\label{cardinal-processes}
\addcontentsline{toc}{subsection}{{[}101{]} Cardinal processes}

\includegraphics{Brachiopod_phylogeny_files/figure-latex/character-mapping-101.pdf}

\begin{quote}
\textbf{Character 101: Sclerites: Dorsal valve: Cardinal processes}

0: Absent\\
1: Present\\
Neomorphic character.
\end{quote}

After Bassett \emph{et al}.
\citeyearpar{Bassett2001Functionalmorphology} character 13. See
\citet{MartiMus2005} for an illustration.\\
Cardinal processes are unlikely to be homologous with the notothyrial
platform, even if their function is similar.

\hypertarget{Bactrotheca-coding-101}{}
\emph{Bactrotheca}: ``Narrow broadly diverging cardinal processes with
subparallel edges'' -- \citet{Valent2012}.

\hypertarget{Clupeafumosus_socialis-coding-101}{}
\emph{Clupeafumosus socialis}: Not reported by Topper \emph{et al}.
\citeyearpar{Topper2013Reappraisalof}.

\hypertarget{Conotheca-coding-101}{}
\emph{Conotheca}: Present \citep{Wrona2003}.

\hypertarget{Cupitheca_holocyclata-coding-101}{}
\emph{Cupitheca holocyclata}: Well-defined \citep{Skovsted2016}.

\hypertarget{Longtancunella_chengjiangensis-coding-101}{}
\emph{Longtancunella chengjiangensis}: Not evident, and ought arguably
to be discernable if present given the quality of preservation.

\hypertarget{Maxilites-coding-101}{}
\emph{Maxilites}: Narrow cardinal processes \citep{Marek1972}.

\hypertarget{Paramicrocornus-coding-101}{}
\emph{Paramicrocornus}: Present \citep{Zhang2018Ahyolithid}.

\hypertarget{Pauxillites-coding-101}{}
\emph{Pauxillites}: \citet{Marek1966}.

\subsection*{{[}102{]} Cardinal teeth}\label{cardinal-teeth}
\addcontentsline{toc}{subsection}{{[}102{]} Cardinal teeth}

\includegraphics{Brachiopod_phylogeny_files/figure-latex/character-mapping-102.pdf}

\begin{quote}
\textbf{Character 102: Sclerites: Dorsal valve: Cardinal teeth}

0: Absent\\
1: Present\\
Neomorphic character.
\end{quote}

Radially arranged teeth, separated by furrows, adorn the cardinal margin
of the operculum of certain hyolithids \citep{Marek1963}. The absence of
corresponding tooth sockets indicates that they do not serve to
articulate the valves; \citet{Marek1967} does not consider the teeth to
be homologous with brachiopod cardinal teeth.

\hypertarget{Maxilites-coding-102}{}
\emph{Maxilites}: Dorsal teeth figured in \citet{MartiMus2005}.

\hypertarget{Pauxillites-coding-102}{}
\emph{Pauxillites}: Present \citep{Marek1966}.

\subsection*{{[}103{]} Clavicles}\label{clavicles}
\addcontentsline{toc}{subsection}{{[}103{]} Clavicles}

\includegraphics{Brachiopod_phylogeny_files/figure-latex/character-mapping-103.pdf}

\begin{quote}
\textbf{Character 103: Sclerites: Dorsal valve: Clavicles}

0: Absent\\
1: Present\\
Neomorphic character.
\end{quote}

Prominent symmetrical ridges on the inner surface of the hyolith
operculum.

\hypertarget{Conotheca-coding-103}{}
\emph{Conotheca}: ``Processes and clavicle-like structures are absent''
-- \citet{Devaere2014}.

\hypertarget{Cupitheca_holocyclata-coding-103}{}
\emph{Cupitheca holocyclata}: ``No traces of clavicles''
\citep{Skovsted2016}.

\hypertarget{Maxilites-coding-103}{}
\emph{Maxilites}: Single pair \citep{Marek1972}.

\hypertarget{Paramicrocornus-coding-103}{}
\emph{Paramicrocornus}: \citet{Zhang2018Ahyolithid}.

\hypertarget{Pauxillites-coding-103}{}
\emph{Pauxillites}: \citet{Marek1966}.

\subsection*{{[}104{]} Clavicles: Type of
clavicles}\label{clavicles-type-of-clavicles}
\addcontentsline{toc}{subsection}{{[}104{]} Clavicles: Type of
clavicles}

\includegraphics{Brachiopod_phylogeny_files/figure-latex/character-mapping-104.pdf}

\begin{quote}
\textbf{Character 104: Sclerites: Dorsal valve: Clavicles: Type of
clavicles}

1: Monoclavicle\\
2: Triclavicle\\
Transformational character.
\end{quote}

Usually the operculum of hyoliths has one pair of clavicles, but in some
taxa of hyolithida there are more than one pair of clavicles, which can
be divided into six types \citep{Marek1967}. The included taxa either
exhibit a single pair of monoclavicles, or three pairs of clavicles.

\hypertarget{Paramicrocornus-coding-104}{}
\emph{Paramicrocornus}: \citet{Zhang2018Ahyolithid}.

\hypertarget{Pauxillites-coding-104}{}
\emph{Pauxillites}: Three pairs of clavicles \citep{Marek1966}.

\section{Sclerites: Ventral valve}\label{sclerites-ventral-valve}

\subsection*{{[}105{]} Growth direction}\label{growth-direction-1}
\addcontentsline{toc}{subsection}{{[}105{]} Growth direction}

\includegraphics{Brachiopod_phylogeny_files/figure-latex/character-mapping-105.pdf}

\begin{quote}
\textbf{Character 105: Sclerites: Ventral valve: Growth direction}

1: Holoperipheral\\
2: Mixoperipheral\\
3: Hemiperipheral\\
Transformational character.
\end{quote}

See Fig. 284 in Williams \emph{et al}.
\citeyearpar{Williams1997Introduction} for depiction of terms.

The growth direction dictates the attitude of the cardinal area relative
to the hinge, which does not therefore represent an independent
character.\\
Crudely put, if, viewed from a dorsal position, the umbo falls within
the outer margin of the shell, growth is holoperipheral; if it falls
outside the margin, it is mixoperipheral; if it falls exactly on the
margin, it is hemiperipheral.

\hypertarget{Clupeafumosus_socialis-coding-105}{}
\emph{Clupeafumosus socialis}: Inferred from Topper \emph{et al}.
\citeyearpar{Topper2013Reappraisalof}.

\hypertarget{Craniops-coding-105}{}
\emph{Craniops}: ``Both valves with growth holoperipheral'' --
\citet{Williams2000LinguliformeaCraniiformea}, p.~164.

\hypertarget{Heliomedusa_orienta-coding-105}{}
\emph{Heliomedusa orienta}: Williams \emph{et al}.
\citeyearpar[2007]{Williams2000LinguliformeaCraniiformea} reconstruct
mixoperipheral growth in the ventral valve {[}though Chen \emph{et al}.
\citeyearpar{Chen2007Reinterpretationof} reconstruct the valves the
other way round, i.e.~it is the ventral valve that grows
holoperipherally, and the dorsal mixoperipherally{]}.

\hypertarget{Paterimitra-coding-105}{}
\emph{Paterimitra}: The apical flange notwithstanding, the umbo of the
S1 sclerite is posterior of the hinge line and the posterior edge of the
lateral plate -- see \citet{Larsson2014iPaterimitra}, fig. 2a, c.

\hypertarget{Siphonobolus_priscus-coding-105}{}
\emph{Siphonobolus priscus}: Initially holoperipheral
\citep[p.~159]{Popov2009Earlyontogeny}, then on the brink of being
mixoperipheral in adulthood, so coded as polymorphic.

\hypertarget{Tonicella-coding-105}{}
\emph{Tonicella}: Growth is hemiperipheral in the anterior valve of
polyplacophorans and holoperipheral in the posterior valves
\citep{Schwabe2010, Connors2012}.

\hypertarget{Ussunia-coding-105}{}
\emph{Ussunia}: Growth ``mixoperipheral in both valves'' in trimerellids
\citep{Williams2000LinguliformeaCraniiformea, Popov1997}.

\subsection*{{[}106{]} Relative size}\label{relative-size}
\addcontentsline{toc}{subsection}{{[}106{]} Relative size}

\includegraphics{Brachiopod_phylogeny_files/figure-latex/character-mapping-106.pdf}

\begin{quote}
\textbf{Character 106: Sclerites: Ventral valve: Relative size}

1: Ventral valve markedly larger than dorsal valve (ventribiconvex)\\
2: Equivalve (subequally biconvex)\\
3: Dorsal valve markedly larger than ventral valve (dorsibiconvex)\\
Transformational character.
\end{quote}

In many brachiopods, the valves are closely similar in size; in others,
the ventral valve is markedly larger than the dorsal, on account of
being more convex. Marginal cases are treated as ambiguous for the
relevant states.

\hypertarget{Antigonambonites_planus-coding-106}{}
\emph{Antigonambonites planus}: Broadly equivalve -- see Williams
\emph{et al}. \citeyearpar{Williams2000LinguliformeaCraniiformea} fig.
508.2c.

\hypertarget{Botsfordia-coding-106}{}
\emph{Botsfordia}: After table 8 in Williams \emph{et al}.
\citeyearpar{Williams2000LinguliformeaCraniiformea}.

\hypertarget{Craniops-coding-106}{}
\emph{Craniops}: ``Shell subequally biconvex'' --
\citet{Williams2000LinguliformeaCraniiformea}.

\hypertarget{Eoobolus-coding-106}{}
\emph{Eoobolus}: ``\emph{Eoobolus} is biconvex'', but in his amended
diagnosis, Balthasar \citeyearpar{Balthasar2009Thebrachiopod} described
it as ``shell inequivalved, dorsibiconvex''.

\hypertarget{Gasconsia-coding-106}{}
\emph{Gasconsia}: Equivalve as juveniles, becoming ``convexiplane''
\citep[p.~187]{Williams2000LinguliformeaCraniiformea} as adults
\citep{Hanken1985Thetaxonomy}.

\hypertarget{Heliomedusa_orienta-coding-106}{}
\emph{Heliomedusa orienta}: Ventral valve larger than the dorsal valve
\citep[p.~659]{Zhang2009Architectureand}.

\hypertarget{Kutorgina_chengjiangensis-coding-106}{}
\emph{Kutorgina chengjiangensis}: Ventral valve larger \citep[see][fig.
125.]{Williams2000LinguliformeaCraniiformea}.

\hypertarget{Longtancunella_chengjiangensis-coding-106}{}
\emph{Longtancunella chengjiangensis}, \emph{Yuganotheca elegans}: The
ventral valve is somewhat, but not markedly, larger than the dorsal; as
such, this character is coded ambiguous for equivalve/ventral valve
larger.

\hypertarget{Mummpikia_nuda-coding-106}{}
\emph{Mummpikia nuda}: Aside from hinge, valves similar in convexity and
size \citep{Balthasar2008iMummpikia}.

\hypertarget{Nisusia_sulcata-coding-106}{}
\emph{Nisusia sulcata}: Ventral valve larger \citep[see][fig.
126.]{Williams2000LinguliformeaCraniiformea}.

\hypertarget{Siphonobolus_priscus-coding-106}{}
\emph{Siphonobolus priscus}: Ventribiconvex
\citep{Popov2009Earlyontogeny}.

\hypertarget{Tonicella-coding-106}{}
\emph{Tonicella}: Coded as ambiguous for equivalve/ventral valve larger:
the posterior embryonic shell field, treated herein as equivalent to the
ventral valve,.

\hypertarget{Ussunia-coding-106}{}
\emph{Ussunia}: Subequally biconvex
\citep[p.~192]{Williams2000LinguliformeaCraniiformea}.

\subsection*{{[}107{]} Posterior surface:
Differentiated}\label{posterior-surface-differentiated-1}
\addcontentsline{toc}{subsection}{{[}107{]} Posterior surface:
Differentiated}

\includegraphics{Brachiopod_phylogeny_files/figure-latex/character-mapping-107.pdf}

\begin{quote}
\textbf{Character 107: Sclerites: Ventral valve: Posterior surface:
Differentiated}

0: Posterior surface of shell not differentiated\\
1: Posterior surface of shell forms distinct cardinal area or
pseudointerarea\\
Neomorphic character.
\end{quote}

In shells that grow by mixoperipheral growth, the triangular area
subtended between each apex and the posterior ends of the lateral
margins is termed the cardinal area. In shells with holoperipheral
growth, a flattened surface on the posterior margin of the valve is
termed a pseudointerarea
\citep[paraphrasing][]{Williams1997Introduction}.

In order for this character to be independent of a shell's growth
direction, we do not distinguish between a ``cardinal area'',
``interarea'' or ``pseudointerarea''.

\hypertarget{Alisina-coding-107}{}
\emph{Alisina}, \emph{Antigonambonites planus}, \emph{Coolinia pecten},
\emph{Glyptoria}, \emph{Kutorgina chengjiangensis}, \emph{Nisusia
sulcata}, \emph{Orthis}, \emph{Salanygolina}, \emph{Tomteluva
perturbata}: Interarea present.

\hypertarget{Bactrotheca-coding-107}{}
\emph{Bactrotheca}: No clear delineation of posterior (functionally
dorsal) surface.

\hypertarget{Clupeafumosus_socialis-coding-107}{}
\emph{Clupeafumosus socialis}: Described by Topper \emph{et al}.
\citeyearpar{Topper2013Reappraisalof}.

\hypertarget{Conotheca-coding-107}{}
\emph{Conotheca}: Round in cross-section, with no clear delineation of
posterior (functionally dorsal) surface \citep{Devaere2014}.

\hypertarget{Cupitheca_holocyclata-coding-107}{}
\emph{Cupitheca holocyclata}: No evidence of differentiation; circular
cross-section \citep{Vendrasco2017}.

\hypertarget{Gasconsia-coding-107}{}
\emph{Gasconsia}: The region corresponding to the ventral
(pseudo)interarea is described as a ``trimerellid ventral cardinal
area'' by Williams \emph{et al}.
\citeyearpar[p.162]{Williams2000LinguliformeaCraniiformea}, who code
both an interarea and a pseudointerarea as absent in trimerellids.

\hypertarget{Heliomedusa_orienta-coding-107}{}
\emph{Heliomedusa orienta}: Zhang \emph{et al}.
\citeyearpar{Zhang2009Architectureand} report a moderate to somewhat
developed ventral pseudointerarea, confirmed by Williams \emph{et al}.
\citeyearpar{Williams2007Supplement}.

\hypertarget{Lingulosacculus-coding-107}{}
\emph{Lingulosacculus}: The conical valve is interpreted as the ventral
valve with an extended pseudointerarea.

\hypertarget{Longtancunella_chengjiangensis-coding-107}{}
\emph{Longtancunella chengjiangensis}: Though ``all evidence of a
pseudointerarea is lacking'' -- \citet{Zhang2011Theexceptionally} -- the
region of the shell between the strophic hinge line and the colleplax
\citep[fig. 2 in][]{Zhang2011Theexceptionally} is distinct from the rest
of the shell; the ends of the strophic hinge line are marked by
prominent nicks in the shell margin. \emph{Longtancunella} is therefore
coded as having a differentiated posterior surface.

\hypertarget{Mickwitzia_muralensis-coding-107}{}
\emph{Mickwitzia muralensis}: Termed an interarea by Balthasar
\citeyearpar{Balthasar2004Shellstructure}.

\hypertarget{Mummpikia_nuda-coding-107}{}
\emph{Mummpikia nuda}: Balthasar \citeyearpar{Balthasar2008iMummpikia}
interprets a pseudointerarea as being present -- e.g.~p273, ``Of
particular interest is the vault that bridges the most anterior portion
of the ventral pseudointerarea and raises it above the visceral
platform.''; ``This pattern is reversed in the ventral valves of
\emph{M. nuda}, where the anterior projection of the pedicle groove is
raised above the valve floor whereas the lateral parts of
pseudointerarea are not''.

\hypertarget{Paterimitra-coding-107}{}
\emph{Paterimitra}: Triangular notch and subapical flange.

\hypertarget{Pedunculotheca_diania-coding-107}{}
\emph{Pedunculotheca diania}: Lateral lines suggest differentiation of
posterior surface, but difficult to discern a change in morphology of
this region. Coded ambiguous.

\hypertarget{Siphonobolus_priscus-coding-107}{}
\emph{Siphonobolus priscus}: ``Ventral pseudointerarea, low, undivided,
poorly defined'' -- \citet{Williams2000LinguliformeaCraniiformea}.

\hypertarget{Terebratulina-coding-107}{}
\emph{Terebratulina}: Interarea.

\hypertarget{Tonicella-coding-107}{}
\emph{Tonicella}: Following the proposed homology model between the
posterior valve of polyplacophorans and the ventral valve of
brachiopods, the ``posterior'' surface of the polyplacophoran valve is
taken to be the surface that would articulate with the anterior valve,
which is anatomically anterior on the living organism.

\hypertarget{Ussunia-coding-107}{}
\emph{Ussunia}: Following char 17 in table 15 of
\citet{Williams2000LinguliformeaCraniiformea}.

\subsection*{{[}108{]} Posterior margin growth
direction}\label{posterior-margin-growth-direction}
\addcontentsline{toc}{subsection}{{[}108{]} Posterior margin growth
direction}

\includegraphics{Brachiopod_phylogeny_files/figure-latex/character-mapping-108.pdf}

\begin{quote}
\textbf{Character 108: Sclerites: Ventral valve: Posterior margin growth
direction}

1: Inward-growing\\
2: Outward-growing\\
Transformational character.
\end{quote}

Balthasar \citeyearpar{Balthasar2008iMummpikia} notes an inward-growing
posterior margin of the pseudointerarea as potentially linking
\emph{Mummpikia} with the linguliform brachiopods.

Coded as inapplicable in taxa without a differentiated posterior margin:
the posterior margin can only grow inwards if it is differentiated from
the anterior margin; else the entire shell would grow in on itself.

\hypertarget{Botsfordia-coding-108}{}
\emph{Botsfordia}: Inward-growing; see Skovsted \& Holmer
\citeyearpar{Skovsted2005EarlyCambrian}, pl. 4.

\hypertarget{Clupeafumosus_socialis-coding-108}{}
\emph{Clupeafumosus socialis}: See Topper \emph{et al}.
\citeyearpar{Topper2013Reappraisalof}.

\hypertarget{Eoobolus-coding-108}{}
\emph{Eoobolus}: See for example Skovsted \& Holmer
\citeyearpar{Skovsted2005EarlyCambrian}, pl. 3.

\hypertarget{Lingulellotreta_malongensis-coding-108}{}
\emph{Lingulellotreta malongensis}: Transverse cross section of ventral
pseudointerarea concave.

\hypertarget{Mummpikia_nuda-coding-108}{}
\emph{Mummpikia nuda}: Balthasar \citeyearpar{Balthasar2008iMummpikia}
interprets an inward-growing posterior margin of the pseudointerarea --
e.g.~p273, ``Of particular interest is the vault that bridges the most
anterior portion of the ventral pseudointerarea and raises it above the
visceral platform {[}\ldots{}{]} An inward-growing posterior margin is
otherwise known only from the pseudointerareas of linguliform
brachiopods''.

\subsection*{{[}109{]} Posterior surface:
Planar}\label{posterior-surface-planar}
\addcontentsline{toc}{subsection}{{[}109{]} Posterior surface: Planar}

\includegraphics{Brachiopod_phylogeny_files/figure-latex/character-mapping-109.pdf}

\begin{quote}
\textbf{Character 109: Sclerites: Ventral valve: Posterior surface:
Planar}

0: Curved lateral profile\\
1: Planar lateral profile\\
Neomorphic character.
\end{quote}

It is possible for a cardinal area or pseudointerarea to be distinct
from the anterior part of the shell, yet to remain curved in lateral
profile.

Taking an undifferentiated posterior margin as primitive, the primitive
condition is curved -- flattening of the posterior margin represents an
additional modification that can only occur once the posterior margin is
differentiated.

A flat and triangular interarea links \emph{Mummpikia} with the
Obolellidae \citep{Balthasar2008iMummpikia} -- but all included taxa
have triangular interareas, so this is not listed as a separate
character.

\hypertarget{Acanthotretella_spinosa-coding-109}{}
\emph{Acanthotretella spinosa}: ventral pseudointerareas are most
similar to those found within the Order Siphonotretida.

\hypertarget{Botsfordia-coding-109}{}
\emph{Botsfordia}: See Skovsted \& Holmer
\citeyearpar{Skovsted2005EarlyCambrian}, pl. 3, fig. 14.

\hypertarget{Clupeafumosus_socialis-coding-109}{}
\emph{Clupeafumosus socialis}: ``Ventral pseudointerarea is gently
procline and is flat in lateral profile''. ---\\
\citep{Topper2013Reappraisalof}.

\hypertarget{Conotheca-coding-109}{}
\emph{Conotheca}: Strongly curved in some species \citep{Wrona2003}.

\hypertarget{Eoobolus-coding-109}{}
\emph{Eoobolus}: Some curvature retained.

\hypertarget{Haplophrentis_carinatus-coding-109}{}
\emph{Haplophrentis carinatus}: Dorsal surface essentially linear
\citep[fig ed6a, ed7a]{Moysiuk2017Hyolithsare}.

\hypertarget{Lingulellotreta_malongensis-coding-109}{}
\emph{Lingulellotreta malongensis}: Transverse cross section of ventral
pseudointerarea concave.

\hypertarget{Longtancunella_chengjiangensis-coding-109}{}
\emph{Longtancunella chengjiangensis}: Flattened, reflecting the
strophic hinge line.

\hypertarget{Maxilites-coding-109}{}
\emph{Maxilites}: Gently sinuous \citep{MartiMus2005}.

\hypertarget{Micromitra-coding-109}{}
\emph{Micromitra}: Essentially planar; see fig. 6 in
\citet{Ushatinskaya2016Protegulumand}.

\hypertarget{Paramicrocornus-coding-109}{}
\emph{Paramicrocornus}: Posterior (functionally dorsal) surface is
linear in adults (though slightly curving in juveniles)
\citep{Zhang2018Ahyolithid}.

\hypertarget{Pauxillites-coding-109}{}
\emph{Pauxillites}: Reconstructed as linear by \citet{Marek1976}.

\hypertarget{Pedunculotheca_diania-coding-109}{}
\emph{Pedunculotheca diania}: Essentially linear.

\hypertarget{Siphonobolus_priscus-coding-109}{}
\emph{Siphonobolus priscus}: `Almost' planar -- see Popov \emph{et al}.
\citeyearpar[fig. 4]{Popov2009Earlyontogeny}. Coded as ambiguous.

\hypertarget{Tonicella-coding-109}{}
\emph{Tonicella}: \citep{Schwabe2010}.

\subsection*{{[}110{]} Ligula}\label{ligula}
\addcontentsline{toc}{subsection}{{[}110{]} Ligula}

\includegraphics{Brachiopod_phylogeny_files/figure-latex/character-mapping-110.pdf}

\begin{quote}
\textbf{Character 110: Sclerites: Ventral valve: Ligula}

0: Absent\\
1: Present\\
Neomorphic character.
\end{quote}

The aperture of many hyolithid hyoliths is characterised by a ligula, a
tongue-like protruding shelf on the functionally ventral surface of
conical shell \citep{MartiMus2005}. This can be recognized by an acute
angle in the lateral profile of the commissure \citep[see second figure
on p.~91 of][]{Marek1966}. No brachiopods display an equivalent feature.

\hypertarget{Paramicrocornus-coding-110}{}
\emph{Paramicrocornus}: ``Very short'' semicircular ligula
\citep{Zhang2018Ahyolithid}.

\subsection*{{[}111{]} Posterior surface:
Extent}\label{posterior-surface-extent}
\addcontentsline{toc}{subsection}{{[}111{]} Posterior surface: Extent}

\includegraphics{Brachiopod_phylogeny_files/figure-latex/character-mapping-111.pdf}

\begin{quote}
\textbf{Character 111: Sclerites: Ventral valve: Posterior surface:
Extent}

1: Low: Wider than deep\\
2: High: Deeper than wide\\
Transformational character.
\end{quote}

Distinguishes taxa whose ventral valve is essentially flat from those
that are essentially conical.

\hypertarget{Antigonambonites_planus-coding-111}{}
\emph{Antigonambonites planus}: Though scored High in data matrix of
Benedetto \citeyearpar{Benedetto2009iChaniella}, this taxon
\citep[see][fig. 508]{Williams2000LinguliformeaCraniiformea} does not
express the deeply conical ventral valve that this character is intended
to reflect. It is charitably coded as ambiguous.

\hypertarget{Clupeafumosus_socialis-coding-111}{}
\emph{Clupeafumosus socialis}: Entire valve length -- see schematic in
Williams \emph{et al}. \citeyearpar{Williams1997Introduction}, fig. 286.

\hypertarget{Coolinia_pecten-coding-111}{}
\emph{Coolinia pecten}: See fig. 485 in
\citet{Williams2000LinguliformeaCraniiformea}.

\hypertarget{Gasconsia-coding-111}{}
\emph{Gasconsia}: ``ventral cardinal interarea low, apsacline, with
narrow, poorly defined homeodeltidium'' --
\citet{Williams2000LinguliformeaCraniiformea}, p.~186.

\hypertarget{Kutorgina_chengjiangensis-coding-111}{}
\emph{Kutorgina chengjiangensis}: This taxon
\citetext{\citealp[see][fig.
129]{Williams2000LinguliformeaCraniiformea}; \citealp[fig.
1]{Popov1992TheCambrian}} comes close to expressing the deeply conical
ventral valve that this character is intended to reflect, though this is
not always so pronounced \citep[e.g.][fig.
125]{Williams2000LinguliformeaCraniiformea}. It is therefore coded as
ambiguous.

\hypertarget{Mickwitzia_muralensis-coding-111}{}
\emph{Mickwitzia muralensis}: Often not prominently high
\citep{Skovsted2003EarlyCambrian, Balthasar2004Shellstructure}, though
in some cases \citep[e.g.][]{Butler2015Exceptionallypreserved} the
ventral valve approaches the conical shape that this character is
intended to capture. Coded as polymorphic.

\hypertarget{Nisusia_sulcata-coding-111}{}
\emph{Nisusia sulcata}: Scored as high in data matrix of Benedetto
\citeyearpar{Benedetto2009iChaniella}, and depicted as such in Williams
\emph{et al}. \citeyearpar[fig.
125]{Williams2000LinguliformeaCraniiformea} and Popov \citeyearpar[fig.
1]{Popov1992TheCambrian}; but not high in all specimens
\citep[e.g.][fig. 126]{Williams2000LinguliformeaCraniiformea}. It is
therefore coded as polymorphic.

\hypertarget{Novocrania-coding-111}{}
\emph{Novocrania}: Low cone.

\hypertarget{Orthis-coding-111}{}
\emph{Orthis}: Scored `Low' for \emph{Eoorthis} by Benedetto
\citeyearpar{Benedetto2009iChaniella}; assumed same in \emph{Orthis}.

\hypertarget{Salanygolina-coding-111}{}
\emph{Salanygolina}: Whereas Williams \emph{et al}.
\citeyearpar[p.~156]{Williams2000LinguliformeaCraniiformea} describe the
ventral pseudointerarea as high, the shell lacks the deeply conical
aspect that this character is intended to capture; we thus code the
taxon as ambiguous.

\subsection*{{[}112{]} Posterior surface:
Delthyrium}\label{posterior-surface-delthyrium}
\addcontentsline{toc}{subsection}{{[}112{]} Posterior surface:
Delthyrium}

\includegraphics{Brachiopod_phylogeny_files/figure-latex/character-mapping-112.pdf}

\begin{quote}
\textbf{Character 112: Sclerites: Ventral valve: Posterior surface:
Delthyrium}

0: Absent\\
1: Present\\
Neomorphic character.
\end{quote}

A delthyrium is an opening in an interarea or pseudointerarea that
accommodates the pedicle, and may be filled with plates.

The homology of the pedicle in the pseudointerarea of obolellids and
botsfordiids with the umbonal pedicle foramen of acrotretids was
proposed by Popov \citeyearpar{Popov1992TheCambrian}, and seemingly
corroborated by observations of Ushatinskaya \& Korovnikov
\citeyearpar{Ushatinskaya2016Revisionof}, who note that the propareas of
the \emph{Botsfordia} ventral valve sometimes merge to form an elongate
teardrop-shaped pedicle foramen.

\hypertarget{Acanthotretella_spinosa-coding-112}{}
\emph{Acanthotretella spinosa}: Origin modelled on \emph{Siphonobolus}.

\hypertarget{Askepasma_toddense-coding-112}{}
\emph{Askepasma toddense}: Homeodeltidium absent
\citep[p.~153]{Williams2000LinguliformeaCraniiformea}; deltidium is open
\citep[see][fig. 4]{Topper2013Theoldest}.

\hypertarget{Botsfordia-coding-112}{}
\emph{Botsfordia}: The homology of the triangular notch or groove in the
pseudointerarea with the umbonal pedicle foramen of acrotretids was
proposed by Popov \citeyearpar{Popov1992TheCambrian}, and seemingly
corroborated by observations of Ushatinskaya \& Korovnikov
\citeyearpar{Ushatinskaya2016Revisionof}, who note that the propareas of
the \emph{Botsfordia} ventral valve sometimes merge to form an elongate
teardrop-shaped pedicle foramen.

\hypertarget{Clupeafumosus_socialis-coding-112}{}
\emph{Clupeafumosus socialis}: Following Popov
\citeyearpar{Popov1992TheCambrian}, the larval delthyrium is sealed in
adults by outgrowths of the posterolateral margins of the shell.

\hypertarget{Eoobolus-coding-112}{}
\emph{Eoobolus}: See for example fig. 5 in
\citet{Balthasar2009Thebrachiopod}.

\hypertarget{Glyptoria-coding-112}{}
\emph{Glyptoria}: ``Delthyrium and notothyrium open, wide'' --
\citet{Cooper1976LowerCambrian}.

\hypertarget{Longtancunella_chengjiangensis-coding-112}{}
\emph{Longtancunella chengjiangensis}: Unclear: a narrow ridge that may
correspond to a pseudodeltidium evident in fig 2a and sketched in fig.
2c is not discussed in the text of \citet{Zhang2011Theexceptionally}, so
the delthyrial region is coded as ambiguous.

\hypertarget{Mickwitzia_muralensis-coding-112}{}
\emph{Mickwitzia muralensis}: A delthyrium is present in young
individuals \citep{Balthasar2004Shellstructure}.

\hypertarget{Micrina-coding-112}{}
\emph{Micrina}: Opening inferred by Holmer \emph{et al}.
\citeyearpar{Holmer2008TheEarly}.

\hypertarget{Pelagodiscus_atlanticus-coding-112}{}
\emph{Pelagodiscus atlanticus}: The listrum (pedicle opening) is
interpreted as originating via a similar mechanism to that of
acrotretids \citep{Popov1992TheCambrian}, and hence corresponding to a
basally sealed delthyrium.

\hypertarget{Siphonobolus_priscus-coding-112}{}
\emph{Siphonobolus priscus}: Ontogeny presumed to resemble that of
acrotretids.

\hypertarget{Tonicella-coding-112}{}
\emph{Tonicella}: The antemucronal area \citep{Schwabe2010} is treated
as equivalent to the brachiopod delthyrium.

\hypertarget{Yuganotheca_elegans-coding-112}{}
\emph{Yuganotheca elegans}: Details of the hinge region are unclear due
to the flattened and overprinted nature of fossil preservation.

\subsection*{{[}113{]} Posterior surface: Delthyrium:
Shape}\label{posterior-surface-delthyrium-shape}
\addcontentsline{toc}{subsection}{{[}113{]} Posterior surface:
Delthyrium: Shape}

\includegraphics{Brachiopod_phylogeny_files/figure-latex/character-mapping-113.pdf}

\begin{quote}
\textbf{Character 113: Sclerites: Ventral valve: Posterior surface:
Delthyrium: Shape}

1: Parallel sided\\
2: Triangular\\
3: Round\\
Transformational character.
\end{quote}

A parallel-sided delthyrium links \emph{Mummpikia} with the Obolellidae
\citep{Balthasar2008iMummpikia}.

Following Popov \citeyearpar{Popov1992TheCambrian}, the larval
delthyrium of acrotretids and allied taxa is understood to be sealed in
adults by outgrowths of the posterolateral margins of the shell. The
resultant round or teardrop-shaped foramen corresponds the delthyrium.

\hypertarget{Askepasma_toddense-coding-113}{}
\emph{Askepasma toddense}: Prominently triangular \citep[see][fig.
2]{Topper2013Theoldest}.

\hypertarget{Clupeafumosus_socialis-coding-113}{}
\emph{Clupeafumosus socialis}: Following the model of Popov
\citeyearpar{Popov1992TheCambrian}.

\hypertarget{Mickwitzia_muralensis-coding-113}{}
\emph{Mickwitzia muralensis}: An opening is incorporated at the base of
the homeodeltidium when the organism switches from early to late
maturity \citep[fig. 10 in][]{Balthasar2004Shellstructure}. This opening
is conceivably homologous with the pedicle foramen of acrotretid
brachiopods and their ilk. To reflect this possible homology,
\emph{Mickwitzia} is coded as polymorphic (triangular/round).

\subsection*{{[}114{]} Posterior surface: Delthyrium: Shape: Aspect of
rounded
opening}\label{posterior-surface-delthyrium-shape-aspect-of-rounded-opening}
\addcontentsline{toc}{subsection}{{[}114{]} Posterior surface:
Delthyrium: Shape: Aspect of rounded opening}

\includegraphics{Brachiopod_phylogeny_files/figure-latex/character-mapping-114.pdf}

\begin{quote}
\textbf{Character 114: Sclerites: Ventral valve: Posterior surface:
Delthyrium: Shape: Aspect of rounded opening}

1: Elongate: oval to rhombic\\
2: Essentially circular\\
3: Wider than long\\
Transformational character.
\end{quote}

Chen \emph{et al}. \citeyearpar{Chen2007Reinterpretationof} propose that
an oval to rhombic foramen characterises the discinids {[}and
\emph{Heliomedusa}, though the foramen in this taxon has since been
reinterpreted by Zhang \emph{et al}.
\citeyearpar{Zhang2009Architectureand} as an impression of internal
tissue{]}.

\hypertarget{Lingulellotreta_malongensis-coding-114}{}
\emph{Lingulellotreta malongensis}: Oval
\citep{Williams2000LinguliformeaCraniiformea}.

\hypertarget{Mickwitzia_muralensis-coding-114}{}
\emph{Mickwitzia muralensis}: Wider than long: see fig. 10 in
\citet{Balthasar2004Shellstructure}.

\subsection*{{[}115{]} Posterior surface: Delthyrium:
Cover}\label{posterior-surface-delthyrium-cover}
\addcontentsline{toc}{subsection}{{[}115{]} Posterior surface:
Delthyrium: Cover}

\includegraphics{Brachiopod_phylogeny_files/figure-latex/character-mapping-115.pdf}

\begin{quote}
\textbf{Character 115: Sclerites: Ventral valve: Posterior surface:
Delthyrium: Cover}

1: Open\\
2: Covered, at least in part\\
Transformational character.
\end{quote}

An open delthyrium links \emph{Mummpikia} with the Obolellidae
\citep{Balthasar2008iMummpikia}.

The delthyrial opening can be covered by one or more deltidial plates,
or a pseudodeltitium.

Inapplicable in taxa with a round delthiruym \citep[generated by
overgrowth of the delthyrial opening by posterolateral parts of the
shell, per][]{Popov1992TheCambrian}.

\hypertarget{Askepasma_toddense-coding-115}{}
\emph{Askepasma toddense}: Open \citep{Topper2013Theoldest}.

\hypertarget{Botsfordia-coding-115}{}
\emph{Botsfordia}: See pl. 3 fig. 15 in Skovsted \& Holmer
\citeyearpar{Skovsted2005EarlyCambrian}.

\hypertarget{Coolinia_pecten-coding-115}{}
\emph{Coolinia pecten}: A convex pseudodeltidium completely covers the
delthyrium in \emph{Coolinia}.

\hypertarget{Glyptoria-coding-115}{}
\emph{Glyptoria}: Coded as open by Williams \emph{et al}.
\citeyearpar{Williams1998Thediversity}.

\hypertarget{Micromitra-coding-115}{}
\emph{Micromitra}: \citet{Williams2000LinguliformeaCraniiformea}, fig.
83.3.

\hypertarget{Nisusia_sulcata-coding-115}{}
\emph{Nisusia sulcata}: ``Covered only apically by a small convex
pseudodeltitium'' -- \citet{Holmer2018Evolutionarysignificance}.

\hypertarget{Paterimitra-coding-115}{}
\emph{Paterimitra}: Covered by subaical flange, in part.

\subsection*{{[}116{]} Posterior surface: Delthyrium: Cover:
Extent}\label{posterior-surface-delthyrium-cover-extent}
\addcontentsline{toc}{subsection}{{[}116{]} Posterior surface:
Delthyrium: Cover: Extent}

\includegraphics{Brachiopod_phylogeny_files/figure-latex/character-mapping-116.pdf}

\begin{quote}
\textbf{Character 116: Sclerites: Ventral valve: Posterior surface:
Delthyrium: Cover: Extent}

1: Covered only partially; partially open\\
2: Completely covered\\
Transformational character.
\end{quote}

\hypertarget{Micrina-coding-116}{}
\emph{Micrina}: Remains somewhat open.

\hypertarget{Micromitra-coding-116}{}
\emph{Micromitra}: Completely covered \citep[fig.
83.3]{Williams2000LinguliformeaCraniiformea}.

\hypertarget{Nisusia_sulcata-coding-116}{}
\emph{Nisusia sulcata}: A well-defined pseudo-deltidium {[}\ldots{}{]}
closes only the apical part of\\
the delthyrium \citep{Rowell1985Theevolutionary}.

\subsection*{{[}117{]} Posterior surface: Delthyrium: Cover:
Identity}\label{posterior-surface-delthyrium-cover-identity}
\addcontentsline{toc}{subsection}{{[}117{]} Posterior surface:
Delthyrium: Cover: Identity}

\includegraphics{Brachiopod_phylogeny_files/figure-latex/character-mapping-117.pdf}

\begin{quote}
\textbf{Character 117: Sclerites: Ventral valve: Posterior surface:
Delthyrium: Cover: Identity}

1: Pseudodeltidium\\
2: Deltidial plate(s)\\
3: Continuation of shell\\
Transformational character.
\end{quote}

This character has the capacity for further resolution (one or more
deltidial plates), but this is unlikely to affect the results of the
present study.

The pseudodelthyrium is also referred to as a homeodeltidium.

The antemucronal area of Polyplacophora is treated as equivalent to the
brachiopod delthyrium, but is not depositionally distinct to the rest of
the shell, so is coded with a distinct character state.

\hypertarget{Alisina-coding-117}{}
\emph{Alisina}: Stated as ``concave pseudodeltidium with median
plication'' -- \citet{Williams2000LinguliformeaCraniiformea}\\
Coded as ``Pseudodeltidium: Covered by concave plate'' by Bassett
\emph{et al}. \citeyearpar{Bassett2001Functionalmorphology}.

\hypertarget{Askepasma_toddense-coding-117}{}
\emph{Askepasma toddense}: No pseudodeltidium
\citep[p.~153]{Williams2000LinguliformeaCraniiformea}.

\hypertarget{Gasconsia-coding-117}{}
\emph{Gasconsia}: A homeodeltidium is illustrated by
\citet{Hanken1985Thetaxonomy}.

\hypertarget{Lingulellotreta_malongensis-coding-117}{}
\emph{Lingulellotreta malongensis}: The subapical flange of the
\emph{Paterimitra} S1 sclerite has been homologised with the ventral
homeodeltidium of \emph{Micromitra} \citep{Larsson2014iPaterimitra}.

\hypertarget{Mickwitzia_muralensis-coding-117}{}
\emph{Mickwitzia muralensis}: Termed a homoedeltidium by Balthasar
\citeyearpar{Balthasar2004Shellstructure}.

\hypertarget{Micrina-coding-117}{}
\emph{Micrina}: ``Ventral valve convex with apsacline interarea bearing
delthyrium, covered by a convex pseudodeltidium'' --
\citet{Holmer2008TheEarly}.

\subsection*{{[}118{]} Posterior surface: Delthyrium: Pseudodeltidium:
Shape}\label{posterior-surface-delthyrium-pseudodeltidium-shape}
\addcontentsline{toc}{subsection}{{[}118{]} Posterior surface:
Delthyrium: Pseudodeltidium: Shape}

\includegraphics{Brachiopod_phylogeny_files/figure-latex/character-mapping-118.pdf}

\begin{quote}
\textbf{Character 118: Sclerites: Ventral valve: Posterior surface:
Delthyrium: Pseudodeltidium: Shape}

1: Concave\\
2: Convex\\
Transformational character.
\end{quote}

A ridge-like (i.e.~convex) pseudodeltitium unites \emph{Salanygolina}
with \emph{Coolinia} and other Chileata
\citep[p.~6]{Holmer2009Theenigmatic}.

\hypertarget{Alisina-coding-118}{}
\emph{Alisina}: ``concave pseudodeltidium with median plication'' --
\citet{Williams2000LinguliformeaCraniiformea}\\
Coded as ``Pseudodeltidium: Covered by concave plate'' by Bassett
\emph{et al}. \citeyearpar{Bassett2001Functionalmorphology}.

\hypertarget{Antigonambonites_planus-coding-118}{}
\emph{Antigonambonites planus}: Convex \citep[fig.
508]{Williams2000LinguliformeaCraniiformea}.

\hypertarget{Gasconsia-coding-118}{}
\emph{Gasconsia}: ``Narrow depressed homeodeltidium'' --
\citet{Hanken1985Thetaxonomy}.

\hypertarget{Kutorgina_chengjiangensis-coding-118}{}
\emph{Kutorgina chengjiangensis}: Difficult to determine based on
material presented in Zhang \emph{et al}.
\citeyearpar{Zhang2007Rhynchonelliformeanbrachiopods}, or indeed for
other species in the genus
\citep[e.g.][]{Williams2000LinguliformeaCraniiformea, Skovsted2005EarlyCambrian, Holmer2018Theattachment}.

\hypertarget{Mickwitzia_muralensis-coding-118}{}
\emph{Mickwitzia muralensis}: Convex \citep[see][fig.
4B]{Balthasar2004Shellstructure}.

\hypertarget{Micrina-coding-118}{}
\emph{Micrina}: Convex deltoid \citep{Holmer2008TheEarly}.

\hypertarget{Micromitra-coding-118}{}
\emph{Micromitra}: Gently convex \citep[see][fig.
83.3]{Williams2000LinguliformeaCraniiformea}.

\hypertarget{Nisusia_sulcata-coding-118}{}
\emph{Nisusia sulcata}: Convex in \emph{Nisusia} \citep[see][fig.
8.4]{Rowell1985Theevolutionary}.

\hypertarget{Paterimitra-coding-118}{}
\emph{Paterimitra}: Gently convex \citep[see][fig.
83.1]{Williams2000LinguliformeaCraniiformea}.

\hypertarget{Salanygolina-coding-118}{}
\emph{Salanygolina}: ``The presence of {[}\ldots{}{]} a narrow
delthyrium covered by a convex pseudodeltidium, places Salanygolinidae
outside the Class Paterinata and strongly suggests affinity to the
Cambrian Chileida'' -- \citet{Holmer2009Theenigmatic}, p.~9.

\hypertarget{Tomteluva_perturbata-coding-118}{}
\emph{Tomteluva perturbata}: Convex \citep{Streng2016Anew}.

\subsection*{{[}119{]} Posterior surface: Delthyrium: Pseudodeltidium:
Hinge
furrows}\label{posterior-surface-delthyrium-pseudodeltidium-hinge-furrows}
\addcontentsline{toc}{subsection}{{[}119{]} Posterior surface:
Delthyrium: Pseudodeltidium: Hinge furrows}

\includegraphics{Brachiopod_phylogeny_files/figure-latex/character-mapping-119.pdf}

\begin{quote}
\textbf{Character 119: Sclerites: Ventral valve: Posterior surface:
Delthyrium: Pseudodeltidium: Hinge furrows}

0: Absent\\
1: Present\\
Neomorphic character.
\end{quote}

After Bassett \emph{et al}.
\citeyearpar{Bassett2001Functionalmorphology} character 18, ``Hinge
furrows on lateral sides of pseudodeltidium''.

\hypertarget{Acanthotretella_spinosa-coding-119}{}
\emph{Bactrotheca}, \emph{Conotheca}, \emph{Haplophrentis carinatus},
\emph{Maxilites}, \emph{Pauxillites}, \emph{Pedunculotheca diania},
\emph{Novocrania}, \emph{Pelagodiscus atlanticus}, \emph{Lingula},
\emph{Terebratulina}, \emph{Phoronis}, \emph{Dailyatia},
\emph{Acanthotretella spinosa}, \emph{Askepasma toddense},
\emph{Micromitra}, \emph{Clupeafumosus socialis}, \emph{Eccentrotheca},
\emph{Heliomedusa orienta}, \emph{Lingulosacculus},
\emph{Lingulellotreta malongensis}, \emph{Micrina}, \emph{Mummpikia
nuda}, \emph{Orthis}, \emph{Paterimitra}: Absent due to inapplicability
of neomorphic character.

\hypertarget{Gasconsia-coding-119}{}
\emph{Gasconsia}: Not evident or illustrated
\citep{Hanken1985Thetaxonomy}.

\hypertarget{Glyptoria-coding-119}{}
\emph{Glyptoria}: Coded as absent in
\citet{Bassett2001Functionalmorphology} (table 18.1).

\hypertarget{Kutorgina_chengjiangensis-coding-119}{}
\emph{Kutorgina chengjiangensis}, \emph{Nisusia sulcata}: Coded as
present in \citet{Bassett2001Functionalmorphology} (table 18.1).

\hypertarget{Salanygolina-coding-119}{}
\emph{Salanygolina}: The presence of this feature is impossible to
determine based on the available material.

\subsection*{{[}120{]} Umbonal perforation}\label{umbonal-perforation}
\addcontentsline{toc}{subsection}{{[}120{]} Umbonal perforation}

\includegraphics{Brachiopod_phylogeny_files/figure-latex/character-mapping-120.pdf}

\begin{quote}
\textbf{Character 120: Sclerites: Ventral valve: Umbonal perforation}

0: Umbo imperforate (or ventral valve absent)\\
1: Umbonal perforation\\
Neomorphic character.
\end{quote}

Certain taxa, particularly those with a colleplax, exhibit a perforation
at the umbo of the ventral valve. This opening is sometimes associated
with a pedicle sheath, which emerges from the umbo of the ventral valve
without any indication of a relationship with the hinge.

In contrast, the pedicle of acrotretids and similar brachiopods is
situated on the larval hinge line, but is later surrounded by the
posterolateral regions of the growing shell to become separated from the
hinge line, and encapsulated in a position close to (or with resorption
of the brephic shell, at) the umbo \citep[see][pp.~407--411 and fig. 3
for discussion]{Popov1992TheCambrian}. In some cases, an internal
pedicle tube attests to this origin -- potentially corresponding to the
pedicle groove of lingulids. As such, the pedicle foramen of acrotretids
and allies is not originally situated at the umbo; it is instead
understood to represent a basally sealed delthyrium.

\hypertarget{Bactrotheca-coding-120}{}
\emph{Bactrotheca}: The apical termination of the conical valve is not
preserved \citep{Valent2012}.

\hypertarget{Clupeafumosus_socialis-coding-120}{}
\emph{Clupeafumosus socialis}: The presumed pedicle foramen reported by
Topper \emph{et al}. \citeyearpar{Topper2013Reappraisalof} is at the
ventral valve umbo. No evidence of an internal pedicle tube is present,
but we follow Popov \citeyearpar{Popov1992TheCambrian} in inferring the
encapsulation of the pedicle foramen.

\hypertarget{Cupitheca_holocyclata-coding-120}{}
\emph{Cupitheca holocyclata}: Decollation generates open umbo, sealed
secondarily with septum.

\hypertarget{Dailyatia-coding-120}{}
\emph{Dailyatia}: The B and C sclerites of \emph{Dailyatia} bear small
umbonal perforations \citep{Skovsted2015Theearly}, but these are not
considered to be homologous with the ventral valve, so this character is
coded as inapplicable -- though the possibility that the perforations
are equivalent is intriguing.

A1 sclerites typically have a pair of perforations, which are
conceivably equivalent to the setal tubes of \emph{Micrina}
\citep{Holmer2011Firstrecord}. The A1 sclerite of D. bacata has a
structure that is arguably similar to the `colleplax' of
\emph{Paterimitra}. But the homology of any of these structures to the
umbonal aperture of brachiopods is difficult to establish.

\hypertarget{Eccentrotheca-coding-120}{}
\emph{Eccentrotheca}: The sclerites of \emph{Eccentrotheca} form a ring
that surrounds the inferred attachment structure; the attachment
structure does not emerge from an aperture within an individual
sclerite. Thus no feature in \emph{Eccentrotheca} is judged to be
potentially homologous with the apical perforation in bivalved
brachiopods.

\hypertarget{Heliomedusa_orienta-coding-120}{}
\emph{Heliomedusa orienta}: There is ``compelling evidence to
demonstrate that the putative pedicle\\
illustrated by Chen \emph{et al}. \citeyearpar[Figs. 4, 6,
7]{Chen2007Reinterpretationof} in fact is the mold of a
three-dimensionally preserved visceral cavity.'' --
\citet{Zhang2009Architectureand}.

\hypertarget{Lingulosacculus-coding-120}{}
\emph{Lingulosacculus}: The apical termination of the fossil is unknown.

\hypertarget{Mickwitzia_muralensis-coding-120}{}
\emph{Mickwitzia muralensis}: The umbo itself is imperforate
\citep{Balthasar2004Shellstructure}.

\hypertarget{Paramicrocornus-coding-120}{}
\emph{Paramicrocornus}: \citet{Zhang2018Ahyolithid}.

\hypertarget{Paterimitra-coding-120}{}
\emph{Paterimitra}: The presumed pedicle foramen is an opening between
the S1 and S2 sclerites, neither of which are perforated
\citep{Skovsted2009Thescleritome}.

\hypertarget{Siphonobolus_priscus-coding-120}{}
\emph{Siphonobolus priscus}: Prominent subcircular perforation at umbo
associated with an internal pedicle tube \citep{Popov2009Earlyontogeny},
thus presumed to share an origin with the acrotretid pedicle foramen.

\hypertarget{Tomteluva_perturbata-coding-120}{}
\emph{Tomteluva perturbata}: Streng \emph{et al}.
\citeyearpar{Streng2016Anew} observe ``an internal tubular structure
probably representing the ventral end of the canal within the posterior
wall of the pedicle tube'', but do not consider this tomteluvid dube to
be homologous with the pedicle tube of acrotretids and their ilk,
stating (p.~274) that it appears to be unique within Brachiopoda.

\subsection*{{[}121{]} Umbonal perforation:
Shape}\label{umbonal-perforation-shape}
\addcontentsline{toc}{subsection}{{[}121{]} Umbonal perforation: Shape}

\includegraphics{Brachiopod_phylogeny_files/figure-latex/character-mapping-121.pdf}

\begin{quote}
\textbf{Character 121: Sclerites: Ventral valve: Umbonal perforation:
Shape}

1: Circular (or subcircular)\\
2: Rhombic to oval\\
3: Arising through decollation\\
Transformational character.
\end{quote}

The perforation in \emph{Cupitheca} seems to have a distinct origin,
arising through decollation; as such, the shape simply reflects the
outline of the shell. This reflects a distinct origin of the perforation
and is therefore provided as a separate state.

\hypertarget{Acanthotretella_spinosa-coding-121}{}
\emph{Acanthotretella spinosa}: Too small to observe given quality of
preservation \citep{Holmer2006Aspinose}.

\hypertarget{Alisina-coding-121}{}
\emph{Alisina}: Seemingly circular \citep{Zhang2011Anobolellate}.

\hypertarget{Antigonambonites_planus-coding-121}{}
\emph{Antigonambonites planus}: Based on p.92, fig.4B.

\hypertarget{Clupeafumosus_socialis-coding-121}{}
\emph{Clupeafumosus socialis}: Taller than wide in some cases, but very
nearly circular in others; see Topper \emph{et al}.
\citeyearpar{Topper2013Reappraisalof}.

\hypertarget{Coolinia_pecten-coding-121}{}
\emph{Coolinia pecten}: Bassett and Popov write ``a dominant feature of
the ventral umbo is a sub-oval perforation about 270 µm long and 250 µm
wide'': the width and height of this structure are almost identical, and
we score it as (sub) circular.

\hypertarget{Heliomedusa_orienta-coding-121}{}
\emph{Heliomedusa orienta}: Rhombic to oval -- seen as evidence for a
discinid affinity \citep{Chen2007Reinterpretationof}.

\hypertarget{Kutorgina_chengjiangensis-coding-121}{}
\emph{Kutorgina chengjiangensis}: The exact size and shape of the apical
perforation is obscured by the emerging pedicle.

\hypertarget{Nisusia_sulcata-coding-121}{}
\emph{Nisusia sulcata}: ``close to circular''
\citep{Holmer2018Evolutionarysignificance}.

\hypertarget{Salanygolina-coding-121}{}
\emph{Salanygolina}: Essentially circular \citep[fig.
4]{Holmer2009Theenigmatic}.

\subsection*{{[}122{]} Colleplax, cicatrix or pedicle
sheath}\label{colleplax-cicatrix-or-pedicle-sheath}
\addcontentsline{toc}{subsection}{{[}122{]} Colleplax, cicatrix or
pedicle sheath}

\includegraphics{Brachiopod_phylogeny_files/figure-latex/character-mapping-122.pdf}

\begin{quote}
\textbf{Character 122: Sclerites: Ventral valve: Colleplax, cicatrix or
pedicle sheath}

0: Absent\\
1: Present\\
Neomorphic character.
\end{quote}

In certain taxa, the umbo of the ventral valve bears a colleplax,
cicatrix or pedicle sheath; Bassett \emph{et al}.
\citeyearpar{Bassett2008Earlyontogeny} consider these structures as
homologous.

\hypertarget{Bactrotheca-coding-122}{}
\emph{Bactrotheca}: The apical termination of the conical valve is not
preserved \citep{Valent2012}.

\hypertarget{Botsfordia-coding-122}{}
\emph{Botsfordia}: Following \citet{Williams1998Thediversity}, appendix
2.

\hypertarget{Clupeafumosus_socialis-coding-122}{}
\emph{Clupeafumosus socialis}: Not reported by Topper \emph{et al}.
\citeyearpar{Topper2013Reappraisalof}.

\hypertarget{Craniops-coding-122}{}
\emph{Craniops}: \emph{Paracraniops} is ``externally similar to
\emph{Craniops}, but lacking cicatrix'' -- indicating that
\emph{Craniops} bears a cicatrix
\citep{Williams2000LinguliformeaCraniiformea}. Also coded as present in
their table 15.

\hypertarget{Heliomedusa_orienta-coding-122}{}
\emph{Heliomedusa orienta}: A cicatrix was reconstructed by
\citet{Jin1992Revisionof} (figs 6b, 7), but has not been reported by
later authors; possibly, as with the `pedicle foramen' of Chen \emph{et
al}. \citeyearpar{Chen2007Reinterpretationof}, this structure represents
internal organs rather than a cicatrix proper
\citep{Zhang2009Architectureand}; as such it has been recorded as
ambiguous.

\hypertarget{Kutorgina_chengjiangensis-coding-122}{}
\emph{Kutorgina chengjiangensis}: The umbonal region of kutorginides
``clearly lacks a pedicle sheath'' \citep{Holmer2018Theattachment}.

\hypertarget{Lingulellotreta_malongensis-coding-122}{}
\emph{Lingulellotreta malongensis}: The pedicle is identified as such
(rather than a pedicle sheath) by the internal pedicle tube.

\hypertarget{Longtancunella_chengjiangensis-coding-122}{}
\emph{Longtancunella chengjiangensis}: A ring-like structure surrounding
the pedicle is interpreted as a colleplax
\citep{Zhang2011Theexceptionally}, though the authors make no comparison
with the pedicle capsule exhibited by extant terebratulids
\citep[see][]{Holmer2018Evolutionarysignificance}.

\hypertarget{Micrina-coding-122}{}
\emph{Micrina}: Absent in \emph{Micrina} \citep{Holmer2011Firstrecord}.

\hypertarget{Pedunculotheca_diania-coding-122}{}
\emph{Pedunculotheca diania}: The flat apical termination of juvenile
individuals possibly functioned as colleplax for attachment, but may
simply represent the brephic shell; we treat it as ambiguous to reflect
this potential homology.

\hypertarget{Siphonobolus_priscus-coding-122}{}
\emph{Siphonobolus priscus}: Coded as present in view of the attachment
scar, which has been considered homologous with the ``adult colleplax
and foramen with attachment pad'' in \emph{Salanygolina}
\citep{Popov2009Earlyontogeny}.

\hypertarget{Tomteluva_perturbata-coding-122}{}
\emph{Tomteluva perturbata}: The internal canal associated with the
pedicle is unique to the tomteluvids, and has an uncertain identity
\citep{Streng2016Anew}. It could conceivably correspond to an
internalized pedicle sheath or an equivalent structure, so this feature
is coded as ambiguous here.

\hypertarget{Ussunia-coding-122}{}
\emph{Ussunia}: Following table 15 in
\citet{Williams2000LinguliformeaCraniiformea}.

\hypertarget{Yuganotheca_elegans-coding-122}{}
\emph{Yuganotheca elegans}: The median collar or conical tube is
conceivably homologous with the pedicle sheath.

\subsection*{{[}123{]} Median septum}\label{median-septum}
\addcontentsline{toc}{subsection}{{[}123{]} Median septum}

\includegraphics{Brachiopod_phylogeny_files/figure-latex/character-mapping-123.pdf}

\begin{quote}
\textbf{Character 123: Sclerites: Ventral valve: Median septum}

0: Absent\\
1: Present\\
Neomorphic character.
\end{quote}

Chen \emph{et al}. \citeyearpar{Chen2007Reinterpretationof} observe a
median septum in what they interpret as the ventral valve of
\emph{Heliomedusa}, and the ventral valve of \emph{Discinisca}, which
they propose points to a close relationship.

\hypertarget{Acanthotretella_spinosa-coding-123}{}
\emph{Acanthotretella spinosa}: Carbonaceous preservation confounds the
identification of internal shell structures; it is possible that this
feature is present, but not observable in the Burgess Shale material.

\hypertarget{Botsfordia-coding-123}{}
\emph{Botsfordia}: Following \citet{Williams1998Thediversity}, appendix
2.

\hypertarget{Clupeafumosus_socialis-coding-123}{}
\emph{Clupeafumosus socialis}: A short medial ridge (septum) is present
in the ventral valve \citep{Topper2013Reappraisalof}.

\hypertarget{Eoobolus-coding-123}{}
\emph{Eoobolus}: Prominent median septum \citep[fig. 4d, e
in][]{Balthasar2009Thebrachiopod}.

\hypertarget{Gasconsia-coding-123}{}
\emph{Gasconsia}: Evident in moulds of ventral valve
\citep{Hanken1985Thetaxonomy, Watkins2002Newrecord}.

\hypertarget{Glyptoria-coding-123}{}
\emph{Glyptoria}: Neither evident nor reported in Williams \emph{et al}.
\citeyearpar{Williams2000LinguliformeaCraniiformea}.

\hypertarget{Haplophrentis_carinatus-coding-123}{}
\emph{Haplophrentis carinatus}: The carina of \emph{H. carinatus} is an
angular elevation of the ventral valve surface, rather than a septum
growing inward on the interior of shell.

\hypertarget{Heliomedusa_orienta-coding-123}{}
\emph{Heliomedusa orienta}: Reported on `ventral' valve by Chen \emph{et
al}. \citeyearpar{Chen2007Reinterpretationof}; we consider the `ventral'
valve to be the dorsal valve.

\hypertarget{Lingulellotreta_malongensis-coding-123}{}
\emph{Lingulellotreta malongensis}: Medial septum visible in ventral
valve in Williams \emph{et al}.
\citeyearpar{Williams2000LinguliformeaCraniiformea}, fig. 34.1c.

\hypertarget{Micromitra-coding-123}{}
\emph{Micromitra}: Ventral ridge characteristic of \emph{Micromitra}
\citep{Skovsted2010EarlyCambrian}.

\hypertarget{Mummpikia_nuda-coding-123}{}
\emph{Mummpikia nuda}: ``Some specimens also reveal that the vault had a
slight median septum, which is now visible as a notch or a groove
dividing the right from the left part'' --
\citet{Balthasar2008iMummpikia}.

\hypertarget{Novocrania-coding-123}{}
\emph{Novocrania}: Valve thin and often unmineralized.

\hypertarget{Pelagodiscus_atlanticus-coding-123}{}
\emph{Pelagodiscus atlanticus}: Described as present in
\emph{Discinisca} by \citet{Chen2007Reinterpretationof}; assumed present
also in \emph{Pelagodiscus}.

\hypertarget{Siphonobolus_priscus-coding-123}{}
\emph{Siphonobolus priscus}: Present; see
\citet{Popov2009Earlyontogeny}, fig. 5J.

\hypertarget{Ussunia-coding-123}{}
\emph{Ussunia}: Following char. 42 in table 15 in
\citet{Williams2000LinguliformeaCraniiformea}.

\section{Sclerites: Ornament}\label{sclerites-ornament}

\subsection*{{[}124{]} Concentric ornament}\label{concentric-ornament}
\addcontentsline{toc}{subsection}{{[}124{]} Concentric ornament}

\includegraphics{Brachiopod_phylogeny_files/figure-latex/character-mapping-124.pdf}

\begin{quote}
\textbf{Character 124: Sclerites: Ornament: Concentric ornament}

1: Smooth, or growth lines only\\
2: Concentric ornament present\\
Transformational character.
\end{quote}

After character 11 in Williams \emph{et al}.
\citeyearpar{Williams1998Thediversity}. Coded as transformational as it
is possible that maintaining a smooth shell without occasional prominent
ridges requires greater secretory control.

\hypertarget{Askepasma_toddense-coding-124}{}
\emph{Askepasma toddense}, \emph{Micromitra}, \emph{Glyptoria},
\emph{Kutorgina chengjiangensis}, \emph{Salanygolina}: Following
appendix 2 in Williams \emph{et al}.
\citeyearpar{Williams1998Thediversity}.

\hypertarget{Bactrotheca-coding-124}{}
\emph{Bactrotheca}: Both valves with fine growth lines only
\citep{Valent2012}.

\hypertarget{Botsfordia-coding-124}{}
\emph{Botsfordia}: Following \citet{Williams1998Thediversity}, appendix
2.\\
Pustules are arranged along concentric growth lines
\citep{Skovsted2005EarlyCambrian}, so are not treated as a distinct
ornamentation.

\hypertarget{Conotheca-coding-124}{}
\emph{Conotheca}: Both valves covered with concentric growth lines
\citep{Devaere2014}, but no further ornament \citep{Wrona2003}.

\hypertarget{Cotyledion_tylodes-coding-124}{}
\emph{Cotyledion tylodes}: \citet{Zhang2013}.

\hypertarget{Cupitheca_holocyclata-coding-124}{}
\emph{Cupitheca holocyclata}: Broad symmetric ridges in some specimens
\citep{Vendrasco2017}.

\hypertarget{Eccentrotheca-coding-124}{}
\emph{Eccentrotheca}: More or less concentric ridges occur on
\emph{Eccentrotheca} sclerites
\citep{Skovsted2011Scleritomeconstruction}.

\hypertarget{Halkieria_evangelista-coding-124}{}
\emph{Halkieria evangelista}: Ridges in shell parallel, but are more
prominent than, growth lines.

\hypertarget{Haplophrentis_carinatus-coding-124}{}
\emph{Haplophrentis carinatus}: A series of regularly spaced concentric
ridges adorn both valves \citep{Moysiuk2017Hyolithsare}; these are more
pronounced than mere growth lines.

\hypertarget{Heliomedusa_orienta-coding-124}{}
\emph{Heliomedusa orienta}: The ornament on shell exterior is described
as concentric fila \citep[P.43]{Chen2007Reinterpretationof}, and also
scored as it in Williams \emph{et al}.
\citeyearpar[pp.160--163]{Williams2000LinguliformeaCraniiformea}.

\hypertarget{Maxilites-coding-124}{}
\emph{Maxilites}: Surfaces of both valves covered with concentric growth
lines \citep{Marek1972, MartiMus2005}.

\hypertarget{Mickwitzia_muralensis-coding-124}{}
\emph{Mickwitzia muralensis}: Symmetric fila.

\hypertarget{Novocrania-coding-124}{}
\emph{Novocrania}: Irregular ridges externally
\citep{Williams2000LinguliformeaCraniiformea}.

\hypertarget{Pauxillites-coding-124}{}
\emph{Pauxillites}: Ventral side of ventral valve and whole dorsal valve
covered with faint growth lines \citep{Valent2015}.

\hypertarget{Pedunculotheca_diania-coding-124}{}
\emph{Pedunculotheca diania}: A series of regularly spaced concentric
ridges adorn the ventral valve; comparatively less regular lines
ornament the operculum.

\hypertarget{Pelagodiscus_atlanticus-coding-124}{}
\emph{Pelagodiscus atlanticus}: Only growth lines evident
\citep{Williams2000LinguliformeaCraniiformea}.

\hypertarget{Terebratulina-coding-124}{}
\emph{Terebratulina}: Single ridge evident in Williams \emph{et al}.
\citeyearpar{Williams2006Rhynchonelliformeapart} fig. 1425.1a
interpreted as interruption ot growth rather than inherent feature, so
coded as absent (i.e.~smooth).

\hypertarget{Tonicella-coding-124}{}
\emph{Tonicella}: No prominent ornamentat in \emph{Tonicella}
\citep{Connors2012}.

\subsection*{{[}125{]} Concentric ornament:
Symmetry}\label{concentric-ornament-symmetry}
\addcontentsline{toc}{subsection}{{[}125{]} Concentric ornament:
Symmetry}

\includegraphics{Brachiopod_phylogeny_files/figure-latex/character-mapping-125.pdf}

\begin{quote}
\textbf{Character 125: Sclerites: Ornament: Concentric ornament:
Symmetry}

1: Asymmetric fila, with outer faces\\
2: Symmetric fila\\
Transformational character.
\end{quote}

After character 11 in Williams \emph{et al}.
\citeyearpar{Williams1998Thediversity}.

\hypertarget{Alisina-coding-125}{}
\emph{Alisina}: Seemingly asymmetric \citetext{\citealp[fig.
122.3c]{Williams2000LinguliformeaCraniiformea}; \citealp[Fig.
1]{Zhang2011Anobolellate}}.

\hypertarget{Askepasma_toddense-coding-125}{}
\emph{Askepasma toddense}, \emph{Micromitra}, \emph{Glyptoria},
\emph{Kutorgina chengjiangensis}, \emph{Salanygolina}: Following
appendix 2 in Williams \emph{et al}.
\citeyearpar{Williams1998Thediversity}.

\hypertarget{Dailyatia-coding-125}{}
\emph{Dailyatia}: Clear asymmetry \citep{Skovsted2015Theearly}.

\hypertarget{Eccentrotheca-coding-125}{}
\emph{Eccentrotheca}: Ornament, such as it is, is asymmetric, with
prominent outer faces \citep{Skovsted2011Scleritomeconstruction}.

\hypertarget{Gasconsia-coding-125}{}
\emph{Gasconsia}: Assymmetric \citep[fig. 3]{Hanken1985Thetaxonomy}.

\hypertarget{Heliomedusa_orienta-coding-125}{}
\emph{Heliomedusa orienta}: See fig. 1715 in Williams \emph{et al}.
\citeyearpar{Williams2007Supplement}.

\hypertarget{Mickwitzia_muralensis-coding-125}{}
\emph{Mickwitzia muralensis}: Symmetric fila
\citep{Balthasar2004Shellstructure}.

\hypertarget{Micrina-coding-125}{}
\emph{Micrina}: No obvious asymmetry, even if not obviously symmetric
either \citep{Holmer2008TheEarly}. Coded as ambiguous.

\hypertarget{Novocrania-coding-125}{}
\emph{Novocrania}: Clear outer faces \citep[fig.
100.2b]{Williams2000LinguliformeaCraniiformea}.

\subsection*{{[}126{]} Radial ornament}\label{radial-ornament}
\addcontentsline{toc}{subsection}{{[}126{]} Radial ornament}

\includegraphics{Brachiopod_phylogeny_files/figure-latex/character-mapping-126.pdf}

\begin{quote}
\textbf{Character 126: Sclerites: Ornament: Radial ornament}

0: Absent\\
1: Present\\
Neomorphic character.
\end{quote}

Ridges radiating from umbo, i.e.~ribs.

\hypertarget{Askepasma_toddense-coding-126}{}
\emph{Askepasma toddense}: ``Ornament of irregularly developed,
concentric growth lamellae; microornament of irregularly arranged,
polygonal pits'' -- \citet{Williams2000LinguliformeaCraniiformea}, p153;
figs on p.155.

\hypertarget{Botsfordia-coding-126}{}
\emph{Botsfordia}: Following \citet{Williams1998Thediversity}, Appendix
2.

\hypertarget{Eoobolus-coding-126}{}
\emph{Eoobolus}: Very faint costellae in some specimens but coded
absent.

\hypertarget{Gasconsia-coding-126}{}
\emph{Gasconsia}: ``Ornament of indistinct low radial ribs'' -- Williams
\emph{et al}. \citeyearpar[p167]{Williams2000LinguliformeaCraniiformea}.

\hypertarget{Glyptoria-coding-126}{}
\emph{Glyptoria}: ``Coarsely costate'' -- Williams \emph{et al}.
\citeyearpar[p710]{Williams2000LinguliformeaCraniiformea}.

\hypertarget{Heliomedusa_orienta-coding-126}{}
\emph{Heliomedusa orienta}: See fig. 1715 in Williams \emph{et al}.
\citeyearpar{Williams2007Supplement}.

\hypertarget{Maxilites-coding-126}{}
\emph{Maxilites}: Lines radiate from the apex of the operculum
\citep[pl.2 fig. 5]{Marek1972} and ornament the conical valve \citep[pl.
2. fig. 3]{Marek1972}.

\hypertarget{Paramicrocornus-coding-126}{}
\emph{Paramicrocornus}: \citet{Zhang2018Ahyolithid}.

\hypertarget{Siphonobolus_priscus-coding-126}{}
\emph{Siphonobolus priscus}: ``Indistinct radial ribs accentuated by
radial rows of tubercles'' -- \citet{Popov2009Earlyontogeny}.

\hypertarget{Ussunia-coding-126}{}
\emph{Ussunia}: Unornamented.

\subsection*{{[}127{]} Shell-penetrating
spines}\label{shell-penetrating-spines}
\addcontentsline{toc}{subsection}{{[}127{]} Shell-penetrating spines}

\includegraphics{Brachiopod_phylogeny_files/figure-latex/character-mapping-127.pdf}

\begin{quote}
\textbf{Character 127: Sclerites: Ornament: Shell-penetrating spines}

0: Absent\\
1: Present\\
Neomorphic character.
\end{quote}

Mineralized or partly mineralized spines are observed in
\emph{Heliomedusa} and \emph{Acanthotretella}.

\hypertarget{Glyptoria-coding-127}{}
\emph{Glyptoria}: Neither evident nor reported in Williams \emph{et al}.
\citeyearpar{Williams2000LinguliformeaCraniiformea}.

\hypertarget{Heliomedusa_orienta-coding-127}{}
\emph{Heliomedusa orienta}: The `spines' reported by Chen \emph{et al}.
\citeyearpar{Chen2007Reinterpretationof} are pyritized spinelike setae
-- see pp.~2580--2590 in Williams \emph{et al}.
\citeyearpar{Williams2007Supplement}.

\hypertarget{Nisusia_sulcata-coding-127}{}
\emph{Nisusia sulcata}: Bears numerous small, hollow spines
\citep{Williams2000LinguliformeaCraniiformea}.

\hypertarget{Tonicella-coding-127}{}
\emph{Tonicella}: Aesthete canals penetrate the main valves of certain
chitons, but are not equivalent to the shell-penetrating spines of
brachiopods.

\section{Sclerites: Composition}\label{sclerites-composition}

\subsection*{{[}128{]} Mineralogy}\label{mineralogy}
\addcontentsline{toc}{subsection}{{[}128{]} Mineralogy}

\includegraphics{Brachiopod_phylogeny_files/figure-latex/character-mapping-128.pdf}

\begin{quote}
\textbf{Character 128: Sclerites: Composition: Mineralogy}

1: Organic (non-mineralized)\\
2: Phosphatic\\
3: Calcitic\\
4: Aragonitic\\
Transformational character.
\end{quote}

\hypertarget{Acanthotretella_spinosa-coding-128}{}
\emph{Acanthotretella spinosa}: Holmer \& Caron
\citeyearpar{Holmer2006Aspinose} note the absence of brittle breakage,
interpreted as indicating the absence of a material mineralized
component to the shells. The preservation is strikingly different from
that of other Burgess Shale brachiopods, ruling out a primarily calcitic
or phosphatic composition. The two-dimensional nature of the
preservation also differs from that of co-occurring aragonitic taxa
\citep[hyoliths;][ p.~273]{Holmer2006Aspinose}, indicating that any
mineralization was minor at best.

Holmer \& Caron \citeyearpar[p.~286]{Holmer2006Aspinose} suggest that it
is more likely that a (minor) mineral component was present than that it
was not, though without providing an uncontestable rationale. To be as
conservative as possible, we therefore code this taxon as ambiguous.

\hypertarget{Clupeafumosus_socialis-coding-128}{}
\emph{Clupeafumosus socialis}: Phosphatic -- hence the conventional
placement within Linguliformea.

\hypertarget{Cotyledion_tylodes-coding-128}{}
\emph{Cotyledion tylodes}: The extensive relief and association with
pyrite framboids indicates original mineralization, but the identity of
the biomineral remains uncertain \citep{Zhang2013}.

\hypertarget{Craniops-coding-128}{}
\emph{Craniops}: Shell calcitic.

\hypertarget{Cupitheca_holocyclata-coding-128}{}
\emph{Cupitheca holocyclata}: Reconstructed as aragonitic from
microstructural fabrics \citep{Vendrasco2017}.

\hypertarget{Eoobolus-coding-128}{}
\emph{Eoobolus}: ``the original shell of \emph{Eoobolus} contained small
calcareous grains that were incorporated into organic-rich layers
alongside apatite'' \citep{Balthasar2007Anearly}.

\hypertarget{Gasconsia-coding-128}{}
\emph{Gasconsia}: Confirmed in Trimerella by
\citet{Balthasar2011Relicaragonite}.

\hypertarget{Heliomedusa_orienta-coding-128}{}
\emph{Heliomedusa orienta}: ``Shell originally organophosphatic, but may
generally have been poorly mineralized'' --
\citet{Williams2007Supplement} -- cf.~ibid, p.~2889, ``These strong
similarities to discinoids in soft-part anatomy imply that the
\emph{Heliomedusa} shell was chitinous or chitinophosphatic, not
calcareous.''

\hypertarget{Lingulellotreta_malongensis-coding-128}{}
\emph{Lingulellotreta malongensis}: Coded as phosphatic by Zhang
\emph{et al}. \citeyearpar{Zhang2014Anearly}, but with no explanation.\\
Cracks within shells of Chengjiang specimens \citep[e.g.][fig.
3]{Zhang2007Noteon} demonstrate that the shells were originally
mineralized, but not the identity of the original biomineral. This said,
phosphatized material from Kazakhstan \citep{Holmer1997EarlyCambrian} is
attributed to the same species; presuming this phosphate to be original
and the material to be conspecific, \emph{L. malongensis} is coded as
having phosphatic shells.

\hypertarget{Lingulosacculus-coding-128}{}
\emph{Lingulosacculus}: The absence of relief in \emph{Lingulosacculus}
rules out a phosphatic or calcitic composition, but co-occurring (and
presumably aragonitic) hyolithids are preserved in the same fashion. Its
constitution was thus either organic or aragonitic
\citep{Balthasar2009EarlyCambrian}.

\hypertarget{Longtancunella_chengjiangensis-coding-128}{}
\emph{Longtancunella chengjiangensis}: ``The original composition of the
shell cannot be determined with certainty'', though it was ``most
probably entirely soft and organic'' --
\citet{Zhang2011Theexceptionally}.

\hypertarget{Mickwitzia_muralensis-coding-128}{}
\emph{Mickwitzia muralensis}: Calcite and silica deemed diagenetic by
Balthasar \citeyearpar{Balthasar2004Shellstructure}.

\hypertarget{Mummpikia_nuda-coding-128}{}
\emph{Mummpikia nuda}: Identified as calcareous by preservational
criteria, and description ``primary\\
calcitic shells of \emph{M. nuda}'' \citep{Balthasar2008iMummpikia}.

\hypertarget{Novocrania-coding-128}{}
\emph{Novocrania}: Ventral valve uncalcified in extant forms or
sometimes thin \citep{Williams2000LinguliformeaCraniiformea}, but coded
as calcitic as calcite-mineralizing pathways are present.

\hypertarget{Salanygolina-coding-128}{}
\emph{Salanygolina}: Original mineralogy unknown, but known to be
mineralised and anticipated to be phosphatic
\citep{Holmer2009Theenigmatic}.

\hypertarget{Ussunia-coding-128}{}
\emph{Ussunia}: Trimerellids were probably aragonitic
\citep{Williams2000LinguliformeaCraniiformea}.

\subsection*{{[}129{]} Cuticle or organic
matrix}\label{cuticle-or-organic-matrix}
\addcontentsline{toc}{subsection}{{[}129{]} Cuticle or organic matrix}

\includegraphics{Brachiopod_phylogeny_files/figure-latex/character-mapping-129.pdf}

\begin{quote}
\textbf{Character 129: Sclerites: Composition: Cuticle or organic
matrix}

1: GAGs, chitin and collagen\\
2: Glycoprotein\\
Transformational character.
\end{quote}

Williams \emph{et al}. \citeyearpar{Williams1996Asupra} identify
glycoprotein-based organic scaffolds as distinct from those comprising
glycosaminoglycans (GAGs), chitin and collagen. This character can only
be scored for extant taxa.

\hypertarget{Lingula-coding-129}{}
\emph{Lingula}: Coded as GAGs, chitin and collagen in lingulids by
Williams \emph{et al}. \citeyearpar{Williams1996Asupra}.

\hypertarget{Novocrania-coding-129}{}
\emph{Novocrania}: Coded as glycoprotein for craniids by Williams
\emph{et al}. \citeyearpar{Williams1996Asupra}.

\hypertarget{Pelagodiscus_atlanticus-coding-129}{}
\emph{Pelagodiscus atlanticus}: Coded as GAGs, chitin and collagen in
discinids by Williams \emph{et al}. \citeyearpar{Williams1996Asupra}.

\hypertarget{Phoronis-coding-129}{}
\emph{Phoronis}: ``The presence of sulphated glycosaminoglycans (GAGs)
in the chitinous cuticle of \emph{Phoronis}
\citep[p.~215]{Herrmann1997Phoronida} would suggest a link with
linguliforms, as GAGs are unknown in rhynchonelliform shells (Fig. 1891,
1896)'' -- \citet{Williams2007Supplement}, p.~2830.

\hypertarget{Siphonobolus_priscus-coding-129}{}
\emph{Siphonobolus priscus}: Lenticular chambers in siphonotretid shells
interpreted as degraded GAG residue
\citep{Williams2004Chemicostructure}.

\hypertarget{Terebratulina-coding-129}{}
\emph{Terebratulina}: Coded as glycoprotein for terebratulids by
Williams \emph{et al}. \citeyearpar{Williams1996Asupra}.

\subsection*{{[}130{]} Incorporation of sedimentary
particles}\label{incorporation-of-sedimentary-particles}
\addcontentsline{toc}{subsection}{{[}130{]} Incorporation of sedimentary
particles}

\includegraphics{Brachiopod_phylogeny_files/figure-latex/character-mapping-130.pdf}

\begin{quote}
\textbf{Character 130: Sclerites: Composition: Incorporation of
sedimentary particles}

0: Absent\\
1: Present\\
Neomorphic character.
\end{quote}

Phoronids and \emph{Yuganotheca} aggulutinate particles into their
sclerites.

\subsection*{{[}131{]} Periostracum:
Flexibility}\label{periostracum-flexibility}
\addcontentsline{toc}{subsection}{{[}131{]} Periostracum: Flexibility}

\includegraphics{Brachiopod_phylogeny_files/figure-latex/character-mapping-131.pdf}

\begin{quote}
\textbf{Character 131: Sclerites: Composition: Periostracum:
Flexibility}

1: Flexible\\
2: Inflexible\\
Transformational character.
\end{quote}

Following character 9 in Williams \emph{et al}.
\citeyearpar{Williams1998Thediversity}; see their p228--230 for a
discussion of how this might be inferred from fossil material.

\hypertarget{Askepasma_toddense-coding-131}{}
\emph{Askepasma toddense}, \emph{Micromitra}, \emph{Glyptoria},
\emph{Kutorgina chengjiangensis}: Following appendix 2 in Williams
\emph{et al}. \citeyearpar{Williams1998Thediversity}.

\hypertarget{Botsfordia-coding-131}{}
\emph{Botsfordia}, \emph{Eoobolus}: Coded as flexible in
\citet{Williams1998Thediversity}, Appendix 2.

\hypertarget{Pelagodiscus_atlanticus-coding-131}{}
\emph{Pelagodiscus atlanticus}: Flexible
\citep{Williams1998Thediversity}.

\hypertarget{Salanygolina-coding-131}{}
\emph{Salanygolina}: Coded as uncertain in appendix 2 in Williams
\emph{et al}. \citeyearpar{Williams1998Thediversity}.

\subsection*{{[}132{]} Microstructure: Number of distinct
layers}\label{microstructure-number-of-distinct-layers}
\addcontentsline{toc}{subsection}{{[}132{]} Microstructure: Number of
distinct layers}

\includegraphics{Brachiopod_phylogeny_files/figure-latex/character-mapping-132.pdf}

\begin{quote}
\textbf{Character 132: Sclerites: Composition: Microstructure: Number of
distinct layers}

1: Single microstructural layer\\
2: Two microstructurally differentiated layers\\
3: Inner and outer laminae enclosing medial void\\
4: Three microstrurally differentiated layers\\
Transformational character.
\end{quote}

Hyolith conchs comprise two mineralized layers of fibrous bundles.
Bundles are measure 5--15 µm across; their constituent fibres are each
0.1--1.0 µm wide. In the inner layer, the fibres are transverse; in the
outer layer, the bundles are inclined towards the umbo, becoming
longitudinal on the outermost margin.

Coded as non-additive as there is no clear necessity to add layers
sequentially: for example, three layers could arise by the addition of a
void within a single pre-existing layer.

Stratiform laminae, shell-penetrating canals and other features above
the scale of crystal organization are not considered as contributing to
the mineralogical microstructure and are coded separately.

Inapplicable in taxa with a non-mineralized shell.

\hypertarget{Bactrotheca-coding-132}{}
\emph{Bactrotheca}: Taxon known only from moulds \citep{Valent2012}.

\hypertarget{Botsfordia-coding-132}{}
\emph{Botsfordia}: ``Composed of a thin primary layer and a laminate
secondary shell exhibiting baculate shell structure'' -- Skovsted \&
Holmer \citeyearpar{Skovsted2005EarlyCambrian}, with reference to
\citet{Skovsted2003EarlyCambrian}.

\hypertarget{Clupeafumosus_socialis-coding-132}{}
\emph{Clupeafumosus socialis}: General acrotretid structure taken from
Zhang \emph{et al}. \citeyearpar{Zhang2016Epithelialcell}.

\hypertarget{Cupitheca_holocyclata-coding-132}{}
\emph{Cupitheca holocyclata}: Inner and outer layers
\citep{Vendrasco2017}.

\hypertarget{Eoobolus-coding-132}{}
\emph{Eoobolus}: ``\emph{Eoobolus} shells exhibit the general
characteristics of modern linguliform shells, i.e.~they were composed of
alternating sets of organic and apatite-rich layers that were separated
by thin sheets of recalcitrant organic layers.'' --
\citet{Balthasar2007Anearly}.

\hypertarget{Halkieria_evangelista-coding-132}{}
\emph{Halkieria evangelista}: Single layer of fibrous aragonite
\citep{Porter2008}.

\hypertarget{Maxilites-coding-132}{}
\emph{Maxilites}: Coded following helen microstructure described in the
similar material of `\emph{Hyolithes}' \citep{MartiMus2007}.

\hypertarget{Mickwitzia_muralensis-coding-132}{}
\emph{Mickwitzia muralensis}: ``the shell structure of \emph{Mickwitzia}
{[}\ldots{}{]} is closely similar to the columnar shell of linguliform
acrotretoid brachiopods as well as to the linguloid
\emph{Lingulellotreta}, in that it has slender columns in the laminar
succession'' -- \citet{Williams2007Supplement}.

\hypertarget{Micrina-coding-132}{}
\emph{Micrina}: Identical to \emph{Mickwitzia} and more derived
linguliforms \citep{Holmer2011Firstrecord}.

\hypertarget{Mummpikia_nuda-coding-132}{}
\emph{Mummpikia nuda}: Balthasar \citeyearpar{Balthasar2008iMummpikia}
considers the single mineralogical layer, which comprises phosphatic
rods and laminae, to characterize Obolellata.

\hypertarget{Namacalathus-coding-132}{}
\emph{Namacalathus}: \emph{Namacalathus} exhibits three layers, none of
which have any obvious correspondence with those of brachiopods.

\hypertarget{Paramicrocornus-coding-132}{}
\emph{Paramicrocornus}: ``The shells of both skeletal components show
very similar structures and are composed of two layers'' --
\citet{Zhang2018Ahyolithid}.

\hypertarget{Siphonobolus_priscus-coding-132}{}
\emph{Siphonobolus priscus}: ``Orthodoxly secreted primary and secondary
layers'' -- \citet{Williams2004Chemicostructure}.

\hypertarget{Tonicella-coding-132}{}
\emph{Tonicella}: From periostracum inwards, Chiton bears three
microstructural layers: fine-grained, nacreous, and regular crossed
lamellar.

\subsection*{{[}133{]} Microstructure:
Format}\label{microstructure-format}
\addcontentsline{toc}{subsection}{{[}133{]} Microstructure: Format}

\includegraphics{Brachiopod_phylogeny_files/figure-latex/character-mapping-133.pdf}

\begin{quote}
\textbf{Character 133: Sclerites: Composition: Microstructure: Format}

1: Laminated\\
2: Fibrous bundles\\
3: Nacreous / crossed lamellar\\
Transformational character.
\end{quote}

Hyolith conchs comprise two mineralized layers of fibrous bundles.
Bundles measure 5--15 μm across; their constituent fibres are each
0.1--1.0 µm wide. In the inner layer, the fibres are transverse; in the
outer layer, the bundles are inclined towards the umbo, becoming
longitudinal on the outermost margin.

Stratiform laminae, shell-penetrating canals and other features above
the scale of crystal organization are not considered as contributing to
the mineralogical microstructure and are coded separately.

The pervasive (not just superficial) polygonal structures in
\emph{Paterimitra} are distinct, and characterize \emph{Askepasma},
\emph{Salanygolina}, \emph{Eccentrotheca} and \emph{Paterimitra}
\citep{Larsson2014iPaterimitra}

Williams \emph{et al}.
\citeyearpar{Williams2000LinguliformeaCraniiformea} identify
cross-bladed laminae as diagnostic of Strophomenata, with the exception
of some older groups that contain fibres or laminar laths.

\hypertarget{Antigonambonites_planus-coding-133}{}
\emph{Antigonambonites planus}: Tabular laminae, not fibrous as
previously thought \citep{Madison2017Laminarshell}.

\hypertarget{Askepasma_toddense-coding-133}{}
\emph{Askepasma toddense}: Lamination present
\citep{Balthasar2009Homologousskeletal}, with imprints of presumed
mantle cells \citep[following][appendix 2]{Williams1998Thediversity}.

\hypertarget{Botsfordia-coding-133}{}
\emph{Botsfordia}: ``Composed of a thin primary layer and a laminate
secondary shell exhibiting baculate shell structure'' -- Skovsted \&
Holmer \citeyearpar{Skovsted2005EarlyCambrian}, with reference to
\citet{Skovsted2003EarlyCambrian}.

\citet[appendix 2]{Williams1998Thediversity} code lamination as present,
with no imprints of presumed mantle cells.

\hypertarget{Coolinia_pecten-coding-133}{}
\emph{Coolinia pecten}: \citet{Dewing2004}.

\hypertarget{Craniops-coding-133}{}
\emph{Craniops}: \citep[fig. 249.1]{Williams1997Introduction}.

\hypertarget{Cupitheca_holocyclata-coding-133}{}
\emph{Cupitheca holocyclata}: Fibrous bundles \citep{Vendrasco2017}.

\hypertarget{Eccentrotheca-coding-133}{}
\emph{Eccentrotheca}, \emph{Paterimitra}: Laminated
\citep{Balthasar2009Homologousskeletal}.

\hypertarget{Eoobolus-coding-133}{}
\emph{Micromitra}, \emph{Eoobolus}: Lamination present, with no imprints
of presumed mantle cells \citep[following][appendix
2]{Williams1998Thediversity}.

\hypertarget{Gasconsia-coding-133}{}
\emph{Gasconsia}: Laminated relict shell structure visible, indicating
original constitution from ``sheet-like laminae''
\citep{Hanken1985Thetaxonomy}.

\hypertarget{Glyptoria-coding-133}{}
\emph{Glyptoria}, \emph{Kutorgina chengjiangensis}: Lamination absent
\citep[following][appendix 2]{Williams1998Thediversity}.

\hypertarget{Lingula-coding-133}{}
\emph{Lingula}: Lingulid laminae are thicker than those of tommotiids or
paterinids, but construed as homologous
\citep{Balthasar2009Homologousskeletal}.

\hypertarget{Maxilites-coding-133}{}
\emph{Maxilites}: Coded following helen microstructure described in the
similar material of `\emph{Hyolithes}' \citep{MartiMus2007}.

\hypertarget{Mickwitzia_muralensis-coding-133}{}
\emph{Mickwitzia muralensis}: Alternation of layers
\citep{Balthasar2004Shellstructure}.

\hypertarget{Micrina-coding-133}{}
\emph{Micrina}: \emph{Micrina} exhibits polygonal imprints on the
internal surfaces of successive second-order laminae, suggesting the
existence of a polygonal organization of these layers
\citep{Balthasar2009Homologousskeletal}.

\hypertarget{Mummpikia_nuda-coding-133}{}
\emph{Mummpikia nuda}: Balthasar \citeyearpar{Balthasar2008iMummpikia}
considers the single mineralogical layer, which comprises phosphatic
rods and laminae, to characterize Obolellata.

\hypertarget{Namacalathus-coding-133}{}
\emph{Namacalathus}: The inner and outer layer are foliated. The
columnar inflections lack canals, and as such we do not consider them to
bear any obvious homology with the hollow pillars of tommotiids and
certain brachiopods, their superficial similarity to strophomenid
pseudopunctae notwithstanding.

\hypertarget{Novocrania-coding-133}{}
\emph{Novocrania}: Laminar secondary layer \citep{Parkinson2005}.

\hypertarget{Orthis-coding-133}{}
\emph{Orthis}: Orithidina have impunctate shells with a fibrous
secondary layer \citep[p.~724]{Williams2000LinguliformeaCraniiformea}.

\hypertarget{Paramicrocornus-coding-133}{}
\emph{Paramicrocornus}: Fibrous \citep{Zhang2018Ahyolithid}.

\hypertarget{Salanygolina-coding-133}{}
\emph{Salanygolina}: ``Apatitic crystals of varying shapes and
dimensions'' -- \citet{Holmer2009Theenigmatic}.

\hypertarget{Siphonobolus_priscus-coding-133}{}
\emph{Siphonobolus priscus}: Cleaved stratified platy laminae, best
preserved in canal walls \citep{Williams2004Chemicostructure}.

\hypertarget{Terebratulina-coding-133}{}
\emph{Terebratulina}: \citet{Parkinson2005}.

\hypertarget{Tomteluva_perturbata-coding-133}{}
\emph{Tomteluva perturbata}: No original structural details evident
\citep{Streng2016Anew}.

\section{Sclerites: Structure}\label{sclerites-structure}

\subsection*{{[}134{]} Stratiform lamellae expressed at
surface}\label{stratiform-lamellae-expressed-at-surface}
\addcontentsline{toc}{subsection}{{[}134{]} Stratiform lamellae
expressed at surface}

\includegraphics{Brachiopod_phylogeny_files/figure-latex/character-mapping-134.pdf}

\begin{quote}
\textbf{Character 134: Sclerites: Structure: Stratiform lamellae
expressed at surface}

1: Lamellae not expressed at surface\\
2: Lamellae correspond to external shell ornament\\
Transformational character.
\end{quote}

In tommotiids, the shell simply comprises a stack of stratiform
lamellae, each corresponding to a circumferential rib at the shell
surface. This is particularly apparent in \emph{Dailyatia}
\citep{Skovsted2015Theearly} and \emph{Paterimitra}
\citep{Larsson2014iPaterimitra}.

\hypertarget{Askepasma_toddense-coding-134}{}
\emph{Askepasma toddense}: \citet{Topper2013Theoldest}.

\hypertarget{Coolinia_pecten-coding-134}{}
\emph{Coolinia pecten}: \citet{Dewing2004}.

\hypertarget{Dailyatia-coding-134}{}
\emph{Dailyatia}: Each lamina corresponds to a ridge on the surface
\citep{Skovsted2015Theearly}.

\hypertarget{Eoobolus-coding-134}{}
\emph{Eoobolus}: \citet{Balthasar2007Anearly}.

\hypertarget{Maxilites-coding-134}{}
\emph{Maxilites}: Coded following helen microstructure described in the
similar material of `\emph{Hyolithes}' \citep{MartiMus2007}.

\hypertarget{Novocrania-coding-134}{}
\emph{Novocrania}: \citet{Parkinson2005}.

\hypertarget{Paterimitra-coding-134}{}
\emph{Paterimitra}: Particularly apparent
\citep{Larsson2014iPaterimitra}.

\hypertarget{Pelagodiscus_atlanticus-coding-134}{}
\emph{Pelagodiscus atlanticus}: \citet{Williams1998Chemicostructural}.

\hypertarget{Salanygolina-coding-134}{}
\emph{Salanygolina}: Laminae seem to terminate at superficial ridges
\citep{Holmer2009Theenigmatic}.

\hypertarget{Siphonobolus_priscus-coding-134}{}
\emph{Siphonobolus priscus}: \citep{Williams2004Chemicostructure}.

\subsection*{{[}135{]} Stratiform laminae
separated}\label{stratiform-laminae-separated}
\addcontentsline{toc}{subsection}{{[}135{]} Stratiform laminae
separated}

\includegraphics{Brachiopod_phylogeny_files/figure-latex/character-mapping-135.pdf}

\begin{quote}
\textbf{Character 135: Sclerites: Structure: Stratiform laminae
separated}

1: Contiguous stratified layer\\
2: Laminae separated by organic layers or voids\\
Transformational character.
\end{quote}

Laminae within, for example, \emph{Salanygolina} are separated by voids
that may originally have contained organic material
\citep[e.g.][]{Holmer2009Theenigmatic}. In contrast, tommotiids and
paterinids exhibit stratification without voids, perhaps representing
periodic fluctuations in phosphate availability
\citep{Balthasar2009Homologousskeletal}.

\hypertarget{Askepasma_toddense-coding-135}{}
\emph{Askepasma toddense}, \emph{Eccentrotheca}: Contiguous
\citep{Balthasar2009Homologousskeletal}.

\hypertarget{Botsfordia-coding-135}{}
\emph{Botsfordia}: \citet{Skovsted2003EarlyCambrian}.

\hypertarget{Clupeafumosus_socialis-coding-135}{}
\emph{Clupeafumosus socialis}: Acrotretid laminae are separated by
column-supported voids.

\hypertarget{Coolinia_pecten-coding-135}{}
\emph{Coolinia pecten}: \citet{Dewing2004}.

\hypertarget{Craniops-coding-135}{}
\emph{Craniops}: \citep[fig. 249.1]{Williams1997Introduction}.

\hypertarget{Dailyatia-coding-135}{}
\emph{Dailyatia}: Contiguous \citep[figs. 54--55]{Skovsted2015Theearly}.

\hypertarget{Eoobolus-coding-135}{}
\emph{Eoobolus}: Essentially contiguous, notwithstanding occasional
laminae of calcite/chlorite \citep{Balthasar2007Anearly}.

\hypertarget{Gasconsia-coding-135}{}
\emph{Gasconsia}: \citet{Hanken1985Thetaxonomy}.

\hypertarget{Maxilites-coding-135}{}
\emph{Maxilites}: Coded following helen microstructure described in the
similar material of `\emph{Hyolithes}' \citep{MartiMus2007}.

\hypertarget{Mickwitzia_muralensis-coding-135}{}
\emph{Mickwitzia muralensis}: \citep{Balthasar2004Shellstructure}.

\hypertarget{Micrina-coding-135}{}
\emph{Micrina}: Matrix filled chambers \citep[fig.
DR1]{Balthasar2009Homologousskeletal}.

\hypertarget{Mummpikia_nuda-coding-135}{}
\emph{Mummpikia nuda}: Seemingly contiguous
\citep{Balthasar2008iMummpikia}.

\hypertarget{Novocrania-coding-135}{}
\emph{Novocrania}: Not in \emph{Novocrania}, though possibly present in
\emph{Neoancistrocrania} \citep{Parkinson2005}.

\hypertarget{Paterimitra-coding-135}{}
\emph{Paterimitra}: Contiguous in some regions, but large internal
cavities present {[}\citet{Balthasar2009Homologousskeletal}; see fig.
DR3{]}.

\hypertarget{Pelagodiscus_atlanticus-coding-135}{}
\emph{Pelagodiscus atlanticus}: \citet{Williams1998Chemicostructural}.

\hypertarget{Salanygolina-coding-135}{}
\emph{Salanygolina}: Polygonally-filled voids
\citep{Holmer2009Theenigmatic}.

\hypertarget{Siphonobolus_priscus-coding-135}{}
\emph{Siphonobolus priscus}: \citep{Williams2004Chemicostructure}.

\subsection*{{[}136{]} Stratiform laminae with polygonal
ornament}\label{stratiform-laminae-with-polygonal-ornament}
\addcontentsline{toc}{subsection}{{[}136{]} Stratiform laminae with
polygonal ornament}

\includegraphics{Brachiopod_phylogeny_files/figure-latex/character-mapping-136.pdf}

\begin{quote}
\textbf{Character 136: Sclerites: Structure: Stratiform laminae with
polygonal ornament}

1: Absent\\
2: Present\\
Transformational character.
\end{quote}

See character 37 in \citet{Williams1998Thediversity}.\\
``A distinct primary layer {[}\ldots{}{]} is characterized by a
polygonal ornament that is mineralized from the polygon walls inward,
while the rest of the shell and/or sclerite is secreted by basal
accretion'' -- \citet{Balthasar2009Homologousskeletal}. Distinguished
from epithelial cell moulds in lingulids, which do not form an integral
part of the shell structure \citep{Balthasar2009Homologousskeletal}.\\
Treated as transformational as ancestral condition is ambiguous.

\hypertarget{Askepasma_toddense-coding-136}{}
\emph{Askepasma toddense}, \emph{Eccentrotheca}, \emph{Paterimitra}:
Present \citep{Balthasar2009Homologousskeletal}.

\hypertarget{Botsfordia-coding-136}{}
\emph{Micromitra}, \emph{Botsfordia}, \emph{Eoobolus}: Lamination
present, with no imprints of presumed mantle cells
\citep[following][appendix 2]{Williams1998Thediversity}.

\hypertarget{Clupeafumosus_socialis-coding-136}{}
\emph{Clupeafumosus socialis}: Epithelial cell moulds present on inner
shell layer in acrotretids \citep{Zhang2016Epithelialcell}.

\hypertarget{Coolinia_pecten-coding-136}{}
\emph{Coolinia pecten}: \citet{Dewing2004}.

\hypertarget{Craniops-coding-136}{}
\emph{Craniops}: \citep[fig. 249.1]{Williams1997Introduction}.

\hypertarget{Dailyatia-coding-136}{}
\emph{Dailyatia}: Polygonal structures on external surface of sclerites
only \citep{Skovsted2015Theearly}; not reported from other camenellans
\citep{Balthasar2009Homologousskeletal}.

\hypertarget{Gasconsia-coding-136}{}
\emph{Gasconsia}: \citet{Hanken1985Thetaxonomy}.

\hypertarget{Lingula-coding-136}{}
\emph{Lingula}: Absent in \emph{Lingula}, though potentially equivalent,
if superficial \citep{Balthasar2009Homologousskeletal}, features adorn
\emph{Lingulella} \citep{Curry1983}.

\hypertarget{Maxilites-coding-136}{}
\emph{Maxilites}: Coded following helen microstructure described in the
similar material of `\emph{Hyolithes}' \citep{MartiMus2007}.

\hypertarget{Micrina-coding-136}{}
\emph{Micrina}: \emph{Micrina} exhibits polygonal imprints on the
internal surfaces of successive second-order laminae, suggesting the
existence of a polygonal organization of these layers
\citep{Balthasar2009Homologousskeletal}.

\hypertarget{Mummpikia_nuda-coding-136}{}
\emph{Mummpikia nuda}: It is conceivable that the rods
\citep{Balthasar2008iMummpikia} correspond to the polygonal ornament
observed in other taxa; coded as ambiguous.

\hypertarget{Novocrania-coding-136}{}
\emph{Novocrania}: \citet{Parkinson2005}.

\hypertarget{Pelagodiscus_atlanticus-coding-136}{}
\emph{Pelagodiscus atlanticus}: \citet{Williams1998Chemicostructural}.

\hypertarget{Salanygolina-coding-136}{}
\emph{Salanygolina}: Prominently present \citep{Holmer2009Theenigmatic}.

\hypertarget{Siphonobolus_priscus-coding-136}{}
\emph{Siphonobolus priscus}: \citep{Williams2004Chemicostructure}.

\subsection*{{[}137{]} Canals}\label{canals}
\addcontentsline{toc}{subsection}{{[}137{]} Canals}

\includegraphics{Brachiopod_phylogeny_files/figure-latex/character-mapping-137.pdf}

\begin{quote}
\textbf{Character 137: Sclerites: Structure: Canals}

0: Absent\\
1: Present\\
Neomorphic character.
\end{quote}

A caniculate microstructure occurs in lingulids; canals are narrower
(\textless{} 1 µm) than punctae, may branch, and do not fully penetrate
the shell, terminating just within the boundaries of a microstructural
layer. See \citet{Williams1997Introduction}, p303ff, and
\citet{Balthasar2008iMummpikia}, p273, for discussion.

Tubules described in hyoliths by Kouchinsky
\citeyearpar{Kouchinsky2000Skeletalmicrostructures} measure around 10 µm
in diameter, making them an order of magnitude wider than lingulid
canals.

This said, Balthasar \citeyearpar{Balthasar2008iMummpikia} considers the
rod-like tubules within the columnar shell microstructure of
\emph{Mickwitzia} cf.~occidens \citep[1--3 µm
wide,][]{Skovsted2003EarlyCambrian}, acrotretides \citep[1 µm wide,
see][\citet{Zhang2016Epithelialcell}]{Holmer1989MiddleOrdovician} and
lingulellotretids \citep[100 nm wide,][]{Cusack1999Chemicostructural} as
equivalent to lingulid canals.

\emph{Micrina} exhibits both punctae and canals
\citep{Harper2017Brachiopodsorigin}, challenging Carlson's contention
\citep[in][]{Williams2007Supplement} that the structures are potentially
homologous as shell perforations.

\hypertarget{Bactrotheca-coding-137}{}
\emph{Bactrotheca}: Taxon known only from moulds \citep{Valent2012}.

\hypertarget{Botsfordia-coding-137}{}
\emph{Botsfordia}: Not evident in section presented by Skovsted \&
Holmer \citeyearpar{Skovsted2003EarlyCambrian}.

\hypertarget{Clupeafumosus_socialis-coding-137}{}
\emph{Clupeafumosus socialis}: Acrotretid laminae bear characteristic
columns \citep[e.g.][]{Zhang2016Epithelialcell}.

Balthasar \citeyearpar{Balthasar2008iMummpikia} considers these columns
as homologous with tubules within the columnar shell microstructure
\emph{Mummpikia}, \emph{Mickwitzia} and lingulellotretids.

\hypertarget{Cupitheca_holocyclata-coding-137}{}
\emph{Cupitheca holocyclata}: Orthogonal tubules \citep{Vendrasco2017}.

\hypertarget{Halkieria_evangelista-coding-137}{}
\emph{Halkieria evangelista}: The chambers in halkieriid sclerites do
not correspond in morphology or dimension to the brachiopod-like canals
documented by this character.

\hypertarget{Longtancunella_chengjiangensis-coding-137}{}
\emph{Longtancunella chengjiangensis}: Preservational resolution not
sufficient to evaluate.

\hypertarget{Maxilites-coding-137}{}
\emph{Maxilites}: Coded following helen microstructure described in the
similar material of `\emph{Hyolithes}' \citep{MartiMus2007}.

\hypertarget{Mickwitzia_muralensis-coding-137}{}
\emph{Mickwitzia muralensis}: Coded as present to reflect similarity of
columnar microstructure remarked on by, among others, Balthasar
\citeyearpar{Balthasar2008iMummpikia}; Williams \emph{et al}.
\citeyearpar{Williams2007Supplement}; Skovsted \& Holmer
\citeyearpar{Skovsted2003EarlyCambrian}.

\hypertarget{Micrina-coding-137}{}
\emph{Micrina}: Acrotretid laminae bear characteristic columns
\citep[e.g.][]{Zhang2016Epithelialcell}; a similar fabric has been
reported, and assumed homologous, in \emph{Micrina}
\citep{Butler2012ConstructingCambrian}.

A similar columnar shell microstructure also occurs in the closely
related \emph{Mickwitzia} \citep{Balthasar2008iMummpikia}.

\hypertarget{Namacalathus-coding-137}{}
\emph{Namacalathus}: Canal-like structures have been reported in
\emph{Namacalathus} \citep{Zhuravlev2015Ediacaranskeletal}, and
interpreted as evidence for a Lophophorate affinity. Though the
structures are not necessarily directly equivalent, the hypothesis of
homology is followed here.

\hypertarget{Paramicrocornus-coding-137}{}
\emph{Paramicrocornus}: Columns, replicated in phosphate and present in
both layers of the shell, have been interpreted as potential homologues
to acrotretid columns \citep{Zhang2018Ahyolithid}.

\hypertarget{Siphonobolus_priscus-coding-137}{}
\emph{Siphonobolus priscus}: The `canals' through the shell have a
diameter of c. 20 µm \citep[text-fig. 2a]{Williams2004Chemicostructure},
falling within the definition of punctae (rather than canals) used
herein.

\hypertarget{Tonicella-coding-137}{}
\emph{Tonicella}: Aesthete canals do not fall within the definition of
this character.

\subsection*{{[}138{]} Punctae}\label{punctae}
\addcontentsline{toc}{subsection}{{[}138{]} Punctae}

\includegraphics{Brachiopod_phylogeny_files/figure-latex/character-mapping-138.pdf}

\begin{quote}
\textbf{Character 138: Sclerites: Structure: Punctae}

0: Absent\\
1: Present\\
Neomorphic character.
\end{quote}

Punctae are 10--20 µm wide canals created by multicellular extensions of
the outer epithelium. They penetrate the full depth of the shell.

Balthasar \citeyearpar{Balthasar2008iMummpikia} writes:

``Vertical shell penetrating structures, such as punctae, pseudopunctae,
extropunctae and canals, are common in many groups of brachiopods and
are distinguished based on their geometry and size
\citep{Williams1997Introduction}. Punctae are 10--20 µm wide and
represent multicellular extensions of the outer epithelium
\citep{Owen1969Thecaecum}. Pseudopunctae and extropunctae are similar in
diameter but, instead of canals, are vertical stacks of conical
deflections of individual shell layers \citep{Williams1993Roleof}. None
of these three types of vertical shell structure, all of which are
confined to calcitic-shelled brachiopods, compares with the much smaller
canals (\textless{} 1 µm in diameter) of \emph{M. nuda}. The only type
of vertical structure that fits the size and nature of the canals of the
Mural obolellids are the canals of linguliform brachiopods, which range
in width from 180 to 740 nm and are occupied by proteinaceous strands in
extant taxa
\citep{Williams1992Structureof, Williams1994Collagenouschitino, Williams1997Introduction}.
In contrast to obolellid canals, however, linguliform canals are not
known to penetrate the entire shell but terminate in organic-rich layers
\citep{Williams1997Introduction}. Based on these considerations it
would, therefore, be misleading to call obolellid shells punctate (they
are as much''punctate" as acrotretids or other linguliforms); rather
their shell structure should be called canaliculate
\citep{Williams1997Introduction}."

\hypertarget{Bactrotheca-coding-138}{}
\emph{Bactrotheca}: Taxon known only from moulds \citep{Valent2012}.

\hypertarget{Craniops-coding-138}{}
\emph{Craniops}: ``impunctate''.

\hypertarget{Haplophrentis_carinatus-coding-138}{}
\emph{Haplophrentis carinatus}: The tubules within the centre of the
bundles of hyolith shells \citep{Kouchinsky2000Skeletalmicrostructures}
are c. 10 µm wide, making them an order of magnitude larger than the
canals that characterize lingulid valves, and a similar scale to
punctae. This said, they have only been reported in a putative
allathecid, so the presence of equivalent structures in hyolithids has
never been demonstrated.

\hypertarget{Heliomedusa_orienta-coding-138}{}
\emph{Heliomedusa orienta}: `Identical' to those in \emph{Mickwitzia} --
see \citet{Williams2007Supplement}.

\hypertarget{Maxilites-coding-138}{}
\emph{Maxilites}: Coded following helen microstructure described in the
similar material of `\emph{Hyolithes}' \citep{MartiMus2007}.

\hypertarget{Mickwitzia_muralensis-coding-138}{}
\emph{Mickwitzia muralensis}: Coded as present to reflect that the
chambers contained setae; following Carlson in
\citet{Williams2007Supplement}, the punctae may or may not be homologous
as punctae, but are likely homologous as shell perforations; both these
perforations and those of \emph{Micrina} were associated with setae,
even if their equivalence bay be with juvenile vs adult setal structures
in modern brachiopods \citep[p.~397]{Balthasar2004Shellstructure}.

\hypertarget{Mummpikia_nuda-coding-138}{}
\emph{Mummpikia nuda}: ``Vertical shell penetrating structures, such as
punctae, pseudopunctae, extropunctae and canals, are common in many
groups of brachiopods and are distinguished based on their geometry and
size \citep{Williams1997Introduction}. Punctae are 10--20 µm wide and
represent multicellular extensions of the outer epithelium
\citep{Owen1969Thecaecum}. {[}\ldots{}{]} None of these three types of
vertical shell structure, all of which are confined to calcitic-shelled
brachiopods, compares with the much smaller canals (\textless{} 1 µm in
diameter) of \emph{M. nuda}. The only type of vertical structure that
fits the size and nature of the canals of the Mural obolellids are the
canals of linguliform brachiopods, which range in width from 180 to 740
nm and are occupied by proteinaceous strands in extant taxa
\citetext{\citealp[1994]{Williams1992Structureof}; \citealp{Williams1997Introduction}}.''
-- \citet{Balthasar2008iMummpikia}.

\hypertarget{Paramicrocornus-coding-138}{}
\emph{Paramicrocornus}: Not evident despite excellent preservation;
interpreted as absent \citep{Zhang2018Ahyolithid}.

\hypertarget{Siphonobolus_priscus-coding-138}{}
\emph{Siphonobolus priscus}: The `canals' through the shell have a
diameter of c. 20 µm \citep[text-fig. 2a]{Williams2004Chemicostructure},
falling within the definition of punctae used herein.

\hypertarget{Terebratulina-coding-138}{}
\emph{Terebratulina}: Endopunctae are relatively large canals, diameter
vary greatly from 5--20 µm.

\subsection*{{[}139{]} Pseudopunctae}\label{pseudopunctae}
\addcontentsline{toc}{subsection}{{[}139{]} Pseudopunctae}

\includegraphics{Brachiopod_phylogeny_files/figure-latex/character-mapping-139.pdf}

\begin{quote}
\textbf{Character 139: Sclerites: Structure: Pseudopunctae}

0: Absent\\
1: Present\\
Neomorphic character.
\end{quote}

Pseudopunctae are not punctae, but deflections of shell laminae. They
characterise Strophomenata in particular.

\hypertarget{Antigonambonites_planus-coding-139}{}
\emph{Antigonambonites planus}, \emph{Glyptoria}, \emph{Nisusia
sulcata}: Scored absent in data matrix of Benedetto
\citeyearpar{Benedetto2009iChaniella}.

\hypertarget{Bactrotheca-coding-139}{}
\emph{Bactrotheca}: Taxon known only from moulds \citep{Valent2012}.

\hypertarget{Maxilites-coding-139}{}
\emph{Maxilites}: Coded following helen microstructure described in the
similar material of `\emph{Hyolithes}' \citep{MartiMus2007}.

\hypertarget{Orthis-coding-139}{}
\emph{Orthis}: Scored absent (in \emph{Eoorthis}) in data matrix of
Benedetto \citeyearpar{Benedetto2009iChaniella}.

\hypertarget{Paramicrocornus-coding-139}{}
\emph{Paramicrocornus}: Absent \citep{Zhang2018Ahyolithid}.

\subsection*{{[}140{]} External polygonal
ornament}\label{external-polygonal-ornament}
\addcontentsline{toc}{subsection}{{[}140{]} External polygonal ornament}

\includegraphics{Brachiopod_phylogeny_files/figure-latex/character-mapping-140.pdf}

\begin{quote}
\textbf{Character 140: Sclerites: Structure: External polygonal
ornament}

0: Absent\\
1: Present\\
Neomorphic character.
\end{quote}

Regular polygonal compartments, around 10 µm in diameter, characterise
\emph{Paterimitra}. Walls between compartments have the cross-section of
an anvil. An external polygonal structure (possible imprints of
epithelial tissue) occurs in \emph{Dailyatia}, but it is a surface
pattern, which is different from the polygonal prisms in the body wall
of other paterinid-like groups.

\hypertarget{Clupeafumosus_socialis-coding-140}{}
\emph{Clupeafumosus socialis}: The polygonal ornament reported in
acrotretids by Zhang \emph{et al}. \citeyearpar{Zhang2016Epithelialcell}
is on the internal surface of the shell.

\section{Gametes}\label{gametes}

\subsection*{{[}141{]} Gonocoel}\label{gonocoel}
\addcontentsline{toc}{subsection}{{[}141{]} Gonocoel}

\includegraphics{Brachiopod_phylogeny_files/figure-latex/character-mapping-141.pdf}

\begin{quote}
\textbf{Character 141: Gametes: Gonocoel}

0: Absent\\
1: Retroperineal gonads\\
Neomorphic character.
\end{quote}

Character 27 in \citet{Haszprunar1996}.

\subsection*{{[}142{]} Ovary wall saccular}\label{ovary-wall-saccular}
\addcontentsline{toc}{subsection}{{[}142{]} Ovary wall saccular}

\includegraphics{Brachiopod_phylogeny_files/figure-latex/character-mapping-142.pdf}

\begin{quote}
\textbf{Character 142: Gametes: Ovary wall saccular}

0: Plain\\
1: Saccular\\
Neomorphic character.
\end{quote}

After character 31 in \citet{Haszprunar1996}.

\subsection*{{[}143{]} Testis wall saccular}\label{testis-wall-saccular}
\addcontentsline{toc}{subsection}{{[}143{]} Testis wall saccular}

\begin{quote}
\textbf{Character 143: Gametes: Testis wall saccular}

0: Plain\\
1: Saccular\\
Neomorphic character.
\end{quote}

After character 31 in \citet{Haszprunar1996}.

\subsection*{{[}144{]} Asexual reproduction}\label{asexual-reproduction}
\addcontentsline{toc}{subsection}{{[}144{]} Asexual reproduction}

\includegraphics{Brachiopod_phylogeny_files/figure-latex/character-mapping-143.pdf}

\begin{quote}
\textbf{Character 144: Gametes: Asexual reproduction}

0: Never exhibited\\
1: Frequently exhibited\\
Neomorphic character.
\end{quote}

After character 30 in \citet{Haszprunar1996}.

\hypertarget{Namacalathus-coding-144}{}
\emph{Namacalathus}: Budding well documented
\citep[e.g.][]{Zhuravlev2015Ediacaranskeletal}.

\subsection*{{[}145{]} Sexes}\label{sexes}
\addcontentsline{toc}{subsection}{{[}145{]} Sexes}

\includegraphics{Brachiopod_phylogeny_files/figure-latex/character-mapping-144.pdf}

\begin{quote}
\textbf{Character 145: Gametes: Sexes}

1: Gonochoristic\\
2: Hermaphroditic\\
Transformational character.
\end{quote}

After characters 1.61 and 2.54 in \citet{SPS1996}.

\hypertarget{Amathia-coding-145}{}
\emph{Amathia}: Hermaphroditic \citep{Reed1988}.

\subsection*{{[}146{]} Fertilization}\label{fertilization}
\addcontentsline{toc}{subsection}{{[}146{]} Fertilization}

\includegraphics{Brachiopod_phylogeny_files/figure-latex/character-mapping-145.pdf}

\begin{quote}
\textbf{Character 146: Gametes: Fertilization}

1: External\\
2: Internal\\
Transformational character.
\end{quote}

After character 62 in \citet{Haszprunar2000}.

\hypertarget{Amathia-coding-146}{}
\emph{Amathia}: Brood pouches in abandoned lophophore.

\section{Gametes: Egg}\label{gametes-egg}

\subsection*{{[}147{]} Size}\label{size}
\addcontentsline{toc}{subsection}{{[}147{]} Size}

\includegraphics{Brachiopod_phylogeny_files/figure-latex/character-mapping-146.pdf}

\begin{quote}
\textbf{Character 147: Gametes: Egg: Size}

1: Small: \textless{} 100 um, little yolk\\
2: Large: \textgreater{} 110 um, much yolk\\
Transformational character.
\end{quote}

Following Carlson \citeyearpar{Carlson1995Phylogeneticrelationships},
character 7. This character is only possible to code in extant taxa. It
is not considered independent of Carlson's character 11, number of
gametes released per spawning, as it is possible to produce more small
eggs than large eggs -- thus this latter character is not reproduced in
the present study. The same goes for Carlson's character 12, gamete
dispersal mode; brooders will tend to brood large eggs.

\hypertarget{Amathia-coding-147}{}
\emph{Amathia}: ``Mature eggs commonly measure about 200 µm in
diameter'' \citep{Franzen1977}; the larva is a similar size
\citep{Reed1982}.

\hypertarget{Dentalium-coding-147}{}
\emph{Dentalium}: Egg size can vary from 60--200 µm in scaphopods, but
in \emph{Dentalium} the eggs are large \citep{DufresneDube1983}.

\hypertarget{Flustra-coding-147}{}
\emph{Flustra}: ``Mature eggs commonly measure about 200 µm in
diameter'' -- \citet{Franzen1977}.

\hypertarget{Lingula-coding-147}{}
\emph{Novocrania}, \emph{Pelagodiscus atlanticus}, \emph{Lingula},
\emph{Terebratulina}: Following coding for class in Carlson
\citeyearpar{Carlson1995Phylogeneticrelationships} appendix 1, character
7.

\hypertarget{Loxosomella-coding-147}{}
\emph{Loxosomella}: Tiny \citep{Nielsen1966}.

\hypertarget{Phoronis-coding-147}{}
\emph{Phoronis}: \emph{Phoronis} has planktotrophic larvae. indicating a
small egg size \citep{Ruppert2004Invertebratezoology}. Carlson
\citeyearpar{Carlson1995Phylogeneticrelationships} codes phoronids as
polymorphic, as some members of the phylum have eggs of each size.

\hypertarget{Serpula-coding-147}{}
\emph{Serpula}: c. 50 µm in \emph{Hydroides} \citep{Miles2007}.

\hypertarget{Siphonobolus_priscus-coding-147}{}
\emph{Siphonobolus priscus}: ``the ventral brephic valve {[}was{]} 50 µm
across, {[}which{]} is close to the known lower limit of the brachiopod
egg size'' -- \citet{Popov2009Earlyontogeny}.

\hypertarget{Sipunculus-coding-147}{}
\emph{Sipunculus}: c. 200 µm in diameter \citep{Rice1988}.

\hypertarget{Tonicella-coding-147}{}
\emph{Tonicella}: \citet{BucklandNicks1988}.

\subsection*{{[}148{]} Protective membrane}\label{protective-membrane}
\addcontentsline{toc}{subsection}{{[}148{]} Protective membrane}

\includegraphics{Brachiopod_phylogeny_files/figure-latex/character-mapping-147.pdf}

\begin{quote}
\textbf{Character 148: Gametes: Egg: Protective membrane}

0: Absent\\
1: Present\\
Neomorphic character.
\end{quote}

After character 4.69 in \citet{SPS1996}.

\hypertarget{Amathia-coding-148}{}
\emph{Flustra}, \emph{Amathia}: ``Eggs have a loose consistency and are
capable of changing form'' \citep{Franzen1977}.

\hypertarget{Phoronis-coding-148}{}
\emph{Phoronis}: Eggs ``are surrounded by a delicate fertilization
membrane'' \citep{Pennerstorfer2012}.

\subsection*{{[}149{]} Site of maturation}\label{site-of-maturation}
\addcontentsline{toc}{subsection}{{[}149{]} Site of maturation}

\includegraphics{Brachiopod_phylogeny_files/figure-latex/character-mapping-148.pdf}

\begin{quote}
\textbf{Character 149: Gametes: Egg: Site of maturation}

1: Body cavity\\
2: Mantle canals\\
3: Ovicell\\
Transformational character.
\end{quote}

After Carlson \citeyearpar{Carlson1995Phylogeneticrelationships},
character 9. Only possible to code in extant taxa.

\hypertarget{Amathia-coding-149}{}
\emph{Flustra}, \emph{Amathia}: Ovicell \citep{Franzen1977}.

\hypertarget{Lingula-coding-149}{}
\emph{Novocrania}, \emph{Pelagodiscus atlanticus}, \emph{Lingula},
\emph{Terebratulina}: Following Hodgson \& Reunov
\citeyearpar{Hodgson1994Ultrastructureof}.

\hypertarget{Phoronis-coding-149}{}
\emph{Phoronis}: Following coding for class in Carlson
\citeyearpar{Carlson1995Phylogeneticrelationships} Appendix 1, character
9.

\section{Gametes: Spermatozoa}\label{gametes-spermatozoa}

\subsection*{{[}150{]} Nucleus: Shape}\label{nucleus-shape}
\addcontentsline{toc}{subsection}{{[}150{]} Nucleus: Shape}

\includegraphics{Brachiopod_phylogeny_files/figure-latex/character-mapping-149.pdf}

\begin{quote}
\textbf{Character 150: Gametes: Spermatozoa: Nucleus: Shape}

0: Equant: length comparable to width\\
1: Elongate: length exaggerated relative to width\\
Neomorphic character.
\end{quote}

After character 41 in \citet{Ponder1997}.

\hypertarget{Amathia-coding-150}{}
\emph{Flustra}, \emph{Amathia}: Elongate \citep{Franzen1981}.

\hypertarget{Dentalium-coding-150}{}
\emph{Dentalium}: Elongate nucleus, 4--6 times longer than wide
\citep{DufresneDube1983}.

\hypertarget{Loxosomella-coding-150}{}
\emph{Loxosomella}: Elongate in \emph{Loxosoma} \citep{Franzen2000}.

\hypertarget{Serpula-coding-150}{}
\emph{Serpula}: \citet{Gherardi2011}.

\hypertarget{Tonicella-coding-150}{}
\emph{Tonicella}: Profoundly elongated nucleus
\citep{BucklandNicks1988}.

\subsection*{{[}151{]} Anterior nuclear
fossa}\label{anterior-nuclear-fossa}
\addcontentsline{toc}{subsection}{{[}151{]} Anterior nuclear fossa}

\includegraphics{Brachiopod_phylogeny_files/figure-latex/character-mapping-150.pdf}

\begin{quote}
\textbf{Character 151: Gametes: Spermatozoa: Anterior nuclear fossa}

0: Absent\\
1: Present\\
Neomorphic character.
\end{quote}

Following \citet{Smith2012}, after character 160 in \citet{Giribet2002}.
A fossa (latin: ditch) is a dent or impression.

\hypertarget{Amathia-coding-151}{}
\emph{Flustra}, \emph{Amathia}: Present \citep[in
\emph{Tubulipora};][]{Franzen1984}.

\hypertarget{Dentalium-coding-151}{}
\emph{Dentalium}: \citet{DufresneDube1983}.

\hypertarget{Loxosomella-coding-151}{}
\emph{Loxosomella}: Present in \emph{Loxosoma} \citep{Franzen2000}.

\hypertarget{Pelagodiscus_atlanticus-coding-151}{}
\emph{Pelagodiscus atlanticus}: Present in \emph{Discinisca}
\emph{tenuis} \citep{Hodgson1994Ultrastructureof}.

\hypertarget{Phoronis-coding-151}{}
\emph{Phoronis}: Nucleus ``almost round''
\citep{Reunov2004Ultrastructuralstudy}.

\hypertarget{Serpula-coding-151}{}
\emph{Serpula}: Absent: subacrosomal space does not impinge on nuclear
envelope \citep{Gherardi2011}.

\hypertarget{Sipunculus-coding-151}{}
\emph{Sipunculus}: Prominent in \emph{Phascolion} \citep{Rice1993}.

\hypertarget{Terebratulina-coding-151}{}
\emph{Terebratulina}: No anterior invagination
\citep{Hodgson1994Ultrastructureof}.

\hypertarget{Tonicella-coding-151}{}
\emph{Tonicella}: \citet{BucklandNicks1988}.

\subsection*{{[}152{]} Acrosome: Shape}\label{acrosome-shape}
\addcontentsline{toc}{subsection}{{[}152{]} Acrosome: Shape}

\includegraphics{Brachiopod_phylogeny_files/figure-latex/character-mapping-151.pdf}

\begin{quote}
\textbf{Character 152: Gametes: Spermatozoa: Acrosome: Shape}

1: Pear-shaped\\
2: Needle-shaped\\
3: Disc-shaped\\
4: Conical\\
Transformational character.
\end{quote}

\hypertarget{Amathia-coding-152}{}
\emph{Flustra}, \emph{Amathia}: Conical \citep[in
\emph{Tubulipora};][]{Franzen1984}.

\hypertarget{Dentalium-coding-152}{}
\emph{Dentalium}: Low conical aspect \citep{DufresneDube1983}.

\hypertarget{Lingula-coding-152}{}
\emph{Lingula}: Pear-shaped \citep{Fukumoto2003Theacrosome}.

\hypertarget{Loxosomella-coding-152}{}
\emph{Loxosomella}: Conical/cylindrical acrosome-like extension in
\emph{Loxosoma} \citep{Franzen2000}.

\hypertarget{Novocrania-coding-152}{}
\emph{Novocrania}: Needle-shaped \citep{Afzelius1978Finestructure}.

\hypertarget{Pelagodiscus_atlanticus-coding-152}{}
\emph{Pelagodiscus atlanticus}: Pear-shaped
\citep{Hodgson1994Ultrastructureof}.

\hypertarget{Phoronis-coding-152}{}
\emph{Phoronis}: Needle-shaped \citep{Reunov2004Ultrastructuralstudy}.

\hypertarget{Serpula-coding-152}{}
\emph{Serpula}: \citet{Gherardi2011}.

\hypertarget{Sipunculus-coding-152}{}
\emph{Sipunculus}: A peaked disc in \emph{Phascolion} \citep{Rice1993}.

\hypertarget{Terebratulina-coding-152}{}
\emph{Terebratulina}: Disc-shaped (in \emph{Kraussina})
\citep{Hodgson1994Ultrastructureof}.

\hypertarget{Tonicella-coding-152}{}
\emph{Tonicella}: Elongate: cylindrical to conical
\citep{BucklandNicks1988}.

\subsection*{{[}153{]} Acrosome: Differentiated
internally}\label{acrosome-differentiated-internally}
\addcontentsline{toc}{subsection}{{[}153{]} Acrosome: Differentiated
internally}

\includegraphics{Brachiopod_phylogeny_files/figure-latex/character-mapping-152.pdf}

\begin{quote}
\textbf{Character 153: Gametes: Spermatozoa: Acrosome: Differentiated
internally}

0: No internal differentiation\\
1: Acrosome differentiated internally\\
Neomorphic character.
\end{quote}

\citet{Hodgson1994Ultrastructureof} describe the \emph{Discinisca}
acrosome as having ``an electron-lucent centre and an electron-dense
outer region'', and state that this trait is characteristic of
inarticulate brachiopods.

\hypertarget{Amathia-coding-153}{}
\emph{Flustra}, \emph{Amathia}: No evidence of internal differentiation
\citep[in \emph{Tubulipora};][]{Franzen1984}.

\hypertarget{Dentalium-coding-153}{}
\emph{Dentalium}: Differentiated membrane only \citep{DufresneDube1983}.

\hypertarget{Lingula-coding-153}{}
\emph{Lingula}: Clear differentiation of marginal area
\citep{Fukumoto2003Theacrosome}.

\hypertarget{Loxosomella-coding-153}{}
\emph{Loxosomella}: Not evident in \emph{Loxosoma} \citep{Franzen2000}.

\hypertarget{Novocrania-coding-153}{}
\emph{Novocrania}: ``Along the inner and outer margins there are
periodically banded layers, and between them there is a less dense
layer'' -- \citet{Afzelius1978Finestructure}.

\hypertarget{Pelagodiscus_atlanticus-coding-153}{}
\emph{Pelagodiscus atlanticus}: Following \emph{Discinisca}
\emph{tenuis}, described in Hodgson \& Reunov
\citeyearpar{Hodgson1994Ultrastructureof}.

\hypertarget{Phoronis-coding-153}{}
\emph{Phoronis}: Acrosome-like structure; no internal division or
surrounding membrane, with possibility that much of the acrosome is
secondarily lost \citep{Reunov2004Ultrastructuralstudy}.

\hypertarget{Serpula-coding-153}{}
\emph{Serpula}: \citet{Gherardi2011}.

\hypertarget{Sipunculus-coding-153}{}
\emph{Sipunculus}: No differentiation within acrosomal vesicle
\citep{Rice1993}.

\hypertarget{Terebratulina-coding-153}{}
\emph{Terebratulina}: Following Hodgson \& Reunov
\citeyearpar{Hodgson1994Ultrastructureof}.

\hypertarget{Tonicella-coding-153}{}
\emph{Tonicella}: ``One can distinguish two components in the acrosome,
an apical and a basal granule'' -- \citet{BucklandNicks1988}.

\subsection*{{[}154{]} Acrosome: Sub-acrosomal
space}\label{acrosome-sub-acrosomal-space}
\addcontentsline{toc}{subsection}{{[}154{]} Acrosome: Sub-acrosomal
space}

\includegraphics{Brachiopod_phylogeny_files/figure-latex/character-mapping-153.pdf}

\begin{quote}
\textbf{Character 154: Gametes: Spermatozoa: Acrosome: Sub-acrosomal
space}

0: Absent\\
1: Present\\
Neomorphic character.
\end{quote}

\hypertarget{Amathia-coding-154}{}
\emph{Flustra}, \emph{Amathia}: No distinct space \citep[in
\emph{Tubulipora};][]{Franzen1984}.

\hypertarget{Dentalium-coding-154}{}
\emph{Dentalium}: \citet{DufresneDube1983}.

\hypertarget{Lingula-coding-154}{}
\emph{Lingula}: Filled with sub-acrosomal substance
\citep{Fukumoto2003Theacrosome}.

\hypertarget{Loxosomella-coding-154}{}
\emph{Loxosomella}: Present in \emph{Loxosoma} \citep{Franzen2000}.

\hypertarget{Novocrania-coding-154}{}
\emph{Novocrania}: Prominent \citep{Afzelius1978Finestructure}.

\hypertarget{Pelagodiscus_atlanticus-coding-154}{}
\emph{Pelagodiscus atlanticus}: Subacrosomal material (in
\emph{Discinisca}) but no subacrosomal space
\citep{Hodgson1994Ultrastructureof}.

\hypertarget{Phoronis-coding-154}{}
\emph{Phoronis}: The filament-like acrosome continues backwards as a
tube-like structure \citep[summarized in
\citet{Jamieson1991FishEvolution}]{Franzen1980Ultrastructureof}.

\hypertarget{Serpula-coding-154}{}
\emph{Serpula}: \citet{Gherardi2011}.

\hypertarget{Sipunculus-coding-154}{}
\emph{Sipunculus}: \citet{Rice1993}.

\hypertarget{Terebratulina-coding-154}{}
\emph{Terebratulina}: No subacrosomal material, let alone a subacrosomal
space \citep[e.g.][]{Hodgson1994Ultrastructureof}.

\hypertarget{Tonicella-coding-154}{}
\emph{Tonicella}: Not evident \citep{BucklandNicks1988}.

\subsection*{{[}155{]} Mid-piece}\label{mid-piece}
\addcontentsline{toc}{subsection}{{[}155{]} Mid-piece}

\includegraphics{Brachiopod_phylogeny_files/figure-latex/character-mapping-154.pdf}

\begin{quote}
\textbf{Character 155: Gametes: Spermatozoa: Mid-piece}

0: Multiple mitochondria\\
1: Single ring-shaped mitochondrion\\
Neomorphic character.
\end{quote}

Following Hodgson \& Reunov \citeyearpar{Hodgson1994Ultrastructureof}.

\hypertarget{Amathia-coding-155}{}
\emph{Flustra}, \emph{Amathia}: Two mitochondrial derivatives in
\emph{Flustra} \citep{Franzen1981, Franzen1977}; four in
\emph{Tubulipora} \citep{Franzen1984}.

\hypertarget{Dentalium-coding-155}{}
\emph{Dentalium}: \citet{DufresneDube1983}.

\hypertarget{Lingula-coding-155}{}
\emph{Lingula}, \emph{Terebratulina}: Following Hodgson \& Reunov
\citeyearpar{Hodgson1994Ultrastructureof}.

\hypertarget{Loxosomella-coding-155}{}
\emph{Loxosomella}: ``The midpiece consists of two long mitochondrial
rods connected with each other by a thin mitochondrial lamella''
\citep[in \emph{Loxosoma}]{Franzen2000}; these are essentially a single
organelle surrounding a central rod of electron-dense material.

\hypertarget{Novocrania-coding-155}{}
\emph{Novocrania}: Four mitochondria \citep{Afzelius1978Finestructure}.

\hypertarget{Pelagodiscus_atlanticus-coding-155}{}
\emph{Pelagodiscus atlanticus}: Following \emph{Discinisca}
\emph{tenuis}, described in Hodgson \& Reunov
\citeyearpar{Hodgson1994Ultrastructureof}.

\hypertarget{Phoronis-coding-155}{}
\emph{Phoronis}: The mitochondria fuse in the middle stage of
spermiogenesis to become a pair of mitochondria
\citep{Reunov2004Ultrastructuralstudy}.

\hypertarget{Serpula-coding-155}{}
\emph{Serpula}: Five mitochondria in ring \citep{Gherardi2011}.

\hypertarget{Sipunculus-coding-155}{}
\emph{Sipunculus}: Ring of five mitochondria around the central
centriole \citep{Rice1993}.

\subsection*{{[}156{]} Centrioles:
Orientation}\label{centrioles-orientation}
\addcontentsline{toc}{subsection}{{[}156{]} Centrioles: Orientation}

\includegraphics{Brachiopod_phylogeny_files/figure-latex/character-mapping-155.pdf}

\begin{quote}
\textbf{Character 156: Gametes: Spermatozoa: Centrioles: Orientation}

0: Orthogonal\\
1: Parallel\\
Neomorphic character.
\end{quote}

Following \citet{Hodgson1994Ultrastructureof}.

\hypertarget{Amathia-coding-156}{}
\emph{Flustra}, \emph{Amathia}: \citep{Franzen1981}.

\hypertarget{Dentalium-coding-156}{}
\emph{Dentalium}: \citet{DufresneDube1983}.

\hypertarget{Lingula-coding-156}{}
\emph{Lingula}, \emph{Terebratulina}: Following Hodgson \& Reunov
\citeyearpar{Hodgson1994Ultrastructureof}.

\hypertarget{Novocrania-coding-156}{}
\emph{Novocrania}: Two orthogonal centrioles
\citep{Afzelius1978Finestructure}.

\hypertarget{Pelagodiscus_atlanticus-coding-156}{}
\emph{Pelagodiscus atlanticus}: Following \emph{Discinisca}
\emph{tenuis}, described in Hodgson \& Reunov
\citeyearpar{Hodgson1994Ultrastructureof}.

\hypertarget{Phoronis-coding-156}{}
\emph{Phoronis}: Only one centriole in spermatzoon
\citep[p.~7]{Reunov2004Ultrastructuralstudy}, but centrioles are
perpendicularly oriented in spermatogonia (ibid., p.~2).

\hypertarget{Serpula-coding-156}{}
\emph{Serpula}: The proximal centriole is parallel to the flagellum
\citep{Gherardi2011}.

\subsection*{{[}157{]} Centrioles: Fusion}\label{centrioles-fusion}
\addcontentsline{toc}{subsection}{{[}157{]} Centrioles: Fusion}

\includegraphics{Brachiopod_phylogeny_files/figure-latex/character-mapping-156.pdf}

\begin{quote}
\textbf{Character 157: Gametes: Spermatozoa: Centrioles: Fusion}

0: Discrete\\
1: Fused\\
Neomorphic character.
\end{quote}

Following \citet{Smith2012}; see \citet{BucklandNicks2008}.

\hypertarget{Amathia-coding-157}{}
\emph{Flustra}, \emph{Amathia}, \emph{Sipunculus}: Proximal centriole
fused anterior to distal centriole.

\hypertarget{Dentalium-coding-157}{}
\emph{Dentalium}: Proximal centriole fused anterior to distal centriole
\citep{DufresneDube1983}.

\hypertarget{Lingula-coding-157}{}
\emph{Loxosomella}, \emph{Novocrania}, \emph{Lingula}, \emph{Phoronis}:
Basal body in deep nuclear fossa.

\hypertarget{Tonicella-coding-157}{}
\emph{Tonicella}: Proximal centriole fused lateral to distal centriole
and offset.

\subsection*{{[}158{]} Satellite fibre
complex}\label{satellite-fibre-complex}
\addcontentsline{toc}{subsection}{{[}158{]} Satellite fibre complex}

\includegraphics{Brachiopod_phylogeny_files/figure-latex/character-mapping-157.pdf}

\begin{quote}
\textbf{Character 158: Gametes: Spermatozoa: Satellite fibre complex}

0: Annulus not associated with satellite fibres\\
1: Annulus associated with satellite fibres\\
Neomorphic character.
\end{quote}

Following \citet{Smith2012}, after character 48 in \citet{Ponder1997}.

\subsection*{{[}159{]} Mitochondria: Shape}\label{mitochondria-shape}
\addcontentsline{toc}{subsection}{{[}159{]} Mitochondria: Shape}

\includegraphics{Brachiopod_phylogeny_files/figure-latex/character-mapping-158.pdf}

\begin{quote}
\textbf{Character 159: Gametes: Spermatozoa: Mitochondria: Shape}

1: Spherical to subspherical\\
2: Rods\\
3: Elongate, sac-like\\
Transformational character.
\end{quote}

After character 5 in \citet{BucklandNicks2008}; see also character 43 in
\citet{Ponder1997}.

\hypertarget{Amathia-coding-159}{}
\emph{Flustra}, \emph{Amathia}: Rods \citep{Franzen1981}.

\hypertarget{Loxosomella-coding-159}{}
\emph{Loxosomella}: Elongate rods in \emph{Loxosoma}
\citep{Franzen2000}.

\hypertarget{Tonicella-coding-159}{}
\emph{Tonicella}: See \citet{BucklandNicks1988}.

\subsection*{{[}160{]} Mitochondria: Cristae:
Configuration}\label{mitochondria-cristae-configuration}
\addcontentsline{toc}{subsection}{{[}160{]} Mitochondria: Cristae:
Configuration}

\includegraphics{Brachiopod_phylogeny_files/figure-latex/character-mapping-159.pdf}

\begin{quote}
\textbf{Character 160: Gametes: Spermatozoa: Mitochondria: Cristae:
Configuration}

0: Unmodified\\
1: Arranged in parallel plates\\
Neomorphic character.
\end{quote}

After character 44 in \citet{Ponder1997}. Cristae are internal
compartments formed by inner mitochondrial membranes.

\hypertarget{Amathia-coding-160}{}
\emph{Flustra}, \emph{Amathia}: ``Typical cristae''; ``Randomly
oriented'' -- \citet{Franzen1984} (in \emph{Tubulipora}).

\hypertarget{Loxosomella-coding-160}{}
\emph{Loxosomella}: in \emph{Loxosoma} \citep{Franzen2000}.

\subsection*{{[}161{]} Mitochondria:
Midpiece}\label{mitochondria-midpiece}
\addcontentsline{toc}{subsection}{{[}161{]} Mitochondria: Midpiece}

\includegraphics{Brachiopod_phylogeny_files/figure-latex/character-mapping-160.pdf}

\begin{quote}
\textbf{Character 161: Gametes: Spermatozoa: Mitochondria: Midpiece}

1: Extremely short\\
2: Long\\
3: Forms continuous sheath\\
Transformational character.
\end{quote}

After \citet{Smith2012}; see also character 43 in \citet{Ponder1997};
character 164 in \citet{Giribet2002}.

\hypertarget{Amathia-coding-161}{}
\emph{Flustra}, \emph{Amathia}: Long \citep{Franzen1981}.

\hypertarget{Loxosomella-coding-161}{}
\emph{Loxosomella}: As long as the flagellum in \emph{Loxosoma}
\citep{Franzen2000}.

\hypertarget{Serpula-coding-161}{}
\emph{Serpula}: Five mitochondria surround the base of the flagellum in
short midpiece, comparable to that of \emph{Sipunculus} and
\emph{Dentalium} \citep{Gherardi2011}.

\hypertarget{Sipunculus-coding-161}{}
\emph{Sipunculus}: Short ring of five mitochondria around the central
centriole \citep{Rice1993}.

\section{Embryo}\label{embryo}

\subsection*{{[}162{]} Micromere size}\label{micromere-size}
\addcontentsline{toc}{subsection}{{[}162{]} Micromere size}

\includegraphics{Brachiopod_phylogeny_files/figure-latex/character-mapping-161.pdf}

\begin{quote}
\textbf{Character 162: Embryo: Micromere size}

1: Similar to macomeres\\
2: Small relative to macromeres\\
Transformational character.
\end{quote}

Following \citet{Hejnol2010}. Blastomeres may undergo significant size
differentiation, generating macromeres and micromeres of prominently
different sizes.

\hypertarget{Amathia-coding-162}{}
\emph{Flustra}, \emph{Amathia}: In \emph{Membranipora}, ``cleavage is
slightly unequal resulting in little larger central\\
blastomeres'' \citep{Gruhl2010M}.

\hypertarget{Lingula-coding-162}{}
\emph{Lingula}, \emph{Terebratulina}: \citet{Williams1997Introduction}.

\hypertarget{Phoronis-coding-162}{}
\emph{Phoronis}: Uniform size \citep{Pennerstorfer2012}.

\hypertarget{Sipunculus-coding-162}{}
\emph{Sipunculus}: Prominent differentiation in Phascolosoma
\citep{Adrianov2011}.

\subsection*{{[}163{]} Equal}\label{equal}
\addcontentsline{toc}{subsection}{{[}163{]} Equal}

\includegraphics{Brachiopod_phylogeny_files/figure-latex/character-mapping-162.pdf}

\begin{quote}
\textbf{Character 163: Embryo: Cleavage: Equal}

1: Unequal\\
2: Equal\\
Transformational character.
\end{quote}

Following character 170 in \citet{Giribet2002}.

\hypertarget{Lingula-coding-163}{}
\emph{Novocrania}, \emph{Pelagodiscus atlanticus}, \emph{Lingula},
\emph{Terebratulina}: Equal, in all brachiopods
\citep{Williams1997Introduction}.

\hypertarget{Phoronis-coding-163}{}
\emph{Phoronis}: ``Cleavage is holoblastic and results in approximately
equal sized, or adequal, blastomeres.'' -- \citet{Pennerstorfer2012}.

\subsection*{{[}164{]} Cross pattern}\label{cross-pattern}
\addcontentsline{toc}{subsection}{{[}164{]} Cross pattern}

\includegraphics{Brachiopod_phylogeny_files/figure-latex/character-mapping-163.pdf}

\begin{quote}
\textbf{Character 164: Embryo: Cleavage: Cross pattern}

0: Absent\\
1: Cross, whether ``molluscan'' or ``annelid''\\
Neomorphic character.
\end{quote}

The ``molluscan cross'' and ``annelid cross'' cannot be systematically
discriminated from one another, so are treated as a single state.\\
See characters 127 \& 128 in \citet{Rouse1999}; 1.49 in
\citet{SPS1996};\\
character 34 in \citet{Haszprunar1996}; 35 in \citet{Haszprunar2000};
172 in \citet{Giribet2002}.

\subsection*{{[}165{]} Polar lobe formation}\label{polar-lobe-formation}
\addcontentsline{toc}{subsection}{{[}165{]} Polar lobe formation}

\includegraphics{Brachiopod_phylogeny_files/figure-latex/character-mapping-164.pdf}

\begin{quote}
\textbf{Character 165: Embryo: Cleavage: Polar lobe formation}

1: Absent\\
2: Present\\
Transformational character.
\end{quote}

Following character 171 in \citet{Giribet2002}.

\subsection*{{[}166{]} Spiral}\label{spiral}
\addcontentsline{toc}{subsection}{{[}166{]} Spiral}

\includegraphics{Brachiopod_phylogeny_files/figure-latex/character-mapping-165.pdf}

\begin{quote}
\textbf{Character 166: Embryo: Cleavage: Spiral}

1: Absent\\
2: Present\\
Transformational character.
\end{quote}

See characters 32--33 in \citet{Haszprunar1996}; character 1.48 in
\citet{SPS1996}; character 29 in \citet{Glenner2004}.

\hypertarget{Amathia-coding-166}{}
\emph{Flustra}, \emph{Amathia}: ``While entoprocts are spiral cleavers,
ectoprocts show a radial cleavage pattern'' -- \citet{Fuchs2008}.

\hypertarget{Phoronis-coding-166}{}
\emph{Phoronis}: ``The observed cleavage displays several characters
consistent with the pattern of spiral cleavage''
\citep{Pennerstorfer2012}.

\subsection*{{[}167{]} Origin of mesoderm}\label{origin-of-mesoderm}
\addcontentsline{toc}{subsection}{{[}167{]} Origin of mesoderm}

\includegraphics{Brachiopod_phylogeny_files/figure-latex/character-mapping-166.pdf}

\begin{quote}
\textbf{Character 167: Embryo: Origin of mesoderm}

1: 4d cell, from the blastopore ridge, or as ectomesoderm\\
2: Archenteron\\
Transformational character.
\end{quote}

After characters 32 in \citet{Grobe2007} and 36--37 in
\citet{Glenner2004}.

\hypertarget{Terebratulina-coding-167}{}
\emph{Terebratulina}: \citet{Williams1997Introduction}.

\section{Larva: Apical organ}\label{larva-apical-organ}

\subsection*{{[}168{]} Muscles extending to the
hyposphere}\label{muscles-extending-to-the-hyposphere}
\addcontentsline{toc}{subsection}{{[}168{]} Muscles extending to the
hyposphere}

\includegraphics{Brachiopod_phylogeny_files/figure-latex/character-mapping-167.pdf}

\begin{quote}
\textbf{Character 168: Larva: Apical organ: Muscles extending to the
hyposphere}

0: Absent\\
1: Present\\
Neomorphic character.
\end{quote}

Character 8 in \citet{Vinther2008}.

\hypertarget{Amathia-coding-168}{}
\emph{Flustra}, \emph{Amathia}: Median muscles extending from apical
organ \citep{Gruhl2008}.

\hypertarget{Dentalium-coding-168}{}
\emph{Dentalium}: Apical organ has disappeared before musculature is set
in place \citep{Wanninger2002M}.

\hypertarget{Phoronis-coding-168}{}
\emph{Phoronis}: Not evident \citep[fig. 2C]{Santagata2004}.

\subsection*{{[}169{]} Serotonergic cells}\label{serotonergic-cells}
\addcontentsline{toc}{subsection}{{[}169{]} Serotonergic cells}

\includegraphics{Brachiopod_phylogeny_files/figure-latex/character-mapping-168.pdf}

\begin{quote}
\textbf{Character 169: Larva: Apical organ: Serotonergic cells}

1: Two flask-shaped cells\\
2: Four flask-shaped cells\\
3: Cluster of c. eight flask-shaped cells\\
4: Aggregation of multiple cells of multiple types\\
Transformational character.
\end{quote}

Character 8 in \citet{Haszprunar2008}.

\hypertarget{Amathia-coding-169}{}
\emph{Flustra}, \emph{Amathia}: Concentration of 30--40 serotonergic
perikarya \citep[in \emph{Fredericella};][]{Gruhl2010F}.

\hypertarget{Lingula-coding-169}{}
\emph{Lingula}: Cluster of ``numerous'' serotonergic cells
\citep{HaySchmidt1992, Altenburger2010}; more than, but probably
equivalent to, the flask-shaped cells of \emph{Terebratalia}
\citep{Luter2016}.

\hypertarget{Loxosomella-coding-169}{}
\emph{Loxosomella}: Six to eight apical cells; eight peripheral cells
\citep{Wanninger2007}, indicating a probable equivalence to
polyplacophorans \citep{Haszprunar2008}.

\hypertarget{Novocrania-coding-169}{}
\emph{Novocrania}: Four flask-shaped cells \citep{Altenburger2010}.

\hypertarget{Phoronis-coding-169}{}
\emph{Phoronis}: Multiple shapes of cells present \citep{Santagata2002};
resembles the linguliform arrangement \citep{Altenburger2010}.

\hypertarget{Sipunculus-coding-169}{}
\emph{Sipunculus}: Cluster of around eight cells, though not quite
countable \citep{Wanninger2005}.

\hypertarget{Terebratulina-coding-169}{}
\emph{Terebratulina}: Eight in \emph{Terebratalia} \citep{Luter2016}.

\hypertarget{Tonicella-coding-169}{}
\emph{Tonicella}: Eight in \emph{Ischnochiton} and \emph{Mopalia}
\citep{Wanninger2007}.

\subsection*{{[}170{]} Develops into adult
brain}\label{develops-into-adult-brain}
\addcontentsline{toc}{subsection}{{[}170{]} Develops into adult brain}

\includegraphics{Brachiopod_phylogeny_files/figure-latex/character-mapping-169.pdf}

\begin{quote}
\textbf{Character 170: Larva: Apical organ: Develops into adult brain}

0: Brain has other origin\\
1: Adult brain derived from larval apical organ / apical pole\\
2:\\
Neomorphic character.
\end{quote}

Character 79 in \citet{Glenner2004}.

\hypertarget{Lingula-coding-170}{}
\emph{Lingula}: ``both the larval apical ganglion and the ventral
ganglion must be retained as\\
the adult nervous system'' \citep{HaySchmidt1992}, but not necessarily
as the brain.

\subsection*{{[}171{]} Brain persists into
adulthood}\label{brain-persists-into-adulthood}
\addcontentsline{toc}{subsection}{{[}171{]} Brain persists into
adulthood}

\includegraphics{Brachiopod_phylogeny_files/figure-latex/character-mapping-170.pdf}

\begin{quote}
\textbf{Character 171: Larva: Brain persists into adulthood}

1: Brain lost\\
2: Brain retained to adulthood\\
Transformational character.
\end{quote}

After character 3 in \citet{Richter2010}.

\subsection*{{[}172{]} Origin of body
cavity}\label{origin-of-body-cavity}
\addcontentsline{toc}{subsection}{{[}172{]} Origin of body cavity}

\includegraphics{Brachiopod_phylogeny_files/figure-latex/character-mapping-171.pdf}

\begin{quote}
\textbf{Character 172: Larva: Origin of body cavity}

1: Mesenchyme\\
2: Coelom\\
Transformational character.
\end{quote}

Character 1.43 \href{mailto:in@SPS1996}{\nolinkurl{in@SPS1996}}.

\subsection*{{[}173{]} Formation of
coelomoducts}\label{formation-of-coelomoducts}
\addcontentsline{toc}{subsection}{{[}173{]} Formation of coelomoducts}

\includegraphics{Brachiopod_phylogeny_files/figure-latex/character-mapping-172.pdf}

\begin{quote}
\textbf{Character 173: Larva: Formation of coelomoducts}

1: Outgrowth\\
2: Ingrowth\\
Transformational character.
\end{quote}

Character 26 in \citet{Haszprunar2000}.

\hypertarget{Loxosomella-coding-173}{}
\emph{Loxosomella}: Coelomoducts absent \citep{Haszprunar2000}.

\subsection*{{[}174{]} Foot}\label{foot-1}
\addcontentsline{toc}{subsection}{{[}174{]} Foot}

\includegraphics{Brachiopod_phylogeny_files/figure-latex/character-mapping-173.pdf}

\begin{quote}
\textbf{Character 174: Larva: Foot}

0: Absent\\
1: Present\\
Neomorphic character.
\end{quote}

Foot or neurotroch present in larval stage, whether or not it is also
present in mature individuals. Following \citet{Wingstrand1985}.

\hypertarget{Loxosomella-coding-174}{}
\emph{Loxosomella}: A foot is present in the creeping-type larva of
\emph{Loxosomella} \emph{murmanica}, though absent in \emph{L. atkinsae}
and the many other entoprocts that have swimming-type larvae
\citep{Fuchs2008}.

\hypertarget{Serpula-coding-174}{}
\emph{Sipunculus}, \emph{Serpula}: \citet{Wingstrand1985} considers the
annelid neurotroch to be potentially homologous with the molluscan and
entoproct foot.

\subsection*{{[}175{]} Pedal gland}\label{pedal-gland}
\addcontentsline{toc}{subsection}{{[}175{]} Pedal gland}

\includegraphics{Brachiopod_phylogeny_files/figure-latex/character-mapping-174.pdf}

\begin{quote}
\textbf{Character 175: Larva: Foot: Pedal gland}

0: Absent\\
1: Present\\
Neomorphic character.
\end{quote}

A pedal gland is considered evidence for homology between the molluscan
and entoproct foot \citep{Haszprunar2008}.

\hypertarget{Amathia-coding-175}{}
\emph{Amathia}: Ciliated clef corresponds to position of foot
\citep{Reed1982}, but dedicated foot not present.

\subsection*{{[}176{]} Paired}\label{paired}
\addcontentsline{toc}{subsection}{{[}176{]} Paired}

\includegraphics{Brachiopod_phylogeny_files/figure-latex/character-mapping-175.pdf}

\begin{quote}
\textbf{Character 176: Larva: Coelom: Paired}

0: Absent\\
1: Paired coelom originating from two teloblasts derived from 4d\\
Neomorphic character.
\end{quote}

Character 2.02 in \citet{Scheltema1993}.

\hypertarget{Amathia-coding-176}{}
\emph{Amathia}: No evidence of pairing \citep{Reed1982}.

\hypertarget{Flustra-coding-176}{}
\emph{Flustra}: Hypostegal coelom separated from principal (perigastric)
body cavity in cheilostomata -- but this is not clearly equivalent to
the paired coelom intended by this character. The coelom of
\emph{Fredericella} is not paired \citep{Gruhl2010F}.

\subsection*{{[}177{]} Paried: Includes
pericardium}\label{paried-includes-pericardium}
\addcontentsline{toc}{subsection}{{[}177{]} Paried: Includes
pericardium}

\includegraphics{Brachiopod_phylogeny_files/figure-latex/character-mapping-176.pdf}

\begin{quote}
\textbf{Character 177: Larva: Coelom: Paried: Includes pericardium}

0: Paired coelom absent, or does not include pericardium\\
1: Paired coelom includes pericardium\\
Neomorphic character.
\end{quote}

Character 1.03 in \citet{Scheltema1993}.

\subsection*{{[}178{]} Feeding}\label{feeding}
\addcontentsline{toc}{subsection}{{[}178{]} Feeding}

\includegraphics{Brachiopod_phylogeny_files/figure-latex/character-mapping-177.pdf}

\begin{quote}
\textbf{Character 178: Larva: Feeding}

1: Lecithotrophic (or otherwise non-feeding)\\
2: Planktotrophic (or otherwise feeding)\\
Transformational character.
\end{quote}

Character 140 in \citet{Rouse1999}. See also character 2.66 in
\citet{SPS1996}; 153 in \citet{Giribet2002}.

\hypertarget{Amathia-coding-178}{}
\emph{Amathia}: Lecithotrophic \citep{Reed1982}.

\hypertarget{Flustra-coding-178}{}
\emph{Flustra}: Metamorphose almost immediately after release from
gonozooid \citep{Zimmer2013}; most bryozoans are lecithotrophic
\citep{Reed1982}.

\section{Larva: Cilia}\label{larva-cilia}

\subsection*{{[}179{]} Metatroch}\label{metatroch}
\addcontentsline{toc}{subsection}{{[}179{]} Metatroch}

\includegraphics{Brachiopod_phylogeny_files/figure-latex/character-mapping-178.pdf}

\begin{quote}
\textbf{Character 179: Larva: Cilia: Metatroch}

0: Absent\\
1: Present\\
Neomorphic character.
\end{quote}

See characters 129 and 131 in \citet{Rouse1999}; 40 in
\citet{Haszprunar1996}.\\
A prototroch is the defining character of a trochophore larva; a
metatroch is a secondary ciliary ring \citep{Rouse1999}.

\hypertarget{Terebratulina-coding-179}{}
\emph{Terebratulina}: \citet{Williams1997Introduction}.

\subsection*{{[}180{]} Telotroch}\label{telotroch}
\addcontentsline{toc}{subsection}{{[}180{]} Telotroch}

\includegraphics{Brachiopod_phylogeny_files/figure-latex/character-mapping-179.pdf}

\begin{quote}
\textbf{Character 180: Larva: Cilia: Telotroch}

0: Absent\\
1: Present\\
Neomorphic character.
\end{quote}

A posterior ciliary band. Character 136 in \citet{Rouse1999}.

\hypertarget{Amathia-coding-180}{}
\emph{Amathia}: Absent; single ciliary field lacks telotroch equivalent
\citep{Reed1982}.

\hypertarget{Flustra-coding-180}{}
\emph{Flustra}: Absent \citep{Zimmer2013}.

\hypertarget{Terebratulina-coding-180}{}
\emph{Terebratulina}: \citet{Williams1997Introduction}.

\subsection*{{[}181{]} Ciliated food groove}\label{ciliated-food-groove}
\addcontentsline{toc}{subsection}{{[}181{]} Ciliated food groove}

\includegraphics{Brachiopod_phylogeny_files/figure-latex/character-mapping-180.pdf}

\begin{quote}
\textbf{Character 181: Larva: Cilia: Ciliated food groove}

0: Absent\\
1: Present\\
Neomorphic character.
\end{quote}

Character 132 in \citet{Rouse1999}.

\hypertarget{Amathia-coding-181}{}
\emph{Amathia}: Coronal cilia do not form a food groove
\citep{Reed1982}.

\hypertarget{Flustra-coding-181}{}
\emph{Flustra}: Cyclostomes are covered in cilia but not arranged in
food groove.

\hypertarget{Terebratulina-coding-181}{}
\emph{Terebratulina}: \citet{Williams1997Introduction}.

\subsection*{{[}182{]} Ciliary bands:
Downstream}\label{ciliary-bands-downstream}
\addcontentsline{toc}{subsection}{{[}182{]} Ciliary bands: Downstream}

\includegraphics{Brachiopod_phylogeny_files/figure-latex/character-mapping-181.pdf}

\begin{quote}
\textbf{Character 182: Larva: Cilia: Ciliary bands: Downstream}

0: Absent\\
1: Present\\
Neomorphic character.
\end{quote}

Downstream-collecting ciliary bands of compound cilia on multiciliated
cells. See character 32 in \citet{Glenner2004}.

\hypertarget{Serpula-coding-182}{}
\emph{Serpula}: ``Groups such as Sabellariidae {[}\ldots{}{]} have
evolved downstream-feeding without the aid of a metatroch'' --
\citep{Rouse2000}.

\hypertarget{Sipunculus-coding-182}{}
\emph{Sipunculus}: ``Taxa such as Sipuncula {[}\ldots{}{]} have a
metatroch and do not have downstream larval-feeding'' --
\citet{Rouse2000}.

\subsection*{{[}183{]} Ciliary bands:
Upstream}\label{ciliary-bands-upstream}
\addcontentsline{toc}{subsection}{{[}183{]} Ciliary bands: Upstream}

\includegraphics{Brachiopod_phylogeny_files/figure-latex/character-mapping-182.pdf}

\begin{quote}
\textbf{Character 183: Larva: Cilia: Ciliary bands: Upstream}

0: Absent\\
1: Present\\
Neomorphic character.
\end{quote}

Upstream-collecting ciliary bands with single cilia on monociliated
cells. See character 32 in \citet{Glenner2004}.

\subsection*{{[}184{]} Adoral ciliary band}\label{adoral-ciliary-band}
\addcontentsline{toc}{subsection}{{[}184{]} Adoral ciliary band}

\includegraphics{Brachiopod_phylogeny_files/figure-latex/character-mapping-183.pdf}

\begin{quote}
\textbf{Character 184: Larva: Cilia: Adoral ciliary band}

0: Absent\\
1: Present\\
Neomorphic character.
\end{quote}

Characters 1.50, 2.66 and 4.68 in \citet{SPS1996}; 2 in
\citet{Vinther2008}. See also characters 39 in \citet{Haszprunar1996}
and 153 in \citet{Giribet2002}.

\subsection*{{[}185{]} Nerve ring underlying ciliated larval swimming
organ}\label{nerve-ring-underlying-ciliated-larval-swimming-organ}
\addcontentsline{toc}{subsection}{{[}185{]} Nerve ring underlying
ciliated larval swimming organ}

\includegraphics{Brachiopod_phylogeny_files/figure-latex/character-mapping-184.pdf}

\begin{quote}
\textbf{Character 185: Larva: Cilia: Nerve ring underlying ciliated
larval swimming organ}

0: Absent\\
1: Present\\
Neomorphic character.
\end{quote}

Following \citet{Wanninger2009}.

\hypertarget{Amathia-coding-185}{}
\emph{Amathia}: Nodular nerve ring underlies corona \citep{Reed1982}.

\hypertarget{Flustra-coding-185}{}
\emph{Flustra}: Present, following schematic in \citet{Gruhl2016}.

\section{Ciliary ultrastructure}\label{ciliary-ultrastructure}

\subsection*{{[}186{]} Accessory centriole}\label{accessory-centriole}
\addcontentsline{toc}{subsection}{{[}186{]} Accessory centriole}

\includegraphics{Brachiopod_phylogeny_files/figure-latex/character-mapping-185.pdf}

\begin{quote}
\textbf{Character 186: Ciliary ultrastructure: Accessory centriole}

0: Absent\\
1: Present\\
Neomorphic character.
\end{quote}

After \citet{Lundin2009}.

\hypertarget{Serpula-coding-186}{}
\emph{Serpula}: Present in certain annelids; not verified in
\emph{Serpula}.

\hypertarget{Terebratulina-coding-186}{}
\emph{Terebratulina}: Present \citep{Luter1995}.

\subsection*{{[}187{]} Aggregation of granules below basal
plate}\label{aggregation-of-granules-below-basal-plate}
\addcontentsline{toc}{subsection}{{[}187{]} Aggregation of granules
below basal plate}

\includegraphics{Brachiopod_phylogeny_files/figure-latex/character-mapping-186.pdf}

\begin{quote}
\textbf{Character 187: Ciliary ultrastructure: Aggregation of granules
below basal plate}

0: Absent\\
1: Present\\
Neomorphic character.
\end{quote}

After \citet{Lundin2009}.

\hypertarget{Serpula-coding-187}{}
\emph{Serpula}: Following \emph{Harmothoe} \citep{Holborow1969}.

\subsection*{{[}188{]} Radiating tubular
fibres}\label{radiating-tubular-fibres}
\addcontentsline{toc}{subsection}{{[}188{]} Radiating tubular fibres}

\includegraphics{Brachiopod_phylogeny_files/figure-latex/character-mapping-187.pdf}

\begin{quote}
\textbf{Character 188: Ciliary ultrastructure: Basal foot: Radiating
tubular fibres}

0: Absent\\
1: Present\\
Neomorphic character.
\end{quote}

After \citet{Lundin2009}. Fibres radiate from the distal end of the
basal foot of the cilia in certain taxa.

\hypertarget{Amathia-coding-188}{}
\emph{Amathia}: \citet{Reed1982}.

\hypertarget{Serpula-coding-188}{}
\emph{Serpula}: Basal foot in \emph{Magelona} is connected to
cytoplasmic microtubules \citep{Bartolomaeus1995}.

\subsection*{{[}189{]} Basal plate}\label{basal-plate}
\addcontentsline{toc}{subsection}{{[}189{]} Basal plate}

\includegraphics{Brachiopod_phylogeny_files/figure-latex/character-mapping-188.pdf}

\begin{quote}
\textbf{Character 189: Ciliary ultrastructure: Basal plate}

1: Thin\\
2: Blurry\\
Transformational character.
\end{quote}

After \citet{Lundin2009}. Also termed ``dense plate''.

\hypertarget{Amathia-coding-189}{}
\emph{Amathia}: \citet{Reed1982}.

\hypertarget{Serpula-coding-189}{}
\emph{Serpula}: Broad and `blurry' in \emph{Magelona}
\citep{Bartolomaeus1995}.

\hypertarget{Terebratulina-coding-189}{}
\emph{Terebratulina}: Thin to thick, but not blurry \citep{Luter1995}.

\subsection*{{[}190{]} Brushborder of
microvilli}\label{brushborder-of-microvilli}
\addcontentsline{toc}{subsection}{{[}190{]} Brushborder of microvilli}

\includegraphics{Brachiopod_phylogeny_files/figure-latex/character-mapping-189.pdf}

\begin{quote}
\textbf{Character 190: Ciliary ultrastructure: Brushborder of
microvilli}

0: Absent\\
1: Present\\
Neomorphic character.
\end{quote}

After \citet{Lundin2009}; coded following \citet{Smith2012}.

\hypertarget{Amathia-coding-190}{}
\emph{Amathia}: Present \citep{Reed1982}.

\hypertarget{Terebratulina-coding-190}{}
\emph{Terebratulina}: Absent \citep{Luter1995}.

\subsection*{{[}191{]} Centriolar triplet derivative in basal
body}\label{centriolar-triplet-derivative-in-basal-body}
\addcontentsline{toc}{subsection}{{[}191{]} Centriolar triplet
derivative in basal body}

\includegraphics{Brachiopod_phylogeny_files/figure-latex/character-mapping-190.pdf}

\begin{quote}
\textbf{Character 191: Ciliary ultrastructure: Centriolar triplet
derivative in basal body}

1: 9 + 2 pattern\\
2: 9 + 3 pattern\\
Transformational character.
\end{quote}

After \citet{Lundin2009}.

\hypertarget{Amathia-coding-191}{}
\emph{Amathia}: \citet{Reed1982}.

\hypertarget{Serpula-coding-191}{}
\emph{Serpula}: Following \emph{Enchytraeus} \citep{Reger1967},
\emph{Magelona} \citep{Bartolomaeus1995} and \emph{Harmothoe}
\citep{Holborow1969}.

\subsection*{{[}192{]} Ciliary necklace with connecting
strands}\label{ciliary-necklace-with-connecting-strands}
\addcontentsline{toc}{subsection}{{[}192{]} Ciliary necklace with
connecting strands}

\includegraphics{Brachiopod_phylogeny_files/figure-latex/character-mapping-191.pdf}

\begin{quote}
\textbf{Character 192: Ciliary ultrastructure: Ciliary necklace with
connecting strands}

0: Absent\\
1: Present\\
Neomorphic character.
\end{quote}

After \citet{Lundin2009}.\\
The ciliary necklace is defined by \citet{Gilula1972} as ``Well-defined
rows or strands of membrane particles that encircle the ciliary shaft''.
It occurs immediately below the basal plate, and comprises three beaded
circles of on the circumference of the cilia membrane.

\hypertarget{Amathia-coding-192}{}
\emph{Amathia}: \citet{Reed1982}.

\hypertarget{Serpula-coding-192}{}
\emph{Serpula}: Not evident in \emph{Enchytraeus} \citep{Reger1967},
\emph{Magelona} \citep{Bartolomaeus1995} or \emph{Harmothoe}
\citep{Holborow1969}.

\hypertarget{Terebratulina-coding-192}{}
\emph{Terebratulina}: \citep{Luter1995}.

\subsection*{{[}193{]} Presence}\label{presence-4}
\addcontentsline{toc}{subsection}{{[}193{]} Presence}

\includegraphics{Brachiopod_phylogeny_files/figure-latex/character-mapping-192.pdf}

\begin{quote}
\textbf{Character 193: Ciliary ultrastructure: Compound cilia: Presence}

0: Absent\\
1: Present\\
Neomorphic character.
\end{quote}

After \citet{Lundin2009}. Compound cilia are motile structures composed
of 10--100 regular cilia used in locomotion or feeding.

\hypertarget{Amathia-coding-193}{}
\emph{Amathia}: \citet{Reed1982}.

\hypertarget{Serpula-coding-193}{}
\emph{Serpula}: \citet{Nielsen1987}.

\subsection*{{[}194{]} Origin}\label{origin-1}
\addcontentsline{toc}{subsection}{{[}194{]} Origin}

\includegraphics{Brachiopod_phylogeny_files/figure-latex/character-mapping-193.pdf}

\begin{quote}
\textbf{Character 194: Ciliary ultrastructure: Compound cilia: Origin}

1: Several monociliate cells\\
2: On multiciliated cell\\
Transformational character.
\end{quote}

Character 14 in \citet{Glenner2004}. Compound cilia can be produced by
the aggregation of cilia from multiple monociliate cells, or from a
single cell bearing multiple cilia \citep{Nielsen1987}.

\hypertarget{Terebratulina-coding-194}{}
\emph{Terebratulina}: ``The coelothelial cells of the metacoel are
monociliated''; ``even some epithelial muscle cells are monociliated''
-- \citet{Luter1995}.

\subsection*{{[}195{]} Glycocalyx
ultrastructure}\label{glycocalyx-ultrastructure}
\addcontentsline{toc}{subsection}{{[}195{]} Glycocalyx ultrastructure}

\includegraphics{Brachiopod_phylogeny_files/figure-latex/character-mapping-194.pdf}

\begin{quote}
\textbf{Character 195: Ciliary ultrastructure: Glycocalyx
ultrastructure}

1: Homogeneous\\
2: Layered\\
Transformational character.
\end{quote}

After \citet{Lundin2009}.

\hypertarget{Amathia-coding-195}{}
\emph{Amathia}: \citet{Reed1982}.

\hypertarget{Terebratulina-coding-195}{}
\emph{Terebratulina}: Homogeneous \citep{Luter1995}.

\subsection*{{[}196{]} Branched}\label{branched}
\addcontentsline{toc}{subsection}{{[}196{]} Branched}

\includegraphics{Brachiopod_phylogeny_files/figure-latex/character-mapping-195.pdf}

\begin{quote}
\textbf{Character 196: Ciliary ultrastructure: Microvilli on epidermal
surface: Branched}

0: Unbranched\\
1: Branched\\
Neomorphic character.
\end{quote}

After \citet{Lundin2009}.

\hypertarget{Amathia-coding-196}{}
\emph{Amathia}: \citet{Reed1982}.

\hypertarget{Terebratulina-coding-196}{}
\emph{Terebratulina}: \citep{Luter1995}.

\subsection*{{[}197{]} Length}\label{length}
\addcontentsline{toc}{subsection}{{[}197{]} Length}

\includegraphics{Brachiopod_phylogeny_files/figure-latex/character-mapping-196.pdf}

\begin{quote}
\textbf{Character 197: Ciliary ultrastructure: Vertical ciliary rootlet:
Length}

1: Short\\
2: Long\\
Transformational character.
\end{quote}

After \citet{Lundin2009}. The vertical ciliary rootlet is also termed
the posterior rootlet.

\hypertarget{Amathia-coding-197}{}
\emph{Amathia}: \citet{Reed1982}.

\hypertarget{Loxosomella-coding-197}{}
\emph{Loxosomella}: Details of ciliary ultrastructure are illustrated in
\citet{Nielsen1976}.

\hypertarget{Terebratulina-coding-197}{}
\emph{Terebratulina}: Long \citep{Luter1995}.

\subsection*{{[}198{]} Shape}\label{shape}
\addcontentsline{toc}{subsection}{{[}198{]} Shape}

\includegraphics{Brachiopod_phylogeny_files/figure-latex/character-mapping-197.pdf}

\begin{quote}
\textbf{Character 198: Ciliary ultrastructure: Vertical ciliary rootlet:
Shape}

1: Conical\\
2: Flat\\
Transformational character.
\end{quote}

After \citet{Lundin2009}. The vertical ciliary rootlet is also termed
the posterior rootlet.

\hypertarget{Amathia-coding-198}{}
\emph{Amathia}: \citet{Reed1982}.

\hypertarget{Serpula-coding-198}{}
\emph{Serpula}: Conical in \emph{Enchytraeus} \citep{Reger1967} and
\emph{Magelona} \citep{Bartolomaeus1995}.

\hypertarget{Terebratulina-coding-198}{}
\emph{Terebratulina}: Conical: tapering to a point \citep{Luter1995}.

\subsection*{{[}199{]} Presence}\label{presence-5}
\addcontentsline{toc}{subsection}{{[}199{]} Presence}

\includegraphics{Brachiopod_phylogeny_files/figure-latex/character-mapping-198.pdf}

\begin{quote}
\textbf{Character 199: Ciliary ultrastructure: Secondary ciliary
rootlet: Presence}

0: Absent\\
1: Present\\
Neomorphic character.
\end{quote}

After \citet{Lundin2009}. The secondary ciliary rootlet is also termed
the anterior ciliary rootlet.

\hypertarget{Amathia-coding-199}{}
\emph{Amathia}: \citet{Reed1982}.

\subsection*{{[}200{]} Length}\label{length-1}
\addcontentsline{toc}{subsection}{{[}200{]} Length}

\includegraphics{Brachiopod_phylogeny_files/figure-latex/character-mapping-199.pdf}

\begin{quote}
\textbf{Character 200: Ciliary ultrastructure: Secondary ciliary
rootlet: Length}

1: Short\\
2: Long\\
Transformational character.
\end{quote}

After \citet{Lundin2009}. The secondary ciliary rootlet is also termed
the anterior ciliary rootlet.

\hypertarget{Amathia-coding-200}{}
\emph{Amathia}: \citet{Reed1982}.

\hypertarget{Serpula-coding-200}{}
\emph{Serpula}: Short in \emph{Enchytraeus} \citep{Reger1967},
\emph{Magelona} \citep{Bartolomaeus1995} and \emph{Harmothoe}
\citep{Holborow1969}.

\hypertarget{Terebratulina-coding-200}{}
\emph{Terebratulina}: ``Very small'' -- \citet{Luter1995}.

\subsection*{{[}201{]} Shape}\label{shape-1}
\addcontentsline{toc}{subsection}{{[}201{]} Shape}

\includegraphics{Brachiopod_phylogeny_files/figure-latex/character-mapping-200.pdf}

\begin{quote}
\textbf{Character 201: Ciliary ultrastructure: Secondary ciliary
rootlet: Shape}

1: Conical\\
2: Flat\\
Transformational character.
\end{quote}

After \citet{Lundin2009}. The secondary ciliary rootlet is also termed
the anterior ciliary rootlet.

\hypertarget{Amathia-coding-201}{}
\emph{Amathia}: \citet{Reed1982}.

\hypertarget{Serpula-coding-201}{}
\emph{Serpula}: Conical in \emph{Magelona} \citep{Bartolomaeus1995}.

\hypertarget{Terebratulina-coding-201}{}
\emph{Terebratulina}: Too small to evaluate.

\section{Nephridia}\label{nephridia}

\subsection*{{[}202{]} Podocytes}\label{podocytes}
\addcontentsline{toc}{subsection}{{[}202{]} Podocytes}

\includegraphics{Brachiopod_phylogeny_files/figure-latex/character-mapping-201.pdf}

\begin{quote}
\textbf{Character 202: Nephridia: Podocytes}

0: Absent\\
1: Present\\
Neomorphic character.
\end{quote}

See characters 21 and 28 in \citet{Haszprunar2000}; 1.12 in
\citet{Scheltema1993}.

\hypertarget{Lingula-coding-202}{}
\emph{Novocrania}, \emph{Pelagodiscus atlanticus}, \emph{Lingula},
\emph{Terebratulina}: ``In Brachiopoda, podocytes have never been
observed'' -- \citet{Luter1995}.

\hypertarget{Phoronis-coding-202}{}
\emph{Phoronis}: Present \citep{Storch1978}.

\hypertarget{Serpula-coding-202}{}
\emph{Serpula}: Present in serpulids \citep{Bartolomaeus2005}.

\subsection*{{[}203{]} Rhogocytes}\label{rhogocytes}
\addcontentsline{toc}{subsection}{{[}203{]} Rhogocytes}

\includegraphics{Brachiopod_phylogeny_files/figure-latex/character-mapping-202.pdf}

\begin{quote}
\textbf{Character 203: Nephridia: Rhogocytes}

0: Absent\\
1: Present\\
Neomorphic character.
\end{quote}

Pore cells. Character 20 in \citet{Haszprunar2000}.

\subsection*{{[}204{]} Serve as excretory
organs}\label{serve-as-excretory-organs}
\addcontentsline{toc}{subsection}{{[}204{]} Serve as excretory organs}

\includegraphics{Brachiopod_phylogeny_files/figure-latex/character-mapping-203.pdf}

\begin{quote}
\textbf{Character 204: Nephridia: Serve as excretory organs}

0: No\\
1: Yes\\
Neomorphic character.
\end{quote}

See character 4.46 in \citet{SPS1996}.

\hypertarget{Lingula-coding-204}{}
\emph{Novocrania}, \emph{Pelagodiscus atlanticus}, \emph{Lingula},
\emph{Terebratulina}: ``The excretory function of the metanephridia in
Brachiopoda must be questioned'' -- \citet{Luter1995}.

\subsection*{{[}205{]} Protonephridia}\label{protonephridia}
\addcontentsline{toc}{subsection}{{[}205{]} Protonephridia}

\includegraphics{Brachiopod_phylogeny_files/figure-latex/character-mapping-204.pdf}

\begin{quote}
\textbf{Character 205: Nephridia: Protonephridia}

0: Absent\\
1: Present\\
Neomorphic character.
\end{quote}

Also termed cyrtocytes. Character 21 in \citet{Grobe2007}; 1.47 in
\citet{SPS1996}; 138 in \citet{Rouse1999}; 20 in \citet{Haszprunar1996};
90 in \citet{Glenner2004}.

\subsection*{{[}206{]} Metanephridia}\label{metanephridia}
\addcontentsline{toc}{subsection}{{[}206{]} Metanephridia}

\includegraphics{Brachiopod_phylogeny_files/figure-latex/character-mapping-205.pdf}

\begin{quote}
\textbf{Character 206: Nephridia: Metanephridia}

0: Absent\\
1: Present\\
Neomorphic character.
\end{quote}

See characters 35 in \citet{Rouse1999}; 28 in \citet{Haszprunar2000}; 93
in \citet{Glenner2004}; 1.47 in \citet{SPS1996}; 21 in
\citet{Grobe2007}; 138 in \citet{Rouse1999}; 20 in
\citet{Haszprunar1996}.

\section{Cuticle}\label{cuticle}

\subsection*{{[}207{]} Layers}\label{layers}
\addcontentsline{toc}{subsection}{{[}207{]} Layers}

\includegraphics{Brachiopod_phylogeny_files/figure-latex/character-mapping-206.pdf}

\begin{quote}
\textbf{Character 207: Cuticle: Layers}

0: Simple (i.e.~glycocalyx)\\
1: Distinct epicuticle and endocuticle\\
Neomorphic character.
\end{quote}

Character 1 in \citet{Haszprunar1996}.

\subsection*{{[}208{]} Composition}\label{composition-1}
\addcontentsline{toc}{subsection}{{[}208{]} Composition}

\includegraphics{Brachiopod_phylogeny_files/figure-latex/character-mapping-207.pdf}

\begin{quote}
\textbf{Character 208: Cuticle: Composition}

1: Chitinous\\
2: Collagen\\
Transformational character.
\end{quote}

Character 2 in \citet{Haszprunar2008}.

\hypertarget{Amathia-coding-208}{}
\emph{Flustra}, \emph{Amathia}: Collagenous \citep{Schopf1967}, though
chitin is associated with the exoskeleton \citep{Hunt1972}.

\hypertarget{Dentalium-coding-208}{}
\emph{Tonicella}, \emph{Dentalium}: \citet{Haszprunar2008}.

\hypertarget{Lingula-coding-208}{}
\emph{Pelagodiscus atlanticus}, \emph{Lingula}, \emph{Terebratulina}:
The brachiopod pedicle has a chitinous cuticle
\citep{Williams1997Introduction, MacKay1978}, but the tentacles are
associated with collagen \citep{Williams1997Introduction}; marked as
polymorphic.

\hypertarget{Loxosomella-coding-208}{}
\emph{Loxosomella}: Absent \citep{Haszprunar2008}. Chitin is
occasionally present in certain species, perhaps in regions where
rigidity is necessary \citep{Borisanova2015}.

\hypertarget{Novocrania-coding-208}{}
\emph{Novocrania}: No (chitinous) pedicle, so only collagenous cuticle
present \citep{Williams1997Introduction}.

\hypertarget{Phoronis-coding-208}{}
\emph{Phoronis}: Collagen fibres in tentacle cuticle
\citep{Bartolomaeus2001U}; chitin only present in tubes
\citep{Jeuniaux1971}.

\hypertarget{Sipunculus-coding-208}{}
\emph{Sipunculus}: Collagenous \citep{Goffinet1978}.

\subsection*{{[}209{]} Fibrous layer with thick
fibrils}\label{fibrous-layer-with-thick-fibrils}
\addcontentsline{toc}{subsection}{{[}209{]} Fibrous layer with thick
fibrils}

\includegraphics{Brachiopod_phylogeny_files/figure-latex/character-mapping-208.pdf}

\begin{quote}
\textbf{Character 209: Cuticle: Fibrous layer with thick fibrils}

0: Absent\\
1: Present\\
Neomorphic character.
\end{quote}

After \citet{Borisanova2015}.

\hypertarget{Amathia-coding-209}{}
\emph{Loxosomella}, \emph{Flustra}, \emph{Amathia}, \emph{Serpula},
\emph{Tonicella}, \emph{Dentalium}: Following table 2 in
\citet{Borisanova2015}.

\hypertarget{Lingula-coding-209}{}
\emph{Lingula}: Pedicle cuticle entirely homogeneous
\citep{Williams1997Introduction}.

\hypertarget{Pelagodiscus_atlanticus-coding-209}{}
\emph{Pelagodiscus atlanticus}: Microvilli in otherwise homogeneous
epidermis \citep{Williams1997Introduction}.

\hypertarget{Phoronis-coding-209}{}
\emph{Phoronis}: Outer layer seemingly fibrous \citep{BereiterHahn1984}.

\hypertarget{Sipunculus-coding-209}{}
\emph{Sipunculus}: Fibrous collagen only \citep{BereiterHahn1984}.

\hypertarget{Terebratulina-coding-209}{}
\emph{Terebratulina}: Not evident in \emph{Notosaria}
\citep{BereiterHahn1984, Williams1997Introduction}.

\subsection*{{[}210{]} Homogeneous layer}\label{homogeneous-layer}
\addcontentsline{toc}{subsection}{{[}210{]} Homogeneous layer}

\includegraphics{Brachiopod_phylogeny_files/figure-latex/character-mapping-209.pdf}

\begin{quote}
\textbf{Character 210: Cuticle: Homogeneous layer}

0: Absent\\
1: Present\\
Neomorphic character.
\end{quote}

After \citet{Borisanova2015}.

\hypertarget{Amathia-coding-210}{}
\emph{Loxosomella}, \emph{Flustra}, \emph{Amathia}, \emph{Serpula},
\emph{Tonicella}, \emph{Dentalium}: Following table 2 in
\citet{Borisanova2015}.

\hypertarget{Lingula-coding-210}{}
\emph{Lingula}: Pedicle cuticle entirely homogeneous
\citep{Williams1997Introduction}.

\hypertarget{Pelagodiscus_atlanticus-coding-210}{}
\emph{Pelagodiscus atlanticus}: Microvilli in otherwise homogeneous
epidermis \citep{Williams1997Introduction}.

\hypertarget{Phoronis-coding-210}{}
\emph{Phoronis}: Not evident \citep{BereiterHahn1984}.

\hypertarget{Sipunculus-coding-210}{}
\emph{Sipunculus}: Fibrous collagen only \citep{BereiterHahn1984}.

\hypertarget{Terebratulina-coding-210}{}
\emph{Terebratulina}: Cuticle is homogeneous in \emph{Notosaria}
\citep{BereiterHahn1984, Williams1997Introduction}.

\subsection*{{[}211{]} Resilience}\label{resilience}
\addcontentsline{toc}{subsection}{{[}211{]} Resilience}

\includegraphics{Brachiopod_phylogeny_files/figure-latex/character-mapping-210.pdf}

\begin{quote}
\textbf{Character 211: Cuticle: Resilience}

0: Labile\\
1: Robust\\
2:\\
Neomorphic character.
\end{quote}

Character 1 in \citet{Haszprunar2000}.

\subsection*{{[}212{]} Microvilli}\label{microvilli}
\addcontentsline{toc}{subsection}{{[}212{]} Microvilli}

\includegraphics{Brachiopod_phylogeny_files/figure-latex/character-mapping-211.pdf}

\begin{quote}
\textbf{Character 212: Cuticle: Microvilli}

0: Absent\\
1: Microvilli present in the cuticle\\
Neomorphic character.
\end{quote}

After \citet{Borisanova2015}.

\hypertarget{Amathia-coding-212}{}
\emph{Loxosomella}, \emph{Flustra}, \emph{Amathia}, \emph{Serpula},
\emph{Tonicella}, \emph{Dentalium}: Following table 2 in
\citet{Borisanova2015}.

\hypertarget{Pelagodiscus_atlanticus-coding-212}{}
\emph{Pelagodiscus atlanticus}: Microvillios inner epithelium in Discina
\citep{Williams1997Introduction}.

\hypertarget{Phoronis-coding-212}{}
\emph{Phoronis}: Present on outer epithelium \citep{BereiterHahn1984}.

\hypertarget{Sipunculus-coding-212}{}
\emph{Sipunculus}: Fibrous collagen only \citep{BereiterHahn1984}.

\section{Muscles}\label{muscles}

\subsection*{{[}213{]} Cytology}\label{cytology}
\addcontentsline{toc}{subsection}{{[}213{]} Cytology}

\includegraphics{Brachiopod_phylogeny_files/figure-latex/character-mapping-212.pdf}

\begin{quote}
\textbf{Character 213: Muscles: Cytology}

1: Smooth\\
2: Obliquely striated\\
3: Smooth on abfrontal face; striated on frontal face\\
Transformational character.
\end{quote}

Character 19 in \citet{Haszprunar1996}; see also character 13 in
\citet{Haszprunar2000}.

\hypertarget{Amathia-coding-213}{}
\emph{Flustra}, \emph{Amathia}: In Bryozoa, myofibrils are ``all
striated'' \citep{Pardos1991}.

\hypertarget{Lingula-coding-213}{}
\emph{Novocrania}, \emph{Pelagodiscus atlanticus}, \emph{Lingula},
\emph{Terebratulina}: In brachiopods, myofibrils ``are striated on the
frontal face and smooth on the abfrontal face'' \citep{Pardos1991}.

\hypertarget{Phoronis-coding-213}{}
\emph{Phoronis}: ``In P. australis {[}\ldots{}{]} all the myofibrils
belong to the smooth type'' -- \citet{Pardos1991}.

\subsection*{{[}214{]} Histology}\label{histology}
\addcontentsline{toc}{subsection}{{[}214{]} Histology}

\includegraphics{Brachiopod_phylogeny_files/figure-latex/character-mapping-213.pdf}

\begin{quote}
\textbf{Character 214: Muscles: Histology}

0: Fibre-type\\
1: Epithelially organized\\
2:\\
Neomorphic character.
\end{quote}

See character 18 in \citet{Haszprunar1996}.

\section{Glands}\label{glands-1}

\subsection*{{[}215{]} Pedal gland}\label{pedal-gland-1}
\addcontentsline{toc}{subsection}{{[}215{]} Pedal gland}

\includegraphics{Brachiopod_phylogeny_files/figure-latex/character-mapping-214.pdf}

\begin{quote}
\textbf{Character 215: Glands: Pedal gland}

0: Absent\\
1: Present\\
Neomorphic character.
\end{quote}

Characters 1.13, 1.40 \& 2.08 in \citet{Scheltema1993}; 114 in
\citet{Giribet2002}; 1.53 in \citet{SPS1996}; 9 in
\citet{Haszprunar1996}.

\subsection*{{[}216{]} Paired pharyngeal
diverticulae}\label{paired-pharyngeal-diverticulae}
\addcontentsline{toc}{subsection}{{[}216{]} Paired pharyngeal
diverticulae}

\includegraphics{Brachiopod_phylogeny_files/figure-latex/character-mapping-215.pdf}

\begin{quote}
\textbf{Character 216: Glands: Paired pharyngeal diverticulae}

0: Absent\\
1: Present\\
Neomorphic character.
\end{quote}

\section{Nervous system}\label{nervous-system}

\subsection*{{[}217{]} Orthogonal}\label{orthogonal}
\addcontentsline{toc}{subsection}{{[}217{]} Orthogonal}

\includegraphics{Brachiopod_phylogeny_files/figure-latex/character-mapping-216.pdf}

\begin{quote}
\textbf{Character 217: Nervous system: Orthogonal}

0: Not orthogonal\\
1: Orthogonal\\
Neomorphic character.
\end{quote}

Character 14 in \citet{Haszprunar1996}. Paired longitudinal nerve cords
regularly interconnected by transversal commissures to form a
rectangular pattern.

\hypertarget{Amathia-coding-217}{}
\emph{Amathia}: \citet{Temereva2016Thenervous}.

\subsection*{{[}218{]} Glial system}\label{glial-system}
\addcontentsline{toc}{subsection}{{[}218{]} Glial system}

\includegraphics{Brachiopod_phylogeny_files/figure-latex/character-mapping-217.pdf}

\begin{quote}
\textbf{Character 218: Nervous system: Glial system}

0: Absent\\
1: Present\\
Neomorphic character.
\end{quote}

Character 16 in \citet{Haszprunar1996}. The Gliointerstitial system
interconnects the nervous and muscle systems.

\hypertarget{Phoronis-coding-218}{}
\emph{Phoronis}: Glial cells are ``abundant''
\citep{Temereva2016Phoronida}.

\subsection*{{[}219{]} Buccal nerve ring}\label{buccal-nerve-ring}
\addcontentsline{toc}{subsection}{{[}219{]} Buccal nerve ring}

\includegraphics{Brachiopod_phylogeny_files/figure-latex/character-mapping-218.pdf}

\begin{quote}
\textbf{Character 219: Nervous system: Buccal nerve ring}

0: Absent\\
1: Present\\
Neomorphic character.
\end{quote}

Character 7b in \citet{Haszprunar2008}.

\hypertarget{Amathia-coding-219}{}
\emph{Amathia}: \citet{Temereva2016Thenervous}.

\subsection*{{[}220{]} Anterior nerve loop}\label{anterior-nerve-loop}
\addcontentsline{toc}{subsection}{{[}220{]} Anterior nerve loop}

\includegraphics{Brachiopod_phylogeny_files/figure-latex/character-mapping-219.pdf}

\begin{quote}
\textbf{Character 220: Nervous system: Anterior nerve loop}

0: Absent\\
1: Present\\
Neomorphic character.
\end{quote}

Character 7c in \citet{Haszprunar2008}. A pre-oral nerve loop is present
in molluscs, \emph{Loxosomella} and certain annelids
\citep{Wanninger2007}.

\hypertarget{Amathia-coding-220}{}
\emph{Amathia}: \citet{Temereva2016Thenervous}.

\subsection*{{[}221{]} Formation of ganglia}\label{formation-of-ganglia}
\addcontentsline{toc}{subsection}{{[}221{]} Formation of ganglia}

\includegraphics{Brachiopod_phylogeny_files/figure-latex/character-mapping-220.pdf}

\begin{quote}
\textbf{Character 221: Nervous system: Formation of ganglia}

1: From cerebral region\\
2: In situ\\
3: Invagination of epithelium\\
Transformational character.
\end{quote}

Character 1.22 in \citet{SPS1996}.

\hypertarget{Amathia-coding-221}{}
\emph{Flustra}, \emph{Amathia}: ``The cerebral ganglion in all bryozoans
is formed as an invagination of a portion\\
of epithelium'' -- \citet{Temereva2016Thenervous}.

\subsection*{{[}222{]} Presence}\label{presence-6}
\addcontentsline{toc}{subsection}{{[}222{]} Presence}

\includegraphics{Brachiopod_phylogeny_files/figure-latex/character-mapping-221.pdf}

\begin{quote}
\textbf{Character 222: Nervous system: Cerebral ganglia: Presence}

0: Absent\\
1: Present\\
Neomorphic character.
\end{quote}

After character 13 in \citet{Haszprunar1996}.

\hypertarget{Amathia-coding-222}{}
\emph{Amathia}: \citet{Temereva2016Thenervous}.

\hypertarget{Phoronis-coding-222}{}
\emph{Phoronis}: We treat the dorsal ganglion, which is formed by two
ends of the tentacular nerve ring \citep{Temereva2016Phoronida}, as
cerebral.

\subsection*{{[}223{]} Fused}\label{fused}
\addcontentsline{toc}{subsection}{{[}223{]} Fused}

\includegraphics{Brachiopod_phylogeny_files/figure-latex/character-mapping-222.pdf}

\begin{quote}
\textbf{Character 223: Nervous system: Cerebral gangila: Fused}

1: Pair of distinct ganglia\\
2: Single ganglion, or fused ganglia\\
Transformational character.
\end{quote}

After character 13 in \citet{Haszprunar1996}.

\hypertarget{Amathia-coding-223}{}
\emph{Amathia}: Fused \citep{Temereva2016Thenervous}.

\subsection*{{[}224{]} Nerve cords}\label{nerve-cords}
\addcontentsline{toc}{subsection}{{[}224{]} Nerve cords}

\includegraphics{Brachiopod_phylogeny_files/figure-latex/character-mapping-223.pdf}

\begin{quote}
\textbf{Character 224: Nervous system: Nerve cords}

1: Ventral nerve cords only\\
2: Tetraneury: one pair of ventral and one pair of lateral nerve cords\\
Transformational character.
\end{quote}

See character 7 in \citet{Haszprunar2008}.

\hypertarget{Amathia-coding-224}{}
\emph{Amathia}: \citet{Temereva2016Thenervous}.

\subsection*{{[}225{]} Ventral longitudinal
nerves}\label{ventral-longitudinal-nerves}
\addcontentsline{toc}{subsection}{{[}225{]} Ventral longitudinal nerves}

\includegraphics{Brachiopod_phylogeny_files/figure-latex/character-mapping-224.pdf}

\begin{quote}
\textbf{Character 225: Nervous system: Ventral longitudinal nerves}

0: Separate\\
1: Paired or secondarily fused\\
Neomorphic character.
\end{quote}

Character 80 in \citet{Glenner2004}; see also character 6 in
\citet{Vinther2008}.

\hypertarget{Amathia-coding-225}{}
\emph{Amathia}: \citet{Temereva2016Thenervous}.

\chapter{Taxonomic implications}\label{taxonomic-implications}

This section briefly places key features of our results in the context
of previous phylogenetic hypotheses. For brevity, we refer to the method
of \citet{Brazeau2018} as ``corrected'' parsimony, as it corrects for
errors arising from the handling of inapplicable data in standard Fitch
parsimony.

Crown- and stem-group terminology has great value in clarifying the
early evolution of major lineages \citep{Budd2000, Carlson2009}. The
crown group of a lineage is defined as the last common ancestor of all
living members of a group, and all its descendants; the stem group as
all taxa more closely related to the crown group than to any other
extant taxon. In our selection of taxa, the brachiopod crown group
corresponds to the last common ancestor of \emph{Terebratulina},
\emph{Novocrania}, \emph{Pelagodiscus} and \emph{Lingula}; the
brachiopod stem group comprises anything between this node and the
branching point of \emph{Phoronis}, which marks the base of the
Brachiozoan crown group.

\begin{description}
\item[Hyoliths]
Hyoliths as a whole are interpreted as stem-group Brachiopods, which
refines the broader phylogenetic position proposed by
\citet{Moysiuk2017Hyolithsare}. This is to say, they sit closer to
brachiopods than the phoronids do, but no analysis places them within
the Brachiopod crown group.

Tommotiids lie both stemwards (e.g. \emph{Eccentrotheca}) and crownwards
(\emph{Mickwitzia}) of hyoliths, which thus represent derived
tommotiids.

Within the hyoliths, orthothecids are consistently recovered as a grade
that is paraphyletic to the hyolithids.
\item[Tommotiids]
Tommotiids represent a basal grade, paraphyletic to phoronids and
crown-group brachiopods, in line with previous interpretations.

\emph{Mickwitzia} is consistently the most crownwards of the tommotiids,
falling as sister to the brachiopod crown group (except in the Bayesian
results, where they fall within the paraphyletic craniiform grade).
\emph{Mickwitzia} is closely related to \emph{Micrina} and
\emph{Heliomedusa}, though the exact nature of this relationship varies
from analysis to analysis. The latter affilitation reflects similarities
emphasized by Holmer and Popov in \citet{Williams2007Supplement}.

\emph{Dailyatia} tends to plot closely with \emph{Halkieria}, reflecting
the similarity in the form of their proposed scleritome
\citep{Skovsted2015Theearly}. Bayesian analysis recovers this pair as
sister to annelids and molluscs; Fitch parsimony places them in the
molluscan stem group. Corrected parsimony instead places these taxa as
sister to hyoliths + brachiopods \citep[cf.][]{Zhao2017}.
\item[Craniiforms]
Trimerellids are reconstructed as paraphyletic with respect to
Craniiforms. This is consistent with the affinity commonly drawn between
these groups \citep[e.g.][]{Williams2000LinguliformeaCraniiformea}, and
helps to account for the stratigraphically late (Ordovician) appearance
of Craniids in the fossil record. (Aragonite is underrepresented in
early Palaeozoic strata due to taphonomic bias.)

The position of the craniiforms is not conclusively resolved; shell
characters point to a relationship with the Rhynchonelliforms, which is
countered by similarities between the spermatozoa of phoronids and
terebratulids, indicating a craniiform + linguliform clade. This latter
relationship is preferred by Fitch and corrected parsimony analyses.

The Bayesian results offer a more surprising interpretation that places
the craniiforms as paraphyletic with respect to all other brachiopods,
with \emph{Gasconsia} representing the basalmost member of the
Rhynchonellid lineage, upholding suggestions
\citep{Holmer2014OrdovicianSilurian} of a chileid rather than
trimerellid affinity. To our knowledge, the hypothesis of a paraphyletic
craniiform + trimerellid grade has never been proposed, and represents a
poor fit to stratigraphic data; potentially it represents an artefact
resulting from the incorrect handling of inapplicable data within the Mk
model.
\item[Rhynchonelliforms]
The position of kutorginids within the rhynchonelliform stem lineage has
been tricky to resolve \citep{Holmer2018Theattachment}. Corrected
parsimony places them sister to Rhynchonelliforms; Fitch parsimony
places them as sister to the Chileids; and Bayesian analysis
conservatively fails to distinguish between these two possibilities.
These results are broadly in accord with previous proposals
\citep{Holmer2018Evolutionarysignificance}. The protorthid
\emph{Glyptoria} is the earliest diverging of the included
rhynchonelliform lineages.

\emph{Salanygolina} has been interpreted as a stem-group
rhynchonelliform based on its combination of paterinid and chileate
features \citep{Holmer2009Theenigmatic}. Our results position
\emph{Salanygolina} as sister either to the Chileids or the Chileids +
Rhynchonelliforms, corroborating this interpretation.

Basal rhynchonellids are characterized by a circular umbonal perforation
in the ventral valve, associated with a colleplax. Partly on this basis,
the aberrant taxa \emph{Yuganotheca} and \emph{Tomteluva} tend to plot
close to the chileids under Fitch and Bayesian analysis, though a
variety of positions in this region of the tree are equally plausible.
Corrected parsimony, in contrast, supports the interpretation of
\emph{Yuganotheca} as a stem-group brachiopod \citep{Zhang2014Anearly}.
\item[Paterinids]
Paterinids have traditionally been placed within the Linguliforms on the
basis of their phosphatic shell \citep{Williams2007Supplement}, which we
identify as ancestral within the brachiopod crown group; consequently,
our analysis places the paterinids within the Rhynchonelliforms instead.
Characters supporting this position include the strophic hinge line,
planar cardinal area, the absence of a pedicle nerve impression, and the
morphology of the mantle canals.

More generally, although some lingulids can be found which share more
generic characters (e.g.~shell growth direction) with paterinids, the
particular combination of characters exhibited in paterinids does not
occur anywhere in the linguliform lineage, but is more similar to that
of basal rhynchonelliforms, particularly \emph{Salanygolina} \citep[as
noted by][]{Holmer2009Theenigmatic}.
\item[Linguliforms]
The reconstruction of Linguloformea comprising Linguloidea as sister to
Discinoidea is as expected. Lingulellotretids also sit within this
linguliform grouping; a position in the phoronid stem lineage
\citep[advocated by][]{Balthasar2009EarlyCambrian} is not upheld.
Acrotretids and Siphonotretids form a clade with \emph{Lingulosacculus}.

More novel is the reconstruction of the calcitic obolellid
\emph{Mummpikia} in the linguliform total group: a rhynchonelliform
affinity has been assumed based on its calcitic mineralogy. This said,
\citet{Balthasar2008iMummpikia} has highlighted the similarities between
obolellids and linguliform brachiopods, including sub-μm vertical canals
and the detailed configuration of the posterior shell margin. Our
analyses uphold the case for a linguliform affinity for
\emph{Mummpikia}; a calcitic shell seemingly arose through an
independent change within this taxon As such, \emph{Mummpikia} has no
direct bearing on the origin of `Calciata', save that shell mineralogy
is perhaps less static than commonly assumed.

More generally, our results identify Class Obolellata as polyphyletic:
\emph{Alisina} (Trematobolidae) plots within Rhynchonellata;
\emph{Tomteluva} is harder to place, but tends to group with
\emph{Salanygolina} stemwards of the chileids.
\item[Outgroup]
All analyses recover the same phylum-level relationships between the
outgroup taxa. Whereas the rooting of Lophotrochozoa is beyond the scope
of this analysis, annelids and molluscs are recovered as more closely
related to each other than to any other group, with Entoprocta
(Kamptozoa) representing their closest relative. Ectoprocta (Bryozoa)
are recovered as the closest relatives of Brachiozoa (=Phoroinda +
Brachiopoda).

As the analysis was not constructed to test the relationships of
outgroup taxa, some caution is due in interpreting these results;
outgroup taxa include single representatives of diverse and ancient
phyla, and are thus prone to long branch error \citep{Parks2014}. The
relationships of the lophotrochozoan phyla were not the primary object
of this study, and have long resisted elucidation; this said, we have
attempted to incorporate all morphological evidence that has been
interpreted as informing relationships between these groups.

In this context, it is perhaps unsurprising that the narrow selection of
fossil representatives of outgroup taxa, included to break long branches
at the base of each clade, plot somewhat differently in different
analyses. \emph{Wiwaxia} consistently plots with a molluscan
affiliation, but Bayesian and corrected parsimony analyses place it as a
stem-group mollusc, whereas Fitch parsimony includes it in the molluscan
crown. \emph{Halkieria}, \emph{Cotyledion}, \emph{Eccentrotheca} and
\emph{Dailyatia} are recovered as close relatives, forming a grade in
the brachiopod stem group under corrected parsimony, a clade in the
molluscan stem group in Fitch parsimony, and a polytomy at a deeper node
under Bayesian analysis.\\
\emph{Namacalathus} is reconstructed as a basal Brachiozoan under
corrected parsimony and Bayesian analysis, and a stem-group brachiopod
under Fitch parsimony.
\end{description}

\hypertarget{figures}{\chapter*{Supplementary Figures}\label{figures}}
\addcontentsline{toc}{chapter}{Supplementary Figures}

\begin{center}\includegraphics[width=0.8\linewidth]{images/image1} \end{center}

\textbf{Fig. S1. \emph{Pedunculotheca diania} Sun, Zhao et Zhu gen. et
sp. nov. from the Chengjiang Biota, Yunnan Province, China.} (a) NIGPAS
166601, external mould of dorsum with dorsal apex and pedicle foramen.
(b) NIGPAS 166597, preserving conical shell, operculum and internal soft
tissue, showing a compressed elliptic cross-section; backscatter
electron micrograph of boxed region shown in (c). (d) NIGPAS 166599b,
counterpart, juvenile conical shell with operculum showing two
longitudinal ventral grooves and circular larval shell. (e) NIGPAS
166602, conical shell with incomplete attachment structure. (f) NIGPAS
166598, broken shell with two ventral furrows and incomplete attachment
structure. (g) NIGPAS 166596, incomplete shell with one medial ventral
furrow and short attachment structure with coelomic cavity; detail of
boxed region shown in (h). (i) NIGPAS 166603, exterior of operculum.
Scale bars: 2mm (for a, b and e--g); 500~µm (for c, h and i).

Abbreviations: an = anus, cc = coelomic cavity, da = dorsal apex, es =
esophagus, in = intestine, mo = mouth, pe = pedicle, st = stomach.

\clearpage

\begin{center}\includegraphics[width=0.8\linewidth]{images/image2} \end{center}

\textbf{Fig. S2. Elemental distribution in the gut of
\emph{Pedunculotheca diania} Sun, Zhao et Zhu gen. et sp. nov.} NIGPAS
166597. Region corresponds to boxed region in Fig. S1c. Scale bar =
100~µm.

Abbreviations: BE = backscatter electron image, O = Oxygen, Si =
Silicon, Al = Aluminium, Fe = Iron, C = Carbon.

\clearpage

\includegraphics{Brachiopod_phylogeny_files/figure-latex/brach-diversity-1.pdf}

\textbf{Fig. S3. Global diversity of brachiopods through the Paleozoic.}
Points represent number of genera reported in each time bin; lines
represent rolling mean diversity over three consecutive time bins. Data
from Paleobiology database.

\clearpage

\hypertarget{table}{\chapter*{Supplementary Table}\label{table}}
\addcontentsline{toc}{chapter}{Supplementary Table}

\begin{tabular}{l|l|l}
\hline
NIGPAS Specimen numbers & Fossil locality & Coordinates\\
\hline
166593, 166617 & Shankou Village, Anning & 24°49’53’’ N, 102°24’47.9” E\\
\hline
166594, 166595 & Yaoying Village, Wuding & 25°36’01.2” N, 102°20’04.6” E\\
\hline
166596--166616 & Ma'anshan Village, Chengjiang & 24°40’37.2” N, 102°58’40.2” E\\
\hline
\end{tabular}

\textbf{Table S1. Provenance of fossil material.} Individuals from the
Yaoying section are usually bigger, with a thicker body wall, and have a
smaller ratio of apertural width to shell length than specimens from
other areas. In the absence of other differentiating features, we
consider these deviations to represent ecophenotypical variation within
a single species, perhaps reflecting the increased energetics and
predation pressure that accompany the shallower water depth reported at
the Yaoying section \citep{Zhao2012}.

\clearpage

\bibliography{References.bib}


\end{document}
