\documentclass[openany]{book}
\usepackage{lmodern}
\usepackage{amssymb,amsmath}
\usepackage{ifxetex,ifluatex}
\usepackage{fixltx2e} % provides \textsubscript
\ifnum 0\ifxetex 1\fi\ifluatex 1\fi=0 % if pdftex
  \usepackage[T1]{fontenc}
  \usepackage[utf8]{inputenc}
\else % if luatex or xelatex
  \ifxetex
    \usepackage{mathspec}
  \else
    \usepackage{fontspec}
  \fi
  \defaultfontfeatures{Ligatures=TeX,Scale=MatchLowercase}
\fi
% use upquote if available, for straight quotes in verbatim environments
\IfFileExists{upquote.sty}{\usepackage{upquote}}{}
% use microtype if available
\IfFileExists{microtype.sty}{%
\usepackage{microtype}
\UseMicrotypeSet[protrusion]{basicmath} % disable protrusion for tt fonts
}{}
\usepackage[margin=1in]{geometry}
\usepackage{hyperref}
\hypersetup{unicode=true,
            pdftitle={Supplementary Information for: Hyoliths with pedicles constrain the origin of the brachiopod body plan},
            pdfauthor={Haijing Sun, Martin R. Smith, Han Zeng, Fangchen Zhao, Guoxiang Li and Maoyan Zhu},
            pdfborder={0 0 0},
            breaklinks=true}
\urlstyle{same}  % don't use monospace font for urls
\usepackage{natbib}
\bibliographystyle{apalike-doi}
\usepackage{color}
\usepackage{fancyvrb}
\newcommand{\VerbBar}{|}
\newcommand{\VERB}{\Verb[commandchars=\\\{\}]}
\DefineVerbatimEnvironment{Highlighting}{Verbatim}{commandchars=\\\{\}}
% Add ',fontsize=\small' for more characters per line
\usepackage{framed}
\definecolor{shadecolor}{RGB}{248,248,248}
\newenvironment{Shaded}{\begin{snugshade}}{\end{snugshade}}
\newcommand{\KeywordTok}[1]{\textcolor[rgb]{0.13,0.29,0.53}{\textbf{#1}}}
\newcommand{\DataTypeTok}[1]{\textcolor[rgb]{0.13,0.29,0.53}{#1}}
\newcommand{\DecValTok}[1]{\textcolor[rgb]{0.00,0.00,0.81}{#1}}
\newcommand{\BaseNTok}[1]{\textcolor[rgb]{0.00,0.00,0.81}{#1}}
\newcommand{\FloatTok}[1]{\textcolor[rgb]{0.00,0.00,0.81}{#1}}
\newcommand{\ConstantTok}[1]{\textcolor[rgb]{0.00,0.00,0.00}{#1}}
\newcommand{\CharTok}[1]{\textcolor[rgb]{0.31,0.60,0.02}{#1}}
\newcommand{\SpecialCharTok}[1]{\textcolor[rgb]{0.00,0.00,0.00}{#1}}
\newcommand{\StringTok}[1]{\textcolor[rgb]{0.31,0.60,0.02}{#1}}
\newcommand{\VerbatimStringTok}[1]{\textcolor[rgb]{0.31,0.60,0.02}{#1}}
\newcommand{\SpecialStringTok}[1]{\textcolor[rgb]{0.31,0.60,0.02}{#1}}
\newcommand{\ImportTok}[1]{#1}
\newcommand{\CommentTok}[1]{\textcolor[rgb]{0.56,0.35,0.01}{\textit{#1}}}
\newcommand{\DocumentationTok}[1]{\textcolor[rgb]{0.56,0.35,0.01}{\textbf{\textit{#1}}}}
\newcommand{\AnnotationTok}[1]{\textcolor[rgb]{0.56,0.35,0.01}{\textbf{\textit{#1}}}}
\newcommand{\CommentVarTok}[1]{\textcolor[rgb]{0.56,0.35,0.01}{\textbf{\textit{#1}}}}
\newcommand{\OtherTok}[1]{\textcolor[rgb]{0.56,0.35,0.01}{#1}}
\newcommand{\FunctionTok}[1]{\textcolor[rgb]{0.00,0.00,0.00}{#1}}
\newcommand{\VariableTok}[1]{\textcolor[rgb]{0.00,0.00,0.00}{#1}}
\newcommand{\ControlFlowTok}[1]{\textcolor[rgb]{0.13,0.29,0.53}{\textbf{#1}}}
\newcommand{\OperatorTok}[1]{\textcolor[rgb]{0.81,0.36,0.00}{\textbf{#1}}}
\newcommand{\BuiltInTok}[1]{#1}
\newcommand{\ExtensionTok}[1]{#1}
\newcommand{\PreprocessorTok}[1]{\textcolor[rgb]{0.56,0.35,0.01}{\textit{#1}}}
\newcommand{\AttributeTok}[1]{\textcolor[rgb]{0.77,0.63,0.00}{#1}}
\newcommand{\RegionMarkerTok}[1]{#1}
\newcommand{\InformationTok}[1]{\textcolor[rgb]{0.56,0.35,0.01}{\textbf{\textit{#1}}}}
\newcommand{\WarningTok}[1]{\textcolor[rgb]{0.56,0.35,0.01}{\textbf{\textit{#1}}}}
\newcommand{\AlertTok}[1]{\textcolor[rgb]{0.94,0.16,0.16}{#1}}
\newcommand{\ErrorTok}[1]{\textcolor[rgb]{0.64,0.00,0.00}{\textbf{#1}}}
\newcommand{\NormalTok}[1]{#1}
\usepackage{longtable,booktabs}
\usepackage{graphicx,grffile}
\makeatletter
\def\maxwidth{\ifdim\Gin@nat@width>\linewidth\linewidth\else\Gin@nat@width\fi}
\def\maxheight{\ifdim\Gin@nat@height>\textheight\textheight\else\Gin@nat@height\fi}
\makeatother
% Scale images if necessary, so that they will not overflow the page
% margins by default, and it is still possible to overwrite the defaults
% using explicit options in \includegraphics[width, height, ...]{}
\setkeys{Gin}{width=\maxwidth,height=\maxheight,keepaspectratio}
\IfFileExists{parskip.sty}{%
\usepackage{parskip}
}{% else
\setlength{\parindent}{0pt}
\setlength{\parskip}{6pt plus 2pt minus 1pt}
}
\setlength{\emergencystretch}{3em}  % prevent overfull lines
\providecommand{\tightlist}{%
  \setlength{\itemsep}{0pt}\setlength{\parskip}{0pt}}
\setcounter{secnumdepth}{5}
% Redefines (sub)paragraphs to behave more like sections
\ifx\paragraph\undefined\else
\let\oldparagraph\paragraph
\renewcommand{\paragraph}[1]{\oldparagraph{#1}\mbox{}}
\fi
\ifx\subparagraph\undefined\else
\let\oldsubparagraph\subparagraph
\renewcommand{\subparagraph}[1]{\oldsubparagraph{#1}\mbox{}}
\fi

%%% Use protect on footnotes to avoid problems with footnotes in titles
\let\rmarkdownfootnote\footnote%
\def\footnote{\protect\rmarkdownfootnote}

%%% Change title format to be more compact
\usepackage{titling}

% Create subtitle command for use in maketitle
\newcommand{\subtitle}[1]{
  \posttitle{
    \begin{center}\large#1\end{center}
    }
}

\setlength{\droptitle}{-2em}
  \title{Supplementary Information for: \newline\newline Hyoliths with pedicles
constrain the origin of the brachiopod body plan}
  \pretitle{\vspace{\droptitle}\centering\huge}
  \posttitle{\par}
  \author{Haijing Sun, Martin R. Smith, Han Zeng, Fangchen Zhao, Guoxiang Li and
Maoyan Zhu}
  \preauthor{\centering\large\emph}
  \postauthor{\par}
  \predate{\centering\large\emph}
  \postdate{\par}
  \date{2018-06-08}

\usepackage{doi} % Adds hyperlinks to dois
\setcitestyle{round}
\usepackage[nottoc]{tocbibind} % list references in TOC
\raggedbottom % already in pnas-new
\usepackage[section]{placeins}

\usepackage{amsthm}
\newtheorem{theorem}{Theorem}[chapter]
\newtheorem{lemma}{Lemma}[chapter]
\newtheorem{corollary}{Corollary}[chapter]
\newtheorem{proposition}{Proposition}[chapter]
\newtheorem{conjecture}{Conjecture}[chapter]
\theoremstyle{definition}
\newtheorem{definition}{Definition}[chapter]
\theoremstyle{definition}
\newtheorem{example}{Example}[chapter]
\theoremstyle{definition}
\newtheorem{exercise}{Exercise}[chapter]
\theoremstyle{remark}
\newtheorem*{remark}{Remark}
\newtheorem*{solution}{Solution}
\begin{document}
\maketitle

{
\setcounter{tocdepth}{1}
\tableofcontents
}
\chapter*{Supplementary Text}\label{supplementary-text}
\addcontentsline{toc}{chapter}{Supplementary Text}

This document comtains supplementary material to
\citet{Sun2018Hyolithswith}. It is best viewed in HTML format at
\href{https://ms609.github.io/hyoliths/}{ms609.github.io/hyoliths}.

It opens with a detailed discussion of
\protect\hyperlink{treesearch}{analyses} of the
\protect\hyperlink{dataset}{morphological dataset} constructed to
accompany \citet{Sun2018Hyolithswith}, and their results.

The results presented in the main paper employ the algorithm described
by \citet{Brazeau2018} for correct handling of inapplicable data in a
parsimony setting. This document depicts how each character is most
parsimoniously \protect\hyperlink{reconstructions}{reconstructed} on an
optimal tree.

For completeness, we also document the results of
\protect\hyperlink{fitch}{standard Fitch parsimony} analysis, and the
results of \protect\hyperlink{bayesian}{Bayesian analysis}, neither of
which treat inapplicable data in a logically consistent fashion.

Supplementary \protect\hyperlink{figures}{figures} and
\protect\hyperlink{table}{tables} appear after the text.

\hypertarget{dataset}{\chapter{Phylogenetic dataset}\label{dataset}}

Analysis was performed on a new matrix of 47 early brachiozoan taxa,
including hyoliths, tommotiids and mickwitziids, which were coded for
220 morphological characters (132 neomorphic, 88 transformational).

\emph{Namacalathus} was incorporated as a 48\textsuperscript{th} taxon,
but preliminary results did not uphold the homology of its potentially
brachiozoan-like features. As such, we excluded it from our analysis due
to its morphological distance from ingroup taxa, a likely source of long
branch error. \emph{Dailyatia} was instead selected as an outgroup as
camenellans have been interpreted as the earliest diverging members of
the Brachiozoa \citep{Skovsted2015Theearly, Zhao2017}.

Characters are coded following the recommendations of
\citet{Brazeau2018}:

\begin{itemize}
\item
  We have employed reductive coding, using a distinct state to mark
  character inapplicability. Character specifications follow the
  structural syntax of \citet{Sereno2007} in order to highlight
  ontological dependence between characters and emphasize the structure
  of the dataset.
\item
  We have distinguished between neomorphic and transformational
  characters \citep[sensu][]{Sereno2007} by reserving the token
  \texttt{0} to refer to the absence of a neomorphic
  (i.e.~presence/absence) character. The states of transformational
  characters (i.e.~characters that describe a property of a feature) are
  represented by the tokens \texttt{1}, \texttt{2}, \texttt{3}, \ldots{}
\item
  We code the absence of neomorphic ontologically dependent characters
  \citep[sensu][]{Vogt2017} as absence, rather than inapplicability.
\end{itemize}

The complete dataset comprises 10560 character codings, of which 912 are
inapplicable and 4355 were neither ambiguous nor inapplicable. The
amount and quality of data that \emph{is} coded is more instructive than
a measure of how many cells are ambiguous \citep{Wiens1998, Wiens2003}.
Of the 220 characters, the number that were coded with an applicable
token for each taxon is:

\begin{tabular}{l|l|l|l|l|l}
\hline
 &  &  &  &  & \\
\hline
\_Namacalathus\_ & 53   \&nbsp; & \_Halkieria evangelista\_ & 58   \&nbsp; & \_Heliomedusa orienta\_ & 63   \&nbsp;\\
\hline
\_Haplophrentis carinatus\_ & 72   \&nbsp; & \_Dailyatia\_ & 50   \&nbsp; & \_Kutorgina chengjiangensis\_ & 78   \&nbsp;\\
\hline
\_Pedunculotheca diania\_ & 64   \&nbsp; & \_Acanthotretella spinosa\_ & 65   \&nbsp; & \_Lingulosacculus\_ & 56   \&nbsp;\\
\hline
\_Cotyledion tylodes\_ & 61   \&nbsp; & \_Alisina\_ & 80   \&nbsp; & \_Lingulellotreta malongensis\_ & 82   \&nbsp;\\
\hline
\_Loxosomella\_ & 161   \&nbsp; & \_Askepasma toddense\_ & 70   \&nbsp; & \_Longtancunella chengjiangensis\_ & 56   \&nbsp;\\
\hline
\_Flustra\_ & 165   \&nbsp; & \_Antigonambonites planus\_ & 78   \&nbsp; & \_Micrina\_ & 63   \&nbsp;\\
\hline
\_Amathia\_ & 152   \&nbsp; & \_Botsfordia\_ & 68   \&nbsp; & \_Micromitra\_ & 72   \&nbsp;\\
\hline
\_Pelagodiscus atlanticus\_ & 155   \&nbsp; & \_Clupeafumosus socialis\_ & 70   \&nbsp; & \_Mickwitzia muralensis\_ & 66   \&nbsp;\\
\hline
\_Terebratulina\_ & 176   \&nbsp; & \_Coolinia pecten\_ & 73   \&nbsp; & \_Mummpikia nuda\_ & 48   \&nbsp;\\
\hline
\_Lingula\_ & 195   \&nbsp; & \_Novocrania\_ & 181   \&nbsp; & \_Nisusia sulcata\_ & 76   \&nbsp;\\
\hline
\_Phoronis\_ & 166   \&nbsp; & \_Craniops\_ & 60   \&nbsp; & \_Orthis\_ & 66   \&nbsp;\\
\hline
\_Sipunculus\_ & 164   \&nbsp; & \_Gasconsia\_ & 64   \&nbsp; & \_Paterimitra\_ & 58   \&nbsp;\\
\hline
\_Serpula\_ & 166   \&nbsp; & \_Ussunia\_ & 51   \&nbsp; & \_Salanygolina\_ & 72   \&nbsp;\\
\hline
\_Tonicella\_ & 181   \&nbsp; & \_Eccentrotheca\_ & 49   \&nbsp; & \_Siphonobolus priscus\_ & 68   \&nbsp;\\
\hline
\_Dentalium\_ & 164   \&nbsp; & \_Eoobolus\_ & 74   \&nbsp; & \_Tomteluva perturbata\_ & 55   \&nbsp;\\
\hline
\_Wiwaxia corrugata\_ & 70   \&nbsp; & \_Glyptoria\_ & 69   \&nbsp; & \_Yuganotheca elegans\_ & 51   \&nbsp;\\
\hline
\end{tabular}

The matrix can be viewed interactively and downloaded at Morphobank
(\href{https://morphobank.org/permalink/?P2800}{project 2800}). {[}This
link will become live on publication of the paper. Referees should
follow the pre-publication link to the dataset that has been provided in
the main manuscript.{]}

A static version of the NEXUS file used to generate this supplementary
information can be downloaded directly from
\url{https://raw.githubusercontent.com/ms609/hyoliths/master/mbank_X24932_6-7-2018_1143.nex}
.

\hypertarget{treesearch}{\chapter{Parsimony analysis}\label{treesearch}}

The phylogenetic dataset contains a considerable proportion of
inapplicable codings (912/10560 = 8.6\% of tokens), which are known to
introduce error and bias to phylogenetic reconstruction when the Fitch
algorithm is employed \citep{Maddison1993, Brazeau2018}. As such, we
employed a new tree-scoring algorithm that correctly handles
inapplicable data \citep{Brazeau2018}, implemented in the
\emph{MorphyLib} C library \citep{Brazeau2017Morphylib}. We employed the
R package \emph{TreeSearch} v0.1.2 \citep{Smith2018TreeSearch} to
conduct phylogenetic tree search with this algorithm.

As this is a new method, we also employed the traditional, Fitch
algorithm, even though this approach is known to generate erroneous
trees. The results of this analysis can be viewed in
\protect\hyperlink{fitch}{a later section}.

\section{Search parameters}\label{search-parameters}

Heuristic searches were conducted using the parsimony ratchet
\citep{Nixon1999} under equal and implied weights \citep{Goloboff1997}.
The consensus tree presented in the main manuscript represents a strict
consensus of all trees that are most parsimonious under one or more of
the concavity constants (\emph{k}) 2, 3, 4.5, 7, 10.5, 16 and 24, an
approach that has been shown to produce higher accuracy (i.e.~more nodes
and quartets resolved correctly) than equal weights at any given level
of precision \citep{Smith2017}.

\section{Analysis}\label{analysis}

The R commands used to conduct the analysis are reproduced below. The
results can most readily be replicated using the
\href{https://github.com/ms609/hyoliths/}{R markdown files} (.Rmd) used
to generate these pages.

\subsection{Initialize and load data}\label{initialize-and-load-data}

\begin{Shaded}
\begin{Highlighting}[]
\CommentTok{# Load data from locally downloaded copy of MorphoBank matrix}
\NormalTok{my_data <-}\StringTok{ }\KeywordTok{ReadAsPhyDat}\NormalTok{(filename)}
\NormalTok{my_data[ignored_taxa] <-}\StringTok{ }\OtherTok{NULL}
\NormalTok{iw_data <-}\StringTok{ }\KeywordTok{PrepareDataIW}\NormalTok{(my_data)}
\end{Highlighting}
\end{Shaded}

\subsection{Generate starting tree}\label{generate-starting-tree}

Start by quickly rearranging a neighbour-joining tree, rooted on the
outgroup.

\begin{Shaded}
\begin{Highlighting}[]
\NormalTok{nj.tree <-}\StringTok{ }\KeywordTok{NJTree}\NormalTok{(my_data)}
\NormalTok{rooted.tree <-}\StringTok{ }\KeywordTok{EnforceOutgroup}\NormalTok{(nj.tree, outgroup)}
\NormalTok{start.tree <-}\StringTok{ }\KeywordTok{TreeSearch}\NormalTok{(}\DataTypeTok{tree=}\NormalTok{rooted.tree, }\DataTypeTok{dataset=}\NormalTok{my_data, }\DataTypeTok{maxIter=}\DecValTok{3000}\NormalTok{,}
                         \DataTypeTok{EdgeSwapper=}\NormalTok{RootedNNISwap, }\DataTypeTok{verbosity=}\DecValTok{0}\NormalTok{)}
\end{Highlighting}
\end{Shaded}

\subsection{Implied weights analysis}\label{implied-weights-analysis}

The position of the root does not affect tree score, so we keep it fixed
(using \texttt{RootedXXXSwap} functions) to avoid unnecessary swaps.

\begin{Shaded}
\begin{Highlighting}[]
\ControlFlowTok{for}\NormalTok{ (k }\ControlFlowTok{in}\NormalTok{ kValues) \{}
\NormalTok{  iw.tree <-}\StringTok{ }\KeywordTok{IWRatchet}\NormalTok{(start.tree, iw_data, }\DataTypeTok{concavity=}\NormalTok{k,}
                       \DataTypeTok{ratchHits =} \DecValTok{60}\NormalTok{, }\DataTypeTok{searchHits=}\DecValTok{55}\NormalTok{,}
                       \DataTypeTok{swappers=}\KeywordTok{list}\NormalTok{(RootedTBRSwap, RootedSPRSwap, RootedNNISwap),}
                       \DataTypeTok{verbosity=}\NormalTok{0L)}
\NormalTok{  score <-}\StringTok{ }\KeywordTok{IWScore}\NormalTok{(iw.tree, iw_data, }\DataTypeTok{concavity=}\NormalTok{k)}
  \CommentTok{# Write a single best tree}
  \KeywordTok{write.nexus}\NormalTok{(iw.tree,}
              \DataTypeTok{file=}\KeywordTok{paste0}\NormalTok{(}\StringTok{"TreeSearch/hy_iw_k"}\NormalTok{, k, }\StringTok{"_"}\NormalTok{, }
                          \KeywordTok{signif}\NormalTok{(score, }\DecValTok{5}\NormalTok{), }\StringTok{".nex"}\NormalTok{, }\DataTypeTok{collapse=}\StringTok{''}\NormalTok{))}

\NormalTok{  iw.consensus <-}\StringTok{ }\KeywordTok{IWRatchetConsensus}\NormalTok{(iw.tree, iw_data, }\DataTypeTok{concavity=}\NormalTok{k,}
                  \DataTypeTok{swappers=}\KeywordTok{list}\NormalTok{(RootedTBRSwap, RootedNNISwap),}
                  \DataTypeTok{searchHits=}\DecValTok{55}\NormalTok{,}
                  \DataTypeTok{nSearch=}\DecValTok{150}\NormalTok{, }\DataTypeTok{verbosity=}\NormalTok{0L)}
  \KeywordTok{write.nexus}\NormalTok{(iw.consensus, }
              \DataTypeTok{file=}\KeywordTok{paste0}\NormalTok{(}\StringTok{"TreeSearch/hy_iw_k"}\NormalTok{, k, }\StringTok{"_"}\NormalTok{, }
                          \KeywordTok{signif}\NormalTok{(}\KeywordTok{IWScore}\NormalTok{(iw.tree, iw_data, }\DataTypeTok{concavity=}\NormalTok{k), }\DecValTok{5}\NormalTok{),}
                          \StringTok{".all.nex"}\NormalTok{, }\DataTypeTok{collapse=}\StringTok{''}\NormalTok{))}
\NormalTok{\}}
\end{Highlighting}
\end{Shaded}

\subsection{Equal weights analysis}\label{equal-weights-analysis}

\begin{Shaded}
\begin{Highlighting}[]
\NormalTok{ew.tree <-}\StringTok{ }\KeywordTok{Ratchet}\NormalTok{(start.tree, my_data, }\DataTypeTok{verbosity=}\NormalTok{0L,}
                   \DataTypeTok{ratchHits =} \DecValTok{25}\NormalTok{, }\DataTypeTok{searchHits=}\DecValTok{55}\NormalTok{, }\CommentTok{# ratchHits = 10 not enough}
                   \DataTypeTok{swappers=}\KeywordTok{list}\NormalTok{(RootedTBRSwap, RootedSPRSwap, RootedNNISwap))}
\NormalTok{ew.consensus <-}\StringTok{ }\KeywordTok{RatchetConsensus}\NormalTok{(ew.tree, my_data, }\DataTypeTok{nSearch=}\DecValTok{250}\NormalTok{, }\DataTypeTok{searchHits =} \DecValTok{55}\NormalTok{,}
                                 \DataTypeTok{swappers=}\KeywordTok{list}\NormalTok{(RootedTBRSwap, RootedNNISwap),}
                                 \DataTypeTok{verbosity=}\NormalTok{0L)}
\KeywordTok{write.nexus}\NormalTok{(ew.consensus, }\DataTypeTok{file=}\KeywordTok{paste0}\NormalTok{(}\DataTypeTok{collapse=}\StringTok{''}\NormalTok{, }\StringTok{"TreeSearch/hy_ew_"}\NormalTok{,}
                                      \KeywordTok{Fitch}\NormalTok{(ew.tree, my_data), }\StringTok{".nex"}\NormalTok{))}
\end{Highlighting}
\end{Shaded}

\section{Results}\label{results}






\begin{figure}
\centering
\includegraphics{Brachiopod_phylogeny_files/figure-latex/treesearch-maj-consensus-1.pdf}
\caption{\label{fig:treesearch-maj-consensus}Consensus of all parsimony results.
Node labels denote the proportion of trees obtained
under all analytical conditions that support the clade.}
\end{figure}




\begin{figure}
\centering
\includegraphics{Brachiopod_phylogeny_files/figure-latex/treesearch-maj-consensus-pruned-1.pdf}
\caption{\label{fig:treesearch-maj-consensus-pruned}Consensus of all parsimony results, with taxa omitted
to show underlying clade support.
Node labels denote the proportion of trees obtained
under all analytical conditions that support the clade.}
\end{figure}





\begin{figure}
\centering
\includegraphics{Brachiopod_phylogeny_files/figure-latex/treesearch-iw-consensus-1.pdf}
\caption{\label{fig:treesearch-iw-consensus}Strict consensus of implied weights analyses at all
values of \emph{k}. Wildcard taxa have been excluded from the consensus
tree shown above to improve resolution.}
\end{figure}








\clearpage 

\begin{verbatim}
## 
##  > Results not available for panel 3
\end{verbatim}

\begin{figure}
\centering
\includegraphics{Brachiopod_phylogeny_files/figure-latex/treesearch-all-iw-results-1.pdf}
\caption{\label{fig:treesearch-all-iw-results}Consensus trees of implied weights analyses
at all values of \emph{k}, and at the individual
values \emph{k} = 2, 3 and 4.5.}
\end{figure}

\clearpage 

\begin{verbatim}
## 
##  > Results not available for panel 4
\end{verbatim}

\begin{verbatim}
## 
##  > Results not available for panel 5
\end{verbatim}

\begin{verbatim}
## 
##  > Results not available for panel 6
\end{verbatim}

\begin{verbatim}
## 
##  > Results not available for panel 7
\end{verbatim}

\clearpage

\begin{figure}
\centering
\includegraphics{Brachiopod_phylogeny_files/figure-latex/treesearch-equal-weights-results-1.pdf}
\caption{\label{fig:treesearch-equal-weights-results}Strict consensus of
most parsimonious trees under equally weighted parsimony}
\end{figure}

\clearpage

\hypertarget{reconstructions}{\chapter{Character
reconstructions}\label{reconstructions}}

This page provides definitions for each of the characters in our matrix,
and justifies codings in particular taxa where relevant. Further
citations for codings that are not discussed in the text can be viewed
by browsing the \protect\hyperlink{dataset}{morphological dataset} on
MorphoBank (\href{https://morphobank.org/permalink/?P2800}{project
2800}). This link will become live on publication of the paper. Referees
should follow the pre-publication link to the dataset that has been
provided in the main manuscript.

Alongside its definition, each character has been mapped onto a tree.
Here, we have arbitrarily selected one most parsimonious tree obtained
under implied weighting, \(k = 4.5\). Other trees can be viewed in the
HTML version of this document at
\href{https://ms609.github.io/hyoliths/reconstructions.html}{ms609.github.io/hyoliths}.
Each tip is labelled according to its coding in the matrix. These states
have been used to reconstruct the condition of each internal node, using
the parsimony method of \citet{Brazeau2018} as implemented in the
\emph{Inapp} \emph{R} package.

We emphasize that different trees will give different reconstructions.
The character mappings are not intended to definitively establish how
each character evolved, but to help the reader quickly establish how
each character has been coded, and to visualize at a glance how each
character fits onto a given tree.

\section{Brephic shell}\label{brephic-shell}

\subsection*{{[}1{]} Embryonic shell}\label{embryonic-shell}
\addcontentsline{toc}{subsection}{{[}1{]} Embryonic shell}

\includegraphics{Brachiopod_phylogeny_files/figure-latex/character-mapping-1.pdf}

\begin{quote}
\textbf{Character 1: Brephic shell: Embryonic shell}

0: Absent\\
1: Present\\
Neomorphic character.
\end{quote}

The embryonic shell or protegulum is secreted by the embryo immediately
before hatching.

\hypertarget{Acanthotretella_spinosa-coding-1}{}
\emph{Acanthotretella spinosa}: On hatching, the polyplacophoran larva
lacks a shell field.

Shell fields develop during the trochophore larva stage. The larva of
the chiton \emph{Mopalia} has two distinct shell fields: that anterior
to the prototroch will develop into the first shell plate; the one
posterior to the prototroch becomes the subsequent plates
\citep{Wanninger2002C}.

This disc-shaped posterior plate, whose position corresponds to the
conchiferan shell field, bears a polygonal ornament and is subdivided by
a series of grooves that prefigure the adult shell plates
\citep{Wanninger2002C}.

\hypertarget{Alisina-coding-1}{}
\emph{Alisina}: The shell does not form until the trochophore larval
stage, which has been exquisitely described in Antalis
\citep{Wanninger2001}.\\
This shell field is initially disc-like, subsequently expanding to fuse
ventrally and produce the cylindrical protoconch. The prototroch is
clearly delineated fro the telotroch in post-metamorphic juveniles
\citep{Wanninger2001}.

\hypertarget{Eoobolus-coding-1}{}
\emph{Eoobolus}: Shell not secreted until after metamorphosis
\citep{Popov2010Earliestontogeny}.

\hypertarget{Heliomedusa_orienta-coding-1}{}
\emph{Heliomedusa orienta}: The earliest shell is not described by
\citet{Hanken1985Thetaxonomy} or \citet{Watkins2002Newrecord}.

\hypertarget{Lingula-coding-1}{}
\emph{Lingula}: Absent, with no possible equivalent \citep{Nielsen1966}.

\hypertarget{Namacalathus-coding-1}{}
\emph{Namacalathus}: Inapplicable insofar as reproduction occurs by
budding; there is no evidence for a free-living larval stage.
Nevertheless, the presence of a sexual reproductive phase in addition to
asexual reproduction cannot be discounted.

\hypertarget{Sipunculus-coding-1}{}
\emph{Sipunculus}: \citet{Reed1982}.

\hypertarget{Ussunia-coding-1}{}
\emph{Ussunia}: Described by Topper \emph{et al}.
\citeyearpar{Topper2013Reappraisalof}.

\subsection*{{[}2{]} Morphology}\label{morphology}
\addcontentsline{toc}{subsection}{{[}2{]} Morphology}

\includegraphics{Brachiopod_phylogeny_files/figure-latex/character-mapping-2.pdf}

\begin{quote}
\textbf{Character 2: Brephic shell: Morphology}

1: Flat, disc-like (cf. \emph{Micrina})\\
2: Three prominent lobes forming a Y (cf. \emph{Paterimitra})\\
3: Spherical\\
Transformational character.
\end{quote}

The brephic shell is the shell possessed by the young organism
\citep[see][ and references therein for discussion of
terminology]{Ushatinskaya2016Revisionof}.

\emph{Micrina} resembles linguliforms \citep{Holmer2011Firstrecord}: in
both, the brephic mitral shell has one pair of setal sacs enclosed by
lateral lobes, whereas the brephic ventral shell has two lateral setal
tubes.

\emph{Paterimitra} and \emph{Salanygolina} have ``identical'' ventral
brephic shells \citep{Holmer2011Firstrecord}, resembling the shape of a
ship's propeller.

\emph{Haplophrentis} is coded following typical hyoliths, which have a
spherical brephic shell; \emph{Pedunculotheca}'s, in contrast, is
seemingly cap-shaped.

\hypertarget{Acanthotretella_spinosa-coding-2}{}
\emph{Acanthotretella spinosa}: Disc-like, subdivided by transverse
grooves \citep{Wanninger2002C}.

\hypertarget{Dentalium-coding-2}{}
\emph{Dentalium}: See fig. 159 in \citet{Williams1997Introduction}.

\hypertarget{Eccentrotheca-coding-2}{}
\emph{Eccentrotheca}: See fig. 3 in \citet{Bassett2017Earliestontogeny}.

\hypertarget{Glyptoria-coding-2}{}
\emph{Glyptoria}: The embryonic shell is more or less circular in
outline -- see \citet{Freeman1999Changesin}, fig. 6A.

\hypertarget{Mummpikia_nuda-coding-2}{}
\emph{Mummpikia nuda}: Disc-like \citep{Li2004}.

\hypertarget{Novocrania-coding-2}{}
\emph{Novocrania}: Renoid -- see fig. 4B3 in
\citet{Topper2013Theoldest}.

\hypertarget{Paterimitra-coding-2}{}
\emph{Paterimitra}: Subtriangular -- essentially round.

\hypertarget{Salanygolina-coding-2}{}
\emph{Salanygolina}: Trifoliate appearance results from prominent
attachment rudiment and bunching of setal sacs
\citep{Balthasar2009Thebrachiopod}.

\hypertarget{Serpula-coding-2}{}
\emph{Serpula}: See e.g.~fig 169 in Williams \emph{et al}.
\citeyearpar{Williams1997Introduction}.

\hypertarget{Ussunia-coding-2}{}
\emph{Ussunia}: The flat larval shell of \emph{Clupeafumosus} resembles
that of \emph{Micrina} in outline
\citetext{\citealp{Topper2013Reappraisalof}; \citealp[cf.][]{Holmer2011Firstrecord}}.

\subsection*{{[}3{]} Embryonic shell extended in
larvae}\label{embryonic-shell-extended-in-larvae}
\addcontentsline{toc}{subsection}{{[}3{]} Embryonic shell extended in
larvae}

\includegraphics{Brachiopod_phylogeny_files/figure-latex/character-mapping-3.pdf}

\begin{quote}
\textbf{Character 3: Brephic shell: Embryonic shell extended in larvae}

1: Not extended; embryonic shell contiguous with adult shell\\
2: Extended into larval shell, separated from adult shell by prominent
nick\\
Transformational character.
\end{quote}

Many taxa add to their embryonic shell (the protegulum possessed by the
embryo upon hatching) during the larval phase of their life cycle. The
shell that exists at metamorphosis, marked by a halo or nick point, is
variously termed the ``first formed shell'', ``metamorphic shell'' or
``larval shell'' \citep{Bassett2017Earliestontogeny}.

\hypertarget{Acanthotretella_spinosa-coding-3}{}
\emph{Acanthotretella spinosa}: \citet{Wanninger2002C}.

\hypertarget{Glyptoria-coding-3}{}
\emph{Glyptoria}: Prominent nick; see \citet{Freeman1999Changesin}, fig.
6A.

\hypertarget{Lingulellotreta_malongensis-coding-3}{}
\emph{Lingulellotreta malongensis}: Nick point indicated by arrows in
fig. 1 of Balthasar \citeyearpar{Balthasar2009Thebrachiopod}.

\hypertarget{Mummpikia_nuda-coding-3}{}
\emph{Mummpikia nuda}: No prominent nick point
\citep{Holmer1997EarlyCambrian, Li2004}.

\hypertarget{Pelagodiscus_atlanticus-coding-3}{}
\emph{Pelagodiscus atlanticus}: The flattened region at the umbo of the
ventral valve in smaller specimens conceivably represents an embryonic
shell, though it may alternatively represent a cicatrix or
colleplax-like structure.

\hypertarget{Ussunia-coding-3}{}
\emph{Ussunia}: Described by Topper \emph{et al}.
\citeyearpar{Topper2013Reappraisalof}.

\subsection*{{[}4{]} Surface ornament}\label{surface-ornament}
\addcontentsline{toc}{subsection}{{[}4{]} Surface ornament}

\includegraphics{Brachiopod_phylogeny_files/figure-latex/character-mapping-4.pdf}

\begin{quote}
\textbf{Character 4: Brephic shell: Surface ornament}

1: Smooth\\
2: Rounded pits\\
3: Polygonal impressions\\
4: Pustulose\\
Transformational character.
\end{quote}

Pitting of the larval shell characterises acrotretids and their
relatives. Pustules occur on Paterinidae. See Character 3 in Williams
\emph{et al}. \citeyearpar{Williams2000LinguliformeaCraniiformea} tables
5--6.

\hypertarget{Dentalium-coding-4}{}
\emph{Dentalium}, \emph{Serpula}: Smooth, following family-level codings
of \citet{Williams2000LinguliformeaCraniiformea}, table 6.

\hypertarget{Lingulellotreta_malongensis-coding-4}{}
\emph{Lingulellotreta malongensis}: Pitted \citep[table
8]{Williams2000LinguliformeaCraniiformea}.

\hypertarget{Mummpikia_nuda-coding-4}{}
\emph{Mummpikia nuda}: Smooth \citep{Holmer1997EarlyCambrian, Li2004}.

\hypertarget{Novocrania-coding-4}{}
\emph{Novocrania}: Indented with hexagonal pits \citep[appendix
2]{Williams1998Thediversity}.

\hypertarget{Orthis-coding-4}{}
\emph{Orthis}: Smooth \citep{Holmer2011Firstrecord}.

\hypertarget{Paterimitra-coding-4}{}
\emph{Paterimitra}: Pustolose in Paterinidae \citep[table
6]{Williams2000LinguliformeaCraniiformea}.

\hypertarget{Ussunia-coding-4}{}
\emph{Ussunia}: ``Larval shells on both valves {[}\ldots{}{]} are
covered by fine, shallow pits'' -- \citet{Topper2013Reappraisalof}.

\hypertarget{NA-coding-4}{}
NA: Polygonal texture present \citep{Holmer2011Firstrecord}, as in the
adult shell.

\hypertarget{NA-coding-4}{}
NA: Smooth \citep{Holmer2009Theenigmatic}.

\hypertarget{NA-coding-4}{}
NA: ``Smooth brephic shell'' -- \citet{Popov2009Earlyontogeny}.

\subsection*{{[}5{]} Larval attachment
structure}\label{larval-attachment-structure}
\addcontentsline{toc}{subsection}{{[}5{]} Larval attachment structure}

\includegraphics{Brachiopod_phylogeny_files/figure-latex/character-mapping-5.pdf}

\begin{quote}
\textbf{Character 5: Brephic shell: Larval attachment structure}

1: Without evidence of pedicle\\
2: With evidence of pedicle\\
Transformational character.
\end{quote}

Embryonic shells of \emph{Micrina} and certain linguliforms exhibit a
transversely folded posterior extension that speaks of the original
presence of a pedicle in the embryo.

This is independent of the presence of an adult pedicle, which may arise
after metamorphosis.

\hypertarget{Lingulellotreta_malongensis-coding-5}{}
\emph{Lingulellotreta malongensis}: Lobe related to the attachment
rudiment \citep[fig. 2]{Balthasar2009Thebrachiopod}.

\hypertarget{Mummpikia_nuda-coding-5}{}
\emph{Mummpikia nuda}: The pedicle foramen intersects the brephic shell
\citep{Holmer1997EarlyCambrian, Li2004}, suggesting larval attachment.

\hypertarget{Salanygolina-coding-5}{}
\emph{Salanygolina}: Note the posterior lobe related to the attachment
rudiment in fig. 2 of \citet{Balthasar2009Thebrachiopod}.

\hypertarget{Ussunia-coding-5}{}
\emph{Ussunia}: The larval shell embraces the pedicle foramen,
suggesting a larval attachment. See fig. 4 of Topper \emph{et al}.
\citeyearpar{Topper2013Reappraisalof}.

\hypertarget{NA-coding-5}{}
NA: Interpreted as having planktotrophic (and thus non-attached) larvae
\citep{Popov2009Earlyontogeny}.

\subsection*{{[}6{]} Setulose}\label{setulose}
\addcontentsline{toc}{subsection}{{[}6{]} Setulose}

\includegraphics{Brachiopod_phylogeny_files/figure-latex/character-mapping-6.pdf}

\begin{quote}
\textbf{Character 6: Brephic shell: Setulose}

0: No evidence of setae in embryonic shell\\
1: Setae present\\
Neomorphic character.
\end{quote}

The protegulum of \emph{Micrina} is penetrated with canals that were
originally associated with setae, a character that it has in common with
linguliforms \citep{Holmer2011Firstrecord}.

\hypertarget{Gasconsia-coding-6}{}
\emph{Gasconsia}: ``One specimen shows fine capillae running laterally
from the posterior tubercles on the dorsal valve (Pl. 3, fig. 5b). This
is possibly the imprints of setae.'' --
\citet{Ushatinskaya2016Revisionof}.

\hypertarget{Mummpikia_nuda-coding-6}{}
\emph{Mummpikia nuda}: Possible suggestion of setal sacs present on
brephic shell \citep{Holmer1997EarlyCambrian, Li2004}, but outline
inadequately preserved to code with confidence; treated as ambiguous.

\hypertarget{Salanygolina-coding-6}{}
\emph{Salanygolina}: Four setal sacs.

\hypertarget{Ussunia-coding-6}{}
\emph{Ussunia}: Setal bundles interpreted as present in acrotretids by
Ushatinskaya \citeyearpar{Ushatinskaya2016Protegulumand}.

\section{Brephic shell: Setal sacs
{[}7{]}}\label{brephic-shell-setal-sacs-7}

\includegraphics{Brachiopod_phylogeny_files/figure-latex/character-mapping-7.pdf}

\begin{quote}
\textbf{Character 7: Brephic shell: Setal sacs}

0: Absent\\
1: Present\\
Neomorphic character.
\end{quote}

Setal sacs are recognizable as raised lumps on the juvenile shell
\citep[see][]{Bassett2017Earliestontogeny}.

\emph{Micrina} and linguliforms have setal sacs on their mitral/dorsal
embryonic shell, whereas these are absent in \emph{Paterimitra}
\citep{Holmer2011Firstrecord}.

\hypertarget{Dentalium-coding-7}{}
\emph{Dentalium}: Lingulids' larval setae are not arranged in bundles
\citep{Carlson1995Phylogeneticrelationships}.

\hypertarget{Eoobolus-coding-7}{}
\emph{Eoobolus}, \emph{Serpula}: Three pairs
\citep{Carlson1995Phylogeneticrelationships}.

\hypertarget{Gasconsia-coding-7}{}
\emph{Gasconsia}: A single pair of low tubercles are \citep[ state ``may
be'']{Ushatinskaya2016Revisionof} located in the middle region of the
dorsal and the ventral brephic valve; these are interpreted as a single
pair of setal sacs, with the identity of the (dorsally unpaired)
tubercles uncertain.

\hypertarget{Mummpikia_nuda-coding-7}{}
\emph{Mummpikia nuda}: Possible suggestion of setal sacs present on
brephic shell \citep{Holmer1997EarlyCambrian, Li2004}, but outline
inadequately preserved to code with confidence; treated as ambiguous.

\hypertarget{Ussunia-coding-7}{}
\emph{Ussunia}: Setal bundles interpreted as present in acrotretids by
Ushatinskaya \citeyearpar{Ushatinskaya2016Protegulumand}.

\subsection*{{[}8{]} Number}\label{number}
\addcontentsline{toc}{subsection}{{[}8{]} Number}

\includegraphics{Brachiopod_phylogeny_files/figure-latex/character-mapping-8.pdf}

\begin{quote}
\textbf{Character 8: Brephic shell: Setal sacs: Number}

1: One pair\\
2: Two pairs\\
3: Three pairs\\
Transformational character.
\end{quote}

Two pairs on e.g.~Coolina; one on e.g. \emph{Micrina}.

\hypertarget{Eoobolus-coding-8}{}
\emph{Eoobolus}, \emph{Serpula}: Three pairs
\citep{Carlson1995Phylogeneticrelationships}.

\hypertarget{Gasconsia-coding-8}{}
\emph{Gasconsia}: ``larval shell with one to three apical tubercles in
ventral valve and two in dorsal valve''
\citep{Williams2000LinguliformeaCraniiformea} -- if these correspond to
setal sacs, then we interpret this as equivalent to one pair.

In \emph{B. minuta}, the ventral valve bears a single medial tubercle
(which in figured material seems to have two bilaterally symmetrical
fields), whereas the dorsal valve bears two apical tubercles
\citep{Li2004} -- supporting the interpretation of a single pair of
setal sacs.

\hypertarget{Salanygolina-coding-8}{}
\emph{Salanygolina}: See fig. 2 in \citet{Balthasar2009Thebrachiopod}.

\hypertarget{Ussunia-coding-8}{}
\emph{Ussunia}: Two pairs identified in acrotretids by Ushatinskaya
\citeyearpar{Ushatinskaya2016Protegulumand}.

\hypertarget{NA-coding-8}{}
NA: Two pairs of setal sacs \citep{Popov2009Earlyontogeny}.

\section{Setae}\label{setae}

\subsection*{{[}9{]} Present in adults}\label{present-in-adults}
\addcontentsline{toc}{subsection}{{[}9{]} Present in adults}

\includegraphics{Brachiopod_phylogeny_files/figure-latex/character-mapping-9.pdf}

\begin{quote}
\textbf{Character 9: Setae: Present in adults}

0: Absent\\
1: Present\\
Neomorphic character.
\end{quote}

Although preservation of setae (in adults) is exceptional, their
presence can be inferred from shelly material
\citep[see][]{Holmer2006Aspinose}.

\hypertarget{Acanthotretella_spinosa-coding-9}{}
\emph{Acanthotretella spinosa}: The girdle elements of certain
polyplacophorans are chitinous and secreted by microvilli
\citep{Fischer1980, Leise1982, Leise1988}; it is therefore likely that
they are homologous with the setae of other lophotrochozoans.

\hypertarget{Askepasma_toddense-coding-9}{}
\emph{Askepasma toddense}: Sclerites likely correspond with
lophotrochozoan setae \citep{Butterfield1990, Smith2014, Zhang2015}.

\hypertarget{Clupeafumosus_socialis-coding-9}{}
\emph{Clupeafumosus socialis}: Note that the setae do not obviously
emerge from tubes, leading Holmer and Caron to question their homology
with the setae of other taxa (\emph{Heliomedusa}, \emph{Mickwitzia}).

Both valves of \emph{Acanthotretella} were covered by long spine-like
and shell penetrating setae. The setae of \emph{A. decaius} are usually
preserved along anterior and anterolateral margins
\citep{Hu2010Softpart}.

\hypertarget{Eoobolus-coding-9}{}
\emph{Eoobolus}: ``Adult craniids are without setae (a feature shared
with the thecideides, the\\
shells of which are also cemented).'' -- \citet{Williams2007Supplement}.

\hypertarget{Halkieria_evangelista-coding-9}{}
\emph{Halkieria evangelista}: The absence of chitin or microvillar
lineations in sipunculan hooks argues against their interpretation as
setae, but they are coded as conceivable homologues, with these
characteristics treated separately.

\hypertarget{Mummpikia_nuda-coding-9}{}
\emph{Mummpikia nuda}: ``Setae appear short, delicate, and are closely
fringed with the entire\\
mantle margin, hardly extending beyond the edge of shell'' --
\citet{Zhang2005}.

\hypertarget{Phoronis-coding-9}{}
\emph{Phoronis}: A gizzard is not present in all bryozoans, and has not
been reported in \emph{Flustra}.

\hypertarget{Sipunculus-coding-9}{}
\emph{Sipunculus}: The teeth of the Bryozoan gizzard have been
homologized with annelid setae \citep{Gordon1975}.

\hypertarget{Ussunia-coding-9}{}
\emph{Ussunia}: Setal bundles interpreted as present in acrotretids by
Ushatinskaya \citeyearpar{Ushatinskaya2016Protegulumand}.

\hypertarget{NA-coding-9}{}
NA: Phosphatised setae emerge from hollow spines
\citep{Popov2009Earlyontogeny}.

\section{Setae: Secretion {[}10{]}}\label{setae-secretion-10}

\includegraphics{Brachiopod_phylogeny_files/figure-latex/character-mapping-10.pdf}

\begin{quote}
\textbf{Character 10: Setae: Secretion}

1: By basal microvilli\\
2: Epicuticular\\
Transformational character.
\end{quote}

The majority of lophotrochozoan sclerites bear a characteristic striated
texture that denotes their secretion by basal microvilli
\citep{Butterfield1990}. The seta-like hooks of sipunculans lack this
texture, suggesting that they may not be homologous with other setae.

\hypertarget{Halkieria_evangelista-coding-10}{}
\emph{Halkieria evangelista}: No evidence of microvillar secretion
\citep[e.g.][]{Schulze2005}.

\subsection*{{[}11{]} Microvillar diameter}\label{microvillar-diameter}
\addcontentsline{toc}{subsection}{{[}11{]} Microvillar diameter}

\includegraphics{Brachiopod_phylogeny_files/figure-latex/character-mapping-11.pdf}

\begin{quote}
\textbf{Character 11: Setae: Secretion: Microvillar diameter}

1: Uniform\\
2: Decreasing towards seta margin\\
Transformational character.
\end{quote}

The diameter of secretory microvilli may vary across the diameter of a
seta \citep{Smith2014}.

\hypertarget{Acanthotretella_spinosa-coding-11}{}
\emph{Acanthotretella spinosa}: Uniform \citep{Fischer1980, Leise1982}.

\hypertarget{Dailyatia-coding-11}{}
\emph{Dailyatia}: Following \emph{Scolelepis} \citep{Hausen2005};
Diasoma \citep{Orrhage1971}.

\hypertarget{Serpula-coding-11}{}
\emph{Serpula}: Slight decrease towards margin in \emph{Discinisca}
\citep{Luter2003}.

\hypertarget{Sipunculus-coding-11}{}
\emph{Sipunculus}: No trend in microvillar size \citep{Gordon1975}.

\hypertarget{Tonicella-coding-11}{}
\emph{Tonicella}: Decreases towards margin in \emph{Terebratalia} larvae
\citep{Gustus1972}.

\section{Setae: Composition {[}12{]}}\label{setae-composition-12}

\includegraphics{Brachiopod_phylogeny_files/figure-latex/character-mapping-12.pdf}

\begin{quote}
\textbf{Character 12: Setae: Composition}

1: Chitin\\
2: Horny protein\\
Transformational character.
\end{quote}

The majority of lophotrochozoan sclerites are chitinous, occasionally
hosting secondary biominerals.

\hypertarget{Halkieria_evangelista-coding-12}{}
\emph{Halkieria evangelista}: Enzymatic test for chitin proved negative
\citep{Rice1993}.

\subsection*{{[}13{]} Enamel}\label{enamel}
\addcontentsline{toc}{subsection}{{[}13{]} Enamel}

\includegraphics{Brachiopod_phylogeny_files/figure-latex/character-mapping-13.pdf}

\begin{quote}
\textbf{Character 13: Setae: Enamel}

0: Absent\\
1: Present\\
Neomorphic character.
\end{quote}

Certain setae are encapsulated in a 20 nm wide electron dense layer,
termed ``enamel'' by \citet{Gustus1973}. Enamel may be absent in larval
setae \citep{Luter2003}; this character refers to the condition in adult
setae.

\hypertarget{Acanthotretella_spinosa-coding-13}{}
\emph{Acanthotretella spinosa}: Not evident
\citep{Leise1988, Fischer1980}.

\hypertarget{Dailyatia-coding-13}{}
\emph{Dailyatia}: Present in \emph{Nereis} \citep{Gustus1973}.

\hypertarget{Sipunculus-coding-13}{}
\emph{Sipunculus}: Not evident \citep{Gordon1975}.

\subsection*{{[}14{]} Distribution}\label{distribution}
\addcontentsline{toc}{subsection}{{[}14{]} Distribution}

\includegraphics{Brachiopod_phylogeny_files/figure-latex/character-mapping-14.pdf}

\begin{quote}
\textbf{Character 14: Setae: Distribution}

1: Uniform\\
2: Only present at margins of shell\\
3: In bundles, repeated on each metamere if serial repetition present\\
4: In digestive tract\\
Transformational character.
\end{quote}

Setae penetrate the valves of many brachiopods. In certain taxa, they
are apparent only at the margins of the valves, in association with the
commissure, being reduced or lost over the surface of the shell.

\hypertarget{Acanthotretella_spinosa-coding-14}{}
\emph{Acanthotretella spinosa}: Uniformly distributed around girdle
(though not within shell) with no serial repetition
\citep{Vinther2005, Leise1988}.

\hypertarget{Lingulosacculus-coding-14}{}
\emph{Lingulosacculus}: Skovsted \emph{et al}.
\citeyearpar{Skovsted2011Scleritomeconstruction} assumed the setae may
have been present along the margin of the adapical opening, but there is
no fossil evidence.

\hypertarget{Micrina-coding-14}{}
\emph{Micrina}: Throughout the shell -- see
\citet{Williams2007Supplement} -- causing the pustulose appearance
remarked upon by \citet{Chen2007Reinterpretationof}.

\hypertarget{Mummpikia_nuda-coding-14}{}
\emph{Mummpikia nuda}: At margin of shell \citep{Zhang2005}.

\subsection*{{[}15{]} Constitution}\label{constitution}
\addcontentsline{toc}{subsection}{{[}15{]} Constitution}

\includegraphics{Brachiopod_phylogeny_files/figure-latex/character-mapping-15.pdf}

\begin{quote}
\textbf{Character 15: Setae: Constitution}

1: Solid, blade-like\\
2: Basal invagination\\
Transformational character.
\end{quote}

Sipunculan ``setae'' are basally invaginated, suggesting that they may
not be homologous with annelid chaetae.

\hypertarget{Sipunculus-coding-15}{}
\emph{Sipunculus}: Cytoplasmic intrusion into a central cavity
\citep{Gordon1975}.

\section{Body organization}\label{body-organization}

\subsection*{{[}16{]} Serial repetition}\label{serial-repetition}
\addcontentsline{toc}{subsection}{{[}16{]} Serial repetition}

\includegraphics{Brachiopod_phylogeny_files/figure-latex/character-mapping-16.pdf}

\begin{quote}
\textbf{Character 16: Body organization: Serial repetition}

0: Absent\\
1: Present\\
Neomorphic character.
\end{quote}

Serial repetition in adult, whether expressed in valves, soft tissues or
exoskeletal elements. See character 13 in \citet{Rouse1999}; 19 in
\citet{Vinther2008}; 38 in \citet{Haszprunar1996}; 40--41 in
\citet{Sutton2012}; \citet{Wanninger2009}.

\hypertarget{Antigonambonites_planus-coding-16}{}
\emph{Antigonambonites planus}: Elements of the \emph{Halkieria}
scleritome adhere to a quincunx arrangement, with different spacing of
elements in each zone; there is no evidence of a metameric arrangement.

\hypertarget{Botsfordia-coding-16}{}
\emph{Botsfordia}: Unknown whether sclerites are serially repeated, or
whether metameres were present in underlying soft anatomy.

\hypertarget{Namacalathus-coding-16}{}
\emph{Namacalathus}: Not evident.

\subsection*{{[}17{]} Foot}\label{foot}
\addcontentsline{toc}{subsection}{{[}17{]} Foot}

\includegraphics{Brachiopod_phylogeny_files/figure-latex/character-mapping-17.pdf}

\begin{quote}
\textbf{Character 17: Body organization: Foot}

0: Absent\\
1: Present\\
Neomorphic character.
\end{quote}

See characters 8 in \citet{Haszprunar1996}; 4 in \citet{Vinther2008};
137 in \citet{Rouse1999}; 21 in \citet{BucklandNicks2008}; 37 in
\citet{Sutton2012}; 1, 3 and 4 in \citet{Haszprunar2008}.\\
It is assumed that the adult foot is homologous with (and thus
contingent on) the larval foot.

\hypertarget{Antigonambonites_planus-coding-17}{}
\emph{Antigonambonites planus}: The ventral surface of \emph{Halkieria}
is unarmoured, but its soft anatomy is unknown.

\hypertarget{Halkieria_evangelista-coding-17}{}
\emph{Halkieria evangelista}: LISTED AS PRESENT IN \citet{Smith2012}:
WHY?.

\hypertarget{Lingula-coding-17}{}
\emph{Lingula}: See \citet{Haszprunar2008}.

\hypertarget{Terebratulina-coding-17}{}
\emph{Terebratulina}: The stalk may conceivably be homologous with the
entoproct foot, but the evidence for homology is weak.

\subsection*{{[}18{]} Coelom}\label{coelom}
\addcontentsline{toc}{subsection}{{[}18{]} Coelom}

\includegraphics{Brachiopod_phylogeny_files/figure-latex/character-mapping-18.pdf}

\begin{quote}
\textbf{Character 18: Body organization: Coelom}

0: Absent: adults acoelomate\\
1: Present: true coelomic cavities differentiated\\
Neomorphic character.
\end{quote}

\hypertarget{Amathia-coding-18}{}
\emph{Amathia}: Internal cavities indicated by differentiation of
internal organs \citep[see][]{Moysiuk2017Hyolithsare}.

\hypertarget{Lingula-coding-18}{}
\emph{Lingula}: ``Adult entoprocts are acoelomate'' --
\citet{Fuchs2008}.

\hypertarget{Phoronis-coding-18}{}
\emph{Phoronis}: ``Adult ectoprocts differentiate true coelomic
cavities'' -- \citet{Fuchs2008}.

\hypertarget{Wiwaxia_corrugata-coding-18}{}
\emph{Wiwaxia corrugata}: \citet{Temereva2017Innervationof}.

\subsection*{{[}19{]} Number}\label{number-1}
\addcontentsline{toc}{subsection}{{[}19{]} Number}

\includegraphics{Brachiopod_phylogeny_files/figure-latex/character-mapping-19.pdf}

\begin{quote}
\textbf{Character 19: Body organization: Coelomoducts: Number}

1: Single\\
2: Multiple\\
Transformational character.
\end{quote}

Character 27 in \citet{Haszprunar2000}. Coelomoducts are excretory
organs derived from the coelom, also in some cases serving as genital
ducts (gonoducts); they replace (and may resemble) nephridia
\citep{Goodrich1945}.

\hypertarget{Lingula-coding-19}{}
\emph{Lingula}: Two coelomoducts pass outwards, meet, and open by a
common pore \citep{Goodrich1945}.

\hypertarget{Phoronis-coding-19}{}
\emph{Phoronis}: Multiple ciliated ducts leading to a common gonopore
\citep{Goodrich1945}.

\hypertarget{Wiwaxia_corrugata-coding-19}{}
\emph{Wiwaxia corrugata}: ``large coelomic funnels serving as genital
ducts'' \citep{Goodrich1945}.

\section{Body organization: Gills
{[}20{]}}\label{body-organization-gills-20}

\includegraphics{Brachiopod_phylogeny_files/figure-latex/character-mapping-20.pdf}

\begin{quote}
\textbf{Character 20: Body organization: Gills}

0: Absent\\
1: Present\\
Neomorphic character.
\end{quote}

Gills (or ctenidia) surround the molluscan foot.\\
Character 1.59--60, 2.09, 4.49 in \citet{SPS1996}; 10--11 in
\citet{Haszprunar2000}; 45 in \citet{Sutton2012}.

\section{Pedicle {[}21{]}}\label{pedicle-21}

\includegraphics{Brachiopod_phylogeny_files/figure-latex/character-mapping-21.pdf}

\begin{quote}
\textbf{Character 21: Pedicle}

0: Absent\\
1: Present\\
Neomorphic character.
\end{quote}

The brachiopod pedicle is a fleshy protuberance that emerges from the
posterior part of the body wall -- as denoted in fossil taxa by its
occurrence between the dorsal and ventral valves.

It is important to distinguish the pedicle from the ``pedicle sheath'',
a tubular extension of the umbo that grows by accretion from an isolated
portion of the ventral mantle. For discussion see
\citet{Holmer2018Theattachment} and \citet{Bassett2017Earliestontogeny}.

\hypertarget{Clupeafumosus_socialis-coding-21}{}
\emph{Clupeafumosus socialis}: The attachment structure of
\emph{Acanthotretella} originates at the margin of the dorsal and
ventral valves; although it emerges from the umbo of the ventral valve,
the presence of an internal pedicle tube betrays its identity as a
pedicle, rather than a pedicle sheath.

The pedicle of \emph{Acanthotretella} emerges from a short extension of
the umbo of the ventral valve. This extension is contiguous with the
valve and presumably grew by accretion; its position and continuity with
the valve suggest its interpretation as a pedicle sheath that is
superseded as an attachment structure. On the other hand, its continuity
with the internal pedicle tube suggests that is may represent an
independent organ.

\hypertarget{Gasconsia-coding-21}{}
\emph{Gasconsia}: Pedicle foramen was not necessarily occupied by a
pedicle (though it presumably was).

\hypertarget{Glyptoria-coding-21}{}
\emph{Glyptoria}: Attached apically by cementation.

\hypertarget{Halkieria_evangelista-coding-21}{}
\emph{Halkieria evangelista}: Absent; there is no clear basis to
homologise the larval attachment structure of certain sipunculans with a
pedicle.

\hypertarget{Lingula-coding-21}{}
\emph{Lingula}: The stalk corresponds to the molluscan foot, rather than
a pedicle.

\hypertarget{Mickwitzia_muralensis-coding-21}{}
\emph{Mickwitzia muralensis}: The absence of a pedicle is inferred from
the absence of an internal pedicle tube, and the absence of a pedicle at
the hinge.

\hypertarget{Micrina-coding-21}{}
\emph{Micrina}: ``It seems unlikely that \emph{H. orienta} possessed a
pedicle that attached it to\\
the soft seafloor, like most other Chengjiang brachiopods.'' \ldots{}\\
``The putative pedicle illustrated by Chen \emph{et al}.
\citeyearpar[Figs 4, 6, 7]{Chen2007Reinterpretationof} in fact is the
mold of a three-dimensionally preserved visceral cavity'' --
\citet{Zhang2009Architectureand}.

\hypertarget{Namacalathus-coding-21}{}
\emph{Namacalathus}: There is no obvious way to homologise the
attachment structure with the ventral pedicle of brachiopods.

\hypertarget{Phoronis-coding-21}{}
\emph{Phoronis}: Grows directly onto the substrate.

\hypertarget{Salanygolina-coding-21}{}
\emph{Salanygolina}: An attachment structure is inferred based on the
presence of an opening \citep{Balthasar2004Shellstructure}; this is
assumed to have been homologous with the brachiopod pedicle.

\hypertarget{Terebratulina-coding-21}{}
\emph{Terebratulina}: The stalk is conceivably homologous with the
brachiopod pedicle, but this possibility is impossible to test.

\hypertarget{Tomteluva_perturbata-coding-21}{}
\emph{Tomteluva perturbata}: Has a pedicle, rather than a pedicle sheath
as in \emph{Kutorgina}
\citep{Holmer2018Evolutionarysignificance, Holmer2018Theattachment}.

\hypertarget{Ussunia-coding-21}{}
\emph{Ussunia}: A pedicle was presumably present, but only the foramen
is preserved.

\hypertarget{Wiwaxia_corrugata-coding-21}{}
\emph{Wiwaxia corrugata}: The tube-bearing stalk of phoronids arises as
an eversion of the metastomal sac, a markedly different origin from the
brachiopod pedicle, which arises from a terminal attachment disc
\citep{Young2002}; the structures are of dubious homology.

\hypertarget{NA-coding-21}{}
NA: ``\emph{Paterimitra} is interpreted to have attached to hard
substrates via a pedicle that emerged through the small posterior
opening'' -- \citet{Skovsted2009Thescleritome}.

\hypertarget{NA-coding-21}{}
NA: Presumed present, based on ventral foramen with colleplax.

\subsection*{{[}22{]} Constitution}\label{constitution-1}
\addcontentsline{toc}{subsection}{{[}22{]} Constitution}

\includegraphics{Brachiopod_phylogeny_files/figure-latex/character-mapping-22.pdf}

\begin{quote}
\textbf{Character 22: Pedicle: Constitution}

1: Massive or uniform\\
2: Densely stacked tabular discs\\
Transformational character.
\end{quote}

The pedicle of certain chengjiang rhynchonelliforms comprises ``densely
stacked, three dimensionally preserved, tabular discs''
\citep{Holmer2018Evolutionarysignificance}.\\
This contrasts with the uniform (`massive') pedicles of living taxa.

\hypertarget{Tonicella-coding-22}{}
\emph{Tonicella}: Extant rhynconellid pedicles are massive, consisting
of a thick outer chitinous cuticle, a pedicle epithelium, and a core
composed of collagen fibres and cartilage-like connective tissue
\citep{Holmer2018Evolutionarysignificance}.

\subsection*{{[}23{]} Biomineralization}\label{biomineralization}
\addcontentsline{toc}{subsection}{{[}23{]} Biomineralization}

\includegraphics{Brachiopod_phylogeny_files/figure-latex/character-mapping-23.pdf}

\begin{quote}
\textbf{Character 23: Pedicle: Biomineralization}

1: Absent\\
2: Present\\
Transformational character.
\end{quote}

The pedicle of strophomenates such as \emph{Antigonambonites} is
biomineralized \citep{Holmer2018Evolutionarysignificance}.

\subsection*{{[}24{]} Bulb}\label{bulb}
\addcontentsline{toc}{subsection}{{[}24{]} Bulb}

\includegraphics{Brachiopod_phylogeny_files/figure-latex/character-mapping-24.pdf}

\begin{quote}
\textbf{Character 24: Pedicle: Bulb}

0: Absent\\
1: Present\\
Neomorphic character.
\end{quote}

A bulb is an expanded region of the distal pedicle, often embedded into
the sediment to improve anchorage.

\hypertarget{Clupeafumosus_socialis-coding-24}{}
\emph{Clupeafumosus socialis}: Holmer and Caron
\citeyearpar{Holmer2006Aspinose} interpret the presence of a bulb as
tentative; we score it as ambiguous.

\subsection*{{[}25{]} Distal rootlets}\label{distal-rootlets}
\addcontentsline{toc}{subsection}{{[}25{]} Distal rootlets}

\includegraphics{Brachiopod_phylogeny_files/figure-latex/character-mapping-25.pdf}

\begin{quote}
\textbf{Character 25: Pedicle: Distal rootlets}

0: Absent\\
1: Present\\
Neomorphic character.
\end{quote}

Observed in \emph{Pedunculotheca} and \emph{Bethia}
\citep{Sutton2005Silurianbrachiopods}.

\subsection*{{[}26{]} Tapering}\label{tapering}
\addcontentsline{toc}{subsection}{{[}26{]} Tapering}

\includegraphics{Brachiopod_phylogeny_files/figure-latex/character-mapping-26.pdf}

\begin{quote}
\textbf{Character 26: Pedicle: Tapering}

1: Uniform thickness\\
2: Tapering\\
Transformational character.
\end{quote}

Holmer \emph{et al}. \citeyearpar{Holmer2018Theattachment} remark that
the tapering aspect of the \emph{Nisusia} pedicle recalls that of
certain Chengjiang taxa (\emph{Alisina}, \emph{Longtancunella}) whilst
distinguishing it from many other taxa (\emph{Eichwaldia},
\emph{Bethia}) in which the pedicle is a constant thickness.

\hypertarget{Craniops-coding-26}{}
\emph{Craniops}: Tapered pedicle sheath with holdfast.

\hypertarget{Pelagodiscus_atlanticus-coding-26}{}
\emph{Pelagodiscus atlanticus}: The pedicle thickness does not obviously
change between the apex of the shell and the holdfast.

\subsection*{{[}27{]} Coelomic region}\label{coelomic-region}
\addcontentsline{toc}{subsection}{{[}27{]} Coelomic region}

\includegraphics{Brachiopod_phylogeny_files/figure-latex/character-mapping-27.pdf}

\begin{quote}
\textbf{Character 27: Pedicle: Coelomic region}

1: Absent\\
2: Present\\
Transformational character.
\end{quote}

Certain brachiopods, such as \emph{Acanthotretella}, exhibit a coelomic
cavity within the pedicle or pedicle sheath.

Treated as transformational as it is not clear that either state is
necessarily ancestral.

\hypertarget{Tomteluva_perturbata-coding-27}{}
\emph{Tomteluva perturbata}: A coleomic canal is inferred based on the
ease with which the pedicle is deformed
\citep{Holmer2018Evolutionarysignificance}, but its presence is not
known for certain so is coded ambiguous.

\subsection*{{[}28{]} Surface ornament}\label{surface-ornament-1}
\addcontentsline{toc}{subsection}{{[}28{]} Surface ornament}

\includegraphics{Brachiopod_phylogeny_files/figure-latex/character-mapping-28.pdf}

\begin{quote}
\textbf{Character 28: Pedicle: Surface ornament}

1: Smooth\\
2: Irregular wrinkles\\
3: Regular annulations\\
Transformational character.
\end{quote}

Annulations are regular rings that surround the pedicle, and are
distinguished from wrinkles, which are irregular in magnitude and
spacing, and may branch or fail to entirely encircle the pedicle.

\hypertarget{Clupeafumosus_socialis-coding-28}{}
\emph{Clupeafumosus socialis}: ``The pedicle surface is ornamented with
pronounced annulated rings, disposed at intervals of about 0.2 mm''.

\hypertarget{Coolinia_pecten-coding-28}{}
\emph{Coolinia pecten}: ``It appears that the pedicle lacks a coelomic
space and is distinctly annulated, with densely stacked tabular bodies''
-- \citet{Zhang2011Anobolellate}.

\hypertarget{Craniops-coding-28}{}
\emph{Craniops}: ``The emerging pedicle has a consistent shape in all
the available specimens and is strongly annulated and distally
tapering'' -- \citet{Holmer2018Evolutionarysignificance}.

\hypertarget{Micromitra-coding-28}{}
\emph{Micromitra}: ``Pronounced concentric annular discs disposed at
intervals of 0.6--1.0 mm'' --
\citet{Zhang2007Rhynchonelliformeanbrachiopods}.

\hypertarget{Mummpikia_nuda-coding-28}{}
\emph{Mummpikia nuda}: Regularly annotated \citep[see fig. 14.9
in][]{Hou2017Brachiopoda}.

\hypertarget{Nisusia_sulcata-coding-28}{}
\emph{Nisusia sulcata}: ``The preserved pedicle has condensed
annulations'' -- \citet{Zhang2011Theexceptionally}.

\hypertarget{Tomteluva_perturbata-coding-28}{}
\emph{Tomteluva perturbata}: The ``strong annulations'' vary
significantly in transverse thickness
\citep{Holmer2018Evolutionarysignificance}, so it is not clear whether
these represent true annulations or wrinkles.

\hypertarget{NA-coding-28}{}
NA: Annulations present in median collar.

\subsection*{{[}29{]} Nerve impression}\label{nerve-impression}
\addcontentsline{toc}{subsection}{{[}29{]} Nerve impression}

\includegraphics{Brachiopod_phylogeny_files/figure-latex/character-mapping-29.pdf}

\begin{quote}
\textbf{Character 29: Pedicle: Nerve impression}

0: Absent\\
1: Present\\
Neomorphic character.
\end{quote}

In certain taxa the impression of the pedicle nerve is evident in the
shell. See character 28 in Williams \emph{et al}.
\citeyearpar{Williams1998Thediversity} appendix 1. Care must be taken
not to code an impression as absent when the preservational quality is
insufficient to safely infer a genuine absence. Treated as neomorphic as
the presence of an innervation is considered a derived state.

\hypertarget{Coolinia_pecten-coding-29}{}
\emph{Coolinia pecten}: Not described by
\citet{Williams2000LinguliformeaCraniiformea}.

\hypertarget{Dentalium-coding-29}{}
\emph{Dentalium}: Present in many lingulids
\citep{Williams2000LinguliformeaCraniiformea}, and coded as present in
Lingulidae \citep[table 6]{Williams2000LinguliformeaCraniiformea}.

\hypertarget{Gasconsia-coding-29}{}
\emph{Gasconsia}: Documented by \citet{Skovsted2017Depthrelated}.

\hypertarget{Longtancunella_chengjiangensis-coding-29}{}
\emph{Longtancunella chengjiangensis}, \emph{Micromitra},
\emph{Novocrania}, \emph{Paterimitra}, NA: Following
\citet{Williams1998Thediversity}, appendix 2.

\hypertarget{Mummpikia_nuda-coding-29}{}
\emph{Mummpikia nuda}: Coded as present in Lingulellotretidae
\citep[table 6]{Williams2000LinguliformeaCraniiformea}.

\hypertarget{Serpula-coding-29}{}
\emph{Serpula}: Coded as present in Discinidae \citep[table
6]{Williams2000LinguliformeaCraniiformea}.

\hypertarget{Siphonobolus_priscus-coding-29}{}
\emph{Siphonobolus priscus}: Balthasar
\citeyearpar[p.~274]{Balthasar2008iMummpikia} identifies a canal as a
probable impression of a pedicle nerve.

\hypertarget{Tomteluva_perturbata-coding-29}{}
\emph{Tomteluva perturbata}, \emph{Yuganotheca elegans}: Not reported in
\citet{Williams2000LinguliformeaCraniiformea}.

\hypertarget{Ussunia-coding-29}{}
\emph{Ussunia}: Coded as absent in Acrotretidae \citep[table
6]{Williams2000LinguliformeaCraniiformea}.

\hypertarget{NA-coding-29}{}
NA: Coded as absent in Siphonotretidae \citep[table
6]{Williams2000LinguliformeaCraniiformea}.

\section{Mantle canals}\label{mantle-canals}

\subsection*{{[}30{]} Presence}\label{presence}
\addcontentsline{toc}{subsection}{{[}30{]} Presence}

\includegraphics{Brachiopod_phylogeny_files/figure-latex/character-mapping-30.pdf}

\begin{quote}
\textbf{Character 30: Mantle canals: Presence}

0: Absent\\
1: Present\\
Neomorphic character.
\end{quote}

Whether impressed on a shell or expressed solely in soft tissue.

\hypertarget{Kutorgina_chengjiangensis-coding-30}{}
\emph{Kutorgina chengjiangensis}: Not preserved along muscle scars
\citep{Nikitin1984}, presumably owing to quality of preservation rather
than genuine absence.

\subsection*{{[}31{]} Morphology}\label{morphology-1}
\addcontentsline{toc}{subsection}{{[}31{]} Morphology}

\includegraphics{Brachiopod_phylogeny_files/figure-latex/character-mapping-31.pdf}

\begin{quote}
\textbf{Character 31: Mantle canals: Morphology}

1: Pinnate (=lemniscate)\\
2: Bifurcate\\
3: Baculate\\
4: Saccate\\
Transformational character.
\end{quote}

The morphology of dorsal and ventral canals is identical in all included
taxa, so is assumed not to be independent -- hence the use of a single
character \citep[contra][]{Williams2000LinguliformeaCraniiformea}.

For a description of terms see Williams \emph{et al}.
\citeyearpar[2000]{Williams1997Introduction}.

Pinnate = ``rapidly branch into a number of subequal, radially disposed
canals''\\
Bifurcate = ``\emph{vascula} \emph{lateralia} in both valves divide
immediately after leaving the body cavity''\\
Baculate = ``extend forward without any major dichotomy or bifurcation''
\citep[ p.~418]{Williams1997Introduction}\\
Saccate = ``pouchlike sinuses lying wholly posterior to the arcuate
\emph{vascula} \emph{media}'' (ibid., p412).

\hypertarget{Clupeafumosus_socialis-coding-31}{}
\emph{Clupeafumosus socialis}: Following Table 6, for Siphonotretidae,
in Williams \emph{et al}.
\citeyearpar{Williams2000LinguliformeaCraniiformea}.

\hypertarget{Coolinia_pecten-coding-31}{}
\emph{Coolinia pecten}, \emph{Tomteluva perturbata}: Following Table 15
in Williams \emph{et al}.
\citeyearpar{Williams2000LinguliformeaCraniiformea}.

\hypertarget{Craniops-coding-31}{}
\emph{Craniops}: Not reported in Treatise
\citep{Williams2000LinguliformeaCraniiformea}.

\hypertarget{Dentalium-coding-31}{}
\emph{Dentalium}, \emph{Mummpikia nuda}: Following table 6 in Williams
\emph{et al}. \citeyearpar{Williams2000LinguliformeaCraniiformea}.

\hypertarget{Eccentrotheca-coding-31}{}
\emph{Eccentrotheca}: Not reported in
\citet{Williams2000LinguliformeaCraniiformea}.

\hypertarget{Eoobolus-coding-31}{}
\emph{Eoobolus}, \emph{Micromitra}: Following table 15 in Williams
\emph{et al}. \citeyearpar{Williams2000LinguliformeaCraniiformea} (for
\emph{Neocrania}).

\hypertarget{Gasconsia-coding-31}{}
\emph{Gasconsia}, \emph{Lingulellotreta malongensis}: Following
\citet{Williams1998Thediversity}, appendix 2, and Williams \emph{et al}.
\citeyearpar{Williams2000LinguliformeaCraniiformea}, table 8.

\hypertarget{Glyptoria-coding-31}{}
\emph{Glyptoria}: Not reported from fossil material.

\hypertarget{Heliomedusa_orienta-coding-31}{}
\emph{Heliomedusa orienta}: Williams \emph{et al}. \citeyearpar[table
15]{Williams2000LinguliformeaCraniiformea} appear to use
Palaeotrimerella \citep[as drawn in][]{Williams1997Introduction} as a
model for \emph{Gasconsia}, which pre-supposes a close relationship. We
are not aware of any report of mantle canals from \emph{Gasconsia}
itself.

\hypertarget{Longtancunella_chengjiangensis-coding-31}{}
\emph{Longtancunella chengjiangensis}: Following appendix 2 (char. 21)
in Williams \emph{et al}. \citeyearpar{Williams1998Thediversity}.

\hypertarget{Mickwitzia_muralensis-coding-31}{}
\emph{Mickwitzia muralensis}: Baculate \emph{vascula} \emph{media} --
Balthasar \& Butterfield \citeyearpar{Balthasar2009EarlyCambrian}.

\hypertarget{Micrina-coding-31}{}
\emph{Micrina}: Described as pinnate by Jin \& Wang
\citeyearpar{Jin1992Revisionof}.

\hypertarget{Nisusia_sulcata-coding-31}{}
\emph{Nisusia sulcata}: Reported by Zhang \emph{et al}.
\citeyearpar[2011T]{Zhang2007Agregarious} though the interpretation is
tentative.

\hypertarget{Novocrania-coding-31}{}
\emph{Novocrania}: Described as pinnate (at least in ventral valve) by
Williams \emph{et al}. \citeyearpar[p.~250]{Williams1998Thediversity}.

\hypertarget{Paterimitra-coding-31}{}
\emph{Paterimitra}: Described as saccate by Williams \emph{et al}.
\citeyearpar{Williams1998Thediversity}.

\hypertarget{Serpula-coding-31}{}
\emph{Serpula}: Following table 6, for Discinidae, in Williams \emph{et
al}. \citeyearpar{Williams2000LinguliformeaCraniiformea}.

\hypertarget{Siphonobolus_priscus-coding-31}{}
\emph{Siphonobolus priscus}: ``Poorly resolved'' --
\citet{Balthasar2008iMummpikia}.

\hypertarget{Tonicella-coding-31}{}
\emph{Tonicella}: ``In modern terebratulides, the \emph{vascula}
\emph{media} are subordinate to the lemniscate or pinnate \emph{vascula}
\emph{genitalia}'' -- \citet{Williams1997Introduction}.

\hypertarget{Ussunia-coding-31}{}
\emph{Ussunia}: Following Table 8 (for Acrotreta) in Williams \emph{et
al}. \citeyearpar{Williams2000LinguliformeaCraniiformea}, and the
general pinnate condition for acrotretoids stated in Williams \emph{et
al}. \citeyearpar{Williams1997Introduction}, p.~420.

\hypertarget{Yuganotheca_elegans-coding-31}{}
\emph{Yuganotheca elegans}: Sacculate (sometimes digitate in dorsal
valve) \citep[p716]{Williams2000LinguliformeaCraniiformea}.

\hypertarget{NA-coding-31}{}
NA: Preservation not adequate to evaluate \citep{Streng2016Anew}.

\hypertarget{NA-coding-31}{}
NA: Coded uncertain in appendix 2 in Williams \emph{et al}.
\citeyearpar{Williams1998Thediversity}.

\hypertarget{NA-coding-31}{}
NA: Interpreted as baculate, following
\citet{Havlicek1982LingulaceaPaterinacea}.

\subsection*{\texorpdfstring{{[}32{]} \emph{vascula}
\emph{lateralia}}{{[}32{]} vascula lateralia}}\label{vascula-lateralia}
\addcontentsline{toc}{subsection}{{[}32{]} \emph{vascula}
\emph{lateralia}}

\includegraphics{Brachiopod_phylogeny_files/figure-latex/character-mapping-32.pdf}

\begin{quote}
\textbf{Character 32: Mantle canals: \emph{vascula} \emph{lateralia}}

0: Absent\\
1: Present\\
Neomorphic character.
\end{quote}

We treat the \emph{vascula} \emph{lateralia} as equivalent to the
\emph{vascula} \emph{genitalia} of articulated brachiopods, allowing
phylogenetic analysis to test their proposed homology.

Williams \emph{et al}. \citeyearpar{Williams1997Introduction} write:
``The mantle canal system of most of the organophosphate-shelled species
consists of a single pair of main trunks in the ventral mantle
(\emph{vascula} \emph{lateralia}) and two pairs in the dorsal mantle,
one pair (\emph{vascula} \emph{lateralia}) occupying a similar position
to the single pair in the ventral mantle and a second pair projecting
from the body cavity near the midline of the valve. This latter pair may
be termed the \emph{vascula} \emph{media}, but whether they are strictly
homologous with the \emph{vascula} \emph{media} of articulated
brachiopods is a matter of opinion. It is also impossible to assert that
the \emph{vascula} \emph{lateralia} are the homologues of the
\emph{vascula} \emph{myaria} or \emph{genitalia} of articulated species,
although they are likely to be so as they arise in a comparable
position.''

``In inarticulated brachiopods, two main mantle canals (\emph{vascula}
\emph{lateralia}) emerge from the main body cavity through muscular
valves and bifurcate distally to produce an increasingly dense array of
blindly ending branches near the periphery of the mantle (fig.
71.1--71.2).''

\hypertarget{Clupeafumosus_socialis-coding-32}{}
\emph{Clupeafumosus socialis}: Following table 8 (which records presence
in Siphonotreta) in Williams \emph{et al}.
\citeyearpar{Williams2000LinguliformeaCraniiformea}.

\hypertarget{Coolinia_pecten-coding-32}{}
\emph{Coolinia pecten}, \emph{Micromitra}, \emph{Tomteluva perturbata}:
Following table 15 in Williams \emph{et al}.
\citeyearpar{Williams2000LinguliformeaCraniiformea}.

\hypertarget{Eoobolus-coding-32}{}
\emph{Eoobolus}: Following table 15 in Williams \emph{et al}.
\citeyearpar{Williams2000LinguliformeaCraniiformea} (for
\emph{Neocrania}), who write that ``Holocene craniides have only a
single pair of main trunks in both valves, corresponding to the
\emph{vascula} \emph{lateralia}''. Williams \emph{et al}.
\citeyearpar{Williams2007Supplement} reiterate this position (p.~2875),
at least for the ventral valve.

\hypertarget{Gasconsia-coding-32}{}
\emph{Gasconsia}: Following Popov \citeyearpar{Popov1992TheCambrian}.

\hypertarget{Heliomedusa_orienta-coding-32}{}
\emph{Heliomedusa orienta}: Williams \emph{et al}. \citeyearpar[table
15]{Williams2000LinguliformeaCraniiformea} appear to use
Palaeotrimerella \citep[as drawn in][]{Williams1997Introduction} as a
model for \emph{Gasconsia}, which pre-supposes a close relationship. We
are not aware of any report of mantle canals from \emph{Gasconsia}
itself.

\hypertarget{Micrina-coding-32}{}
\emph{Micrina}: Present: Williams \emph{et al}.
\citeyearpar{Williams2000LinguliformeaCraniiformea}; Jin \& Wang
\citeyearpar{Jin1992Revisionof}.

\hypertarget{Mummpikia_nuda-coding-32}{}
\emph{Mummpikia nuda}: Present
\citep{Williams2000LinguliformeaCraniiformea}.

\hypertarget{Nisusia_sulcata-coding-32}{}
\emph{Nisusia sulcata}: Presence is possible but requires interpretation
that is not unambiguous:

``In the dorsal valve, there can be seen two baculate grooves that arise
from the\\
anterior body wall at an antero-lateral position. These two grooves
(Figs 4H, 5D) could be taken to represent the \emph{vascula}
\emph{lateralia}'' -- \citet{Zhang2007Agregarious}.

\hypertarget{Novocrania-coding-32}{}
\emph{Novocrania}, \emph{Paterimitra}: ``Laurie
\citeyearpar{Laurie1987Themusculature} has shown that arcuate
\emph{vascula} \emph{media} were present in the mantles of both valves
as were pouchlike \emph{vascula} \emph{genitalia}, especially in the
ventral valve'' -- \citet{Williams1997Introduction}.

\hypertarget{Serpula-coding-32}{}
\emph{Serpula}: Following \emph{Lochkothele} (Discinidae), Fig. 43.4a in
Williams \emph{et al}.
\citeyearpar{Williams2000LinguliformeaCraniiformea}.

\hypertarget{Tonicella-coding-32}{}
\emph{Tonicella}, \emph{Yuganotheca elegans}: = \emph{vascula}
\emph{genitalia}.

\hypertarget{Ussunia-coding-32}{}
\emph{Ussunia}: Presence indicated in Table 8 (for Acrotreta) in
Williams \emph{et al}.
\citeyearpar{Williams2000LinguliformeaCraniiformea}.

\hypertarget{NA-coding-32}{}
NA: Preservation not adequate to evaluate \citep{Streng2016Anew}.

\hypertarget{NA-coding-32}{}
NA: Based on the figures and sketches in \citet{Zhang2014Anearly} (and
supplementary material), the mantle canals are interpreted as lateral,
with no clear \emph{vascula} \emph{media} present.

\hypertarget{NA-coding-32}{}
NA: Noted in \emph{Siphonobolus} by Williams \emph{et al}.
\citeyearpar{Williams2000LinguliformeaCraniiformea}, with reference to
Havlicek \citeyearpar{Havlicek1982LingulaceaPaterinacea}.

\subsection*{\texorpdfstring{{[}33{]} \emph{vascula}
\emph{media}}{{[}33{]} vascula media}}\label{vascula-media}
\addcontentsline{toc}{subsection}{{[}33{]} \emph{vascula} \emph{media}}

\includegraphics{Brachiopod_phylogeny_files/figure-latex/character-mapping-33.pdf}

\begin{quote}
\textbf{Character 33: Mantle canals: \emph{vascula} \emph{media}}

0: Absent\\
1: Present (in dorsal valve)\\
Neomorphic character.
\end{quote}

Williams \emph{et al}. \citeyearpar{Williams1997Introduction} note that
in addition to the \emph{vascula} \emph{lateralia}, ``\emph{Discinisca}
has two additional mantle canals emanating from the body cavity into the
dorsal mantle (\emph{vascula} \emph{media}).''

These structures are only evident in the dorsal valve for the included
taxa, so only a single character is necessary.

\hypertarget{Clupeafumosus_socialis-coding-33}{}
\emph{Clupeafumosus socialis}: Following table 6 (for Siphonotretidae)
in Williams \emph{et al}.
\citeyearpar{Williams2000LinguliformeaCraniiformea}.

\hypertarget{Coolinia_pecten-coding-33}{}
\emph{Coolinia pecten}, \emph{Micromitra}, \emph{Tomteluva perturbata}:
Following table 15 in Williams \emph{et al}.
\citeyearpar{Williams2000LinguliformeaCraniiformea}.

\hypertarget{Dentalium-coding-33}{}
\emph{Dentalium}, \emph{Mummpikia nuda}: Following table 6 in Williams
\emph{et al}. \citeyearpar{Williams2000LinguliformeaCraniiformea}.

\hypertarget{Eoobolus-coding-33}{}
\emph{Eoobolus}: Williams \emph{et al}.
\citeyearpar{Williams2000LinguliformeaCraniiformea} write ``Holocene
craniides have only a single pair of main trunks in both valves,
corresponding to the \emph{vascula} \emph{lateralia}'' -- an observation
reflected in their table 15 (for \emph{Neocrania}).\\
But in contrast, \citet{Williams2007Supplement}, p.~2875, identify the
dorsal valve's canals as a \emph{vascula} \emph{media} in living
cranidds (though both are \emph{lateralia} in Ordoviian craniides). This
character is therefore coded as ambiguous.

\hypertarget{Gasconsia-coding-33}{}
\emph{Gasconsia}: Following Popov \citeyearpar[fig.
2]{Popov1992TheCambrian}.

\hypertarget{Heliomedusa_orienta-coding-33}{}
\emph{Heliomedusa orienta}: Williams \emph{et al}. \citeyearpar[table
15]{Williams2000LinguliformeaCraniiformea} appear to use
Palaeotrimerella \citep[as drawn in][]{Williams1997Introduction} as a
model for \emph{Gasconsia}, which pre-supposes a close relationship. We
are not aware of any report of mantle canals from \emph{Gasconsia}
itself.

\hypertarget{Lingulellotreta_malongensis-coding-33}{}
\emph{Lingulellotreta malongensis}: Fig. 5 in
\citet{Balthasar2009Thebrachiopod}.

\hypertarget{Longtancunella_chengjiangensis-coding-33}{}
\emph{Longtancunella chengjiangensis}: Present and divergent
\citep{Williams2000LinguliformeaCraniiformea}.

\hypertarget{Micrina-coding-33}{}
\emph{Micrina}: Present: Williams \emph{et al}.
\citeyearpar{Williams2000LinguliformeaCraniiformea} p162, Jin \& Wang
\citeyearpar{Jin1992Revisionof}.

\hypertarget{Nisusia_sulcata-coding-33}{}
\emph{Nisusia sulcata}: Reported by Zhang \emph{et al}.
\citeyearpar{Zhang2007Agregarious} though the interpretation is
tentative.

\hypertarget{Novocrania-coding-33}{}
\emph{Novocrania}: Following table 6 (for Paterinidae) in Williams
\emph{et al}. \citeyearpar{Williams2000LinguliformeaCraniiformea}.

\hypertarget{Paterimitra-coding-33}{}
\emph{Paterimitra}: Reported by Williams \emph{et al}.
\citeyearpar{Williams1998Thediversity}.

\hypertarget{Serpula-coding-33}{}
\emph{Serpula}: Following table 6 (for Discinidae) in Williams \emph{et
al}. \citeyearpar{Williams2000LinguliformeaCraniiformea}.

\hypertarget{Tonicella-coding-33}{}
\emph{Tonicella}: ``In modern terebratulides, the \emph{vascula}
\emph{media} are subordinate to the lemniscate or pinnate \emph{vascula}
\emph{genitalia}'' -- \citet{Williams1997Introduction} p417.

\hypertarget{Ussunia-coding-33}{}
\emph{Ussunia}: Following \emph{Hadrotreta} schematic in Williams
\emph{et al}. \citeyearpar{Williams2000LinguliformeaCraniiformea}.

\hypertarget{Yuganotheca_elegans-coding-33}{}
\emph{Yuganotheca elegans}: From idealised morphology in Williams
\emph{et al}. \citeyearpar{Williams2000LinguliformeaCraniiformea}.

\hypertarget{NA-coding-33}{}
NA: Preservation not adequate to evaluate \citep{Streng2016Anew}.

\hypertarget{NA-coding-33}{}
NA: Based on the figures and sketches in \citet{Zhang2014Anearly} (and
supplementary material), the mantle canals are interpreted as lateral,
with no clear \emph{vascula} \emph{media} present.

\hypertarget{NA-coding-33}{}
NA: Noted in \emph{Siphonobolus} by Havlicek
\citeyearpar{Havlicek1982LingulaceaPaterinacea}.

\subsection*{\texorpdfstring{{[}34{]} \emph{vascula}
\emph{terminalia}}{{[}34{]} vascula terminalia}}\label{vascula-terminalia}
\addcontentsline{toc}{subsection}{{[}34{]} \emph{vascula}
\emph{terminalia}}

\includegraphics{Brachiopod_phylogeny_files/figure-latex/character-mapping-34.pdf}

\begin{quote}
\textbf{Character 34: Mantle canals: \emph{vascula} \emph{terminalia}}

1: Exclusively marginal (peripheral)\\
2: Directed peripherally and (intero)medially\\
Transformational character.
\end{quote}

Presumed to be connected with setal follicles in life
\citep{Williams1998Thediversity}. See Williams \emph{et al}.
\citeyearpar{Williams2000LinguliformeaCraniiformea} for discussion.

\hypertarget{Clupeafumosus_socialis-coding-34}{}
\emph{Clupeafumosus socialis}: Preservation not clear enough to score
with certainty \citep{Holmer2006Aspinose}.

\hypertarget{Coolinia_pecten-coding-34}{}
\emph{Coolinia pecten}: Interomedial \emph{vascula} \emph{terminalia}
not reported by Williams \emph{et al}.
\citeyearpar{Williams2000LinguliformeaCraniiformea}.

\hypertarget{Dentalium-coding-34}{}
\emph{Dentalium}: Peripheral and medial for all Lingulata
\citep{Williams2000LinguliformeaCraniiformea}.

\hypertarget{Eoobolus-coding-34}{}
\emph{Eoobolus}: Peripheral only
\citep[p.158]{Williams2000LinguliformeaCraniiformea}.

\hypertarget{Gasconsia-coding-34}{}
\emph{Gasconsia}, \emph{Lingulellotreta malongensis}: Following
\citet{Williams1998Thediversity}, appendix 2.

\hypertarget{Longtancunella_chengjiangensis-coding-34}{}
\emph{Longtancunella chengjiangensis}: Following appendix 2 in Williams
\emph{et al}. \citeyearpar{Williams1998Thediversity}.

\hypertarget{Mickwitzia_muralensis-coding-34}{}
\emph{Mickwitzia muralensis}: Strong indication of medially directed
\emph{vascula} \emph{terminalia} from \emph{vascula} \emph{lateralia};
see fig. 1.A1 in \citet{Balthasar2009EarlyCambrian}.

\hypertarget{Micrina-coding-34}{}
\emph{Micrina}: Inferred from Jin \& Wang
\citeyearpar{Jin1992Revisionof}.

\hypertarget{Micromitra-coding-34}{}
\emph{Micromitra}, NA: Coded uncertain in appendix 2 in Williams
\emph{et al}. \citeyearpar{Williams1998Thediversity}.

\hypertarget{Mummpikia_nuda-coding-34}{}
\emph{Mummpikia nuda}: Not described in Williams \emph{et al}.
\citeyearpar{Williams2000LinguliformeaCraniiformea}.

\hypertarget{Novocrania-coding-34}{}
\emph{Novocrania}, \emph{Paterimitra}: Peripheral only
\citep{Williams1998Thediversity, Williams2000LinguliformeaCraniiformea}.

\hypertarget{Serpula-coding-34}{}
\emph{Serpula}: Following \emph{Lochkothele} (Discinidae), fig. 43.4a in
Williams \emph{et al}.
\citeyearpar{Williams2000LinguliformeaCraniiformea}.

\hypertarget{Tonicella-coding-34}{}
\emph{Tonicella}: Following idealised plectolophous terebratulid of Emig
\citeyearpar{Emig1992Functionaldisposition}.

\hypertarget{Yuganotheca_elegans-coding-34}{}
\emph{Yuganotheca elegans}: See schematics in Williams \emph{et al}.
\citeyearpar{Williams2000LinguliformeaCraniiformea}.

\hypertarget{NA-coding-34}{}
NA: Not reported in \citet{Havlicek1982LingulaceaPaterinacea} or
\citet{Williams2000LinguliformeaCraniiformea}.

\section{Perioral tentacular
apparatus}\label{perioral-tentacular-apparatus}

\subsection*{{[}35{]} Presence}\label{presence-1}
\addcontentsline{toc}{subsection}{{[}35{]} Presence}

\includegraphics{Brachiopod_phylogeny_files/figure-latex/character-mapping-35.pdf}

\begin{quote}
\textbf{Character 35: Perioral tentacular apparatus: Presence}

0: Absent\\
1: Present\\
Neomorphic character.
\end{quote}

The lophophore is a ring of tentacles that surrounds the mouth.
\citet{Temereva2017Innervationof} suggests that true lophophores must
also encompass the anus, which excludes the tentacular apparatus of
entoprocts from the definition; as homology between the tentacular
apparatuses of entoprocts and other lophophorates has often been
assumed, we prefer to take a more inclusive stance and code the
structures as potentially homologous.

It is unlikely that the tentacles of annelids and sipunculans correspond
to the lophophore, yet homology is not inconceivable. In order that the
tentacular apparatus of \emph{Haplophrentis} can be compared with both
organs without prejudice, we capture the presence of a tentacular
apparatus in this very broad character, with arguments against homology
reflected in separate transformation series.

\hypertarget{Alisina-coding-35}{}
\emph{Alisina}: The scaphopod captacula is conceivably equivalent to the
tentacular apparatus of other lophotrochozoans. It is developmentally
pre-oral, and has tentatively been homologised with the pre-oral
tentacles of Monoplacophora and Gastropoda \citep{Steiner1992}, though
their musculature and late development suggests instead that they may
derive from the molluscan foot, as do the arms of cephalopods
\citep{Wanninger2002M}.

\hypertarget{Amathia-coding-35}{}
\emph{Amathia}: \citet{Moysiuk2017Hyolithsare}.

\hypertarget{Terebratulina-coding-35}{}
\emph{Terebratulina}: The tentacular crown \citep{Zhang2013} is
interpreted as a lophophore.

\subsection*{{[}36{]} Origin}\label{origin}
\addcontentsline{toc}{subsection}{{[}36{]} Origin}

\includegraphics{Brachiopod_phylogeny_files/figure-latex/character-mapping-36.pdf}

\begin{quote}
\textbf{Character 36: Perioral tentacular apparatus: Origin}

1: Prostomium (i.e.~anterior of larval prototroch)\\
2: Second pair of coelomic sacs, at metamorphosis\\
3: Mid-trunk, prior to metamorphosis\\
Transformational character.
\end{quote}

The tentacles of annelids and sipunculans originate from a dorsal pair
of buds on the prostomium \citep{Adrianov2006}, whereas the brachiopod
lophophore arises from the second pair of coelomic sacs
\citep{Nielsen1991}.

\hypertarget{Alisina-coding-36}{}
\emph{Alisina}: The captacula arise close to the mouth after
metamorphosis \citep{Wanninger2002M}, in a position not dissimilar from
that of the phoronid tentacles \citep{Santagata2004}.

\hypertarget{Eoobolus-coding-36}{}
\emph{Eoobolus}: ``At metamorphosis {[}\ldots{}.{]} the second pair of
coelomic sacs develop small attachment areas at the edge of the dorsal
valve and become the lophophore coelom'' \citep{Nielsen1991}

``The larval lobes are retained during the first steps of metamorphosis
and are\\
subsequently remodeled to form the lophophore and other adult organs''
-- \citet{Altenburger2013}.

\hypertarget{Halkieria_evangelista-coding-36}{}
\emph{Halkieria evangelista}: \citep{Adrianov2006}.

\hypertarget{Lingula-coding-36}{}
\emph{Lingula}: Arising after metamorphosis \citep{Nielsen1971}.

\hypertarget{Phoronis-coding-36}{}
\emph{Phoronis}: The tentacles appear at metamorphosis, seemingly from
below the corona (=prototroch) \citep{Young2002}.

\hypertarget{Tonicella-coding-36}{}
\emph{Tonicella}: Lophophore of \emph{Terebratalia} arises post
metamorphosis \citep{Young2002}; lophophore conceivably arising from
vesicular bodies at base of apical lobe?.

\hypertarget{Wiwaxia_corrugata-coding-36}{}
\emph{Wiwaxia corrugata}: At the posterior of the head, at the late
larval stage \citep{Santagata2004}.

\subsection*{{[}37{]} Tentacle disposition}\label{tentacle-disposition}
\addcontentsline{toc}{subsection}{{[}37{]} Tentacle disposition}

\includegraphics{Brachiopod_phylogeny_files/figure-latex/character-mapping-37.pdf}

\begin{quote}
\textbf{Character 37: Perioral tentacular apparatus: Tentacle
disposition}

1: Single side\\
2: Both sides\\
Transformational character.
\end{quote}

Tentacles may occur along one or both sides of the axis of the
lophophore arm \citep{Carlson1995Phylogeneticrelationships}.

\hypertarget{Alisina-coding-37}{}
\emph{Alisina}: On rim of basal lobe only \citep{Morton1959}.

\hypertarget{Clupeafumosus_socialis-coding-37}{}
\emph{Clupeafumosus socialis}: Preservation insufficient to evaluate
\citep{Holmer2006Aspinose}.

\hypertarget{Coolinia_pecten-coding-37}{}
\emph{Coolinia pecten}, \emph{Mickwitzia muralensis}, \emph{Nisusia
sulcata}: Preservation inadequate.

\hypertarget{Dentalium-coding-37}{}
\emph{Dentalium}, \emph{Eoobolus}, \emph{Serpula}, \emph{Tonicella},
\emph{Wiwaxia corrugata}: Following coding for higher group in
\citet{Carlson1995Phylogeneticrelationships}, appendix 1, character 36.

\hypertarget{Halkieria_evangelista-coding-37}{}
\emph{Halkieria evangelista}: Both sides in tentacle-breathers such as
\emph{Themiste} \citep{Ruppert1995, Adrianov2006}; only one side in
\emph{Sipunculus} \citep{Ruppert1995, Adrianov2006}.

\hypertarget{Lingula-coding-37}{}
\emph{Lingula}: Single side \citep{Nielsen1966}.

\hypertarget{Micrina-coding-37}{}
\emph{Micrina}: ``Each lophophoral arm bears a row of long, slender
flexible tentacles'' -- \citet{Zhang2009Architectureand}.

\hypertarget{Micromitra-coding-37}{}
\emph{Micromitra}: Tentacles ``cannot be confidently demonstrated in the
available specimens.'' --
\citet{Zhang2007Rhynchonelliformeanbrachiopods}.

\hypertarget{Mummpikia_nuda-coding-37}{}
\emph{Mummpikia nuda}: ``The tentacles are clearly visible, and closely
arranged in a single palisade'' -- \citet{Zhang2004Newdata}.

\hypertarget{Phoronis-coding-37}{}
\emph{Phoronis}: Both sides \citep{Schwaha2015, Shunkina2015}.

\hypertarget{Sipunculus-coding-37}{}
\emph{Sipunculus}: Single side \citep{Temereva2016Thenervous}.

\hypertarget{Terebratulina-coding-37}{}
\emph{Terebratulina}: Tentacles seemingly occupy a single side of the
lophophore \citep{Zhang2013}.

\subsection*{{[}38{]} Tentacle rows per side in trocholophe
stage}\label{tentacle-rows-per-side-in-trocholophe-stage}
\addcontentsline{toc}{subsection}{{[}38{]} Tentacle rows per side in
trocholophe stage}

\includegraphics{Brachiopod_phylogeny_files/figure-latex/character-mapping-38.pdf}

\begin{quote}
\textbf{Character 38: Perioral tentacular apparatus: Tentacle rows per
side in trocholophe stage}

0: No additional ablabial row\\
1: Adlabial and ablabial row\\
Neomorphic character.
\end{quote}

After Carlson \citeyearpar{Carlson1995Phylogeneticrelationships},
character 37. Lophophore tentacles are commonly arranged into an
ablabial and adlabial row, with ablabial tentacles sometimes added later
in development.

\hypertarget{Dentalium-coding-38}{}
\emph{Dentalium}, \emph{Serpula}, \emph{Tonicella}, \emph{Wiwaxia
corrugata}: Following coding for higher taxon in Carlson
\citeyearpar{Carlson1995Phylogeneticrelationships}, appendix 1,
character 37.

\hypertarget{Eoobolus-coding-38}{}
\emph{Eoobolus}: Following coding for higher taxon in Carlson
\citeyearpar{Carlson1995Phylogeneticrelationships}, appendix 1,
character 37. Also states in
\citet{Williams2000LinguliformeaCraniiformea}, p.~158.

\hypertarget{Lingula-coding-38}{}
\emph{Lingula}, \emph{Phoronis}: Inapplicable.

\hypertarget{Sipunculus-coding-38}{}
\emph{Sipunculus}: \citep{Temereva2016Thenervous}.

\subsection*{{[}39{]} Tentacle rows per side in post-trocholophe
stage}\label{tentacle-rows-per-side-in-post-trocholophe-stage}
\addcontentsline{toc}{subsection}{{[}39{]} Tentacle rows per side in
post-trocholophe stage}

\includegraphics{Brachiopod_phylogeny_files/figure-latex/character-mapping-39.pdf}

\begin{quote}
\textbf{Character 39: Perioral tentacular apparatus: Tentacle rows per
side in post-trocholophe stage}

0: No additional ablabial row\\
1: Adlabial and ablabial row\\
Neomorphic character.
\end{quote}

After Carlson \citeyearpar{Carlson1995Phylogeneticrelationships},
character 37. Lophophore tentacles are commonly arranged into an
ablabial and adlabial row, with ablabial tentacles sometimes added later
in development (and thus interpreted as a neomorphic addition).

\hypertarget{Clupeafumosus_socialis-coding-39}{}
\emph{Clupeafumosus socialis}: Preservation insufficient to evaluate
\citep{Holmer2006Aspinose}.

\hypertarget{Dentalium-coding-39}{}
\emph{Dentalium}, \emph{Eoobolus}, \emph{Serpula}, \emph{Tonicella},
\emph{Wiwaxia corrugata}: Following coding for higher taxon in Carlson
\citeyearpar{Carlson1995Phylogeneticrelationships}, appendix 1,
character 37.

\hypertarget{Lingula-coding-39}{}
\emph{Lingula}: \citet{Nielsen1966}.

\hypertarget{Mickwitzia_muralensis-coding-39}{}
\emph{Mickwitzia muralensis}: Preservation insufficient to evaluate.

\hypertarget{Micrina-coding-39}{}
\emph{Micrina}: ``The lophophoral arms bear laterofrontal tentacles with
a double row of cilia along their lateral edge, as in extant lingulid
brachiopods'' -- \citet{Zhang2009Architectureand}.

\hypertarget{Micromitra-coding-39}{}
\emph{Micromitra}: Tentacles ``cannot be confidently demonstrated in the
available specimens.'' --
\citet{Zhang2007Rhynchonelliformeanbrachiopods}.

\hypertarget{Mummpikia_nuda-coding-39}{}
\emph{Mummpikia nuda}: Single palisade \citep{Zhang2004Newdata}.

\hypertarget{Sipunculus-coding-39}{}
\emph{Sipunculus}: \citep{Temereva2016Thenervous}.

\hypertarget{Terebratulina-coding-39}{}
\emph{Terebratulina}: Additional row not evident \citep{Zhang2013}.

\hypertarget{NA-coding-39}{}
NA: ``helical lophophore fringed with a single row of thick, widely
spaced, parallel-sided and hollow tentacles'' --
\citet{Zhang2014Anearly}.

\subsection*{{[}40{]} Median tentacle in early
development}\label{median-tentacle-in-early-development}
\addcontentsline{toc}{subsection}{{[}40{]} Median tentacle in early
development}

\includegraphics{Brachiopod_phylogeny_files/figure-latex/character-mapping-40.pdf}

\begin{quote}
\textbf{Character 40: Perioral tentacular apparatus: Median tentacle in
early development}

0: Absent\\
1: Present\\
Neomorphic character.
\end{quote}

Following character 28 in \citet{Carlson1995Phylogeneticrelationships}.
Certain taxa exhibit a median tentacle early in development that is lost
during ontogeny.

\hypertarget{Amathia-coding-40}{}
\emph{Amathia}, \emph{Botsfordia}, \emph{Clupeafumosus socialis},
\emph{Coolinia pecten}, \emph{Craniops}, \emph{Eccentrotheca},
\emph{Heliomedusa orienta}, \emph{Lingulosacculus}, \emph{Longtancunella
chengjiangensis}, \emph{Mickwitzia muralensis}, \emph{Micrina},
\emph{Micromitra}, \emph{Mummpikia nuda}, \emph{Namacalathus},
\emph{Nisusia sulcata}, \emph{Novocrania}, \emph{Orthis},
\emph{Paterimitra}, \emph{Pelagodiscus atlanticus}, \emph{Siphonobolus
priscus}, \emph{Tomteluva perturbata}, \emph{Ussunia}, \emph{Yuganotheca
elegans}, NA, NA, NA, NA: Lophophore ontogeny presently unknown.

\hypertarget{Lingula-coding-40}{}
\emph{Lingula}: \citet{Nielsen1966}.

\subsection*{{[}41{]} Site of tentacle
addition}\label{site-of-tentacle-addition}
\addcontentsline{toc}{subsection}{{[}41{]} Site of tentacle addition}

\includegraphics{Brachiopod_phylogeny_files/figure-latex/character-mapping-41.pdf}

\begin{quote}
\textbf{Character 41: Perioral tentacular apparatus: Site of tentacle
addition}

1: At two points located behind the mouth\\
2: At the tip of each lophophore arm\\
Transformational character.
\end{quote}

Following \citet{Temereva2017Innervationof}.

\hypertarget{Dentalium-coding-41}{}
\emph{Dentalium}, \emph{Eoobolus}, \emph{Phoronis}, \emph{Serpula},
\emph{Sipunculus}, \emph{Tonicella}: Following
\citet{Temereva2017Innervationof}.

\hypertarget{Halkieria_evangelista-coding-41}{}
\emph{Halkieria evangelista}: New branches added at each lateral
extreme, behind mouth \citep{Adrianov2006}.

\hypertarget{Wiwaxia_corrugata-coding-41}{}
\emph{Wiwaxia corrugata}: Following \citet{Temereva2017Innervationof} --
though in larvae, tentacles are added at the tips of the developing
lophophore.

\subsection*{{[}42{]} Innervation}\label{innervation}
\addcontentsline{toc}{subsection}{{[}42{]} Innervation}

\includegraphics{Brachiopod_phylogeny_files/figure-latex/character-mapping-42.pdf}

\begin{quote}
\textbf{Character 42: Perioral tentacular apparatus: Innervation}

1: Extensions of a circumoral nerve ring\\
2: Nerve rings within the tentacle ring itself\\
Transformational character.
\end{quote}

Annelid tentacles are innervated by palp nerves \citep{Orrhage2005};
lophophores ancestrally contained a pair of nerve rings
\citep{Temereva2017Innervationof}.

\hypertarget{Alisina-coding-42}{}
\emph{Alisina}: The captacula each bear an individual nerve fibre
emanating from the cerebral ganglia, which is also associated with the
circumoesophageal nerve ring \citep{SumnerRooney2015}, recalling the
situation in annelids and sipunculans.

\hypertarget{Dailyatia-coding-42}{}
\emph{Dailyatia}: \citet{Orrhage2005}.

\hypertarget{Halkieria_evangelista-coding-42}{}
\emph{Halkieria evangelista}: \citet{Rice1993}.

\hypertarget{Lingula-coding-42}{}
\emph{Lingula}: Tentacle nerves originate laterally from the cerebral
ganglion, branching three times and leading to a single nerve within
each tentacle \citep{Fuchs2006}.

\hypertarget{Phoronis-coding-42}{}
\emph{Phoronis}, \emph{Sipunculus}: Following
\citet{Temereva2017Innervationof}.

\subsection*{{[}43{]} Inner nerve ring}\label{inner-nerve-ring}
\addcontentsline{toc}{subsection}{{[}43{]} Inner nerve ring}

\includegraphics{Brachiopod_phylogeny_files/figure-latex/character-mapping-43.pdf}

\begin{quote}
\textbf{Character 43: Perioral tentacular apparatus: Inner nerve ring}

0: Not reduced (whether present or absent due to absence of lophophore
nerve rings)\\
1: Reduced, weakly developed or absent in adults\\
Neomorphic character.
\end{quote}

Juvenile lophophorates exhibit two nerve rings in the tentacles; one of
these rings is often reduced or lost at adulthood
\citep{Temereva2017Innervationof}.

\hypertarget{Dentalium-coding-43}{}
\emph{Dentalium}: \citet{Temereva2017Thefirst}.

\hypertarget{Eoobolus-coding-43}{}
\emph{Eoobolus}: Probably only a single ring is present, but only
available illustrations are 19th century sketches \citep{Luter2016}.

\hypertarget{Lingula-coding-43}{}
\emph{Lingula}: Nerves present in tentacles, but not forming rings
\citep{Fuchs2006}.

\hypertarget{Phoronis-coding-43}{}
\emph{Phoronis}, \emph{Sipunculus}: Following
\citet{Temereva2017Innervationof}.

\hypertarget{Tonicella-coding-43}{}
\emph{Tonicella}: In \emph{Gryphus} \citep{Temereva2017Thefirst}.

\hypertarget{Wiwaxia_corrugata-coding-43}{}
\emph{Wiwaxia corrugata}: \citep{Temereva2017Innervationof}.

\subsection*{{[}44{]} Outer nerve ring}\label{outer-nerve-ring}
\addcontentsline{toc}{subsection}{{[}44{]} Outer nerve ring}

\includegraphics{Brachiopod_phylogeny_files/figure-latex/character-mapping-44.pdf}

\begin{quote}
\textbf{Character 44: Perioral tentacular apparatus: Outer nerve ring}

0: Not reduced (whether present or absent due to absence of lophophore
nerve rings)\\
1: Reduced, weakly developed or absent in adults\\
Neomorphic character.
\end{quote}

Juvenile lophophorates exhibit two nerve rings in the tentacles; one of
these rings is often reduced or lost at adulthood
\citep{Temereva2017Innervationof}.

\hypertarget{Dentalium-coding-44}{}
\emph{Dentalium}, \emph{Tonicella}: \citet{Temereva2017Innervationof}.

\hypertarget{Eoobolus-coding-44}{}
\emph{Eoobolus}: Probably only a single ring is present, but only
available illustrations are 19th century sketches \citep{Luter2016}.

\hypertarget{Lingula-coding-44}{}
\emph{Lingula}: Nerves present in tentacles, but not forming rings
\citep{Fuchs2006}.

\hypertarget{Phoronis-coding-44}{}
\emph{Phoronis}: ``Most species of bryozoans have only the inner'' nerve
ring -- \citet{Temereva2017Innervationof}.

\hypertarget{Sipunculus-coding-44}{}
\emph{Sipunculus}: Following \citet{Temereva2017Innervationof}; only one
tentacle nerve ring evident in \citet{Temereva2016Thenervous}.

\hypertarget{Wiwaxia_corrugata-coding-44}{}
\emph{Wiwaxia corrugata}: \citet{Temereva2017Thefirst}.

\subsection*{{[}45{]} Musculature}\label{musculature}
\addcontentsline{toc}{subsection}{{[}45{]} Musculature}

\includegraphics{Brachiopod_phylogeny_files/figure-latex/character-mapping-45.pdf}

\begin{quote}
\textbf{Character 45: Perioral tentacular apparatus: Musculature}

1: Outer main tentacle muscle; two pairs of inner longitudinal muscles\\
2: Peripheral muscle fibres\\
3: Core of longitudinal muscle fibres surrounded by circular muscles\\
Transformational character.
\end{quote}

\hypertarget{Alisina-coding-45}{}
\emph{Alisina}: Six to eight elongate muscle cells in core
\citep{Shimek1988}, surrounded by circular muscles \citep{Byrum1994}.

\hypertarget{Dailyatia-coding-45}{}
\emph{Dailyatia}: Peripheral muscle fibres \citep{Hanson1949}.

\hypertarget{Dentalium-coding-45}{}
\emph{Dentalium}, \emph{Eoobolus}, \emph{Serpula}, \emph{Tonicella}:
``Inner coelomic epithelium underlain by muscle fibers, or in the
tentacles, myoepithelial cells.'' -- \citet{Williams1997Introduction}.

\hypertarget{Halkieria_evangelista-coding-45}{}
\emph{Halkieria evangelista}: Peripheral to main tentacle cavity
\citep{Pilger1982}.

\hypertarget{Lingula-coding-45}{}
\emph{Lingula}: Outer main tentacle muscle; two pairs of inner
longitudinal muscles \citep{Fuchs2006}.

\hypertarget{Wiwaxia_corrugata-coding-45}{}
\emph{Wiwaxia corrugata}: \citep{Pardos1991}.

\subsection*{{[}46{]} Forms closed loop}\label{forms-closed-loop}
\addcontentsline{toc}{subsection}{{[}46{]} Forms closed loop}

\includegraphics{Brachiopod_phylogeny_files/figure-latex/character-mapping-46.pdf}

\begin{quote}
\textbf{Character 46: Perioral tentacular apparatus: Forms closed loop}

1: Diverging laterally\\
2: Closed loop\\
Transformational character.
\end{quote}

Whereas the lophophore of crown-group brachiopods typically forms a
closed loop, those of \emph{Haplophrentis} and \emph{Heliomedusa}
diverge laterally \citep{Moysiuk2017Hyolithsare}.

\hypertarget{Dailyatia-coding-46}{}
\emph{Dailyatia}, \emph{Halkieria evangelista}: Growing to encircle
mouth in adults.

\hypertarget{Lingula-coding-46}{}
\emph{Lingula}: \citet{Nielsen1966}.

\hypertarget{Mickwitzia_muralensis-coding-46}{}
\emph{Mickwitzia muralensis}: Two diverging arms of the lophophore are
preserved \citep{Balthasar2009EarlyCambrian}.

\hypertarget{Namacalathus-coding-46}{}
\emph{Namacalathus}: The existence of a lophophore is speculative.

\hypertarget{Nisusia_sulcata-coding-46}{}
\emph{Nisusia sulcata}: Two distinct, diverging arms reconstructed by
\citet{Zhang2007Agregarious}.

\hypertarget{Sipunculus-coding-46}{}
\emph{Sipunculus}: Ends of arms meet to form closed loop
\citep{Temereva2016Thenervous}.

\hypertarget{Terebratulina-coding-46}{}
\emph{Terebratulina}: Tentacles form almost complete circular crown.

\hypertarget{Tomteluva_perturbata-coding-46}{}
\emph{Tomteluva perturbata}: No specimens of \emph{Nisusia} preserve the
lophophore.

\hypertarget{Wiwaxia_corrugata-coding-46}{}
\emph{Wiwaxia corrugata}: Two lophophore arms rather than a single
continuous loop, but with tips close together to form functional loop
\citep{Torrey1901}.

\subsection*{{[}47{]} Coiling direction}\label{coiling-direction}
\addcontentsline{toc}{subsection}{{[}47{]} Coiling direction}

\includegraphics{Brachiopod_phylogeny_files/figure-latex/character-mapping-47.pdf}

\begin{quote}
\textbf{Character 47: Perioral tentacular apparatus: Coiling direction}

1: Anteriad\\
2: Posteriad\\
Transformational character.
\end{quote}

The lophophore arms of \emph{Heliomedusa} and \emph{Haplophrentis} arch
posteriad, rather than anteriad as in lingulids. See
\citet{Zhang2009Architectureand}; \citet{Moysiuk2017Hyolithsare}.

\hypertarget{Clupeafumosus_socialis-coding-47}{}
\emph{Clupeafumosus socialis}, \emph{Mummpikia nuda}: Arms proceed
anteriad before recurving.

\hypertarget{Lingula-coding-47}{}
\emph{Lingula}: Posterior (anal side) of lophophore has short stretch
lacking tentacles.

\hypertarget{Phoronis-coding-47}{}
\emph{Phoronis}, \emph{Sipunculus}: Bryozoan arms reach in anal
(i.e.~posterior) direction \citep{Shunkina2015}.

\hypertarget{Serpula-coding-47}{}
\emph{Serpula}: ``converging anteriorly and coiling anterior to the body
cavity'' -- \citet{Zhang2009Architectureand}.

\hypertarget{Terebratulina-coding-47}{}
\emph{Terebratulina}: Cannot establish without distinguishing gut from
anus.

\hypertarget{Wiwaxia_corrugata-coding-47}{}
\emph{Wiwaxia corrugata}: Coiling in direction of anus (i.e.~posteriad).

\subsection*{{[}48{]} Adjustor muscle}\label{adjustor-muscle}
\addcontentsline{toc}{subsection}{{[}48{]} Adjustor muscle}

\includegraphics{Brachiopod_phylogeny_files/figure-latex/character-mapping-48.pdf}

\begin{quote}
\textbf{Character 48: Perioral tentacular apparatus: Adjustor muscle}

0: Absent\\
1: Present\\
Neomorphic character.
\end{quote}

Following character 55 in Carlson
\citeyearpar{Carlson1995Phylogeneticrelationships}. Not possible to code
in most fossil taxa.

\hypertarget{Amathia-coding-48}{}
\emph{Amathia}, \emph{Botsfordia}, \emph{Clupeafumosus socialis},
\emph{Coolinia pecten}, \emph{Craniops}, \emph{Eccentrotheca},
\emph{Heliomedusa orienta}, \emph{Lingulosacculus}, \emph{Longtancunella
chengjiangensis}, \emph{Mickwitzia muralensis}, \emph{Micrina},
\emph{Micromitra}, \emph{Mummpikia nuda}, \emph{Namacalathus},
\emph{Nisusia sulcata}, \emph{Novocrania}, \emph{Orthis},
\emph{Paterimitra}, \emph{Pelagodiscus atlanticus}, \emph{Siphonobolus
priscus}, \emph{Tomteluva perturbata}, \emph{Ussunia}, \emph{Yuganotheca
elegans}, NA, NA, NA, NA: Preservation not adequate to evaluate presence
or absence of this muscle.

\hypertarget{Dentalium-coding-48}{}
\emph{Dentalium}, \emph{Eoobolus}, \emph{Serpula}, \emph{Tonicella},
\emph{Wiwaxia corrugata}: Following coding for higher taxon in Carlson
\citeyearpar{Carlson1995Phylogeneticrelationships}, appendix 1,
character 55.

\section{Digestive tract}\label{digestive-tract}

\subsection*{{[}49{]} Prominent pharynx}\label{prominent-pharynx}
\addcontentsline{toc}{subsection}{{[}49{]} Prominent pharynx}

\includegraphics{Brachiopod_phylogeny_files/figure-latex/character-mapping-49.pdf}

\begin{quote}
\textbf{Character 49: Digestive tract: Prominent pharynx}

0: Absent\\
1: Present\\
Neomorphic character.
\end{quote}

Hyoliths exhibit a prominent protrusible muscular pharynx at the base of
the lophophore \citep{Moysiuk2017Hyolithsare}. This is considered as
potentially equivalent to the anterior projection of the visceral cavity
in \emph{Heliomedusa}, and, by extension, in \emph{Lingulosacculus} and
Lingulotreta.

\hypertarget{Halkieria_evangelista-coding-49}{}
\emph{Halkieria evangelista}: Eversible pharynx (introvert).

\hypertarget{Lingulellotreta_malongensis-coding-49}{}
\emph{Lingulellotreta malongensis}: Prominent extension of dorsal
visceral platform \citep{Balthasar2009Thebrachiopod}.

\hypertarget{Mickwitzia_muralensis-coding-49}{}
\emph{Mickwitzia muralensis}: The prominent anterior extension of the
visceral area noted by Balthasar \& Butterfield
\citeyearpar{Balthasar2009EarlyCambrian} is considered as potentially
homologous with that of \emph{Heliomedusa}
\citep{Zhang2009Architectureand} and, by extension, \emph{Haplophrentis}
\citep{Moysiuk2017Hyolithsare}.

\hypertarget{Micrina-coding-49}{}
\emph{Micrina}: Corresponding to the ``neck'' of the vase-shaped
visceral cavity reported by \citet{Zhang2009Architectureand}.

\hypertarget{Mummpikia_nuda-coding-49}{}
\emph{Mummpikia nuda}: An anterior projection of the visceral area is
noted by Williams \emph{et al}.
\citeyearpar{Williams2000LinguliformeaCraniiformea} and considered
equivalent to that observed in \emph{Lingulosacculus}
\citep{Balthasar2009EarlyCambrian}.

\hypertarget{NA-coding-49}{}
NA: Possibly present, following interpretation of mouth \citep[see fig.
2c, d in][]{Zhang2014Anearly}.

\subsection*{{[}50{]} Radula}\label{radula}
\addcontentsline{toc}{subsection}{{[}50{]} Radula}

\includegraphics{Brachiopod_phylogeny_files/figure-latex/character-mapping-50.pdf}

\begin{quote}
\textbf{Character 50: Digestive tract: Radula}

0: Absent\\
1: Present\\
Neomorphic character.
\end{quote}

Any apparatus comprising multiple denticulate rows arranged serially in
the sagittal plane is treated as potentially homologous with the
molluscan radula.

\hypertarget{Askepasma_toddense-coding-50}{}
\emph{Askepasma toddense}: \citet{Smith2012M}.

\subsection*{{[}51{]} Oesophageal folds}\label{oesophageal-folds}
\addcontentsline{toc}{subsection}{{[}51{]} Oesophageal folds}

\includegraphics{Brachiopod_phylogeny_files/figure-latex/character-mapping-51.pdf}

\begin{quote}
\textbf{Character 51: Digestive tract: Oesophageal folds}

0: Absent\\
1: Present\\
Neomorphic character.
\end{quote}

Following character 86 in \citet{Giribet2002}.

\hypertarget{Wiwaxia_corrugata-coding-51}{}
\emph{Wiwaxia corrugata}: Ciliated ridge in oesophagus
\citep{Torrey1901}.

\subsection*{{[}52{]} Oral sphincter}\label{oral-sphincter}
\addcontentsline{toc}{subsection}{{[}52{]} Oral sphincter}

\includegraphics{Brachiopod_phylogeny_files/figure-latex/character-mapping-52.pdf}

\begin{quote}
\textbf{Character 52: Digestive tract: Oral sphincter}

0: Absent\\
1: Present\\
Neomorphic character.
\end{quote}

Character 133 in \citet{Grobe2007}.

\hypertarget{Alisina-coding-52}{}
\emph{Alisina}: Present, but secondarily reduced.

\subsection*{{[}53{]} Locomotory cilia}\label{locomotory-cilia}
\addcontentsline{toc}{subsection}{{[}53{]} Locomotory cilia}

\includegraphics{Brachiopod_phylogeny_files/figure-latex/character-mapping-53.pdf}

\begin{quote}
\textbf{Character 53: Digestive tract: Foregut: Locomotory cilia}

0: Absent\\
1: Present\\
Neomorphic character.
\end{quote}

Character 66 in \citet{Haszprunar2000}.

\section{Digestive tract: Midgut}\label{digestive-tract-midgut}

\subsection*{{[}54{]} Subdivisions}\label{subdivisions}
\addcontentsline{toc}{subsection}{{[}54{]} Subdivisions}

\includegraphics{Brachiopod_phylogeny_files/figure-latex/character-mapping-54.pdf}

\begin{quote}
\textbf{Character 54: Digestive tract: Midgut: Subdivisions}

0: Not subdivided\\
1: Functional subdivisions\\
Neomorphic character.
\end{quote}

The molluscan midgut is functionally subdivided into a sorting area
(stomach), digestion area (midgut sac or gland), and transport tube
(intestine). Characters 42 in \citet{Haszprunar2000}, 1.38 in
\citet{SPS1996}.

\hypertarget{Askepasma_toddense-coding-54}{}
\emph{Askepasma toddense}: Subdivided, presumably functionally, but with
some ambiguity {[}\citet{Smith2012M};Smith2014{]}.

\subsection*{{[}55{]} Glands}\label{glands}
\addcontentsline{toc}{subsection}{{[}55{]} Glands}

\includegraphics{Brachiopod_phylogeny_files/figure-latex/character-mapping-55.pdf}

\begin{quote}
\textbf{Character 55: Digestive tract: Midgut: Glands}

0: Absent\\
1: Present\\
Neomorphic character.
\end{quote}

Characters 1.40, 2.30 and 4.59 in \citet{SPS1996}; 42 in
\citet{Haszprunar2000}.

\hypertarget{Askepasma_toddense-coding-55}{}
\emph{Askepasma toddense}: Annex to midgut interpreted as a gland
\citep{Smith2012M}.

\section{Digestive tract: Anus}\label{digestive-tract-anus}

\subsection*{{[}56{]} Presence}\label{presence-2}
\addcontentsline{toc}{subsection}{{[}56{]} Presence}

\includegraphics{Brachiopod_phylogeny_files/figure-latex/character-mapping-56.pdf}

\begin{quote}
\textbf{Character 56: Digestive tract: Anus: Presence}

0: Anus present: through-gut\\
1: Anus lost: digestive tract is blind sac\\
Neomorphic character.
\end{quote}

The digestive tract may either constitute a blind sac, or a through gut
with anus. The loss of an anus is known to be derived within spiralia,
so this character is treated as neomorphic.

\hypertarget{Longtancunella_chengjiangensis-coding-56}{}
\emph{Longtancunella chengjiangensis}: Scored according to familial
level feature.

\hypertarget{Micromitra-coding-56}{}
\emph{Micromitra}: Although ``the possibility of a blind ending may not
be completely eliminated {[}\ldots{}{]} the weight of evidence
{[}\ldots{}{]} leads us to reject the possibility of a blind-ending
intestine'' -- \citet{Zhang2007Rhynchonelliformeanbrachiopods}, p.~1399.

\subsection*{{[}57{]} Location}\label{location}
\addcontentsline{toc}{subsection}{{[}57{]} Location}

\includegraphics{Brachiopod_phylogeny_files/figure-latex/character-mapping-57.pdf}

\begin{quote}
\textbf{Character 57: Digestive tract: Anus: Location}

1: Straight gut with posterior anus\\
2: Anus migrated posteriad to create U-shaped gut\\
3: Anus opening to rear of pedal sole, causing slight U-shape to gut\\
Transformational character.
\end{quote}

``The relative position of the mouth and anus in the larvae of
brachiopods and phoronids is similar: posterior anus and anterior
mouth'' -- \citet{Williams2007Supplement}, p.~2884\\
See also character 6 in \citet{Haszprunar2008}.

\hypertarget{Alisina-coding-57}{}
\emph{Alisina}: The U-shaped gut of scaphopods arises by exaggeration of
the dorsal surface, rather than migration of the anus
\citep{Steiner1992}.

\hypertarget{Micromitra-coding-57}{}
\emph{Micromitra}: ``Five specimens have an exceptionally preserved
digestive tract, dorsally curved, with a putative dorso-terminal anus
located near the proximal end of a pedicle'' --
\citet{Zhang2007Rhynchonelliformeanbrachiopods}.

\hypertarget{Tonicella-coding-57}{}
\emph{Tonicella}: ``In rhynchonelliforms, the gut curves somewhat into a
C-shape and the (blind) anus becomes posteroventral in position.'' --
\citet{Williams2007Supplement}, p.\\
2884.

\subsection*{{[}58{]} Migration: Within ring of
tentacles}\label{migration-within-ring-of-tentacles}
\addcontentsline{toc}{subsection}{{[}58{]} Migration: Within ring of
tentacles}

\includegraphics{Brachiopod_phylogeny_files/figure-latex/character-mapping-58.pdf}

\begin{quote}
\textbf{Character 58: Digestive tract: Anus: Migration: Within ring of
tentacles}

1: Not within ring of tentacles\\
2: Anterior - within ring of feeding tentacles\\
Transformational character.
\end{quote}

A migrated anus may be located laterally or within the lophophore ring
(as in entoprocts).

\hypertarget{Micromitra-coding-58}{}
\emph{Micromitra}: ``Presumed to terminate in a functional anus located
near the proximal end of the pedicle.'' --
\citet{Zhang2007Rhynchonelliformeanbrachiopods}.

\subsection*{{[}59{]} Migration: Position}\label{migration-position}
\addcontentsline{toc}{subsection}{{[}59{]} Migration: Position}

\includegraphics{Brachiopod_phylogeny_files/figure-latex/character-mapping-59.pdf}

\begin{quote}
\textbf{Character 59: Digestive tract: Anus: Migration: Position}

1: Left\\
2: Right\\
3: Dorsally\\
4: Ventrally\\
Transformational character.
\end{quote}

If the anus is not within the ring of tentacles, in which direction is
it oriented?.

\hypertarget{Alisina-coding-59}{}
\emph{Alisina}: An alternative interpretation would be that the
posterior of the scaphopod has been extended to generate the relatively
anterior position of the originally ventral anus.

\hypertarget{Amathia-coding-59}{}
\emph{Amathia}: Opening to the right -- see figures 1, 3, and extended
data 5 in Moysiuk \emph{et al}. \citeyearpar{Moysiuk2017Hyolithsare}.
The text states in error that the anus is to the left of the midline.

\hypertarget{Dentalium-coding-59}{}
\emph{Dentalium}: ``In the lingulids, the {[}intestine{]} follows an
oblique course anteriorly to open at the anus on the right body wall.''
-- \citet{Williams1997Introduction}, p.~89.

\hypertarget{Mickwitzia_muralensis-coding-59}{}
\emph{Mickwitzia muralensis}: ``This same arrangement occurs in \emph{L.
nuda}, with the looped dark line tracking the same course as the
exceptionally preserved guts of Chengjiang lingulellotretids, including
the median position of its posterior loop and the sharp right turn as it
exits the posterior extension of the ventral valve''
\citep[p.310]{Balthasar2009EarlyCambrian}.

\hypertarget{Micromitra-coding-59}{}
\emph{Micromitra}: ``Five specimens have an exceptionally preserved
digestive tract, dorsally curved, with a putative dorso-terminal anus
located near the proximal end of a pedicle'' --
\citet{Zhang2007Rhynchonelliformeanbrachiopods}.

\hypertarget{Mummpikia_nuda-coding-59}{}
\emph{Mummpikia nuda}: ``finally terminating in an anal opening on the
right anterior body wall'' \citep[p.66]{Zhang2007Noteon}.

\hypertarget{Nisusia_sulcata-coding-59}{}
\emph{Nisusia sulcata}: ``The intestine extends posteriorly, and then
turns right to continue as a tortuous strand, finally terminating at the
latero-median position of the anterior body wall'' --
\citet{Zhang2007Agregarious}.

\hypertarget{Phoronis-coding-59}{}
\emph{Phoronis}, \emph{Sipunculus}: Anus remains on ventral surface.
Arguably, rather than the anus migrating, the dorsal surface of the
animal has become extended.

\hypertarget{Tonicella-coding-59}{}
\emph{Tonicella}: ``In rhynchonelliforms, the gut curves somewhat into a
C-shape and the (blind) anus becomes posteroventral in position.'' --
\citet{Williams2007Supplement}, p.\\
2884.

\hypertarget{NA-coding-59}{}
NA: The identification of the ``very poorly impressed possible anus at
the lateral side of the anterior body wall'' is not yet confident, so
this character is coded as not presently available.

\section{Sclerites}\label{sclerites}

\subsection*{{[}60{]} Present in adult}\label{present-in-adult}
\addcontentsline{toc}{subsection}{{[}60{]} Present in adult}

\includegraphics{Brachiopod_phylogeny_files/figure-latex/character-mapping-60.pdf}

\begin{quote}
\textbf{Character 60: Sclerites: Present in adult}

0: Absent\\
1: Present\\
Neomorphic character.
\end{quote}

Plate-like (wider than tall) skeletal elements, whether mineralized or
non-mineralized.\\
The definition deliberately excludes setae (which are taller than wide).

\hypertarget{Acanthotretella_spinosa-coding-60}{}
\emph{Acanthotretella spinosa}, \emph{Alisina}: Molluscan valves are
treated as potential homologues of brachiopod valves.

\hypertarget{Antigonambonites_planus-coding-60}{}
\emph{Antigonambonites planus}: Halkieriid sclerites are interpreted as
potentially homologous with those of \emph{Dailyatia} and hence the
brachiopods \citep{Zhao2017}.

\hypertarget{Askepasma_toddense-coding-60}{}
\emph{Askepasma toddense}: The scales of \emph{Wiwaxia} are treated as
homologous with the chaetae of annelids and brachiopods
\citep{Butterfield1990, Smith2014, Zhang2015}, rather than brachiopod
shell.

\hypertarget{Dailyatia-coding-60}{}
\emph{Dailyatia}: Annelid setae are not considered to represent
potential homologues with the brachiopod shell.

\hypertarget{Halkieria_evangelista-coding-60}{}
\emph{Halkieria evangelista}: Hooks are present, though the absence of
chitin or microvillar impressions indicates that they are not homologous
with those of other lophotrochozoans.

\hypertarget{Namacalathus-coding-60}{}
\emph{Namacalathus}: The mineralized endoskeleton of \emph{Namacalathus}
is not interpreted as a sclerite.

\section{Sclerites: Bivalved {[}61{]}}\label{sclerites-bivalved-61}

\includegraphics{Brachiopod_phylogeny_files/figure-latex/character-mapping-61.pdf}

\begin{quote}
\textbf{Character 61: Sclerites: Bivalved}

0: Scleritomous: without differentiated dorsal and ventral valves\\
1: Bivalved: scleritome dominated by prominent dorsal and ventral
valve\\
Neomorphic character.
\end{quote}

Scleritome dominated by prominent differentiated dorsal and ventral
valves.

\hypertarget{Acanthotretella_spinosa-coding-61}{}
\emph{Acanthotretella spinosa}: As larvae, polyplacophorans exhibit an
anterior and a posterior shell field \citep{Wanninger2002C}; subsequent
subdivision of the posterior field gives rise to the posterior seven
valves. \emph{Tonicella} is thus tentatively coded as `bivalved' to
reflect the potential (if perhaps unlikely) homology with the paired
elements of brachiopods.

\subsection*{{[}62{]} Accessory sclerites
reduced}\label{accessory-sclerites-reduced}
\addcontentsline{toc}{subsection}{{[}62{]} Accessory sclerites reduced}

\includegraphics{Brachiopod_phylogeny_files/figure-latex/character-mapping-62.pdf}

\begin{quote}
\textbf{Character 62: Sclerites: Bivalved: Accessory sclerites reduced}

1: Accessory sclerites not reduced\\
2: Accessory sclerites absent: two valves only\\
Transformational character.
\end{quote}

Taxa in the bivalved condition may retain sclerites as small additional
elements, such as the L-elements of \emph{Paterimitra}
\citep{Skovsted2015Theearly}.

This character is treated as neomorphic, with accessory sclerites
ancestrally present, recognizing the likely origin of brachiozoans (and
Lophotrochozoans more generally) from a scleritomous organism.

\hypertarget{Acanthotretella_spinosa-coding-62}{}
\emph{Acanthotretella spinosa}: The intermediate shell plates arise by
subdivision of the posterior shell field \citep{Wanninger2002C}, and are
thus treated as equivalent to the posterior valve rather than as
distinct elements.\\
The girdle elements are homologous with annelid chaetae / brachiopod
setae \citep{Leise1982}, rather than sclerites.

\hypertarget{Alisina-coding-62}{}
\emph{Alisina}: The scaphopod valve arises posterior of the prototroch
and is thus homologous with the posterior valves of Chiton, assuming
that molluscan shell fields are homologous features.

\hypertarget{Amathia-coding-62}{}
\emph{Amathia}: Coded as ambiguous to recognize the possibility that
helens may correspond to L-elements of \emph{Paterimitra}
\citep{Moysiuk2017Hyolithsare}.

\hypertarget{NA-coding-62}{}
NA: L-sclerites \citep{Skovsted2009Thescleritome}.

\section{Sclerites: Accessory
sclerites}\label{sclerites-accessory-sclerites}

\subsection*{{[}63{]} Arrangement}\label{arrangement}
\addcontentsline{toc}{subsection}{{[}63{]} Arrangement}

\includegraphics{Brachiopod_phylogeny_files/figure-latex/character-mapping-63.pdf}

\begin{quote}
\textbf{Character 63: Sclerites: Accessory sclerites: Arrangement}

0: Single field\\
1: Peripheral and medial fields with distinct sclerite arrangements\\
Neomorphic character.
\end{quote}

Following \citet{Zhao2017}.

\hypertarget{Botsfordia-coding-63}{}
\emph{Botsfordia}: Following the reconstruction of
\citet{Skovsted2015Theearly}.

\subsection*{{[}64{]} Symmetry}\label{symmetry}
\addcontentsline{toc}{subsection}{{[}64{]} Symmetry}

\includegraphics{Brachiopod_phylogeny_files/figure-latex/character-mapping-64.pdf}

\begin{quote}
\textbf{Character 64: Sclerites: Accessory sclerites: Symmetry}

0: Dextral and sinistral forms absent\\
1: Occuring in dextral and sinistral forms\\
Neomorphic character.
\end{quote}

Following \citet{Zhao2017}.

\hypertarget{Amathia-coding-64}{}
\emph{Amathia}: Coded as ambiguous to recognize the possibility that
helens may correspond to L-elements of \emph{Paterimitra}
\citep{Moysiuk2017Hyolithsare}.

\hypertarget{Lingulosacculus-coding-64}{}
\emph{Lingulosacculus}: \citet{Skovsted2008Thescleritome}.

\section{Sclerites: Bivalved}\label{sclerites-bivalved}

\subsection*{{[}65{]} Hinge line shape}\label{hinge-line-shape}
\addcontentsline{toc}{subsection}{{[}65{]} Hinge line shape}

\includegraphics{Brachiopod_phylogeny_files/figure-latex/character-mapping-65.pdf}

\begin{quote}
\textbf{Character 65: Sclerites: Bivalved: Hinge line shape}

1: Astrophic\\
2: Strophic\\
Transformational character.
\end{quote}

\hypertarget{Acanthotretella_spinosa-coding-65}{}
\emph{Acanthotretella spinosa}: A linear hinge articulation does not
exist between valves 1 and 2; nor would it exist between valves 1 and 8
were these adjacent \citep{Connors2012}.

\hypertarget{Antigonambonites_planus-coding-65}{}
\emph{Antigonambonites planus}, \emph{Salanygolina}: Non-strophic.

\hypertarget{Eoobolus-coding-65}{}
\emph{Eoobolus}: Craniides have a strophic posterior valve edge
\citep[table 39 on p.~2853]{Williams2007Supplement}: \emph{Novocrania}'s
``dorsal posterior margin'' is ``straight''
\citep[p.~171]{Williams2000LinguliformeaCraniiformea}.

\hypertarget{Gasconsia-coding-65}{}
\emph{Gasconsia}: Coded as dissociated in Williams \emph{et al}.
\citeyearpar{Williams1998Thediversity}, appendix 2.

\hypertarget{Glyptoria-coding-65}{}
\emph{Glyptoria}: Astrophic: rounded posterior margin \citep[see fig. 91
in][]{Williams2000LinguliformeaCraniiformea}.

\hypertarget{Heliomedusa_orienta-coding-65}{}
\emph{Heliomedusa orienta}: The straight posterior margin of
\emph{Gasconsia} contributes to an overall resemblance with the Chileids
\citep{Holmer2014Ordovician96}.

\hypertarget{Micromitra-coding-65}{}
\emph{Micromitra}: Williams \emph{et al}.
\citeyearpar[p.~208]{Williams2000LinguliformeaCraniiformea} consider the
hinge of \emph{Kutorgina} to be stropic, whereas Bassett \emph{et al}.
\citeyearpar{Bassett2001Functionalmorphology} argue for an astropic
interpretation -- whilst noting that the arrangement is prominently
different from other astrophic taxa. We therefore code this taxon as
ambiguous.

\hypertarget{Nisusia_sulcata-coding-65}{}
\emph{Nisusia sulcata}: ``\emph{Longtancunella} has an oval to
subcircular shell with a very short strophic hinge line'' --
\citet{Zhang2011Theexceptionally}.

\hypertarget{Orthis-coding-65}{}
\emph{Orthis}: Non-strophic: see \citet{Holmer2008TheEarly}.

\hypertarget{Tomteluva_perturbata-coding-65}{}
\emph{Tomteluva perturbata}: ``The strophic, articulated shells of the
Kutorginata rotated on simple hinge mechanisms that are different from
those of other rhynchonelliforms''
\citep[p.~208]{Williams2000LinguliformeaCraniiformea}.

\hypertarget{NA-coding-65}{}
NA: ``Tomteluvid taxa all have a strongly ventribiconvex, astrophic
shell with a unisulcate commissure'' -- \citet{Streng2016Anew}, p5.

\hypertarget{NA-coding-65}{}
NA: Not evident from fossil material; the possibility of a short
strophic hinge line (as in \emph{Longtancunella}) is difficult to
discount.

\subsection*{{[}66{]} Apophyses}\label{apophyses}
\addcontentsline{toc}{subsection}{{[}66{]} Apophyses}

\includegraphics{Brachiopod_phylogeny_files/figure-latex/character-mapping-66.pdf}

\begin{quote}
\textbf{Character 66: Sclerites: Bivalved: Apophyses}

0: Absent\\
1: Present\\
Neomorphic character.
\end{quote}

Many brachiopods, in addition to \emph{Micrina} and others, bear
tooth-like structures or processes that articulate the two primary
valves.\\
Caution must be applied before taxa are coded as ``absent'', as teeth
can be subtle and may be overlooked.

Kutorginata don't have teeth or dental sockets, but their shells are
articulated by ``two triangular plates formed by dorsal interarea,
bearing oblique ridges on the inner sides''
\citep[p.~211]{Williams2000LinguliformeaCraniiformea}; this simple hinge
mechanism is different from other rhynchonelliforms
\citep[p.208]{Williams2000LinguliformeaCraniiformea}, but serves an
equivalent purpose and is thus potentially homologous. We thus code
kutorginids as present, using a subsequent character to capture
difference in tooth morphology.

\hypertarget{Acanthotretella_spinosa-coding-66}{}
\emph{Acanthotretella spinosa}: The sutural laminae correspond in
function and position to brachiopod apophyses \citep{Connors2012}, and
so are coded as potentially homologous.

\hypertarget{Coolinia_pecten-coding-66}{}
\emph{Coolinia pecten}: ``Strophic articulation with paired, ventral
denticles, composed of secondary shell'' -- definition of family
Trematobolidae in \citet{Williams2000LinguliformeaCraniiformea}.

\hypertarget{Heliomedusa_orienta-coding-66}{}
\emph{Heliomedusa orienta}: ``Articulatory structure comprising ventral
cardinal socket and dorsal hinge plate {[}\ldots{}{]} The shape of the
shell probably correlates strongly with the unique type of articulation,
which consists of a dorsal hinge plate that fits tightly into a cardinal
socket in the ventral valve, with a concave homeodeltidium in the center
of the ventral interarea'' --
\citet{Williams2000LinguliformeaCraniiformea}, p.184, concerning order
Trimerellida.

\hypertarget{Kutorgina_chengjiangensis-coding-66}{}
\emph{Kutorgina chengjiangensis}: ``articulatory structures poorly
developed'' -- \citet{Williams2000LinguliformeaCraniiformea}, p.~192.

\hypertarget{Micromitra-coding-66}{}
\emph{Micromitra}: ``Articulation characterized by two triangular plates
formed by dorsal interarea, bearing oblique ridges on the inner sides''
-- \citet{Williams2000LinguliformeaCraniiformea}, p.~211.

\hypertarget{Salanygolina-coding-66}{}
\emph{Salanygolina}: Not reported by or evident in Balthasar
\citeyearpar{Balthasar2004Shellstructure}.

\hypertarget{Siphonobolus_priscus-coding-66}{}
\emph{Siphonobolus priscus}: No articulation structures are evident;
instead, the propareas are rotated inwards
\citep{Balthasar2008iMummpikia}. The definition of Family Obolellidae in
Williams \emph{et al}.
\citeyearpar{Williams2000LinguliformeaCraniiformea} notes that
articulation may be lacking or vestigial in the group.

\hypertarget{Tomteluva_perturbata-coding-66}{}
\emph{Tomteluva perturbata}: Pseudodont articulation: teeth formed by
distal lateral extensions from the ventral pseudodeltidium --
\citet{Holmer2018Evolutionarysignificance}.

\hypertarget{Ussunia-coding-66}{}
\emph{Ussunia}: No articulating processes evident or reported by Topper
\emph{et al}. \citeyearpar{Topper2013Reappraisalof}.

\hypertarget{NA-coding-66}{}
NA: Tomteluvids {[}\ldots{}{]} lack articulation structures such as
teeth and sockets \citep{Streng2016Anew}.

\subsection*{{[}67{]} Apophyses: Morphology}\label{apophyses-morphology}
\addcontentsline{toc}{subsection}{{[}67{]} Apophyses: Morphology}

\includegraphics{Brachiopod_phylogeny_files/figure-latex/character-mapping-67.pdf}

\begin{quote}
\textbf{Character 67: Sclerites: Bivalved: Apophyses: Morphology}

1: Deltidiodont\\
2: Cyrtomatodont\\
3: Pseudodont\\
Transformational character.
\end{quote}

Deltidiodont teeth are simple hinge teeth developed by the distal
accretion of secondary shell; Cyrtomatodont teeth are knoblike or
hook-shaped hinge teeth developed by differential secretion and
resorption of the secondary shell \citep[fig. 322
in][]{Williams1997Introduction}.

Kutorginata (here represented by \emph{Kutorgina} and \emph{Nisusia})
don't have teeth (apophyses) or dental sockets, but their shells are
articulated by ``two triangular plates formed by dorsal interarea,
bearing oblique ridges on the inner sides''
\citep[p.~211]{Williams2000LinguliformeaCraniiformea}; this simple hinge
mechanism is different from other rhynchonelliforms
{[}\citet{Williams2000LinguliformeaCraniiformea}, p.208; table 13
character 30{]}, and is described as a ``pseudodont articulation''
\citep{Holmer2018Evolutionarysignificance}.

\hypertarget{Acanthotretella_spinosa-coding-67}{}
\emph{Acanthotretella spinosa}: Chiton apophyses (sutural laminae) are
accretions deriving from the ventral shell layer of the intermediate and
tail valves \citep{Schwabe2010}, so correspond to the deltidiodont
situation in brachiopods.

\hypertarget{Craniops-coding-67}{}
\emph{Craniops}, \emph{Longtancunella chengjiangensis}: Coded as
deltidiodont in Benedetto \citeyearpar{Benedetto2009iChaniella}.

\hypertarget{Micromitra-coding-67}{}
\emph{Micromitra}: ``Articulation characterized by two triangular plates
formed by dorsal interarea, bearing oblique ridges on the inner sides''
-- \citet{Williams2000LinguliformeaCraniiformea}, p.~211.

\hypertarget{Orthis-coding-67}{}
\emph{Orthis}: The simple knob-like teeth of \emph{Micrina} show no
evidence of resprobtion or the hook-like shape that characterises
Cyrtomatodont teeth.

\hypertarget{Tomteluva_perturbata-coding-67}{}
\emph{Tomteluva perturbata}: The `teeth' are formed by the distal
lateral extensions from the ventral\\
pseudodeltidium fitting into the `sockets' on the inner side of the
dorsal interarea \citep{Holmer2018Evolutionarysignificance}. {[}Coded as
``deltidiodont teeth absent'' in Benedetto
\citeyearpar{Benedetto2009iChaniella}.{]}.

\hypertarget{Tonicella-coding-67}{}
\emph{Tonicella}: Cyrtomatodont -- see fig. 322 in Williams \emph{et
al}. \citeyearpar{Williams2000LinguliformeaCraniiformea}.

\hypertarget{Yuganotheca_elegans-coding-67}{}
\emph{Yuganotheca elegans}: Coded as deltidiodont (in \emph{Eoorthis})
in Benedetto \citeyearpar{Benedetto2009iChaniella}.

\subsection*{{[}68{]} Apophyses: Dental
plates}\label{apophyses-dental-plates}
\addcontentsline{toc}{subsection}{{[}68{]} Apophyses: Dental plates}

\includegraphics{Brachiopod_phylogeny_files/figure-latex/character-mapping-68.pdf}

\begin{quote}
\textbf{Character 68: Sclerites: Bivalved: Apophyses: Dental plates}

0: Absent\\
1: Present\\
Neomorphic character.
\end{quote}

\citet{Williams1997Introduction} (p362) write: ``Teeth {[}\ldots{}{]}
are commonly supported by a pair of variably disposed plates also built
up exclusively of secondary shell and known as dental plates (Fig.
323.1, 323.3).''

Dewing \citeyearpar{Dewing2001Hingemodifications} elaborates: ``Dental
plates are near-vertical, narrow sheets of shell tissue between the
anteromedian edge of the teeth and floor of the ventral valve. They are
a composite structure, resulting from the growth of teeth over the ridge
that bounds the ventral-valve muscle field.''

\citet{Williams2000LinguliformeaCraniiformea} (p.201) write: ``The
denticles lack supporting structures in all Obolellida, but in Naukatida
they are supported by an arcuate plate below the\\
interarea, the anterise (Fig. 119.3a)''.

The anterise is conceivably homologous with the dental plates, thus the
presence of either is coded ``present'' for this character.

\hypertarget{Craniops-coding-68}{}
\emph{Craniops}: Coded as present (well developed) in Benedetto
\citeyearpar{Benedetto2009iChaniella}.

\hypertarget{Eccentrotheca-coding-68}{}
\emph{Eccentrotheca}: Coded as present following Dewing
\citeyearpar{Dewing2001Hingemodifications}, who seems to use the term
Strophomenoids to encompass \emph{Coolinia}, and attests to the presence
of dental plates.

\hypertarget{Heliomedusa_orienta-coding-68}{}
\emph{Heliomedusa orienta}: Coded ambiguous to reflect the possibility
that the hinge plate in trimerellids is homologous to the dental plates
of other taxa, and has replaced the teeth themselves as the primary
articulatory mechanism \citep[see][p.~184, for details of the
articulation]{Williams2000LinguliformeaCraniiformea}.

\hypertarget{Longtancunella_chengjiangensis-coding-68}{}
\emph{Longtancunella chengjiangensis}, \emph{Tomteluva perturbata}:
Coded as absent in Benedetto \citeyearpar{Benedetto2009iChaniella}.

\hypertarget{Yuganotheca_elegans-coding-68}{}
\emph{Yuganotheca elegans}: Coded as present (short and recessive, in
\emph{Eoorthis}) in Benedetto \citeyearpar{Benedetto2009iChaniella}.

\subsection*{{[}69{]} Sockets}\label{sockets}
\addcontentsline{toc}{subsection}{{[}69{]} Sockets}

\includegraphics{Brachiopod_phylogeny_files/figure-latex/character-mapping-69.pdf}

\begin{quote}
\textbf{Character 69: Sclerites: Bivalved: Sockets}

0: Absent\\
1: Present\\
Neomorphic character.
\end{quote}

Simplified from Bassett \emph{et al}.
\citeyearpar{Bassett2001Functionalmorphology} character 16.\\
This character is independent of apophyses, as several taxa bear sockets
without corresponding teeth; the function of these sockets is unknown.\\
See figs 323ff in Williams \emph{et al}.
\citeyearpar{Williams1997Introduction}.

\hypertarget{Coolinia_pecten-coding-69}{}
\emph{Coolinia pecten}: ``bearing sockets, bounded by low ridges'' --
\citet{Williams2000LinguliformeaCraniiformea}.

\hypertarget{Craniops-coding-69}{}
\emph{Craniops}: Coded as present in Benedetto
\citeyearpar{Benedetto2009iChaniella}.

\hypertarget{Heliomedusa_orienta-coding-69}{}
\emph{Heliomedusa orienta}: ``Articulatory structure comprising ventral
cardinal socket and dorsal hinge plate'' --
\citet{Williams2000LinguliformeaCraniiformea}, p.~184.

\hypertarget{Kutorgina_chengjiangensis-coding-69}{}
\emph{Kutorgina chengjiangensis}: Following table 15 in
\citet{Williams2000LinguliformeaCraniiformea}.

\hypertarget{Longtancunella_chengjiangensis-coding-69}{}
\emph{Longtancunella chengjiangensis}, \emph{Tomteluva perturbata}:
Coded as absent in Benedetto \citeyearpar{Benedetto2009iChaniella}.

\hypertarget{Salanygolina-coding-69}{}
\emph{Salanygolina}: Not reported by or evident in Balthasar
\citeyearpar{Balthasar2004Shellstructure}.

\hypertarget{NA-coding-69}{}
NA: Tomteluvids {[}\ldots{}{]} lack articulation structures such as
teeth and sockets \citep{Streng2016Anew}.

\subsection*{{[}70{]} Socket ridges}\label{socket-ridges}
\addcontentsline{toc}{subsection}{{[}70{]} Socket ridges}

\includegraphics{Brachiopod_phylogeny_files/figure-latex/character-mapping-70.pdf}

\begin{quote}
\textbf{Character 70: Sclerites: Bivalved: Socket ridges}

0: Absent\\
1: Present\\
Neomorphic character.
\end{quote}

After Bassett \emph{et al}.
\citeyearpar{Bassett2001Functionalmorphology} character 17. May be
difficult to distinguish from a brachiophore \citep[see Fig 323
in][]{Williams1997Introduction}, so the two structures are not
distinguished here.

\hypertarget{Coolinia_pecten-coding-70}{}
\emph{Coolinia pecten}: ``bearing sockets, bounded by low ridges'' --
\citet{Williams2000LinguliformeaCraniiformea}.

\hypertarget{Craniops-coding-70}{}
\emph{Craniops}: Coded as present in Benedetto
\citeyearpar{Benedetto2009iChaniella}.

\hypertarget{Longtancunella_chengjiangensis-coding-70}{}
\emph{Longtancunella chengjiangensis}, \emph{Tomteluva perturbata}:
Coded as absent in Benedetto \citeyearpar{Benedetto2009iChaniella}.

\hypertarget{NA-coding-70}{}
NA: Tomteluvids {[}\ldots{}{]} lack articulation structures such as
teeth and sockets \citep{Streng2016Anew}.

\subsection*{{[}71{]} Enclosing filtration
chamber}\label{enclosing-filtration-chamber}
\addcontentsline{toc}{subsection}{{[}71{]} Enclosing filtration chamber}

\includegraphics{Brachiopod_phylogeny_files/figure-latex/character-mapping-71.pdf}

\begin{quote}
\textbf{Character 71: Sclerites: Bivalved: Enclosing filtration chamber}

0: No filtration chamber, or open chamber\\
1: Shells close to form enclosed filtration chamber\\
Neomorphic character.
\end{quote}

In crown-group brachiopods, the two primary shells close to form an
enclosed filtration chamber. Further down the stem, taxa such as
\emph{Micrina} do not.

\subsection*{{[}72{]} Commissure}\label{commissure}
\addcontentsline{toc}{subsection}{{[}72{]} Commissure}

\includegraphics{Brachiopod_phylogeny_files/figure-latex/character-mapping-72.pdf}

\begin{quote}
\textbf{Character 72: Sclerites: Bivalved: Commissure}

1: Rectimarginate\\
2: Uniplicate\\
3: Sulcate\\
Transformational character.
\end{quote}

The anterior commissure can be rectimarginate (i.e.~straight),
uniplicate (i.e.~median sulcus in ventral valve), or sulcate (with
median sulcus in dorsal valve).

\hypertarget{Longtancunella_chengjiangensis-coding-72}{}
\emph{Longtancunella chengjiangensis}, \emph{Micromitra},
\emph{Paterimitra}, NA: Following appendix 2 in Williams \emph{et al}.
\citeyearpar{Williams1998Thediversity}.

\hypertarget{Novocrania-coding-72}{}
\emph{Novocrania}: ``ventral valve weakly to moderately sulcate''
\citep{Topper2013Theoldest}; a similar description is provided by
Williams \emph{et al}.
\citeyearpar{Williams2000LinguliformeaCraniiformea}.

\hypertarget{Tonicella-coding-72}{}
\emph{Tonicella}: ``Anterior commissure rectimarginate to uniplicate''
-- uniplicate in fig. 1425.1c of Williams \emph{et al}.
\citeyearpar{Williams2006Rhynchonelliformeapart}.

\subsection*{{[}73{]} Muscle scars: Ventral}\label{muscle-scars-ventral}
\addcontentsline{toc}{subsection}{{[}73{]} Muscle scars: Ventral}

\includegraphics{Brachiopod_phylogeny_files/figure-latex/character-mapping-73.pdf}

\begin{quote}
\textbf{Character 73: Sclerites: Bivalved: Muscle scars: Ventral }

0: Absent\\
1: Present\\
Neomorphic character.
\end{quote}

After Bassett \emph{et al}.
\citeyearpar{Bassett2001Functionalmorphology} character 6.

\hypertarget{Acanthotretella_spinosa-coding-73}{}
\emph{Acanthotretella spinosa}: Absent \citep{Schwabe2010}.

\hypertarget{Antigonambonites_planus-coding-73}{}
\emph{Antigonambonites planus}: Muscle scars are known from the Type A,
but not Type B, morphs of the halkieriid \emph{Oikozetetes}
\citep{Paterson2009, Jacquet2014}.

\hypertarget{Coolinia_pecten-coding-73}{}
\emph{Coolinia pecten}: Muscle scars scored based on \emph{Alisina}
\emph{comleyensis} \citep{Bassett2001Functionalmorphology}.

\hypertarget{Kutorgina_chengjiangensis-coding-73}{}
\emph{Kutorgina chengjiangensis}: Scars preserved on ventral valve
\citep{Nikitin1984}.

\hypertarget{Orthis-coding-73}{}
\emph{Orthis}: Prominent ventral muscle scars -- see e.g.
\citet{Holmer2008TheEarly}, fig. 1f.

\hypertarget{Salanygolina-coding-73}{}
\emph{Salanygolina}: Scars absent; instead, cones ornament shell's
internal surface.

\subsection*{{[}74{]} Muscle scars: Ventral:
Position}\label{muscle-scars-ventral-position}
\addcontentsline{toc}{subsection}{{[}74{]} Muscle scars: Ventral:
Position}

\includegraphics{Brachiopod_phylogeny_files/figure-latex/character-mapping-74.pdf}

\begin{quote}
\textbf{Character 74: Sclerites: Bivalved: Muscle scars: Ventral:
Position}

1: Posterolateral and medial attachments\\
2: Medial attachments only\\
Transformational character.
\end{quote}

Muscles can attach to the ventral valve posterolaterally to, as well as
between, the \emph{vascula} \emph{lateralia}
\citep{Popov1992TheCambrian}.

\hypertarget{Clupeafumosus_socialis-coding-74}{}
\emph{Clupeafumosus socialis}: ``Individual muscle scars cannot be
distinguished'' -- \citet{Holmer2006Aspinose}.

\hypertarget{Coolinia_pecten-coding-74}{}
\emph{Coolinia pecten}: Following reconstruction of Gorjansky \& Popov
\citeyearpar{Gorjansky1986Onthe}.

\hypertarget{Eoobolus-coding-74}{}
\emph{Eoobolus}: Posterior adductor muscles attach posterolaterally to
ventral mantle canal \citep{Robinson2014Themuscles}.

\hypertarget{Glyptoria-coding-74}{}
\emph{Glyptoria}: See fig. 89 in Williams \emph{et al}.
\citeyearpar{Williams2000LinguliformeaCraniiformea}.

\hypertarget{Heliomedusa_orienta-coding-74}{}
\emph{Heliomedusa orienta}: Musculature described in Hanken \& Harper
\citeyearpar{Hanken1985Thetaxonomy}.

\hypertarget{Kutorgina_chengjiangensis-coding-74}{}
\emph{Kutorgina chengjiangensis}: Internal anatomy not adequately
preserved to evaluate \citep{Nikitin1984}.

\hypertarget{Lingulellotreta_malongensis-coding-74}{}
\emph{Lingulellotreta malongensis}: The `laterals' of Balthasar
\citeyearpar[fig. 5]{Balthasar2009Thebrachiopod} are situated almost
upon the \emph{vascula} \emph{lateralia}; they are interpreted as
sitting posterolateral to them.

\hypertarget{Longtancunella_chengjiangensis-coding-74}{}
\emph{Longtancunella chengjiangensis}: Posterolateral reflected by
diductor attachments; see fig. 18.3.2 in
\citet{Bassett2001Functionalmorphology}.

\hypertarget{Micromitra-coding-74}{}
\emph{Micromitra}: Following situation in \emph{Nisusia}; see fig. 18.2
in Bassett \emph{et al}. \citeyearpar{Bassett2001Functionalmorphology}.

\hypertarget{Mummpikia_nuda-coding-74}{}
\emph{Mummpikia nuda}: See fig. 5 in \citet{Holmer1997EarlyCambrian}.

\hypertarget{Novocrania-coding-74}{}
\emph{Novocrania}: Restricted to medial field, following the
interpretation of the musculature presented by Williams \emph{et al}.
\citeyearpar{Williams2000LinguliformeaCraniiformea}, fig. 81.

\hypertarget{Paterimitra-coding-74}{}
\emph{Paterimitra}: Posteriomedial muscle field \citep[text-fig.
6]{Williams1998Thediversity} treated as equivalent to posterolateral
muscles.

\hypertarget{Serpula-coding-74}{}
\emph{Serpula}: Inapplicable as vascular system not directly equivalent
to the canonical; see. fig 6b in Balthasar
\citeyearpar{Balthasar2009Thebrachiopod}.

\hypertarget{Tomteluva_perturbata-coding-74}{}
\emph{Tomteluva perturbata}: Posterolateral diductors \citep[fig. 18.2
in][]{Bassett2001Functionalmorphology}.

\hypertarget{Ussunia-coding-74}{}
\emph{Ussunia}: Coded following \emph{Hadrotreta}, as illustrated in
Popov \citeyearpar{Popov1992TheCambrian}.

\hypertarget{Yuganotheca_elegans-coding-74}{}
\emph{Yuganotheca elegans}: Not applicable: \emph{vascula}
\emph{lateralia} not comparable to those of other taxa.

\hypertarget{NA-coding-74}{}
NA: Ventral musculature not clearly constrained
\citep{Holmer2009Theenigmatic}.

\hypertarget{NA-coding-74}{}
NA: Coded following general siphonotretid condition described by Popov
\citeyearpar[p.~407]{Popov1992TheCambrian}.

\subsection*{{[}75{]} Muscle scars:
Adjustor}\label{muscle-scars-adjustor}
\addcontentsline{toc}{subsection}{{[}75{]} Muscle scars: Adjustor}

\includegraphics{Brachiopod_phylogeny_files/figure-latex/character-mapping-75.pdf}

\begin{quote}
\textbf{Character 75: Sclerites: Bivalved: Muscle scars: Adjustor}

0: Absent\\
1: Present\\
Neomorphic character.
\end{quote}

After Bassett \emph{et al}.
\citeyearpar{Bassett2001Functionalmorphology} character 7.\\
This character is contingent on the presence of a pedicle. Extreme
caution must be used in inferring an absent state, as adjustor scars can
be extremely difficult to distinguish from the adductor scars.

\hypertarget{Coolinia_pecten-coding-75}{}
\emph{Coolinia pecten}: Muscle scars scored based on \emph{Alisina}
\emph{comleyensis} \citep{Bassett2001Functionalmorphology}. The presence
of an adjustor is marked as not presently available, as it is not clear
that a scar, if present, could be distinguished from the diminutive
muscle scars present.

\hypertarget{Gasconsia-coding-75}{}
\emph{Gasconsia}: Not described in Popov
\citeyearpar{Popov1992TheCambrian}.

\hypertarget{Heliomedusa_orienta-coding-75}{}
\emph{Heliomedusa orienta}: No mention of an adjustor muscle in
\emph{Gasconsia} or Trimerellida more generally on pp.~184--185 of
\citet{Williams2000LinguliformeaCraniiformea}, nor in discussion in
\citet{Williams2007Supplement} (p.~2850). Coded as absent.

\hypertarget{Novocrania-coding-75}{}
\emph{Novocrania}: Following the interpretation of the musculature
presented by Williams \emph{et al}.
\citeyearpar{Williams2000LinguliformeaCraniiformea}, fig. 81.

\hypertarget{Salanygolina-coding-75}{}
\emph{Salanygolina}: Scars absent; instead, cones ornament shell's
internal surface.

\hypertarget{Ussunia-coding-75}{}
\emph{Ussunia}: Not known in any acrotretid
\citep{Williams2000LinguliformeaCraniiformea}; not evident in
\emph{Clupeafumosus} \citep{Topper2013Reappraisalof}.

\hypertarget{NA-coding-75}{}
NA: Ventral musculature poorly constrained
\citep{Williams2000LinguliformeaCraniiformea, Popov2009Earlyontogeny}.

\subsection*{{[}76{]} Muscle scars: Dorsal
adductors}\label{muscle-scars-dorsal-adductors}
\addcontentsline{toc}{subsection}{{[}76{]} Muscle scars: Dorsal
adductors}

\includegraphics{Brachiopod_phylogeny_files/figure-latex/character-mapping-76.pdf}

\begin{quote}
\textbf{Character 76: Sclerites: Bivalved: Muscle scars: Dorsal
adductors}

1: Dispersed\\
2: Radially arranged\\
3: Quadripartite\\
Transformational character.
\end{quote}

After Bassett \emph{et al}.
\citeyearpar{Bassett2001Functionalmorphology} character 8, and Williams
\emph{et al}. {[}\citet{Williams1996Asupra}, character 35; 2000, p.~160,
character 54{]}

In the dorsal valve, the anterior and posterior adductor scars of
articulated brachiopods form a single (quadripartite) muscle field
\citep[p.~201]{Williams2000LinguliformeaCraniiformea}

In contrast, the anterior and posterior scars of e.g.~trimerellids have
prominently separate attachment points, with anterior and posterior
muscle fields clearly distinct, and coded as ``dispersed''.

In e.g.~kutorginates, adductor muscles are separated into left and right
fields; the same is the case in lingulids, where there are more separate
muscle groups and the left and right fields conspire to produce a radial
arrangement; both of these configurations are scored as ``radially
arranged''.

\hypertarget{Amathia-coding-76}{}
\emph{Amathia}: Laterally dispersed, based on interpretation of Moysiuk
\emph{et al}. \citeyearpar{Moysiuk2017Hyolithsare}, and consistent with
general situation in hyoliths \citep[see][]{Dzik1980Ontogenyof}.

\hypertarget{Antigonambonites_planus-coding-76}{}
\emph{Antigonambonites planus}: It is unclear whether the paired muscle
scars of \emph{Oikozetetes} may be homologous to brachiopod adductors.

\hypertarget{Coolinia_pecten-coding-76}{}
\emph{Coolinia pecten}: Following Williams \emph{et al}.
\citeyearpar{Williams2000LinguliformeaCraniiformea} table 15 (their
character 54).

\hypertarget{Craniops-coding-76}{}
\emph{Craniops}: Treatise.

\hypertarget{Eccentrotheca-coding-76}{}
\emph{Eccentrotheca}: ``radially arranged adductor scars'' --
\citet{Bassett2017Earliestontogeny}, p1.

\hypertarget{Eoobolus-coding-76}{}
\emph{Eoobolus}: Craniids scored as ``open, quadripartite'' by Williams
\emph{et al}. \citeyearpar{Williams1996Asupra}.

\hypertarget{Gasconsia-coding-76}{}
\emph{Gasconsia}: Following \citet{Williams1998Thediversity}, appendix
2.

\hypertarget{Heliomedusa_orienta-coding-76}{}
\emph{Heliomedusa orienta}: Following the coding of Williams \emph{et
al}. \citeyearpar{Williams2000LinguliformeaCraniiformea}, table 15.

\hypertarget{Kutorgina_chengjiangensis-coding-76}{}
\emph{Kutorgina chengjiangensis}: Following coding with state 0
(dispersed) in table 15 in
\citet{Williams2000LinguliformeaCraniiformea}.

\hypertarget{Longtancunella_chengjiangensis-coding-76}{}
\emph{Longtancunella chengjiangensis}: Scored as ``dispersed'' by
Williams \emph{et al}. \citeyearpar{Williams1998Thediversity} \ldots{}
but then so is \emph{Kutorgina}, which Bassett \emph{et al}.
\citeyearpar{Bassett2001Functionalmorphology} score as radial.

Williams \emph{et al}.
\citeyearpar{Williams2000LinguliformeaCraniiformea} state, for
superfamily Protorthida, ``dorsal adductor scars probably linear'',
which fits in the category of ``radial'' employed herein -- so that's
what we follow.

\hypertarget{Micrina-coding-76}{}
\emph{Micrina}: Distinct anterior and posterior fields
\citep{Chen2007Reinterpretationof}; coded as ``dispersed'' by Williams
\emph{et al}. \citeyearpar{Williams2000LinguliformeaCraniiformea} in
table 15.

\hypertarget{Novocrania-coding-76}{}
\emph{Novocrania}: Separate left and right fields, so radially arranged
-- following the interpretation of the musculature presented by Williams
\emph{et al}. \citeyearpar{Williams2000LinguliformeaCraniiformea}, fig.
81.

\hypertarget{Paterimitra-coding-76}{}
\emph{Paterimitra}: Williams \emph{et al}.
\citeyearpar{Williams1998Thediversity} code as ``dispersed'', but have a
less divided scheme of character states and disagree with other sources
in some codings \citep[e.g.][in
Kutorginates]{Bassett2001Functionalmorphology}. Williams \emph{et al}.
\citeyearpar{Williams2000LinguliformeaCraniiformea} do not describe
\emph{Micromitra} musculature and we were unable to find any reliable
description of the scars, so we code as ``not presently available''.

\hypertarget{Salanygolina-coding-76}{}
\emph{Salanygolina}: Scars absent; instead, cones ornament shell's
internal surface.

\hypertarget{Serpula-coding-76}{}
\emph{Serpula}: Discinids scored as ``open, quadripartite'' by Williams
\emph{et al}. \citeyearpar{Williams1996Asupra}.

\hypertarget{Tonicella-coding-76}{}
\emph{Tonicella}: Coded as ``grouped, quadripartite'' by Williams
\emph{et al}. \citeyearpar{Williams1996Asupra}.

\hypertarget{Ussunia-coding-76}{}
\emph{Ussunia}: Following reconstruction of \emph{Hadrotreta} by
Williams \citeyearpar{Williams2000LinguliformeaCraniiformea}, fig. 51,
which exhibits distinct left and right fields.

\hypertarget{NA-coding-76}{}
NA: ``The dorsal valve of \emph{Salanygolina} has a radial arrangement
of adductor muscle scars and the scars of posteromedially placed
internal oblique muscles, which are also characteristic of paterinates
and chileates'' -- Holmer \emph{et al}.
\citeyearpar{Holmer2009Theenigmatic}.

\hypertarget{NA-coding-76}{}
NA: Ventral musculature poorly constrained
\citep{Williams2000LinguliformeaCraniiformea, Popov2009Earlyontogeny}.

\subsection*{{[}77{]} Muscle scars: Adductors:
Position}\label{muscle-scars-adductors-position}
\addcontentsline{toc}{subsection}{{[}77{]} Muscle scars: Adductors:
Position}

\includegraphics{Brachiopod_phylogeny_files/figure-latex/character-mapping-77.pdf}

\begin{quote}
\textbf{Character 77: Sclerites: Bivalved: Muscle scars: Adductors:
Position}

1: Oblique\\
2: At high angle\\
Transformational character.
\end{quote}

Position of adductor muscles relative to commissural plane.\\
After Bassett \emph{et al}.
\citeyearpar{Bassett2001Functionalmorphology} character 11.

\hypertarget{Eccentrotheca-coding-77}{}
\emph{Eccentrotheca}: Not reported by Williams \emph{et al}.
\citeyearpar{Williams2000LinguliformeaCraniiformea}, nor Bassett \&
Popov \citeyearpar{Bassett2017Earliestontogeny}, nor explicitly by
Dewing \citeyearpar{Dewing2001Hingemodifications}.

\hypertarget{Gasconsia-coding-77}{}
\emph{Gasconsia}: Following description of Popov
\citeyearpar{Popov1992TheCambrian}.

\hypertarget{Heliomedusa_orienta-coding-77}{}
\emph{Heliomedusa orienta}: See discussion under Trimerellida in
Williams \emph{et al}.
\citeyearpar{Williams2000LinguliformeaCraniiformea}.

\hypertarget{Lingulellotreta_malongensis-coding-77}{}
\emph{Lingulellotreta malongensis}: ``\emph{Eoobolus} should have
anterior and posterior adductors and a variety of oblique muscles which
were probably arranged in criss-crossing pairs'' --
\citet{Balthasar2009Thebrachiopod}.

\hypertarget{Novocrania-coding-77}{}
\emph{Novocrania}: Following the interpretation of the musculature
presented by Williams \emph{et al}.
\citeyearpar{Williams2000LinguliformeaCraniiformea}, fig. 81.

\hypertarget{Salanygolina-coding-77}{}
\emph{Salanygolina}: Scars absent; instead, cones ornament shell's
internal surface.

\hypertarget{Serpula-coding-77}{}
\emph{Serpula}: Musculature considered essentially equivalent to
\emph{Lingula} by \citet{Williams2000LinguliformeaCraniiformea}, so
\emph{Lingula} coding followed here.

\hypertarget{NA-coding-77}{}
NA: Ventral musculature poorly constrained
\citep{Williams2000LinguliformeaCraniiformea, Popov2009Earlyontogeny}.

\subsection*{{[}78{]} Muscle scars: Dermal
muscles}\label{muscle-scars-dermal-muscles}
\addcontentsline{toc}{subsection}{{[}78{]} Muscle scars: Dermal muscles}

\includegraphics{Brachiopod_phylogeny_files/figure-latex/character-mapping-78.pdf}

\begin{quote}
\textbf{Character 78: Sclerites: Bivalved: Muscle scars: Dermal muscles}

0: Absent or weakly developed\\
1: Strongly developed\\
Neomorphic character.
\end{quote}

Based on character 11 in Zhang \emph{et al}.
\citeyearpar{Zhang2014Anearly}.\\
Well developed dermal muscles present in the body wall of recent
lingulates, which are absent in all calcareous-shelled brachiopods.
These muscles are responsible for the hydraulic shell-opening mechanism,
and possibly present in all organophosphatic-shelled brachiopods, with
the possible exception of the paterinates
\citep[p.~32]{Williams2000LinguliformeaCraniiformea}.

\hypertarget{Coolinia_pecten-coding-78}{}
\emph{Coolinia pecten}, \emph{Craniops}, \emph{Heliomedusa orienta},
\emph{Longtancunella chengjiangensis}, \emph{Tomteluva perturbata},
\emph{Yuganotheca elegans}, NA: According to the statement of Williams
\emph{et al}. \citeyearpar[p.~32]{Williams2000LinguliformeaCraniiformea}
that these muscle are absent in all carbonate- shelled brachiopods.

\hypertarget{Eccentrotheca-coding-78}{}
\emph{Eccentrotheca}: According to the statement of Williams \emph{et
al}. \citeyearpar[p.~32]{Williams2000LinguliformeaCraniiformea} that
these muscle are absent in all carbonate-shelled brachiopods.

\hypertarget{Eoobolus-coding-78}{}
\emph{Eoobolus}: Following Zhang \emph{et al}.
\citeyearpar{Zhang2014Anearly}, and the statement of Williams \emph{et
al}. \citeyearpar{Williams2000LinguliformeaCraniiformea} that such
muscles are absent in all calcite-shelled brachiopods.

\hypertarget{Gasconsia-coding-78}{}
\emph{Gasconsia}: Implicitly taken as present in Popov
\citeyearpar{Popov1992TheCambrian}, though not marked in diagrams --
suggesting not strongly developed.

\hypertarget{Lingulellotreta_malongensis-coding-78}{}
\emph{Lingulellotreta malongensis}: Not remarked upon by Balthasar
\citeyearpar{Balthasar2009Thebrachiopod}.

\hypertarget{Micromitra-coding-78}{}
\emph{Micromitra}: According to the statement of Williams \emph{et al}.
\citeyearpar[p.~32]{Williams2000LinguliformeaCraniiformea} that these
muscle are absent in all carbonate- shelled brachiopods, and the coding
for kutorginids in Zhang \emph{et al}. \citeyearpar{Zhang2014Anearly}.

\hypertarget{Novocrania-coding-78}{}
\emph{Novocrania}: According to the statement of Williams \emph{et al}.
\citeyearpar[p.~32]{Williams2000LinguliformeaCraniiformea} that the
presence of these muscles in paterinates is uncertain.

\hypertarget{Paterimitra-coding-78}{}
\emph{Paterimitra}: Williams \emph{et al}.
\citeyearpar[p.~32]{Williams2000LinguliformeaCraniiformea} are uncertain
about the presence of these muscles in the paterinates. Zhang \emph{et
al}. \citeyearpar{Zhang2014Anearly} code absence in Paterinida, but
without specifying evidence; we follow their coding here.

\hypertarget{Serpula-coding-78}{}
\emph{Serpula}: Musculature considered essentially equivalent to
\emph{Lingula} by \citet{Williams2000LinguliformeaCraniiformea}, so
\emph{Lingula} coding followed here.

\hypertarget{Siphonobolus_priscus-coding-78}{}
\emph{Siphonobolus priscus}, NA: Though Williams \emph{et al}.
\citeyearpar[p.~32]{Williams2000LinguliformeaCraniiformea} state that
these muscles are absent in all carbonate-shelled brachiopods, their
existence cannot be discounted with certainty in this taxon, which is
therefore coded not presently available.

\hypertarget{Tonicella-coding-78}{}
\emph{Tonicella}: Williams \emph{et al}.
\citeyearpar[p.~32]{Williams2000LinguliformeaCraniiformea} state that
these muscles are absent in all carbonate-shelled brachiopods.

\hypertarget{Ussunia-coding-78}{}
\emph{Ussunia}: This character is coded based on the score of Acrotreta
in Zhang \emph{et al}. \citeyearpar{Zhang2014Anearly}, and statement in
Williams \emph{et al}.
\citeyearpar[P.32]{Williams2000LinguliformeaCraniiformea}.

\hypertarget{NA-coding-78}{}
NA: Ventral musculature poorly constrained
\citep{Williams2000LinguliformeaCraniiformea, Popov2009Earlyontogeny}.

\subsection*{\texorpdfstring{{[}79{]} Muscle scars: Unpaired median
(\emph{levator
ani})}{{[}79{]} Muscle scars: Unpaired median (levator ani)}}\label{muscle-scars-unpaired-median-levator-ani}
\addcontentsline{toc}{subsection}{{[}79{]} Muscle scars: Unpaired median
(\emph{levator ani})}

\includegraphics{Brachiopod_phylogeny_files/figure-latex/character-mapping-79.pdf}

\begin{quote}
\textbf{Character 79: Sclerites: Bivalved: Muscle scars: Unpaired median
(\emph{levator ani})}

0: Absent\\
1: Present\\
Neomorphic character.
\end{quote}

The \emph{levator ani} is a diminutive unpaired medial muscle found in
certain calcitic brachiopods
{[}\citet{Williams2000LinguliformeaCraniiformea}; see fig. 89, character
34 in table 13{]}.

\hypertarget{Coolinia_pecten-coding-79}{}
\emph{Coolinia pecten}, \emph{Micromitra}, \emph{Tomteluva perturbata}:
Following table 13 in \citet{Williams2000LinguliformeaCraniiformea}.

\hypertarget{Eccentrotheca-coding-79}{}
\emph{Eccentrotheca}: Not reported in Dewing
\citeyearpar{Dewing2001Hingemodifications}.

\hypertarget{Eoobolus-coding-79}{}
\emph{Eoobolus}: Following table 13 in
\citet{Williams2000LinguliformeaCraniiformea} (for \emph{Novocrania}).

\hypertarget{Glyptoria-coding-79}{}
\emph{Glyptoria}: See fig. 90 in
\citet{Williams2000LinguliformeaCraniiformea}.

\hypertarget{Heliomedusa_orienta-coding-79}{}
\emph{Heliomedusa orienta}:
\citet{Williams2000LinguliformeaCraniiformea} code an unpaired medial
muscle scar as present in their table 13, but give no reference for this
coding, which perhaps arises from their interpretation of the taxon as a
trimerellid. Hanken and Harper \citeyearpar[p.~249 and text-fig.
2]{Hanken1985Thetaxonomy} explicitly identify a pair of central muscles,
so we code a \emph{levator ani} as absent.

\hypertarget{Kutorgina_chengjiangensis-coding-79}{}
\emph{Kutorgina chengjiangensis}: Following table 15 in
\citet{Williams2000LinguliformeaCraniiformea}.

\hypertarget{Micrina-coding-79}{}
\emph{Micrina}: Poor preservation of minor muscle scars noted by Chen
\emph{et al}. \citeyearpar{Chen2007Reinterpretationof}.

\hypertarget{Salanygolina-coding-79}{}
\emph{Salanygolina}: Scars absent; instead, cones ornament shell's
internal surface.

\hypertarget{Serpula-coding-79}{}
\emph{Serpula}: Musculature considered essentially equivalent to
\emph{Lingula} by \citet{Williams2000LinguliformeaCraniiformea}, so
\emph{Lingula} coding followed here.

\hypertarget{NA-coding-79}{}
NA: Ventral musculature poorly constrained
\citep{Williams2000LinguliformeaCraniiformea, Popov2009Earlyontogeny}.

\subsection*{{[}80{]} Muscle scars: Dorsal
diductor}\label{muscle-scars-dorsal-diductor}
\addcontentsline{toc}{subsection}{{[}80{]} Muscle scars: Dorsal
diductor}

\includegraphics{Brachiopod_phylogeny_files/figure-latex/character-mapping-80.pdf}

\begin{quote}
\textbf{Character 80: Sclerites: Bivalved: Muscle scars: Dorsal
diductor}

0: Absent\\
1: Present\\
Neomorphic character.
\end{quote}

After Bassett \emph{et al}.
\citeyearpar{Bassett2001Functionalmorphology} character 9.

\hypertarget{Antigonambonites_planus-coding-80}{}
\emph{Antigonambonites planus}: It is unclear whether the paired muscle
scars of \emph{Oikozetetes} are homologous to brachiopod diductors.

\hypertarget{Clupeafumosus_socialis-coding-80}{}
\emph{Clupeafumosus socialis}: Not observable in \emph{Acanthotretella}
itself, so coded as ambiguous -- though it is likely based on the
anticipated phylogenetic affinities of \emph{Acanthotretella} that the
muscles are absent.

\hypertarget{Heliomedusa_orienta-coding-80}{}
\emph{Heliomedusa orienta}: Internal oblique muscles serve as diductors.

\hypertarget{Kutorgina_chengjiangensis-coding-80}{}
\emph{Kutorgina chengjiangensis}: Internal oblique muscles present
\citep{Nikitin1984} and taken to serve as diductors by analogy with
\emph{Gasconsia}.

\hypertarget{Novocrania-coding-80}{}
\emph{Novocrania}: Not reconstructed in the the interpretation of the
musculature presented by Williams \emph{et al}.
\citeyearpar{Williams2000LinguliformeaCraniiformea}, fig. 81, but
presence cannot be confidently excluded.

\hypertarget{Ussunia-coding-80}{}
\emph{Ussunia}: Not reported by Topper \emph{et al}.
\citeyearpar{Topper2013Reappraisalof}, nor reconstructed in generic
acrotretid by Williams \emph{et al}.
\citeyearpar{Williams2000LinguliformeaCraniiformea}.

\hypertarget{NA-coding-80}{}
NA: Ventral musculature poorly constrained
\citep{Williams2000LinguliformeaCraniiformea, Popov2009Earlyontogeny}.

\subsection*{{[}81{]} Muscle scars: Dorsal diductor:
Position}\label{muscle-scars-dorsal-diductor-position}
\addcontentsline{toc}{subsection}{{[}81{]} Muscle scars: Dorsal
diductor: Position}

\includegraphics{Brachiopod_phylogeny_files/figure-latex/character-mapping-81.pdf}

\begin{quote}
\textbf{Character 81: Sclerites: Bivalved: Muscle scars: Dorsal
diductor: Position}

1: Close to commissural plane\\
2: Oblique to commissural plane\\
3: At high angle to commissural plane\\
Transformational character.
\end{quote}

After Bassett \emph{et al}.
\citeyearpar{Bassett2001Functionalmorphology} character 10.

\hypertarget{NA-coding-81}{}
NA: Ventral musculature poorly constrained
\citep{Williams2000LinguliformeaCraniiformea, Popov2009Earlyontogeny}.

\section{Sclerite: Dorsal valve}\label{sclerite-dorsal-valve}

\subsection*{{[}82{]} Morphology}\label{morphology-2}
\addcontentsline{toc}{subsection}{{[}82{]} Morphology}

\includegraphics{Brachiopod_phylogeny_files/figure-latex/character-mapping-82.pdf}

\begin{quote}
\textbf{Character 82: Sclerite: Dorsal valve: Morphology}

0: Cap- shaped (operculum of hyoliths)\\
1: Non cap-shaped\\
Neomorphic character.
\end{quote}

{[}NOTE: It is not clear why the shells of brachiopods, \emph{Halkieria}
etc are not scored as cap-shaped, nor why \emph{Micromitra} is coded as
inapplicable. Many other characters already exist that reflect the
morphology of the dorsal valve; is this character independent? How can
it be scored objectively?{]}.

Oh dear! \textbf{You included the inapplicable token in a neomorphic
character!}\\
That's really very naughty, as \citet{Brazeau2018} will tell you.\\
Unless you are very sure that you understand the consequences, you
should re-code

\begin{itemize}
\tightlist
\item
  Namacalathus\\
\item
  Cotyledion\_tylodes\\
\item
  Loxosomella\\
\item
  Flustra\\
\item
  Amathia\\
\item
  Phoronis\\
\item
  Sipunculus\\
\item
  Serpula\\
\item
  Dentalium\\
\item
  Wiwaxia\_corrugata\\
\item
  Halkieria\_evangelista\\
\item
  Dailyatia\\
\item
  Eccentrotheca\\
\item
  Micromitra
\end{itemize}

\subsection*{{[}83{]} Outline of cap-shaped
shell}\label{outline-of-cap-shaped-shell}
\addcontentsline{toc}{subsection}{{[}83{]} Outline of cap-shaped shell}

\includegraphics{Brachiopod_phylogeny_files/figure-latex/character-mapping-83.pdf}

\begin{quote}
\textbf{Character 83: Sclerite: Dorsal valve: Outline of cap-shaped
shell}

0: Circular or subcircular\\
1: Triangular or subtriangular\\
2: Trapezoid or subtrapezoid\\
Neomorphic character.
\end{quote}

{[}NOTE: This character seems not to be independent of other characters,
for example, the shape of the hinge line and the growth of the shell. It
is not clear why the character is inapplicable in non-hyolith taxa.{]}.

Oh dear! \textbf{You included the inapplicable token in a neomorphic
character!}\\
That's really very naughty, as \citet{Brazeau2018} will tell you.\\
Unless you are very sure that you understand the consequences, you
should re-code

\begin{itemize}
\tightlist
\item
  Namacalathus\\
\item
  Cotyledion\_tylodes\\
\item
  Loxosomella\\
\item
  Flustra\\
\item
  Amathia\\
\item
  Pelagodiscus\_atlanticus\\
\item
  Terebratulina\\
\item
  Lingula\\
\item
  Phoronis\\
\item
  Sipunculus\\
\item
  Serpula\\
\item
  Tonicella\\
\item
  Dentalium\\
\item
  Wiwaxia\_corrugata\\
\item
  Halkieria\_evangelista\\
\item
  Dailyatia\\
\item
  Acanthotretella\_spinosa\\
\item
  Alisina\\
\item
  Askepasma\_toddense\\
\item
  Antigonambonites\_planus\\
\item
  Botsfordia\\
\item
  Clupeafumosus\_socialis\\
\item
  Coolinia\_pecten\\
\item
  Novocrania\\
\item
  Craniops\\
\item
  Gasconsia\\
\item
  Ussunia\\
\item
  Eccentrotheca\\
\item
  Eoobolus\\
\item
  Glyptoria\\
\item
  Heliomedusa\_orienta\\
\item
  Kutorgina\_chengjiangensis\\
\item
  Lingulosacculus\\
\item
  Lingulellotreta\_malongensis\\
\item
  Longtancunella\_chengjiangensis\\
\item
  Micrina\\
\item
  Micromitra\\
\item
  Mickwitzia\_muralensis\\
\item
  Mummpikia\_nuda\\
\item
  Nisusia\_sulcata\\
\item
  Orthis\\
\item
  Paterimitra\\
\item
  Salanygolina\\
\item
  Siphonobolus\_priscus\\
\item
  Tomteluva\_perturbata\\
\item
  Yuganotheca\_elegans
\end{itemize}

\section{Sclerites: Dorsal valve}\label{sclerites-dorsal-valve}

\subsection*{{[}84{]} Growth direction}\label{growth-direction}
\addcontentsline{toc}{subsection}{{[}84{]} Growth direction}

\includegraphics{Brachiopod_phylogeny_files/figure-latex/character-mapping-84.pdf}

\begin{quote}
\textbf{Character 84: Sclerites: Dorsal valve: Growth direction}

1: Holoperipheral\\
2: Mixoperipheral\\
3: Hemiperipheral\\
Transformational character.
\end{quote}

See Fig. 284 in Williams \emph{et al}.
\citeyearpar{Williams1997Introduction}.\\
The growth direction dictates the attitude of the cardinal area relative
to the hinge, which does not therefore represent an independent
character.\\
Crudely put, if, viewed from a dorsal position, the umbo falls within
the outer margin of the shell, growth is holoperipheral; if it falls
outside the margin, it is mixoperipheral; if it falls exactly on the
margin, it is hemiperipheral.

\hypertarget{Acanthotretella_spinosa-coding-84}{}
\emph{Acanthotretella spinosa}: For the purposes of this analysis, we
must treat polyplacophoran and brachiopod valves as potentially
homologous.

In brachiopods, the dorsal valve bears the lophophore, which arises from
the anterior lobe of the larva \citep{Altenburger2013} -- indicating
that the dorsal shell field is associated with the anterior lobe.

In polyplacophorans, the head valve arises from a shell field on the
anterior (pre-prototroch) lobe of the larva \citep{Wanninger2002C},
which we therefore treat as homologous with the brachiopod dorsal valve.

In support of this hypothesis, we note that the posterior (but not
anterior) valves of chitons bear apophyses
\citep{Schwabe2010, Connors2012}, which are most prominent in the
ventral (but not dorsal) valves of brachiopods \citep[fig.
322]{Williams1997Introduction}, and which occur in the morph A shell of
\emph{Oikozetetes}, which is interpreted as the posterior valve of a
halkieriid \citep{Paterson2009}.

As the single posterior shell field of polyplacophorans subdivides to
give rise to the six intermediate valves plus the tail valve
\citep{Wanninger2002C}, we prefer to consider the intermediate valves as
representing ``subdivisions'' of a single valve rather than additional
valves added to the body plan.

Growth is hemiperipheral in the anterior valve of polyplacophorans and
holoperipheral in the posterior valves \citep{Schwabe2010, Connors2012}.

\hypertarget{Glyptoria-coding-84}{}
\emph{Glyptoria}: ``Both valves with growth holoperipheral'' --
\citet{Williams2000LinguliformeaCraniiformea}, p.~164.

\hypertarget{Kutorgina_chengjiangensis-coding-84}{}
\emph{Kutorgina chengjiangensis}: Growth ``mixoperipheral in both
valves'' in trimerellids
\citep{Williams2000LinguliformeaCraniiformea, Popov1997}.

\hypertarget{Micrina-coding-84}{}
\emph{Micrina}: ``holoperipheral growth in dorsal valve'' --
\citet{Williams2007Supplement}.

Zhang \emph{et al}. \citeyearpar{Zhang2009Architectureand} conclude that
Chen \emph{et al}. \citeyearpar{Chen2007Reinterpretationof} misidentify
the dorsal valve as the ventral valve.

\hypertarget{Orthis-coding-84}{}
\emph{Orthis}: See Holmer \emph{et al}.
\citeyearpar{Holmer2008TheEarly}.

\hypertarget{Ussunia-coding-84}{}
\emph{Ussunia}: Appears hemiperipheral in fig. 3 in Topper \emph{et al}.
\citeyearpar{Topper2013Reappraisalof}, though bordering on
holoperipheral, so scored as ambiguous.

\hypertarget{NA-coding-84}{}
NA: S2 and L sclerites are clearly holoperipheral. See
\citet{Larsson2014iPaterimitra}, fig. 2.

\subsection*{{[}85{]} Posterior surface:
Differentiated}\label{posterior-surface-differentiated}
\addcontentsline{toc}{subsection}{{[}85{]} Posterior surface:
Differentiated}

\includegraphics{Brachiopod_phylogeny_files/figure-latex/character-mapping-85.pdf}

\begin{quote}
\textbf{Character 85: Sclerites: Dorsal valve: Posterior surface:
Differentiated}

0: Posterior face of dorsal valve not differentiated\\
1: Posterior face of dorsal valve forms distinct cardinal area or
pseudointerarea\\
Neomorphic character.
\end{quote}

In shells that grow by mixoperipheral growth, the triangular area
subtended between each apex and the posterior ends of the lateral
margins is termed the cardinal area. In shells with holoperipheral
growth, a flattened surface on the posterior margin of the valve is
termed a pseudointerarea
\citep[paraphrasing][]{Williams1997Introduction}.

In order for this character to be independent of a shell's growth
direction, we do not distinguish between a ``cardinal area'',
``interarea'' or ``pseudointerarea''.

\hypertarget{Acanthotretella_spinosa-coding-85}{}
\emph{Acanthotretella spinosa}: V-shaped notch in anterior valve
\citep{Schwabe2010}.

\hypertarget{Amathia-coding-85}{}
\emph{Amathia}: A very short pseudointerarea appears to be present
\citep{Moysiuk2017Hyolithsare}.

\hypertarget{Clupeafumosus_socialis-coding-85}{}
\emph{Clupeafumosus socialis}: Pseudointerarea present, following
Siphonotretidae coding in Williams \emph{et al}.
\citeyearpar{Williams2000LinguliformeaCraniiformea}, table 6.

\hypertarget{Coolinia_pecten-coding-85}{}
\emph{Coolinia pecten}, \emph{Craniops}, \emph{Eccentrotheca},
\emph{Longtancunella chengjiangensis}, \emph{Micromitra},
\emph{Yuganotheca elegans}, NA, NA: Cardinal area (interarea) present.

\hypertarget{Dentalium-coding-85}{}
\emph{Dentalium}, \emph{Mummpikia nuda}: Pseudointerarea present,
following Williams \emph{et al}.
\citeyearpar{Williams2000LinguliformeaCraniiformea}, table 6.

\hypertarget{Eoobolus-coding-85}{}
\emph{Eoobolus}, \emph{Pelagodiscus atlanticus}, NA: Pseudointerarea.

\hypertarget{Gasconsia-coding-85}{}
\emph{Gasconsia}: ``dorsal pseudointerarea vestigial, divided by median
groove'' -- \citet{Williams2000LinguliformeaCraniiformea}.

\hypertarget{Glyptoria-coding-85}{}
\emph{Glyptoria}: ``Only some craniopsids (Lingulapholis,
Pseudopholidops {[}not \emph{Craniops}{]}) have well-developed
pseudointerareas.'' -- \citet{Williams2000LinguliformeaCraniiformea}.

\hypertarget{Heliomedusa_orienta-coding-85}{}
\emph{Heliomedusa orienta}: Absent: the dorsal (branchial)
pseudointerarea of \emph{G. schucherti} is ``reduced or obsolete''; that
of \emph{G. worsleyi} ``short, virtually obsolete''
\citep{Hanken1985Thetaxonomy}.

\hypertarget{Kutorgina_chengjiangensis-coding-85}{}
\emph{Kutorgina chengjiangensis}: Following table 15 in
\citet{Williams2000LinguliformeaCraniiformea}.

\hypertarget{Mickwitzia_muralensis-coding-85}{}
\emph{Mickwitzia muralensis}: Unclear from fossil material.

\hypertarget{Micrina-coding-85}{}
\emph{Micrina}: Pseudointerea in ventral valve, but not dorsal valve
\citep[2007]{Williams2000LinguliformeaCraniiformea}.

\hypertarget{Nisusia_sulcata-coding-85}{}
\emph{Nisusia sulcata}: Zhang \emph{et al}.
\citeyearpar{Zhang2011Theexceptionally} note that ``all evidence of a
pseudointerarea is lacking'', but the two-dimensional preservation style
of Chengjiang material makes details of dorsal valve difficult to
distinguish, and the possibility of a diminutive pseudointerarea cannot
be excluded with total confidence.

\hypertarget{Novocrania-coding-85}{}
\emph{Novocrania}: Well-defined pseudointerarea
\citep[p153]{Williams2000LinguliformeaCraniiformea}.

\hypertarget{Orthis-coding-85}{}
\emph{Orthis}: = Sellate sclerite duplicature
\citep{Holmer2008TheEarly}.

\hypertarget{Paterimitra-coding-85}{}
\emph{Paterimitra}: ``Dorsal pseudointerarea usually well defined, low,
anacline to catacline'' --
\citet{Williams2000LinguliformeaCraniiformea}.

\hypertarget{Salanygolina-coding-85}{}
\emph{Salanygolina}: Shell flat.

\hypertarget{Serpula-coding-85}{}
\emph{Serpula}: Absent, following entry for Discinidae in Williams
\emph{et al}. \citeyearpar{Williams2000LinguliformeaCraniiformea}, table
6.

\hypertarget{Siphonobolus_priscus-coding-85}{}
\emph{Siphonobolus priscus}: ``Information on the dorsal interarea is
inconclusive {[}\ldots{}{]} no obvious\\
interarea is recognisable; whether or not this is the primary state or a
taphonomic artefact is difficult to assess'' --
\citet{Balthasar2008iMummpikia}, p.~276.

\hypertarget{Tomteluva_perturbata-coding-85}{}
\emph{Tomteluva perturbata}: Cardinal area (interarea) present -- with
reference to Holmer \emph{et al}.
\citeyearpar{Holmer2018Evolutionarysignificance}.

\hypertarget{Tonicella-coding-85}{}
\emph{Tonicella}: Interarea present.

\hypertarget{Ussunia-coding-85}{}
\emph{Ussunia}: Pseudointerarea present; figured by Topper \emph{et al}.
\citeyearpar{Topper2013Reappraisalof}, fig. 3j.

\hypertarget{NA-coding-85}{}
NA: A differentiated region is not obvious in fossil material or its
reconstruction \citep{Zhang2014Anearly}, but the two-dimensional
preservation style of Chengjiang material makes details of dorsal valve
difficult to distinguish, and the possibility of a diminutive
pseudointerarea cannot be excluded with confidence.

\hypertarget{NA-coding-85}{}
NA: ``Dorsal pseudointerarea weakly anacline, undivided, elevated above
the valve floor'' -- \citet{Popov2009Earlyontogeny}.

\subsection*{{[}86{]} Differentiated posterior surface:
Morphology}\label{differentiated-posterior-surface-morphology}
\addcontentsline{toc}{subsection}{{[}86{]} Differentiated posterior
surface: Morphology}

\includegraphics{Brachiopod_phylogeny_files/figure-latex/character-mapping-86.pdf}

\begin{quote}
\textbf{Character 86: Sclerites: Dorsal valve: Differentiated posterior
surface: Morphology}

0: Curved lateral profile\\
1: Planar lateral profile\\
Neomorphic character.
\end{quote}

It is possible for a cardinal area or pseudointerarea to be distinct
from the anterior part of the shell, yet to remain curved in lateral
profile.

Taking an undifferentiated posterior margin as primitive, the primitive
condition is curved -- flattening of the posterior margin represents an
additional modification that can only occur once the posterior margin is
differentiated.

\hypertarget{Acanthotretella_spinosa-coding-86}{}
\emph{Acanthotretella spinosa}: Essentially planar, though open in
aspect \citep[following Chiton in][]{Schwabe2010}.

\hypertarget{Gasconsia-coding-86}{}
\emph{Gasconsia}: ``Curved pseudointerarea'' --
\citet{Skovsted2017Depthrelated}.

\hypertarget{Heliomedusa_orienta-coding-86}{}
\emph{Heliomedusa orienta}, \emph{Kutorgina chengjiangensis},
\emph{Micrina}, \emph{Salanygolina}, \emph{Serpula}: Posterior surface
cannot be flat if it is not differentiated.

\hypertarget{Lingulellotreta_malongensis-coding-86}{}
\emph{Lingulellotreta malongensis}: Essentially planar; see Balthasar
\citeyearpar{Balthasar2009Thebrachiopod}, fig. 4a.

\hypertarget{Paterimitra-coding-86}{}
\emph{Paterimitra}: Essentially straight; see fig. 3.7 in
\citet{Ushatinskaya2016Protegulumand}.

\hypertarget{Pelagodiscus_atlanticus-coding-86}{}
\emph{Pelagodiscus atlanticus}: Difficult to evaluate based on present
material, given low nature of valve and compressed preservation.

\hypertarget{Ussunia-coding-86}{}
\emph{Ussunia}: Truncated but essentially planar surface; see e.g.~p196
of \citet{Topper2013Reappraisalof}.

\hypertarget{NA-coding-86}{}
NA: The short interarea appears planar (see for example Popov et a. 2009
fig. 6A), but its short length makes it difficult to establish whether
slight curvature is present.

\subsection*{{[}87{]} Posterior surface: Medial
groove}\label{posterior-surface-medial-groove}
\addcontentsline{toc}{subsection}{{[}87{]} Posterior surface: Medial
groove}

\includegraphics{Brachiopod_phylogeny_files/figure-latex/character-mapping-87.pdf}

\begin{quote}
\textbf{Character 87: Sclerites: Dorsal valve: Posterior surface: Medial
groove}

0: Absent\\
1: Present\\
Neomorphic character.
\end{quote}

Following character 29 in Williams \emph{et al}.
\citeyearpar{Williams2000LinguliformeaCraniiformea}, table 9 (which
relates to pseudointerarea).

\hypertarget{Clupeafumosus_socialis-coding-87}{}
\emph{Clupeafumosus socialis}: The dorsal pseudointerarea is poorly
preserved, but appears to have a median groove
\citep{Holmer2006Aspinose}.

\hypertarget{Gasconsia-coding-87}{}
\emph{Gasconsia}: ``dorsal pseudointerarea vestigial, divided by median
groove'' -- \citet{Williams2000LinguliformeaCraniiformea}.

\hypertarget{Lingulellotreta_malongensis-coding-87}{}
\emph{Lingulellotreta malongensis}: Prominent medial groove
\citep{Balthasar2009Thebrachiopod}.

\hypertarget{Micrina-coding-87}{}
\emph{Micrina}: ``A posteriorly protruding dorsal pseudointerarea with
no median groove and no flexure lines'' --
\citet{Chen2007Reinterpretationof}.

\hypertarget{Mummpikia_nuda-coding-87}{}
\emph{Mummpikia nuda}: Dorsal pseudointerarea with wide, concave median
groove and short propareas" --
\citet{Williams2000LinguliformeaCraniiformea}.

\hypertarget{Ussunia-coding-87}{}
\emph{Ussunia}: Present; figured by Topper \emph{et al}.
\citeyearpar{Topper2013Reappraisalof}, fig. 3j.

\hypertarget{NA-coding-87}{}
NA: The dorsal pseudointerarea of \emph{S. priscus} is undivided
\citep{Popov2009Earlyontogeny}, but in other species it is divided by a
``wide, poorly defined median groove''
\citep{Williams2000LinguliformeaCraniiformea}. Coded, therefore, as
polymorphic.

\subsection*{{[}88{]} Posterior surface:
Notothyrium}\label{posterior-surface-notothyrium}
\addcontentsline{toc}{subsection}{{[}88{]} Posterior surface:
Notothyrium}

\includegraphics{Brachiopod_phylogeny_files/figure-latex/character-mapping-88.pdf}

\begin{quote}
\textbf{Character 88: Sclerites: Dorsal valve: Posterior surface:
Notothyrium}

0: Absent\\
1: Present\\
Neomorphic character.
\end{quote}

A notothyrium is an opening in an interarea that accommodates the
pedicle, and may be filled with plates.

\hypertarget{Acanthotretella_spinosa-coding-88}{}
\emph{Acanthotretella spinosa}: The deep V-shaped notch \citep[fig.
8]{Schwabe2010} is positionally equivalent to the brachiopod
notothyrium.

\hypertarget{Gasconsia-coding-88}{}
\emph{Gasconsia}: Following \citet{Williams1998Thediversity}, appendix
2.

\hypertarget{Nisusia_sulcata-coding-88}{}
\emph{Nisusia sulcata}: No evidence or report of an opening at the hinge
line in fossil material in \citet{Zhang2007Agregarious} or
\citet{Zhang2011Theexceptionally}.

\subsection*{{[}89{]} Posterior surface: Notothyrium:
Shape}\label{posterior-surface-notothyrium-shape}
\addcontentsline{toc}{subsection}{{[}89{]} Posterior surface:
Notothyrium: Shape}

\includegraphics{Brachiopod_phylogeny_files/figure-latex/character-mapping-89.pdf}

\begin{quote}
\textbf{Character 89: Sclerites: Dorsal valve: Posterior surface:
Notothyrium: Shape}

1: Parallel-sided cleft\\
2: Triangular\\
Transformational character.
\end{quote}

A notothyrium is an opening in an interarea that accommodates the
pedicle, and may be filled with plates.

A simplification of character 5 in
\citet{Bassett2001Functionalmorphology}.

\subsection*{{[}90{]} Posterior surface: Notothyrium: Chilidial
plates}\label{posterior-surface-notothyrium-chilidial-plates}
\addcontentsline{toc}{subsection}{{[}90{]} Posterior surface:
Notothyrium: Chilidial plates}

\includegraphics{Brachiopod_phylogeny_files/figure-latex/character-mapping-90.pdf}

\begin{quote}
\textbf{Character 90: Sclerites: Dorsal valve: Posterior surface:
Notothyrium: Chilidial plates}

1: Open\\
2: Covered by chilidial plates\\
Transformational character.
\end{quote}

A notothyrium may be open or covered by a chilidium or two chilidial
plates.\\
No included taxa exhibit more than one chilidial plate.\\
Transformational as it is not self-evident whether the ancestral taxon
had an open or closed notothyrium.

\subsection*{{[}91{]} Notothyrial platform}\label{notothyrial-platform}
\addcontentsline{toc}{subsection}{{[}91{]} Notothyrial platform}

\includegraphics{Brachiopod_phylogeny_files/figure-latex/character-mapping-91.pdf}

\begin{quote}
\textbf{Character 91: Sclerites: Dorsal valve: Notothyrial platform}

0: Absent\\
1: Present\\
Neomorphic character.
\end{quote}

After Bassett \emph{et al}.
\citeyearpar{Bassett2001Functionalmorphology} character 12.\\
The presence or absence of a notothyrial platform, which often serves as
an attachment point for the diductors in a similar fashion to the
cardinal processes, is independent of the presence of a notothyrium.

\hypertarget{Coolinia_pecten-coding-91}{}
\emph{Coolinia pecten}, \emph{Longtancunella chengjiangensis}: Bassett
\emph{et al}. \citeyearpar{Bassett2001Functionalmorphology} score as
present in Table 18.1.

\hypertarget{Eccentrotheca-coding-91}{}
\emph{Eccentrotheca}: Referred to as the ``posterior platform'' in
Dewing \citeyearpar{Dewing2001Hingemodifications}.

\hypertarget{Kutorgina_chengjiangensis-coding-91}{}
\emph{Kutorgina chengjiangensis}: ``Visceral platforms absent in both
valves'' -- \citet{Williams2000LinguliformeaCraniiformea}, p.~192.

\hypertarget{Micromitra-coding-91}{}
\emph{Micromitra}: ``Dorsal diductor scars impressed on floor of
notothyrial cavity'': \citet{Williams2000LinguliformeaCraniiformea},
regarding Kutorginata.\\
Bassett \emph{et al}. \citeyearpar{Bassett2001Functionalmorphology}
score as absent in Table 18.1.

\hypertarget{Tomteluva_perturbata-coding-91}{}
\emph{Tomteluva perturbata}: Bassett \emph{et al}.
\citeyearpar{Bassett2001Functionalmorphology} score as absent in Table
18.1.\\
``Dorsal diductor scars impressed on floor of notothyrial cavity'':
\citet{Williams2000LinguliformeaCraniiformea}, regarding Kutorginata.

\subsection*{{[}92{]} Cardinal shield}\label{cardinal-shield}
\addcontentsline{toc}{subsection}{{[}92{]} Cardinal shield}

\includegraphics{Brachiopod_phylogeny_files/figure-latex/character-mapping-92.pdf}

\begin{quote}
\textbf{Character 92: Sclerites: Dorsal valve: Cardinal shield}

0: Absent\\
1: Present\\
Neomorphic character.
\end{quote}

A prominent platform in the hyolith operculum. With no obvious sites for
muscle attachment, it is unlikely to be homologous to the notothyrial
platform.

\subsection*{{[}93{]} Cardinal processes}\label{cardinal-processes}
\addcontentsline{toc}{subsection}{{[}93{]} Cardinal processes}

\includegraphics{Brachiopod_phylogeny_files/figure-latex/character-mapping-93.pdf}

\begin{quote}
\textbf{Character 93: Sclerites: Dorsal valve: Cardinal processes}

0: Absent\\
1: Present\\
Neomorphic character.
\end{quote}

After Bassett \emph{et al}.
\citeyearpar{Bassett2001Functionalmorphology} character 13. See
\citet{MartiMus2005} for an illustration.\\
Cardinal processes are unlikely to be homologous with the notothyrial
platform, even if their function is similar.

\hypertarget{Nisusia_sulcata-coding-93}{}
\emph{Nisusia sulcata}: Not evident, and ought arguably to be
discernable if present given the quality of preservation.

\hypertarget{Ussunia-coding-93}{}
\emph{Ussunia}: Not reported by Topper \emph{et al}.
\citeyearpar{Topper2013Reappraisalof}.

\subsection*{{[}94{]} Medial septum}\label{medial-septum}
\addcontentsline{toc}{subsection}{{[}94{]} Medial septum}

\includegraphics{Brachiopod_phylogeny_files/figure-latex/character-mapping-94.pdf}

\begin{quote}
\textbf{Character 94: Sclerites: Dorsal valve: Medial septum}

0: Absent\\
1: Present\\
Neomorphic character.
\end{quote}

The dorsal valve of many taxa is exhibits a septum or process (or
myophragm) along the medial line. See character 25 in Benedetto
\citeyearpar{Benedetto2009iChaniella}.

\hypertarget{Clupeafumosus_socialis-coding-94}{}
\emph{Clupeafumosus socialis}: Not described by Holmer \& Caron
\citeyearpar{Holmer2006Aspinose}, but an unannotated linear feature
corresponds to the position of a median septum. Without detailed study
of the specimen, we opt to score this as ambiguous.

\hypertarget{Craniops-coding-94}{}
\emph{Craniops}: Weakly developed septum evident in internal cast:
\citet{Williams2000LinguliformeaCraniiformea}, fig. 508.2e.

\hypertarget{Eoobolus-coding-94}{}
\emph{Eoobolus}: Median process evident: Williams \emph{et al}.
\citeyearpar{Williams2000LinguliformeaCraniiformea} fig. 100.2a, d.

\hypertarget{Gasconsia-coding-94}{}
\emph{Gasconsia}: ``dorsal interior with narrow anterior projection
extending to midvalve, bisected by median ridge'' --
\citet{Williams2000LinguliformeaCraniiformea}.

\hypertarget{Kutorgina_chengjiangensis-coding-94}{}
\emph{Kutorgina chengjiangensis}: Following char 42 in table 15 in
\citet{Williams2000LinguliformeaCraniiformea}.

\hypertarget{Lingulellotreta_malongensis-coding-94}{}
\emph{Lingulellotreta malongensis}: A ``median projection'' is present
\citep[fig. 4g in][]{Balthasar2009Thebrachiopod}.

\hypertarget{Longtancunella_chengjiangensis-coding-94}{}
\emph{Longtancunella chengjiangensis}: Neither evident nor reported in
Williams \emph{et al}.
\citeyearpar{Williams2000LinguliformeaCraniiformea}.

\hypertarget{Mickwitzia_muralensis-coding-94}{}
\emph{Mickwitzia muralensis}: It is not possible to determine, based on
the material presented in Balthasar \& Butterfield
\citeyearpar{Balthasar2009EarlyCambrian}, whether the anterior
projection of the visceral area in the dorsal valve corresponds to a
medial septum in the underlying shell.

\hypertarget{Micrina-coding-94}{}
\emph{Micrina}: Reported on `ventral' valve by Chen \emph{et al}.
\citeyearpar{Chen2007Reinterpretationof}; we consider their `ventral'
valve to be the dorsal valve.

The structure is unambiguously figured \citep[e.g.~fig. 5.1
in][]{Chen2007Reinterpretationof}, contra its coding as absent in
\citet{Williams2000LinguliformeaCraniiformea} and its lack of mention in
\citet{Williams2007Supplement} or \citet{Zhang2009Architectureand}.

\hypertarget{Micromitra-coding-94}{}
\emph{Micromitra}: Absent -- fig. 129.1f in Williams \emph{et al}.
\citeyearpar{Williams2000LinguliformeaCraniiformea}.

\hypertarget{Mummpikia_nuda-coding-94}{}
\emph{Mummpikia nuda}: Very weakly developed but seemingly present
between muscle scars in \emph{Lingulellotreta}, more prominent in
Aboriginella (also Lingulellotretidae) \citep[fig.
34]{Williams2000LinguliformeaCraniiformea}.

\hypertarget{Siphonobolus_priscus-coding-94}{}
\emph{Siphonobolus priscus}: See pl. 2 panel 6 in Balthasar
\citeyearpar{Balthasar2008iMummpikia}.

\hypertarget{Tomteluva_perturbata-coding-94}{}
\emph{Tomteluva perturbata}: Fig. 125 in Williams \emph{et al}.
\citeyearpar{Williams2000LinguliformeaCraniiformea}.

\hypertarget{Ussunia-coding-94}{}
\emph{Ussunia}: Prominent process evident
\citep{Topper2013Reappraisalof}.

\hypertarget{Yuganotheca_elegans-coding-94}{}
\emph{Yuganotheca elegans}: Short medial process (``low median ridge'',
p.~724) present in dorsal valve; see Fig. 523.3b in Williams \emph{et
al}. \citeyearpar{Williams2000LinguliformeaCraniiformea}.

\hypertarget{NA-coding-94}{}
NA: ``Dorsal interior {[}\ldots{}{]} bisected by a short median ridge.''
-- \citet{Popov2009Earlyontogeny}.

\subsection*{{[}95{]} Clavicles}\label{clavicles}
\addcontentsline{toc}{subsection}{{[}95{]} Clavicles}

\includegraphics{Brachiopod_phylogeny_files/figure-latex/character-mapping-95.pdf}

\begin{quote}
\textbf{Character 95: Sclerites: Dorsal valve: Clavicles}

0: Absent\\
1: Present\\
Neomorphic character.
\end{quote}

Prominent symmetrical ridges on the inner surface of the hyolith
operculum.

\subsection*{{[}96{]} Clavicles: Type of
clavicles}\label{clavicles-type-of-clavicles}
\addcontentsline{toc}{subsection}{{[}96{]} Clavicles: Type of clavicles}

All taxa are coded as ambiguous for this character.

\begin{quote}
\textbf{Character 96: Sclerites: Dorsal valve: Clavicles: Type of
clavicles}

0: Monoclavicle\\
1: Platyclavicle\\
2: Biclavicle\\
3: Triclavicle\\
4: Tetraclavicle\\
5: Polyclavicle\\
Neomorphic character.
\end{quote}

Usually the operculum of hyoliths has one pair of clavicles, but in some
taxa of hyolithida there are more than one pair of clavicles, which can
be divided into six types (Marek, 1967).\\
Monoclavicle:one pair of clavicles with a keel-shaped or rpunded, narrow
cross section; Platyclavicle: one pair of broad, flat clavicles
enclosing a great number of small channels;\\
Biclavicle: two pairs of clavicles;\\
Triclavicle: three pairs of clavicles;\\
Tetraclavicle: four pairs of clavicles;\\
Polyclavicle: more than four pairs of clavicles.

{[}NOTE: This character seems to combine two different features: the
number of clavicles, and the shape of each clavicle. Would it work
better as two separate characters?{]}.

\section{Sclerite: Helens {[}97{]}}\label{sclerite-helens-97}

All taxa are coded as ambiguous for this character.

\begin{quote}
\textbf{Character 97: Sclerite: Helens}

0: Absent\\
1: Present\\
Neomorphic character.
\end{quote}

Helens are a pair of curved lateral skeletal elements of hyolithide
hyoliths, so far no parallel structures have been found in
lophotrochozoans.

{[}NOTE: Hyolith helens are currently coded as being potentially
homologous with the L-elements of \emph{Paterimitra}; see character
``Sclerites: Bivalved: Accessory sclerites reduced''. We should not code
the presence of the same feature twice; if we treat helens as definitely
not homologous with the \emph{Paterimitra} L-sclerites, then we need to
unarguably defend why the two structures cannot be homologous in the
character definition. If we cannot do this then we should code hyoliths
with helens as having accessory sclerites instead. We could add a
separate character describing the form/shape of the accessory sclerites
if this helped to group the hyolithides.{]}.

\subsection*{{[}98{]} Cardinal teeth}\label{cardinal-teeth}
\addcontentsline{toc}{subsection}{{[}98{]} Cardinal teeth}

All taxa are coded as ambiguous for this character.

\begin{quote}
\textbf{Character 98: Sclerite: Dorsal valve: Cardinal teeth}

0: Absent\\
1: Present\\
Neomorphic character.
\end{quote}

Dentiform processes on the internal cardinal sheield margin of some
hyolith opercula,but no corresponding tooth sockets, which is supposed
not for articulation.

{[}NOTE: The character description should make clear how these differ
from cardinal processes, character ``Sclerites: Dorsal valve: Cardinal
processes''{]}.

\section{Sclerites: Ventral valve}\label{sclerites-ventral-valve}

\subsection*{{[}99{]} Relative size}\label{relative-size}
\addcontentsline{toc}{subsection}{{[}99{]} Relative size}

\includegraphics{Brachiopod_phylogeny_files/figure-latex/character-mapping-96.pdf}

\begin{quote}
\textbf{Character 99: Sclerites: Ventral valve: Relative size}

1: Ventral valve markedly larger than dorsal valve (ventribiconvex)\\
2: Equivalve (subequally biconvex)\\
3: Dorsal valve markedly larger than ventral valve (dorsibiconvex)\\
Transformational character.
\end{quote}

In many brachiopods, the valves are closely similar in size; in others,
the ventral valve is markedly larger than the dorsal, on account of
being more convex. Marginal cases are treated as ambiguous for the
relevant states.

\hypertarget{Acanthotretella_spinosa-coding-99}{}
\emph{Acanthotretella spinosa}: Coded as ambiguous for equivalve/ventral
valve larger: the posterior embryonic shell field, treated herein as
equivalent to the ventral valve,.

\hypertarget{Craniops-coding-99}{}
\emph{Craniops}: Broadly equivalve -- see Williams \emph{et al}.
\citeyearpar{Williams2000LinguliformeaCraniiformea} fig. 508.2c.

\hypertarget{Gasconsia-coding-99}{}
\emph{Gasconsia}: After table 8 in Williams \emph{et al}.
\citeyearpar{Williams2000LinguliformeaCraniiformea}.

\hypertarget{Glyptoria-coding-99}{}
\emph{Glyptoria}: ``Shell subequally biconvex'' --
\citet{Williams2000LinguliformeaCraniiformea}.

\hypertarget{Heliomedusa_orienta-coding-99}{}
\emph{Heliomedusa orienta}: Equivalve as juveniles, becoming
``convexiplane'' \citep[p.~187]{Williams2000LinguliformeaCraniiformea}
as adults \citep{Hanken1985Thetaxonomy}.

\hypertarget{Kutorgina_chengjiangensis-coding-99}{}
\emph{Kutorgina chengjiangensis}: Subequally biconvex
\citep[p.~192]{Williams2000LinguliformeaCraniiformea}.

\hypertarget{Lingulellotreta_malongensis-coding-99}{}
\emph{Lingulellotreta malongensis}: ``\emph{Eoobolus} is biconvex'', but
in his amended diagnosis, Balthasar
\citeyearpar{Balthasar2009Thebrachiopod} described it as ``shell
inequivalved, dorsibiconvex''.

\hypertarget{Micrina-coding-99}{}
\emph{Micrina}: Ventral valve larger than the dorsal valve
\citep[p.~659]{Zhang2009Architectureand}.

\hypertarget{Micromitra-coding-99}{}
\emph{Micromitra}: Ventral valve larger \citep[see][fig.
125.]{Williams2000LinguliformeaCraniiformea}.

\hypertarget{Nisusia_sulcata-coding-99}{}
\emph{Nisusia sulcata}, NA: The ventral valve is somewhat, but not
markedly, larger than the dorsal; as such, this character is coded
ambiguous for equivalve/ventral valve larger.

\hypertarget{Siphonobolus_priscus-coding-99}{}
\emph{Siphonobolus priscus}: Aside from hinge, valves similar in
convexity and size \citep{Balthasar2008iMummpikia}.

\hypertarget{Tomteluva_perturbata-coding-99}{}
\emph{Tomteluva perturbata}: Ventral valve larger \citep[see][fig.
126.]{Williams2000LinguliformeaCraniiformea}.

\hypertarget{NA-coding-99}{}
NA: Ventribiconvex \citep{Popov2009Earlyontogeny}.

\subsection*{{[}100{]} Growth direction}\label{growth-direction-1}
\addcontentsline{toc}{subsection}{{[}100{]} Growth direction}

\includegraphics{Brachiopod_phylogeny_files/figure-latex/character-mapping-97.pdf}

\begin{quote}
\textbf{Character 100: Sclerites: Ventral valve: Growth direction}

1: Holoperipheral\\
2: Mixoperipheral\\
3: Hemiperipheral\\
Transformational character.
\end{quote}

See Fig. 284 in Williams \emph{et al}.
\citeyearpar{Williams1997Introduction} for depiction of terms.

The growth direction dictates the attitude of the cardinal area relative
to the hinge, which does not therefore represent an independent
character.\\
Crudely put, if, viewed from a dorsal position, the umbo falls within
the outer margin of the shell, growth is holoperipheral; if it falls
outside the margin, it is mixoperipheral; if it falls exactly on the
margin, it is hemiperipheral.

\hypertarget{Acanthotretella_spinosa-coding-100}{}
\emph{Acanthotretella spinosa}: Growth is hemiperipheral in the anterior
valve of polyplacophorans and holoperipheral in the posterior valves
\citep{Schwabe2010, Connors2012}.

\hypertarget{Glyptoria-coding-100}{}
\emph{Glyptoria}: ``Both valves with growth holoperipheral'' --
\citet{Williams2000LinguliformeaCraniiformea}, p.~164.

\hypertarget{Kutorgina_chengjiangensis-coding-100}{}
\emph{Kutorgina chengjiangensis}: Growth ``mixoperipheral in both
valves'' in trimerellids
\citep{Williams2000LinguliformeaCraniiformea, Popov1997}.

\hypertarget{Micrina-coding-100}{}
\emph{Micrina}: Williams \emph{et al}.
\citeyearpar[2007]{Williams2000LinguliformeaCraniiformea} reconstruct
mixoperipheral growth in the ventral valve {[}though Chen \emph{et al}.
\citeyearpar{Chen2007Reinterpretationof} reconstruct the valves the
other way round, i.e.~it is the ventral valve that grows
holoperipherally, and the dorsal mixoperipherally{]}.

\hypertarget{Ussunia-coding-100}{}
\emph{Ussunia}: Inferred from Topper \emph{et al}.
\citeyearpar{Topper2013Reappraisalof}.

\hypertarget{NA-coding-100}{}
NA: The apical flange notwithstanding, the umbo of the S1 sclerite is
posterior of the hinge line and the posterior edge of the lateral plate
-- see \citet{Larsson2014iPaterimitra}, fig. 2a, c.

\hypertarget{NA-coding-100}{}
NA: Initially holoperipheral \citep[p.~159]{Popov2009Earlyontogeny},
then on the brink of being mixoperipheral in adulthood, so coded as
polymorphic.

\subsection*{{[}101{]} Posterior surface:
Differentiated}\label{posterior-surface-differentiated-1}
\addcontentsline{toc}{subsection}{{[}101{]} Posterior surface:
Differentiated}

\includegraphics{Brachiopod_phylogeny_files/figure-latex/character-mapping-98.pdf}

\begin{quote}
\textbf{Character 101: Sclerites: Ventral valve: Posterior surface:
Differentiated}

0: Posterior surface of shell not differentiated\\
1: Posterior surface of shell forms distinct cardinal area or
pseudointerarea\\
Neomorphic character.
\end{quote}

In shells that grow by mixoperipheral growth, the triangular area
subtended between each apex and the posterior ends of the lateral
margins is termed the cardinal area. In shells with holoperipheral
growth, a flattened surface on the posterior margin of the valve is
termed a pseudointerarea
\citep[paraphrasing][]{Williams1997Introduction}.

In order for this character to be independent of a shell's growth
direction, we do not distinguish between a ``cardinal area'',
``interarea'' or ``pseudointerarea''.

\hypertarget{Acanthotretella_spinosa-coding-101}{}
\emph{Acanthotretella spinosa}: Following the proposed homology model
between the posterior valve of polyplacophorans and the ventral valve of
brachiopods, the ``posterior'' surface of the polyplacophoran valve is
taken to be the surface that would articulate with the anterior valve,
which is anatomically anterior on the living organism.

\hypertarget{Coolinia_pecten-coding-101}{}
\emph{Coolinia pecten}, \emph{Craniops}, \emph{Eccentrotheca},
\emph{Longtancunella chengjiangensis}, \emph{Micromitra},
\emph{Tomteluva perturbata}, \emph{Yuganotheca elegans}, NA, NA:
Interarea present.

\hypertarget{Heliomedusa_orienta-coding-101}{}
\emph{Heliomedusa orienta}: The region corresponding to the ventral
(pseudo)interarea is described as a ``trimerellid ventral cardinal
area'' by Williams \emph{et al}.
\citeyearpar[p.162]{Williams2000LinguliformeaCraniiformea}, who code
both an interarea and a pseudointerarea as absent in trimerellids.

\hypertarget{Kutorgina_chengjiangensis-coding-101}{}
\emph{Kutorgina chengjiangensis}: Following char 17 in table 15 of
\citet{Williams2000LinguliformeaCraniiformea}.

\hypertarget{Mickwitzia_muralensis-coding-101}{}
\emph{Mickwitzia muralensis}: The conical valve is interpreted as the
ventral valve with an extended pseudointerarea.

\hypertarget{Micrina-coding-101}{}
\emph{Micrina}: Zhang \emph{et al}.
\citeyearpar{Zhang2009Architectureand} report a moderate to somewhat
developed ventral pseudointerarea, confirmed by Williams \emph{et al}.
\citeyearpar{Williams2007Supplement}.

\hypertarget{Nisusia_sulcata-coding-101}{}
\emph{Nisusia sulcata}: Though ``all evidence of a pseudointerarea is
lacking'' -- \citet{Zhang2011Theexceptionally} -- the region of the
shell between the strophic hinge line and the colleplax \citep[fig. 2
in][]{Zhang2011Theexceptionally} is distinct from the rest of the shell;
the ends of the strophic hinge line are marked by prominent nicks in the
shell margin. \emph{Longtancunella} is therefore coded as having a
differentiated posterior surface.

\hypertarget{Salanygolina-coding-101}{}
\emph{Salanygolina}: Termed an interarea by Balthasar
\citeyearpar{Balthasar2004Shellstructure}.

\hypertarget{Siphonobolus_priscus-coding-101}{}
\emph{Siphonobolus priscus}: Balthasar
\citeyearpar{Balthasar2008iMummpikia} interprets a pseudointerarea as
being present -- e.g.~p273, ``Of particular interest is the vault that
bridges the most anterior portion of the ventral pseudointerarea and
raises it above the visceral platform.''; ``This pattern is reversed in
the ventral valves of \emph{M. nuda}, where the anterior projection of
the pedicle groove is raised above the valve floor whereas the lateral
parts of pseudointerarea are not''.

\hypertarget{Tonicella-coding-101}{}
\emph{Tonicella}: Interarea.

\hypertarget{Ussunia-coding-101}{}
\emph{Ussunia}: Described by Topper \emph{et al}.
\citeyearpar{Topper2013Reappraisalof}.

\hypertarget{NA-coding-101}{}
NA: Triangular notch and subapical flange.

\hypertarget{NA-coding-101}{}
NA: ``Ventral pseudointerarea, low, undivided, poorly defined'' --
\citet{Williams2000LinguliformeaCraniiformea}.

\subsection*{{[}102{]} Posterior margin growth
direction}\label{posterior-margin-growth-direction}
\addcontentsline{toc}{subsection}{{[}102{]} Posterior margin growth
direction}

\includegraphics{Brachiopod_phylogeny_files/figure-latex/character-mapping-99.pdf}

\begin{quote}
\textbf{Character 102: Sclerites: Ventral valve: Posterior margin growth
direction}

1: Inward-growing\\
2: Outward-growing\\
Transformational character.
\end{quote}

Balthasar \citeyearpar{Balthasar2008iMummpikia} notes an inward-growing
posterior margin of the pseudointerarea as potentially linking
\emph{Mummpikia} with the linguliform brachiopods.

Coded as inapplicable in taxa without a differentiated posterior margin:
the posterior margin can only grow inwards if it is differentiated from
the anterior margin; else the entire shell would grow in on itself.

\hypertarget{Gasconsia-coding-102}{}
\emph{Gasconsia}: Inward-growing; see Skovsted \& Holmer
\citeyearpar{Skovsted2005EarlyCambrian}, pl. 4.

\hypertarget{Lingulellotreta_malongensis-coding-102}{}
\emph{Lingulellotreta malongensis}: See for example Skovsted \& Holmer
\citeyearpar{Skovsted2005EarlyCambrian}, pl. 3.

\hypertarget{Mummpikia_nuda-coding-102}{}
\emph{Mummpikia nuda}: Transverse cross section of ventral
pseudointerarea concave.

\hypertarget{Siphonobolus_priscus-coding-102}{}
\emph{Siphonobolus priscus}: Balthasar
\citeyearpar{Balthasar2008iMummpikia} interprets an inward-growing
posterior margin of the pseudointerarea -- e.g.~p273, ``Of particular
interest is the vault that bridges the most anterior portion of the
ventral pseudointerarea and raises it above the visceral platform
{[}\ldots{}{]} An inward-growing posterior margin is otherwise known
only from the pseudointerareas of linguliform brachiopods''.

\hypertarget{Ussunia-coding-102}{}
\emph{Ussunia}: See Topper \emph{et al}.
\citeyearpar{Topper2013Reappraisalof}.

\subsection*{{[}103{]} Posterior surface:
Planar}\label{posterior-surface-planar}
\addcontentsline{toc}{subsection}{{[}103{]} Posterior surface: Planar}

\includegraphics{Brachiopod_phylogeny_files/figure-latex/character-mapping-100.pdf}

\begin{quote}
\textbf{Character 103: Sclerites: Ventral valve: Posterior surface:
Planar}

0: Curved lateral profile\\
1: Planar lateral profile\\
Neomorphic character.
\end{quote}

It is possible for a cardinal area or pseudointerarea to be distinct
from the anterior part of the shell, yet to remain curved in lateral
profile.

Taking an undifferentiated posterior margin as primitive, the primitive
condition is curved -- flattening of the posterior margin represents an
additional modification that can only occur once the posterior margin is
differentiated.

A flat and triangular interarea links \emph{Mummpikia} with the
Obolellidae \citep{Balthasar2008iMummpikia} -- but all included taxa
have triangular interareas, so this is not listed as a separate
character.

\hypertarget{Acanthotretella_spinosa-coding-103}{}
\emph{Acanthotretella spinosa}: \citep{Schwabe2010}.

\hypertarget{Clupeafumosus_socialis-coding-103}{}
\emph{Clupeafumosus socialis}: ventral pseudointerareas are most similar
to those found within the Order Siphonotretida.

\hypertarget{Gasconsia-coding-103}{}
\emph{Gasconsia}: See Skovsted \& Holmer
\citeyearpar{Skovsted2005EarlyCambrian}, pl. 3, fig. 14.

\hypertarget{Lingulellotreta_malongensis-coding-103}{}
\emph{Lingulellotreta malongensis}: Some curvature retained.

\hypertarget{Mummpikia_nuda-coding-103}{}
\emph{Mummpikia nuda}: Transverse cross section of ventral
pseudointerarea concave.

\hypertarget{Nisusia_sulcata-coding-103}{}
\emph{Nisusia sulcata}: Flattened, reflecting the strophic hinge line.

\hypertarget{Paterimitra-coding-103}{}
\emph{Paterimitra}: Essentially planar; see fig. 6 in
\citet{Ushatinskaya2016Protegulumand}.

\hypertarget{Ussunia-coding-103}{}
\emph{Ussunia}: ``Ventral pseudointerarea is gently procline and is flat
in lateral profile''. ---\\
\citep{Topper2013Reappraisalof}.

\hypertarget{NA-coding-103}{}
NA: `Almost' planar -- see Popov \emph{et al}. \citeyearpar[fig.
4]{Popov2009Earlyontogeny}. Coded as ambiguous.

\subsection*{{[}104{]} apertural ligular extension of conical
shell}\label{apertural-ligular-extension-of-conical-shell}
\addcontentsline{toc}{subsection}{{[}104{]} apertural ligular extension
of conical shell}

All taxa are coded as ambiguous for this character.

\begin{quote}
\textbf{Character 104: Sclerites: Ventral valve: apertural ligular
extension of conical shell}

0: Absent\\
1: Present\\
Neomorphic character.
\end{quote}

Ligula of concial shell is a characteristic feature of hyolithid
hyoliths.

{[}NOTE: It is not clear from the definition how this character ought to
be coded in taxa that are not hyoliths. A more general definition is
needed to ensure that other taxa can be coded objectively.{]}.

\subsection*{{[}105{]} Posterior surface:
Extent}\label{posterior-surface-extent}
\addcontentsline{toc}{subsection}{{[}105{]} Posterior surface: Extent}

\includegraphics{Brachiopod_phylogeny_files/figure-latex/character-mapping-101.pdf}

\begin{quote}
\textbf{Character 105: Sclerites: Ventral valve: Posterior surface:
Extent}

1: Low: Wider than deep\\
2: High: Deeper than wide\\
Transformational character.
\end{quote}

Distinguishes taxa whose ventral valve is essentially flat from those
that are essentially conical.

\hypertarget{Craniops-coding-105}{}
\emph{Craniops}: Though scored High in data matrix of Benedetto
\citeyearpar{Benedetto2009iChaniella}, this taxon \citep[see][fig.
508]{Williams2000LinguliformeaCraniiformea} does not express the deeply
conical ventral valve that this character is intended to reflect. It is
charitably coded as ambiguous.

\hypertarget{Eccentrotheca-coding-105}{}
\emph{Eccentrotheca}: See fig. 485 in
\citet{Williams2000LinguliformeaCraniiformea}.

\hypertarget{Eoobolus-coding-105}{}
\emph{Eoobolus}: Low cone.

\hypertarget{Heliomedusa_orienta-coding-105}{}
\emph{Heliomedusa orienta}: ``ventral cardinal interarea low, apsacline,
with narrow, poorly defined homeodeltidium'' --
\citet{Williams2000LinguliformeaCraniiformea}, p.~186.

\hypertarget{Micromitra-coding-105}{}
\emph{Micromitra}: This taxon \citetext{\citealp[see][fig.
129]{Williams2000LinguliformeaCraniiformea}; \citealp[fig.
1]{Popov1992TheCambrian}} comes close to expressing the deeply conical
ventral valve that this character is intended to reflect, though this is
not always so pronounced \citep[e.g.][fig.
125]{Williams2000LinguliformeaCraniiformea}. It is therefore coded as
ambiguous.

\hypertarget{Salanygolina-coding-105}{}
\emph{Salanygolina}: Often not prominently high
\citep{Skovsted2003EarlyCambrian, Balthasar2004Shellstructure}, though
in some cases \citep[e.g.][]{Butler2015Exceptionallypreserved} the
ventral valve approaches the conical shape that this character is
intended to capture. Coded as polymorphic.

\hypertarget{Tomteluva_perturbata-coding-105}{}
\emph{Tomteluva perturbata}: Scored as high in data matrix of Benedetto
\citeyearpar{Benedetto2009iChaniella}, and depicted as such in Williams
\emph{et al}. \citeyearpar[fig.
125]{Williams2000LinguliformeaCraniiformea} and Popov \citeyearpar[fig.
1]{Popov1992TheCambrian}; but not high in all specimens
\citep[e.g.][fig. 126]{Williams2000LinguliformeaCraniiformea}. It is
therefore coded as polymorphic.

\hypertarget{Ussunia-coding-105}{}
\emph{Ussunia}: Entire valve length -- see schematic in Williams
\emph{et al}. \citeyearpar{Williams1997Introduction}, fig. 286.

\hypertarget{Yuganotheca_elegans-coding-105}{}
\emph{Yuganotheca elegans}: Scored `Low' for \emph{Eoorthis} by
Benedetto \citeyearpar{Benedetto2009iChaniella}; assumed same in
\emph{Orthis}.

\hypertarget{NA-coding-105}{}
NA: Whereas Williams \emph{et al}.
\citeyearpar[p.~156]{Williams2000LinguliformeaCraniiformea} describe the
ventral pseudointerarea as high, the shell lacks the deeply conical
aspect that this character is intended to capture; we thus code the
taxon as ambiguous.

\subsection*{{[}106{]} Posterior surface:
Delthyrium}\label{posterior-surface-delthyrium}
\addcontentsline{toc}{subsection}{{[}106{]} Posterior surface:
Delthyrium}

\includegraphics{Brachiopod_phylogeny_files/figure-latex/character-mapping-102.pdf}

\begin{quote}
\textbf{Character 106: Sclerites: Ventral valve: Posterior surface:
Delthyrium}

0: Absent\\
1: Present\\
Neomorphic character.
\end{quote}

A delthyrium is an opening in an interarea or pseudointerarea that
accommodates the pedicle, and may be filled with plates.

The homology of the pedicle in the pseudointerarea of obolellids and
botsfordiids with the umbonal pedicle foramen of acrotretids was
proposed by Popov \citeyearpar{Popov1992TheCambrian}, and seemingly
corroborated by observations of Ushatinskaya \& Korovnikov
\citeyearpar{Ushatinskaya2016Revisionof}, who note that the propareas of
the \emph{Botsfordia} ventral valve sometimes merge to form an elongate
teardrop-shaped pedicle foramen.

\hypertarget{Acanthotretella_spinosa-coding-106}{}
\emph{Acanthotretella spinosa}: The antemucronal area
\citep{Schwabe2010} is treated as equivalent to the brachiopod
delthyrium.

\hypertarget{Clupeafumosus_socialis-coding-106}{}
\emph{Clupeafumosus socialis}: Origin modelled on \emph{Siphonobolus}.

\hypertarget{Gasconsia-coding-106}{}
\emph{Gasconsia}: The homology of the triangular notch or groove in the
pseudointerarea with the umbonal pedicle foramen of acrotretids was
proposed by Popov \citeyearpar{Popov1992TheCambrian}, and seemingly
corroborated by observations of Ushatinskaya \& Korovnikov
\citeyearpar{Ushatinskaya2016Revisionof}, who note that the propareas of
the \emph{Botsfordia} ventral valve sometimes merge to form an elongate
teardrop-shaped pedicle foramen.

\hypertarget{Lingulellotreta_malongensis-coding-106}{}
\emph{Lingulellotreta malongensis}: See for example fig. 5 in
\citet{Balthasar2009Thebrachiopod}.

\hypertarget{Longtancunella_chengjiangensis-coding-106}{}
\emph{Longtancunella chengjiangensis}: ``Delthyrium and notothyrium
open, wide'' -- \citet{Cooper1976LowerCambrian}.

\hypertarget{Nisusia_sulcata-coding-106}{}
\emph{Nisusia sulcata}: Unclear: a narrow ridge that may correspond to a
pseudodeltidium evident in fig 2a and sketched in fig. 2c is not
discussed in the text of \citet{Zhang2011Theexceptionally}, so the
delthyrial region is coded as ambiguous.

\hypertarget{Novocrania-coding-106}{}
\emph{Novocrania}: Homeodeltidium absent
\citep[p.~153]{Williams2000LinguliformeaCraniiformea}; deltidium is open
\citep[see][fig. 4]{Topper2013Theoldest}.

\hypertarget{Orthis-coding-106}{}
\emph{Orthis}: Opening inferred by Holmer \emph{et al}.
\citeyearpar{Holmer2008TheEarly}.

\hypertarget{Salanygolina-coding-106}{}
\emph{Salanygolina}: A delthyrium is present in young individuals
\citep{Balthasar2004Shellstructure}.

\hypertarget{Serpula-coding-106}{}
\emph{Serpula}: The listrum (pedicle opening) is interpreted as
originating via a similar mechanism to that of acrotretids
\citep{Popov1992TheCambrian}, and hence corresponding to a basally
sealed delthyrium.

\hypertarget{Ussunia-coding-106}{}
\emph{Ussunia}: Following Popov \citeyearpar{Popov1992TheCambrian}, the
larval delthyrium is sealed in adults by outgrowths of the
posterolateral margins of the shell.

\hypertarget{NA-coding-106}{}
NA: Details of the hinge region are unclear due to the flattened and
overprinted nature of fossil preservation.

\hypertarget{NA-coding-106}{}
NA: Ontogeny presumed to resemble that of acrotretids.

\subsection*{{[}107{]} Posterior surface: Delthyrium:
Shape}\label{posterior-surface-delthyrium-shape}
\addcontentsline{toc}{subsection}{{[}107{]} Posterior surface:
Delthyrium: Shape}

\includegraphics{Brachiopod_phylogeny_files/figure-latex/character-mapping-103.pdf}

\begin{quote}
\textbf{Character 107: Sclerites: Ventral valve: Posterior surface:
Delthyrium: Shape}

1: Parallel sided\\
2: Triangular\\
3: Round\\
Transformational character.
\end{quote}

A parallel-sided delthyrium links \emph{Mummpikia} with the Obolellidae
\citep{Balthasar2008iMummpikia}.

Following Popov \citeyearpar{Popov1992TheCambrian}, the larval
delthyrium of acrotretids and allied taxa is understood to be sealed in
adults by outgrowths of the posterolateral margins of the shell. The
resultant round or teardrop-shaped foramen corresponds the delthyrium.

\hypertarget{Novocrania-coding-107}{}
\emph{Novocrania}: Prominently triangular \citep[see][fig.
2]{Topper2013Theoldest}.

\hypertarget{Salanygolina-coding-107}{}
\emph{Salanygolina}: An opening is incorporated at the base of the
homeodeltidium when the organism switches from early to late maturity
\citep[fig. 10 in][]{Balthasar2004Shellstructure}. This opening is
conceivably homologous with the pedicle foramen of acrotretid
brachiopods and their ilk. To reflect this possible homology,
\emph{Mickwitzia} is coded as polymorphic (triangular/round).

\hypertarget{Ussunia-coding-107}{}
\emph{Ussunia}: Following the model of Popov
\citeyearpar{Popov1992TheCambrian}.

\subsection*{{[}108{]} Posterior surface: Delthyrium: Shape: Aspect of
rounded
opening}\label{posterior-surface-delthyrium-shape-aspect-of-rounded-opening}
\addcontentsline{toc}{subsection}{{[}108{]} Posterior surface:
Delthyrium: Shape: Aspect of rounded opening}

\includegraphics{Brachiopod_phylogeny_files/figure-latex/character-mapping-104.pdf}

\begin{quote}
\textbf{Character 108: Sclerites: Ventral valve: Posterior surface:
Delthyrium: Shape: Aspect of rounded opening}

1: Elongate: oval to rhombic\\
2: Essentially circular\\
3: Wider than long\\
Transformational character.
\end{quote}

Chen \emph{et al}. \citeyearpar{Chen2007Reinterpretationof} propose that
an oval to rhombic foramen characterises the discinids {[}and
\emph{Heliomedusa}, though the foramen in this taxon has since been
reinterpreted by Zhang \emph{et al}.
\citeyearpar{Zhang2009Architectureand} as an impression of internal
tissue{]}.

\hypertarget{Mummpikia_nuda-coding-108}{}
\emph{Mummpikia nuda}: Oval
\citep{Williams2000LinguliformeaCraniiformea}.

\hypertarget{Salanygolina-coding-108}{}
\emph{Salanygolina}: Wider than long: see fig. 10 in
\citet{Balthasar2004Shellstructure}.

\subsection*{{[}109{]} Posterior surface: Delthyrium:
Cover}\label{posterior-surface-delthyrium-cover}
\addcontentsline{toc}{subsection}{{[}109{]} Posterior surface:
Delthyrium: Cover}

\includegraphics{Brachiopod_phylogeny_files/figure-latex/character-mapping-105.pdf}

\begin{quote}
\textbf{Character 109: Sclerites: Ventral valve: Posterior surface:
Delthyrium: Cover}

1: Open\\
2: Covered, at least in part\\
Transformational character.
\end{quote}

An open delthyrium links \emph{Mummpikia} with the Obolellidae
\citep{Balthasar2008iMummpikia}.

The delthyrial opening can be covered by one or more deltidial plates,
or a pseudodeltitium.

Inapplicable in taxa with a round delthiruym \citep[generated by
overgrowth of the delthyrial opening by posterolateral parts of the
shell, per][]{Popov1992TheCambrian}.

\hypertarget{Eccentrotheca-coding-109}{}
\emph{Eccentrotheca}: A convex pseudodeltidium completely covers the
delthyrium in \emph{Coolinia}.

\hypertarget{Gasconsia-coding-109}{}
\emph{Gasconsia}: See pl. 3 fig. 15 in Skovsted \& Holmer
\citeyearpar{Skovsted2005EarlyCambrian}.

\hypertarget{Longtancunella_chengjiangensis-coding-109}{}
\emph{Longtancunella chengjiangensis}: Coded as open by Williams
\emph{et al}. \citeyearpar{Williams1998Thediversity}.

\hypertarget{Novocrania-coding-109}{}
\emph{Novocrania}: Open \citep{Topper2013Theoldest}.

\hypertarget{Tomteluva_perturbata-coding-109}{}
\emph{Tomteluva perturbata}: ``Covered only apically by a small convex
pseudodeltitium'' -- \citet{Holmer2018Evolutionarysignificance}.

\hypertarget{NA-coding-109}{}
NA: Covered by subaical flange, in part.

\subsection*{{[}110{]} Posterior surface: Delthyrium: Cover:
Extent}\label{posterior-surface-delthyrium-cover-extent}
\addcontentsline{toc}{subsection}{{[}110{]} Posterior surface:
Delthyrium: Cover: Extent}

\includegraphics{Brachiopod_phylogeny_files/figure-latex/character-mapping-106.pdf}

\begin{quote}
\textbf{Character 110: Sclerites: Ventral valve: Posterior surface:
Delthyrium: Cover: Extent}

1: Covered only partially; partially open\\
2: Completely covered\\
Transformational character.
\end{quote}

\hypertarget{Orthis-coding-110}{}
\emph{Orthis}: Remains somewhat open.

\hypertarget{Tomteluva_perturbata-coding-110}{}
\emph{Tomteluva perturbata}: A well-defined pseudo-deltidium
{[}\ldots{}{]} closes only the apical part of\\
the delthyrium \citep{Rowell1985Theevolutionary}.

\subsection*{{[}111{]} Posterior surface: Delthyrium: Cover:
Identity}\label{posterior-surface-delthyrium-cover-identity}
\addcontentsline{toc}{subsection}{{[}111{]} Posterior surface:
Delthyrium: Cover: Identity}

\includegraphics{Brachiopod_phylogeny_files/figure-latex/character-mapping-107.pdf}

\begin{quote}
\textbf{Character 111: Sclerites: Ventral valve: Posterior surface:
Delthyrium: Cover: Identity}

1: Pseudodeltidium\\
2: Deltidial plate(s)\\
3: Continuation of shell\\
Transformational character.
\end{quote}

This character has the capacity for further resolution (one or more
deltidial plates), but this is unlikely to affect the results of the
present study.

The pseudodelthyrium is also referred to as a homeodeltidium.

The antemucronal area of Polyplacophora is treated as equivalent to the
brachiopod delthyrium, but is not depositionally distinct to the rest of
the shell, so is coded with a distinct character state.

\hypertarget{Coolinia_pecten-coding-111}{}
\emph{Coolinia pecten}: Stated as ``concave pseudodeltidium with median
plication'' -- \citet{Williams2000LinguliformeaCraniiformea}\\
Coded as ``Pseudodeltidium: Covered by concave plate'' by Bassett
\emph{et al}. \citeyearpar{Bassett2001Functionalmorphology}.

\hypertarget{Heliomedusa_orienta-coding-111}{}
\emph{Heliomedusa orienta}: A homeodeltidium is illustrated by
\citet{Hanken1985Thetaxonomy}.

\hypertarget{Mummpikia_nuda-coding-111}{}
\emph{Mummpikia nuda}: The subapical flange of the \emph{Paterimitra} S1
sclerite has been homologised with the ventral homeodeltidium of
\emph{Micromitra} \citep{Larsson2014iPaterimitra}.

\hypertarget{Novocrania-coding-111}{}
\emph{Novocrania}: No pseudodeltidium
\citep[p.~153]{Williams2000LinguliformeaCraniiformea}.

\hypertarget{Orthis-coding-111}{}
\emph{Orthis}: ``Ventral valve convex with apsacline interarea bearing
delthyrium, covered by a convex pseudodeltidium'' --
\citet{Holmer2008TheEarly}.

\hypertarget{Salanygolina-coding-111}{}
\emph{Salanygolina}: Termed a homoedeltidium by Balthasar
\citeyearpar{Balthasar2004Shellstructure}.

\subsection*{{[}112{]} Posterior surface: Delthyrium: Pseudodeltidium:
Shape}\label{posterior-surface-delthyrium-pseudodeltidium-shape}
\addcontentsline{toc}{subsection}{{[}112{]} Posterior surface:
Delthyrium: Pseudodeltidium: Shape}

\includegraphics{Brachiopod_phylogeny_files/figure-latex/character-mapping-108.pdf}

\begin{quote}
\textbf{Character 112: Sclerites: Ventral valve: Posterior surface:
Delthyrium: Pseudodeltidium: Shape}

1: Concave\\
2: Convex\\
Transformational character.
\end{quote}

A ridge-like (i.e.~convex) pseudodeltitium unites \emph{Salanygolina}
with \emph{Coolinia} and other Chileata
\citep[p.~6]{Holmer2009Theenigmatic}.

\hypertarget{Coolinia_pecten-coding-112}{}
\emph{Coolinia pecten}: ``concave pseudodeltidium with median
plication'' -- \citet{Williams2000LinguliformeaCraniiformea}\\
Coded as ``Pseudodeltidium: Covered by concave plate'' by Bassett
\emph{et al}. \citeyearpar{Bassett2001Functionalmorphology}.

\hypertarget{Craniops-coding-112}{}
\emph{Craniops}: Convex \citep[fig.
508]{Williams2000LinguliformeaCraniiformea}.

\hypertarget{Heliomedusa_orienta-coding-112}{}
\emph{Heliomedusa orienta}: ``Narrow depressed homeodeltidium'' --
\citet{Hanken1985Thetaxonomy}.

\hypertarget{Micromitra-coding-112}{}
\emph{Micromitra}: Difficult to determine based on material presented in
Zhang \emph{et al}.
\citeyearpar{Zhang2007Rhynchonelliformeanbrachiopods}, or indeed for
other species in the genus
\citep[e.g.][]{Williams2000LinguliformeaCraniiformea, Skovsted2005EarlyCambrian, Holmer2018Theattachment}.

\hypertarget{Orthis-coding-112}{}
\emph{Orthis}: Convex deltoid \citep{Holmer2008TheEarly}.

\hypertarget{Paterimitra-coding-112}{}
\emph{Paterimitra}: Gently convex \citep[see][fig.
83.3]{Williams2000LinguliformeaCraniiformea}.

\hypertarget{Salanygolina-coding-112}{}
\emph{Salanygolina}: Convex \citep[see][fig.
4B]{Balthasar2004Shellstructure}.

\hypertarget{Tomteluva_perturbata-coding-112}{}
\emph{Tomteluva perturbata}: Convex in \emph{Nisusia} \citep[see][fig.
8.4]{Rowell1985Theevolutionary}.

\hypertarget{NA-coding-112}{}
NA: Gently convex \citep[see][fig.
83.1]{Williams2000LinguliformeaCraniiformea}.

\hypertarget{NA-coding-112}{}
NA: Convex \citep{Streng2016Anew}.

\hypertarget{NA-coding-112}{}
NA: ``The presence of {[}\ldots{}{]} a narrow delthyrium covered by a
convex pseudodeltidium, places Salanygolinidae outside the Class
Paterinata and strongly suggests affinity to the Cambrian Chileida'' --
\citet{Holmer2009Theenigmatic}, p.~9.

\subsection*{{[}113{]} Posterior surface: Delthyrium: Pseudodeltidium:
Hinge
furrows}\label{posterior-surface-delthyrium-pseudodeltidium-hinge-furrows}
\addcontentsline{toc}{subsection}{{[}113{]} Posterior surface:
Delthyrium: Pseudodeltidium: Hinge furrows}

\includegraphics{Brachiopod_phylogeny_files/figure-latex/character-mapping-109.pdf}

\begin{quote}
\textbf{Character 113: Sclerites: Ventral valve: Posterior surface:
Delthyrium: Pseudodeltidium: Hinge furrows}

0: Absent\\
1: Present\\
Neomorphic character.
\end{quote}

After Bassett \emph{et al}.
\citeyearpar{Bassett2001Functionalmorphology} character 18, ``Hinge
furrows on lateral sides of pseudodeltidium''.

\hypertarget{Amathia-coding-113}{}
\emph{Amathia}, \emph{Botsfordia}, \emph{Clupeafumosus socialis},
\emph{Dentalium}, \emph{Eoobolus}, \emph{Lingulosacculus},
\emph{Mickwitzia muralensis}, \emph{Micrina}, \emph{Mummpikia nuda},
\emph{Novocrania}, \emph{Orthis}, \emph{Paterimitra}, \emph{Pelagodiscus
atlanticus}, \emph{Serpula}, \emph{Siphonobolus priscus},
\emph{Tonicella}, \emph{Ussunia}, \emph{Wiwaxia corrugata},
\emph{Yuganotheca elegans}, NA: Absent due to inapplicability of
neomorphic character.

\hypertarget{Heliomedusa_orienta-coding-113}{}
\emph{Heliomedusa orienta}: Not evident or illustrated
\citep{Hanken1985Thetaxonomy}.

\hypertarget{Longtancunella_chengjiangensis-coding-113}{}
\emph{Longtancunella chengjiangensis}: Coded as absent in
\citet{Bassett2001Functionalmorphology} (table 18.1).

\hypertarget{Micromitra-coding-113}{}
\emph{Micromitra}, \emph{Tomteluva perturbata}: Coded as present in
\citet{Bassett2001Functionalmorphology} (table 18.1).

\hypertarget{NA-coding-113}{}
NA: The presence of this feature is impossible to determine based on the
available material.

\subsection*{{[}114{]} Umbonal perforation}\label{umbonal-perforation}
\addcontentsline{toc}{subsection}{{[}114{]} Umbonal perforation}

\includegraphics{Brachiopod_phylogeny_files/figure-latex/character-mapping-110.pdf}

\begin{quote}
\textbf{Character 114: Sclerites: Ventral valve: Umbonal perforation}

0: Umbo imperforate (or ventral valve absent)\\
1: Umbonal perforation\\
Neomorphic character.
\end{quote}

Certain taxa, particularly those with a colleplax, exhibit a perforation
at the umbo of the ventral valve. This opening is sometimes associated
with a pedicle sheath, which emerges from the umbo of the ventral valve
without any indication of a relationship with the hinge.

In contrast, the pedicle of acrotretids and similar brachiopods is
situated on the larval hinge line, but is later surrounded by the
posterolateral regions of the growing shell to become separated from the
hinge line, and encapsulated in a position close to (or with resorption
of the brephic shell, at) the umbo \citep[see][pp.~407--411 and fig. 3
for discussion]{Popov1992TheCambrian}. In some cases, an internal
pedicle tube attests to this origin -- potentially corresponding to the
pedicle groove of lingulids. As such, the pedicle foramen of acrotretids
and allies is not originally situated at the umbo; it is instead
understood to represent a basally sealed delthyrium.

\hypertarget{Botsfordia-coding-114}{}
\emph{Botsfordia}: The B and C sclerites of \emph{Dailyatia} bear small
umbonal perforations \citep{Skovsted2015Theearly}, but these are not
considered to be homologous with the ventral valve, so this character is
coded as inapplicable -- though the possibility that the perforations
are equivalent is intriguing.

A1 sclerites typically have a pair of perforations, which are
conceivably equivalent to the setal tubes of \emph{Micrina}
\citep{Holmer2011Firstrecord}. The A1 sclerite of D. bacata has a
structure that is arguably similar to the `colleplax' of
\emph{Paterimitra}. But the homology of any of these structures to the
umbonal aperture of brachiopods is difficult to establish.

\hypertarget{Lingulosacculus-coding-114}{}
\emph{Lingulosacculus}: The sclerites of \emph{Eccentrotheca} form a
ring that surrounds the inferred attachment structure; the attachment
structure does not emerge from an aperture within an individual
sclerite. Thus no feature in \emph{Eccentrotheca} is judged to be
potentially homologous with the apical perforation in bivalved
brachiopods.

\hypertarget{Mickwitzia_muralensis-coding-114}{}
\emph{Mickwitzia muralensis}: The apical termination of the fossil is
unknown.

\hypertarget{Micrina-coding-114}{}
\emph{Micrina}: There is ``compelling evidence to demonstrate that the
putative pedicle\\
illustrated by Chen \emph{et al}. \citeyearpar[Figs. 4, 6,
7]{Chen2007Reinterpretationof} in fact is the mold of a
three-dimensionally preserved visceral cavity.'' --
\citet{Zhang2009Architectureand}.

\hypertarget{Salanygolina-coding-114}{}
\emph{Salanygolina}: The umbo itself is imperforate
\citep{Balthasar2004Shellstructure}.

\hypertarget{Ussunia-coding-114}{}
\emph{Ussunia}: The presumed pedicle foramen reported by Topper \emph{et
al}. \citeyearpar{Topper2013Reappraisalof} is at the ventral valve umbo.
No evidence of an internal pedicle tube is present, but we follow Popov
\citeyearpar{Popov1992TheCambrian} in inferring the encapsulation of the
pedicle foramen.

\hypertarget{NA-coding-114}{}
NA: The presumed pedicle foramen is an opening between the S1 and S2
sclerites, neither of which are perforated
\citep{Skovsted2009Thescleritome}.

\hypertarget{NA-coding-114}{}
NA: Streng \emph{et al}. \citeyearpar{Streng2016Anew} observe ``an
internal tubular structure probably representing the ventral end of the
canal within the posterior wall of the pedicle tube'', but do not
consider this tomteluvid dube to be homologous with the pedicle tube of
acrotretids and their ilk, stating (p.~274) that it appears to be unique
within Brachiopoda.

\hypertarget{NA-coding-114}{}
NA: Prominent subcircular perforation at umbo associated with an
internal pedicle tube \citep{Popov2009Earlyontogeny}, thus presumed to
share an origin with the acrotretid pedicle foramen.

\subsection*{{[}115{]} Umbonal perforation:
Shape}\label{umbonal-perforation-shape}
\addcontentsline{toc}{subsection}{{[}115{]} Umbonal perforation: Shape}

\includegraphics{Brachiopod_phylogeny_files/figure-latex/character-mapping-111.pdf}

\begin{quote}
\textbf{Character 115: Sclerites: Ventral valve: Umbonal perforation:
Shape}

1: Circular (or subcircular)\\
2: Rhombic to oval\\
Transformational character.
\end{quote}

\hypertarget{Clupeafumosus_socialis-coding-115}{}
\emph{Clupeafumosus socialis}: Too small to observe given quality of
preservation \citep{Holmer2006Aspinose}.

\hypertarget{Coolinia_pecten-coding-115}{}
\emph{Coolinia pecten}: Seemingly circular
\citep{Zhang2011Anobolellate}.

\hypertarget{Craniops-coding-115}{}
\emph{Craniops}: Based on p.92, fig.4B.

\hypertarget{Eccentrotheca-coding-115}{}
\emph{Eccentrotheca}: Bassett and Popov write ``a dominant feature of
the ventral umbo is a sub-oval perforation about 270 µm long and 250 µm
wide'': the width and height of this structure are almost identical, and
we score it as (sub) circular.

\hypertarget{Micrina-coding-115}{}
\emph{Micrina}: Rhombic to oval -- seen as evidence for a discinid
affinity \citep{Chen2007Reinterpretationof}.

\hypertarget{Micromitra-coding-115}{}
\emph{Micromitra}: The exact size and shape of the apical perforation is
obscured by the emerging pedicle.

\hypertarget{Tomteluva_perturbata-coding-115}{}
\emph{Tomteluva perturbata}: ``close to circular''
\citep{Holmer2018Evolutionarysignificance}.

\hypertarget{Ussunia-coding-115}{}
\emph{Ussunia}: Taller than wide in some cases, but very nearly circular
in others; see Topper \emph{et al}.
\citeyearpar{Topper2013Reappraisalof}.

\hypertarget{NA-coding-115}{}
NA: Essentially circular \citep[fig. 4]{Holmer2009Theenigmatic}.

\subsection*{{[}116{]} Colleplax, cicatrix or pedicle
sheath}\label{colleplax-cicatrix-or-pedicle-sheath}
\addcontentsline{toc}{subsection}{{[}116{]} Colleplax, cicatrix or
pedicle sheath}

\includegraphics{Brachiopod_phylogeny_files/figure-latex/character-mapping-112.pdf}

\begin{quote}
\textbf{Character 116: Sclerites: Ventral valve: Colleplax, cicatrix or
pedicle sheath}

0: Absent\\
1: Present\\
Neomorphic character.
\end{quote}

In certain taxa, the umbo of the ventral valve bears a colleplax,
cicatrix or pedicle sheath; Bassett \emph{et al}.
\citeyearpar{Bassett2008Earlyontogeny} consider these structures as
homologous.

\hypertarget{Gasconsia-coding-116}{}
\emph{Gasconsia}: Following \citet{Williams1998Thediversity}, appendix
2.

\hypertarget{Glyptoria-coding-116}{}
\emph{Glyptoria}: \emph{Paracraniops} is ``externally similar to
\emph{Craniops}, but lacking cicatrix'' -- indicating that
\emph{Craniops} bears a cicatrix
\citep{Williams2000LinguliformeaCraniiformea}. Also coded as present in
their table 15.

\hypertarget{Kutorgina_chengjiangensis-coding-116}{}
\emph{Kutorgina chengjiangensis}: Following table 15 in
\citet{Williams2000LinguliformeaCraniiformea}.

\hypertarget{Micrina-coding-116}{}
\emph{Micrina}: A cicatrix was reconstructed by
\citet{Jin1992Revisionof} (figs 6b, 7), but has not been reported by
later authors; possibly, as with the `pedicle foramen' of Chen \emph{et
al}. \citeyearpar{Chen2007Reinterpretationof}, this structure represents
internal organs rather than a cicatrix proper
\citep{Zhang2009Architectureand}; as such it has been recorded as
ambiguous.

\hypertarget{Micromitra-coding-116}{}
\emph{Micromitra}: The umbonal region of kutorginides ``clearly lacks a
pedicle sheath'' \citep{Holmer2018Theattachment}.

\hypertarget{Mummpikia_nuda-coding-116}{}
\emph{Mummpikia nuda}: The pedicle is identified as such (rather than a
pedicle sheath) by the internal pedicle tube.

\hypertarget{Nisusia_sulcata-coding-116}{}
\emph{Nisusia sulcata}: A ring-like structure surrounding the pedicle is
interpreted as a colleplax \citep{Zhang2011Theexceptionally}, though the
authors make no comparison with the pedicle capsule exhibited by extant
terebratulids \citep[see][]{Holmer2018Evolutionarysignificance}.

\hypertarget{Orthis-coding-116}{}
\emph{Orthis}: Absent in \emph{Micrina} \citep{Holmer2011Firstrecord}.

\hypertarget{Pelagodiscus_atlanticus-coding-116}{}
\emph{Pelagodiscus atlanticus}: The flat apical termination of juvenile
individuals possibly functioned as colleplax for attachment, but may
simply represent the brephic shell; we treat it as ambiguous to reflect
this potential homology.

\hypertarget{Ussunia-coding-116}{}
\emph{Ussunia}: Not reported by Topper \emph{et al}.
\citeyearpar{Topper2013Reappraisalof}.

\hypertarget{NA-coding-116}{}
NA: The internal canal associated with the pedicle is unique to the
tomteluvids, and has an uncertain identity \citep{Streng2016Anew}. It
could conceivably correspond to an internalized pedicle sheath or an
equivalent structure, so this feature is coded as ambiguous here.

\hypertarget{NA-coding-116}{}
NA: The median collar or conical tube is conceivably homologous with the
pedicle sheath.

\hypertarget{NA-coding-116}{}
NA: Coded as present in view of the attachment scar, which has been
considered homologous with the ``adult colleplax and foramen with
attachment pad'' in \emph{Salanygolina} \citep{Popov2009Earlyontogeny}.

\subsection*{{[}117{]} Median septum}\label{median-septum}
\addcontentsline{toc}{subsection}{{[}117{]} Median septum}

\includegraphics{Brachiopod_phylogeny_files/figure-latex/character-mapping-113.pdf}

\begin{quote}
\textbf{Character 117: Sclerites: Ventral valve: Median septum}

0: Absent\\
1: Present\\
Neomorphic character.
\end{quote}

Chen \emph{et al}. \citeyearpar{Chen2007Reinterpretationof} observe a
median septum in what they interpret as the ventral valve of
\emph{Heliomedusa}, and the ventral valve of \emph{Discinisca}, which
they propose points to a close relationship.

\hypertarget{Amathia-coding-117}{}
\emph{Amathia}: The carina of \emph{H. carinatus} is an angular
elevation of the ventral valve surface, rather than a septum growing
inward on the interior of shell.

\hypertarget{Clupeafumosus_socialis-coding-117}{}
\emph{Clupeafumosus socialis}: Carbonaceous preservation confounds the
identification of internal shell structures; it is possible that this
feature is present, but not observable in the Burgess Shale material.

\hypertarget{Eoobolus-coding-117}{}
\emph{Eoobolus}: Valve thin and often unmineralized.

\hypertarget{Gasconsia-coding-117}{}
\emph{Gasconsia}: Following \citet{Williams1998Thediversity}, appendix
2.

\hypertarget{Heliomedusa_orienta-coding-117}{}
\emph{Heliomedusa orienta}: Evident in moulds of ventral valve
\citep{Hanken1985Thetaxonomy, Watkins2002Newrecord}.

\hypertarget{Kutorgina_chengjiangensis-coding-117}{}
\emph{Kutorgina chengjiangensis}: Following char. 42 in table 15 in
\citet{Williams2000LinguliformeaCraniiformea}.

\hypertarget{Lingulellotreta_malongensis-coding-117}{}
\emph{Lingulellotreta malongensis}: Prominent median septum \citep[fig.
4d, e in][]{Balthasar2009Thebrachiopod}.

\hypertarget{Longtancunella_chengjiangensis-coding-117}{}
\emph{Longtancunella chengjiangensis}: Neither evident nor reported in
Williams \emph{et al}.
\citeyearpar{Williams2000LinguliformeaCraniiformea}.

\hypertarget{Micrina-coding-117}{}
\emph{Micrina}: Reported on `ventral' valve by Chen \emph{et al}.
\citeyearpar{Chen2007Reinterpretationof}; we consider the `ventral'
valve to be the dorsal valve.

\hypertarget{Mummpikia_nuda-coding-117}{}
\emph{Mummpikia nuda}: Medial septum visible in ventral valve in
Williams \emph{et al}.
\citeyearpar{Williams2000LinguliformeaCraniiformea}, fig. 34.1c.

\hypertarget{Paterimitra-coding-117}{}
\emph{Paterimitra}: Ventral ridge characteristic of \emph{Micromitra}
\citep{Skovsted2010EarlyCambrian}.

\hypertarget{Serpula-coding-117}{}
\emph{Serpula}: Described as present in \emph{Discinisca} by
\citet{Chen2007Reinterpretationof}; assumed present also in
\emph{Pelagodiscus}.

\hypertarget{Siphonobolus_priscus-coding-117}{}
\emph{Siphonobolus priscus}: ``Some specimens also reveal that the vault
had a slight median septum, which is now visible as a notch or a groove
dividing the right from the left part'' --
\citet{Balthasar2008iMummpikia}.

\hypertarget{Ussunia-coding-117}{}
\emph{Ussunia}: A short medial ridge (septum) is present in the ventral
valve \citep{Topper2013Reappraisalof}.

\hypertarget{NA-coding-117}{}
NA: Present; see \citet{Popov2009Earlyontogeny}, fig. 5J.

\section{Sclerites: Ornament}\label{sclerites-ornament}

\subsection*{{[}118{]} Concentric ornament}\label{concentric-ornament}
\addcontentsline{toc}{subsection}{{[}118{]} Concentric ornament}

\includegraphics{Brachiopod_phylogeny_files/figure-latex/character-mapping-114.pdf}

\begin{quote}
\textbf{Character 118: Sclerites: Ornament: Concentric ornament}

1: Smooth, or growth lines only\\
2: Concentric ornament present\\
Transformational character.
\end{quote}

After character 11 in Williams \emph{et al}.
\citeyearpar{Williams1998Thediversity}. Coded as transformational as it
is possible that maintaining a smooth shell without occasional prominent
ridges requires greater secretory control.

\hypertarget{Acanthotretella_spinosa-coding-118}{}
\emph{Acanthotretella spinosa}: No prominent ornamentat in
\emph{Tonicella} \citep{Connors2012}.

\hypertarget{Amathia-coding-118}{}
\emph{Amathia}: A series of regularly spaced concentric ridges adorn
both valves \citep{Moysiuk2017Hyolithsare}; these are more pronounced
than mere growth lines.

\hypertarget{Antigonambonites_planus-coding-118}{}
\emph{Antigonambonites planus}: Ridges in shell parallel, but are more
prominent than, growth lines.

\hypertarget{Eoobolus-coding-118}{}
\emph{Eoobolus}: Irregular ridges externally
\citep{Williams2000LinguliformeaCraniiformea}.

\hypertarget{Gasconsia-coding-118}{}
\emph{Gasconsia}: Following \citet{Williams1998Thediversity}, appendix
2.\\
Pustules are arranged along concentric growth lines
\citep{Skovsted2005EarlyCambrian}, so are not treated as a distinct
ornamentation.

\hypertarget{Lingulosacculus-coding-118}{}
\emph{Lingulosacculus}: More or less concentric ridges occur on
\emph{Eccentrotheca} sclerites
\citep{Skovsted2011Scleritomeconstruction}.

\hypertarget{Longtancunella_chengjiangensis-coding-118}{}
\emph{Longtancunella chengjiangensis}, \emph{Micromitra},
\emph{Novocrania}, \emph{Paterimitra}, NA: Following appendix 2 in
Williams \emph{et al}. \citeyearpar{Williams1998Thediversity}.

\hypertarget{Micrina-coding-118}{}
\emph{Micrina}: The ornament on shell exterior is described as
concentric fila \citep[P.43]{Chen2007Reinterpretationof}, and also
scored as it in Williams \emph{et al}.
\citeyearpar[pp.160--163]{Williams2000LinguliformeaCraniiformea}.

\hypertarget{Pelagodiscus_atlanticus-coding-118}{}
\emph{Pelagodiscus atlanticus}: A series of regularly spaced concentric
ridges adorn the ventral valve; comparatively less regular lines
ornament the operculum.

\hypertarget{Salanygolina-coding-118}{}
\emph{Salanygolina}: Symmetric fila.

\hypertarget{Serpula-coding-118}{}
\emph{Serpula}: Only growth lines evident
\citep{Williams2000LinguliformeaCraniiformea}.

\hypertarget{Terebratulina-coding-118}{}
\emph{Terebratulina}: \citet{Zhang2013}.

\hypertarget{Tonicella-coding-118}{}
\emph{Tonicella}: Single ridge evident in Williams \emph{et al}.
\citeyearpar{Williams2006Rhynchonelliformeapart} fig. 1425.1a
interpreted as interruption ot growth rather than inherent feature, so
coded as absent (i.e.~smooth).

\subsection*{{[}119{]} Concentric ornament:
Symmetry}\label{concentric-ornament-symmetry}
\addcontentsline{toc}{subsection}{{[}119{]} Concentric ornament:
Symmetry}

\includegraphics{Brachiopod_phylogeny_files/figure-latex/character-mapping-115.pdf}

\begin{quote}
\textbf{Character 119: Sclerites: Ornament: Concentric ornament:
Symmetry}

1: Asymmetric fila, with outer faces\\
2: Symmetric fila\\
Transformational character.
\end{quote}

After character 11 in Williams \emph{et al}.
\citeyearpar{Williams1998Thediversity}.

\hypertarget{Botsfordia-coding-119}{}
\emph{Botsfordia}: Clear asymmetry \citep{Skovsted2015Theearly}.

\hypertarget{Coolinia_pecten-coding-119}{}
\emph{Coolinia pecten}: Seemingly asymmetric \citetext{\citealp[fig.
122.3c]{Williams2000LinguliformeaCraniiformea}; \citealp[Fig.
1]{Zhang2011Anobolellate}}.

\hypertarget{Eoobolus-coding-119}{}
\emph{Eoobolus}: Clear outer faces \citep[fig.
100.2b]{Williams2000LinguliformeaCraniiformea}.

\hypertarget{Heliomedusa_orienta-coding-119}{}
\emph{Heliomedusa orienta}: Assymmetric \citep[fig.
3]{Hanken1985Thetaxonomy}.

\hypertarget{Lingulosacculus-coding-119}{}
\emph{Lingulosacculus}: Ornament, such as it is, is asymmetric, with
prominent outer faces \citep{Skovsted2011Scleritomeconstruction}.

\hypertarget{Longtancunella_chengjiangensis-coding-119}{}
\emph{Longtancunella chengjiangensis}, \emph{Micromitra},
\emph{Novocrania}, \emph{Paterimitra}, NA: Following appendix 2 in
Williams \emph{et al}. \citeyearpar{Williams1998Thediversity}.

\hypertarget{Micrina-coding-119}{}
\emph{Micrina}: See fig. 1715 in Williams \emph{et al}.
\citeyearpar{Williams2007Supplement}.

\hypertarget{Orthis-coding-119}{}
\emph{Orthis}: No obvious asymmetry, even if not obviously symmetric
either \citep{Holmer2008TheEarly}. Coded as ambiguous.

\hypertarget{Salanygolina-coding-119}{}
\emph{Salanygolina}: Symmetric fila \citep{Balthasar2004Shellstructure}.

\subsection*{{[}120{]} Radial ornament}\label{radial-ornament}
\addcontentsline{toc}{subsection}{{[}120{]} Radial ornament}

\includegraphics{Brachiopod_phylogeny_files/figure-latex/character-mapping-116.pdf}

\begin{quote}
\textbf{Character 120: Sclerites: Ornament: Radial ornament}

0: Absent\\
1: Present\\
Neomorphic character.
\end{quote}

Ridges radiating from umbo, i.e.~ribs.

\hypertarget{Gasconsia-coding-120}{}
\emph{Gasconsia}: Following \citet{Williams1998Thediversity}, Appendix
2.

\hypertarget{Heliomedusa_orienta-coding-120}{}
\emph{Heliomedusa orienta}: ``Ornament of indistinct low radial ribs''
-- Williams \emph{et al}.
\citeyearpar[p167]{Williams2000LinguliformeaCraniiformea}.

\hypertarget{Kutorgina_chengjiangensis-coding-120}{}
\emph{Kutorgina chengjiangensis}: Unornamented.

\hypertarget{Lingulellotreta_malongensis-coding-120}{}
\emph{Lingulellotreta malongensis}: Very faint costellae in some
specimens but coded absent.

\hypertarget{Longtancunella_chengjiangensis-coding-120}{}
\emph{Longtancunella chengjiangensis}: ``Coarsely costate'' -- Williams
\emph{et al}. \citeyearpar[p710]{Williams2000LinguliformeaCraniiformea}.

\hypertarget{Micrina-coding-120}{}
\emph{Micrina}: See fig. 1715 in Williams \emph{et al}.
\citeyearpar{Williams2007Supplement}.

\hypertarget{Novocrania-coding-120}{}
\emph{Novocrania}: ``Ornament of irregularly developed, concentric
growth lamellae; microornament of irregularly arranged, polygonal pits''
-- \citet{Williams2000LinguliformeaCraniiformea}, p153; figs on p.155.

\hypertarget{NA-coding-120}{}
NA: ``Indistinct radial ribs accentuated by radial rows of tubercles''
-- \citet{Popov2009Earlyontogeny}.

\subsection*{{[}121{]} Shell-penetrating
spines}\label{shell-penetrating-spines}
\addcontentsline{toc}{subsection}{{[}121{]} Shell-penetrating spines}

\includegraphics{Brachiopod_phylogeny_files/figure-latex/character-mapping-117.pdf}

\begin{quote}
\textbf{Character 121: Sclerites: Ornament: Shell-penetrating spines}

0: Absent\\
1: Present\\
Neomorphic character.
\end{quote}

Mineralized or partly mineralized spines are observed in
\emph{Heliomedusa} and \emph{Acanthotretella}.

\hypertarget{Acanthotretella_spinosa-coding-121}{}
\emph{Acanthotretella spinosa}: Aesthete canals penetrate the main
valves of certain chitons, but are not equivalent to the
shell-penetrating spines of brachiopods.

\hypertarget{Longtancunella_chengjiangensis-coding-121}{}
\emph{Longtancunella chengjiangensis}: Neither evident nor reported in
Williams \emph{et al}.
\citeyearpar{Williams2000LinguliformeaCraniiformea}.

\hypertarget{Micrina-coding-121}{}
\emph{Micrina}: The `spines' reported by Chen \emph{et al}.
\citeyearpar{Chen2007Reinterpretationof} are pyritized spinelike\\
setae -- see pp.~2580--2590 in Williams \emph{et al}.
\citeyearpar{Williams2007Supplement}.

\hypertarget{Tomteluva_perturbata-coding-121}{}
\emph{Tomteluva perturbata}: Bears numerous small, hollow spines
\citep{Williams2000LinguliformeaCraniiformea}.

\section{Sclerites: Composition}\label{sclerites-composition}

\subsection*{{[}122{]} Mineralogy}\label{mineralogy}
\addcontentsline{toc}{subsection}{{[}122{]} Mineralogy}

\includegraphics{Brachiopod_phylogeny_files/figure-latex/character-mapping-118.pdf}

\begin{quote}
\textbf{Character 122: Sclerites: Composition: Mineralogy}

1: Organic (non-mineralized)\\
2: Phosphatic\\
3: Calcitic\\
4: Aragonitic\\
Transformational character.
\end{quote}

\hypertarget{Clupeafumosus_socialis-coding-122}{}
\emph{Clupeafumosus socialis}: Holmer \& Caron
\citeyearpar{Holmer2006Aspinose} note the absence of brittle breakage,
interpreted as indicating the absence of a material mineralized
component to the shells. The preservation is strikingly different from
that of other Burgess Shale brachiopods, ruling out a primarily calcitic
or phosphatic composition. The two-dimensional nature of the
preservation also differs from that of co-occurring aragonitic taxa
\citep[hyoliths;][ p.~273]{Holmer2006Aspinose}, indicating that any
mineralization was minor at best.

Holmer \& Caron \citeyearpar[p.~286]{Holmer2006Aspinose} suggest that it
is more likely that a (minor) mineral component was present than that it
was not, though without providing an uncontestable rationale. To be as
conservative as possible, we therefore code this taxon as ambiguous.

\hypertarget{Eoobolus-coding-122}{}
\emph{Eoobolus}: Ventral valve uncalcified in extant forms or sometimes
thin \citep{Williams2000LinguliformeaCraniiformea}, but coded as
calcitic as calcite-mineralizing pathways are present.

\hypertarget{Glyptoria-coding-122}{}
\emph{Glyptoria}: Shell calcitic.

\hypertarget{Heliomedusa_orienta-coding-122}{}
\emph{Heliomedusa orienta}: Confirmed in Trimerella by
\citet{Balthasar2011Relicaragonite}.

\hypertarget{Kutorgina_chengjiangensis-coding-122}{}
\emph{Kutorgina chengjiangensis}: Trimerellids were probably aragonitic
\citep{Williams2000LinguliformeaCraniiformea}.

\hypertarget{Lingulellotreta_malongensis-coding-122}{}
\emph{Lingulellotreta malongensis}: ``the original shell of
\emph{Eoobolus} contained small calcareous grains that were incorporated
into organic-rich layers alongside apatite''
\citep{Balthasar2007Anearly}.

\hypertarget{Mickwitzia_muralensis-coding-122}{}
\emph{Mickwitzia muralensis}: The absence of relief in
\emph{Lingulosacculus} rules out a phosphatic or calcitic composition,
but co-occurring (and presumably aragonitic) hyolithids are preserved in
the same fashion. Its constitution was thus either organic or aragonitic
\citep{Balthasar2009EarlyCambrian}.

\hypertarget{Micrina-coding-122}{}
\emph{Micrina}: ``Shell originally organophosphatic, but may generally
have been poorly mineralized'' -- \citet{Williams2007Supplement} --
cf.~ibid, p.~2889, ``These strong similarities to discinoids in
soft-part anatomy imply that the \emph{Heliomedusa} shell was chitinous
or chitinophosphatic, not calcareous.''

\hypertarget{Mummpikia_nuda-coding-122}{}
\emph{Mummpikia nuda}: Coded as phosphatic by Zhang \emph{et al}.
\citeyearpar{Zhang2014Anearly}, but with no explanation.\\
Cracks within shells of Chengjiang specimens \citep[e.g.][fig.
3]{Zhang2007Noteon} demonstrate that the shells were originally
mineralized, but not the identity of the original biomineral. This said,
phosphatized material from Kazakhstan \citep{Holmer1997EarlyCambrian} is
attributed to the same species; presuming this phosphate to be original
and the material to be conspecific, \emph{L. malongensis} is coded as
having phosphatic shells.

\hypertarget{Nisusia_sulcata-coding-122}{}
\emph{Nisusia sulcata}: ``The original composition of the shell cannot
be determined with certainty'', though it was ``most probably entirely
soft and organic'' -- \citet{Zhang2011Theexceptionally}.

\hypertarget{Salanygolina-coding-122}{}
\emph{Salanygolina}: Calcite and silica deemed diagenetic by Balthasar
\citeyearpar{Balthasar2004Shellstructure}.

\hypertarget{Siphonobolus_priscus-coding-122}{}
\emph{Siphonobolus priscus}: Identified as calcareous by preservational
criteria, and description ``primary\\
calcitic shells of \emph{M. nuda}'' \citep{Balthasar2008iMummpikia}.

\hypertarget{Terebratulina-coding-122}{}
\emph{Terebratulina}: The extensive relief and association with pyrite
framboids indicates original mineralization, but the identity of the
biomineral remains uncertain \citep{Zhang2013}.

\hypertarget{Ussunia-coding-122}{}
\emph{Ussunia}: Phosphatic -- hence the conventional placement within
Linguliformea.

\hypertarget{NA-coding-122}{}
NA: Original mineralogy unknown, but known to be mineralised and
anticipated to be phosphatic \citep{Holmer2009Theenigmatic}.

\subsection*{{[}123{]} Cuticle or organic
matrix}\label{cuticle-or-organic-matrix}
\addcontentsline{toc}{subsection}{{[}123{]} Cuticle or organic matrix}

\includegraphics{Brachiopod_phylogeny_files/figure-latex/character-mapping-119.pdf}

\begin{quote}
\textbf{Character 123: Sclerites: Composition: Cuticle or organic
matrix}

1: GAGs, chitin and collagen\\
2: Glycoprotein\\
Transformational character.
\end{quote}

Williams \emph{et al}. \citeyearpar{Williams1996Asupra} identify
glycoprotein-based organic scaffolds as distinct from those comprising
glycosaminoglycans (GAGs), chitin and collagen. This character can only
be scored for extant taxa.

\hypertarget{Dentalium-coding-123}{}
\emph{Dentalium}: Coded as GAGs, chitin and collagen in lingulids by
Williams \emph{et al}. \citeyearpar{Williams1996Asupra}.

\hypertarget{Eoobolus-coding-123}{}
\emph{Eoobolus}: Coded as glycoprotein for craniids by Williams \emph{et
al}. \citeyearpar{Williams1996Asupra}.

\hypertarget{Serpula-coding-123}{}
\emph{Serpula}: Coded as GAGs, chitin and collagen in discinids by
Williams \emph{et al}. \citeyearpar{Williams1996Asupra}.

\hypertarget{Tonicella-coding-123}{}
\emph{Tonicella}: Coded as glycoprotein for terebratulids by Williams
\emph{et al}. \citeyearpar{Williams1996Asupra}.

\hypertarget{Wiwaxia_corrugata-coding-123}{}
\emph{Wiwaxia corrugata}: ``The presence of sulphated glycosaminoglycans
(GAGs) in the chitinous cuticle of \emph{Phoronis}
\citep[p.~215]{Herrmann1997Phoronida} would suggest a link with
linguliforms, as GAGs are unknown in rhynchonelliform shells (Fig. 1891,
1896)'' -- \citet{Williams2007Supplement}, p.~2830.

\hypertarget{NA-coding-123}{}
NA: Lenticular chambers in siphonotretid shells interpreted as degraded
GAG residue \citep{Williams2004Chemicostructure}.

\subsection*{{[}124{]} Incorporation of sedimentary
particles}\label{incorporation-of-sedimentary-particles}
\addcontentsline{toc}{subsection}{{[}124{]} Incorporation of sedimentary
particles}

\includegraphics{Brachiopod_phylogeny_files/figure-latex/character-mapping-120.pdf}

\begin{quote}
\textbf{Character 124: Sclerites: Composition: Incorporation of
sedimentary particles}

0: Absent\\
1: Present\\
Neomorphic character.
\end{quote}

Phoronids and \emph{Yuganotheca} aggulutinate particles into their
sclerites.

\subsection*{{[}125{]} Periostracum:
Flexibility}\label{periostracum-flexibility}
\addcontentsline{toc}{subsection}{{[}125{]} Periostracum: Flexibility}

\includegraphics{Brachiopod_phylogeny_files/figure-latex/character-mapping-121.pdf}

\begin{quote}
\textbf{Character 125: Sclerites: Composition: Periostracum:
Flexibility}

1: Flexible\\
2: Inflexible\\
Transformational character.
\end{quote}

Following character 9 in Williams \emph{et al}.
\citeyearpar{Williams1998Thediversity}; see their p228--230 for a
discussion of how this might be inferred from fossil material.

\hypertarget{Gasconsia-coding-125}{}
\emph{Gasconsia}, \emph{Lingulellotreta malongensis}: Coded as flexible
in \citet{Williams1998Thediversity}, Appendix 2.

\hypertarget{Longtancunella_chengjiangensis-coding-125}{}
\emph{Longtancunella chengjiangensis}, \emph{Micromitra},
\emph{Novocrania}, \emph{Paterimitra}: Following appendix 2 in Williams
\emph{et al}. \citeyearpar{Williams1998Thediversity}.

\hypertarget{Serpula-coding-125}{}
\emph{Serpula}: Flexible \citep{Williams1998Thediversity}.

\hypertarget{NA-coding-125}{}
NA: Coded as uncertain in appendix 2 in Williams \emph{et al}.
\citeyearpar{Williams1998Thediversity}.

\subsection*{{[}126{]} Microstructure:
Layers}\label{microstructure-layers}
\addcontentsline{toc}{subsection}{{[}126{]} Microstructure: Layers}

\includegraphics{Brachiopod_phylogeny_files/figure-latex/character-mapping-122.pdf}

\begin{quote}
\textbf{Character 126: Sclerites: Composition: Microstructure: Layers}

1: Single microstructural layer\\
2: Two microstructurally differentiated layers\\
3: Inner and outer laminae enclosing medial void\\
4: Three microstrurally differentiated layers\\
Transformational character.
\end{quote}

Hyolith conchs comprise two mineralized layers of fibrous bundles.
Bundles are measure 5--15 µm across; their constituent fibres are each
0.1--1.0 µm wide. In the inner layer, the fibres are transverse; in the
outer layer, the bundles are inclined towards the umbo, becoming
longitudinal on the outermost margin.

Obolellids comprise a single laminated mineralogical layer
\citep{Balthasar2008iMummpikia}. Shell-penetrating canals are not
considered as contributing to the mineralogical microstructure and are
coded separately.

Coded as non-additive as there is no clear necessity to pass through the
brachiopod-like construction: the three layers could arise by the
addition of a void to a single pre-existing layer, for example.

Inapplicable in taxa with a non-mineralized shell.

\hypertarget{Acanthotretella_spinosa-coding-126}{}
\emph{Acanthotretella spinosa}: From periostracum inwards, Chiton bears
three microstructural layers: fine-grained, nacreous, and regular
crossed lamellar.

\hypertarget{Amathia-coding-126}{}
\emph{Amathia}: Assumed to be equivalent to the hyoliths described by
Kouchinsky \citeyearpar{Kouchinsky2000Skeletalmicrostructures}.

\hypertarget{Antigonambonites_planus-coding-126}{}
\emph{Antigonambonites planus}: Single layer of fibrous aragonite
\citep{Porter2008}.

\hypertarget{Gasconsia-coding-126}{}
\emph{Gasconsia}: ``Composed of a thin primary layer and a laminate
secondary shell exhibiting baculate shell structure'' -- Skovsted \&
Holmer \citeyearpar{Skovsted2005EarlyCambrian}, with reference to
\citet{Skovsted2003EarlyCambrian}.

\hypertarget{Lingulellotreta_malongensis-coding-126}{}
\emph{Lingulellotreta malongensis}: ``\emph{Eoobolus} shells exhibit the
general characteristics of modern linguliform shells, i.e.~they were
composed of alternating sets of organic and apatite-rich layers that
were separated by thin sheets of recalcitrant organic layers.'' --
\citet{Balthasar2007Anearly}.

\hypertarget{Namacalathus-coding-126}{}
\emph{Namacalathus}: \emph{Namacalathus} exhibits three layers, none of
which have any obvious correspondence with those of brachiopods.

\hypertarget{Orthis-coding-126}{}
\emph{Orthis}: Identical to \emph{Mickwitzia} and more derived
linguliforms \citep{Holmer2011Firstrecord}.

\hypertarget{Salanygolina-coding-126}{}
\emph{Salanygolina}: ``the shell structure of \emph{Mickwitzia}
{[}\ldots{}{]} is closely similar to the columnar shell of linguliform
acrotretoid brachiopods as well as to the linguloid
\emph{Lingulellotreta}, in that it has slender columns in the laminar
succession'' -- \citet{Williams2007Supplement}.

\hypertarget{Ussunia-coding-126}{}
\emph{Ussunia}: General acrotretid structure taken from Zhang \emph{et
al}. \citeyearpar{Zhang2016Epithelialcell}.

\hypertarget{NA-coding-126}{}
NA: ``Orthodoxly secreted primary and secondary layers'' --
\citet{Williams2004Chemicostructure}.

\subsection*{{[}127{]} Microstructure: Crystal
format}\label{microstructure-crystal-format}
\addcontentsline{toc}{subsection}{{[}127{]} Microstructure: Crystal
format}

\includegraphics{Brachiopod_phylogeny_files/figure-latex/character-mapping-123.pdf}

\begin{quote}
\textbf{Character 127: Sclerites: Composition: Microstructure: Crystal
format}

1: Laminated\\
2: Fibrous bundles\\
3: Polygonal columns\\
4: Nacreous / crossed lamellar\\
Transformational character.
\end{quote}

Hyolith conchs comprise two mineralized layers of fibrous bundles.
Bundles measure 5--15 μm across; their constituent fibres are each
0.1--1.0 µm wide. In the inner layer, the fibres are transverse; in the
outer layer, the bundles are inclined towards the umbo, becoming
longitudinal on the outermost margin.

Obolellids comprise a single laminated mineralogical layer
\citep{Balthasar2008iMummpikia}. Shell-penetrating canals are not
considered as contributing to the mineralogical microstructure and are
coded separately.

The pervasive (not just superficial) polygonal structures in
\emph{Paterimitra} are distinct, and characterize \emph{Askepasma},
\emph{Salanygolina}, \emph{Eccentrotheca} and \emph{Paterimitra}
\citep{Larsson2014iPaterimitra}

Williams \emph{et al}.
\citeyearpar{Williams2000LinguliformeaCraniiformea} identify
cross-bladed laminae as diagnostic of Strophomenata, with the exception
of some older groups that contain fibres or laminar laths.

\hypertarget{Amathia-coding-127}{}
\emph{Amathia}: Inferred from other hyolithids
\citep[e.g.][]{Moore2018Plywoodlike}.

\hypertarget{Craniops-coding-127}{}
\emph{Craniops}: Shell structure of this taxon is laminated, rather than
fibrous as previously considered.

\hypertarget{Gasconsia-coding-127}{}
\emph{Gasconsia}: ``Composed of a thin primary layer and a laminate
secondary shell exhibiting baculate shell structure'' -- Skovsted \&
Holmer \citeyearpar{Skovsted2005EarlyCambrian}, with reference to
\citet{Skovsted2003EarlyCambrian}.

\hypertarget{Glyptoria-coding-127}{}
\emph{Glyptoria}: ``with calcitic or possibly aragonitic inarticulated
shells with laminar (tabular) secondary layers''
\citep{Williams2000LinguliformeaCraniiformea}.

\hypertarget{Namacalathus-coding-127}{}
\emph{Namacalathus}: The inner and outer layer are foliated. The
columnar inflections lack canals, and as such we do not consider them to
bear any obvious homology with the hollow pillars of tommotiids and
certain brachiopods, their superficial similarity to strophomenid
pseudopunctae notwithstanding.

\hypertarget{Pelagodiscus_atlanticus-coding-127}{}
\emph{Pelagodiscus atlanticus}: Assumed to be fibrous by analogy with
the allothecomorph orthothecid described by Kouchinsky
\citeyearpar{Kouchinsky2000Skeletalmicrostructures}.

\hypertarget{NA-coding-127}{}
NA: Prominent laminations; see Williams \emph{et al}.
\citeyearpar{Williams2004Chemicostructure}.

\subsection*{{[}128{]} Microstructure:
Punctae}\label{microstructure-punctae}
\addcontentsline{toc}{subsection}{{[}128{]} Microstructure: Punctae}

\includegraphics{Brachiopod_phylogeny_files/figure-latex/character-mapping-124.pdf}

\begin{quote}
\textbf{Character 128: Sclerites: Composition: Microstructure: Punctae}

0: Absent\\
1: Present\\
Neomorphic character.
\end{quote}

Punctae are 10--20 µm wide canals created by multicellular extensions of
the outer epithelium. They penetrate the full depth of the shell.

Balthasar \citeyearpar{Balthasar2008iMummpikia} writes:

``Vertical shell penetrating structures, such as punctae, pseudopunctae,
extropunctae and canals, are common in many groups of brachiopods and
are distinguished based on their geometry and size
\citep{Williams1997Introduction}. Punctae are 10--20 µm wide and
represent multicellular extensions of the outer epithelium
\citep{Owen1969Thecaecum}. Pseudopunctae and extropunctae are similar in
diameter but, instead of canals, are vertical stacks of conical
deflections of individual shell layers \citep{Williams1993Roleof}. None
of these three types of vertical shell structure, all of which are
confined to calcitic-shelled brachiopods, compares with the much smaller
canals (\textless{} 1 µm in diameter) of \emph{M. nuda}. The only type
of vertical structure that fits the size and nature of the canals of the
Mural obolellids are the canals of linguliform brachiopods, which range
in width from 180 to 740 nm and are occupied by proteinaceous strands in
extant taxa
\citep{Williams1992Structureof, Williams1994Collagenouschitino, Williams1997Introduction}.
In contrast to obolellid canals, however, linguliform canals are not
known to penetrate the entire shell but terminate in organic-rich layers
\citep{Williams1997Introduction}. Based on these considerations it
would, therefore, be misleading to call obolellid shells punctate (they
are as much''punctate" as acrotretids or other linguliforms); rather
their shell structure should be called canaliculate
\citep{Williams1997Introduction}."

\hypertarget{Amathia-coding-128}{}
\emph{Amathia}: The tubules within the centre of the bundles of hyolith
shells \citep{Kouchinsky2000Skeletalmicrostructures} are c. 10 µm wide,
making them an order of magnitude larger than the canals that
characterize lingulid valves, and a similar scale to punctae. This said,
they have only been reported in a putative allathecid, so the presence
of equivalent structures in hyolithids has never been demonstrated.

\hypertarget{Glyptoria-coding-128}{}
\emph{Glyptoria}: ``impunctate''.

\hypertarget{Micrina-coding-128}{}
\emph{Micrina}: `Identical' to those in \emph{Mickwitzia} -- see
\citet{Williams2007Supplement}.

\hypertarget{Salanygolina-coding-128}{}
\emph{Salanygolina}: Coded as present to reflect that the chambers
contained setae; following Carlson in \citet{Williams2007Supplement},
the punctae may or may not be homologous as punctae, but are likely
homologous as shell perforations; both these perforations and those of
\emph{Micrina} were associated with setae, even if their equivalence bay
be with juvenile vs adult setal structures in modern brachiopods
\citep[p.~397]{Balthasar2004Shellstructure}.

\hypertarget{Siphonobolus_priscus-coding-128}{}
\emph{Siphonobolus priscus}: ``Vertical shell penetrating structures,
such as punctae, pseudopunctae, extropunctae and canals, are common in
many groups of brachiopods and are distinguished based on their geometry
and size \citep{Williams1997Introduction}. Punctae are 10--20 µm wide
and represent multicellular extensions of the outer epithelium
\citep{Owen1969Thecaecum}. {[}\ldots{}{]} None of these three types of
vertical shell structure, all of which are confined to calcitic-shelled
brachiopods, compares with the much smaller canals (\textless{} 1 µm in
diameter) of \emph{M. nuda}. The only type of vertical structure that
fits the size and nature of the canals of the Mural obolellids are the
canals of linguliform brachiopods, which range in width from 180 to 740
nm and are occupied by proteinaceous strands in extant taxa
\citetext{\citealp[1994]{Williams1992Structureof}; \citealp{Williams1997Introduction}}.''
-- \citet{Balthasar2008iMummpikia}.

\hypertarget{Tonicella-coding-128}{}
\emph{Tonicella}: Endopunctae are relatively large canals, diameter vary
greatly from 5--20 µm.

\hypertarget{NA-coding-128}{}
NA: The `canals' through the shell have a diameter of c. 20 µm
\citep[text-fig. 2a]{Williams2004Chemicostructure}, falling within the
definition of punctae used herein.

\subsection*{{[}129{]} Microstructure:
Canals}\label{microstructure-canals}
\addcontentsline{toc}{subsection}{{[}129{]} Microstructure: Canals}

\includegraphics{Brachiopod_phylogeny_files/figure-latex/character-mapping-125.pdf}

\begin{quote}
\textbf{Character 129: Sclerites: Composition: Microstructure: Canals}

0: Absent\\
1: Present\\
Neomorphic character.
\end{quote}

A caniculate microstructure occurs in lingulids; canals are narrower
(\textless{} 1 µm) than punctae, may branch, and do not fully penetrate
the shell, terminating just within the boundaries of a microstructural
layer. See \citet{Williams1997Introduction}, p303ff, and
\citet{Balthasar2008iMummpikia}, p273, for discussion.

Tubules described in hyoliths by Kouchinsky
\citeyearpar{Kouchinsky2000Skeletalmicrostructures} measure around 10 µm
in diameter, making them an order of magnitude wider than lingulid
canals.

This said, Balthasar \citeyearpar{Balthasar2008iMummpikia} considers the
tubules within the columnar shell microstructure of \emph{Mickwitzia}
cf.~occidens \citep[1--3 µm wide,][]{Skovsted2003EarlyCambrian},
acrotretides \citep[1 µm wide,
see][\citet{Zhang2016Epithelialcell}]{Holmer1989MiddleOrdovician} and
lingulellotretids \citep[100 nm wide,][]{Cusack1999Chemicostructural} as
equivalent to lingulid canals.

\emph{Micrina} exhibits both punctae and canals
\citep{Harper2017Brachiopodsorigin}, challenging Carlson's contention
\citep[in][]{Williams2007Supplement} that the structures are potentially
homologous as shell perforations.

\hypertarget{Acanthotretella_spinosa-coding-129}{}
\emph{Acanthotretella spinosa}: Aesthete canals do not fall within the
definition of this character.

\hypertarget{Amathia-coding-129}{}
\emph{Amathia}: Zhang \emph{et al}. \citeyearpar{Zhang2018Ahyolithid}
have reported um-scale canals, replicated in phosphate, within the shell
of the hyolithid \emph{Paramicrocornus}; as shell microstructure is not
preserved in \emph{Haplophrentis}, this latter taxon is taken as a
model.

\hypertarget{Antigonambonites_planus-coding-129}{}
\emph{Antigonambonites planus}: The chambers in halkieriid sclerites do
not correspond in morphology or dimension to the brachiopod-like canals
documented by this character.

\hypertarget{Gasconsia-coding-129}{}
\emph{Gasconsia}: Not evident in section presented by Skovsted \& Holmer
\citeyearpar{Skovsted2003EarlyCambrian}.

\hypertarget{Namacalathus-coding-129}{}
\emph{Namacalathus}: Canal-like structures have been reported in
\emph{Namacalathus} \citep{Zhuravlev2015Ediacaranskeletal}, and
interpreted as evidence for a Lophophorate affinity. Though the
structures are not necessarily directly equivalent, the hypothesis of
homology is followed here.

\hypertarget{Nisusia_sulcata-coding-129}{}
\emph{Nisusia sulcata}: Preservational resolution not sufficient to
evaluate.

\hypertarget{Orthis-coding-129}{}
\emph{Orthis}: Acrotretid laminae bear characteristic columns
\citep[e.g.][]{Zhang2016Epithelialcell}; a similar fabric has been
reported, and assumed homologous, in \emph{Micrina}
\citep{Butler2012ConstructingCambrian}.

A similar columnar shell microstructure also occurs in the closely
related \emph{Mickwitzia} \citep{Balthasar2008iMummpikia}.

\hypertarget{Salanygolina-coding-129}{}
\emph{Salanygolina}: Coded as present to reflect similarity of columnar
microstructure remarked on by, among others, Balthasar
\citeyearpar{Balthasar2008iMummpikia}; Williams \emph{et al}.
\citeyearpar{Williams2007Supplement}; Skovsted \& Holmer
\citeyearpar{Skovsted2003EarlyCambrian}.

\hypertarget{Ussunia-coding-129}{}
\emph{Ussunia}: Acrotretid laminae bear characteristic columns
\citep[e.g.][]{Zhang2016Epithelialcell}.

Balthasar \citeyearpar{Balthasar2008iMummpikia} considers these columns
as homologous with tubules within the columnar shell microstructure
\emph{Mummpikia}, \emph{Mickwitzia} and lingulellotretids.

\hypertarget{NA-coding-129}{}
NA: The `canals' through the shell have a diameter of c. 20 µm
\citep[text-fig. 2a]{Williams2004Chemicostructure}, falling within the
definition of punctae (rather than canals) used herein.

\subsection*{{[}130{]} Microstructure:
Pseudopunctae}\label{microstructure-pseudopunctae}
\addcontentsline{toc}{subsection}{{[}130{]} Microstructure:
Pseudopunctae}

\includegraphics{Brachiopod_phylogeny_files/figure-latex/character-mapping-126.pdf}

\begin{quote}
\textbf{Character 130: Sclerites: Composition: Microstructure:
Pseudopunctae}

0: Absent\\
1: Present\\
Neomorphic character.
\end{quote}

Pseudopunctae are not punctae, but deflections of shell laminae. They
characterise Strophomenata in particular.

\hypertarget{Craniops-coding-130}{}
\emph{Craniops}, \emph{Longtancunella chengjiangensis}, \emph{Tomteluva
perturbata}: Scored absent in data matrix of Benedetto
\citeyearpar{Benedetto2009iChaniella}.

\hypertarget{Yuganotheca_elegans-coding-130}{}
\emph{Yuganotheca elegans}: Scored absent (in \emph{Eoorthis}) in data
matrix of Benedetto \citeyearpar{Benedetto2009iChaniella}.

\subsection*{{[}131{]} Microstructure: External polygonal
ornament}\label{microstructure-external-polygonal-ornament}
\addcontentsline{toc}{subsection}{{[}131{]} Microstructure: External
polygonal ornament}

\includegraphics{Brachiopod_phylogeny_files/figure-latex/character-mapping-127.pdf}

\begin{quote}
\textbf{Character 131: Sclerites: Composition: Microstructure: External
polygonal ornament}

0: Absent\\
1: Present\\
Neomorphic character.
\end{quote}

Regular polygonal compartments, around 10 µm in diameter, characterise
\emph{Paterimitra}. Walls between compartments have the cross-section of
an anvil. An external polygonal structure (possible imprints of
epithelial tissue) occurs in \emph{Dailyatia}, but it is a surface
pattern, which is different from the polygonal prisms in the body wall
of other paterinid-like groups.

\hypertarget{Ussunia-coding-131}{}
\emph{Ussunia}: The polygonal ornament reported in acrotretids by Zhang
\emph{et al}. \citeyearpar{Zhang2016Epithelialcell} is on the internal
surface of the shell.

\section{Sclerites}\label{sclerites-1}

\subsection*{{[}132{]} Periodically shed and
replaced}\label{periodically-shed-and-replaced}
\addcontentsline{toc}{subsection}{{[}132{]} Periodically shed and
replaced}

\includegraphics{Brachiopod_phylogeny_files/figure-latex/character-mapping-128.pdf}

\begin{quote}
\textbf{Character 132: Sclerites: Periodically shed and replaced}

0: Absent\\
1: Present\\
Neomorphic character.
\end{quote}

Certain taxa periodically slough and replace some of their individual
sclerites during growth.

\section{Gametes}\label{gametes}

\subsection*{{[}133{]} Egg size}\label{egg-size}
\addcontentsline{toc}{subsection}{{[}133{]} Egg size}

\includegraphics{Brachiopod_phylogeny_files/figure-latex/character-mapping-129.pdf}

\begin{quote}
\textbf{Character 133: Gametes: Egg size}

1: Small: \textless{} 100 um, little yolk\\
2: Large: \textgreater{} 110 um, much yolk\\
Transformational character.
\end{quote}

Following Carlson \citeyearpar{Carlson1995Phylogeneticrelationships},
character 7. This character is only possible to code in extant taxa. It
is not considered independent of Carlson's character 11, number of
gametes released per spawning, as it is possible to produce more small
eggs than large eggs -- thus this latter character is not reproduced in
the present study. The same goes for Carlson's character 12, gamete
dispersal mode; brooders will tend to brood large eggs.

\hypertarget{Acanthotretella_spinosa-coding-133}{}
\emph{Acanthotretella spinosa}: \citet{BucklandNicks1988}.

\hypertarget{Alisina-coding-133}{}
\emph{Alisina}: Egg size can vary from 60--200 µm in scaphopods, but in
\emph{Dentalium} the eggs are large \citep{DufresneDube1983}.

\hypertarget{Dailyatia-coding-133}{}
\emph{Dailyatia}: c. 50 µm in \emph{Hydroides} \citep{Miles2007}.

\hypertarget{Dentalium-coding-133}{}
\emph{Dentalium}, \emph{Eoobolus}, \emph{Serpula}, \emph{Tonicella}:
Following coding for class in Carlson
\citeyearpar{Carlson1995Phylogeneticrelationships} appendix 1, character
7.

\hypertarget{Halkieria_evangelista-coding-133}{}
\emph{Halkieria evangelista}: c. 200 µm in diameter \citep{Rice1988}.

\hypertarget{Lingula-coding-133}{}
\emph{Lingula}: Tiny \citep{Nielsen1966}.

\hypertarget{Phoronis-coding-133}{}
\emph{Phoronis}: ``Mature eggs commonly measure about 200 µm in
diameter'' -- \citet{Franzen1977}.

\hypertarget{Sipunculus-coding-133}{}
\emph{Sipunculus}: ``Mature eggs commonly measure about 200 µm in
diameter'' \citep{Franzen1977}; the larva is a similar size
\citep{Reed1982}.

\hypertarget{Wiwaxia_corrugata-coding-133}{}
\emph{Wiwaxia corrugata}: \emph{Phoronis} has planktotrophic larvae.
indicating a small egg size \citep{Ruppert2004Invertebratezoology}.
Carlson \citeyearpar{Carlson1995Phylogeneticrelationships} codes
phoronids as polymorphic, as some members of the phylum have eggs of
each size.

\hypertarget{NA-coding-133}{}
NA: ``the ventral brephic valve {[}was{]} 50 µm across, {[}which{]} is
close to the known lower limit of the brachiopod egg size'' --
\citet{Popov2009Earlyontogeny}.

\subsection*{{[}134{]} Gonocoel}\label{gonocoel}
\addcontentsline{toc}{subsection}{{[}134{]} Gonocoel}

\includegraphics{Brachiopod_phylogeny_files/figure-latex/character-mapping-130.pdf}

\begin{quote}
\textbf{Character 134: Gametes: Gonocoel}

0: Absent\\
1: Retroperineal gonads\\
Neomorphic character.
\end{quote}

Character 27 in \citet{Haszprunar1996}.

\subsection*{{[}135{]} Ovary wall saccular}\label{ovary-wall-saccular}
\addcontentsline{toc}{subsection}{{[}135{]} Ovary wall saccular}

\includegraphics{Brachiopod_phylogeny_files/figure-latex/character-mapping-131.pdf}

\begin{quote}
\textbf{Character 135: Gametes: Ovary wall saccular}

0: Plain\\
1: Saccular\\
Neomorphic character.
\end{quote}

After character 31 in \citet{Haszprunar1996}.

\subsection*{{[}136{]} Testis wall saccular}\label{testis-wall-saccular}
\addcontentsline{toc}{subsection}{{[}136{]} Testis wall saccular}

\begin{quote}
\textbf{Character 136: Gametes: Testis wall saccular}

0: Plain\\
1: Saccular\\
Neomorphic character.
\end{quote}

After character 31 in \citet{Haszprunar1996}.

\subsection*{{[}137{]} Asexual reproduction}\label{asexual-reproduction}
\addcontentsline{toc}{subsection}{{[}137{]} Asexual reproduction}

\includegraphics{Brachiopod_phylogeny_files/figure-latex/character-mapping-132.pdf}

\begin{quote}
\textbf{Character 137: Gametes: Asexual reproduction}

0: Never exhibited\\
1: Frequently exhibited\\
Neomorphic character.
\end{quote}

After character 30 in \citet{Haszprunar1996}.

\hypertarget{Namacalathus-coding-137}{}
\emph{Namacalathus}: Budding well documented
\citep[e.g.][]{Zhuravlev2015Ediacaranskeletal}.

\subsection*{{[}138{]} Fertilization}\label{fertilization}
\addcontentsline{toc}{subsection}{{[}138{]} Fertilization}

\includegraphics{Brachiopod_phylogeny_files/figure-latex/character-mapping-133.pdf}

\begin{quote}
\textbf{Character 138: Gametes: Fertilization}

1: External\\
2: Internal\\
Transformational character.
\end{quote}

After character 62 in \citet{Haszprunar2000}.

\hypertarget{Sipunculus-coding-138}{}
\emph{Sipunculus}: Brood pouches in abandoned lophophore.

\subsection*{{[}139{]} Sexes}\label{sexes}
\addcontentsline{toc}{subsection}{{[}139{]} Sexes}

\includegraphics{Brachiopod_phylogeny_files/figure-latex/character-mapping-134.pdf}

\begin{quote}
\textbf{Character 139: Gametes: Sexes}

1: Gonochoristic\\
2: Hermaphroditic\\
Transformational character.
\end{quote}

After characters 1.61 and 2.54 in \citet{SPS1996}.

\hypertarget{Sipunculus-coding-139}{}
\emph{Sipunculus}: Hermaphroditic \citep{Reed1988}.

\subsection*{{[}140{]} Protective membrane}\label{protective-membrane}
\addcontentsline{toc}{subsection}{{[}140{]} Protective membrane}

\includegraphics{Brachiopod_phylogeny_files/figure-latex/character-mapping-135.pdf}

\begin{quote}
\textbf{Character 140: Gametes: Egg: Protective membrane}

0: Absent\\
1: Present\\
Neomorphic character.
\end{quote}

After character 4.69 in \citet{SPS1996}.

\hypertarget{Phoronis-coding-140}{}
\emph{Phoronis}, \emph{Sipunculus}: ``Eggs have a loose consistency and
are capable of changing form'' \citep{Franzen1977}.

\hypertarget{Wiwaxia_corrugata-coding-140}{}
\emph{Wiwaxia corrugata}: Eggs ``are surrounded by a delicate
fertilization membrane'' \citep{Pennerstorfer2012}.

\section{Gametes: Site of maturation
{[}141{]}}\label{gametes-site-of-maturation-141}

\includegraphics{Brachiopod_phylogeny_files/figure-latex/character-mapping-136.pdf}

\begin{quote}
\textbf{Character 141: Gametes: Site of maturation}

0: Body cavity\\
1: Mantle canals\\
2: Ovicell\\
Neomorphic character.
\end{quote}

After Carlson \citeyearpar{Carlson1995Phylogeneticrelationships},
character 9. Only possible to code in extant taxa. Mantle canals is
considered the derived state, as it represents a migration from the body
cavity, where gametes are produced.

\hypertarget{Dentalium-coding-141}{}
\emph{Dentalium}, \emph{Eoobolus}, \emph{Serpula}, \emph{Tonicella}:
Following Hodgson \& Reunov \citeyearpar{Hodgson1994Ultrastructureof}.

\hypertarget{Phoronis-coding-141}{}
\emph{Phoronis}, \emph{Sipunculus}: Ovicell \citep{Franzen1977}.

\hypertarget{Wiwaxia_corrugata-coding-141}{}
\emph{Wiwaxia corrugata}: Following coding for class in Carlson
\citeyearpar{Carlson1995Phylogeneticrelationships} Appendix 1, character
9.

\section{Gametes: Spermatozoa}\label{gametes-spermatozoa}

\subsection*{{[}142{]} Nucleus: Shape}\label{nucleus-shape}
\addcontentsline{toc}{subsection}{{[}142{]} Nucleus: Shape}

\includegraphics{Brachiopod_phylogeny_files/figure-latex/character-mapping-137.pdf}

\begin{quote}
\textbf{Character 142: Gametes: Spermatozoa: Nucleus: Shape}

0: Equant: length comparable to width\\
1: Elongate: length exaggerated relative to width\\
Neomorphic character.
\end{quote}

After character 41 in \citet{Ponder1997}.

\hypertarget{Acanthotretella_spinosa-coding-142}{}
\emph{Acanthotretella spinosa}: Profoundly elongated nucleus
\citep{BucklandNicks1988}.

\hypertarget{Alisina-coding-142}{}
\emph{Alisina}: Elongate nucleus, 4--6 times longer than wide
\citep{DufresneDube1983}.

\hypertarget{Dailyatia-coding-142}{}
\emph{Dailyatia}: \citet{Gherardi2011}.

\hypertarget{Lingula-coding-142}{}
\emph{Lingula}: Elongate in \emph{Loxosoma} {[}@Franzen 2000{]}.

\hypertarget{Phoronis-coding-142}{}
\emph{Phoronis}, \emph{Sipunculus}: Elongate \citep{Franzen1981}.

\subsection*{{[}143{]} Anterior nuclear
fossa}\label{anterior-nuclear-fossa}
\addcontentsline{toc}{subsection}{{[}143{]} Anterior nuclear fossa}

\includegraphics{Brachiopod_phylogeny_files/figure-latex/character-mapping-138.pdf}

\begin{quote}
\textbf{Character 143: Gametes: Spermatozoa: Anterior nuclear fossa}

0: Absent\\
1: Present\\
Neomorphic character.
\end{quote}

Following \citet{Smith2012}, after character 160 in \citet{Giribet2002}.
A fossa (latin: ditch) is a dent or impression.

\hypertarget{Acanthotretella_spinosa-coding-143}{}
\emph{Acanthotretella spinosa}: \citet{BucklandNicks1988}.

\hypertarget{Alisina-coding-143}{}
\emph{Alisina}: \citet{DufresneDube1983}.

\hypertarget{Dailyatia-coding-143}{}
\emph{Dailyatia}: Absent: subacrosomal space does not impinge on nuclear
envelope \citep{Gherardi2011}.

\hypertarget{Halkieria_evangelista-coding-143}{}
\emph{Halkieria evangelista}: Prominent in \emph{Phascolion}
\citep{Rice1993}.

\hypertarget{Lingula-coding-143}{}
\emph{Lingula}: Present in \emph{Loxosoma} {[}@Franzen 2000{]}.

\hypertarget{Phoronis-coding-143}{}
\emph{Phoronis}, \emph{Sipunculus}: Present \citep[in
\emph{Tubulipora};][]{Franzen1984}.

\hypertarget{Serpula-coding-143}{}
\emph{Serpula}: Present in \emph{Discinisca} \emph{tenuis}
\citep{Hodgson1994Ultrastructureof}.

\hypertarget{Tonicella-coding-143}{}
\emph{Tonicella}: No anterior invagination
\citep{Hodgson1994Ultrastructureof}.

\hypertarget{Wiwaxia_corrugata-coding-143}{}
\emph{Wiwaxia corrugata}: Nucleus ``almost round''
\citep{Reunov2004Ultrastructuralstudy}.

\subsection*{{[}144{]} Acrosome: Shape}\label{acrosome-shape}
\addcontentsline{toc}{subsection}{{[}144{]} Acrosome: Shape}

\includegraphics{Brachiopod_phylogeny_files/figure-latex/character-mapping-139.pdf}

\begin{quote}
\textbf{Character 144: Gametes: Spermatozoa: Acrosome: Shape}

1: Pear-shaped\\
2: Needle-shaped\\
3: Disc-shaped\\
4: Conical\\
Transformational character.
\end{quote}

\hypertarget{Acanthotretella_spinosa-coding-144}{}
\emph{Acanthotretella spinosa}: Elongate: cylindrical to conical
\citep{BucklandNicks1988}.

\hypertarget{Alisina-coding-144}{}
\emph{Alisina}: Low conical aspect \citep{DufresneDube1983}.

\hypertarget{Dailyatia-coding-144}{}
\emph{Dailyatia}: \citet{Gherardi2011}.

\hypertarget{Dentalium-coding-144}{}
\emph{Dentalium}: Pear-shaped \citep{Fukumoto2003Theacrosome}.

\hypertarget{Eoobolus-coding-144}{}
\emph{Eoobolus}: Needle-shaped \citep{Afzelius1978Finestructure}.

\hypertarget{Halkieria_evangelista-coding-144}{}
\emph{Halkieria evangelista}: A peaked disc in \emph{Phascolion}
\citep{Rice1993}.

\hypertarget{Lingula-coding-144}{}
\emph{Lingula}: Conical/cylindrical acrosome-like extension in
\emph{Loxosoma} {[}@Franzen 2000{]}.

\hypertarget{Phoronis-coding-144}{}
\emph{Phoronis}, \emph{Sipunculus}: Conical \citep[in
\emph{Tubulipora};][]{Franzen1984}.

\hypertarget{Serpula-coding-144}{}
\emph{Serpula}: Pear-shaped \citep{Hodgson1994Ultrastructureof}.

\hypertarget{Tonicella-coding-144}{}
\emph{Tonicella}: Disc-shaped (in \emph{Kraussina})
\citep{Hodgson1994Ultrastructureof}.

\hypertarget{Wiwaxia_corrugata-coding-144}{}
\emph{Wiwaxia corrugata}: Needle-shaped
\citep{Reunov2004Ultrastructuralstudy}.

\subsection*{{[}145{]} Acrosome: Differentiated
internally}\label{acrosome-differentiated-internally}
\addcontentsline{toc}{subsection}{{[}145{]} Acrosome: Differentiated
internally}

\includegraphics{Brachiopod_phylogeny_files/figure-latex/character-mapping-140.pdf}

\begin{quote}
\textbf{Character 145: Gametes: Spermatozoa: Acrosome: Differentiated
internally}

0: No internal differentiation\\
1: Acrosome differentiated internally\\
Neomorphic character.
\end{quote}

\citet{Hodgson1994Ultrastructureof} describe the \emph{Discinisca}
acrosome as having ``an electron-lucent centre and an electron-dense
outer region'', and state that this trait is characteristic of
inarticulate brachiopods.

\hypertarget{Acanthotretella_spinosa-coding-145}{}
\emph{Acanthotretella spinosa}: ``One can distinguish two components in
the acrosome, an apical and a basal granule'' --
\citet{BucklandNicks1988}.

\hypertarget{Alisina-coding-145}{}
\emph{Alisina}: Differentiated membrane only \citep{DufresneDube1983}.

\hypertarget{Dailyatia-coding-145}{}
\emph{Dailyatia}: \citet{Gherardi2011}.

\hypertarget{Dentalium-coding-145}{}
\emph{Dentalium}: Clear differentiation of marginal area
\citep{Fukumoto2003Theacrosome}.

\hypertarget{Eoobolus-coding-145}{}
\emph{Eoobolus}: ``Along the inner and outer margins there are
periodically banded layers, and between them there is a less dense
layer'' -- \citet{Afzelius1978Finestructure}.

\hypertarget{Halkieria_evangelista-coding-145}{}
\emph{Halkieria evangelista}: No differentiation within acrosomal
vesicle \citep{Rice1993}.

\hypertarget{Lingula-coding-145}{}
\emph{Lingula}: Not evident in \emph{Loxosoma} {[}@Franzen 2000{]}.

\hypertarget{Phoronis-coding-145}{}
\emph{Phoronis}, \emph{Sipunculus}: No evidence of internal
differentiation \citep[in \emph{Tubulipora};][]{Franzen1984}.

\hypertarget{Serpula-coding-145}{}
\emph{Serpula}: Following \emph{Discinisca} \emph{tenuis}, described in
Hodgson \& Reunov \citeyearpar{Hodgson1994Ultrastructureof}.

\hypertarget{Tonicella-coding-145}{}
\emph{Tonicella}: Following Hodgson \& Reunov
\citeyearpar{Hodgson1994Ultrastructureof}.

\hypertarget{Wiwaxia_corrugata-coding-145}{}
\emph{Wiwaxia corrugata}: Acrosome-like structure; no internal division
or surrounding membrane, with possibility that much of the acrosome is
secondarily lost \citep{Reunov2004Ultrastructuralstudy}.

\subsection*{{[}146{]} Acrosome: Sub-acrosomal
space}\label{acrosome-sub-acrosomal-space}
\addcontentsline{toc}{subsection}{{[}146{]} Acrosome: Sub-acrosomal
space}

\includegraphics{Brachiopod_phylogeny_files/figure-latex/character-mapping-141.pdf}

\begin{quote}
\textbf{Character 146: Gametes: Spermatozoa: Acrosome: Sub-acrosomal
space}

0: Absent\\
1: Present\\
Neomorphic character.
\end{quote}

\hypertarget{Acanthotretella_spinosa-coding-146}{}
\emph{Acanthotretella spinosa}: Not evident \citep{BucklandNicks1988}.

\hypertarget{Alisina-coding-146}{}
\emph{Alisina}: \citet{DufresneDube1983}.

\hypertarget{Dailyatia-coding-146}{}
\emph{Dailyatia}: \citet{Gherardi2011}.

\hypertarget{Dentalium-coding-146}{}
\emph{Dentalium}: Filled with sub-acrosomal substance
\citep{Fukumoto2003Theacrosome}.

\hypertarget{Eoobolus-coding-146}{}
\emph{Eoobolus}: Prominent \citep{Afzelius1978Finestructure}.

\hypertarget{Halkieria_evangelista-coding-146}{}
\emph{Halkieria evangelista}: \citet{Rice1993}.

\hypertarget{Lingula-coding-146}{}
\emph{Lingula}: Present in \emph{Loxosoma} {[}@Franzen 2000{]}.

\hypertarget{Phoronis-coding-146}{}
\emph{Phoronis}, \emph{Sipunculus}: No distinct space \citep[in
\emph{Tubulipora};][]{Franzen1984}.

\hypertarget{Serpula-coding-146}{}
\emph{Serpula}: Subacrosomal material (in \emph{Discinisca}) but no
subacrosomal space \citep{Hodgson1994Ultrastructureof}.

\hypertarget{Tonicella-coding-146}{}
\emph{Tonicella}: No subacrosomal material, let alone a subacrosomal
space \citep[e.g.][]{Hodgson1994Ultrastructureof}.

\hypertarget{Wiwaxia_corrugata-coding-146}{}
\emph{Wiwaxia corrugata}: The filament-like acrosome continues backwards
as a tube-like structure \citep[summarized in
\citet{Jamieson1991FishEvolution}]{Franzen1980Ultrastructureof}.

\subsection*{{[}147{]} Mid-piece}\label{mid-piece}
\addcontentsline{toc}{subsection}{{[}147{]} Mid-piece}

\includegraphics{Brachiopod_phylogeny_files/figure-latex/character-mapping-142.pdf}

\begin{quote}
\textbf{Character 147: Gametes: Spermatozoa: Mid-piece}

0: Multiple mitochondria\\
1: Single ring-shaped mitochondrion\\
Neomorphic character.
\end{quote}

Following Hodgson \& Reunov \citeyearpar{Hodgson1994Ultrastructureof}.

\hypertarget{Alisina-coding-147}{}
\emph{Alisina}: \citet{DufresneDube1983}.

\hypertarget{Dailyatia-coding-147}{}
\emph{Dailyatia}: Five mitochondria in ring \citep{Gherardi2011}.

\hypertarget{Dentalium-coding-147}{}
\emph{Dentalium}, \emph{Tonicella}: Following Hodgson \& Reunov
\citeyearpar{Hodgson1994Ultrastructureof}.

\hypertarget{Eoobolus-coding-147}{}
\emph{Eoobolus}: Four mitochondria \citep{Afzelius1978Finestructure}.

\hypertarget{Halkieria_evangelista-coding-147}{}
\emph{Halkieria evangelista}: Ring of five mitochondria around the
central centriole \citep{Rice1993}.

\hypertarget{Lingula-coding-147}{}
\emph{Lingula}: ``The midpiece consists of two long mitochondrial rods
connected with each other by a thin mitochondrial lamella'' \citep[in
\emph{Loxosoma}]{Franzen2000}; these are essentially a single organelle
surrounding a central rod of electron-dense material.

\hypertarget{Phoronis-coding-147}{}
\emph{Phoronis}, \emph{Sipunculus}: Two mitochondrial derivatives in
\emph{Flustra} \citep{Franzen1981, Franzen1977}; four in
\emph{Tubulipora} \citep{Franzen1984}.

\hypertarget{Serpula-coding-147}{}
\emph{Serpula}: Following \emph{Discinisca} \emph{tenuis}, described in
Hodgson \& Reunov \citeyearpar{Hodgson1994Ultrastructureof}.

\hypertarget{Wiwaxia_corrugata-coding-147}{}
\emph{Wiwaxia corrugata}: The mitochondria fuse in the middle stage of
spermiogenesis to become a pair of mitochondria
\citep{Reunov2004Ultrastructuralstudy}.

\subsection*{{[}148{]} Centrioles:
Orientation}\label{centrioles-orientation}
\addcontentsline{toc}{subsection}{{[}148{]} Centrioles: Orientation}

\includegraphics{Brachiopod_phylogeny_files/figure-latex/character-mapping-143.pdf}

\begin{quote}
\textbf{Character 148: Gametes: Spermatozoa: Centrioles: Orientation}

0: Orthogonal\\
1: Parallel\\
Neomorphic character.
\end{quote}

Following \citet{Hodgson1994Ultrastructureof}.

\hypertarget{Alisina-coding-148}{}
\emph{Alisina}: \citet{DufresneDube1983}.

\hypertarget{Dailyatia-coding-148}{}
\emph{Dailyatia}: The proximal centriole is parallel to the flagellum
\citep{Gherardi2011}.

\hypertarget{Dentalium-coding-148}{}
\emph{Dentalium}, \emph{Tonicella}: Following Hodgson \& Reunov
\citeyearpar{Hodgson1994Ultrastructureof}.

\hypertarget{Eoobolus-coding-148}{}
\emph{Eoobolus}: Two orthogonal centrioles
\citep{Afzelius1978Finestructure}.

\hypertarget{Phoronis-coding-148}{}
\emph{Phoronis}, \emph{Sipunculus}: \citep{Franzen1981}.

\hypertarget{Serpula-coding-148}{}
\emph{Serpula}: Following \emph{Discinisca} \emph{tenuis}, described in
Hodgson \& Reunov \citeyearpar{Hodgson1994Ultrastructureof}.

\hypertarget{Wiwaxia_corrugata-coding-148}{}
\emph{Wiwaxia corrugata}: Only one centriole in spermatzoon
\citep[p.~7]{Reunov2004Ultrastructuralstudy}, but centrioles are
perpendicularly oriented in spermatogonia (ibid., p.~2).

\subsection*{{[}149{]} Centrioles: Fusion}\label{centrioles-fusion}
\addcontentsline{toc}{subsection}{{[}149{]} Centrioles: Fusion}

\includegraphics{Brachiopod_phylogeny_files/figure-latex/character-mapping-144.pdf}

\begin{quote}
\textbf{Character 149: Gametes: Spermatozoa: Centrioles: Fusion}

0: Discrete\\
1: Fused\\
Neomorphic character.
\end{quote}

Following \citet{Smith2012}; see \citet{BucklandNicks2008}.

\hypertarget{Acanthotretella_spinosa-coding-149}{}
\emph{Acanthotretella spinosa}: Proximal centriole fused lateral to
distal centriole and offset.

\hypertarget{Alisina-coding-149}{}
\emph{Alisina}: Proximal centriole fused anterior to distal centriole
\citep{DufresneDube1983}.

\hypertarget{Dentalium-coding-149}{}
\emph{Dentalium}, \emph{Eoobolus}, \emph{Lingula}, \emph{Wiwaxia
corrugata}: Basal body in deep nuclear fossa.

\hypertarget{Halkieria_evangelista-coding-149}{}
\emph{Halkieria evangelista}, \emph{Phoronis}, \emph{Sipunculus}:
Proximal centriole fused anterior to distal centriole.

\subsection*{{[}150{]} Satellite fibre
complex}\label{satellite-fibre-complex}
\addcontentsline{toc}{subsection}{{[}150{]} Satellite fibre complex}

\includegraphics{Brachiopod_phylogeny_files/figure-latex/character-mapping-145.pdf}

\begin{quote}
\textbf{Character 150: Gametes: Spermatozoa: Satellite fibre complex}

0: Annulus not associated with satellite fibres\\
1: Annulus associated with satellite fibres\\
Neomorphic character.
\end{quote}

Following \citet{Smith2012}, after character 48 in \citet{Ponder1997}.

\subsection*{{[}151{]} Mitochondria: Shape}\label{mitochondria-shape}
\addcontentsline{toc}{subsection}{{[}151{]} Mitochondria: Shape}

\includegraphics{Brachiopod_phylogeny_files/figure-latex/character-mapping-146.pdf}

\begin{quote}
\textbf{Character 151: Gametes: Spermatozoa: Mitochondria: Shape}

1: Spherical to subspherical\\
2: Rods\\
3: Elongate, sac-like\\
Transformational character.
\end{quote}

After character 5 in \citet{BucklandNicks2008}; see also character 43 in
\citet{Ponder1997}.

\hypertarget{Acanthotretella_spinosa-coding-151}{}
\emph{Acanthotretella spinosa}: See \citet{BucklandNicks1988}.

\hypertarget{Lingula-coding-151}{}
\emph{Lingula}: Elongate rods in \emph{Loxosoma} \citep{Franzen2000}.

\hypertarget{Phoronis-coding-151}{}
\emph{Phoronis}, \emph{Sipunculus}: Rods \citep{Franzen1981}.

\subsection*{{[}152{]} Mitochondria: Cristae:
Configuration}\label{mitochondria-cristae-configuration}
\addcontentsline{toc}{subsection}{{[}152{]} Mitochondria: Cristae:
Configuration}

\includegraphics{Brachiopod_phylogeny_files/figure-latex/character-mapping-147.pdf}

\begin{quote}
\textbf{Character 152: Gametes: Spermatozoa: Mitochondria: Cristae:
Configuration}

0: Unmodified\\
1: Arranged in parallel plates\\
Neomorphic character.
\end{quote}

After character 44 in \citet{Ponder1997}. Cristae are internal
compartments formed by inner mitochondrial membranes.

\hypertarget{Lingula-coding-152}{}
\emph{Lingula}: in \emph{Loxosoma} \citep{Franzen2000}.

\hypertarget{Phoronis-coding-152}{}
\emph{Phoronis}, \emph{Sipunculus}: ``Typical cristae''; ``Randomly
oriented'' -- \citet{Franzen1984} (in \emph{Tubulipora}).

\subsection*{{[}153{]} Mitochondria:
Midpiece}\label{mitochondria-midpiece}
\addcontentsline{toc}{subsection}{{[}153{]} Mitochondria: Midpiece}

\includegraphics{Brachiopod_phylogeny_files/figure-latex/character-mapping-148.pdf}

\begin{quote}
\textbf{Character 153: Gametes: Spermatozoa: Mitochondria: Midpiece}

1: Extremely short\\
2: Long\\
3: Forms continuous sheath\\
Transformational character.
\end{quote}

After \citet{Smith2012}; see also character 43 in \citet{Ponder1997};
character 164 in \citet{Giribet2002}.

\hypertarget{Dailyatia-coding-153}{}
\emph{Dailyatia}: Five mitochondria surround the base of the flagellum
in short midpiece, comparable to that of \emph{Sipunculus} and
\emph{Dentalium} \citep{Gherardi2011}.

\hypertarget{Halkieria_evangelista-coding-153}{}
\emph{Halkieria evangelista}: Short ring of five mitochondria around the
central centriole \citep{Rice1993}.

\hypertarget{Lingula-coding-153}{}
\emph{Lingula}: As long as the flagellum in \emph{Loxosoma}
\citep{Franzen2000}.

\hypertarget{Phoronis-coding-153}{}
\emph{Phoronis}, \emph{Sipunculus}: Long \citep{Franzen1981}.

\section{Embryo: Cleavage}\label{embryo-cleavage}

\subsection*{{[}154{]} Equal}\label{equal}
\addcontentsline{toc}{subsection}{{[}154{]} Equal}

\includegraphics{Brachiopod_phylogeny_files/figure-latex/character-mapping-149.pdf}

\begin{quote}
\textbf{Character 154: Embryo: Cleavage: Equal}

1: Unequal\\
2: Equal\\
Transformational character.
\end{quote}

Following character 170 in \citet{Giribet2002}.

\hypertarget{Dentalium-coding-154}{}
\emph{Dentalium}, \emph{Eoobolus}, \emph{Serpula}, \emph{Tonicella}:
Equal, in all brachiopods \citep{Williams1997Introduction}.

\hypertarget{Wiwaxia_corrugata-coding-154}{}
\emph{Wiwaxia corrugata}: ``Cleavage is holoblastic and results in
approximately equal sized, or adequal, blastomeres.'' --
\citet{Pennerstorfer2012}.

\subsection*{{[}155{]} Cross pattern}\label{cross-pattern}
\addcontentsline{toc}{subsection}{{[}155{]} Cross pattern}

\includegraphics{Brachiopod_phylogeny_files/figure-latex/character-mapping-150.pdf}

\begin{quote}
\textbf{Character 155: Embryo: Cleavage: Cross pattern}

0: Absent\\
1: Cross, whether ``molluscan'' or ``annelid''\\
Neomorphic character.
\end{quote}

The ``molluscan cross'' and ``annelid cross'' cannot be systematically
discriminated from one another, so are treated as a single state.\\
See characters 127 \& 128 in \citet{Rouse1999}; 1.49 in
\citet{SPS1996};\\
character 34 in \citet{Haszprunar1996}; 35 in \citet{Haszprunar2000};
172 in \citet{Giribet2002}.

\subsection*{{[}156{]} Polar lobe formation}\label{polar-lobe-formation}
\addcontentsline{toc}{subsection}{{[}156{]} Polar lobe formation}

\includegraphics{Brachiopod_phylogeny_files/figure-latex/character-mapping-151.pdf}

\begin{quote}
\textbf{Character 156: Embryo: Cleavage: Polar lobe formation}

1: Absent\\
2: Present\\
Transformational character.
\end{quote}

Following character 171 in \citet{Giribet2002}.

\subsection*{{[}157{]} Spiral}\label{spiral}
\addcontentsline{toc}{subsection}{{[}157{]} Spiral}

\includegraphics{Brachiopod_phylogeny_files/figure-latex/character-mapping-152.pdf}

\begin{quote}
\textbf{Character 157: Embryo: Cleavage: Spiral}

1: Absent\\
2: Present\\
Transformational character.
\end{quote}

See characters 32--33 in \citet{Haszprunar1996}; character 1.48 in
\citet{SPS1996}; character 29 in \citet{Glenner2004}.

\hypertarget{Phoronis-coding-157}{}
\emph{Phoronis}, \emph{Sipunculus}: ``While entoprocts are spiral
cleavers, ectoprocts show a radial cleavage pattern'' --
\citet{Fuchs2008}.

\hypertarget{Wiwaxia_corrugata-coding-157}{}
\emph{Wiwaxia corrugata}: ``The observed cleavage displays several
characters consistent with the pattern of spiral cleavage''
\citep{Pennerstorfer2012}.

\section{Embryo: Micromere size
{[}158{]}}\label{embryo-micromere-size-158}

\includegraphics{Brachiopod_phylogeny_files/figure-latex/character-mapping-153.pdf}

\begin{quote}
\textbf{Character 158: Embryo: Micromere size}

1: Similar to macomeres\\
2: Small relative to macromeres\\
Transformational character.
\end{quote}

Following \citet{Hejnol2010}. Blastomeres may undergo significant size
differentiation, generating macromeres and micromeres of prominently
different sizes.

\hypertarget{Dentalium-coding-158}{}
\emph{Dentalium}, \emph{Tonicella}: \citet{Williams1997Introduction}.

\hypertarget{Halkieria_evangelista-coding-158}{}
\emph{Halkieria evangelista}: Prominent differentiation in Phascolosoma
\citep{Adrianov2011}.

\hypertarget{Phoronis-coding-158}{}
\emph{Phoronis}, \emph{Sipunculus}: In \emph{Membranipora}, ``cleavage
is slightly unequal resulting in little larger central\\
blastomeres'' \citep{Gruhl2010M}.

\hypertarget{Wiwaxia_corrugata-coding-158}{}
\emph{Wiwaxia corrugata}: Uniform size \citep{Pennerstorfer2012}.

\subsection*{{[}159{]} Origin of mesoderm}\label{origin-of-mesoderm}
\addcontentsline{toc}{subsection}{{[}159{]} Origin of mesoderm}

\includegraphics{Brachiopod_phylogeny_files/figure-latex/character-mapping-154.pdf}

\begin{quote}
\textbf{Character 159: Embryo: Origin of mesoderm}

1: 4d cell, from the blastopore ridge, or as ectomesoderm\\
2: Archenteron\\
Transformational character.
\end{quote}

After characters 32 in \citet{Grobe2007} and 36--37 in
\citet{Glenner2004}.

\hypertarget{Tonicella-coding-159}{}
\emph{Tonicella}: \citet{Williams1997Introduction}.

\section{Larva}\label{larva}

\subsection*{{[}160{]} Metatroch}\label{metatroch}
\addcontentsline{toc}{subsection}{{[}160{]} Metatroch}

\includegraphics{Brachiopod_phylogeny_files/figure-latex/character-mapping-155.pdf}

\begin{quote}
\textbf{Character 160: Larva: Metatroch}

0: Absent\\
1: Present\\
Neomorphic character.
\end{quote}

See characters 129 and 131 in \citet{Rouse1999}; 40 in
\citet{Haszprunar1996}.\\
A prototroch is the defining character of a trochophore larva; a
metatroch is a secondary ciliary ring \citep{Rouse1999}.

\hypertarget{Tonicella-coding-160}{}
\emph{Tonicella}: \citet{Williams1997Introduction}.

\subsection*{{[}161{]} Telotroch}\label{telotroch}
\addcontentsline{toc}{subsection}{{[}161{]} Telotroch}

\includegraphics{Brachiopod_phylogeny_files/figure-latex/character-mapping-156.pdf}

\begin{quote}
\textbf{Character 161: Larva: Telotroch}

0: Absent\\
1: Present\\
Neomorphic character.
\end{quote}

A posterior ciliary band. Character 136 in \citet{Rouse1999}.

\hypertarget{Phoronis-coding-161}{}
\emph{Phoronis}: Absent \citep{Zimmer2013}.

\hypertarget{Sipunculus-coding-161}{}
\emph{Sipunculus}: Absent; single ciliary field lacks telotroch
equivalent \citep{Reed1982}.

\hypertarget{Tonicella-coding-161}{}
\emph{Tonicella}: \citet{Williams1997Introduction}.

\subsection*{{[}162{]} Feeding}\label{feeding}
\addcontentsline{toc}{subsection}{{[}162{]} Feeding}

\includegraphics{Brachiopod_phylogeny_files/figure-latex/character-mapping-157.pdf}

\begin{quote}
\textbf{Character 162: Larva: Feeding}

1: Lecithotrophic (or otherwise non-feeding)\\
2: Planktotrophic (or otherwise feeding)\\
Transformational character.
\end{quote}

Character 140 in \citet{Rouse1999}. See also \citet{Collin1997};
character 2.66 in \citet{SPS1996}; 153 in \citet{Giribet2002}.

\hypertarget{Phoronis-coding-162}{}
\emph{Phoronis}: Metamorphose almost immediately after release from
gonozooid \citep{Zimmer2013}; most bryozoans are lecithotrophic
\citep{Reed1982}.

\hypertarget{Sipunculus-coding-162}{}
\emph{Sipunculus}: Lecithotrophic \citep{Reed1982}.

\section{Larva: Cilia}\label{larva-cilia}

\subsection*{{[}163{]} Ciliated food groove}\label{ciliated-food-groove}
\addcontentsline{toc}{subsection}{{[}163{]} Ciliated food groove}

\includegraphics{Brachiopod_phylogeny_files/figure-latex/character-mapping-158.pdf}

\begin{quote}
\textbf{Character 163: Larva: Cilia: Ciliated food groove}

0: Absent\\
1: Present\\
Neomorphic character.
\end{quote}

Character 132 in \citet{Rouse1999}.

\hypertarget{Phoronis-coding-163}{}
\emph{Phoronis}: Cyclostomes are covered in cilia but not arranged in
food groove.

\hypertarget{Sipunculus-coding-163}{}
\emph{Sipunculus}: Coronal cilia do not form a food groove
\citep{Reed1982}.

\hypertarget{Tonicella-coding-163}{}
\emph{Tonicella}: \citet{Williams1997Introduction}.

\subsection*{{[}164{]} Ciliary bands:
Downstream}\label{ciliary-bands-downstream}
\addcontentsline{toc}{subsection}{{[}164{]} Ciliary bands: Downstream}

\includegraphics{Brachiopod_phylogeny_files/figure-latex/character-mapping-159.pdf}

\begin{quote}
\textbf{Character 164: Larva: Cilia: Ciliary bands: Downstream}

0: Absent\\
1: Present\\
Neomorphic character.
\end{quote}

Downstream-collecting ciliary bands of compound cilia on multiciliated
cells. See character 32 in \citet{Glenner2004}.

\hypertarget{Dailyatia-coding-164}{}
\emph{Dailyatia}: ``Groups such as Sabellariidae {[}\ldots{}{]} have
evolved downstream-feeding without the aid of a metatroch'' --
\citep{Rouse2000}.

\hypertarget{Halkieria_evangelista-coding-164}{}
\emph{Halkieria evangelista}: ``Taxa such as Sipuncula {[}\ldots{}{]}
have a metatroch and do not have downstream larval-feeding'' --
\citet{Rouse2000}.

\subsection*{{[}165{]} Ciliary bands:
Upstream}\label{ciliary-bands-upstream}
\addcontentsline{toc}{subsection}{{[}165{]} Ciliary bands: Upstream}

\includegraphics{Brachiopod_phylogeny_files/figure-latex/character-mapping-160.pdf}

\begin{quote}
\textbf{Character 165: Larva: Cilia: Ciliary bands: Upstream}

0: Absent\\
1: Present\\
Neomorphic character.
\end{quote}

Upstream-collecting ciliary bands with single cilia on monociliated
cells. See character 32 in \citet{Glenner2004}.

\subsection*{{[}166{]} Adoral ciliary band}\label{adoral-ciliary-band}
\addcontentsline{toc}{subsection}{{[}166{]} Adoral ciliary band}

\includegraphics{Brachiopod_phylogeny_files/figure-latex/character-mapping-161.pdf}

\begin{quote}
\textbf{Character 166: Larva: Cilia: Adoral ciliary band}

0: Absent\\
1: Present\\
Neomorphic character.
\end{quote}

Characters 1.50, 2.66 and 4.68 in \citet{SPS1996}; 2 in
\citet{Vinther2008}. See also characters 39 in \citet{Haszprunar1996}
and 153 in \citet{Giribet2002}.

\section{Larva: Nerve ring underlying ciliated larval swimming organ
{[}167{]}}\label{larva-nerve-ring-underlying-ciliated-larval-swimming-organ-167}

\includegraphics{Brachiopod_phylogeny_files/figure-latex/character-mapping-162.pdf}

\begin{quote}
\textbf{Character 167: Larva: Nerve ring underlying ciliated larval
swimming organ}

0: Absent\\
1: Present\\
Neomorphic character.
\end{quote}

Following \citet{Wanninger2009}.

\hypertarget{Phoronis-coding-167}{}
\emph{Phoronis}: Present, following schematic in \citet{Gruhl2016}.

\hypertarget{Sipunculus-coding-167}{}
\emph{Sipunculus}: Nodular nerve ring underlies corona \citep{Reed1982}.

\subsection*{{[}168{]} Paired dorsal setal
bundles}\label{paired-dorsal-setal-bundles}
\addcontentsline{toc}{subsection}{{[}168{]} Paired dorsal setal bundles}

\includegraphics{Brachiopod_phylogeny_files/figure-latex/character-mapping-163.pdf}

\begin{quote}
\textbf{Character 168: Larva: Paired dorsal setal bundles}

0: Absent\\
1: Present\\
Neomorphic character.
\end{quote}

Annelid chaetae are equivalent to the bundled setae expressed in certain
brachiopod larvae. See character 12 in \citet{Vinther2008}.

\hypertarget{Phoronis-coding-168}{}
\emph{Phoronis}: Absent \citep{Zimmer2013}.

\hypertarget{Sipunculus-coding-168}{}
\emph{Sipunculus}: \citep{Reed1982}.

\hypertarget{Tonicella-coding-168}{}
\emph{Tonicella}: \citet{Williams1997Introduction}.

\section{Larva: Apical organ}\label{larva-apical-organ}

\subsection*{{[}169{]} Muscles extending to the
hyposphere}\label{muscles-extending-to-the-hyposphere}
\addcontentsline{toc}{subsection}{{[}169{]} Muscles extending to the
hyposphere}

\includegraphics{Brachiopod_phylogeny_files/figure-latex/character-mapping-164.pdf}

\begin{quote}
\textbf{Character 169: Larva: Apical organ: Muscles extending to the
hyposphere}

0: Absent\\
1: Present\\
Neomorphic character.
\end{quote}

Character 8 in \citet{Vinther2008}.

\hypertarget{Alisina-coding-169}{}
\emph{Alisina}: Apical organ has disappeared before musculature is set
in place \citep{Wanninger2002M}.

\hypertarget{Phoronis-coding-169}{}
\emph{Phoronis}, \emph{Sipunculus}: Median muscles extending from apical
organ \citep{Gruhl2008}.

\hypertarget{Wiwaxia_corrugata-coding-169}{}
\emph{Wiwaxia corrugata}: Not evident \citep[fig. 2C]{Santagata2004}.

\subsection*{{[}170{]} Serotonergic cells}\label{serotonergic-cells}
\addcontentsline{toc}{subsection}{{[}170{]} Serotonergic cells}

\includegraphics{Brachiopod_phylogeny_files/figure-latex/character-mapping-165.pdf}

\begin{quote}
\textbf{Character 170: Larva: Apical organ: Serotonergic cells}

1: Two flask-shaped cells\\
2: Four flask-shaped cells\\
3: Cluster of c. eight flask-shaped cells\\
4: Aggregation of multiple cells of multiple types\\
Transformational character.
\end{quote}

Character 8 in \citet{Haszprunar2008}.

\hypertarget{Acanthotretella_spinosa-coding-170}{}
\emph{Acanthotretella spinosa}: Eight in \emph{Ischnochiton} and
\emph{Mopalia} \citep{Wanninger2007}.

\hypertarget{Dentalium-coding-170}{}
\emph{Dentalium}: Cluster of ``numerous'' serotonergic cells
\citep{HaySchmidt1992, Altenburger2010}; more than, but probably
equivalent to, the flask-shaped cells of \emph{Terebratalia}
\citep{Luter2016}.

\hypertarget{Eoobolus-coding-170}{}
\emph{Eoobolus}: Four flask-shaped cells \citep{Altenburger2010}.

\hypertarget{Halkieria_evangelista-coding-170}{}
\emph{Halkieria evangelista}: Cluster of around eight cells, though not
quite countable \citep{Wanninger2005}.

\hypertarget{Lingula-coding-170}{}
\emph{Lingula}: Six to eight apical cells; eight peripheral cells
\citep{Wanninger2007}, indicating a probable equivalence to
polyplacophorans \citep{Haszprunar2008}.

\hypertarget{Phoronis-coding-170}{}
\emph{Phoronis}, \emph{Sipunculus}: Concentration of 30--40 serotonergic
perikarya \citep[in \emph{Fredericella};][]{Gruhl2010F}.

\hypertarget{Tonicella-coding-170}{}
\emph{Tonicella}: Eight in \emph{Terebratalia} \citep{Luter2016}.

\hypertarget{Wiwaxia_corrugata-coding-170}{}
\emph{Wiwaxia corrugata}: Multiple shapes of cells present
\citep{Santagata2002}; resembles the linguliform arrangement
\citep{Altenburger2010}.

\subsection*{{[}171{]} Develops into adult
brain}\label{develops-into-adult-brain}
\addcontentsline{toc}{subsection}{{[}171{]} Develops into adult brain}

\includegraphics{Brachiopod_phylogeny_files/figure-latex/character-mapping-166.pdf}

\begin{quote}
\textbf{Character 171: Larva: Apical organ: Develops into adult brain}

0: Brain has other origin\\
1: Adult brain derived from larval apical organ / apical pole\\
2:\\
Neomorphic character.
\end{quote}

Character 79 in \citet{Glenner2004}.

\hypertarget{Dentalium-coding-171}{}
\emph{Dentalium}: ``both the larval apical ganglion and the ventral
ganglion must be retained as\\
the adult nervous system'' \citep{HaySchmidt1992}, but not necessarily
as the brain.

\section{Larva: Brain persists into adulthood
{[}172{]}}\label{larva-brain-persists-into-adulthood-172}

\includegraphics{Brachiopod_phylogeny_files/figure-latex/character-mapping-167.pdf}

\begin{quote}
\textbf{Character 172: Larva: Brain persists into adulthood}

1: Brain lost\\
2: Brain retained to adulthood\\
Transformational character.
\end{quote}

After character 3 in \citet{Richter2010}.

\subsection*{{[}173{]} Serotonin-like immunoreactivity in apical
organ}\label{serotonin-like-immunoreactivity-in-apical-organ}
\addcontentsline{toc}{subsection}{{[}173{]} Serotonin-like
immunoreactivity in apical organ}

\includegraphics{Brachiopod_phylogeny_files/figure-latex/character-mapping-168.pdf}

\begin{quote}
\textbf{Character 173: Larva: Serotonin-like immunoreactivity in apical
organ}

1: Two lateral cells in the apical ganglion and the lateral serotonergic
projection to the prototroch;\\
2: Many serotonergic cells in the apical ganglion and a caudal
serotonergic projection to the ciliary band\\
Transformational character.
\end{quote}

(SLI) See \citet{Haszprunar2000}; \citet{Richter2010}\\
Absent (or faint) in \emph{Phascolion} only, so this character only
details the location in taxa in which SLR is readily evident.

\subsection*{{[}174{]} Origin of body
cavity}\label{origin-of-body-cavity}
\addcontentsline{toc}{subsection}{{[}174{]} Origin of body cavity}

\includegraphics{Brachiopod_phylogeny_files/figure-latex/character-mapping-169.pdf}

\begin{quote}
\textbf{Character 174: Larva: Origin of body cavity}

1: Mesenchyme\\
2: Coelom\\
Transformational character.
\end{quote}

Character 1.43 \href{mailto:in@SPS1996}{\nolinkurl{in@SPS1996}}.

\subsection*{{[}175{]} Formation of
coelomoducts}\label{formation-of-coelomoducts}
\addcontentsline{toc}{subsection}{{[}175{]} Formation of coelomoducts}

\includegraphics{Brachiopod_phylogeny_files/figure-latex/character-mapping-170.pdf}

\begin{quote}
\textbf{Character 175: Larva: Formation of coelomoducts}

1: Outgrowth\\
2: Ingrowth\\
Transformational character.
\end{quote}

Character 26 in \citet{Haszprunar2000}.

\hypertarget{Lingula-coding-175}{}
\emph{Lingula}: Coelomoducts absent \citep{Haszprunar2000}.

\subsection*{{[}176{]} Pedal gland}\label{pedal-gland}
\addcontentsline{toc}{subsection}{{[}176{]} Pedal gland}

\includegraphics{Brachiopod_phylogeny_files/figure-latex/character-mapping-171.pdf}

\begin{quote}
\textbf{Character 176: Larva: Foot: Pedal gland}

0: Absent\\
1: Present\\
Neomorphic character.
\end{quote}

A pedal gland is considered evidence for homology between the molluscan
and entoproct foot \citep{Haszprunar2008}.

\hypertarget{Sipunculus-coding-176}{}
\emph{Sipunculus}: Ciliated clef corresponds to position of foot
\citep{Reed1982}, but dedicated foot not present.

\section{Larva: Coelom}\label{larva-coelom}

\subsection*{{[}177{]} Paired}\label{paired}
\addcontentsline{toc}{subsection}{{[}177{]} Paired}

\includegraphics{Brachiopod_phylogeny_files/figure-latex/character-mapping-172.pdf}

\begin{quote}
\textbf{Character 177: Larva: Coelom: Paired}

0: Absent\\
1: Paired coelom originating from two teloblasts derived from 4d\\
Neomorphic character.
\end{quote}

Character 2.02 in \citet{Scheltema1993}.

\hypertarget{Phoronis-coding-177}{}
\emph{Phoronis}: Hypostegal coelom separated from principal
(perigastric) body cavity in cheilostomata -- but this is not clearly
equivalent to the paired coelom intended by this character. The coelom
of \emph{Fredericella} is not paired \citep{Gruhl2010F}.

\hypertarget{Sipunculus-coding-177}{}
\emph{Sipunculus}: No evidence of pairing \citep{Reed1982}.

\subsection*{{[}178{]} Paried: Includes
pericardium}\label{paried-includes-pericardium}
\addcontentsline{toc}{subsection}{{[}178{]} Paried: Includes
pericardium}

\includegraphics{Brachiopod_phylogeny_files/figure-latex/character-mapping-173.pdf}

\begin{quote}
\textbf{Character 178: Larva: Coelom: Paried: Includes pericardium}

0: Paired coelom absent, or does not include pericardium\\
1: Paired coelom includes pericardium\\
Neomorphic character.
\end{quote}

Character 1.03 in \citet{Scheltema1993}.

\section{Larva}\label{larva-1}

\subsection*{{[}179{]} Foot}\label{foot-1}
\addcontentsline{toc}{subsection}{{[}179{]} Foot}

\includegraphics{Brachiopod_phylogeny_files/figure-latex/character-mapping-174.pdf}

\begin{quote}
\textbf{Character 179: Larva: Foot}

0: Absent\\
1: Present\\
Neomorphic character.
\end{quote}

Foot or neurotroch present in larval stage, whether or not it is also
present in mature individuals. Following \citet{Wingstrand1985}.

\hypertarget{Dailyatia-coding-179}{}
\emph{Dailyatia}, \emph{Halkieria evangelista}: \citet{Wingstrand1985}
considers the annelid neurotroch to be potentially homologous with the
molluscan and entoproct foot.

\hypertarget{Lingula-coding-179}{}
\emph{Lingula}: A foot is present in the creeping-type larva of
\emph{Loxosomella} \emph{murmanica}, though absent in \emph{L. atkinsae}
and the many other entoprocts that have swimming-type larvae
\citep{Fuchs2008}.

\section{Ciliary ultrastructure}\label{ciliary-ultrastructure}

\subsection*{{[}180{]} Accessory centriole}\label{accessory-centriole}
\addcontentsline{toc}{subsection}{{[}180{]} Accessory centriole}

\includegraphics{Brachiopod_phylogeny_files/figure-latex/character-mapping-175.pdf}

\begin{quote}
\textbf{Character 180: Ciliary ultrastructure: Accessory centriole}

0: Absent\\
1: Present\\
Neomorphic character.
\end{quote}

After \citet{Lundin2009}.

\hypertarget{Dailyatia-coding-180}{}
\emph{Dailyatia}: Present in certain annelids; not verified in
\emph{Serpula}.

\hypertarget{Tonicella-coding-180}{}
\emph{Tonicella}: Present \citep{Luter1995}.

\subsection*{{[}181{]} Aggregation of granules below basal
plate}\label{aggregation-of-granules-below-basal-plate}
\addcontentsline{toc}{subsection}{{[}181{]} Aggregation of granules
below basal plate}

\includegraphics{Brachiopod_phylogeny_files/figure-latex/character-mapping-176.pdf}

\begin{quote}
\textbf{Character 181: Ciliary ultrastructure: Aggregation of granules
below basal plate}

0: Absent\\
1: Present\\
Neomorphic character.
\end{quote}

After \citet{Lundin2009}.

\hypertarget{Dailyatia-coding-181}{}
\emph{Dailyatia}: Following \emph{Harmothoe} \citep{Holborow1969}.

\subsection*{{[}182{]} Radiating tubular
fibres}\label{radiating-tubular-fibres}
\addcontentsline{toc}{subsection}{{[}182{]} Radiating tubular fibres}

\includegraphics{Brachiopod_phylogeny_files/figure-latex/character-mapping-177.pdf}

\begin{quote}
\textbf{Character 182: Ciliary ultrastructure: Basal foot: Radiating
tubular fibres}

0: Absent\\
1: Present\\
Neomorphic character.
\end{quote}

After \citet{Lundin2009}. Fibres radiate from the distal end of the
basal foot of the cilia in certain taxa.

\hypertarget{Dailyatia-coding-182}{}
\emph{Dailyatia}: Basal foot in \emph{Magelona} is connected to
cytoplasmic microtubules \citep{Bartolomaeus1995}.

\hypertarget{Sipunculus-coding-182}{}
\emph{Sipunculus}: \citet{Reed1982}.

\section{Ciliary ultrastructure: Basal plate
{[}183{]}}\label{ciliary-ultrastructure-basal-plate-183}

\includegraphics{Brachiopod_phylogeny_files/figure-latex/character-mapping-178.pdf}

\begin{quote}
\textbf{Character 183: Ciliary ultrastructure: Basal plate}

1: Thin\\
2: Blurry\\
Transformational character.
\end{quote}

After \citet{Lundin2009}. Also termed ``dense plate''.

\hypertarget{Dailyatia-coding-183}{}
\emph{Dailyatia}: Broad and `blurry' in \emph{Magelona}
\citep{Bartolomaeus1995}.

\hypertarget{Sipunculus-coding-183}{}
\emph{Sipunculus}: \citet{Reed1982}.

\hypertarget{Tonicella-coding-183}{}
\emph{Tonicella}: Thin to thick, but not blurry \citep{Luter1995}.

\subsection*{{[}184{]} Brushborder of
microvilli}\label{brushborder-of-microvilli}
\addcontentsline{toc}{subsection}{{[}184{]} Brushborder of microvilli}

\includegraphics{Brachiopod_phylogeny_files/figure-latex/character-mapping-179.pdf}

\begin{quote}
\textbf{Character 184: Ciliary ultrastructure: Brushborder of
microvilli}

0: Absent\\
1: Present\\
Neomorphic character.
\end{quote}

After \citet{Lundin2009}; coded following \citet{Smith2012}.

\hypertarget{Sipunculus-coding-184}{}
\emph{Sipunculus}: Present \citep{Reed1982}.

\hypertarget{Tonicella-coding-184}{}
\emph{Tonicella}: Absent \citep{Luter1995}.

\subsection*{{[}185{]} Centriolar triplet derivative in basal
body}\label{centriolar-triplet-derivative-in-basal-body}
\addcontentsline{toc}{subsection}{{[}185{]} Centriolar triplet
derivative in basal body}

\includegraphics{Brachiopod_phylogeny_files/figure-latex/character-mapping-180.pdf}

\begin{quote}
\textbf{Character 185: Ciliary ultrastructure: Centriolar triplet
derivative in basal body}

1: 9 + 2 pattern\\
2: 9 + 3 pattern\\
Transformational character.
\end{quote}

After \citet{Lundin2009}.

\hypertarget{Dailyatia-coding-185}{}
\emph{Dailyatia}: Following \emph{Enchytraeus} \citep{Reger1967},
\emph{Magelona} \citep{Bartolomaeus1995} and \emph{Harmothoe}
\citep{Holborow1969}.

\hypertarget{Sipunculus-coding-185}{}
\emph{Sipunculus}: \citet{Reed1982}.

\subsection*{{[}186{]} Ciliary necklace with connecting
strands}\label{ciliary-necklace-with-connecting-strands}
\addcontentsline{toc}{subsection}{{[}186{]} Ciliary necklace with
connecting strands}

\includegraphics{Brachiopod_phylogeny_files/figure-latex/character-mapping-181.pdf}

\begin{quote}
\textbf{Character 186: Ciliary ultrastructure: Ciliary necklace with
connecting strands}

0: Absent\\
1: Present\\
Neomorphic character.
\end{quote}

After \citet{Lundin2009}.\\
The ciliary necklace is defined by \citet{Gilula1972} as ``Well-defined
rows or strands of membrane particles that encircle the ciliary shaft''.
It occurs immediately below the basal plate, and comprises three beaded
circles of on the circumference of the cilia membrane.

\hypertarget{Dailyatia-coding-186}{}
\emph{Dailyatia}: Not evident in \emph{Enchytraeus} \citep{Reger1967},
\emph{Magelona} \citep{Bartolomaeus1995} or \emph{Harmothoe}
\citep{Holborow1969}.

\hypertarget{Sipunculus-coding-186}{}
\emph{Sipunculus}: \citet{Reed1982}.

\hypertarget{Tonicella-coding-186}{}
\emph{Tonicella}: \citep{Luter1995}.

\section{Ciliary ultrastructure: Compound
cilia}\label{ciliary-ultrastructure-compound-cilia}

\subsection*{{[}187{]} Presence}\label{presence-3}
\addcontentsline{toc}{subsection}{{[}187{]} Presence}

\includegraphics{Brachiopod_phylogeny_files/figure-latex/character-mapping-182.pdf}

\begin{quote}
\textbf{Character 187: Ciliary ultrastructure: Compound cilia: Presence}

0: Absent\\
1: Present\\
Neomorphic character.
\end{quote}

After \citet{Lundin2009}. Compound cilia are motile structures composed
of 10--100 regular cilia used in locomotion or feeding.

\hypertarget{Dailyatia-coding-187}{}
\emph{Dailyatia}: \citet{Nielsen1987}.

\hypertarget{Sipunculus-coding-187}{}
\emph{Sipunculus}: \citet{Reed1982}.

\subsection*{{[}188{]} Origin}\label{origin-1}
\addcontentsline{toc}{subsection}{{[}188{]} Origin}

\includegraphics{Brachiopod_phylogeny_files/figure-latex/character-mapping-183.pdf}

\begin{quote}
\textbf{Character 188: Ciliary ultrastructure: Compound cilia: Origin}

1: Several monociliate cells\\
2: On multiciliated cell\\
Transformational character.
\end{quote}

Character 14 in \citet{Glenner2004}. Compound cilia can be produced by
the aggregation of cilia from multiple monociliate cells, or from a
single cell bearing multiple cilia \citep{Nielsen1987}.

\hypertarget{Tonicella-coding-188}{}
\emph{Tonicella}: ``The coelothelial cells of the metacoel are
monociliated''; ``even some epithelial muscle cells are monociliated''
-- \citet{Luter1995}.

\section{Ciliary ultrastructure: Glycocalyx ultrastructure
{[}189{]}}\label{ciliary-ultrastructure-glycocalyx-ultrastructure-189}

\includegraphics{Brachiopod_phylogeny_files/figure-latex/character-mapping-184.pdf}

\begin{quote}
\textbf{Character 189: Ciliary ultrastructure: Glycocalyx
ultrastructure}

1: Homogeneous\\
2: Layered\\
Transformational character.
\end{quote}

After \citet{Lundin2009}.

\hypertarget{Sipunculus-coding-189}{}
\emph{Sipunculus}: \citet{Reed1982}.

\hypertarget{Tonicella-coding-189}{}
\emph{Tonicella}: Homogeneous \citep{Luter1995}.

\subsection*{{[}190{]} Branched}\label{branched}
\addcontentsline{toc}{subsection}{{[}190{]} Branched}

\includegraphics{Brachiopod_phylogeny_files/figure-latex/character-mapping-185.pdf}

\begin{quote}
\textbf{Character 190: Ciliary ultrastructure: Microvilli on epidermal
surface: Branched}

0: Unbranched\\
1: Branched\\
Neomorphic character.
\end{quote}

After \citet{Lundin2009}.

\hypertarget{Sipunculus-coding-190}{}
\emph{Sipunculus}: \citet{Reed1982}.

\hypertarget{Tonicella-coding-190}{}
\emph{Tonicella}: \citep{Luter1995}.

\section{Ciliary ultrastructure: Vertical ciliary
rootlet}\label{ciliary-ultrastructure-vertical-ciliary-rootlet}

\subsection*{{[}191{]} Length}\label{length}
\addcontentsline{toc}{subsection}{{[}191{]} Length}

\includegraphics{Brachiopod_phylogeny_files/figure-latex/character-mapping-186.pdf}

\begin{quote}
\textbf{Character 191: Ciliary ultrastructure: Vertical ciliary rootlet:
Length}

1: Short\\
2: Long\\
Transformational character.
\end{quote}

After \citet{Lundin2009}. The vertical ciliary rootlet is also termed
the posterior rootlet.

\hypertarget{Lingula-coding-191}{}
\emph{Lingula}: Details of ciliary ultrastructure are illustrated in
\citet{Nielsen1976}.

\hypertarget{Sipunculus-coding-191}{}
\emph{Sipunculus}: \citet{Reed1982}.

\hypertarget{Tonicella-coding-191}{}
\emph{Tonicella}: Long \citep{Luter1995}.

\subsection*{{[}192{]} Shape}\label{shape}
\addcontentsline{toc}{subsection}{{[}192{]} Shape}

\includegraphics{Brachiopod_phylogeny_files/figure-latex/character-mapping-187.pdf}

\begin{quote}
\textbf{Character 192: Ciliary ultrastructure: Vertical ciliary rootlet:
Shape}

1: Conical\\
2: Flat\\
Transformational character.
\end{quote}

After \citet{Lundin2009}. The vertical ciliary rootlet is also termed
the posterior rootlet.

\hypertarget{Dailyatia-coding-192}{}
\emph{Dailyatia}: Conical in \emph{Enchytraeus} \citep{Reger1967} and
\emph{Magelona} \citep{Bartolomaeus1995}.

\hypertarget{Sipunculus-coding-192}{}
\emph{Sipunculus}: \citet{Reed1982}.

\hypertarget{Tonicella-coding-192}{}
\emph{Tonicella}: Conical: tapering to a point \citep{Luter1995}.

\section{Ciliary ultrastructure: Secondary ciliary
rootlet}\label{ciliary-ultrastructure-secondary-ciliary-rootlet}

\subsection*{{[}193{]} Presence}\label{presence-4}
\addcontentsline{toc}{subsection}{{[}193{]} Presence}

\includegraphics{Brachiopod_phylogeny_files/figure-latex/character-mapping-188.pdf}

\begin{quote}
\textbf{Character 193: Ciliary ultrastructure: Secondary ciliary
rootlet: Presence}

0: Absent\\
1: Present\\
Neomorphic character.
\end{quote}

After \citet{Lundin2009}. The secondary ciliary rootlet is also termed
the anterior ciliary rootlet.

\hypertarget{Sipunculus-coding-193}{}
\emph{Sipunculus}: \citet{Reed1982}.

\subsection*{{[}194{]} Length}\label{length-1}
\addcontentsline{toc}{subsection}{{[}194{]} Length}

\includegraphics{Brachiopod_phylogeny_files/figure-latex/character-mapping-189.pdf}

\begin{quote}
\textbf{Character 194: Ciliary ultrastructure: Secondary ciliary
rootlet: Length}

1: Short\\
2: Long\\
Transformational character.
\end{quote}

After \citet{Lundin2009}. The secondary ciliary rootlet is also termed
the anterior ciliary rootlet.

\hypertarget{Dailyatia-coding-194}{}
\emph{Dailyatia}: Short in \emph{Enchytraeus} \citep{Reger1967},
\emph{Magelona} \citep{Bartolomaeus1995} and \emph{Harmothoe}
\citep{Holborow1969}.

\hypertarget{Sipunculus-coding-194}{}
\emph{Sipunculus}: \citet{Reed1982}.

\hypertarget{Tonicella-coding-194}{}
\emph{Tonicella}: ``Very small'' -- \citet{Luter1995}.

\subsection*{{[}195{]} Shape}\label{shape-1}
\addcontentsline{toc}{subsection}{{[}195{]} Shape}

\includegraphics{Brachiopod_phylogeny_files/figure-latex/character-mapping-190.pdf}

\begin{quote}
\textbf{Character 195: Ciliary ultrastructure: Secondary ciliary
rootlet: Shape}

1: Conical\\
2: Flat\\
Transformational character.
\end{quote}

After \citet{Lundin2009}. The secondary ciliary rootlet is also termed
the anterior ciliary rootlet.

\hypertarget{Dailyatia-coding-195}{}
\emph{Dailyatia}: Conical in \emph{Magelona} \citep{Bartolomaeus1995}.

\hypertarget{Sipunculus-coding-195}{}
\emph{Sipunculus}: \citet{Reed1982}.

\hypertarget{Tonicella-coding-195}{}
\emph{Tonicella}: Too small to evaluate.

\section{Nephridia}\label{nephridia}

\subsection*{{[}196{]} Podocytes}\label{podocytes}
\addcontentsline{toc}{subsection}{{[}196{]} Podocytes}

\includegraphics{Brachiopod_phylogeny_files/figure-latex/character-mapping-191.pdf}

\begin{quote}
\textbf{Character 196: Nephridia: Podocytes}

0: Absent\\
1: Present\\
Neomorphic character.
\end{quote}

See characters 21 and 28 in \citet{Haszprunar2000}; 1.12 in
\citet{Scheltema1993}.

\hypertarget{Dailyatia-coding-196}{}
\emph{Dailyatia}: Present in serpulids \citep{Bartolomaeus2005}.

\hypertarget{Dentalium-coding-196}{}
\emph{Dentalium}, \emph{Eoobolus}, \emph{Serpula}, \emph{Tonicella}:
``In Brachiopoda, podocytes have never been observed'' --
\citet{Luter1995}.

\hypertarget{Wiwaxia_corrugata-coding-196}{}
\emph{Wiwaxia corrugata}: Present \citep{Storch1978}.

\subsection*{{[}197{]} Rhogocytes}\label{rhogocytes}
\addcontentsline{toc}{subsection}{{[}197{]} Rhogocytes}

\includegraphics{Brachiopod_phylogeny_files/figure-latex/character-mapping-192.pdf}

\begin{quote}
\textbf{Character 197: Nephridia: Rhogocytes}

0: Absent\\
1: Present\\
Neomorphic character.
\end{quote}

Pore cells. Character 20 in \citet{Haszprunar2000}.

\subsection*{{[}198{]} Serve as excretory
organs}\label{serve-as-excretory-organs}
\addcontentsline{toc}{subsection}{{[}198{]} Serve as excretory organs}

\includegraphics{Brachiopod_phylogeny_files/figure-latex/character-mapping-193.pdf}

\begin{quote}
\textbf{Character 198: Nephridia: Serve as excretory organs}

0: No\\
1: Yes\\
Neomorphic character.
\end{quote}

See character 4.46 in \citet{SPS1996}.

\hypertarget{Dentalium-coding-198}{}
\emph{Dentalium}, \emph{Eoobolus}, \emph{Serpula}, \emph{Tonicella}:
``The excretory function of the metanephridia in Brachiopoda must be
questioned'' -- \citet{Luter1995}.

\subsection*{{[}199{]} Protonephridia}\label{protonephridia}
\addcontentsline{toc}{subsection}{{[}199{]} Protonephridia}

\includegraphics{Brachiopod_phylogeny_files/figure-latex/character-mapping-194.pdf}

\begin{quote}
\textbf{Character 199: Nephridia: Protonephridia}

0: Absent\\
1: Present\\
Neomorphic character.
\end{quote}

Also termed cyrtocytes. Character 21 in \citet{Grobe2007}; 1.47 in
\citet{SPS1996}; 138 in \citet{Rouse1999}; 20 in \citet{Haszprunar1996};
90 in \citet{Glenner2004}.

\subsection*{{[}200{]} Metanephridia}\label{metanephridia}
\addcontentsline{toc}{subsection}{{[}200{]} Metanephridia}

\includegraphics{Brachiopod_phylogeny_files/figure-latex/character-mapping-195.pdf}

\begin{quote}
\textbf{Character 200: Nephridia: Metanephridia}

0: Absent\\
1: Present\\
Neomorphic character.
\end{quote}

See characters 35 in \citet{Rouse1999}; 28 in \citet{Haszprunar2000}; 93
in \citet{Glenner2004}; 1.47 in \citet{SPS1996}; 21 in
\citet{Grobe2007}; 138 in \citet{Rouse1999}; 20 in
\citet{Haszprunar1996}.

\section{Cuticle}\label{cuticle}

\subsection*{{[}201{]} Layers}\label{layers}
\addcontentsline{toc}{subsection}{{[}201{]} Layers}

\includegraphics{Brachiopod_phylogeny_files/figure-latex/character-mapping-196.pdf}

\begin{quote}
\textbf{Character 201: Cuticle: Layers}

0: Simple (i.e.~glycocalyx)\\
1: Distinct epicuticle and endocuticle\\
Neomorphic character.
\end{quote}

Character 1 in \citet{Haszprunar1996}.

\subsection*{{[}202{]} Composition}\label{composition}
\addcontentsline{toc}{subsection}{{[}202{]} Composition}

\includegraphics{Brachiopod_phylogeny_files/figure-latex/character-mapping-197.pdf}

\begin{quote}
\textbf{Character 202: Cuticle: Composition}

1: Chitinous\\
2: Collagen\\
Transformational character.
\end{quote}

Character 2 in \citet{Haszprunar2008}.

\hypertarget{Acanthotretella_spinosa-coding-202}{}
\emph{Acanthotretella spinosa}, \emph{Alisina}: \citet{Haszprunar2008}.

\hypertarget{Dentalium-coding-202}{}
\emph{Dentalium}, \emph{Serpula}, \emph{Tonicella}: The brachiopod
pedicle has a chitinous cuticle
\citep{Williams1997Introduction, MacKay1978}, but the tentacles are
associated with collagen \citep{Williams1997Introduction}; marked as
polymorphic.

\hypertarget{Eoobolus-coding-202}{}
\emph{Eoobolus}: No (chitinous) pedicle, so only collagenous cuticle
present \citep{Williams1997Introduction}.

\hypertarget{Halkieria_evangelista-coding-202}{}
\emph{Halkieria evangelista}: Collagenous \citep{Goffinet1978}.

\hypertarget{Lingula-coding-202}{}
\emph{Lingula}: Absent \citep{Haszprunar2008}. Chitin is occasionally
present in certain species, perhaps in regions where rigidity is
necessary \citep{Borisanova2015}.

\hypertarget{Phoronis-coding-202}{}
\emph{Phoronis}, \emph{Sipunculus}: Collagenous \citep{Schopf1967},
though chitin is associated with the exoskeleton \citep{Hunt1972}.

\hypertarget{Wiwaxia_corrugata-coding-202}{}
\emph{Wiwaxia corrugata}: Collagen fibres in tentacle cuticle
\citep{Bartolomaeus2001U}; chitin only present in tubes
\citep{Jeuniaux1971}.

\subsection*{{[}203{]} Fibrous layer with thick
fibrils}\label{fibrous-layer-with-thick-fibrils}
\addcontentsline{toc}{subsection}{{[}203{]} Fibrous layer with thick
fibrils}

\includegraphics{Brachiopod_phylogeny_files/figure-latex/character-mapping-198.pdf}

\begin{quote}
\textbf{Character 203: Cuticle: Fibrous layer with thick fibrils}

0: Absent\\
1: Present\\
Neomorphic character.
\end{quote}

After \citet{Borisanova2015}.

\hypertarget{Acanthotretella_spinosa-coding-203}{}
\emph{Acanthotretella spinosa}, \emph{Alisina}, \emph{Dailyatia},
\emph{Lingula}, \emph{Phoronis}, \emph{Sipunculus}: Following table 2 in
\citet{Borisanova2015}.

\hypertarget{Dentalium-coding-203}{}
\emph{Dentalium}: Pedicle cuticle entirely homogeneous
\citep{Williams1997Introduction}.

\hypertarget{Halkieria_evangelista-coding-203}{}
\emph{Halkieria evangelista}: Fibrous collagen only
\citep{BereiterHahn1984}.

\hypertarget{Serpula-coding-203}{}
\emph{Serpula}: Microvilli in otherwise homogeneous epidermis
\citep{Williams1997Introduction}.

\hypertarget{Tonicella-coding-203}{}
\emph{Tonicella}: Not evident in \emph{Notosaria}
\citep{BereiterHahn1984, Williams1997Introduction}.

\hypertarget{Wiwaxia_corrugata-coding-203}{}
\emph{Wiwaxia corrugata}: Outer layer seemingly fibrous
\citep{BereiterHahn1984}.

\subsection*{{[}204{]} Homogeneous layer}\label{homogeneous-layer}
\addcontentsline{toc}{subsection}{{[}204{]} Homogeneous layer}

\includegraphics{Brachiopod_phylogeny_files/figure-latex/character-mapping-199.pdf}

\begin{quote}
\textbf{Character 204: Cuticle: Homogeneous layer}

0: Absent\\
1: Present\\
Neomorphic character.
\end{quote}

After \citet{Borisanova2015}.

\hypertarget{Acanthotretella_spinosa-coding-204}{}
\emph{Acanthotretella spinosa}, \emph{Alisina}, \emph{Dailyatia},
\emph{Lingula}, \emph{Phoronis}, \emph{Sipunculus}: Following table 2 in
\citet{Borisanova2015}.

\hypertarget{Dentalium-coding-204}{}
\emph{Dentalium}: Pedicle cuticle entirely homogeneous
\citep{Williams1997Introduction}.

\hypertarget{Halkieria_evangelista-coding-204}{}
\emph{Halkieria evangelista}: Fibrous collagen only
\citep{BereiterHahn1984}.

\hypertarget{Serpula-coding-204}{}
\emph{Serpula}: Microvilli in otherwise homogeneous epidermis
\citep{Williams1997Introduction}.

\hypertarget{Tonicella-coding-204}{}
\emph{Tonicella}: Cuticle is homogeneous in \emph{Notosaria}
\citep{BereiterHahn1984, Williams1997Introduction}.

\hypertarget{Wiwaxia_corrugata-coding-204}{}
\emph{Wiwaxia corrugata}: Not evident \citep{BereiterHahn1984}.

\subsection*{{[}205{]} Resilience}\label{resilience}
\addcontentsline{toc}{subsection}{{[}205{]} Resilience}

\includegraphics{Brachiopod_phylogeny_files/figure-latex/character-mapping-200.pdf}

\begin{quote}
\textbf{Character 205: Cuticle: Resilience}

0: Labile\\
1: Robust\\
2:\\
Neomorphic character.
\end{quote}

Character 1 in \citet{Haszprunar2000}.

\subsection*{{[}206{]} Microvilli}\label{microvilli}
\addcontentsline{toc}{subsection}{{[}206{]} Microvilli}

\includegraphics{Brachiopod_phylogeny_files/figure-latex/character-mapping-201.pdf}

\begin{quote}
\textbf{Character 206: Cuticle: Microvilli}

0: Absent\\
1: Microvilli present in the cuticle\\
Neomorphic character.
\end{quote}

After \citet{Borisanova2015}.

\hypertarget{Acanthotretella_spinosa-coding-206}{}
\emph{Acanthotretella spinosa}, \emph{Alisina}, \emph{Dailyatia},
\emph{Lingula}, \emph{Phoronis}, \emph{Sipunculus}: Following table 2 in
\citet{Borisanova2015}.

\hypertarget{Halkieria_evangelista-coding-206}{}
\emph{Halkieria evangelista}: Fibrous collagen only
\citep{BereiterHahn1984}.

\hypertarget{Serpula-coding-206}{}
\emph{Serpula}: Microvillios inner epithelium in Discina
\citep{Williams1997Introduction}.

\hypertarget{Wiwaxia_corrugata-coding-206}{}
\emph{Wiwaxia corrugata}: Present on outer epithelium
\citep{BereiterHahn1984}.

\section{Muscles}\label{muscles}

\subsection*{{[}207{]} Cytology}\label{cytology}
\addcontentsline{toc}{subsection}{{[}207{]} Cytology}

\includegraphics{Brachiopod_phylogeny_files/figure-latex/character-mapping-202.pdf}

\begin{quote}
\textbf{Character 207: Muscles: Cytology}

1: Smooth\\
2: Obliquely striated\\
3: Smooth on abfrontal face; striated on frontal face\\
Transformational character.
\end{quote}

Character 19 in \citet{Haszprunar1996}; see also character 13 in
\citet{Haszprunar2000}.

\hypertarget{Dentalium-coding-207}{}
\emph{Dentalium}, \emph{Eoobolus}, \emph{Serpula}, \emph{Tonicella}: In
brachiopods, myofibrils ``are striated on the frontal face and smooth on
the abfrontal face'' \citep{Pardos1991}.

\hypertarget{Phoronis-coding-207}{}
\emph{Phoronis}, \emph{Sipunculus}: In Bryozoa, myofibrils are ``all
striated'' \citep{Pardos1991}.

\hypertarget{Wiwaxia_corrugata-coding-207}{}
\emph{Wiwaxia corrugata}: ``In P. australis {[}\ldots{}{]} all the
myofibrils belong to the smooth type'' -- \citet{Pardos1991}.

\subsection*{{[}208{]} Histology}\label{histology}
\addcontentsline{toc}{subsection}{{[}208{]} Histology}

\includegraphics{Brachiopod_phylogeny_files/figure-latex/character-mapping-203.pdf}

\begin{quote}
\textbf{Character 208: Muscles: Histology}

0: Fibre-type\\
1: Epithelially organized\\
2:\\
Neomorphic character.
\end{quote}

See character 18 in \citet{Haszprunar1996}.

\section{Glands}\label{glands-1}

\subsection*{{[}209{]} Pedal gland}\label{pedal-gland-1}
\addcontentsline{toc}{subsection}{{[}209{]} Pedal gland}

\includegraphics{Brachiopod_phylogeny_files/figure-latex/character-mapping-204.pdf}

\begin{quote}
\textbf{Character 209: Glands: Pedal gland}

0: Absent\\
1: Present\\
Neomorphic character.
\end{quote}

Characters 1.13, 1.40 \& 2.08 in \citet{Scheltema1993}; 114 in
\citet{Giribet2002}; 1.53 in \citet{SPS1996}; 9 in
\citet{Haszprunar1996}.

\subsection*{{[}210{]} Paired pharyngeal
diverticulae}\label{paired-pharyngeal-diverticulae}
\addcontentsline{toc}{subsection}{{[}210{]} Paired pharyngeal
diverticulae}

\includegraphics{Brachiopod_phylogeny_files/figure-latex/character-mapping-205.pdf}

\begin{quote}
\textbf{Character 210: Glands: Paired pharyngeal diverticulae}

0: Absent\\
1: Present\\
Neomorphic character.
\end{quote}

\section{Body organization: Circulatory system
{[}211{]}}\label{body-organization-circulatory-system-211}

\includegraphics{Brachiopod_phylogeny_files/figure-latex/character-mapping-206.pdf}

\begin{quote}
\textbf{Character 211: Body organization: Circulatory system}

1: Epithelially lined\\
2: Poorly defined, involving sinuses and lacunae\\
3: Closed circulatory system\\
Transformational character.
\end{quote}

After character 23 in \citet{Haszprunar1996}; 24 in
\citet{Haszprunar2000}; 41 in \citet{Rouse1999}; 16 in
\citet{Scheltema1993}; 16 in \citet{Vinther2008}; 5 in
\citet{Haszprunar2008}.

\hypertarget{Acanthotretella_spinosa-coding-211}{}
\emph{Acanthotretella spinosa}, \emph{Alisina}, \emph{Lingula}: See
\citet{Haszprunar2008}.

\hypertarget{Halkieria_evangelista-coding-211}{}
\emph{Halkieria evangelista}: Open circulatory system.

\hypertarget{Phoronis-coding-211}{}
\emph{Phoronis}, \emph{Sipunculus}: As Brachiopods, sipunculans and
relatives \citep{Ruppert1983}.

\section{Nervous system}\label{nervous-system}

\subsection*{{[}212{]} Orthogonal}\label{orthogonal}
\addcontentsline{toc}{subsection}{{[}212{]} Orthogonal}

\includegraphics{Brachiopod_phylogeny_files/figure-latex/character-mapping-207.pdf}

\begin{quote}
\textbf{Character 212: Nervous system: Orthogonal}

0: Not orthogonal\\
1: Orthogonal\\
Neomorphic character.
\end{quote}

Character 14 in \citet{Haszprunar1996}. Paired longitudinal nerve cords
regularly interconnected by transversal commissures to form a
rectangular pattern.

\hypertarget{Sipunculus-coding-212}{}
\emph{Sipunculus}: \citet{Temereva2016Thenervous}.

\subsection*{{[}213{]} Glial system}\label{glial-system}
\addcontentsline{toc}{subsection}{{[}213{]} Glial system}

\includegraphics{Brachiopod_phylogeny_files/figure-latex/character-mapping-208.pdf}

\begin{quote}
\textbf{Character 213: Nervous system: Glial system}

0: Absent\\
1: Present\\
Neomorphic character.
\end{quote}

Character 16 in \citet{Haszprunar1996}. The Gliointerstitial system
interconnects the nervous and muscle systems.

\hypertarget{Wiwaxia_corrugata-coding-213}{}
\emph{Wiwaxia corrugata}: Glial cells are ``abundant''
\citep{Temereva2016Phoronida}.

\subsection*{{[}214{]} Buccal nerve ring}\label{buccal-nerve-ring}
\addcontentsline{toc}{subsection}{{[}214{]} Buccal nerve ring}

\includegraphics{Brachiopod_phylogeny_files/figure-latex/character-mapping-209.pdf}

\begin{quote}
\textbf{Character 214: Nervous system: Buccal nerve ring}

0: Absent\\
1: Present\\
Neomorphic character.
\end{quote}

Character 7b in \citet{Haszprunar2008}.

\hypertarget{Sipunculus-coding-214}{}
\emph{Sipunculus}: \citet{Temereva2016Thenervous}.

\subsection*{{[}215{]} Anterior nerve loop}\label{anterior-nerve-loop}
\addcontentsline{toc}{subsection}{{[}215{]} Anterior nerve loop}

\includegraphics{Brachiopod_phylogeny_files/figure-latex/character-mapping-210.pdf}

\begin{quote}
\textbf{Character 215: Nervous system: Anterior nerve loop}

0: Absent\\
1: Present\\
Neomorphic character.
\end{quote}

Character 7c in \citet{Haszprunar2008}. A pre-oral nerve loop is present
in molluscs, \emph{Loxosomella} and certain annelids
\citep{Wanninger2007}.

\hypertarget{Sipunculus-coding-215}{}
\emph{Sipunculus}: \citet{Temereva2016Thenervous}.

\subsection*{{[}216{]} Formation of ganglia}\label{formation-of-ganglia}
\addcontentsline{toc}{subsection}{{[}216{]} Formation of ganglia}

\includegraphics{Brachiopod_phylogeny_files/figure-latex/character-mapping-211.pdf}

\begin{quote}
\textbf{Character 216: Nervous system: Formation of ganglia}

1: From cerebral region\\
2: In situ\\
3: Invagination of epithelium\\
Transformational character.
\end{quote}

Character 1.22 in \citet{SPS1996}.

\hypertarget{Phoronis-coding-216}{}
\emph{Phoronis}, \emph{Sipunculus}: ``The cerebral ganglion in all
bryozoans is formed as an invagination of a portion\\
of epithelium'' -- \citet{Temereva2016Thenervous}.

\subsection*{{[}217{]} Presence}\label{presence-5}
\addcontentsline{toc}{subsection}{{[}217{]} Presence}

\includegraphics{Brachiopod_phylogeny_files/figure-latex/character-mapping-212.pdf}

\begin{quote}
\textbf{Character 217: Nervous system: Cerebral ganglia: Presence}

0: Absent\\
1: Present\\
Neomorphic character.
\end{quote}

After character 13 in \citet{Haszprunar1996}.

\hypertarget{Sipunculus-coding-217}{}
\emph{Sipunculus}: \citet{Temereva2016Thenervous}.

\hypertarget{Wiwaxia_corrugata-coding-217}{}
\emph{Wiwaxia corrugata}: We treat the dorsal ganglion, which is formed
by two ends of the tentacular nerve ring \citep{Temereva2016Phoronida},
as cerebral.

\section{Nervous system}\label{nervous-system-1}

\subsection*{{[}218{]} Fused}\label{fused}
\addcontentsline{toc}{subsection}{{[}218{]} Fused}

\includegraphics{Brachiopod_phylogeny_files/figure-latex/character-mapping-213.pdf}

\begin{quote}
\textbf{Character 218: Nervous system: Cerebral gangila: Fused}

1: Pair of distinct ganglia\\
2: Single ganglion, or fused ganglia\\
Transformational character.
\end{quote}

After character 13 in \citet{Haszprunar1996}.

\hypertarget{Sipunculus-coding-218}{}
\emph{Sipunculus}: Fused \citep{Temereva2016Thenervous}.

\section{Nervous system: Nerve cords
{[}219{]}}\label{nervous-system-nerve-cords-219}

\includegraphics{Brachiopod_phylogeny_files/figure-latex/character-mapping-214.pdf}

\begin{quote}
\textbf{Character 219: Nervous system: Nerve cords}

1: Ventral nerve cords only\\
2: Tetraneury: one pair of ventral and one pair of lateral nerve cords\\
Transformational character.
\end{quote}

See character 7 in \citet{Haszprunar2008}.

\hypertarget{Sipunculus-coding-219}{}
\emph{Sipunculus}: \citet{Temereva2016Thenervous}.

\subsection*{{[}220{]} Ventral longitudinal
nerves}\label{ventral-longitudinal-nerves}
\addcontentsline{toc}{subsection}{{[}220{]} Ventral longitudinal nerves}

\includegraphics{Brachiopod_phylogeny_files/figure-latex/character-mapping-215.pdf}

\begin{quote}
\textbf{Character 220: Nervous system: Ventral longitudinal nerves}

0: Separate\\
1: Paired or secondarily fused\\
Neomorphic character.
\end{quote}

Character 80 in \citet{Glenner2004}; see also character 6 in
\citet{Vinther2008}.

\hypertarget{Sipunculus-coding-220}{}
\emph{Sipunculus}: \citet{Temereva2016Thenervous}.

\hypertarget{fitch}{\chapter{Fitch parsimony}\label{fitch}}

Parsimony search was conducted in TNT v1.5 \citep{Goloboff2016} using
ratchet and tree drifting heuristics \citep{Goloboff1999, Nixon1999},
repeating the search until the optimal score had been hit by 1500
independent searches:

\begin{quote}
xmult:rat10 drift10 hits 1500 level 4 chklevel 5;
\end{quote}

Searches were conducted under equal weights and results saved to file:

\begin{quote}
piwe-; xmult; {/* Conduct search with equal weighting */}

tsav *TNT/ew.tre;sav;tsav/; {/* Save results to file */}

keep 0; hold 10000; {/* Clear trees from memory */}
\end{quote}

Further searches were conducted under extended implied weighting
\citep{Goloboff1997, Goloboff2014}, under the concavity constants 2, 3,
4.5, 7, 10.5, 16 and 24:

\begin{quote}
xpiwe=; {/* Enable extended implied weighting */}

piwe=2; xmult; {/* Conduct analysis at k = 2 */}

tsav *TNT/xpiwe2.tre; sav; tsav/; {/* Save results to file */}

keep 0; hold 10000; {/* Clear trees from memory */}

piwe=3; xmult; {/* Conduct analysis at k = 3 */}

tsav *TNT/xpiwe3.tre; sav ;tsav/; {/* Save results to file */}
\end{quote}

We acknowledge the Willi Hennig Society for their sponsorship of the TNT
software.

\section{Results}\label{results-1}









\begin{figure}
\centering
\includegraphics{Brachiopod_phylogeny_files/figure-latex/tnt-iw-consensus-1.pdf}
\caption{\label{fig:tnt-iw-consensus}Strict consensus of all trees recovered by TNT
using Fitch parsimony with implied weighting at all values of \emph{k}, and at the individual
values \emph{k} = 2, 3 and 4.5.
The consensus of all implied weights runs is
not very well resolved, largely due to a few wildcard taxa, particularly
at \(k = 4.5\), which obscures a consistent set of relationships between
the remaining taxa.}
\end{figure}

\begin{figure}
\centering
\includegraphics{Brachiopod_phylogeny_files/figure-latex/tnt-iw-7-24-1.pdf}
\caption{\label{fig:tnt-iw-7-24}Strict consensus of all trees recovered by TNT
using Fitch parsimony with implied weighting, at \emph{k} = 7, 10.5, 16 and 24.}
\end{figure}

\newpage










\begin{figure}
\centering
\includegraphics{Brachiopod_phylogeny_files/figure-latex/tnt-ew-consensus-1.pdf}
\caption{\label{fig:tnt-ew-consensus}Consensus of all trees obtained using equally weighted
Fitch parsimony in TNT. \emph{Mickwitzia} and \emph{Micrina} may equally
parsimoniously be reconstructed in the basal region of the linguliform
or rhynchonelliform lineages; as such, the inclusion of these taxa in
the consensus tree reduces resolution. These taxa were still included in
the analysis used to generate this tree, but were removed from each MPT
before the consensus was calculated in order that the relationships that
are present in each tree might be more easily observed.}
\end{figure}

\hypertarget{bayesian}{\chapter{Bayesian analysis}\label{bayesian}}

Bayesian search was conducted in MrBayes v3.2.6 \citep{Ronquist2012}
using the Mk model \citep{Lewis2001} with gamma-distributed rate
variation across characters:

\begin{quote}
lset coding=variable rates=gamma;
\end{quote}

Branch length was drawn from a dirichlet prior distribution, which is
less informative than an exponential model \citep{Rannala2012}, but
requires a prior mean tree length within about two orders of magnitude
of the true value \citep{Zhang2012}. To satisfy this latter criterion,
we specified the prior mean tree length to be equal to the length of the
most parsimonious tree under equal weights, using a Dirichlet prior with
\(\alpha_T = 1\), \(\beta_T = 1/(\)\emph{equal weights tree
length}\(/\)\emph{number of characters}\()\), \(\alpha = c = 1\):

\begin{quote}
prset brlenspr = unconstrained: gammadir(1, 0.36, 1, 1);
\end{quote}

Neomorphic and transformational characters
\citep[\emph{sensu}][]{Sereno2007} were allocated to two separate
partitions whose proportion of invariant characters and gamma shape
parameters were allowed to vary independently:

\begin{quote}
charset Neomorphic = 1 6 7 9 13 16 17 18 20 21 24 25 29 30 32 33 35 38
39 40 43 44 48 49 50 51 52 53 54 55 56 60 61 63 64 66 68 69 70 71 73 75
78 79 80 82 83 85 86 87 88 91 92 93 94 95 96 97 98 101 103 104 106 113
114 116 117 120 121 124 128 129 130 131 132 134 135 136 137 140 141 142
143 145 146 147 148 149 150 152 155 160 161 163 164 165 166 167 168 169
171 176 177 178 179 180 181 182 184 186 187 190 193 196 197 198 199 200
201 203 204 205 206 208 209 210 212 213 214 215 217 220;

charset Transformational = 2 3 4 5 8 10 11 12 14 15 19 22 23 26 27 28 31
34 36 37 41 42 45 46 47 57 58 59 62 65 67 72 74 76 77 81 84 89 90 99 100
102 105 107 108 109 110 111 112 115 118 119 122 123 125 126 127 133 138
139 144 151 153 154 156 157 158 159 162 170 172 173 174 175 183 185 188
189 191 192 194 195 202 207 211 216 218 219;

partition chartype = 2: Neomorphic, Transformational;

set partition = chartype;

unlink shape=(all) pinvar=(all);
\end{quote}

Neomorphic characters were not assumed to have a symmetrical transition
rate -- that is, the probability of the absent → present transition was
allowed to differ from that of the present → absent transition, being
drawn from a uniform prior:

\begin{quote}
prset applyto=(1) symdirihyperpr=fixed(1.0);
\end{quote}

The rate of variation in neomorphic characters was also allowed to vary
from that of transformational characters:

\begin{quote}
prset applyto=(1) ratepr=variable;
\end{quote}

\emph{Flustra} was selected as an outgroup:

\begin{quote}
outgroup Flustra;
\end{quote}

Four MrBayes runs were executed, each sampling eight chains for
5~000~000 generations, with samples taken every 500 generations. The
first 10\% of samples were discarded as burn-in.

\begin{quote}
mcmcp ngen=5000000 samplefreq=500 nruns=4 nchains=8 burninfrac=0.1;
\end{quote}

A posterior tree topology was derived from the combined posterior sample
of all runs. Convergence was indicated by PSRF = 1.00 and an estimated
sample size of \textgreater{} 200 for each parameter.

\section{Parameter estimates}\label{parameter-estimates}

\section{Results}\label{results-2}

\chapter{Taxonomic implications}\label{taxonomic-implications}

This section briefly places key features of our results in the context
of previous phylogenetic hypotheses.

\begin{description}
\item[Outgroup]
We advise caution in the interpretation of outgroup relationships.
Outgroup taxa include single representatives of diverse and ancient
phyla, and are thus prone to long branch error. The relationships of the
lophotrochozoan phyla were not the primary object of this study, and
have long resisted elucidation; this said, we have attempted to
incorporate all morphological evidence that has been interpreted as
informing relationships between these groups.
\item[Brachiopod crown and stem group]
Crown- and stem-group terminology has great value in clarifying the
early evolution of major lineages \citep{Budd2000, Carlson2009}. The
crown group of a lineage is defined as the last common ancestor of all
living members of a group, and all its descendants; the stem group as
all taxa more closely related to the crown group than to any other
extant taxon. In our analyses, the brachiopod crown group corresponds to
the last common ancestor of \emph{Terebratulina} and \emph{Lingula}; the
brachiopod stem group comprises anything between this node and the
branching point of \emph{Phoronis}.
\item[Craniiforms]
Trimerellids are reconstructed as paraphyletic with respect to
Craniiforms. This is consistent with the affinity commonly drawn between
these groups \citep[e.g.][]{Williams2000LinguliformeaCraniiformea}, and
helps to account for the stratigraphically late (Ordovician) appearance
of Craniids in the fossil record. (Aragonite is underrepresented in
early Palaeozoic strata due to taphonomic bias.)

The relationship of Craniiforms with respect to Linguliforms and
Rhynchonelliforms remains unclear. Shell characters point to a
relationship with the Rhynchonelliforms, which is countered by
similarities between the spermatozoa of phoronids and terebratulids,
which indicate a craniiform + linguliform clade.

It's worth noting that Bayesian and Fitch analyses place
\emph{Gasconsia} as the basalmost member of the Rhynchonellid lineage,
upholding suggestions \citep{Holmer2014OrdovicianSilurian} of a chileid
rather than trimerellid affinity. This placement presumably represents
an artefact resulting from the incorrect handling of inapplicable data.
But if true, \emph{Gasconsia} would be a close analogue for the common
ancestor of Rhynchonelliforms + Craniiforms (+Linguliforms?).
\item[Rhynchonelliforms]
The position of kutorginids within the rhynchonelliform stem lineage has
been tricky to resolve \citep{Holmer2018Theattachment}; we resolve them
as paraphyletic with respect to Rhynconellata (which encompasses the
obolellate \emph{Alisina}), which is broadly in accord to previous
proposals \citep{Holmer2018Evolutionarysignificance}. Chileids form the
adelphiotaxon to this clade. \emph{Longtancunella}
\citep{Zhang2011Theexceptionally} nests crownwards of the protorthid
\emph{Glyptoria}, but stemward of the obolellid \emph{Alisina}.

\emph{Salanygolina} has been interpreted as a stem-group
rhynchonelliform based on its combination of paterinid and chileate
features \citep{Holmer2009Theenigmatic}. Our results position
\emph{Salanygolina} between paterinids and chileids, which directly
corroborates this proposed phylogenetic position.

Basal rhynchonellids are characterized by a circular umbonal perforation
in the ventral valve, associated with a colleplax. Partly on this basis,
the aberrant taxa \emph{Yuganotheca} and \emph{Tomteluva} plot close to
\emph{Salanygolina}, the three often forming a clade -- though the
reliability of this grouping is perhaps liable to change as additional
data comes to light. Nevertheless, an interpretation of
\emph{Yuganotheca} as a stem-group brachiopod \citep{Zhang2014Anearly}
is difficult to reconcile with the increasingly well-constrained nature
of the early brachiopod body plan.
\item[Linguliforms]
The reconstruction of Linguloformea comprising Linguloidea as sister to
Discinoidea is as expected, though it is notable that Acrotretids and
Siphonotretids plot more closely to Linguloidea than Discinoidea does.

Lingulellotretids also sit within this lingulid grouping; a position in
the phoronid stem lineage \citep[advocated
by][]{Balthasar2009EarlyCambrian} is not upheld.

More novel is the reconstruction of the calcitic obolellid
\emph{Mummpikia} in the linguliform total group: a rhynchonelliform
affinity has been assumed based on its calcitic mineralogy. This said,
\citet{Balthasar2008iMummpikia} has highlighted the similarities between
obolellids and linguliform brachiopods, including sub-μm vertical canals
and the detailed configuration of the posterior shell margin. Our
analysis upholds the case for a linguliform affinity for
\emph{Mummpikia}; a calcitic shell seemingly arose through an
independent change within this taxon As such, \emph{Mummpikia} has no
direct bearing on the origin of `Calciata', save that shell mineralogy
is perhaps less static than commonly assumed.

More generally, our results identify Class Obolellata as polyphyletic:
\emph{Alisina} (Trematobolidae) plots within Rhynchonellata;
\emph{Tomteluva} is harder to place, but tends to group with
\emph{Salanygolina} stemwards of the chileids.
\item[Paterinids]
Paterinids have traditionally been placed within the Linguliforms on the
basis of their phosphatic shell \citep{Williams2007Supplement}, which
our analysis identifies as ancestral within the brachiopod crown group;
our analysis places them within the Rhynchonelliforms instead.
Characters supporting this position include the strophic hinge line,
planar cardinal area, the absence of a pedicle nerve impression, and the
morphology of the mantle canals.

More generally, although some lingulids can be found which share more
generic characters (e.g.~shell growth direction) with paterinids, the
particular combination of characters exhibited in paterinids does not
occur anywhere in the linguliform lineage, but is more similar to that
of basal rhynchonelliforms, particularly \emph{Salanygolina}.
\item[Tommotiids]
Tommotiids represent a basal grade, paraphyletic to phoronids and
crown-group brachiopods, in line with previous interpretations.

\emph{Micrina} and \emph{Mickwitzia} are the most crownwards of the
tommotiids, but beyond this, their position is somewhat difficult to pin
down; certain analytical configruations reconstruct then as
stem-brachiopods; others place them closer to the discinids, the
lingulids or the craniiforms. \emph{Heliomedusa} is commonly associated
closely with \emph{Mickwitzia}, reflecting the similarities emphasized
by Holmer and Popov in \citet{Williams2007Supplement}, but plots instead
within the Craniiforms under certain analytical conditions, in line with
earlier interpretations \citep{Williams2000LinguliformeaCraniiformea}.
\item[Hyoliths]
Hyoliths are interpreted as stem-group Brachiopods, which refines the
broader phylogenetic position proposed by
\citet{Moysiuk2017Hyolithsare}. This is to say, they sit closer to
brachiopods than the phoronids do, but no analysis places them within
the Brachiopod crown group.

Hyoliths thus represent derived tommotiids, and are the closest
relatives to the Brachiopod crown group.
\end{description}

\bibliography{References.bib}


\end{document}
