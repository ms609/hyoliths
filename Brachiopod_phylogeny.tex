\documentclass[openany]{book}
\usepackage{lmodern}
\usepackage{amssymb,amsmath}
\usepackage{ifxetex,ifluatex}
\usepackage{fixltx2e} % provides \textsubscript
\ifnum 0\ifxetex 1\fi\ifluatex 1\fi=0 % if pdftex
  \usepackage[T1]{fontenc}
  \usepackage[utf8]{inputenc}
\else % if luatex or xelatex
  \ifxetex
    \usepackage{mathspec}
  \else
    \usepackage{fontspec}
  \fi
  \defaultfontfeatures{Ligatures=TeX,Scale=MatchLowercase}
\fi
% use upquote if available, for straight quotes in verbatim environments
\IfFileExists{upquote.sty}{\usepackage{upquote}}{}
% use microtype if available
\IfFileExists{microtype.sty}{%
\usepackage{microtype}
\UseMicrotypeSet[protrusion]{basicmath} % disable protrusion for tt fonts
}{}
\usepackage[margin=1in]{geometry}
\usepackage{hyperref}
\hypersetup{unicode=true,
            pdftitle={Supplementary Information for: Hyoliths with pedicles constrain the origin of the brachiopod body plan},
            pdfauthor={Haijing Sun, Martin R. Smith, Han Zeng, Fangchen Zhao, Guoxiang Li and Maoyan Zhu},
            pdfborder={0 0 0},
            breaklinks=true}
\urlstyle{same}  % don't use monospace font for urls
\usepackage{natbib}
\bibliographystyle{apalike-doi}
\usepackage{color}
\usepackage{fancyvrb}
\newcommand{\VerbBar}{|}
\newcommand{\VERB}{\Verb[commandchars=\\\{\}]}
\DefineVerbatimEnvironment{Highlighting}{Verbatim}{commandchars=\\\{\}}
% Add ',fontsize=\small' for more characters per line
\usepackage{framed}
\definecolor{shadecolor}{RGB}{248,248,248}
\newenvironment{Shaded}{\begin{snugshade}}{\end{snugshade}}
\newcommand{\KeywordTok}[1]{\textcolor[rgb]{0.13,0.29,0.53}{\textbf{#1}}}
\newcommand{\DataTypeTok}[1]{\textcolor[rgb]{0.13,0.29,0.53}{#1}}
\newcommand{\DecValTok}[1]{\textcolor[rgb]{0.00,0.00,0.81}{#1}}
\newcommand{\BaseNTok}[1]{\textcolor[rgb]{0.00,0.00,0.81}{#1}}
\newcommand{\FloatTok}[1]{\textcolor[rgb]{0.00,0.00,0.81}{#1}}
\newcommand{\ConstantTok}[1]{\textcolor[rgb]{0.00,0.00,0.00}{#1}}
\newcommand{\CharTok}[1]{\textcolor[rgb]{0.31,0.60,0.02}{#1}}
\newcommand{\SpecialCharTok}[1]{\textcolor[rgb]{0.00,0.00,0.00}{#1}}
\newcommand{\StringTok}[1]{\textcolor[rgb]{0.31,0.60,0.02}{#1}}
\newcommand{\VerbatimStringTok}[1]{\textcolor[rgb]{0.31,0.60,0.02}{#1}}
\newcommand{\SpecialStringTok}[1]{\textcolor[rgb]{0.31,0.60,0.02}{#1}}
\newcommand{\ImportTok}[1]{#1}
\newcommand{\CommentTok}[1]{\textcolor[rgb]{0.56,0.35,0.01}{\textit{#1}}}
\newcommand{\DocumentationTok}[1]{\textcolor[rgb]{0.56,0.35,0.01}{\textbf{\textit{#1}}}}
\newcommand{\AnnotationTok}[1]{\textcolor[rgb]{0.56,0.35,0.01}{\textbf{\textit{#1}}}}
\newcommand{\CommentVarTok}[1]{\textcolor[rgb]{0.56,0.35,0.01}{\textbf{\textit{#1}}}}
\newcommand{\OtherTok}[1]{\textcolor[rgb]{0.56,0.35,0.01}{#1}}
\newcommand{\FunctionTok}[1]{\textcolor[rgb]{0.00,0.00,0.00}{#1}}
\newcommand{\VariableTok}[1]{\textcolor[rgb]{0.00,0.00,0.00}{#1}}
\newcommand{\ControlFlowTok}[1]{\textcolor[rgb]{0.13,0.29,0.53}{\textbf{#1}}}
\newcommand{\OperatorTok}[1]{\textcolor[rgb]{0.81,0.36,0.00}{\textbf{#1}}}
\newcommand{\BuiltInTok}[1]{#1}
\newcommand{\ExtensionTok}[1]{#1}
\newcommand{\PreprocessorTok}[1]{\textcolor[rgb]{0.56,0.35,0.01}{\textit{#1}}}
\newcommand{\AttributeTok}[1]{\textcolor[rgb]{0.77,0.63,0.00}{#1}}
\newcommand{\RegionMarkerTok}[1]{#1}
\newcommand{\InformationTok}[1]{\textcolor[rgb]{0.56,0.35,0.01}{\textbf{\textit{#1}}}}
\newcommand{\WarningTok}[1]{\textcolor[rgb]{0.56,0.35,0.01}{\textbf{\textit{#1}}}}
\newcommand{\AlertTok}[1]{\textcolor[rgb]{0.94,0.16,0.16}{#1}}
\newcommand{\ErrorTok}[1]{\textcolor[rgb]{0.64,0.00,0.00}{\textbf{#1}}}
\newcommand{\NormalTok}[1]{#1}
\usepackage{longtable,booktabs}
\usepackage{graphicx,grffile}
\makeatletter
\def\maxwidth{\ifdim\Gin@nat@width>\linewidth\linewidth\else\Gin@nat@width\fi}
\def\maxheight{\ifdim\Gin@nat@height>\textheight\textheight\else\Gin@nat@height\fi}
\makeatother
% Scale images if necessary, so that they will not overflow the page
% margins by default, and it is still possible to overwrite the defaults
% using explicit options in \includegraphics[width, height, ...]{}
\setkeys{Gin}{width=\maxwidth,height=\maxheight,keepaspectratio}
\IfFileExists{parskip.sty}{%
\usepackage{parskip}
}{% else
\setlength{\parindent}{0pt}
\setlength{\parskip}{6pt plus 2pt minus 1pt}
}
\setlength{\emergencystretch}{3em}  % prevent overfull lines
\providecommand{\tightlist}{%
  \setlength{\itemsep}{0pt}\setlength{\parskip}{0pt}}
\setcounter{secnumdepth}{5}
% Redefines (sub)paragraphs to behave more like sections
\ifx\paragraph\undefined\else
\let\oldparagraph\paragraph
\renewcommand{\paragraph}[1]{\oldparagraph{#1}\mbox{}}
\fi
\ifx\subparagraph\undefined\else
\let\oldsubparagraph\subparagraph
\renewcommand{\subparagraph}[1]{\oldsubparagraph{#1}\mbox{}}
\fi

%%% Use protect on footnotes to avoid problems with footnotes in titles
\let\rmarkdownfootnote\footnote%
\def\footnote{\protect\rmarkdownfootnote}

%%% Change title format to be more compact
\usepackage{titling}

% Create subtitle command for use in maketitle
\newcommand{\subtitle}[1]{
  \posttitle{
    \begin{center}\large#1\end{center}
    }
}

\setlength{\droptitle}{-2em}
  \title{Supplementary Information for: \newline\newline Hyoliths with pedicles
constrain the origin of the brachiopod body plan}
  \pretitle{\vspace{\droptitle}\centering\huge}
  \posttitle{\par}
  \author{Haijing Sun, Martin R. Smith, Han Zeng, Fangchen Zhao, Guoxiang Li and
Maoyan Zhu}
  \preauthor{\centering\large\emph}
  \postauthor{\par}
  \predate{\centering\large\emph}
  \postdate{\par}
  \date{2018-05-30}

\usepackage{doi} % Adds hyperlinks to dois
\setcitestyle{round}
\usepackage[nottoc]{tocbibind} % list references in TOC
\raggedbottom % already in pnas-new
\usepackage[section]{placeins}

\usepackage{amsthm}
\newtheorem{theorem}{Theorem}[chapter]
\newtheorem{lemma}{Lemma}[chapter]
\newtheorem{corollary}{Corollary}[chapter]
\newtheorem{proposition}{Proposition}[chapter]
\newtheorem{conjecture}{Conjecture}[chapter]
\theoremstyle{definition}
\newtheorem{definition}{Definition}[chapter]
\theoremstyle{definition}
\newtheorem{example}{Example}[chapter]
\theoremstyle{definition}
\newtheorem{exercise}{Exercise}[chapter]
\theoremstyle{remark}
\newtheorem*{remark}{Remark}
\newtheorem*{solution}{Solution}
\begin{document}
\maketitle

{
\setcounter{tocdepth}{1}
\tableofcontents
}
\chapter*{Supplementary Text}\label{supplementary-text}
\addcontentsline{toc}{chapter}{Supplementary Text}

This document comtains supplementary material to
\citet{Sun2018Hyolithswith}. It is best viewed in HTML format at
\href{https://ms609.github.io/hyoliths/}{ms609.github.io/hyoliths}.

It opens with a detailed discussion of
\protect\hyperlink{treesearch}{analyses} of the
\protect\hyperlink{dataset}{morphological dataset} constructed to
accompany \citet{Sun2018Hyolithswith}, and their results.

The results presented in the main paper employ the algorithm described
by \citet{Brazeau2018} for correct handling of inapplicable data in a
parsimony setting. This document depicts how each character is most
parsimoniously \protect\hyperlink{reconstructions}{reconstructed} on an
optimal tree.

For completeness, we also document the results of
\protect\hyperlink{fitch}{standard Fitch parsimony} analysis, and the
results of \protect\hyperlink{bayesian}{Bayesian analysis}, neither of
which treat inapplicable data in a logically consistent fashion.

Supplementary \protect\hyperlink{figures}{figures} and
\protect\hyperlink{table}{tables} appear after the text.

\hypertarget{dataset}{\chapter{Phylogenetic dataset}\label{dataset}}

Analysis was performed on a new matrix of 45 early brachiozoan taxa,
including hyoliths, tommotiids and mickwitziids, which were coded for
196 morphological characters (127 neomorphic, 69 transformational).

\emph{Namacalathus} was incorporated as a 46\textsuperscript{th} taxon,
but preliminary results did not uphold the homology of its potentially
brachiozoan-like features. As such, we excluded it from our analysis due
to its morphological distance from ingroup taxa, a likely source of long
branch error. \emph{Dailyatia} was instead selected as an outgroup as
camenellans have been interpreted as the earliest diverging members of
the Brachiozoa \citep{Skovsted2015Theearly, Zhao2017}.

Characters are coded following the recommendations of
\citet{Brazeau2018}:

\begin{itemize}
\item
  We have employed reductive coding, using a distinct state to mark
  character inapplicability. Character specifications follow the
  structural syntax of \citet{Sereno2007} in order to highlight
  ontological dependence between characters and emphasize the structure
  of the dataset.
\item
  We have distinguished between neomorphic and transformational
  characters \citep[sensu][]{Sereno2007} by reserving the token
  \texttt{0} to refer to the absence of a neomorphic
  (i.e.~presence/absence) character. The states of transformational
  characters (i.e.~characters that describe a property of a feature) are
  represented by the tokens \texttt{1}, \texttt{2}, \texttt{3}, \ldots{}
\item
  We code the absence of neomorphic ontologically dependent characters
  \citep[sensu][]{Vogt2017} as absence, rather than inapplicability.
\end{itemize}

The complete dataset comprises 9016 character codings, of which 795 are
inapplicable and 3801 were neither ambiguous nor inapplicable. The
amount and quality of data that \emph{is} coded is more instructive than
a measure of how many cells are ambiguous \citep{Wiens1998, Wiens2003}.
Of the 196 characters, the number that were coded with an applicable
token for each taxon is:

\begin{tabular}{l|l|l|l|l|l}
\hline
 &  &  &  &  & \\
\hline
\_Cotyledion tylodes\_ & 59   \&nbsp; & \_Alisina\_ & 78   \&nbsp; & \_Longtancunella chengjiangensis\_ & 52   \&nbsp;\\
\hline
\_Loxosomella\_ & 147   \&nbsp; & \_Askepasma toddense\_ & 68   \&nbsp; & \_Micrina\_ & 61   \&nbsp;\\
\hline
\_Flustra\_ & 146   \&nbsp; & \_Antigonambonites planus\_ & 75   \&nbsp; & \_Micromitra\_ & 72   \&nbsp;\\
\hline
\_Novocrania\_ & 171   \&nbsp; & \_Botsfordia\_ & 67   \&nbsp; & \_Mickwitzia muralensis\_ & 65   \&nbsp;\\
\hline
\_Lingula\_ & 177   \&nbsp; & \_Clupeafumosus socialis\_ & 69   \&nbsp; & \_Mummpikia nuda\_ & 47   \&nbsp;\\
\hline
\_Phoronis\_ & 121   \&nbsp; & \_Coolinia pecten\_ & 71   \&nbsp; & \_Nisusia sulcata\_ & 76   \&nbsp;\\
\hline
\_Sipunculus\_ & 146   \&nbsp; & \_Craniops\_ & 59   \&nbsp; & \_Orthis\_ & 65   \&nbsp;\\
\hline
\_Serpula\_ & 149   \&nbsp; & \_Eccentrotheca\_ & 36   \&nbsp; & \_Paterimitra\_ & 57   \&nbsp;\\
\hline
\_Chiton\_ & 156   \&nbsp; & \_Eoobolus\_ & 73   \&nbsp; & \_Pedunculotheca diania\_ & 62   \&nbsp;\\
\hline
\_Dentalium\_ & 154   \&nbsp; & \_Glyptoria\_ & 68   \&nbsp; & \_Salanygolina\_ & 69   \&nbsp;\\
\hline
\_Pelagodiscus atlanticus\_ & 132   \&nbsp; & \_Gasconsia\_ & 63   \&nbsp; & \_Siphonobolus priscus\_ & 66   \&nbsp;\\
\hline
\_Terebratulina\_ & 133   \&nbsp; & \_Haplophrentis carinatus\_ & 67   \&nbsp; & \_Ussunia\_ & 44   \&nbsp;\\
\hline
\_Wiwaxia corrugata\_ & 62   \&nbsp; & \_Heliomedusa orienta\_ & 60   \&nbsp; & \_Tomteluva perturbata\_ & 53   \&nbsp;\\
\hline
\_Halkieria evangelista\_ & 55   \&nbsp; & \_Kutorgina chengjiangensis\_ & 78   \&nbsp; & \_Yuganotheca elegans\_ & 48   \&nbsp;\\
\hline
\_Dailyatia\_ & 37   \&nbsp; & \_Lingulosacculus\_ & 52   \&nbsp; &  & \\
\hline
\_Acanthotretella spinosa\_ & 62   \&nbsp; & \_Lingulellotreta malongensis\_ & 73   \&nbsp; &  & \\
\hline
\end{tabular}

The matrix can be viewed interactively and downloaded at Morphobank
(\href{https://morphobank.org/permalink/?P2800}{project 2800}). {[}This
link will become live on publication of the paper. Referees should
follow the pre-publication link to the dataset that has been provided in
the main manuscript.{]}

A static version of the NEXUS file used to generate this supplementary
information can be downloaded directly from
\url{https://raw.githubusercontent.com/ms609/hyoliths/master/mbank_X24932_5-30-2018_536.nex}
.

\hypertarget{treesearch}{\chapter{Parsimony analysis}\label{treesearch}}

The phylogenetic dataset contains a considerable proportion of
inapplicable codings (795/9016 = 8.8\% of tokens), which are known to
introduce error and bias to phylogenetic reconstruction when the Fitch
algorithm is employed \citep{Maddison1993, Brazeau2018}. As such, we
employed a new tree-scoring algorithm that correctly handles
inapplicable data \citep{Brazeau2018}, implemented in the
\emph{MorphyLib} C library \citep{Brazeau2017Morphylib}. We employed the
R package \emph{TreeSearch} v0.1.2 \citep{Smith2018TreeSearch} to
conduct phylogenetic tree search with this algorithm.

As this is a new method, we also employed the traditional, Fitch
algorithm, even though this approach is known to generate erroneous
trees. The results of this analysis can be viewed in
\protect\hyperlink{fitch}{a later section}.

\section{Search parameters}\label{search-parameters}

Heuristic searches were conducted using the parsimony ratchet
\citep{Nixon1999} under equal and implied weights \citep{Goloboff1997}.
The consensus tree presented in the main manuscript represents a strict
consensus of all trees that are most parsimonious under one or more of
the concavity constants (\emph{k}) 2, 3, 4.5, 7, 10.5, 16 and 24, an
approach that has been shown to produce higher accuracy (i.e.~more nodes
and quartets resolved correctly) than equal weights at any given level
of precision \citep{Smith2017}.

\section{Analysis}\label{analysis}

The R commands used to conduct the analysis are reproduced below. The
results can most readily be replicated using the
\href{https://github.com/ms609/hyoliths/}{R markdown files} (.Rmd) used
to generate these pages.

\subsection{Initialize and load data}\label{initialize-and-load-data}

\begin{Shaded}
\begin{Highlighting}[]
\CommentTok{# Load data from locally downloaded copy of MorphoBank matrix}
\NormalTok{my_data <-}\StringTok{ }\KeywordTok{ReadAsPhyDat}\NormalTok{(filename)}
\CommentTok{# my_data$Namacalathus <- NULL # Exclude Namacalathus}
\NormalTok{iw_data <-}\StringTok{ }\KeywordTok{PrepareDataIW}\NormalTok{(my_data)}
\end{Highlighting}
\end{Shaded}

\subsection{Generate starting tree}\label{generate-starting-tree}

Start by quickly rearranging a neighbour-joining tree, rooted on the
outgroup.

\begin{Shaded}
\begin{Highlighting}[]
\NormalTok{nj.tree <-}\StringTok{ }\KeywordTok{NJTree}\NormalTok{(my_data)}
\NormalTok{rooted.tree <-}\StringTok{ }\KeywordTok{EnforceOutgroup}\NormalTok{(nj.tree, outgroup)}
\NormalTok{start.tree <-}\StringTok{ }\KeywordTok{TreeSearch}\NormalTok{(}\DataTypeTok{tree=}\NormalTok{rooted.tree, }\DataTypeTok{dataset=}\NormalTok{my_data, }\DataTypeTok{maxIter=}\DecValTok{3000}\NormalTok{,}
                         \DataTypeTok{EdgeSwapper=}\NormalTok{RootedNNISwap, }\DataTypeTok{verbosity=}\DecValTok{0}\NormalTok{)}
\end{Highlighting}
\end{Shaded}

\subsection{Implied weights analysis}\label{implied-weights-analysis}

The position of the root does not affect tree score, so we keep it fixed
(using \texttt{RootedXXXSwap} functions) to avoid unnecessary swaps.

\begin{Shaded}
\begin{Highlighting}[]
\ControlFlowTok{for}\NormalTok{ (k }\ControlFlowTok{in}\NormalTok{ kValues) \{}
\NormalTok{  iw.tree <-}\StringTok{ }\KeywordTok{IWRatchet}\NormalTok{(start.tree, iw_data, }\DataTypeTok{concavity=}\NormalTok{k,}
                       \DataTypeTok{ratchHits =} \DecValTok{60}\NormalTok{, }\DataTypeTok{searchHits=}\DecValTok{55}\NormalTok{,}
                       \DataTypeTok{swappers=}\KeywordTok{list}\NormalTok{(RootedTBRSwap, RootedSPRSwap, RootedNNISwap),}
                       \DataTypeTok{verbosity=}\NormalTok{0L)}
\NormalTok{  score <-}\StringTok{ }\KeywordTok{IWScore}\NormalTok{(iw.tree, iw_data, }\DataTypeTok{concavity=}\NormalTok{k)}
  \CommentTok{# Write a single best tree}
  \KeywordTok{write.nexus}\NormalTok{(iw.tree,}
              \DataTypeTok{file=}\KeywordTok{paste0}\NormalTok{(}\StringTok{"TreeSearch/hy_iw_k"}\NormalTok{, k, }\StringTok{"_"}\NormalTok{, }
                          \KeywordTok{signif}\NormalTok{(score, }\DecValTok{5}\NormalTok{), }\StringTok{".nex"}\NormalTok{, }\DataTypeTok{collapse=}\StringTok{''}\NormalTok{))}

\NormalTok{  iw.consensus <-}\StringTok{ }\KeywordTok{IWRatchetConsensus}\NormalTok{(iw.tree, iw_data, }\DataTypeTok{concavity=}\NormalTok{k,}
                  \DataTypeTok{swappers=}\KeywordTok{list}\NormalTok{(RootedTBRSwap, RootedNNISwap),}
                  \DataTypeTok{searchHits=}\DecValTok{55}\NormalTok{,}
                  \DataTypeTok{nSearch=}\DecValTok{150}\NormalTok{, }\DataTypeTok{verbosity=}\NormalTok{0L)}
  \KeywordTok{write.nexus}\NormalTok{(iw.consensus, }
              \DataTypeTok{file=}\KeywordTok{paste0}\NormalTok{(}\StringTok{"TreeSearch/hy_iw_k"}\NormalTok{, k, }\StringTok{"_"}\NormalTok{, }
                          \KeywordTok{signif}\NormalTok{(}\KeywordTok{IWScore}\NormalTok{(iw.tree, iw_data, }\DataTypeTok{concavity=}\NormalTok{k), }\DecValTok{5}\NormalTok{),}
                          \StringTok{".all.nex"}\NormalTok{, }\DataTypeTok{collapse=}\StringTok{''}\NormalTok{))}
\NormalTok{\}}
\end{Highlighting}
\end{Shaded}

\subsection{Equal weights analysis}\label{equal-weights-analysis}

\begin{Shaded}
\begin{Highlighting}[]
\NormalTok{ew.tree <-}\StringTok{ }\KeywordTok{Ratchet}\NormalTok{(start.tree, my_data, }\DataTypeTok{verbosity=}\NormalTok{0L,}
                   \DataTypeTok{ratchHits =} \DecValTok{25}\NormalTok{, }\DataTypeTok{searchHits=}\DecValTok{55}\NormalTok{, }\CommentTok{# ratchHits = 10 not enough}
                   \DataTypeTok{swappers=}\KeywordTok{list}\NormalTok{(RootedTBRSwap, RootedSPRSwap, RootedNNISwap))}
\NormalTok{ew.consensus <-}\StringTok{ }\KeywordTok{RatchetConsensus}\NormalTok{(ew.tree, my_data, }\DataTypeTok{nSearch=}\DecValTok{250}\NormalTok{, }\DataTypeTok{searchHits =} \DecValTok{55}\NormalTok{,}
                                 \DataTypeTok{swappers=}\KeywordTok{list}\NormalTok{(RootedTBRSwap, RootedNNISwap),}
                                 \DataTypeTok{verbosity=}\NormalTok{0L)}
\KeywordTok{write.nexus}\NormalTok{(ew.consensus, }\DataTypeTok{file=}\KeywordTok{paste0}\NormalTok{(}\DataTypeTok{collapse=}\StringTok{''}\NormalTok{, }\StringTok{"TreeSearch/hy_ew_"}\NormalTok{,}
                                      \KeywordTok{Fitch}\NormalTok{(ew.tree, my_data), }\StringTok{".nex"}\NormalTok{))}
\end{Highlighting}
\end{Shaded}

\section{Results}\label{results}






\begin{figure}
\centering
\includegraphics{Brachiopod_phylogeny_files/figure-latex/treesearch-maj-consensus-1.pdf}
\caption{\label{fig:treesearch-maj-consensus}Consensus of all parsimony results.
Node labels denote the proportion of trees obtained
under all analytical conditions that support the clade.}
\end{figure}




\begin{figure}
\centering
\includegraphics{Brachiopod_phylogeny_files/figure-latex/treesearch-maj-consensus-pruned-1.pdf}
\caption{\label{fig:treesearch-maj-consensus-pruned}Consensus of all parsimony results.
Node labels denote the proportion of trees obtained
under all analytical conditions that support the clade.}
\end{figure}





\begin{figure}
\centering
\includegraphics{Brachiopod_phylogeny_files/figure-latex/treesearch-iw-consensus-1.pdf}
\caption{\label{fig:treesearch-iw-consensus}Consensus of implied weights analyses at all values of
\emph{k}. Wildcard taxa have been excluded from the consensus tree shown
above to improve resolution.}
\end{figure}








\clearpage 

\begin{figure}
\centering
\includegraphics{Brachiopod_phylogeny_files/figure-latex/treesearch-all-iw-results-1.pdf}
\caption{\label{fig:treesearch-all-iw-results}Consensus trees of implied weights analyses
at all values of \emph{k}, and at the individual
values \emph{k} = 2, 3 and 4.5.}
\end{figure}

\clearpage 

\begin{verbatim}
## 
##  > Results not available for panel 5
\end{verbatim}

\begin{verbatim}
## 
##  > Results not available for panel 6
\end{verbatim}

\begin{verbatim}
## 
##  > Results not available for panel 7
\end{verbatim}

\begin{figure}
\centering
\includegraphics{Brachiopod_phylogeny_files/figure-latex/treesearch-iw-results-3-1.pdf}
\caption{\label{fig:treesearch-iw-results-3}Consensus trees of implied weights analyses
at \emph{k} = 7, NA, NA and NA.}
\end{figure}

\clearpage

\begin{figure}
\centering
\includegraphics{Brachiopod_phylogeny_files/figure-latex/treesearch-equal-weights-results-1.pdf}
\caption{\label{fig:treesearch-equal-weights-results}Strict consensus of
most parsimonious trees under equally weighted parsimony}
\end{figure}

\clearpage

\hypertarget{reconstructions}{\chapter{Character
reconstructions}\label{reconstructions}}

This page provides definitions for each of the characters in our matrix,
and justifies codings in particular taxa where relevant. Further
citations for codings that are not discussed in the text can be viewed
by browsing the \protect\hyperlink{dataset}{morphological dataset} on
MorphoBank (\href{https://morphobank.org/permalink/?P2800}{project
2800}). This link will become live on publication of the paper. Referees
should follow the pre-publication link to the dataset that has been
provided in the main manuscript.

Alongside its definition, each character has been mapped onto a tree.
Here, we have arbitrarily selected one most parsimonious tree obtained
under implied weighting, \(k = 4.5\). Other trees can be viewed in the
HTML version of this document at
\href{https://ms609.github.io/hyoliths/reconstructions.html}{ms609.github.io/hyoliths}.
Each tip is labelled according to its coding in the matrix. These states
have been used to reconstruct the condition of each internal node, using
the parsimony method of \citet{Brazeau2018} as implemented in the
\emph{Inapp} \emph{R} package.

We emphasize that different trees will give different reconstructions.
The character mappings are not intended to definitively establish how
each character evolved, but to help the reader quickly establish how
each character has been coded, and to visualize at a glance how well the
character fits onto the given tree. Consistency indices
\citep{Archie1989} are also provided for each character.

\section{Sclerites}\label{sclerites}

\subsection*{{[}1{]} Present in adult}\label{present-in-adult}
\addcontentsline{toc}{subsection}{{[}1{]} Present in adult}

\includegraphics{Brachiopod_phylogeny_files/figure-latex/character-mapping-1.pdf}

\begin{quote}
\textbf{Character 1: Sclerites: Present in adult}

0: Absent\\
1: Present\\
Neomorphic character.
\end{quote}

Plate-like (wider than tall) skeletal elements, whether mineralized or
non-mineralized.\\
The definition deliberately excludes setae (which are taller than wide).

\hypertarget{Flustra-coding-1}{}
\emph{Flustra}: Gizzard elements.

\hypertarget{Halkieria_evangelista-coding-1}{}
\emph{Halkieria evangelista}: Halkieriid sclerites are interpreted as
potentially homologous with those of \emph{Dailyatia} and hence the
brachiopods \citep{Zhao2017}.

\hypertarget{Namacalathus-coding-1}{}
\emph{Namacalathus}: The mineralized endoskeleton of \emph{Namacalathus}
is not interpreted as a sclerite.

\hypertarget{Serpula-coding-1}{}
\emph{Serpula}: Annelid setae are not considered to represent potential
homologues with the brachiopod shell.

\hypertarget{Sipunculus-coding-1}{}
\emph{Sipunculus}: Hooks are present, though the absence of chitin or
microvillar impressions indicates that they are not homologous with
those of other lophotrochozoans.

\hypertarget{Wiwaxia_corrugata-coding-1}{}
\emph{Wiwaxia corrugata}: The scales of \emph{Wiwaxia} are treated as
homologous with the chaetae of annelids and brachiopods
\citep{Butterfield1990, Smith2014, Zhang2015}, rather than brachiopod
shell.

\section{Sclerites: Bivalved {[}2{]}}\label{sclerites-bivalved-2}

\includegraphics{Brachiopod_phylogeny_files/figure-latex/character-mapping-2.pdf}

\begin{quote}
\textbf{Character 2: Sclerites: Bivalved}

0: Scleritomous: without differentiated dorsal and ventral valves\\
1: Bivalved: scleritome dominated by prominent dorsal and ventral
valve\\
Neomorphic character.
\end{quote}

Scleritome dominated by prominent differentiated dorsal and ventral
valves.

\hypertarget{Chiton-coding-2}{}
\emph{Chiton}: Although \emph{Chiton} exhibits a dorsal shell-field, its
dorsal valves do not express the bivalved construction that
characterises \emph{Micrina} and crown-group brachiopods.

\subsection*{{[}3{]} Accessory sclerites
reduced}\label{accessory-sclerites-reduced}
\addcontentsline{toc}{subsection}{{[}3{]} Accessory sclerites reduced}

\includegraphics{Brachiopod_phylogeny_files/figure-latex/character-mapping-3.pdf}

\begin{quote}
\textbf{Character 3: Sclerites: Bivalved: Accessory sclerites reduced}

0: Accessory sclerites not reduced\\
1: Accessory sclerites absent: two valves only\\
Neomorphic character.
\end{quote}

Taxa in the bivalved condition may retain sclerites as small additional
elements, such as the L-elements of \emph{Paterimitra}
\citep{Skovsted2015Theearly}.

This character is treated as neomorphic, with accessory sclerites
ancestrally present, recognizing the likely origin of brachiozoans (and
Lophotrochozoans more generally) from a scleritomous organism.

\hypertarget{Haplophrentis_carinatus-coding-3}{}
\emph{Haplophrentis carinatus}: Coded as ambiguous to recognize the
possibility that helens may correspond to L-elements of
\emph{Paterimitra} \citep{Moysiuk2017Hyolithsare}.

\hypertarget{Paterimitra-coding-3}{}
\emph{Paterimitra}: L-sclerites \citep{Skovsted2009Thescleritome}.

\section{Sclerites: Accessory
sclerites}\label{sclerites-accessory-sclerites}

\subsection*{{[}4{]} Arrangement}\label{arrangement}
\addcontentsline{toc}{subsection}{{[}4{]} Arrangement}

\includegraphics{Brachiopod_phylogeny_files/figure-latex/character-mapping-4.pdf}

\begin{quote}
\textbf{Character 4: Sclerites: Accessory sclerites: Arrangement}

0: Single field\\
1: Peripheral and medial fields with distinct sclerite arrangements\\
Neomorphic character.
\end{quote}

Following \citet{Zhao2017}.

\hypertarget{Chiton-coding-4}{}
\emph{Chiton}: The eight valves are interpreted as corresponding to a
medial field; the girdle elements to a peripheral field.

\hypertarget{Dailyatia-coding-4}{}
\emph{Dailyatia}: Following the reconstruction of
\citet{Skovsted2015Theearly}.

\subsection*{{[}5{]} Symmetry}\label{symmetry}
\addcontentsline{toc}{subsection}{{[}5{]} Symmetry}

\includegraphics{Brachiopod_phylogeny_files/figure-latex/character-mapping-5.pdf}

\begin{quote}
\textbf{Character 5: Sclerites: Accessory sclerites: Symmetry}

0: Dextral and sinistral forms absent\\
1: Occuring in dextral and sinistral forms\\
Neomorphic character.
\end{quote}

Following \citet{Zhao2017}.

\hypertarget{Eccentrotheca-coding-5}{}
\emph{Eccentrotheca}: \citet{Skovsted2008Thescleritome}.

\hypertarget{Haplophrentis_carinatus-coding-5}{}
\emph{Haplophrentis carinatus}: Coded as ambiguous to recognize the
possibility that helens may correspond to L-elements of
\emph{Paterimitra} \citep{Moysiuk2017Hyolithsare}.

\section{Sclerites: Bivalved}\label{sclerites-bivalved}

\subsection*{{[}6{]} Hinge line shape}\label{hinge-line-shape}
\addcontentsline{toc}{subsection}{{[}6{]} Hinge line shape}

\includegraphics{Brachiopod_phylogeny_files/figure-latex/character-mapping-6.pdf}

\begin{quote}
\textbf{Character 6: Sclerites: Bivalved: Hinge line shape}

1: Astrophic\\
2: Strophic\\
Transformational character.
\end{quote}

\hypertarget{Botsfordia-coding-6}{}
\emph{Botsfordia}: Coded as dissociated in Williams \emph{et al}.
\citeyearpar{Williams1998Thediversity}, appendix 2.

\hypertarget{Craniops-coding-6}{}
\emph{Craniops}: Astrophic: rounded posterior margin \citep[see fig. 91
in][]{Williams2000LinguliformeaCraniiformea}.

\hypertarget{Gasconsia-coding-6}{}
\emph{Gasconsia}: The straight posterior margin of \emph{Gasconsia}
contributes to an overall resemblance with the Chileids
\citep{Holmer2014Ordovician96}.

\hypertarget{Halkieria_evangelista-coding-6}{}
\emph{Halkieria evangelista}: Inapplicable: not articulated.

\hypertarget{Kutorgina_chengjiangensis-coding-6}{}
\emph{Kutorgina chengjiangensis}: Williams \emph{et al}.
\citeyearpar[p.~208]{Williams2000LinguliformeaCraniiformea} consider the
hinge of \emph{Kutorgina} to be stropic, whereas Bassett \emph{et al}.
\citeyearpar{Bassett2001Functionalmorphology} argue for an astropic
interpretation -- whilst noting that the arrangement is prominently
different from other astrophic taxa. We therefore code this taxon as
ambiguous.

\hypertarget{Longtancunella_chengjiangensis-coding-6}{}
\emph{Longtancunella chengjiangensis}: ``\emph{Longtancunella} has an
oval to subcircular shell with a very short strophic hinge line'' --
\citet{Zhang2011Theexceptionally}.

\hypertarget{Mickwitzia_muralensis-coding-6}{}
\emph{Mickwitzia muralensis}: non-strophic.

\hypertarget{Micrina-coding-6}{}
\emph{Micrina}: See \citet{Holmer2008TheEarly}.

\hypertarget{Nisusia_sulcata-coding-6}{}
\emph{Nisusia sulcata}: ``The strophic, articulated shells of the
Kutorginata rotated on simple hinge mechanisms that are different from
those of other rhynchonelliforms''
\citep[p.~208]{Williams2000LinguliformeaCraniiformea}.

\hypertarget{Novocrania-coding-6}{}
\emph{Novocrania}: Craniides have a strophic posterior valve edge
\citep[table 39 on p.~2853]{Williams2007Supplement}: \emph{Novocrania}'s
``dorsal posterior margin'' is ``straight''
\citep[p.~171]{Williams2000LinguliformeaCraniiformea}.

\hypertarget{Tomteluva_perturbata-coding-6}{}
\emph{Tomteluva perturbata}: ``Tomteluvid taxa all have a strongly
ventribiconvex, astrophic shell with a unisulcate commissure'' --
\citet{Streng2016Anew}, p5.

\hypertarget{Yuganotheca_elegans-coding-6}{}
\emph{Yuganotheca elegans}: Not evident from fossil material; the
possibility of a short strophic hinge line (as in \emph{Longtancunella})
is difficult to discount.

\subsection*{{[}7{]} Apophyses}\label{apophyses}
\addcontentsline{toc}{subsection}{{[}7{]} Apophyses}

\includegraphics{Brachiopod_phylogeny_files/figure-latex/character-mapping-7.pdf}

\begin{quote}
\textbf{Character 7: Sclerites: Bivalved: Apophyses}

0: Absent\\
1: Present\\
Neomorphic character.
\end{quote}

Many brachiopods, in addition to \emph{Micrina} and others, bear
tooth-like structures or processes that articulate the two primary
valves.\\
Caution must be applied before taxa are coded as ``absent'', as teeth
can be subtle and may be overlooked.

Kutorginata don't have teeth or dental sockets, but their shells are
articulated by ``two triangular plates formed by dorsal interarea,
bearing oblique ridges on the inner sides''
\citep[p.~211]{Williams2000LinguliformeaCraniiformea}; this simple hinge
mechanism is different from other rhynchonelliforms
\citep[p.208]{Williams2000LinguliformeaCraniiformea}, but serves an
equivalent purpose and is thus potentially homologous. We thus code
kutorginids as present, using a subsequent character to capture
difference in tooth morphology.

\hypertarget{Alisina-coding-7}{}
\emph{Alisina}: ``Strophic articulation with paired, ventral denticles,
composed of secondary shell'' -- definition of family Trematobolidae in
\citet{Williams2000LinguliformeaCraniiformea}.

\hypertarget{Clupeafumosus_socialis-coding-7}{}
\emph{Clupeafumosus socialis}: No articulating processes evident or
reported by Topper \emph{et al}. \citeyearpar{Topper2013Reappraisalof}.

\hypertarget{Gasconsia-coding-7}{}
\emph{Gasconsia}: ``Articulatory structure comprising ventral cardinal
socket and dorsal hinge plate {[}\ldots{}{]} The shape of the shell
probably correlates strongly with the unique type of articulation, which
consists of a dorsal hinge plate that fits tightly into a cardinal
socket in the ventral valve, with a concave homeodeltidium in the center
of the ventral interarea'' --
\citet{Williams2000LinguliformeaCraniiformea}, p.184, concerning order
Trimerellida.

\hypertarget{Kutorgina_chengjiangensis-coding-7}{}
\emph{Kutorgina chengjiangensis}: ``Articulation characterized by two
triangular plates formed by dorsal interarea, bearing oblique ridges on
the inner sides'' -- \citet{Williams2000LinguliformeaCraniiformea},
p.~211.

\hypertarget{Mickwitzia_muralensis-coding-7}{}
\emph{Mickwitzia muralensis}: Not reported by or evident in Balthasar
\citeyearpar{Balthasar2004Shellstructure}.

\hypertarget{Mummpikia_nuda-coding-7}{}
\emph{Mummpikia nuda}: No articulation structures are evident; instead,
the propareas are rotated inwards \citep{Balthasar2008iMummpikia}. The
definition of Family Obolellidae in Williams \emph{et al}.
\citeyearpar{Williams2000LinguliformeaCraniiformea} notes that
articulation may be lacking or vestigial in the group.

\hypertarget{Nisusia_sulcata-coding-7}{}
\emph{Nisusia sulcata}: Pseudodont articulation: teeth formed by distal
lateral extensions from the ventral pseudodeltidium --
\citet{Holmer2018Evolutionarysignificance}.

\hypertarget{Tomteluva_perturbata-coding-7}{}
\emph{Tomteluva perturbata}: Tomteluvids {[}\ldots{}{]} lack
articulation structures such as teeth and sockets
\citep{Streng2016Anew}.

\hypertarget{Ussunia-coding-7}{}
\emph{Ussunia}: ``articulatory structures poorly developed'' --
\citet{Williams2000LinguliformeaCraniiformea}, p.~192.

\subsection*{{[}8{]} Apophyses: Morphology}\label{apophyses-morphology}
\addcontentsline{toc}{subsection}{{[}8{]} Apophyses: Morphology}

\includegraphics{Brachiopod_phylogeny_files/figure-latex/character-mapping-8.pdf}

\begin{quote}
\textbf{Character 8: Sclerites: Bivalved: Apophyses: Morphology}

1: Deltidiodont\\
2: Cyrtomatodont\\
3: Pseudodont\\
Transformational character.
\end{quote}

Deltidiodont teeth are simple hinge teeth developed by the distal
accretion of secondary shell; Cyrtomatodont teeth are knoblike or
hook-shaped hinge teeth developed by differential secretion and
resorption of the secondary shell \citep[fig. 322
in][]{Williams2000LinguliformeaCraniiformea}.

Kutorginata (here represented by \emph{Kutorgina} and \emph{Nisusia})
don't have teeth (apophyses) or dental sockets, but their shells are
articulated by ``two triangular plates formed by dorsal interarea,
bearing oblique ridges on the inner sides''
\citep[p.~211]{Williams2000LinguliformeaCraniiformea}; this simple hinge
mechanism is different from other rhynchonelliforms
{[}\citet{Williams2000LinguliformeaCraniiformea}, p.208; table 13
character 30{]}, and is described as a ``pseudodont articulation''
\citep{Holmer2018Evolutionarysignificance}.

\hypertarget{Antigonambonites_planus-coding-8}{}
\emph{Antigonambonites planus}, \emph{Glyptoria}: Coded as deltidiodont
in Benedetto \citeyearpar{Benedetto2009iChaniella}.

\hypertarget{Kutorgina_chengjiangensis-coding-8}{}
\emph{Kutorgina chengjiangensis}: ``Articulation characterized by two
triangular plates formed by dorsal interarea, bearing oblique ridges on
the inner sides'' -- \citet{Williams2000LinguliformeaCraniiformea},
p.~211.

\hypertarget{Micrina-coding-8}{}
\emph{Micrina}: The simple knob-like teeth of \emph{Micrina} show no
evidence of resprobtion or the hook-like shape that characterises
Cyrtomatodont teeth.

\hypertarget{Nisusia_sulcata-coding-8}{}
\emph{Nisusia sulcata}: The `teeth' are formed by the distal lateral
extensions from the ventral\\
pseudodeltidium fitting into the `sockets' on the inner side of the
dorsal interarea \citep{Holmer2018Evolutionarysignificance}. {[}Coded as
``deltidiodont teeth absent'' in Benedetto
\citeyearpar{Benedetto2009iChaniella}.{]}.

\hypertarget{Orthis-coding-8}{}
\emph{Orthis}: Coded as deltidiodont (in \emph{Eoorthis}) in Benedetto
\citeyearpar{Benedetto2009iChaniella}.

\hypertarget{Terebratulina-coding-8}{}
\emph{Terebratulina}: Cyrtomatodont -- see fig. 322 in Williams \emph{et
al}. \citeyearpar{Williams2000LinguliformeaCraniiformea}.

\subsection*{{[}9{]} Apophyses: Dental
plates}\label{apophyses-dental-plates}
\addcontentsline{toc}{subsection}{{[}9{]} Apophyses: Dental plates}

\includegraphics{Brachiopod_phylogeny_files/figure-latex/character-mapping-9.pdf}

\begin{quote}
\textbf{Character 9: Sclerites: Bivalved: Apophyses: Dental plates}

0: Absent\\
1: Present\\
Neomorphic character.
\end{quote}

\citet{Williams1997Introduction} (p362) write: ``Teeth {[}\ldots{}{]}
are commonly supported by a pair of variably disposed plates also built
up exclusively of secondary shell and known as dental plates (Fig.
323.1, 323.3).''

Dewing \citeyearpar{Dewing2001Hingemodifications} elaborates: ``Dental
plates are near-vertical, narrow sheets of shell tissue between the
anteromedian edge of the teeth and floor of the ventral valve. They are
a composite structure, resulting from the growth of teeth over the ridge
that bounds the ventral-valve muscle field.''

\citet{Williams2000LinguliformeaCraniiformea} (p.201) write: ``The
denticles lack supporting structures in all Obolellida, but in Naukatida
they are supported by an arcuate plate below the\\
interarea, the anterise (Fig. 119.3a)''.

The anterise is conceivably homologous with the dental plates, thus the
presence of either is coded ``present'' for this character.

\hypertarget{Antigonambonites_planus-coding-9}{}
\emph{Antigonambonites planus}: Coded as present (well developed) in
Benedetto \citeyearpar{Benedetto2009iChaniella}.

\hypertarget{Coolinia_pecten-coding-9}{}
\emph{Coolinia pecten}: Coded as present following Dewing
\citeyearpar{Dewing2001Hingemodifications}, who seems to use the term
Strophomenoids to encompass \emph{Coolinia}, and attests to the presence
of dental plates.

\hypertarget{Gasconsia-coding-9}{}
\emph{Gasconsia}: Coded ambiguous to reflect the possibility that the
hinge plate in trimerellids is homologous to the dental plates of other
taxa, and has replaced the teeth themselves as the primary articulatory
mechanism \citep[see][p.~184, for details of the
articulation]{Williams2000LinguliformeaCraniiformea}.

\hypertarget{Glyptoria-coding-9}{}
\emph{Glyptoria}, \emph{Nisusia sulcata}: Coded as absent in Benedetto
\citeyearpar{Benedetto2009iChaniella}.

\hypertarget{Orthis-coding-9}{}
\emph{Orthis}: Coded as present (short and recessive, in
\emph{Eoorthis}) in Benedetto \citeyearpar{Benedetto2009iChaniella}.

\subsection*{{[}10{]} Sockets}\label{sockets}
\addcontentsline{toc}{subsection}{{[}10{]} Sockets}

\includegraphics{Brachiopod_phylogeny_files/figure-latex/character-mapping-10.pdf}

\begin{quote}
\textbf{Character 10: Sclerites: Bivalved: Sockets}

0: Absent\\
1: Present\\
Neomorphic character.
\end{quote}

Simplified from Bassett \emph{et al}.
\citeyearpar{Bassett2001Functionalmorphology} character 16.\\
This character is independent of apophyses, as several taxa bear sockets
without corresponding teeth; the function of these sockets is unknown.\\
See figs 323ff in Williams \emph{et al}.
\citeyearpar{Williams1997Introduction}.

\hypertarget{Alisina-coding-10}{}
\emph{Alisina}: ``bearing sockets, bounded by low ridges'' --
\citet{Williams2000LinguliformeaCraniiformea}.

\hypertarget{Antigonambonites_planus-coding-10}{}
\emph{Antigonambonites planus}: Coded as present in Benedetto
\citeyearpar{Benedetto2009iChaniella}.

\hypertarget{Gasconsia-coding-10}{}
\emph{Gasconsia}: ``Articulatory structure comprising ventral cardinal
socket and dorsal hinge plate'' --
\citet{Williams2000LinguliformeaCraniiformea}, p.~184.

\hypertarget{Glyptoria-coding-10}{}
\emph{Glyptoria}, \emph{Nisusia sulcata}: Coded as absent in Benedetto
\citeyearpar{Benedetto2009iChaniella}.

\hypertarget{Mickwitzia_muralensis-coding-10}{}
\emph{Mickwitzia muralensis}: Not reported by or evident in Balthasar
\citeyearpar{Balthasar2004Shellstructure}.

\hypertarget{Tomteluva_perturbata-coding-10}{}
\emph{Tomteluva perturbata}: Tomteluvids {[}\ldots{}{]} lack
articulation structures such as teeth and sockets
\citep{Streng2016Anew}.

\hypertarget{Ussunia-coding-10}{}
\emph{Ussunia}: Following table 15 in
\citet{Williams2000LinguliformeaCraniiformea}.

\subsection*{{[}11{]} Socket ridges}\label{socket-ridges}
\addcontentsline{toc}{subsection}{{[}11{]} Socket ridges}

\includegraphics{Brachiopod_phylogeny_files/figure-latex/character-mapping-11.pdf}

\begin{quote}
\textbf{Character 11: Sclerites: Bivalved: Socket ridges}

0: Absent\\
1: Present\\
Neomorphic character.
\end{quote}

After Bassett \emph{et al}.
\citeyearpar{Bassett2001Functionalmorphology} character 17. May be
difficult to distinguish from a brachiophore \citep[see Fig 323
in][]{Williams1997Introduction}, so the two structures are not
distinguished here.

\hypertarget{Alisina-coding-11}{}
\emph{Alisina}: ``bearing sockets, bounded by low ridges'' --
\citet{Williams2000LinguliformeaCraniiformea}.

\hypertarget{Antigonambonites_planus-coding-11}{}
\emph{Antigonambonites planus}: Coded as present in Benedetto
\citeyearpar{Benedetto2009iChaniella}.

\hypertarget{Glyptoria-coding-11}{}
\emph{Glyptoria}, \emph{Nisusia sulcata}: Coded as absent in Benedetto
\citeyearpar{Benedetto2009iChaniella}.

\hypertarget{Tomteluva_perturbata-coding-11}{}
\emph{Tomteluva perturbata}: Tomteluvids {[}\ldots{}{]} lack
articulation structures such as teeth and sockets
\citep{Streng2016Anew}.

\subsection*{{[}12{]} Enclosing filtration
chamber}\label{enclosing-filtration-chamber}
\addcontentsline{toc}{subsection}{{[}12{]} Enclosing filtration chamber}

\includegraphics{Brachiopod_phylogeny_files/figure-latex/character-mapping-12.pdf}

\begin{quote}
\textbf{Character 12: Sclerites: Bivalved: Enclosing filtration chamber}

0: No filtration chamber, or open chamber\\
1: Shells close to form enclosed filtration chamber\\
Neomorphic character.
\end{quote}

In crown-group brachiopods, the two primary shells close to form an
enclosed filtration chamber. Further down the stem, taxa such as
\emph{Micrina} do not.

\subsection*{{[}13{]} Commissure}\label{commissure}
\addcontentsline{toc}{subsection}{{[}13{]} Commissure}

\includegraphics{Brachiopod_phylogeny_files/figure-latex/character-mapping-13.pdf}

\begin{quote}
\textbf{Character 13: Sclerites: Bivalved: Commissure}

1: Rectimarginate\\
2: Uniplicate\\
3: Sulcate\\
Transformational character.
\end{quote}

The anterior commissure can be rectimarginate (i.e.~straight),
uniplicate (i.e.~median sulcus in ventral valve), or sulcate (with
median sulcus in dorsal valve).

\hypertarget{Askepasma_toddense-coding-13}{}
\emph{Askepasma toddense}: ``ventral valve weakly to moderately
sulcate'' \citep{Topper2013Theoldest}; a similar description is provided
by Williams \emph{et al}.
\citeyearpar{Williams2000LinguliformeaCraniiformea}.

\hypertarget{Glyptoria-coding-13}{}
\emph{Glyptoria}, \emph{Kutorgina chengjiangensis}, \emph{Micromitra},
\emph{Salanygolina}: Following appendix 2 in Williams \emph{et al}.
\citeyearpar{Williams1998Thediversity}.

\hypertarget{Terebratulina-coding-13}{}
\emph{Terebratulina}: ``Anterior commissure rectimarginate to
uniplicate'' -- uniplicate in fig. 1425.1c of Williams \emph{et al}.
\citeyearpar{Williams2006Rhynchonelliformeapart}.

\subsection*{{[}14{]} Muscle scars: Ventral}\label{muscle-scars-ventral}
\addcontentsline{toc}{subsection}{{[}14{]} Muscle scars: Ventral}

\includegraphics{Brachiopod_phylogeny_files/figure-latex/character-mapping-14.pdf}

\begin{quote}
\textbf{Character 14: Sclerites: Bivalved: Muscle scars: Ventral }

0: Absent\\
1: Present\\
Neomorphic character.
\end{quote}

After Bassett \emph{et al}.
\citeyearpar{Bassett2001Functionalmorphology} character 6.

\hypertarget{Alisina-coding-14}{}
\emph{Alisina}: Muscle scars scored based on \emph{Alisina}
\emph{comleyensis} \citep{Bassett2001Functionalmorphology}.

\hypertarget{Halkieria_evangelista-coding-14}{}
\emph{Halkieria evangelista}: Muscle scars are known from the Type A,
but not Type B, morphs of the halkieriid \emph{Oikozetetes}
\citep{Paterson2009, Jacquet2014}.

\hypertarget{Mickwitzia_muralensis-coding-14}{}
\emph{Mickwitzia muralensis}: Scars absent; instead, cones ornament
shell's internal surface.

\hypertarget{Micrina-coding-14}{}
\emph{Micrina}: Prominent ventral muscle scars -- see e.g.
\citet{Holmer2008TheEarly}, fig. 1f.

\subsection*{{[}15{]} Muscle scars: Ventral:
Position}\label{muscle-scars-ventral-position}
\addcontentsline{toc}{subsection}{{[}15{]} Muscle scars: Ventral:
Position}

\includegraphics{Brachiopod_phylogeny_files/figure-latex/character-mapping-15.pdf}

\begin{quote}
\textbf{Character 15: Sclerites: Bivalved: Muscle scars: Ventral:
Position}

1: Posterolateral and medial attachments\\
2: Medial attachments only\\
Transformational character.
\end{quote}

Muscles can attach to the ventral valve posterolaterally to, as well as
between, the \emph{vascula} \emph{lateralia}
\citep{Popov1992TheCambrian}.

\hypertarget{Acanthotretella_spinosa-coding-15}{}
\emph{Acanthotretella spinosa}: ``Individual muscle scars cannot be
distinguished'' -- \citet{Holmer2006Aspinose}.

\hypertarget{Alisina-coding-15}{}
\emph{Alisina}: Following reconstruction of Gorjansky \& Popov
\citeyearpar{Gorjansky1986Onthe}.

\hypertarget{Askepasma_toddense-coding-15}{}
\emph{Askepasma toddense}: Restricted to medial field, following the
interpretation of the musculature presented by Williams \emph{et al}.
\citeyearpar{Williams2000LinguliformeaCraniiformea}, fig. 81.

\hypertarget{Clupeafumosus_socialis-coding-15}{}
\emph{Clupeafumosus socialis}: Coded following \emph{Hadrotreta}, as
illustrated in Popov \citeyearpar{Popov1992TheCambrian}.

\hypertarget{Craniops-coding-15}{}
\emph{Craniops}: See fig. 89 in Williams \emph{et al}.
\citeyearpar{Williams2000LinguliformeaCraniiformea}.

\hypertarget{Eoobolus-coding-15}{}
\emph{Eoobolus}: The `laterals' of Balthasar \citeyearpar[fig.
5]{Balthasar2009Thebrachiopod} are situated almost upon the
\emph{vascula} \emph{lateralia}; they are interpreted as sitting
posterolateral to them.

\hypertarget{Gasconsia-coding-15}{}
\emph{Gasconsia}: Musculature described in Hanken \& Harper
\citeyearpar{Hanken1985Thetaxonomy}.

\hypertarget{Glyptoria-coding-15}{}
\emph{Glyptoria}: Posterolateral reflected by diductor attachments; see
fig. 18.3.2 in \citet{Bassett2001Functionalmorphology}.

\hypertarget{Kutorgina_chengjiangensis-coding-15}{}
\emph{Kutorgina chengjiangensis}: Following situation in \emph{Nisusia};
see fig. 18.2 in Bassett \emph{et al}.
\citeyearpar{Bassett2001Functionalmorphology}.

\hypertarget{Lingulellotreta_malongensis-coding-15}{}
\emph{Lingulellotreta malongensis}: See fig. 5 in
\citet{Holmer1997EarlyCambrian}.

\hypertarget{Micromitra-coding-15}{}
\emph{Micromitra}: Posteriomedial muscle field \citep[text-fig.
6]{Williams1998Thediversity} treated as equivalent to posterolateral
muscles.

\hypertarget{Nisusia_sulcata-coding-15}{}
\emph{Nisusia sulcata}: Posterolateral diductors \citep[fig. 18.2
in][]{Bassett2001Functionalmorphology}.

\hypertarget{Novocrania-coding-15}{}
\emph{Novocrania}: Posterior adductor muscles attach posterolaterally to
ventral mantle canal \citep{Robinson2014Themuscles}.

\hypertarget{Orthis-coding-15}{}
\emph{Orthis}: Not applicable: \emph{vascula} \emph{lateralia} not
comparable to those of other taxa.

\hypertarget{Pelagodiscus_atlanticus-coding-15}{}
\emph{Pelagodiscus atlanticus}: Inapplicable as vascular system not
directly equivalent to the canonical; see. fig 6b in Balthasar
\citeyearpar{Balthasar2009Thebrachiopod}.

\hypertarget{Salanygolina-coding-15}{}
\emph{Salanygolina}: Ventral musculature not clearly constrained
\citep{Holmer2009Theenigmatic}.

\hypertarget{Siphonobolus_priscus-coding-15}{}
\emph{Siphonobolus priscus}: Coded following general siphonotretid
condition described by Popov \citeyearpar[p.~407]{Popov1992TheCambrian}.

\subsection*{{[}16{]} Muscle scars:
Adjustor}\label{muscle-scars-adjustor}
\addcontentsline{toc}{subsection}{{[}16{]} Muscle scars: Adjustor}

\includegraphics{Brachiopod_phylogeny_files/figure-latex/character-mapping-16.pdf}

\begin{quote}
\textbf{Character 16: Sclerites: Bivalved: Muscle scars: Adjustor}

0: Absent\\
1: Present\\
Neomorphic character.
\end{quote}

After Bassett \emph{et al}.
\citeyearpar{Bassett2001Functionalmorphology} character 7.\\
This character is contingent on the presence of a pedicle. Extreme
caution must be used in inferring an absent state, as adjustor scars can
be extremely difficult to distinguish from the adductor scars.

\hypertarget{Alisina-coding-16}{}
\emph{Alisina}: Muscle scars scored based on \emph{Alisina}
\emph{comleyensis} \citep{Bassett2001Functionalmorphology}. The presence
of an adjustor is marked as not presently available, as it is not clear
that a scar, if present, could be distinguished from the diminutive
muscle scars present.

\hypertarget{Askepasma_toddense-coding-16}{}
\emph{Askepasma toddense}: Following the interpretation of the
musculature presented by Williams \emph{et al}.
\citeyearpar{Williams2000LinguliformeaCraniiformea}, fig. 81.

\hypertarget{Botsfordia-coding-16}{}
\emph{Botsfordia}: Not described in Popov
\citeyearpar{Popov1992TheCambrian}.

\hypertarget{Clupeafumosus_socialis-coding-16}{}
\emph{Clupeafumosus socialis}: Not known in any acrotretid
\citep{Williams2000LinguliformeaCraniiformea}; not evident in
\emph{Clupeafumosus} \citep{Topper2013Reappraisalof}.

\hypertarget{Gasconsia-coding-16}{}
\emph{Gasconsia}: No mention of an adjustor muscle in \emph{Gasconsia}
or Trimerellida more generally on pp.~184--185 of
\citet{Williams2000LinguliformeaCraniiformea}, nor in discussion in
\citet{Williams2007Supplement} (p.~2850). Coded as absent.

\hypertarget{Mickwitzia_muralensis-coding-16}{}
\emph{Mickwitzia muralensis}: Scars absent; instead, cones ornament
shell's internal surface.

\hypertarget{Siphonobolus_priscus-coding-16}{}
\emph{Siphonobolus priscus}: Ventral musculature poorly constrained
\citep{Williams2000LinguliformeaCraniiformea, Popov2009Earlyontogeny}.

\subsection*{{[}17{]} Muscle scars: Dorsal
adductors}\label{muscle-scars-dorsal-adductors}
\addcontentsline{toc}{subsection}{{[}17{]} Muscle scars: Dorsal
adductors}

\includegraphics{Brachiopod_phylogeny_files/figure-latex/character-mapping-17.pdf}

\begin{quote}
\textbf{Character 17: Sclerites: Bivalved: Muscle scars: Dorsal
adductors}

1: Dispersed\\
2: Radially arranged\\
3: Quadripartite\\
Transformational character.
\end{quote}

After Bassett \emph{et al}.
\citeyearpar{Bassett2001Functionalmorphology} character 8, and Williams
\emph{et al}. {[}\citet{Williams1996Asupra}, character 35; 2000, p.~160,
character 54{]}

In the dorsal valve, the anterior and posterior adductor scars of
articulated brachiopods form a single (quadripartite) muscle field
\citep[p.~201]{Williams2000LinguliformeaCraniiformea}

In contrast, the anterior and posterior scars of e.g.~trimerellids have
prominently separate attachment points, with anterior and posterior
muscle fields clearly distinct, and coded as ``dispersed''.

In e.g.~kutorginates, adductor muscles are separated into left and right
fields; the same is the case in lingulids, where there are more separate
muscle groups and the left and right fields conspire to produce a radial
arrangement; both of these configurations are scored as ``radially
arranged''.

\hypertarget{Alisina-coding-17}{}
\emph{Alisina}: Following Williams \emph{et al}.
\citeyearpar{Williams2000LinguliformeaCraniiformea} table 15 (their
character 54).

\hypertarget{Antigonambonites_planus-coding-17}{}
\emph{Antigonambonites planus}: Treatise.

\hypertarget{Askepasma_toddense-coding-17}{}
\emph{Askepasma toddense}: Separate left and right fields, so radially
arranged -- following the interpretation of the musculature presented by
Williams \emph{et al}.
\citeyearpar{Williams2000LinguliformeaCraniiformea}, fig. 81.

\hypertarget{Botsfordia-coding-17}{}
\emph{Botsfordia}: Following \citet{Williams1998Thediversity}, appendix
2.

\hypertarget{Clupeafumosus_socialis-coding-17}{}
\emph{Clupeafumosus socialis}: Following reconstruction of
\emph{Hadrotreta} by Williams
\citeyearpar{Williams2000LinguliformeaCraniiformea}, fig. 51, which
exhibits distinct left and right fields.

\hypertarget{Coolinia_pecten-coding-17}{}
\emph{Coolinia pecten}: ``radially arranged adductor scars'' --
\citet{Bassett2017Earliestontogeny}, p1.

\hypertarget{Gasconsia-coding-17}{}
\emph{Gasconsia}: Following the coding of Williams \emph{et al}.
\citeyearpar{Williams2000LinguliformeaCraniiformea}, table 15.

\hypertarget{Glyptoria-coding-17}{}
\emph{Glyptoria}: Scored as ``dispersed'' by Williams \emph{et al}.
\citeyearpar{Williams1998Thediversity} \ldots{} but then so is
\emph{Kutorgina}, which Bassett \emph{et al}.
\citeyearpar{Bassett2001Functionalmorphology} score as radial.

Williams \emph{et al}.
\citeyearpar{Williams2000LinguliformeaCraniiformea} state, for
superfamily Protorthida, ``dorsal adductor scars probably linear'',
which fits in the category of ``radial'' employed herein -- so that's
what we follow.

\hypertarget{Halkieria_evangelista-coding-17}{}
\emph{Halkieria evangelista}: It is unclear whether the paired muscle
scars of \emph{Oikozetetes} may be homologous to brachiopod adductors.

\hypertarget{Haplophrentis_carinatus-coding-17}{}
\emph{Haplophrentis carinatus}: Laterally dispersed, based on
interpretation of Moysiuk \emph{et al}.
\citeyearpar{Moysiuk2017Hyolithsare}, and consistent with general
situation in hyoliths \citep[see][]{Dzik1980Ontogenyof}.

\hypertarget{Heliomedusa_orienta-coding-17}{}
\emph{Heliomedusa orienta}: Distinct anterior and posterior fields
\citep{Chen2007Reinterpretationof}; coded as ``dispersed'' by Williams
\emph{et al}. \citeyearpar{Williams2000LinguliformeaCraniiformea} in
table 15.

\hypertarget{Mickwitzia_muralensis-coding-17}{}
\emph{Mickwitzia muralensis}: Scars absent; instead, cones ornament
shell's internal surface.

\hypertarget{Micromitra-coding-17}{}
\emph{Micromitra}: Williams \emph{et al}.
\citeyearpar{Williams1998Thediversity} code as ``dispersed'', but have a
less divided scheme of character states and disagree with other sources
in some codings \citep[e.g.][in
Kutorginates]{Bassett2001Functionalmorphology}. Williams \emph{et al}.
\citeyearpar{Williams2000LinguliformeaCraniiformea} do not describe
\emph{Micromitra} musculature and we were unable to find any reliable
description of the scars, so we code as ``not presently available''.

\hypertarget{Novocrania-coding-17}{}
\emph{Novocrania}: Craniids scored as ``open, quadripartite'' by
Williams \emph{et al}. \citeyearpar{Williams1996Asupra}.

\hypertarget{Pelagodiscus_atlanticus-coding-17}{}
\emph{Pelagodiscus atlanticus}: Discinids scored as ``open,
quadripartite'' by Williams \emph{et al}.
\citeyearpar{Williams1996Asupra}.

\hypertarget{Salanygolina-coding-17}{}
\emph{Salanygolina}: ``The dorsal valve of \emph{Salanygolina} has a
radial arrangement of adductor muscle scars and the scars of
posteromedially placed internal oblique muscles, which are also
characteristic of paterinates and chileates'' -- Holmer \emph{et al}.
\citeyearpar{Holmer2009Theenigmatic}.

\hypertarget{Siphonobolus_priscus-coding-17}{}
\emph{Siphonobolus priscus}: Ventral musculature poorly constrained
\citep{Williams2000LinguliformeaCraniiformea, Popov2009Earlyontogeny}.

\hypertarget{Terebratulina-coding-17}{}
\emph{Terebratulina}: Coded as ``grouped, quadripartite'' by Williams
\emph{et al}. \citeyearpar{Williams1996Asupra}.

\hypertarget{Ussunia-coding-17}{}
\emph{Ussunia}: Following table 15 in
\citet{Williams2000LinguliformeaCraniiformea}.

\subsection*{{[}18{]} Muscle scars: Adductors:
Position}\label{muscle-scars-adductors-position}
\addcontentsline{toc}{subsection}{{[}18{]} Muscle scars: Adductors:
Position}

\includegraphics{Brachiopod_phylogeny_files/figure-latex/character-mapping-18.pdf}

\begin{quote}
\textbf{Character 18: Sclerites: Bivalved: Muscle scars: Adductors:
Position}

1: Oblique\\
2: At high angle\\
Transformational character.
\end{quote}

Position of adductor muscles relative to commissural plane.\\
After Bassett \emph{et al}.
\citeyearpar{Bassett2001Functionalmorphology} character 11.

\hypertarget{Askepasma_toddense-coding-18}{}
\emph{Askepasma toddense}: Following the interpretation of the
musculature presented by Williams \emph{et al}.
\citeyearpar{Williams2000LinguliformeaCraniiformea}, fig. 81.

\hypertarget{Botsfordia-coding-18}{}
\emph{Botsfordia}: Following description of Popov
\citeyearpar{Popov1992TheCambrian}.

\hypertarget{Coolinia_pecten-coding-18}{}
\emph{Coolinia pecten}: Not reported by Williams \emph{et al}.
\citeyearpar{Williams2000LinguliformeaCraniiformea}, nor Bassett \&
Popov \citeyearpar{Bassett2017Earliestontogeny}, nor explicitly by
Dewing \citeyearpar{Dewing2001Hingemodifications}.

\hypertarget{Eoobolus-coding-18}{}
\emph{Eoobolus}: ``\emph{Eoobolus} should have anterior and posterior
adductors and a variety of oblique muscles which were probably arranged
in criss-crossing pairs'' -- \citet{Balthasar2009Thebrachiopod}.

\hypertarget{Gasconsia-coding-18}{}
\emph{Gasconsia}: See discussion under Trimerellida in Williams \emph{et
al}. \citeyearpar{Williams2000LinguliformeaCraniiformea}.

\hypertarget{Mickwitzia_muralensis-coding-18}{}
\emph{Mickwitzia muralensis}: Scars absent; instead, cones ornament
shell's internal surface.

\hypertarget{Pelagodiscus_atlanticus-coding-18}{}
\emph{Pelagodiscus atlanticus}: Musculature considered essentially
equivalent to \emph{Lingula} by
\citet{Williams2000LinguliformeaCraniiformea}, so \emph{Lingula} coding
followed here.

\hypertarget{Siphonobolus_priscus-coding-18}{}
\emph{Siphonobolus priscus}: Ventral musculature poorly constrained
\citep{Williams2000LinguliformeaCraniiformea, Popov2009Earlyontogeny}.

\subsection*{{[}19{]} Muscle scars: Dermal
muscles}\label{muscle-scars-dermal-muscles}
\addcontentsline{toc}{subsection}{{[}19{]} Muscle scars: Dermal muscles}

\includegraphics{Brachiopod_phylogeny_files/figure-latex/character-mapping-19.pdf}

\begin{quote}
\textbf{Character 19: Sclerites: Bivalved: Muscle scars: Dermal muscles}

0: Absent or weakly developed\\
1: Strongly developed\\
Neomorphic character.
\end{quote}

Based on character 11 in Zhang \emph{et al}.
\citeyearpar{Zhang2014Anearly}.\\
Well developed dermal muscles present in the body wall of recent
lingulates, which are absent in all calcareous-shelled brachiopods.
These muscles are responsible for the hydraulic shell-opening mechanism,
and possibly present in all organophosphatic-shelled brachiopods, with
the possible exception of the paterinates
\citep[p.~32]{Williams2000LinguliformeaCraniiformea}.

\hypertarget{Alisina-coding-19}{}
\emph{Alisina}, \emph{Antigonambonites planus}, \emph{Gasconsia},
\emph{Glyptoria}, \emph{Nisusia sulcata}, \emph{Orthis},
\emph{Salanygolina}: According to the statement of Williams \emph{et
al}. \citeyearpar[p.~32]{Williams2000LinguliformeaCraniiformea} that
these muscle are absent in all carbonate- shelled brachiopods.

\hypertarget{Askepasma_toddense-coding-19}{}
\emph{Askepasma toddense}: According to the statement of Williams
\emph{et al}. \citeyearpar[p.~32]{Williams2000LinguliformeaCraniiformea}
that the presence of these muscles in paterinates is uncertain.

\hypertarget{Botsfordia-coding-19}{}
\emph{Botsfordia}: Implicitly taken as present in Popov
\citeyearpar{Popov1992TheCambrian}, though not marked in diagrams --
suggesting not strongly developed.

\hypertarget{Clupeafumosus_socialis-coding-19}{}
\emph{Clupeafumosus socialis}: This character is coded based on the
score of Acrotreta in Zhang \emph{et al}.
\citeyearpar{Zhang2014Anearly}, and statement in Williams \emph{et al}.
\citeyearpar[P.32]{Williams2000LinguliformeaCraniiformea}.

\hypertarget{Coolinia_pecten-coding-19}{}
\emph{Coolinia pecten}: According to the statement of Williams \emph{et
al}. \citeyearpar[p.~32]{Williams2000LinguliformeaCraniiformea} that
these muscle are absent in all carbonate-shelled brachiopods.

\hypertarget{Eoobolus-coding-19}{}
\emph{Eoobolus}: Not remarked upon by Balthasar
\citeyearpar{Balthasar2009Thebrachiopod}.

\hypertarget{Kutorgina_chengjiangensis-coding-19}{}
\emph{Kutorgina chengjiangensis}: According to the statement of Williams
\emph{et al}. \citeyearpar[p.~32]{Williams2000LinguliformeaCraniiformea}
that these muscle are absent in all carbonate- shelled brachiopods, and
the coding for kutorginids in Zhang \emph{et al}.
\citeyearpar{Zhang2014Anearly}.

\hypertarget{Micromitra-coding-19}{}
\emph{Micromitra}: Williams \emph{et al}.
\citeyearpar[p.~32]{Williams2000LinguliformeaCraniiformea} are uncertain
about the presence of these muscles in the paterinates. Zhang \emph{et
al}. \citeyearpar{Zhang2014Anearly} code absence in Paterinida, but
without specifying evidence; we follow their coding here.

\hypertarget{Mummpikia_nuda-coding-19}{}
\emph{Mummpikia nuda}, \emph{Tomteluva perturbata}: Though Williams
\emph{et al}. \citeyearpar[p.~32]{Williams2000LinguliformeaCraniiformea}
state that these muscles are absent in all carbonate-shelled
brachiopods, their existence cannot be discounted with certainty in this
taxon, which is therefore coded not presently available.

\hypertarget{Novocrania-coding-19}{}
\emph{Novocrania}: Following Zhang \emph{et al}.
\citeyearpar{Zhang2014Anearly}, and the statement of Williams \emph{et
al}. \citeyearpar{Williams2000LinguliformeaCraniiformea} that such
muscles are absent in all calcite-shelled brachiopods.

\hypertarget{Pelagodiscus_atlanticus-coding-19}{}
\emph{Pelagodiscus atlanticus}: Musculature considered essentially
equivalent to \emph{Lingula} by
\citet{Williams2000LinguliformeaCraniiformea}, so \emph{Lingula} coding
followed here.

\hypertarget{Siphonobolus_priscus-coding-19}{}
\emph{Siphonobolus priscus}: Ventral musculature poorly constrained
\citep{Williams2000LinguliformeaCraniiformea, Popov2009Earlyontogeny}.

\hypertarget{Terebratulina-coding-19}{}
\emph{Terebratulina}: Williams \emph{et al}.
\citeyearpar[p.~32]{Williams2000LinguliformeaCraniiformea} state that
these muscles are absent in all carbonate-shelled brachiopods.

\subsection*{\texorpdfstring{{[}20{]} Muscle scars: Unpaired median
(\emph{levator
ani})}{{[}20{]} Muscle scars: Unpaired median (levator ani)}}\label{muscle-scars-unpaired-median-levator-ani}
\addcontentsline{toc}{subsection}{{[}20{]} Muscle scars: Unpaired median
(\emph{levator ani})}

\includegraphics{Brachiopod_phylogeny_files/figure-latex/character-mapping-20.pdf}

\begin{quote}
\textbf{Character 20: Sclerites: Bivalved: Muscle scars: Unpaired median
(\emph{levator ani})}

0: Absent\\
1: Present\\
Neomorphic character.
\end{quote}

The \emph{levator ani} is a diminutive unpaired medial muscle found in
certain calcitic brachiopods
{[}\citet{Williams2000LinguliformeaCraniiformea}; see fig. 89, character
34 in table 13{]}.

\hypertarget{Alisina-coding-20}{}
\emph{Alisina}, \emph{Kutorgina chengjiangensis}, \emph{Nisusia
sulcata}: Following table 13 in
\citet{Williams2000LinguliformeaCraniiformea}.

\hypertarget{Coolinia_pecten-coding-20}{}
\emph{Coolinia pecten}: Not reported in Dewing
\citeyearpar{Dewing2001Hingemodifications}.

\hypertarget{Craniops-coding-20}{}
\emph{Craniops}: See fig. 90 in
\citet{Williams2000LinguliformeaCraniiformea}.

\hypertarget{Gasconsia-coding-20}{}
\emph{Gasconsia}: \citet{Williams2000LinguliformeaCraniiformea} code an
unpaired medial muscle scar as present in their table 13, but give no
reference for this coding, which perhaps arises from their
interpretation of the taxon as a trimerellid. Hanken and Harper
\citeyearpar[p.~249 and text-fig. 2]{Hanken1985Thetaxonomy} explicitly
identify a pair of central muscles, so we code a \emph{levator ani} as
absent.

\hypertarget{Heliomedusa_orienta-coding-20}{}
\emph{Heliomedusa orienta}: Poor preservation of minor muscle scars
noted by Chen \emph{et al}. \citeyearpar{Chen2007Reinterpretationof}.

\hypertarget{Mickwitzia_muralensis-coding-20}{}
\emph{Mickwitzia muralensis}: Scars absent; instead, cones ornament
shell's internal surface.

\hypertarget{Novocrania-coding-20}{}
\emph{Novocrania}: Following table 13 in
\citet{Williams2000LinguliformeaCraniiformea} (for \emph{Novocrania}).

\hypertarget{Pelagodiscus_atlanticus-coding-20}{}
\emph{Pelagodiscus atlanticus}: Musculature considered essentially
equivalent to \emph{Lingula} by
\citet{Williams2000LinguliformeaCraniiformea}, so \emph{Lingula} coding
followed here.

\hypertarget{Siphonobolus_priscus-coding-20}{}
\emph{Siphonobolus priscus}: Ventral musculature poorly constrained
\citep{Williams2000LinguliformeaCraniiformea, Popov2009Earlyontogeny}.

\hypertarget{Ussunia-coding-20}{}
\emph{Ussunia}: Following table 15 in
\citet{Williams2000LinguliformeaCraniiformea}.

\subsection*{{[}21{]} Muscle scars: Dorsal
diductor}\label{muscle-scars-dorsal-diductor}
\addcontentsline{toc}{subsection}{{[}21{]} Muscle scars: Dorsal
diductor}

\includegraphics{Brachiopod_phylogeny_files/figure-latex/character-mapping-21.pdf}

\begin{quote}
\textbf{Character 21: Sclerites: Bivalved: Muscle scars: Dorsal
diductor}

0: Absent\\
1: Present\\
Neomorphic character.
\end{quote}

After Bassett \emph{et al}.
\citeyearpar{Bassett2001Functionalmorphology} character 9.

\hypertarget{Acanthotretella_spinosa-coding-21}{}
\emph{Acanthotretella spinosa}: Not observable in \emph{Acanthotretella}
itself, so coded as ambiguous -- though it is likely based on the
anticipated phylogenetic affinities of \emph{Acanthotretella} that the
muscles are absent.

\hypertarget{Askepasma_toddense-coding-21}{}
\emph{Askepasma toddense}: Not reconstructed in the the interpretation
of the musculature presented by Williams \emph{et al}.
\citeyearpar{Williams2000LinguliformeaCraniiformea}, fig. 81, but
presence cannot be confidently excluded.

\hypertarget{Clupeafumosus_socialis-coding-21}{}
\emph{Clupeafumosus socialis}: Not reported by Topper \emph{et al}.
\citeyearpar{Topper2013Reappraisalof}, nor reconstructed in generic
acrotretid by Williams \emph{et al}.
\citeyearpar{Williams2000LinguliformeaCraniiformea}.

\hypertarget{Gasconsia-coding-21}{}
\emph{Gasconsia}: Internal oblique muscles serve as diductors.

\hypertarget{Halkieria_evangelista-coding-21}{}
\emph{Halkieria evangelista}: It is unclear whether the paired muscle
scars of \emph{Oikozetetes} are homologous to brachiopod diductors.

\hypertarget{Siphonobolus_priscus-coding-21}{}
\emph{Siphonobolus priscus}: Ventral musculature poorly constrained
\citep{Williams2000LinguliformeaCraniiformea, Popov2009Earlyontogeny}.

\subsection*{{[}22{]} Muscle scars: Dorsal diductor:
Position}\label{muscle-scars-dorsal-diductor-position}
\addcontentsline{toc}{subsection}{{[}22{]} Muscle scars: Dorsal
diductor: Position}

\includegraphics{Brachiopod_phylogeny_files/figure-latex/character-mapping-22.pdf}

\begin{quote}
\textbf{Character 22: Sclerites: Bivalved: Muscle scars: Dorsal
diductor: Position}

1: Close to commissural plane\\
2: Oblique to commissural plane\\
3: At high angle to commissural plane\\
Transformational character.
\end{quote}

After Bassett \emph{et al}.
\citeyearpar{Bassett2001Functionalmorphology} character 10.

\hypertarget{Siphonobolus_priscus-coding-22}{}
\emph{Siphonobolus priscus}: Ventral musculature poorly constrained
\citep{Williams2000LinguliformeaCraniiformea, Popov2009Earlyontogeny}.

\section{Sclerites: Dorsal valve}\label{sclerites-dorsal-valve}

\subsection*{{[}23{]} Growth direction}\label{growth-direction}
\addcontentsline{toc}{subsection}{{[}23{]} Growth direction}

\includegraphics{Brachiopod_phylogeny_files/figure-latex/character-mapping-23.pdf}

\begin{quote}
\textbf{Character 23: Sclerites: Dorsal valve: Growth direction}

1: Holoperipheral\\
2: Mixoperipheral\\
3: Hemiperipheral\\
Transformational character.
\end{quote}

See Fig. 284 in Williams \emph{et al}.
\citeyearpar{Williams1997Introduction}.\\
The growth direction dictates the attitude of the cardinal area relative
to the hinge, which does not therefore represent an independent
character.\\
Crudely put, if, viewed from a dorsal position, the umbo falls within
the outer margin of the shell, growth is holoperipheral; if it falls
outside the margin, it is mixoperipheral; if it falls exactly on the
margin, it is hemiperipheral.

\hypertarget{Clupeafumosus_socialis-coding-23}{}
\emph{Clupeafumosus socialis}: Appears hemiperipheral in fig. 3 in
Topper \emph{et al}. \citeyearpar{Topper2013Reappraisalof}, though
bordering on holoperipheral, so scored as ambiguous.

\hypertarget{Craniops-coding-23}{}
\emph{Craniops}: ``both valves with growth holoperipheral'' --
\citet{Williams2000LinguliformeaCraniiformea} p164.

\hypertarget{Heliomedusa_orienta-coding-23}{}
\emph{Heliomedusa orienta}: ``holoperipheral growth in dorsal valve'' --
\citet{Williams2007Supplement}.

The insinuation from Zhang \emph{et al}.
\citeyearpar{Zhang2009Architectureand} is that Chen \emph{et al}.
\citeyearpar{Chen2007Reinterpretationof} misidentify the dorsal valve as
the ventral valve.

\hypertarget{Micrina-coding-23}{}
\emph{Micrina}: See Holmer \emph{et al}.
\citeyearpar{Holmer2008TheEarly}.

\hypertarget{Paterimitra-coding-23}{}
\emph{Paterimitra}: S2 and L sclerites are clearly holoperipheral. See
\citet{Larsson2014iPaterimitra}, fig. 2.

\hypertarget{Ussunia-coding-23}{}
\emph{Ussunia}: Following description of order in
\citet{Williams2000LinguliformeaCraniiformea}.

\subsection*{{[}24{]} Posterior surface:
Differentiated}\label{posterior-surface-differentiated}
\addcontentsline{toc}{subsection}{{[}24{]} Posterior surface:
Differentiated}

\includegraphics{Brachiopod_phylogeny_files/figure-latex/character-mapping-24.pdf}

\begin{quote}
\textbf{Character 24: Sclerites: Dorsal valve: Posterior surface:
Differentiated}

0: Posterior shell not differentiated\\
1: Posterior shell forms distinct cardinal area or pseudointerarea\\
Neomorphic character.
\end{quote}

In shells that grow by mixoperipheral growth, the triangular area
subtended between each apex and the posterior ends of the lateral
margins is termed the cardinal area. In shells with holoperipheral
growth, a flattened surface on the posterior margin of the valve is
termed a pseudointerarea
\citep[paraphrasing][]{Williams1997Introduction}.

In order for this character to be independent of a shell's growth
direction, we do not distinguish between a ``cardinal area'',
``interarea'' or ``pseudointerarea''.

\hypertarget{Acanthotretella_spinosa-coding-24}{}
\emph{Acanthotretella spinosa}: Pseudointerarea present, following
Siphonotretidae coding in Williams \emph{et al}.
\citeyearpar{Williams2000LinguliformeaCraniiformea}, table 6.

\hypertarget{Alisina-coding-24}{}
\emph{Alisina}, \emph{Antigonambonites planus}, \emph{Coolinia pecten},
\emph{Glyptoria}, \emph{Kutorgina chengjiangensis}, \emph{Orthis},
\emph{Salanygolina}, \emph{Tomteluva perturbata}: Cardinal area
(interarea) present.

\hypertarget{Askepasma_toddense-coding-24}{}
\emph{Askepasma toddense}: Well-defined pseudointerarea
\citep[p153]{Williams2000LinguliformeaCraniiformea}.

\hypertarget{Botsfordia-coding-24}{}
\emph{Botsfordia}: ``dorsal pseudointerarea vestigial, divided by median
groove'' -- \citet{Williams2000LinguliformeaCraniiformea}.

\hypertarget{Clupeafumosus_socialis-coding-24}{}
\emph{Clupeafumosus socialis}: Pseudointerarea present; figured by
Topper \emph{et al}. \citeyearpar{Topper2013Reappraisalof}, fig. 3j.

\hypertarget{Craniops-coding-24}{}
\emph{Craniops}: ``Only some craniopsids (Lingulapholis, Pseudopholidops
{[}not \emph{Craniops}{]}) have well-developed pseudointerareas.'' --
\citet{Williams2000LinguliformeaCraniiformea}.

\hypertarget{Gasconsia-coding-24}{}
\emph{Gasconsia}: Absent: the dorsal (branchial) pseudointerarea of
\emph{G. schucherti} is ``reduced or obsolete''; that of \emph{G.
worsleyi} ``short, virtually obsolete'' \citep{Hanken1985Thetaxonomy}.

\hypertarget{Haplophrentis_carinatus-coding-24}{}
\emph{Haplophrentis carinatus}: A very short pseudointerarea appears to
be present \citep{Moysiuk2017Hyolithsare}.

\hypertarget{Heliomedusa_orienta-coding-24}{}
\emph{Heliomedusa orienta}: Pseudointerea in ventral valve, but not
dorsal valve \citep[2007]{Williams2000LinguliformeaCraniiformea}.

\hypertarget{Lingula-coding-24}{}
\emph{Lingula}, \emph{Lingulellotreta malongensis}: Pseudointerarea
present, following Williams \emph{et al}.
\citeyearpar{Williams2000LinguliformeaCraniiformea}, table 6.

\hypertarget{Lingulosacculus-coding-24}{}
\emph{Lingulosacculus}: Unclear from fossil material.

\hypertarget{Longtancunella_chengjiangensis-coding-24}{}
\emph{Longtancunella chengjiangensis}: Zhang \emph{et al}.
\citeyearpar{Zhang2011Theexceptionally} note that ``all evidence of a
pseudointerarea is lacking'', but the two-dimensional preservation style
of Chengjiang material makes details of dorsal valve difficult to
distinguish, and the possibility of a diminutive pseudointerarea cannot
be excluded with total confidence.

\hypertarget{Mickwitzia_muralensis-coding-24}{}
\emph{Mickwitzia muralensis}: Shell flat.

\hypertarget{Micrina-coding-24}{}
\emph{Micrina}: = Sellate sclerite duplicature
\citep{Holmer2008TheEarly}.

\hypertarget{Micromitra-coding-24}{}
\emph{Micromitra}: ``Dorsal pseudointerarea usually well defined, low,
anacline to catacline'' --
\citet{Williams2000LinguliformeaCraniiformea}.

\hypertarget{Mummpikia_nuda-coding-24}{}
\emph{Mummpikia nuda}: ``Information on the dorsal interarea is
inconclusive {[}\ldots{}{]} no obvious\\
interarea is recognisable; whether or not this is the primary state or a
taphonomic artefact is difficult to assess'' --
\citet{Balthasar2008iMummpikia}, p.~276.

\hypertarget{Nisusia_sulcata-coding-24}{}
\emph{Nisusia sulcata}: Cardinal area (interarea) present -- with
reference to Holmer \emph{et al}.
\citeyearpar{Holmer2018Evolutionarysignificance}.

\hypertarget{Novocrania-coding-24}{}
\emph{Novocrania}, \emph{Paterimitra}, \emph{Pedunculotheca diania}:
Pseudointerarea.

\hypertarget{Pelagodiscus_atlanticus-coding-24}{}
\emph{Pelagodiscus atlanticus}: Absent, following entry for Discinidae
in Williams \emph{et al}.
\citeyearpar{Williams2000LinguliformeaCraniiformea}, table 6.

\hypertarget{Siphonobolus_priscus-coding-24}{}
\emph{Siphonobolus priscus}: ``Dorsal pseudointerarea weakly anacline,
undivided, elevated above the valve floor'' --
\citet{Popov2009Earlyontogeny}.

\hypertarget{Terebratulina-coding-24}{}
\emph{Terebratulina}: Interarea present.

\hypertarget{Ussunia-coding-24}{}
\emph{Ussunia}: Following table 15 in
\citet{Williams2000LinguliformeaCraniiformea}.

\hypertarget{Yuganotheca_elegans-coding-24}{}
\emph{Yuganotheca elegans}: A differentiated region is not obvious in
fossil material or its reconstruction \citep{Zhang2014Anearly}, but the
two-dimensional preservation style of Chengjiang material makes details
of dorsal valve difficult to distinguish, and the possibility of a
diminutive pseudointerarea cannot be excluded with confidence.

\subsection*{{[}25{]} Differentiated posterior surface:
Morphology}\label{differentiated-posterior-surface-morphology}
\addcontentsline{toc}{subsection}{{[}25{]} Differentiated posterior
surface: Morphology}

\includegraphics{Brachiopod_phylogeny_files/figure-latex/character-mapping-25.pdf}

\begin{quote}
\textbf{Character 25: Sclerites: Dorsal valve: Differentiated posterior
surface: Morphology}

0: Curved lateral profile\\
1: Planar lateral profile\\
Neomorphic character.
\end{quote}

It is possible for a cardinal area or pseudointerarea to be distinct
from the anterior part of the shell, yet to remain curved in lateral
profile.

Taking an undifferentiated posterior margin as primitive, the primitive
condition is curved -- flattening of the posterior margin represents an
additional modification that can only occur once the posterior margin is
differentiated.

\hypertarget{Botsfordia-coding-25}{}
\emph{Botsfordia}: ``Curved pseudointerarea'' --
\citet{Skovsted2017Depthrelated}.

\hypertarget{Clupeafumosus_socialis-coding-25}{}
\emph{Clupeafumosus socialis}: Truncated but essentially planar surface;
see e.g.~p196 of \citet{Topper2013Reappraisalof}.

\hypertarget{Eoobolus-coding-25}{}
\emph{Eoobolus}: Essentially planar; see Balthasar
\citeyearpar{Balthasar2009Thebrachiopod}, fig. 4a.

\hypertarget{Gasconsia-coding-25}{}
\emph{Gasconsia}, \emph{Heliomedusa orienta}, \emph{Mickwitzia
muralensis}, \emph{Pelagodiscus atlanticus}, \emph{Ussunia}: Posterior
surface cannot be flat if it is not differentiated.

\hypertarget{Micromitra-coding-25}{}
\emph{Micromitra}: Essentially straight; see fig. 3.7 in
\citet{Ushatinskaya2016Protegulumand}.

\hypertarget{Pedunculotheca_diania-coding-25}{}
\emph{Pedunculotheca diania}: Difficult to evaluate based on present
material, given low nature of valve and compressed preservation.

\hypertarget{Siphonobolus_priscus-coding-25}{}
\emph{Siphonobolus priscus}: The short interarea appears planar (see for
example Popov et a. 2009 fig. 6A), but its short length makes it
difficult to establish whether slight curvature is present.

\subsection*{{[}26{]} Posterior surface: Medial
groove}\label{posterior-surface-medial-groove}
\addcontentsline{toc}{subsection}{{[}26{]} Posterior surface: Medial
groove}

\includegraphics{Brachiopod_phylogeny_files/figure-latex/character-mapping-26.pdf}

\begin{quote}
\textbf{Character 26: Sclerites: Dorsal valve: Posterior surface: Medial
groove}

0: Absent\\
1: Present\\
Neomorphic character.
\end{quote}

Following character 29 in Williams \emph{et al}.
\citeyearpar{Williams2000LinguliformeaCraniiformea}, table 9 (which
relates to pseudointerarea).

\hypertarget{Acanthotretella_spinosa-coding-26}{}
\emph{Acanthotretella spinosa}: The dorsal pseudointerarea is poorly
preserved, but appears to have a median groove
\citep{Holmer2006Aspinose}.

\hypertarget{Botsfordia-coding-26}{}
\emph{Botsfordia}: ``dorsal pseudointerarea vestigial, divided by median
groove'' -- \citet{Williams2000LinguliformeaCraniiformea}.

\hypertarget{Clupeafumosus_socialis-coding-26}{}
\emph{Clupeafumosus socialis}: Present; figured by Topper \emph{et al}.
\citeyearpar{Topper2013Reappraisalof}, fig. 3j.

\hypertarget{Eoobolus-coding-26}{}
\emph{Eoobolus}: Prominent medial groove
\citep{Balthasar2009Thebrachiopod}.

\hypertarget{Heliomedusa_orienta-coding-26}{}
\emph{Heliomedusa orienta}: ``A posteriorly protruding dorsal
pseudointerarea with no median groove and no flexure lines'' --
\citet{Chen2007Reinterpretationof}.

\hypertarget{Lingulellotreta_malongensis-coding-26}{}
\emph{Lingulellotreta malongensis}: Dorsal pseudointerarea with wide,
concave median groove and short propareas" --
\citet{Williams2000LinguliformeaCraniiformea}.

\hypertarget{Siphonobolus_priscus-coding-26}{}
\emph{Siphonobolus priscus}: The dorsal pseudointerarea of \emph{S.
priscus} is undivided \citep{Popov2009Earlyontogeny}, but in other
species it is divided by a ``wide, poorly defined median groove''
\citep{Williams2000LinguliformeaCraniiformea}. Coded, therefore, as
polymorphic.

\subsection*{{[}27{]} Posterior surface:
Notothyrium}\label{posterior-surface-notothyrium}
\addcontentsline{toc}{subsection}{{[}27{]} Posterior surface:
Notothyrium}

\includegraphics{Brachiopod_phylogeny_files/figure-latex/character-mapping-27.pdf}

\begin{quote}
\textbf{Character 27: Sclerites: Dorsal valve: Posterior surface:
Notothyrium}

0: Absent\\
1: Present\\
Neomorphic character.
\end{quote}

A notothyrium is an opening in an interarea that accommodates the
pedicle, and may be filled with plates.

\hypertarget{Botsfordia-coding-27}{}
\emph{Botsfordia}: Following \citet{Williams1998Thediversity}, appendix
2.

\hypertarget{Longtancunella_chengjiangensis-coding-27}{}
\emph{Longtancunella chengjiangensis}: No evidence or report of an
opening at the hinge line in fossil material in
\citet{Zhang2007Agregarious} or \citet{Zhang2011Theexceptionally}.

\subsection*{{[}28{]} Posterior surface: Notothyrium:
Shape}\label{posterior-surface-notothyrium-shape}
\addcontentsline{toc}{subsection}{{[}28{]} Posterior surface:
Notothyrium: Shape}

\includegraphics{Brachiopod_phylogeny_files/figure-latex/character-mapping-28.pdf}

\begin{quote}
\textbf{Character 28: Sclerites: Dorsal valve: Posterior surface:
Notothyrium: Shape}

1: Parallel-sided cleft\\
2: Triangular\\
Transformational character.
\end{quote}

A notothyrium is an opening in an interarea that accommodates the
pedicle, and may be filled with plates.

A simplification of character 5 in
\citet{Bassett2001Functionalmorphology}.

\subsection*{{[}29{]} Posterior surface: Notothyrium: Chilidial
plates}\label{posterior-surface-notothyrium-chilidial-plates}
\addcontentsline{toc}{subsection}{{[}29{]} Posterior surface:
Notothyrium: Chilidial plates}

\includegraphics{Brachiopod_phylogeny_files/figure-latex/character-mapping-29.pdf}

\begin{quote}
\textbf{Character 29: Sclerites: Dorsal valve: Posterior surface:
Notothyrium: Chilidial plates}

1: Open\\
2: Covered by chilidial plates\\
Transformational character.
\end{quote}

A notothyrium may be open or covered by a chilidium or two chilidial
plates.\\
No included taxa exhibit more than one chilidial plate.\\
Transformational as it is not self-evident whether the ancestral taxon
had an open or closed notothyrium.

\subsection*{{[}30{]} Notothyrial platform}\label{notothyrial-platform}
\addcontentsline{toc}{subsection}{{[}30{]} Notothyrial platform}

\includegraphics{Brachiopod_phylogeny_files/figure-latex/character-mapping-30.pdf}

\begin{quote}
\textbf{Character 30: Sclerites: Dorsal valve: Notothyrial platform}

0: Absent\\
1: Present\\
Neomorphic character.
\end{quote}

After Bassett \emph{et al}.
\citeyearpar{Bassett2001Functionalmorphology} character 12.\\
The presence or absence of a notothyrial platform, which often serves as
an attachment point for the diductors in a similar fashion to the
cardinal processes, is independent of the presence of a notothyrium.

\hypertarget{Alisina-coding-30}{}
\emph{Alisina}, \emph{Glyptoria}: Bassett \emph{et al}.
\citeyearpar{Bassett2001Functionalmorphology} score as present in Table
18.1.

\hypertarget{Coolinia_pecten-coding-30}{}
\emph{Coolinia pecten}: Referred to as the ``posterior platform'' in
Dewing \citeyearpar{Dewing2001Hingemodifications}.

\hypertarget{Kutorgina_chengjiangensis-coding-30}{}
\emph{Kutorgina chengjiangensis}: ``Dorsal diductor scars impressed on
floor of notothyrial cavity'':
\citet{Williams2000LinguliformeaCraniiformea}, regarding Kutorginata.\\
Bassett \emph{et al}. \citeyearpar{Bassett2001Functionalmorphology}
score as absent in Table 18.1.

\hypertarget{Nisusia_sulcata-coding-30}{}
\emph{Nisusia sulcata}: Bassett \emph{et al}.
\citeyearpar{Bassett2001Functionalmorphology} score as absent in Table
18.1.\\
``Dorsal diductor scars impressed on floor of notothyrial cavity'':
\citet{Williams2000LinguliformeaCraniiformea}, regarding Kutorginata.

\hypertarget{Ussunia-coding-30}{}
\emph{Ussunia}: ``Visceral platforms absent in both valves'' --
\citet{Williams2000LinguliformeaCraniiformea}, p.~192.

\subsection*{{[}31{]} Cardinal shield}\label{cardinal-shield}
\addcontentsline{toc}{subsection}{{[}31{]} Cardinal shield}

\includegraphics{Brachiopod_phylogeny_files/figure-latex/character-mapping-31.pdf}

\begin{quote}
\textbf{Character 31: Sclerites: Dorsal valve: Cardinal shield}

0: Absent\\
1: Present\\
Neomorphic character.
\end{quote}

A prominent platform in the hyolith operculum. With no obvious sites for
muscle attachment, it is unlikely to be homologous to the notothyrial
platform.

\subsection*{{[}32{]} Cardinal processes}\label{cardinal-processes}
\addcontentsline{toc}{subsection}{{[}32{]} Cardinal processes}

\includegraphics{Brachiopod_phylogeny_files/figure-latex/character-mapping-32.pdf}

\begin{quote}
\textbf{Character 32: Sclerites: Dorsal valve: Cardinal processes}

0: Absent\\
1: Present\\
Neomorphic character.
\end{quote}

After Bassett \emph{et al}.
\citeyearpar{Bassett2001Functionalmorphology} character 13.\\
Cardinal processes are unlikely to be homologous with the notothyrial
platform, even if their function is similar.

\hypertarget{Clupeafumosus_socialis-coding-32}{}
\emph{Clupeafumosus socialis}: Not reported by Topper \emph{et al}.
\citeyearpar{Topper2013Reappraisalof}.

\hypertarget{Longtancunella_chengjiangensis-coding-32}{}
\emph{Longtancunella chengjiangensis}: Not evident, and ought arguably
to be discernable if present given the quality of preservation.

\subsection*{{[}33{]} Medial septum}\label{medial-septum}
\addcontentsline{toc}{subsection}{{[}33{]} Medial septum}

\includegraphics{Brachiopod_phylogeny_files/figure-latex/character-mapping-33.pdf}

\begin{quote}
\textbf{Character 33: Sclerites: Dorsal valve: Medial septum}

0: Absent\\
1: Present\\
Neomorphic character.
\end{quote}

The dorsal valve of many taxa is exhibits a septum or process (or
myophragm) along the medial line. See character 25 in Benedetto
\citeyearpar{Benedetto2009iChaniella}.

\hypertarget{Acanthotretella_spinosa-coding-33}{}
\emph{Acanthotretella spinosa}: Not described by Holmer \& Caron
\citeyearpar{Holmer2006Aspinose}, but an unannotated linear feature
corresponds to the position of a median septum. Without detailed study
of the specimen, we opt to score this as ambiguous.

\hypertarget{Antigonambonites_planus-coding-33}{}
\emph{Antigonambonites planus}: Weakly developed septum evident in
internal cast: \citet{Williams2000LinguliformeaCraniiformea}, fig.
508.2e.

\hypertarget{Botsfordia-coding-33}{}
\emph{Botsfordia}: ``dorsal interior with narrow anterior projection
extending to midvalve, bisected by median ridge'' --
\citet{Williams2000LinguliformeaCraniiformea}.

\hypertarget{Clupeafumosus_socialis-coding-33}{}
\emph{Clupeafumosus socialis}: Prominent process evident
\citep{Topper2013Reappraisalof}.

\hypertarget{Eoobolus-coding-33}{}
\emph{Eoobolus}: A ``median projection'' is present \citep[fig. 4g
in][]{Balthasar2009Thebrachiopod}.

\hypertarget{Glyptoria-coding-33}{}
\emph{Glyptoria}: Neither evident nor reported in Williams \emph{et al}.
\citeyearpar{Williams2000LinguliformeaCraniiformea}.

\hypertarget{Heliomedusa_orienta-coding-33}{}
\emph{Heliomedusa orienta}: Reported on `ventral' valve by Chen \emph{et
al}. \citeyearpar{Chen2007Reinterpretationof}; we consider their
`ventral' valve to be the dorsal valve.

The structure is unambiguously figured \citep[e.g.~fig. 5.1
in][]{Chen2007Reinterpretationof}, contra its coding as absent in
\citet{Williams2000LinguliformeaCraniiformea} and its lack of mention in
\citet{Williams2007Supplement} or \citet{Zhang2009Architectureand}.

\hypertarget{Kutorgina_chengjiangensis-coding-33}{}
\emph{Kutorgina chengjiangensis}: Absent -- fig. 129.1f in Williams
\emph{et al}. \citeyearpar{Williams2000LinguliformeaCraniiformea}.

\hypertarget{Lingulellotreta_malongensis-coding-33}{}
\emph{Lingulellotreta malongensis}: Very weakly developed but seemingly
present between muscle scars in \emph{Lingulellotreta}, more prominent
in Aboriginella (also Lingulellotretidae) \citep[fig.
34]{Williams2000LinguliformeaCraniiformea}.

\hypertarget{Lingulosacculus-coding-33}{}
\emph{Lingulosacculus}: It is not possible to determine, based on the
material presented in Balthasar \& Butterfield
\citeyearpar{Balthasar2009EarlyCambrian}, whether the anterior
projection of the visceral area in the dorsal valve corresponds to a
medial septum in the underlying shell.

\hypertarget{Mummpikia_nuda-coding-33}{}
\emph{Mummpikia nuda}: See pl. 2 panel 6 in Balthasar
\citeyearpar{Balthasar2008iMummpikia}.

\hypertarget{Nisusia_sulcata-coding-33}{}
\emph{Nisusia sulcata}: Fig. 125 in Williams \emph{et al}.
\citeyearpar{Williams2000LinguliformeaCraniiformea}.

\hypertarget{Novocrania-coding-33}{}
\emph{Novocrania}: Median process evident: Williams \emph{et al}.
\citeyearpar{Williams2000LinguliformeaCraniiformea} fig. 100.2a, d.

\hypertarget{Orthis-coding-33}{}
\emph{Orthis}: Short medial process (``low median ridge'', p.~724)
present in dorsal valve; see Fig. 523.3b in Williams \emph{et al}.
\citeyearpar{Williams2000LinguliformeaCraniiformea}.

\hypertarget{Siphonobolus_priscus-coding-33}{}
\emph{Siphonobolus priscus}: ``Dorsal interior {[}\ldots{}{]} bisected
by a short median ridge.'' -- \citet{Popov2009Earlyontogeny}.

\hypertarget{Ussunia-coding-33}{}
\emph{Ussunia}: Following char 42 in table 15 in
\citet{Williams2000LinguliformeaCraniiformea}.

\subsection*{{[}34{]} Clavicles}\label{clavicles}
\addcontentsline{toc}{subsection}{{[}34{]} Clavicles}

\includegraphics{Brachiopod_phylogeny_files/figure-latex/character-mapping-34.pdf}

\begin{quote}
\textbf{Character 34: Sclerites: Dorsal valve: Clavicles}

0: Absent\\
1: Present\\
Neomorphic character.
\end{quote}

Prominent symmetrical ridges on the inner surface of the hyolith
operculum.

\section{Sclerites: Ventral valve}\label{sclerites-ventral-valve}

\subsection*{{[}35{]} Relative size}\label{relative-size}
\addcontentsline{toc}{subsection}{{[}35{]} Relative size}

\includegraphics{Brachiopod_phylogeny_files/figure-latex/character-mapping-35.pdf}

\begin{quote}
\textbf{Character 35: Sclerites: Ventral valve: Relative size}

1: Ventral valve markedly larger than dorsal valve (ventribiconvex)\\
2: Equivalve (subequally biconvex)\\
3: Dorsal valve markedly larger than ventral valve (dorsibiconvex)\\
Transformational character.
\end{quote}

In many brachiopods, the valves are closely similar in size; in others,
the ventral valve is markedly larger than the dorsal, on account of
being more convex. Marginal cases are treated as ambiguous for the
relevant states.

\hypertarget{Antigonambonites_planus-coding-35}{}
\emph{Antigonambonites planus}: Broadly equivalve -- see Williams
\emph{et al}. \citeyearpar{Williams2000LinguliformeaCraniiformea} fig.
508.2c.

\hypertarget{Botsfordia-coding-35}{}
\emph{Botsfordia}: After table 8 in Williams \emph{et al}.
\citeyearpar{Williams2000LinguliformeaCraniiformea}.

\hypertarget{Craniops-coding-35}{}
\emph{Craniops}: ``Shell subequally biconvex'' --
\citet{Williams2000LinguliformeaCraniiformea}.

\hypertarget{Eoobolus-coding-35}{}
\emph{Eoobolus}: ``\emph{Eoobolus} is biconvex'', but in his amended
diagnosis, Balthasar \citeyearpar{Balthasar2009Thebrachiopod} described
it as ``shell inequivalved, dorsibiconvex''.

\hypertarget{Gasconsia-coding-35}{}
\emph{Gasconsia}: Convexiplane
\citep[p.~187]{Williams2000LinguliformeaCraniiformea}.

\hypertarget{Heliomedusa_orienta-coding-35}{}
\emph{Heliomedusa orienta}: Ventral valve larger than the dorsal valve
\citep[p.~659]{Zhang2009Architectureand}.

\hypertarget{Kutorgina_chengjiangensis-coding-35}{}
\emph{Kutorgina chengjiangensis}: Ventral valve larger \citep[see][fig.
125.]{Williams2000LinguliformeaCraniiformea}.

\hypertarget{Longtancunella_chengjiangensis-coding-35}{}
\emph{Longtancunella chengjiangensis}, \emph{Yuganotheca elegans}: The
ventral valve is somewhat, but not markedly, larger than the dorsal; as
such, this character is coded ambiguous for equivalve/ventral valve
larger.

\hypertarget{Mummpikia_nuda-coding-35}{}
\emph{Mummpikia nuda}: Aside from hinge, valves similar in convexity and
size \citep{Balthasar2008iMummpikia}.

\hypertarget{Nisusia_sulcata-coding-35}{}
\emph{Nisusia sulcata}: Ventral valve larger \citep[see][fig.
126.]{Williams2000LinguliformeaCraniiformea}.

\hypertarget{Siphonobolus_priscus-coding-35}{}
\emph{Siphonobolus priscus}: Ventribiconvex
\citep{Popov2009Earlyontogeny}.

\hypertarget{Ussunia-coding-35}{}
\emph{Ussunia}: Subequally biconvex
\citep[p.~192]{Williams2000LinguliformeaCraniiformea}.

\subsection*{{[}36{]} Growth direction}\label{growth-direction-1}
\addcontentsline{toc}{subsection}{{[}36{]} Growth direction}

\includegraphics{Brachiopod_phylogeny_files/figure-latex/character-mapping-36.pdf}

\begin{quote}
\textbf{Character 36: Sclerites: Ventral valve: Growth direction}

1: Holoperipheral\\
2: Mixoperipheral\\
3: Hemiperipheral\\
Transformational character.
\end{quote}

See Fig. 284 in Williams \emph{et al}.
\citeyearpar{Williams1997Introduction} for depiction of terms.

The growth direction dictates the attitude of the cardinal area relative
to the hinge, which does not therefore represent an independent
character.\\
Crudely put, if, viewed from a dorsal position, the umbo falls within
the outer margin of the shell, growth is holoperipheral; if it falls
outside the margin, it is mixoperipheral; if it falls exactly on the
margin, it is hemiperipheral.

\hypertarget{Clupeafumosus_socialis-coding-36}{}
\emph{Clupeafumosus socialis}: Inferred from Topper \emph{et al}.
\citeyearpar{Topper2013Reappraisalof}.

\hypertarget{Heliomedusa_orienta-coding-36}{}
\emph{Heliomedusa orienta}: Williams \emph{et al}.
\citeyearpar[2007]{Williams2000LinguliformeaCraniiformea} reconstruct
mixoperipheral growth in the ventral valve {[}though Chen \emph{et al}.
\citeyearpar{Chen2007Reinterpretationof} reconstruct the valves the
other way round, i.e.~it is the ventral valve that grows
holoperipherally, and the dorsal mixoperipherally{]}.

\hypertarget{Paterimitra-coding-36}{}
\emph{Paterimitra}: The apical flange notwithstanding, the umbo of the
S1 sclerite is posterior of the hinge line and the posterior edge of the
lateral plate -- see \citet{Larsson2014iPaterimitra}, fig. 2a, c.

\hypertarget{Siphonobolus_priscus-coding-36}{}
\emph{Siphonobolus priscus}: Initially holoperipheral
\citep[p.~159]{Popov2009Earlyontogeny}, then on the brink of being
mixoperipheral in adulthood, so coded as polymorphic.

\hypertarget{Ussunia-coding-36}{}
\emph{Ussunia}: Following description of order in
\citet{Williams2000LinguliformeaCraniiformea}.

\subsection*{{[}37{]} Posterior surface:
Differentiated}\label{posterior-surface-differentiated-1}
\addcontentsline{toc}{subsection}{{[}37{]} Posterior surface:
Differentiated}

\includegraphics{Brachiopod_phylogeny_files/figure-latex/character-mapping-37.pdf}

\begin{quote}
\textbf{Character 37: Sclerites: Ventral valve: Posterior surface:
Differentiated}

0: Posterior surface of shell not differentiated\\
1: Posterior surface of shell forms distinct cardinal area or
pseudointerarea\\
Neomorphic character.
\end{quote}

In shells that grow by mixoperipheral growth, the triangular area
subtended between each apex and the posterior ends of the lateral
margins is termed the cardinal area. In shells with holoperipheral
growth, a flattened surface on the posterior margin of the valve is
termed a pseudointerarea
\citep[paraphrasing][]{Williams1997Introduction}.

In order for this character to be independent of a shell's growth
direction, we do not distinguish between a ``cardinal area'',
``interarea'' or ``pseudointerarea''.

\hypertarget{Alisina-coding-37}{}
\emph{Alisina}, \emph{Antigonambonites planus}, \emph{Coolinia pecten},
\emph{Glyptoria}, \emph{Kutorgina chengjiangensis}, \emph{Nisusia
sulcata}, \emph{Orthis}, \emph{Salanygolina}, \emph{Tomteluva
perturbata}: Interarea present.

\hypertarget{Clupeafumosus_socialis-coding-37}{}
\emph{Clupeafumosus socialis}: Described by Topper \emph{et al}.
\citeyearpar{Topper2013Reappraisalof}.

\hypertarget{Gasconsia-coding-37}{}
\emph{Gasconsia}: The region corresponding to the ventral
(pseudo)interarea is described as a ``trimerellid ventral cardinal
area'' by Williams \emph{et al}.
\citeyearpar[p.162]{Williams2000LinguliformeaCraniiformea}, who code
both an interarea and a pseudointerarea as absent in trimerellids.

\hypertarget{Heliomedusa_orienta-coding-37}{}
\emph{Heliomedusa orienta}: Zhang \emph{et al}.
\citeyearpar{Zhang2009Architectureand} report a moderate to somewhat
developed ventral pseudointerarea, confirmed by Williams \emph{et al}.
\citeyearpar{Williams2007Supplement}.

\hypertarget{Lingulosacculus-coding-37}{}
\emph{Lingulosacculus}: The conical valve is interpreted as the ventral
valve with an extended pseudointerarea.

\hypertarget{Longtancunella_chengjiangensis-coding-37}{}
\emph{Longtancunella chengjiangensis}: Though ``all evidence of a
pseudointerarea is lacking'' -- \citet{Zhang2011Theexceptionally} -- the
region of the shell between the strophic hinge line and the colleplax
\citep[fig. 2 in][]{Zhang2011Theexceptionally} is distinct from the rest
of the shell; the ends of the strophic hinge line are marked by
prominent nicks in the shell margin. \emph{Longtancunella} is therefore
coded as having a differentiated posterior surface.

\hypertarget{Mickwitzia_muralensis-coding-37}{}
\emph{Mickwitzia muralensis}: Termed an interarea by Balthasar
\citeyearpar{Balthasar2004Shellstructure}.

\hypertarget{Mummpikia_nuda-coding-37}{}
\emph{Mummpikia nuda}: Balthasar \citeyearpar{Balthasar2008iMummpikia}
interprets a pseudointerarea as being present -- e.g.~p273, ``Of
particular interest is the vault that bridges the most anterior portion
of the ventral pseudointerarea and raises it above the visceral
platform.''; ``This pattern is reversed in the ventral valves of
\emph{M. nuda}, where the anterior projection of the pedicle groove is
raised above the valve floor whereas the lateral parts of
pseudointerarea are not''.

\hypertarget{Paterimitra-coding-37}{}
\emph{Paterimitra}: Triangular notch and subapical flange.

\hypertarget{Siphonobolus_priscus-coding-37}{}
\emph{Siphonobolus priscus}: ``Ventral pseudointerarea, low, undivided,
poorly defined'' -- \citet{Williams2000LinguliformeaCraniiformea}.

\hypertarget{Terebratulina-coding-37}{}
\emph{Terebratulina}: Interarea.

\hypertarget{Ussunia-coding-37}{}
\emph{Ussunia}: Following char 17 in table 15 in
\citet{Williams2000LinguliformeaCraniiformea}.

\subsection*{{[}38{]} Posterior margin growth
direction}\label{posterior-margin-growth-direction}
\addcontentsline{toc}{subsection}{{[}38{]} Posterior margin growth
direction}

\includegraphics{Brachiopod_phylogeny_files/figure-latex/character-mapping-38.pdf}

\begin{quote}
\textbf{Character 38: Sclerites: Ventral valve: Posterior margin growth
direction}

1: Inward-growing\\
2: Outward-growing\\
Transformational character.
\end{quote}

Balthasar \citeyearpar{Balthasar2008iMummpikia} notes an inward-growing
posterior margin of the pseudointerarea as potentially linking
\emph{Mummpikia} with the linguliform brachiopods.

Coded as inapplicable in taxa without a differentiated posterior margin:
the posterior margin can only grow inwards if it is differentiated from
the anterior margin; else the entire shell would grow in on itself.

\hypertarget{Botsfordia-coding-38}{}
\emph{Botsfordia}: Inward-growing; see Skovsted \& Holmer
\citeyearpar{Skovsted2005EarlyCambrian}, pl. 4.

\hypertarget{Clupeafumosus_socialis-coding-38}{}
\emph{Clupeafumosus socialis}: See Topper \emph{et al}.
\citeyearpar{Topper2013Reappraisalof}.

\hypertarget{Eoobolus-coding-38}{}
\emph{Eoobolus}: See for example Skovsted \& Holmer
\citeyearpar{Skovsted2005EarlyCambrian}, pl. 3.

\hypertarget{Lingulellotreta_malongensis-coding-38}{}
\emph{Lingulellotreta malongensis}: Transverse cross section of ventral
pseudointerarea concave.

\hypertarget{Mummpikia_nuda-coding-38}{}
\emph{Mummpikia nuda}: Balthasar \citeyearpar{Balthasar2008iMummpikia}
interprets an inward-growing posterior margin of the pseudointerarea --
e.g.~p273, ``Of particular interest is the vault that bridges the most
anterior portion of the ventral pseudointerarea and raises it above the
visceral platform {[}\ldots{}{]} An inward-growing posterior margin is
otherwise known only from the pseudointerareas of linguliform
brachiopods''.

\subsection*{{[}39{]} Posterior surface:
Planar}\label{posterior-surface-planar}
\addcontentsline{toc}{subsection}{{[}39{]} Posterior surface: Planar}

\includegraphics{Brachiopod_phylogeny_files/figure-latex/character-mapping-39.pdf}

\begin{quote}
\textbf{Character 39: Sclerites: Ventral valve: Posterior surface:
Planar}

0: Curved lateral profile\\
1: Planar lateral profile\\
Neomorphic character.
\end{quote}

It is possible for a cardinal area or pseudointerarea to be distinct
from the anterior part of the shell, yet to remain curved in lateral
profile.

Taking an undifferentiated posterior margin as primitive, the primitive
condition is curved -- flattening of the posterior margin represents an
additional modification that can only occur once the posterior margin is
differentiated.

A flat and triangular interarea links \emph{Mummpikia} with the
Obolellidae \citep{Balthasar2008iMummpikia} -- but all included taxa
have triangular interareas, so this is not listed as a separate
character.

\hypertarget{Acanthotretella_spinosa-coding-39}{}
\emph{Acanthotretella spinosa}: ventral pseudointerareas are most
similar to those found within the Order Siphonotretida.

\hypertarget{Botsfordia-coding-39}{}
\emph{Botsfordia}: See Skovsted \& Holmer
\citeyearpar{Skovsted2005EarlyCambrian}, pl. 3, fig. 14.

\hypertarget{Clupeafumosus_socialis-coding-39}{}
\emph{Clupeafumosus socialis}: ``Ventral pseudointerarea is gently
procline and is flat in lateral profile''. ---\\
\citep{Topper2013Reappraisalof}.

\hypertarget{Eoobolus-coding-39}{}
\emph{Eoobolus}: Some curvature retained.

\hypertarget{Lingulellotreta_malongensis-coding-39}{}
\emph{Lingulellotreta malongensis}: Transverse cross section of ventral
pseudointerarea concave.

\hypertarget{Longtancunella_chengjiangensis-coding-39}{}
\emph{Longtancunella chengjiangensis}: Flattened, reflecting the
strophic hinge line.

\hypertarget{Micromitra-coding-39}{}
\emph{Micromitra}: Essentially planar; see fig. 6 in
\citet{Ushatinskaya2016Protegulumand}.

\hypertarget{Siphonobolus_priscus-coding-39}{}
\emph{Siphonobolus priscus}: `Almost' planar -- see Popov \emph{et al}.
\citeyearpar[fig. 4]{Popov2009Earlyontogeny}. Coded as ambiguous.

\subsection*{{[}40{]} Posterior surface:
Extent}\label{posterior-surface-extent}
\addcontentsline{toc}{subsection}{{[}40{]} Posterior surface: Extent}

\includegraphics{Brachiopod_phylogeny_files/figure-latex/character-mapping-40.pdf}

\begin{quote}
\textbf{Character 40: Sclerites: Ventral valve: Posterior surface:
Extent}

1: Low\\
2: High\\
Transformational character.
\end{quote}

Distinguishes taxa whose ventral valve is essentially flat from those
that are essentially conical.

\hypertarget{Antigonambonites_planus-coding-40}{}
\emph{Antigonambonites planus}: Though scored High in data matrix of
Benedetto \citeyearpar{Benedetto2009iChaniella}, this taxon
\citep[see][fig. 508]{Williams2000LinguliformeaCraniiformea} does not
express the deeply conical ventral valve that this character is intended
to reflect. It is charitably coded as ambiguous.

\hypertarget{Clupeafumosus_socialis-coding-40}{}
\emph{Clupeafumosus socialis}: Entire valve length -- see schematic in
Williams \emph{et al}. \citeyearpar{Williams1997Introduction}, fig. 286.

\hypertarget{Coolinia_pecten-coding-40}{}
\emph{Coolinia pecten}: See fig. 485 in
\citet{Williams2000LinguliformeaCraniiformea}.

\hypertarget{Gasconsia-coding-40}{}
\emph{Gasconsia}: ``ventral cardinal interarea low, apsacline, with
narrow, poorly defined homeodeltidium'' --
\citet{Williams2000LinguliformeaCraniiformea}, p.~186.

\hypertarget{Kutorgina_chengjiangensis-coding-40}{}
\emph{Kutorgina chengjiangensis}: This taxon
\citetext{\citealp[see][fig.
129]{Williams2000LinguliformeaCraniiformea}; \citealp[fig.
1]{Popov1992TheCambrian}} comes close to expressing the deeply conical
ventral valve that this character is intended to reflect, though this is
not always so pronounced \citep[e.g.][fig.
125]{Williams2000LinguliformeaCraniiformea}. It is therefore coded as
ambiguous.

\hypertarget{Mickwitzia_muralensis-coding-40}{}
\emph{Mickwitzia muralensis}: Often not prominently high
\citep{Skovsted2003EarlyCambrian, Balthasar2004Shellstructure}, though
in some cases \citep[e.g.][]{Butler2015Exceptionallypreserved} the
ventral valve approaches the conical shape that this character is
intended to capture. Coded as polymorphic.

\hypertarget{Nisusia_sulcata-coding-40}{}
\emph{Nisusia sulcata}: Scored as high in data matrix of Benedetto
\citeyearpar{Benedetto2009iChaniella}, and depicted as such in Williams
\emph{et al}. \citeyearpar[fig.
125]{Williams2000LinguliformeaCraniiformea} and Popov \citeyearpar[fig.
1]{Popov1992TheCambrian}; but not high in all specimens
\citep[e.g.][fig. 126]{Williams2000LinguliformeaCraniiformea}. It is
therefore coded as polymorphic.

\hypertarget{Orthis-coding-40}{}
\emph{Orthis}: Scored `Low' for \emph{Eoorthis} by Benedetto
\citeyearpar{Benedetto2009iChaniella}; assumed same in \emph{Orthis}.

\hypertarget{Salanygolina-coding-40}{}
\emph{Salanygolina}: Whereas Williams \emph{et al}.
\citeyearpar[p.~156]{Williams2000LinguliformeaCraniiformea} describe the
ventral pseudointerarea as high, the shell lacks the deeply conical
aspect that this character is intended to capture; we thus code the
taxon as ambiguous.

\subsection*{{[}41{]} Posterior surface:
Delthyrium}\label{posterior-surface-delthyrium}
\addcontentsline{toc}{subsection}{{[}41{]} Posterior surface:
Delthyrium}

\includegraphics{Brachiopod_phylogeny_files/figure-latex/character-mapping-41.pdf}

\begin{quote}
\textbf{Character 41: Sclerites: Ventral valve: Posterior surface:
Delthyrium}

0: Absent\\
1: Present\\
Neomorphic character.
\end{quote}

A delthyrium is an opening in an interarea or pseudointerarea that
accommodates the pedicle, and may be filled with plates.

The homology of the pedicle in the pseudointerarea of obolellids and
botsfordiids with the umbonal pedicle foramen of acrotretids was
proposed by Popov \citeyearpar{Popov1992TheCambrian}, and seemingly
corroborated by observations of Ushatinskaya \& Korovnikov
\citeyearpar{Ushatinskaya2016Revisionof}, who note that the propareas of
the \emph{Botsfordia} ventral valve sometimes merge to form an elongate
teardrop-shaped pedicle foramen.

\hypertarget{Acanthotretella_spinosa-coding-41}{}
\emph{Acanthotretella spinosa}: Origin modelled on \emph{Siphonobolus}.

\hypertarget{Askepasma_toddense-coding-41}{}
\emph{Askepasma toddense}: Homeodeltidium absent
\citep[p.~153]{Williams2000LinguliformeaCraniiformea}; deltidium is open
\citep[see][fig. 4]{Topper2013Theoldest}.

\hypertarget{Botsfordia-coding-41}{}
\emph{Botsfordia}: The homology of the triangular notch or groove in the
pseudointerarea with the umbonal pedicle foramen of acrotretids was
proposed by Popov \citeyearpar{Popov1992TheCambrian}, and seemingly
corroborated by observations of Ushatinskaya \& Korovnikov
\citeyearpar{Ushatinskaya2016Revisionof}, who note that the propareas of
the \emph{Botsfordia} ventral valve sometimes merge to form an elongate
teardrop-shaped pedicle foramen.

\hypertarget{Clupeafumosus_socialis-coding-41}{}
\emph{Clupeafumosus socialis}: Following Popov
\citeyearpar{Popov1992TheCambrian}, the larval delthyrium is sealed in
adults by outgrowths of the posterolateral margins of the shell.

\hypertarget{Eoobolus-coding-41}{}
\emph{Eoobolus}: See for example fig. 5 in
\citet{Balthasar2009Thebrachiopod}.

\hypertarget{Glyptoria-coding-41}{}
\emph{Glyptoria}: ``Delthyrium and notothyrium open, wide'' --
\citet{Cooper1976LowerCambrian}.

\hypertarget{Longtancunella_chengjiangensis-coding-41}{}
\emph{Longtancunella chengjiangensis}: Unclear: a narrow ridge that may
correspond to a pseudodeltidium evident in fig 2a and sketched in fig.
2c is not discussed in the text of \citet{Zhang2011Theexceptionally}, so
the delthyrial region is coded as ambiguous.

\hypertarget{Mickwitzia_muralensis-coding-41}{}
\emph{Mickwitzia muralensis}: A delthyrium is present in young
individuals \citep{Balthasar2004Shellstructure}.

\hypertarget{Micrina-coding-41}{}
\emph{Micrina}: Opening inferred by Holmer \emph{et al}.
\citeyearpar{Holmer2008TheEarly}.

\hypertarget{Pelagodiscus_atlanticus-coding-41}{}
\emph{Pelagodiscus atlanticus}: The listrum (pedicle opening) is
interpreted as originating via a similar mechanism to that of
acrotretids \citep{Popov1992TheCambrian}, and hence corresponding to a
basally sealed delthyrium.

\hypertarget{Siphonobolus_priscus-coding-41}{}
\emph{Siphonobolus priscus}: Ontogeny presumed to resemble that of
acrotretids.

\hypertarget{Yuganotheca_elegans-coding-41}{}
\emph{Yuganotheca elegans}: Details of the hinge region are unclear due
to the flattened and overprinted nature of fossil preservation.

\subsection*{{[}42{]} Posterior surface: Delthyrium:
Shape}\label{posterior-surface-delthyrium-shape}
\addcontentsline{toc}{subsection}{{[}42{]} Posterior surface:
Delthyrium: Shape}

\includegraphics{Brachiopod_phylogeny_files/figure-latex/character-mapping-42.pdf}

\begin{quote}
\textbf{Character 42: Sclerites: Ventral valve: Posterior surface:
Delthyrium: Shape}

1: Parallel sided\\
2: Triangular\\
3: Round\\
Transformational character.
\end{quote}

A parallel-sided delthyrium links \emph{Mummpikia} with the Obolellidae
\citep{Balthasar2008iMummpikia}.

Following Popov \citeyearpar{Popov1992TheCambrian}, the larval
delthyrium of acrotretids and allied taxa is understood to be sealed in
adults by outgrowths of the posterolateral margins of the shell. The
resultant round or teardrop-shaped foramen corresponds the delthyrium.

\hypertarget{Askepasma_toddense-coding-42}{}
\emph{Askepasma toddense}: Prominently triangular \citep[see][fig.
2]{Topper2013Theoldest}.

\hypertarget{Clupeafumosus_socialis-coding-42}{}
\emph{Clupeafumosus socialis}: Following the model of Popov
\citeyearpar{Popov1992TheCambrian}.

\hypertarget{Mickwitzia_muralensis-coding-42}{}
\emph{Mickwitzia muralensis}: An opening is incorporated at the base of
the homeodeltidium when the organism switches from early to late
maturity \citep[fig. 10 in][]{Balthasar2004Shellstructure}. This opening
is conceivably homologous with the pedicle foramen of acrotretid
brachiopods and their ilk. To reflect this possible homology,
\emph{Mickwitzia} is coded as polymorphic (triangular/round).

\subsection*{{[}43{]} Posterior surface: Delthyrium: Shape: Aspect of
rounded
opening}\label{posterior-surface-delthyrium-shape-aspect-of-rounded-opening}
\addcontentsline{toc}{subsection}{{[}43{]} Posterior surface:
Delthyrium: Shape: Aspect of rounded opening}

\includegraphics{Brachiopod_phylogeny_files/figure-latex/character-mapping-43.pdf}

\begin{quote}
\textbf{Character 43: Sclerites: Ventral valve: Posterior surface:
Delthyrium: Shape: Aspect of rounded opening}

1: Elongate: oval to rhombic\\
2: Essentially circular\\
3: Wider than long\\
Transformational character.
\end{quote}

Chen \emph{et al}. \citeyearpar{Chen2007Reinterpretationof} propose that
an oval to rhombic foramen characterises the discinids {[}and
\emph{Heliomedusa}, though the foramen in this taxon has since been
reinterpreted by Zhang \emph{et al}.
\citeyearpar{Zhang2009Architectureand} as an impression of internal
tissue{]}.

\hypertarget{Lingulellotreta_malongensis-coding-43}{}
\emph{Lingulellotreta malongensis}: Oval
\citep{Williams2000LinguliformeaCraniiformea}.

\hypertarget{Mickwitzia_muralensis-coding-43}{}
\emph{Mickwitzia muralensis}: Wider than long: see fig. 10 in
\citet{Balthasar2004Shellstructure}.

\subsection*{{[}44{]} Posterior surface: Delthyrium:
Cover}\label{posterior-surface-delthyrium-cover}
\addcontentsline{toc}{subsection}{{[}44{]} Posterior surface:
Delthyrium: Cover}

\includegraphics{Brachiopod_phylogeny_files/figure-latex/character-mapping-44.pdf}

\begin{quote}
\textbf{Character 44: Sclerites: Ventral valve: Posterior surface:
Delthyrium: Cover}

1: Open\\
2: Covered, at least in part\\
Transformational character.
\end{quote}

An open delthyrium links \emph{Mummpikia} with the Obolellidae
\citep{Balthasar2008iMummpikia}.

The delthyrial opening can be covered by one or more deltidial plates,
or a pseudodeltitium.

Inapplicable in taxa with a round delthiruym \citep[generated by
overgrowth of the delthyrial opening by posterolateral parts of the
shell, per][]{Popov1992TheCambrian}.

\hypertarget{Askepasma_toddense-coding-44}{}
\emph{Askepasma toddense}: Open \citep{Topper2013Theoldest}.

\hypertarget{Botsfordia-coding-44}{}
\emph{Botsfordia}: See pl. 3 fig. 15 in Skovsted \& Holmer
\citeyearpar{Skovsted2005EarlyCambrian}.

\hypertarget{Coolinia_pecten-coding-44}{}
\emph{Coolinia pecten}: A convex pseudodeltidium completely covers the
delthyrium in \emph{Coolinia}.

\hypertarget{Glyptoria-coding-44}{}
\emph{Glyptoria}: Coded as open by Williams \emph{et al}.
\citeyearpar{Williams1998Thediversity}.

\hypertarget{Nisusia_sulcata-coding-44}{}
\emph{Nisusia sulcata}: ``Covered only apically by a small convex
pseudodeltitium'' -- \citet{Holmer2018Evolutionarysignificance}.

\hypertarget{Paterimitra-coding-44}{}
\emph{Paterimitra}: Covered by subaical flange, in part.

\subsection*{{[}45{]} Posterior surface: Delthyrium: Cover:
Extent}\label{posterior-surface-delthyrium-cover-extent}
\addcontentsline{toc}{subsection}{{[}45{]} Posterior surface:
Delthyrium: Cover: Extent}

\includegraphics{Brachiopod_phylogeny_files/figure-latex/character-mapping-45.pdf}

\begin{quote}
\textbf{Character 45: Sclerites: Ventral valve: Posterior surface:
Delthyrium: Cover: Extent}

1: Covered only partially; partially open\\
2: Completely covered\\
Transformational character.
\end{quote}

\hypertarget{Micrina-coding-45}{}
\emph{Micrina}: Remains somewhat open.

\hypertarget{Nisusia_sulcata-coding-45}{}
\emph{Nisusia sulcata}: A well-defined pseudo-deltidium {[}\ldots{}{]}
closes only the apical part of\\
the delthyrium \citep{Rowell1985Theevolutionary}.

\subsection*{{[}46{]} Posterior surface: Delthyrium: Cover:
Identity}\label{posterior-surface-delthyrium-cover-identity}
\addcontentsline{toc}{subsection}{{[}46{]} Posterior surface:
Delthyrium: Cover: Identity}

\includegraphics{Brachiopod_phylogeny_files/figure-latex/character-mapping-46.pdf}

\begin{quote}
\textbf{Character 46: Sclerites: Ventral valve: Posterior surface:
Delthyrium: Cover: Identity}

1: Pseudodeltidium\\
2: Deltidial plate(s)\\
Transformational character.
\end{quote}

This character has the capacity for further resolution (one or more
deltidial plates), but this is unlikely to affect the results of the
present study.

The pseudodelthyrium is also referred to as a homeodeltidium.

\hypertarget{Alisina-coding-46}{}
\emph{Alisina}: Stated as ``concave pseudodeltidium with median
plication'' -- \citet{Williams2000LinguliformeaCraniiformea}\\
Coded as ``Pseudodeltidium: Covered by concave plate'' by Bassett
\emph{et al}. \citeyearpar{Bassett2001Functionalmorphology}.

\hypertarget{Askepasma_toddense-coding-46}{}
\emph{Askepasma toddense}: No pseudodeltidium
\citep[p.~153]{Williams2000LinguliformeaCraniiformea}.

\hypertarget{Lingulellotreta_malongensis-coding-46}{}
\emph{Lingulellotreta malongensis}: The subapical flange of the
\emph{Paterimitra} S1 sclerite has been homologised with the ventral
homeodeltidium of \emph{Micromitra} \citep{Larsson2014iPaterimitra}.

\hypertarget{Mickwitzia_muralensis-coding-46}{}
\emph{Mickwitzia muralensis}: Termed a homoedeltidium by Balthasar
\citeyearpar{Balthasar2004Shellstructure}.

\hypertarget{Micrina-coding-46}{}
\emph{Micrina}: ``Ventral valve convex with apsacline interarea bearing
delthyrium, covered by a convex pseudodeltidium'' --
\citet{Holmer2008TheEarly}.

\subsection*{{[}47{]} Posterior surface: Delthyrium: Pseudodeltidium:
Shape}\label{posterior-surface-delthyrium-pseudodeltidium-shape}
\addcontentsline{toc}{subsection}{{[}47{]} Posterior surface:
Delthyrium: Pseudodeltidium: Shape}

\includegraphics{Brachiopod_phylogeny_files/figure-latex/character-mapping-47.pdf}

\begin{quote}
\textbf{Character 47: Sclerites: Ventral valve: Posterior surface:
Delthyrium: Pseudodeltidium: Shape}

1: Concave\\
2: Convex\\
Transformational character.
\end{quote}

A ridge-like (i.e.~convex) pseudodeltitium unites \emph{Salanygolina}
with \emph{Coolinia} and other Chileata
\citep[p.~6]{Holmer2009Theenigmatic}.

\hypertarget{Alisina-coding-47}{}
\emph{Alisina}: ``concave pseudodeltidium with median plication'' --
\citet{Williams2000LinguliformeaCraniiformea}\\
Coded as ``Pseudodeltidium: Covered by concave plate'' by Bassett
\emph{et al}. \citeyearpar{Bassett2001Functionalmorphology}.

\hypertarget{Antigonambonites_planus-coding-47}{}
\emph{Antigonambonites planus}: Convex \citep[fig.
508]{Williams2000LinguliformeaCraniiformea}.

\hypertarget{Gasconsia-coding-47}{}
\emph{Gasconsia}: \emph{Gasconsia} possesses narrow concave
homeodeltidium, but absent pseudodeltidium.

\hypertarget{Kutorgina_chengjiangensis-coding-47}{}
\emph{Kutorgina chengjiangensis}: Difficult to determine based on
material presented in Zhang \emph{et al}.
\citeyearpar{Zhang2007Rhynchonelliformeanbrachiopods}, or indeed for
other species in the genus
\citep[e.g.][]{Williams2000LinguliformeaCraniiformea, Skovsted2005EarlyCambrian, Holmer2018Theattachment}.

\hypertarget{Mickwitzia_muralensis-coding-47}{}
\emph{Mickwitzia muralensis}: Convex \citep[see][fig.
4B]{Balthasar2004Shellstructure}.

\hypertarget{Micrina-coding-47}{}
\emph{Micrina}: Convex deltoid \citep{Holmer2008TheEarly}.

\hypertarget{Micromitra-coding-47}{}
\emph{Micromitra}: Gently convex \citep[see][fig.
83.3]{Williams2000LinguliformeaCraniiformea}.

\hypertarget{Nisusia_sulcata-coding-47}{}
\emph{Nisusia sulcata}: Convex in \emph{Nisusia} \citep[see][fig.
8.4]{Rowell1985Theevolutionary}.

\hypertarget{Paterimitra-coding-47}{}
\emph{Paterimitra}: Gently convex \citep[see][fig.
83.1]{Williams2000LinguliformeaCraniiformea}.

\hypertarget{Salanygolina-coding-47}{}
\emph{Salanygolina}: ``The presence of {[}\ldots{}{]} a narrow
delthyrium covered by a convex pseudodeltidium, places Salanygolinidae
outside the Class Paterinata and strongly suggests affinity to the
Cambrian Chileida'' -- \citet{Holmer2009Theenigmatic}, p.~9.

\hypertarget{Tomteluva_perturbata-coding-47}{}
\emph{Tomteluva perturbata}: Convex \citep{Streng2016Anew}.

\subsection*{{[}48{]} Posterior surface: Delthyrium: Pseudodeltidium:
Hinge
furrows}\label{posterior-surface-delthyrium-pseudodeltidium-hinge-furrows}
\addcontentsline{toc}{subsection}{{[}48{]} Posterior surface:
Delthyrium: Pseudodeltidium: Hinge furrows}

\includegraphics{Brachiopod_phylogeny_files/figure-latex/character-mapping-48.pdf}

\begin{quote}
\textbf{Character 48: Sclerites: Ventral valve: Posterior surface:
Delthyrium: Pseudodeltidium: Hinge furrows}

0: Absent\\
1: Present\\
Neomorphic character.
\end{quote}

After Bassett \emph{et al}.
\citeyearpar{Bassett2001Functionalmorphology} character 18, ``Hinge
furrows on lateral sides of pseudodeltidium''.

\hypertarget{Acanthotretella_spinosa-coding-48}{}
\emph{Acanthotretella spinosa}, \emph{Askepasma toddense},
\emph{Clupeafumosus socialis}, \emph{Dailyatia}, \emph{Eccentrotheca},
\emph{Glyptoria}, \emph{Haplophrentis carinatus}, \emph{Heliomedusa
orienta}, \emph{Lingula}, \emph{Lingulellotreta malongensis},
\emph{Lingulosacculus}, \emph{Micrina}, \emph{Micromitra},
\emph{Mummpikia nuda}, \emph{Novocrania}, \emph{Orthis},
\emph{Paterimitra}, \emph{Pedunculotheca diania}, \emph{Pelagodiscus
atlanticus}, \emph{Phoronis}, \emph{Terebratulina}: Absent due to
inapplicability of neomorphic character.

\hypertarget{Salanygolina-coding-48}{}
\emph{Salanygolina}: The presence of this feature is impossible to
determine based on the available material.

\subsection*{{[}49{]} Umbonal perforation}\label{umbonal-perforation}
\addcontentsline{toc}{subsection}{{[}49{]} Umbonal perforation}

\includegraphics{Brachiopod_phylogeny_files/figure-latex/character-mapping-49.pdf}

\begin{quote}
\textbf{Character 49: Sclerites: Ventral valve: Umbonal perforation}

0: Umbo imperforate (or ventral valve absent)\\
1: Umbonal perforation\\
Neomorphic character.
\end{quote}

Certain taxa, particularly those with a colleplax, exhibit a perforation
at the umbo of the ventral valve. This opening is sometimes associated
with a pedicle sheath, which emerges from the umbo of the ventral valve
without any indication of a relationship with the hinge.

In contrast, the pedicle of acrotretids and similar brachiopods is
situated on the larval hinge line, but is later surrounded by the
posterolateral regions of the growing shell to become separated from the
hinge line, and encapsulated in a position close to (or with resorption
of the brephic shell, at) the umbo \citep[see][pp.~407--411 and fig. 3
for discussion]{Popov1992TheCambrian}. In some cases, an internal
pedicle tube attests to this origin -- potentially corresponding to the
pedicle groove of lingulids. As such, the pedicle foramen of acrotretids
and allies is not originally situated at the umbo; it is instead
understood to represent a basally sealed delthyrium.

\hypertarget{Clupeafumosus_socialis-coding-49}{}
\emph{Clupeafumosus socialis}: The presumed pedicle foramen reported by
Topper \emph{et al}. \citeyearpar{Topper2013Reappraisalof} is at the
ventral valve umbo. No evidence of an internal pedicle tube is present,
but we follow Popov \citeyearpar{Popov1992TheCambrian} in inferring the
encapsulation of the pedicle foramen.

\hypertarget{Dailyatia-coding-49}{}
\emph{Dailyatia}: The B and C sclerites of \emph{Dailyatia} bear small
umbonal perforations \citep{Skovsted2015Theearly}, but these are not
considered to be homologous with the ventral valve, so this character is
coded as inapplicable -- though the possibility that the perforations
are equivalent is intriguing.

A1 sclerites typically have a pair of perforations, which are
conceivably equivalent to the setal tubes of \emph{Micrina}
\citep{Holmer2011Firstrecord}. The A1 sclerite of D. bacata has a
structure that is arguably similar to the `colleplax' of
\emph{Paterimitra}. But the homology of any of these structures to the
umbonal aperture of brachiopods is difficult to establish.

\hypertarget{Eccentrotheca-coding-49}{}
\emph{Eccentrotheca}: The sclerites of \emph{Eccentrotheca} form a ring
that surrounds the inferred attachment structure; the attachment
structure does not emerge from an aperture within an individual
sclerite. Thus no feature in \emph{Eccentrotheca} is judged to be
potentially homologous with the apical perforation in bivalved
brachiopods.

\hypertarget{Heliomedusa_orienta-coding-49}{}
\emph{Heliomedusa orienta}: There is ``compelling evidence to
demonstrate that the putative pedicle\\
illustrated by Chen \emph{et al}. \citeyearpar[Figs. 4, 6,
7]{Chen2007Reinterpretationof} in fact is the mold of a
three-dimensionally preserved visceral cavity.'' --
\citet{Zhang2009Architectureand}.

\hypertarget{Lingulosacculus-coding-49}{}
\emph{Lingulosacculus}: The apical termination of the fossil is unknown.

\hypertarget{Mickwitzia_muralensis-coding-49}{}
\emph{Mickwitzia muralensis}: The umbo itself is imperforate
\citep{Balthasar2004Shellstructure}.

\hypertarget{Paterimitra-coding-49}{}
\emph{Paterimitra}: The presumed pedicle foramen is an opening between
the S1 and S2 sclerites, neither of which are perforated
\citep{Skovsted2009Thescleritome}.

\hypertarget{Siphonobolus_priscus-coding-49}{}
\emph{Siphonobolus priscus}: Prominent subcircular perforation at umbo
associated with an internal pedicle tube \citep{Popov2009Earlyontogeny},
thus presumed to share an origin with the acrotretid pedicle foramen.

\hypertarget{Tomteluva_perturbata-coding-49}{}
\emph{Tomteluva perturbata}: Streng \emph{et al}.
\citeyearpar{Streng2016Anew} observe ``an internal tubular structure
probably representing the ventral end of the canal within the posterior
wall of the pedicle tube'', but do not consider this tomteluvid dube to
be homologous with the pedicle tube of acrotretids and their ilk,
stating (p.~274) that it appears to be unique within Brachiopoda.

\subsection*{{[}50{]} Umbonal perforation:
Shape}\label{umbonal-perforation-shape}
\addcontentsline{toc}{subsection}{{[}50{]} Umbonal perforation: Shape}

\includegraphics{Brachiopod_phylogeny_files/figure-latex/character-mapping-50.pdf}

\begin{quote}
\textbf{Character 50: Sclerites: Ventral valve: Umbonal perforation:
Shape}

1: Circular (or subcircular)\\
2: Rhombic to oval\\
Transformational character.
\end{quote}

\hypertarget{Acanthotretella_spinosa-coding-50}{}
\emph{Acanthotretella spinosa}: Too small to observe given quality of
preservation \citep{Holmer2006Aspinose}.

\hypertarget{Alisina-coding-50}{}
\emph{Alisina}: Seemingly circular \citep{Zhang2011Anobolellate}.

\hypertarget{Antigonambonites_planus-coding-50}{}
\emph{Antigonambonites planus}: Based on p.92, fig.4B.

\hypertarget{Clupeafumosus_socialis-coding-50}{}
\emph{Clupeafumosus socialis}: Taller than wide in some cases, but very
nearly circular in others; see Topper \emph{et al}.
\citeyearpar{Topper2013Reappraisalof}.

\hypertarget{Coolinia_pecten-coding-50}{}
\emph{Coolinia pecten}: Bassett and Popov write ``a dominant feature of
the ventral umbo is a sub-oval perforation about 270 µm long and 250 µm
wide'': the width and height of this structure are almost identical, and
we score it as (sub) circular.

\hypertarget{Heliomedusa_orienta-coding-50}{}
\emph{Heliomedusa orienta}: Rhombic to oval -- seen as evidence for a
discinid affinity \citep{Chen2007Reinterpretationof}.

\hypertarget{Kutorgina_chengjiangensis-coding-50}{}
\emph{Kutorgina chengjiangensis}: The exact size and shape of the apical
perforation is obscured by the emerging pedicle.

\hypertarget{Nisusia_sulcata-coding-50}{}
\emph{Nisusia sulcata}: ``close to circular''
\citep{Holmer2018Evolutionarysignificance}.

\hypertarget{Salanygolina-coding-50}{}
\emph{Salanygolina}: Essentially circular \citep[fig.
4]{Holmer2009Theenigmatic}.

\subsection*{{[}51{]} Colleplax, cicatrix or pedicle
sheath}\label{colleplax-cicatrix-or-pedicle-sheath}
\addcontentsline{toc}{subsection}{{[}51{]} Colleplax, cicatrix or
pedicle sheath}

\includegraphics{Brachiopod_phylogeny_files/figure-latex/character-mapping-51.pdf}

\begin{quote}
\textbf{Character 51: Sclerites: Ventral valve: Colleplax, cicatrix or
pedicle sheath}

0: Absent\\
1: Present\\
Neomorphic character.
\end{quote}

In certain taxa, the umbo of the ventral valve bears a colleplax,
cicatrix or pedicle sheath; Bassett \emph{et al}.
\citeyearpar{Bassett2008Earlyontogeny} consider these structures as
homologous.

\hypertarget{Botsfordia-coding-51}{}
\emph{Botsfordia}: Following \citet{Williams1998Thediversity}, appendix
2.

\hypertarget{Clupeafumosus_socialis-coding-51}{}
\emph{Clupeafumosus socialis}: Not reported by Topper \emph{et al}.
\citeyearpar{Topper2013Reappraisalof}.

\hypertarget{Craniops-coding-51}{}
\emph{Craniops}: \emph{Paracraniops} is ``externally similar to
\emph{Craniops}, but lacking cicatrix'' -- indicating that
\emph{Craniops} bears a cicatrix
\citep{Williams2000LinguliformeaCraniiformea}. Also coded as present in
their table 15.

\hypertarget{Heliomedusa_orienta-coding-51}{}
\emph{Heliomedusa orienta}: A cicatrix was reconstructed by
\citet{Jin1992Revisionof} (figs 6b, 7), but has not been reported by
later authors; possibly, as with the `pedicle foramen' of Chen \emph{et
al}. \citeyearpar{Chen2007Reinterpretationof}, this structure represents
internal organs rather than a cicatrix proper
\citep{Zhang2009Architectureand}; as such it has been recorded as
ambiguous.

\hypertarget{Kutorgina_chengjiangensis-coding-51}{}
\emph{Kutorgina chengjiangensis}: The umbonal region of kutorginides
``clearly lacks a pedicle sheath'' \citep{Holmer2018Theattachment}.

\hypertarget{Lingulellotreta_malongensis-coding-51}{}
\emph{Lingulellotreta malongensis}: The pedicle is identified as such
(rather than a pedicle sheath) by the internal pedicle tube.

\hypertarget{Longtancunella_chengjiangensis-coding-51}{}
\emph{Longtancunella chengjiangensis}: A ring-like structure surrounding
the pedicle is interpreted as a colleplax
\citep{Zhang2011Theexceptionally}, though the authors make no comparison
with the pedicle capsule exhibited by extant terebratulids
\citep[see][]{Holmer2018Evolutionarysignificance}.

\hypertarget{Micrina-coding-51}{}
\emph{Micrina}: Absent in \emph{Micrina} \citep{Holmer2011Firstrecord}.

\hypertarget{Pedunculotheca_diania-coding-51}{}
\emph{Pedunculotheca diania}: The flat apical termination of juvenile
individuals possibly functioned as colleplax for attachment, but may
simply represent the brephic shell; we treat it as ambiguous to reflect
this potential homology.

\hypertarget{Siphonobolus_priscus-coding-51}{}
\emph{Siphonobolus priscus}: Coded as present in view of the attachment
scar, which has been considered homologous with the ``adult colleplax
and foramen with attachment pad'' in \emph{Salanygolina}
\citep{Popov2009Earlyontogeny}.

\hypertarget{Tomteluva_perturbata-coding-51}{}
\emph{Tomteluva perturbata}: The internal canal associated with the
pedicle is unique to the tomteluvids, and has an uncertain identity
\citep{Streng2016Anew}. It could conceivably correspond to an
internalized pedicle sheath or an equivalent structure, so this feature
is coded as ambiguous here.

\hypertarget{Ussunia-coding-51}{}
\emph{Ussunia}: Following table 15 in
\citet{Williams2000LinguliformeaCraniiformea}.

\hypertarget{Yuganotheca_elegans-coding-51}{}
\emph{Yuganotheca elegans}: The median collar or conical tube is
conceivably homologous with the pedicle sheath.

\subsection*{{[}52{]} Median septum}\label{median-septum}
\addcontentsline{toc}{subsection}{{[}52{]} Median septum}

\includegraphics{Brachiopod_phylogeny_files/figure-latex/character-mapping-52.pdf}

\begin{quote}
\textbf{Character 52: Sclerites: Ventral valve: Median septum}

0: Absent\\
1: Present\\
Neomorphic character.
\end{quote}

Chen \emph{et al}. \citeyearpar{Chen2007Reinterpretationof} observe a
median septum in what they interpret as the ventral valve of
\emph{Heliomedusa}, and the ventral valve of \emph{Discinisca}, which
they propose points to a close relationship.

\hypertarget{Acanthotretella_spinosa-coding-52}{}
\emph{Acanthotretella spinosa}: Carbonaceous preservation confounds the
identification of internal shell structures; it is possible that this
feature is present, but not observable in the Burgess Shale material.

\hypertarget{Botsfordia-coding-52}{}
\emph{Botsfordia}: Following \citet{Williams1998Thediversity}, appendix
2.

\hypertarget{Clupeafumosus_socialis-coding-52}{}
\emph{Clupeafumosus socialis}: A short medial ridge (septum) is present
in the ventral valve \citep{Topper2013Reappraisalof}.

\hypertarget{Eoobolus-coding-52}{}
\emph{Eoobolus}: Prominent median septum \citep[fig. 4d, e
in][]{Balthasar2009Thebrachiopod}.

\hypertarget{Gasconsia-coding-52}{}
\emph{Gasconsia}: Evident in moulds of ventral valve; see Watkins
\citeyearpar{Watkins2002Newrecord}.

\hypertarget{Glyptoria-coding-52}{}
\emph{Glyptoria}: Neither evident nor reported in Williams \emph{et al}.
\citeyearpar{Williams2000LinguliformeaCraniiformea}.

\hypertarget{Haplophrentis_carinatus-coding-52}{}
\emph{Haplophrentis carinatus}: The carina of \emph{H. carinatus} is an
angular elevation of the ventral valve surface, rather than a septum
growing inward on the interior of shell.

\hypertarget{Heliomedusa_orienta-coding-52}{}
\emph{Heliomedusa orienta}: Reported on `ventral' valve by Chen \emph{et
al}. \citeyearpar{Chen2007Reinterpretationof}; we consider the `ventral'
valve to be the dorsal valve.

\hypertarget{Lingulellotreta_malongensis-coding-52}{}
\emph{Lingulellotreta malongensis}: Medial septum visible in ventral
valve in Williams \emph{et al}.
\citeyearpar{Williams2000LinguliformeaCraniiformea}, fig. 34.1c.

\hypertarget{Micromitra-coding-52}{}
\emph{Micromitra}: Ventral ridge characteristic of \emph{Micromitra}
\citep{Skovsted2010EarlyCambrian}.

\hypertarget{Mummpikia_nuda-coding-52}{}
\emph{Mummpikia nuda}: ``Some specimens also reveal that the vault had a
slight median septum, which is now visible as a notch or a groove
dividing the right from the left part'' --
\citet{Balthasar2008iMummpikia}.

\hypertarget{Novocrania-coding-52}{}
\emph{Novocrania}: Valve thin and often unmineralized.

\hypertarget{Pelagodiscus_atlanticus-coding-52}{}
\emph{Pelagodiscus atlanticus}: Described as present in
\emph{Discinisca} by \citet{Chen2007Reinterpretationof}; assumed present
also in \emph{Pelagodiscus}.

\hypertarget{Siphonobolus_priscus-coding-52}{}
\emph{Siphonobolus priscus}: Present; see
\citet{Popov2009Earlyontogeny}, fig. 5J.

\hypertarget{Ussunia-coding-52}{}
\emph{Ussunia}: Following char. 42 in table 15 in
\citet{Williams2000LinguliformeaCraniiformea}.

\section{Sclerites: Ornament}\label{sclerites-ornament}

\subsection*{{[}53{]} Concentric ornament}\label{concentric-ornament}
\addcontentsline{toc}{subsection}{{[}53{]} Concentric ornament}

\includegraphics{Brachiopod_phylogeny_files/figure-latex/character-mapping-53.pdf}

\begin{quote}
\textbf{Character 53: Sclerites: Ornament: Concentric ornament}

0: Smooth, or growth lines only\\
1: Concentric ornament present\\
Neomorphic character.
\end{quote}

After character 11 in Williams \emph{et al}.
\citeyearpar{Williams1998Thediversity}.

\hypertarget{Askepasma_toddense-coding-53}{}
\emph{Askepasma toddense}, \emph{Glyptoria}, \emph{Kutorgina
chengjiangensis}, \emph{Micromitra}, \emph{Salanygolina}: Following
appendix 2 in Williams \emph{et al}.
\citeyearpar{Williams1998Thediversity}.

\hypertarget{Botsfordia-coding-53}{}
\emph{Botsfordia}: Following \citet{Williams1998Thediversity}, appendix
2.\\
Pustules are arranged along concentric growth lines
\citep{Skovsted2005EarlyCambrian}, so are not treated as a distinct
ornamentation.

\hypertarget{Cotyledion_tylodes-coding-53}{}
\emph{Cotyledion tylodes}: \citet{Zhang2013}.

\hypertarget{Eccentrotheca-coding-53}{}
\emph{Eccentrotheca}: More or less concentric ridges occur on
\emph{Eccentrotheca} sclerites
\citep{Skovsted2011Scleritomeconstruction}.

\hypertarget{Halkieria_evangelista-coding-53}{}
\emph{Halkieria evangelista}: Ridges in shell parallel, but are more
prominent than, growth lines.

\hypertarget{Haplophrentis_carinatus-coding-53}{}
\emph{Haplophrentis carinatus}: A series of regularly spaced concentric
ridges adorn both valves \citep{Moysiuk2017Hyolithsare}; these are more
pronounced than mere growth lines.

\hypertarget{Heliomedusa_orienta-coding-53}{}
\emph{Heliomedusa orienta}: The ornament on shell exterior is described
as concentric fila \citep[P.43]{Chen2007Reinterpretationof}, and also
scored as it in Williams \emph{et al}.
\citeyearpar[pp.160--163]{Williams2000LinguliformeaCraniiformea}.

\hypertarget{Mickwitzia_muralensis-coding-53}{}
\emph{Mickwitzia muralensis}: Symmetric fila.

\hypertarget{Novocrania-coding-53}{}
\emph{Novocrania}: Irregular ridges externally
\citep{Williams2000LinguliformeaCraniiformea}.

\hypertarget{Pedunculotheca_diania-coding-53}{}
\emph{Pedunculotheca diania}: A series of regularly spaced concentric
ridges adorn the ventral valve; comparatively less regular lines
ornament the operculum.

\hypertarget{Pelagodiscus_atlanticus-coding-53}{}
\emph{Pelagodiscus atlanticus}: Only growth lines evident
\citep{Williams2000LinguliformeaCraniiformea}.

\hypertarget{Terebratulina-coding-53}{}
\emph{Terebratulina}: Single ridge evident in Williams \emph{et al}.
\citeyearpar{Williams2006Rhynchonelliformeapart} fig. 1425.1a
interpreted as interruption ot growth rather than inherent feature, so
coded as absent (i.e.~smooth).

\subsection*{{[}54{]} Concentric ornament:
Symmetry}\label{concentric-ornament-symmetry}
\addcontentsline{toc}{subsection}{{[}54{]} Concentric ornament:
Symmetry}

\includegraphics{Brachiopod_phylogeny_files/figure-latex/character-mapping-54.pdf}

\begin{quote}
\textbf{Character 54: Sclerites: Ornament: Concentric ornament:
Symmetry}

0: Symmetric fila\\
1: Asymmetric fila, with outer faces\\
Neomorphic character.
\end{quote}

After character 11 in Williams \emph{et al}.
\citeyearpar{Williams1998Thediversity}.

\hypertarget{Alisina-coding-54}{}
\emph{Alisina}: Seemingly asymmetric \citetext{\citealp[fig.
122.3c]{Williams2000LinguliformeaCraniiformea}; \citealp[Fig.
1]{Zhang2011Anobolellate}}.

\hypertarget{Askepasma_toddense-coding-54}{}
\emph{Askepasma toddense}, \emph{Glyptoria}, \emph{Kutorgina
chengjiangensis}, \emph{Micromitra}, \emph{Salanygolina}: Following
appendix 2 in Williams \emph{et al}.
\citeyearpar{Williams1998Thediversity}.

\hypertarget{Dailyatia-coding-54}{}
\emph{Dailyatia}: Clear asymmetry \citep{Skovsted2015Theearly}.

\hypertarget{Eccentrotheca-coding-54}{}
\emph{Eccentrotheca}: Ornament, such as it is, is asymmetric, with
prominent outer faces \citep{Skovsted2011Scleritomeconstruction}.

\hypertarget{Heliomedusa_orienta-coding-54}{}
\emph{Heliomedusa orienta}: See fig. 1715 in Williams \emph{et al}.
\citeyearpar{Williams2007Supplement}.

\hypertarget{Mickwitzia_muralensis-coding-54}{}
\emph{Mickwitzia muralensis}: Symmetric fila
\citep{Balthasar2004Shellstructure}.

\hypertarget{Micrina-coding-54}{}
\emph{Micrina}: No obvious asymmetry, even if not obviously symmetric
either \citep{Holmer2008TheEarly}. Coded as ambiguous.

\hypertarget{Novocrania-coding-54}{}
\emph{Novocrania}: Clear outer faces \citep[fig.
100.2b]{Williams2000LinguliformeaCraniiformea}.

\subsection*{{[}55{]} Radial ornament}\label{radial-ornament}
\addcontentsline{toc}{subsection}{{[}55{]} Radial ornament}

\includegraphics{Brachiopod_phylogeny_files/figure-latex/character-mapping-55.pdf}

\begin{quote}
\textbf{Character 55: Sclerites: Ornament: Radial ornament}

0: Absent\\
1: Present\\
Neomorphic character.
\end{quote}

Ridges radiating from umbo, i.e.~ribs.

\hypertarget{Askepasma_toddense-coding-55}{}
\emph{Askepasma toddense}: ``Ornament of irregularly developed,
concentric growth lamellae; microornament of irregularly arranged,
polygonal pits'' -- \citet{Williams2000LinguliformeaCraniiformea}, p153;
figs on p.155.

\hypertarget{Botsfordia-coding-55}{}
\emph{Botsfordia}: Following \citet{Williams1998Thediversity}, Appendix
2.

\hypertarget{Eoobolus-coding-55}{}
\emph{Eoobolus}: Very faint costellae in some specimens but coded
absent.

\hypertarget{Gasconsia-coding-55}{}
\emph{Gasconsia}: ``Ornament of indistinct low radial ribs'' -- Williams
\emph{et al}. \citeyearpar[p167]{Williams2000LinguliformeaCraniiformea}.

\hypertarget{Glyptoria-coding-55}{}
\emph{Glyptoria}: ``Coarsely costate'' -- Williams \emph{et al}.
\citeyearpar[p710]{Williams2000LinguliformeaCraniiformea}.

\hypertarget{Heliomedusa_orienta-coding-55}{}
\emph{Heliomedusa orienta}: See fig. 1715 in Williams \emph{et al}.
\citeyearpar{Williams2007Supplement}.

\hypertarget{Siphonobolus_priscus-coding-55}{}
\emph{Siphonobolus priscus}: ``Indistinct radial ribs accentuated by
radial rows of tubercles'' -- \citet{Popov2009Earlyontogeny}.

\hypertarget{Ussunia-coding-55}{}
\emph{Ussunia}: Unornamented.

\subsection*{{[}56{]} Shell-penetrating
spines}\label{shell-penetrating-spines}
\addcontentsline{toc}{subsection}{{[}56{]} Shell-penetrating spines}

\includegraphics{Brachiopod_phylogeny_files/figure-latex/character-mapping-56.pdf}

\begin{quote}
\textbf{Character 56: Sclerites: Ornament: Shell-penetrating spines}

0: Absent\\
1: Present\\
Neomorphic character.
\end{quote}

Mineralized or partly mineralized spines are observed in
\emph{Heliomedusa} and \emph{Acanthotretella}.

\hypertarget{Chiton-coding-56}{}
\emph{Chiton}: Aesthete canals penetrate the main valves of certain
chitons, but are not equivalent to the shell-penetrating spines of
brachiopods.

\hypertarget{Glyptoria-coding-56}{}
\emph{Glyptoria}: Neither evident nor reported in Williams \emph{et al}.
\citeyearpar{Williams2000LinguliformeaCraniiformea}.

\hypertarget{Heliomedusa_orienta-coding-56}{}
\emph{Heliomedusa orienta}: The `spines' reported by Chen \emph{et al}.
\citeyearpar{Chen2007Reinterpretationof} are pyritized spinelike\\
setae -- see pp.~2580--2590 in Williams \emph{et al}.
\citeyearpar{Williams2007Supplement}.

\hypertarget{Nisusia_sulcata-coding-56}{}
\emph{Nisusia sulcata}: Bears numerous small, hollow spines
\citep{Williams2000LinguliformeaCraniiformea}.

\section{Sclerites: Composition}\label{sclerites-composition}

\subsection*{{[}57{]} Mineralogy}\label{mineralogy}
\addcontentsline{toc}{subsection}{{[}57{]} Mineralogy}

\includegraphics{Brachiopod_phylogeny_files/figure-latex/character-mapping-57.pdf}

\begin{quote}
\textbf{Character 57: Sclerites: Composition: Mineralogy}

1: Organic (non-mineralized)\\
2: Phosphatic\\
3: Calcitic\\
4: Aragonitic\\
Transformational character.
\end{quote}

\hypertarget{Acanthotretella_spinosa-coding-57}{}
\emph{Acanthotretella spinosa}: Holmer \& Caron
\citeyearpar{Holmer2006Aspinose} note the absence of brittle breakage,
interpreted as indicating the absence of a material mineralized
component to the shells. The preservation is strikingly different from
that of other Burgess Shale brachiopods, ruling out a primarily calcitic
or phosphatic composition. The two-dimensional nature of the
preservation also differs from that of co-occurring aragonitic taxa
\citep[hyoliths;][ p.~273]{Holmer2006Aspinose}, indicating that any
mineralization was minor at best.

Holmer \& Caron \citeyearpar[p.~286]{Holmer2006Aspinose} suggest that it
is more likely that a (minor) mineral component was present than that it
was not, though without providing an uncontestable rationale. To be as
conservative as possible, we therefore code this taxon as ambiguous.

\hypertarget{Clupeafumosus_socialis-coding-57}{}
\emph{Clupeafumosus socialis}: Phosphatic -- hence the conventional
placement within Linguliformea.

\hypertarget{Cotyledion_tylodes-coding-57}{}
\emph{Cotyledion tylodes}: The extensive relief and association with
pyrite framboids indicates original mineralization, but the identity of
the biomineral remains uncertain \citep{Zhang2013}.

\hypertarget{Craniops-coding-57}{}
\emph{Craniops}: Shell calcitic.

\hypertarget{Eoobolus-coding-57}{}
\emph{Eoobolus}: ``the original shell of \emph{Eoobolus} contained small
calcareous grains that were incorporated into organic-rich layers
alongside apatite'' \citep{Balthasar2007Anearly}.

\hypertarget{Gasconsia-coding-57}{}
\emph{Gasconsia}: Confirmed in Trimerella by
\citet{Balthasar2011Relicaragonite}.

\hypertarget{Heliomedusa_orienta-coding-57}{}
\emph{Heliomedusa orienta}: ``Shell originally organophosphatic, but may
generally have been poorly mineralized'' --
\citet{Williams2007Supplement} -- cf.~ibid, p.~2889, ``These strong
similarities to discinoids in soft-part anatomy imply that the
\emph{Heliomedusa} shell was chitinous or chitinophosphatic, not
calcareous.''

\hypertarget{Lingulellotreta_malongensis-coding-57}{}
\emph{Lingulellotreta malongensis}: Coded as phosphatic by Zhang
\emph{et al}. \citeyearpar{Zhang2014Anearly}, but with no explanation.\\
Cracks within shells of Chengjiang specimens \citep[e.g.][fig.
3]{Zhang2007Noteon} demonstrate that the shells were originally
mineralized, but not the identity of the original biomineral. This said,
phosphatized material from Kazakhstan \citep{Holmer1997EarlyCambrian} is
attributed to the same species; presuming this phosphate to be original
and the material to be conspecific, \emph{L. malongensis} is coded as
having phosphatic shells.

\hypertarget{Lingulosacculus-coding-57}{}
\emph{Lingulosacculus}: The absence of relief in \emph{Lingulosacculus}
rules out a phosphatic or calcitic composition, but co-occurring (and
presumably aragonitic) hyolithids are preserved in the same fashion. Its
constitution was thus either organic or aragonitic
\citep{Balthasar2009EarlyCambrian}.

\hypertarget{Longtancunella_chengjiangensis-coding-57}{}
\emph{Longtancunella chengjiangensis}: ``The original composition of the
shell cannot be determined with certainty'', though it was ``most
probably entirely soft and organic'' --
\citet{Zhang2011Theexceptionally}.

\hypertarget{Mickwitzia_muralensis-coding-57}{}
\emph{Mickwitzia muralensis}: Calcite and silica deemed diagenetic by
Balthasar \citeyearpar{Balthasar2004Shellstructure}.

\hypertarget{Mummpikia_nuda-coding-57}{}
\emph{Mummpikia nuda}: Identified as calcareous by preservational
criteria, and description ``primary\\
calcitic shells of \emph{M. nuda}'' \citep{Balthasar2008iMummpikia}.

\hypertarget{Novocrania-coding-57}{}
\emph{Novocrania}: Ventral valve uncalcified in extant forms or
sometimes thin \citep{Williams2000LinguliformeaCraniiformea}, but coded
as calcitic as calcite-mineralizing pathways are present.

\hypertarget{Salanygolina-coding-57}{}
\emph{Salanygolina}: Original mineralogy unknown, but known to be
mineralised and anticipated to be phosphatic
\citep{Holmer2009Theenigmatic}.

\subsection*{{[}58{]} Cuticle or organic
matrix}\label{cuticle-or-organic-matrix}
\addcontentsline{toc}{subsection}{{[}58{]} Cuticle or organic matrix}

\includegraphics{Brachiopod_phylogeny_files/figure-latex/character-mapping-58.pdf}

\begin{quote}
\textbf{Character 58: Sclerites: Composition: Cuticle or organic matrix}

1: GAGs, chitin and collagen\\
2: Glycoprotein\\
Transformational character.
\end{quote}

Williams \emph{et al}. \citeyearpar{Williams1996Asupra} identify
glycoprotein-based organic scaffolds as distinct from those comprising
glycosaminoglycans (GAGs), chitin and collagen. This character can only
be scored for extant taxa.

\hypertarget{Lingula-coding-58}{}
\emph{Lingula}: Coded as GAGs, chitin and collagen in lingulids by
Williams \emph{et al}. \citeyearpar{Williams1996Asupra}.

\hypertarget{Novocrania-coding-58}{}
\emph{Novocrania}: Coded as glycoprotein for craniids by Williams
\emph{et al}. \citeyearpar{Williams1996Asupra}.

\hypertarget{Pelagodiscus_atlanticus-coding-58}{}
\emph{Pelagodiscus atlanticus}: Coded as GAGs, chitin and collagen in
discinids by Williams \emph{et al}. \citeyearpar{Williams1996Asupra}.

\hypertarget{Phoronis-coding-58}{}
\emph{Phoronis}: ``The presence of sulphated glycosaminoglycans (GAGs)
in the chitinous cuticle of \emph{Phoronis}
\citep[p.~215]{Herrmann1997Phoronida} would suggest a link with
linguliforms, as GAGs are unknown in rhynchonelliform shells (Fig. 1891,
1896)'' -- \citet{Williams2007Supplement}, p.~2830.

\hypertarget{Siphonobolus_priscus-coding-58}{}
\emph{Siphonobolus priscus}: Lenticular chambers in siphonotretid shells
interpreted as degraded GAG residue
\citep{Williams2004Chemicostructure}.

\hypertarget{Terebratulina-coding-58}{}
\emph{Terebratulina}: Coded as glycoprotein for terebratulids by
Williams \emph{et al}. \citeyearpar{Williams1996Asupra}.

\subsection*{{[}59{]} Incorporation of sedimentary
particles}\label{incorporation-of-sedimentary-particles}
\addcontentsline{toc}{subsection}{{[}59{]} Incorporation of sedimentary
particles}

\includegraphics{Brachiopod_phylogeny_files/figure-latex/character-mapping-59.pdf}

\begin{quote}
\textbf{Character 59: Sclerites: Composition: Incorporation of
sedimentary particles}

0: Absent\\
1: Present\\
Neomorphic character.
\end{quote}

Phoronids and \emph{Yuganotheca} aggulutinate particles into their
sclerites.

\subsection*{{[}60{]} Periostracum:
Flexibility}\label{periostracum-flexibility}
\addcontentsline{toc}{subsection}{{[}60{]} Periostracum: Flexibility}

\includegraphics{Brachiopod_phylogeny_files/figure-latex/character-mapping-60.pdf}

\begin{quote}
\textbf{Character 60: Sclerites: Composition: Periostracum: Flexibility}

1: Flexible\\
2: Inflexible\\
Transformational character.
\end{quote}

Following character 9 in Williams \emph{et al}.
\citeyearpar{Williams1998Thediversity}; see their p228--230 for a
discussion of how this might be inferred from fossil material.

\hypertarget{Askepasma_toddense-coding-60}{}
\emph{Askepasma toddense}, \emph{Glyptoria}, \emph{Kutorgina
chengjiangensis}, \emph{Micromitra}: Following appendix 2 in Williams
\emph{et al}. \citeyearpar{Williams1998Thediversity}.

\hypertarget{Botsfordia-coding-60}{}
\emph{Botsfordia}, \emph{Eoobolus}: Coded as flexible in
\citet{Williams1998Thediversity}, Appendix 2.

\hypertarget{Pelagodiscus_atlanticus-coding-60}{}
\emph{Pelagodiscus atlanticus}: Flexible
\citep{Williams1998Thediversity}.

\hypertarget{Salanygolina-coding-60}{}
\emph{Salanygolina}: Coded as uncertain in appendix 2 in Williams
\emph{et al}. \citeyearpar{Williams1998Thediversity}.

\subsection*{{[}61{]} Microstructure:
Layers}\label{microstructure-layers}
\addcontentsline{toc}{subsection}{{[}61{]} Microstructure: Layers}

\includegraphics{Brachiopod_phylogeny_files/figure-latex/character-mapping-61.pdf}

\begin{quote}
\textbf{Character 61: Sclerites: Composition: Microstructure: Layers}

1: Single microstructural layer\\
2: Two microstructurally differentiated layers\\
3: Inner and outer laminae enclosing medial void\\
4: Three microstrurally differentiated layers\\
Transformational character.
\end{quote}

Hyolith conchs comprise two mineralized layers of fibrous bundles.
Bundles are measure 5--15 µm across; their constituent fibres are each
0.1--1.0 µm wide. In the inner layer, the fibres are transverse; in the
outer layer, the bundles are inclined towards the umbo, becoming
longitudinal on the outermost margin.

Obolellids comprise a single laminated mineralogical layer
\citep{Balthasar2008iMummpikia}. Shell-penetrating canals are not
considered as contributing to the mineralogical microstructure and are
coded separately.

Coded as non-additive as there is no clear necessity to pass through the
brachiopod-like construction: the three layers could arise by the
addition of a void to a single pre-existing layer, for example.

Inapplicable in taxa with a non-mineralized shell.

\hypertarget{Botsfordia-coding-61}{}
\emph{Botsfordia}: ``Composed of a thin primary layer and a laminate
secondary shell exhibiting baculate shell structure'' -- Skovsted \&
Holmer \citeyearpar{Skovsted2005EarlyCambrian}, with reference to
\citet{Skovsted2003EarlyCambrian}.

\hypertarget{Chiton-coding-61}{}
\emph{Chiton}: From periostracum inwards, \emph{Chiton} bears three
microstructural layers: fine-grained, nacreous, and regular crossed
lamellar.

\hypertarget{Clupeafumosus_socialis-coding-61}{}
\emph{Clupeafumosus socialis}: General acrotretid structure taken from
Zhang \emph{et al}. \citeyearpar{Zhang2016Epithelialcell}.

\hypertarget{Eoobolus-coding-61}{}
\emph{Eoobolus}: ``\emph{Eoobolus} shells exhibit the general
characteristics of modern linguliform shells, i.e.~they were composed of
alternating sets of organic and apatite-rich layers that were separated
by thin sheets of recalcitrant organic layers.'' --
\citet{Balthasar2007Anearly}.

\hypertarget{Halkieria_evangelista-coding-61}{}
\emph{Halkieria evangelista}: Single layer of fibrous aragonite
\citep{Porter2008}.

\hypertarget{Haplophrentis_carinatus-coding-61}{}
\emph{Haplophrentis carinatus}: Assumed to be equivalent to the hyoliths
described by Kouchinsky
\citeyearpar{Kouchinsky2000Skeletalmicrostructures}.

\hypertarget{Mickwitzia_muralensis-coding-61}{}
\emph{Mickwitzia muralensis}: ``the shell structure of \emph{Mickwitzia}
{[}\ldots{}{]} is closely similar to the columnar shell of linguliform
acrotretoid brachiopods as well as to the linguloid
\emph{Lingulellotreta}, in that it has slender columns in the laminar
succession'' -- \citet{Williams2007Supplement}.

\hypertarget{Micrina-coding-61}{}
\emph{Micrina}: Identical to \emph{Mickwitzia} and more derived
linguliforms \citep{Holmer2011Firstrecord}.

\hypertarget{Namacalathus-coding-61}{}
\emph{Namacalathus}: \emph{Namacalathus} exhibits three layers, none of
which have any obvious correspondence with those of brachiopods.

\hypertarget{Siphonobolus_priscus-coding-61}{}
\emph{Siphonobolus priscus}: ``Orthodoxly secreted primary and secondary
layers'' -- \citet{Williams2004Chemicostructure}.

\subsection*{{[}62{]} Microstructure: Crystal
format}\label{microstructure-crystal-format}
\addcontentsline{toc}{subsection}{{[}62{]} Microstructure: Crystal
format}

\includegraphics{Brachiopod_phylogeny_files/figure-latex/character-mapping-62.pdf}

\begin{quote}
\textbf{Character 62: Sclerites: Composition: Microstructure: Crystal
format}

1: Laminated\\
2: Fibrous bundles\\
3: Polygonal columns\\
4: Nacreous / crossed lamellar\\
Transformational character.
\end{quote}

Hyolith conchs comprise two mineralized layers of fibrous bundles.
Bundles measure 5--15 μm across; their constituent fibres are each
0.1--1.0 µm wide. In the inner layer, the fibres are transverse; in the
outer layer, the bundles are inclined towards the umbo, becoming
longitudinal on the outermost margin.

Obolellids comprise a single laminated mineralogical layer
\citep{Balthasar2008iMummpikia}. Shell-penetrating canals are not
considered as contributing to the mineralogical microstructure and are
coded separately.

The pervasive (not just superficial) polygonal structures in
\emph{Paterimitra} are distinct, and characterize \emph{Askepasma},
\emph{Salanygolina}, \emph{Eccentrotheca} and \emph{Paterimitra}
\citep{Larsson2014iPaterimitra}

Williams \emph{et al}.
\citeyearpar{Williams2000LinguliformeaCraniiformea} identify
cross-bladed laminae as diagnostic of Strophomenata, with the exception
of some older groups that contain fibres or laminar laths.

\hypertarget{Antigonambonites_planus-coding-62}{}
\emph{Antigonambonites planus}: Shell structure of this taxon is
laminated, rather than fibrous as previously considered.

\hypertarget{Botsfordia-coding-62}{}
\emph{Botsfordia}: ``Composed of a thin primary layer and a laminate
secondary shell exhibiting baculate shell structure'' -- Skovsted \&
Holmer \citeyearpar{Skovsted2005EarlyCambrian}, with reference to
\citet{Skovsted2003EarlyCambrian}.

\hypertarget{Craniops-coding-62}{}
\emph{Craniops}: ``with calcitic or possibly aragonitic inarticulated
shells with laminar (tabular) secondary layers''
\citep{Williams2000LinguliformeaCraniiformea}.

\hypertarget{Haplophrentis_carinatus-coding-62}{}
\emph{Haplophrentis carinatus}: Inferred from other hyolithids
\citep[e.g.][]{Moore2018Plywoodlike}.

\hypertarget{Namacalathus-coding-62}{}
\emph{Namacalathus}: The inner and outer layer are foliated. The
columnar inflections lack canals, and as such we do not consider them to
bear any obvious homology with the hollow pillars of tommotiids and
certain brachiopods, their superficial similarity to strophomenid
pseudopunctae notwithstanding.

\hypertarget{Pedunculotheca_diania-coding-62}{}
\emph{Pedunculotheca diania}: Assumed to be fibrous by analogy with the
allothecomorph orthothecid described by Kouchinsky
\citeyearpar{Kouchinsky2000Skeletalmicrostructures}.

\hypertarget{Siphonobolus_priscus-coding-62}{}
\emph{Siphonobolus priscus}: Prominent laminations; see Williams
\emph{et al}. \citeyearpar{Williams2004Chemicostructure}.

\subsection*{{[}63{]} Microstructure:
Punctae}\label{microstructure-punctae}
\addcontentsline{toc}{subsection}{{[}63{]} Microstructure: Punctae}

\includegraphics{Brachiopod_phylogeny_files/figure-latex/character-mapping-63.pdf}

\begin{quote}
\textbf{Character 63: Sclerites: Composition: Microstructure: Punctae}

0: Absent\\
1: Present\\
Neomorphic character.
\end{quote}

Punctae are 10--20 µm wide canals created by multicellular extensions of
the outer epithelium. They penetrate the full depth of the shell.

Balthasar \citeyearpar{Balthasar2008iMummpikia} writes:

``Vertical shell penetrating structures, such as punctae, pseudopunctae,
extropunctae and canals, are common in many groups of brachiopods and
are distinguished based on their geometry and size
\citep{Williams1997Introduction}. Punctae are 10--20 µm wide and
represent multicellular extensions of the outer epithelium
\citep{Owen1969Thecaecum}. Pseudopunctae and extropunctae are similar in
diameter but, instead of canals, are vertical stacks of conical
deflections of individual shell layers \citep{Williams1993Roleof}. None
of these three types of vertical shell structure, all of which are
confined to calcitic-shelled brachiopods, compares with the much smaller
canals (\textless{} 1 µm in diameter) of \emph{M. nuda}. The only type
of vertical structure that fits the size and nature of the canals of the
Mural obolellids are the canals of linguliform brachiopods, which range
in width from 180 to 740 nm and are occupied by proteinaceous strands in
extant taxa
\citep{Williams1992Structureof, Williams1994Collagenouschitino, Williams1997Introduction}.
In contrast to obolellid canals, however, linguliform canals are not
known to penetrate the entire shell but terminate in organic-rich layers
\citep{Williams1997Introduction}. Based on these considerations it
would, therefore, be misleading to call obolellid shells punctate (they
are as much''punctate" as acrotretids or other linguliforms); rather
their shell structure should be called canaliculate
\citep{Williams1997Introduction}."

\hypertarget{Craniops-coding-63}{}
\emph{Craniops}: ``impunctate''.

\hypertarget{Haplophrentis_carinatus-coding-63}{}
\emph{Haplophrentis carinatus}: The tubules within the centre of the
bundles of hyolith shells \citep{Kouchinsky2000Skeletalmicrostructures}
are c. 10 µm wide, making them an order of magnitude larger than the
canals that characterize lingulid valves, and a similar scale to
punctae. This said, they have only been reported in a putative
allathecid, so the presence of equivalent structures in hyolithids has
never been demonstrated.

\hypertarget{Heliomedusa_orienta-coding-63}{}
\emph{Heliomedusa orienta}: `Identical' to those in \emph{Mickwitzia} --
see \citet{Williams2007Supplement}.

\hypertarget{Mickwitzia_muralensis-coding-63}{}
\emph{Mickwitzia muralensis}: Coded as present to reflect that the
chambers contained setae; following Carlson in
\citet{Williams2007Supplement}, the punctae may or may not be homologous
as punctae, but are likely homologous as shell perforations; both these
perforations and those of \emph{Micrina} were associated with setae,
even if their equivalence bay be with juvenile vs adult setal structures
in modern brachiopods \citep[p.~397]{Balthasar2004Shellstructure}.

\hypertarget{Mummpikia_nuda-coding-63}{}
\emph{Mummpikia nuda}: ``Vertical shell penetrating structures, such as
punctae, pseudopunctae, extropunctae and canals, are common in many
groups of brachiopods and are distinguished based on their geometry and
size \citep{Williams1997Introduction}. Punctae are 10--20 µm wide and
represent multicellular extensions of the outer epithelium
\citep{Owen1969Thecaecum}. {[}\ldots{}{]} None of these three types of
vertical shell structure, all of which are confined to calcitic-shelled
brachiopods, compares with the much smaller canals (\textless{} 1 µm in
diameter) of \emph{M. nuda}. The only type of vertical structure that
fits the size and nature of the canals of the Mural obolellids are the
canals of linguliform brachiopods, which range in width from 180 to 740
nm and are occupied by proteinaceous strands in extant taxa
\citetext{\citealp[1994]{Williams1992Structureof}; \citealp{Williams1997Introduction}}.''
-- \citet{Balthasar2008iMummpikia}.

\hypertarget{Siphonobolus_priscus-coding-63}{}
\emph{Siphonobolus priscus}: The `canals' through the shell have a
diameter of c. 20 µm \citep[text-fig. 2a]{Williams2004Chemicostructure},
falling within the definition of punctae used herein.

\hypertarget{Terebratulina-coding-63}{}
\emph{Terebratulina}: Endopunctae are relatively large canals, diameter
vary greatly from 5--20 m.

\subsection*{{[}64{]} Microstructure:
Canals}\label{microstructure-canals}
\addcontentsline{toc}{subsection}{{[}64{]} Microstructure: Canals}

\includegraphics{Brachiopod_phylogeny_files/figure-latex/character-mapping-64.pdf}

\begin{quote}
\textbf{Character 64: Sclerites: Composition: Microstructure: Canals}

0: Absent\\
1: Present\\
Neomorphic character.
\end{quote}

A caniculate microstructure occurs in lingulids; canals are narrower
(\textless{} 1 µm) than punctae, may branch, and do not fully penetrate
the shell, terminating just within the boundaries of a microstructural
layer. See \citet{Williams1997Introduction}, p303ff, and
\citet{Balthasar2008iMummpikia}, p273, for discussion.

Tubules described in hyoliths by Kouchinsky
\citeyearpar{Kouchinsky2000Skeletalmicrostructures} measure around 10 µm
in diameter, making them an order of magnitude wider than lingulid
canals.

This said, Balthasar \citeyearpar{Balthasar2008iMummpikia} considers the
tubules within the columnar shell microstructure of \emph{Mickwitzia}
cf.~occidens \citep[1--3 µm wide,][]{Skovsted2003EarlyCambrian},
acrotretides \citep[1 µm wide,
see][\citet{Zhang2016Epithelialcell}]{Holmer1989MiddleOrdovician} and
lingulellotretids \citep[100 nm wide,][]{Cusack1999Chemicostructural} as
equivalent to lingulid canals.

\emph{Micrina} exhibits both punctae and canals
\citep{Harper2017Brachiopodsorigin}, challenging Carlson's contention
\citep[in][]{Williams2007Supplement} that the structures are potentially
homologous as shell perforations.

\hypertarget{Botsfordia-coding-64}{}
\emph{Botsfordia}: Not evident in section presented by Skovsted \&
Holmer \citeyearpar{Skovsted2003EarlyCambrian}.

\hypertarget{Chiton-coding-64}{}
\emph{Chiton}: Aesthete canals do not fall within the definition of this
character.

\hypertarget{Clupeafumosus_socialis-coding-64}{}
\emph{Clupeafumosus socialis}: Acrotretid laminae bear characteristic
columns \citep[e.g.][]{Zhang2016Epithelialcell}.

Balthasar \citeyearpar{Balthasar2008iMummpikia} considers these columns
as homologous with tubules within the columnar shell microstructure
\emph{Mummpikia}, \emph{Mickwitzia} and lingulellotretids.

\hypertarget{Halkieria_evangelista-coding-64}{}
\emph{Halkieria evangelista}: The chambers in halkieriid sclerites do
not correspond in morphology or dimension to the brachiopod-like canals
documented by this character.

\hypertarget{Haplophrentis_carinatus-coding-64}{}
\emph{Haplophrentis carinatus}: Zhang \emph{et al}.
\citeyearpar{Zhang2018Ahyolithid} have reported um-scale canals,
replicated in phosphate, within the shell of the hyolithid
\emph{Paramicrocornus}; as shell microstructure is not preserved in
\emph{Haplophrentis}, this latter taxon is taken as a model.

\hypertarget{Longtancunella_chengjiangensis-coding-64}{}
\emph{Longtancunella chengjiangensis}: Preservational resolution not
sufficient to evaluate.

\hypertarget{Mickwitzia_muralensis-coding-64}{}
\emph{Mickwitzia muralensis}: Coded as present to reflect similarity of
columnar microstructure remarked on by, among others, Balthasar
\citeyearpar{Balthasar2008iMummpikia}; Williams \emph{et al}.
\citeyearpar{Williams2007Supplement}; Skovsted \& Holmer
\citeyearpar{Skovsted2003EarlyCambrian}.

\hypertarget{Micrina-coding-64}{}
\emph{Micrina}: Acrotretid laminae bear characteristic columns
\citep[e.g.][]{Zhang2016Epithelialcell}; a similar fabric has been
reported, and assumed homologous, in \emph{Micrina}
\citep{Butler2012ConstructingCambrian}.

A similar columnar shell microstructure also occurs in the closely
related \emph{Mickwitzia} \citep{Balthasar2008iMummpikia}.

\hypertarget{Namacalathus-coding-64}{}
\emph{Namacalathus}: Canal-like structures have been reported in
\emph{Namacalathus} \citep{Zhuravlev2015Ediacaranskeletal}, and
interpreted as evidence for a Lophophorate affinity. Though the
structures are not necessarily directly equivalent, the hypothesis of
homology is followed here.

\hypertarget{Siphonobolus_priscus-coding-64}{}
\emph{Siphonobolus priscus}: The `canals' through the shell have a
diameter of c. 20 µm \citep[text-fig. 2a]{Williams2004Chemicostructure},
falling within the definition of punctae (rather than canals) used
herein.

\subsection*{{[}65{]} Microstructure:
Pseudopunctae}\label{microstructure-pseudopunctae}
\addcontentsline{toc}{subsection}{{[}65{]} Microstructure:
Pseudopunctae}

\includegraphics{Brachiopod_phylogeny_files/figure-latex/character-mapping-65.pdf}

\begin{quote}
\textbf{Character 65: Sclerites: Composition: Microstructure:
Pseudopunctae}

0: Absent\\
1: Present\\
Neomorphic character.
\end{quote}

Pseudopunctae are not punctae, but deflections of shell laminae. They
characterise Strophomenata in particular.

\hypertarget{Antigonambonites_planus-coding-65}{}
\emph{Antigonambonites planus}, \emph{Glyptoria}, \emph{Nisusia
sulcata}: Scored absent in data matrix of Benedetto
\citeyearpar{Benedetto2009iChaniella}.

\hypertarget{Orthis-coding-65}{}
\emph{Orthis}: Scored absent (in \emph{Eoorthis}) in data matrix of
Benedetto \citeyearpar{Benedetto2009iChaniella}.

\subsection*{{[}66{]} Microstructure: External polygonal
ornament}\label{microstructure-external-polygonal-ornament}
\addcontentsline{toc}{subsection}{{[}66{]} Microstructure: External
polygonal ornament}

\includegraphics{Brachiopod_phylogeny_files/figure-latex/character-mapping-66.pdf}

\begin{quote}
\textbf{Character 66: Sclerites: Composition: Microstructure: External
polygonal ornament}

0: Absent\\
1: Present\\
Neomorphic character.
\end{quote}

Regular polygonal compartments, around 10 µm in diameter, characterise
\emph{Paterimitra}. Walls between compartments have the cross-section of
an anvil. An external polygonal structure (possible imprints of
epithelial tissue) occurs in \emph{Dailyatia}, but it is a surface
pattern, which is different from the polygonal prisms in the body wall
of other paterinid-like groups.

\hypertarget{Clupeafumosus_socialis-coding-66}{}
\emph{Clupeafumosus socialis}: The polygonal ornament reported in
acrotretids by Zhang \emph{et al}. \citeyearpar{Zhang2016Epithelialcell}
is on the internal surface of the shell.

\section{Sclerites}\label{sclerites-1}

\subsection*{{[}67{]} Periodically shed and
replaced}\label{periodically-shed-and-replaced}
\addcontentsline{toc}{subsection}{{[}67{]} Periodically shed and
replaced}

\includegraphics{Brachiopod_phylogeny_files/figure-latex/character-mapping-67.pdf}

\begin{quote}
\textbf{Character 67: Sclerites: Periodically shed and replaced}

0: Absent\\
1: Present\\
Neomorphic character.
\end{quote}

Certain taxa periodically slough and replace some of their individual
sclerites during growth.

\section{Gametes}\label{gametes}

\subsection*{{[}68{]} Egg size}\label{egg-size}
\addcontentsline{toc}{subsection}{{[}68{]} Egg size}

\includegraphics{Brachiopod_phylogeny_files/figure-latex/character-mapping-68.pdf}

\begin{quote}
\textbf{Character 68: Gametes: Egg size}

1: Small: \textless{} 100 um, little yolk\\
2: Large: \textgreater{} 110 um, much yolk\\
Transformational character.
\end{quote}

Following Carlson \citeyearpar{Carlson1995Phylogeneticrelationships},
character 7. This character is only possible to code in extant taxa. It
is not considered independent of Carlson's character 11, number of
gametes released per spawning, as it is possible to produce more small
eggs than large eggs -- thus this latter character is not reproduced in
the present study. The same goes for Carlson's character 12, gamete
dispersal mode; brooders will tend to brood large eggs.

\hypertarget{Chiton-coding-68}{}
\emph{Chiton}: \citet{BucklandNicks1988}.

\hypertarget{Dentalium-coding-68}{}
\emph{Dentalium}: Egg size can vary from 60--200 µm in scaphopods, but
in \emph{Dentalium} the eggs are large \citep{DufresneDube1983}.

\hypertarget{Flustra-coding-68}{}
\emph{Flustra}: ``Mature eggs commonly measure about 200 µm in
diameter'' --.

\hypertarget{Lingula-coding-68}{}
\emph{Lingula}, \emph{Novocrania}, \emph{Pelagodiscus atlanticus},
\emph{Terebratulina}: Following coding for class in Carlson
\citeyearpar{Carlson1995Phylogeneticrelationships} appendix 1, character
7.

\hypertarget{Loxosomella-coding-68}{}
\emph{Loxosomella}: Tiny \citep{Nielsen1966}.

\hypertarget{Phoronis-coding-68}{}
\emph{Phoronis}: \emph{Phoronis} has planktotrophic larvae. indicating a
small egg size \citep{Ruppert2004Invertebratezoology}. Carlson
\citeyearpar{Carlson1995Phylogeneticrelationships} codes phoronids as
polymorphic, as some members of the phylum have eggs of each size.

\hypertarget{Serpula-coding-68}{}
\emph{Serpula}: c. 50 µm in \emph{Hydroides} \citep{Miles2007}.

\hypertarget{Siphonobolus_priscus-coding-68}{}
\emph{Siphonobolus priscus}: ``the ventral brephic valve {[}was{]} 50 µm
across, {[}which{]} is close to the known lower limit of the brachiopod
egg size'' -- \citet{Popov2009Earlyontogeny}.

\hypertarget{Sipunculus-coding-68}{}
\emph{Sipunculus}: c. 200 µm in diameter \citep{Rice1988}.

\subsection*{{[}69{]} Gonocoel}\label{gonocoel}
\addcontentsline{toc}{subsection}{{[}69{]} Gonocoel}

\includegraphics{Brachiopod_phylogeny_files/figure-latex/character-mapping-69.pdf}

\begin{quote}
\textbf{Character 69: Gametes: Gonocoel}

0: Absent\\
1: Retroperineal gonads\\
Neomorphic character.
\end{quote}

Character 27 in \citet{Haszprunar1996}.

\subsection*{{[}70{]} Ovary wall saccular}\label{ovary-wall-saccular}
\addcontentsline{toc}{subsection}{{[}70{]} Ovary wall saccular}

\includegraphics{Brachiopod_phylogeny_files/figure-latex/character-mapping-70.pdf}

\begin{quote}
\textbf{Character 70: Gametes: Ovary wall saccular}

0: Plain\\
1: Saccular\\
Neomorphic character.
\end{quote}

After character 31 in \citet{Haszprunar1996}.

\subsection*{{[}71{]} Testis wall saccular}\label{testis-wall-saccular}
\addcontentsline{toc}{subsection}{{[}71{]} Testis wall saccular}

\begin{quote}
\textbf{Character 71: Gametes: Testis wall saccular}

0: Plain\\
1: Saccular\\
Neomorphic character.
\end{quote}

After character 31 in \citet{Haszprunar1996}.

\subsection*{{[}72{]} Asexual reproduction}\label{asexual-reproduction}
\addcontentsline{toc}{subsection}{{[}72{]} Asexual reproduction}

\includegraphics{Brachiopod_phylogeny_files/figure-latex/character-mapping-71.pdf}

\begin{quote}
\textbf{Character 72: Gametes: Asexual reproduction}

0: Never exhibited\\
1: Frequently exhibited\\
Neomorphic character.
\end{quote}

After character 30 in \citet{Haszprunar1996}.

\subsection*{{[}73{]} Fertilization}\label{fertilization}
\addcontentsline{toc}{subsection}{{[}73{]} Fertilization}

\includegraphics{Brachiopod_phylogeny_files/figure-latex/character-mapping-72.pdf}

\begin{quote}
\textbf{Character 73: Gametes: Fertilization}

0: External\\
1: Internal\\
2:\\
Neomorphic character.
\end{quote}

After character 62 in \citet{Haszprunar2000}.

\subsection*{{[}74{]} Sexes}\label{sexes}
\addcontentsline{toc}{subsection}{{[}74{]} Sexes}

\includegraphics{Brachiopod_phylogeny_files/figure-latex/character-mapping-73.pdf}

\begin{quote}
\textbf{Character 74: Gametes: Sexes}

1: Gonochoristic\\
2: Hermaphroditic\\
Transformational character.
\end{quote}

After characters 1.61 and 2.54 in \citet{SPS1996}.

\subsection*{{[}75{]} Protective membrane}\label{protective-membrane}
\addcontentsline{toc}{subsection}{{[}75{]} Protective membrane}

\includegraphics{Brachiopod_phylogeny_files/figure-latex/character-mapping-74.pdf}

\begin{quote}
\textbf{Character 75: Gametes: Egg: Protective membrane}

0: Absent\\
1: Present\\
Neomorphic character.
\end{quote}

After character 4.69 in \citet{SPS1996}.

\hypertarget{Flustra-coding-75}{}
\emph{Flustra}: ``Eggs have a loose consistency and are capable of
changing form'' \citep{Franzen2013}.

\section{Gametes: Site of maturation
{[}76{]}}\label{gametes-site-of-maturation-76}

\includegraphics{Brachiopod_phylogeny_files/figure-latex/character-mapping-75.pdf}

\begin{quote}
\textbf{Character 76: Gametes: Site of maturation}

0: Body cavity\\
1: Mantle canals\\
2: Ovicell\\
Neomorphic character.
\end{quote}

After Carlson \citeyearpar{Carlson1995Phylogeneticrelationships},
character 9. Only possible to code in extant taxa. Mantle canals is
considered the derived state, as it represents a migration from the body
cavity, where gametes are produced.

\hypertarget{Flustra-coding-76}{}
\emph{Flustra}: Ovicell \citep{Franzen2013}.

\hypertarget{Lingula-coding-76}{}
\emph{Lingula}, \emph{Novocrania}, \emph{Pelagodiscus atlanticus},
\emph{Terebratulina}: Following Hodgson \& Reunov
\citeyearpar{Hodgson1994Ultrastructureof}.

\hypertarget{Phoronis-coding-76}{}
\emph{Phoronis}: Following coding for class in Carlson
\citeyearpar{Carlson1995Phylogeneticrelationships} Appendix 1, character
9.

\section{Gametes: Spermatozoa}\label{gametes-spermatozoa}

\subsection*{{[}77{]} Nucleus: Shape}\label{nucleus-shape}
\addcontentsline{toc}{subsection}{{[}77{]} Nucleus: Shape}

\includegraphics{Brachiopod_phylogeny_files/figure-latex/character-mapping-76.pdf}

\begin{quote}
\textbf{Character 77: Gametes: Spermatozoa: Nucleus: Shape}

0: Equant: length comparable to width\\
1: Elongate: length exaggerated relative to width\\
Neomorphic character.
\end{quote}

After character 41 in \citet{Ponder1997}.

\hypertarget{Chiton-coding-77}{}
\emph{Chiton}: Profoundly elongated nucleus \citep{BucklandNicks1988}.

\hypertarget{Dentalium-coding-77}{}
\emph{Dentalium}: Elongate nucleus, 4--6 times longer than wide
\citep{DufresneDube1983}.

\hypertarget{Flustra-coding-77}{}
\emph{Flustra}: Elongate \citep{Franzen1981}.

\hypertarget{Loxosomella-coding-77}{}
\emph{Loxosomella}: Elongate in \emph{Loxosoma} {[}@Franzen 2000{]}.

\hypertarget{Serpula-coding-77}{}
\emph{Serpula}: \citet{Gherardi2011}.

\subsection*{{[}78{]} Anterior nuclear
fossa}\label{anterior-nuclear-fossa}
\addcontentsline{toc}{subsection}{{[}78{]} Anterior nuclear fossa}

\includegraphics{Brachiopod_phylogeny_files/figure-latex/character-mapping-77.pdf}

\begin{quote}
\textbf{Character 78: Gametes: Spermatozoa: Anterior nuclear fossa}

0: Absent\\
1: Present\\
Neomorphic character.
\end{quote}

Following \citet{Smith2012}, after character 160 in \citet{Giribet2002}.
A fossa (latin: ditch) is a dent or impression.

\hypertarget{Chiton-coding-78}{}
\emph{Chiton}: \citet{BucklandNicks1988}.

\hypertarget{Dentalium-coding-78}{}
\emph{Dentalium}: \citet{DufresneDube1983}.

\hypertarget{Flustra-coding-78}{}
\emph{Flustra}: Present \citep[in \emph{Tubulipora};][]{Franzen1984}.

\hypertarget{Loxosomella-coding-78}{}
\emph{Loxosomella}: Present in \emph{Loxosoma} {[}@Franzen 2000{]}.

\hypertarget{Pelagodiscus_atlanticus-coding-78}{}
\emph{Pelagodiscus atlanticus}: Present in \emph{Discinisca}
\emph{tenuis} \citep{Hodgson1994Ultrastructureof}.

\hypertarget{Phoronis-coding-78}{}
\emph{Phoronis}: Nucleus ``almost round''
\citep{Reunov2004Ultrastructuralstudy}.

\hypertarget{Serpula-coding-78}{}
\emph{Serpula}: Absent: subacrosomal space does not impinge on nuclear
envelope \citep{Gherardi2011}.

\hypertarget{Sipunculus-coding-78}{}
\emph{Sipunculus}: Prominent in \emph{Phascolion} \citep{Rice1993}.

\hypertarget{Terebratulina-coding-78}{}
\emph{Terebratulina}: No anterior invagination
\citep{Hodgson1994Ultrastructureof}.

\subsection*{{[}79{]} Acrosome: Shape}\label{acrosome-shape}
\addcontentsline{toc}{subsection}{{[}79{]} Acrosome: Shape}

\includegraphics{Brachiopod_phylogeny_files/figure-latex/character-mapping-78.pdf}

\begin{quote}
\textbf{Character 79: Gametes: Spermatozoa: Acrosome: Shape}

1: Pear-shaped\\
2: Needle-shaped\\
3: Disc-shaped\\
4: Conical\\
Transformational character.
\end{quote}

\hypertarget{Chiton-coding-79}{}
\emph{Chiton}: Elongate: cylindrical to conical
\citep{BucklandNicks1988}.

\hypertarget{Dentalium-coding-79}{}
\emph{Dentalium}: Low conical aspect \citep{DufresneDube1983}.

\hypertarget{Flustra-coding-79}{}
\emph{Flustra}: Conical \citep[in \emph{Tubulipora};][]{Franzen1984}.

\hypertarget{Lingula-coding-79}{}
\emph{Lingula}: Pear-shaped \citep{Fukumoto2003Theacrosome}.

\hypertarget{Loxosomella-coding-79}{}
\emph{Loxosomella}: Conical/cylindrical acrosome-like extension in
\emph{Loxosoma} {[}@Franzen 2000{]}.

\hypertarget{Novocrania-coding-79}{}
\emph{Novocrania}: Needle-shaped \citep{Afzelius1978Finestructure}.

\hypertarget{Pelagodiscus_atlanticus-coding-79}{}
\emph{Pelagodiscus atlanticus}: Pear-shaped
\citep{Hodgson1994Ultrastructureof}.

\hypertarget{Phoronis-coding-79}{}
\emph{Phoronis}: Needle-shaped \citep{Reunov2004Ultrastructuralstudy}.

\hypertarget{Serpula-coding-79}{}
\emph{Serpula}: \citet{Gherardi2011}.

\hypertarget{Sipunculus-coding-79}{}
\emph{Sipunculus}: A peaked disc in \emph{Phascolion} \citep{Rice1993}.

\hypertarget{Terebratulina-coding-79}{}
\emph{Terebratulina}: Disc-shaped (in \emph{Kraussina})
\citep{Hodgson1994Ultrastructureof}.

\subsection*{{[}80{]} Acrosome: Differentiated
internally}\label{acrosome-differentiated-internally}
\addcontentsline{toc}{subsection}{{[}80{]} Acrosome: Differentiated
internally}

\includegraphics{Brachiopod_phylogeny_files/figure-latex/character-mapping-79.pdf}

\begin{quote}
\textbf{Character 80: Gametes: Spermatozoa: Acrosome: Differentiated
internally}

0: No internal differentiation\\
1: Acrosome differentiated internally\\
Neomorphic character.
\end{quote}

\citet{Hodgson1994Ultrastructureof} describe the \emph{Discinisca}
acrosome as having ``an electron-lucent centre and an electron-dense
outer region'', and state that this trait is characteristic of
inarticulate brachiopods.

\hypertarget{Chiton-coding-80}{}
\emph{Chiton}: ``One can distinguish two components in the acrosome, an
apical and a basal granule'' -- \citet{BucklandNicks1988}.

\hypertarget{Dentalium-coding-80}{}
\emph{Dentalium}: Differentiated membrane only \citep{DufresneDube1983}.

\hypertarget{Flustra-coding-80}{}
\emph{Flustra}: No evidence of internal differentiation \citep[in
\emph{Tubulipora};][]{Franzen1984}.

\hypertarget{Lingula-coding-80}{}
\emph{Lingula}: Clear differentiation of marginal area
\citep{Fukumoto2003Theacrosome}.

\hypertarget{Loxosomella-coding-80}{}
\emph{Loxosomella}: Not evident in \emph{Loxosoma} {[}@Franzen 2000{]}.

\hypertarget{Novocrania-coding-80}{}
\emph{Novocrania}: ``Along the inner and outer margins there are
periodically banded layers, and between them there is a less dense
layer'' -- \citet{Afzelius1978Finestructure}.

\hypertarget{Pelagodiscus_atlanticus-coding-80}{}
\emph{Pelagodiscus atlanticus}: Following \emph{Discinisca}
\emph{tenuis}, described in Hodgson \& Reunov
\citeyearpar{Hodgson1994Ultrastructureof}.

\hypertarget{Phoronis-coding-80}{}
\emph{Phoronis}: Acrosome-like structure; no internal division or
surrounding membrane, with possibility that much of the acrosome is
secondarily lost \citep{Reunov2004Ultrastructuralstudy}.

\hypertarget{Serpula-coding-80}{}
\emph{Serpula}: \citet{Gherardi2011}.

\hypertarget{Sipunculus-coding-80}{}
\emph{Sipunculus}: No differentiation within acrosomal vesicle
\citep{Rice1993}.

\hypertarget{Terebratulina-coding-80}{}
\emph{Terebratulina}: Following Hodgson \& Reunov
\citeyearpar{Hodgson1994Ultrastructureof}.

\subsection*{{[}81{]} Acrosome: Sub-acrosomal
space}\label{acrosome-sub-acrosomal-space}
\addcontentsline{toc}{subsection}{{[}81{]} Acrosome: Sub-acrosomal
space}

\includegraphics{Brachiopod_phylogeny_files/figure-latex/character-mapping-80.pdf}

\begin{quote}
\textbf{Character 81: Gametes: Spermatozoa: Acrosome: Sub-acrosomal
space}

0: Absent\\
1: Present\\
Neomorphic character.
\end{quote}

\hypertarget{Chiton-coding-81}{}
\emph{Chiton}: Not evident \citep{BucklandNicks1988}.

\hypertarget{Dentalium-coding-81}{}
\emph{Dentalium}: \citet{DufresneDube1983}.

\hypertarget{Flustra-coding-81}{}
\emph{Flustra}: No distinct space \citep[in
\emph{Tubulipora};][]{Franzen1984}.

\hypertarget{Lingula-coding-81}{}
\emph{Lingula}: Filled with sub-acrosomal substance
\citep{Fukumoto2003Theacrosome}.

\hypertarget{Loxosomella-coding-81}{}
\emph{Loxosomella}: Present in \emph{Loxosoma} {[}@Franzen 2000{]}.

\hypertarget{Novocrania-coding-81}{}
\emph{Novocrania}: Prominent \citep{Afzelius1978Finestructure}.

\hypertarget{Pelagodiscus_atlanticus-coding-81}{}
\emph{Pelagodiscus atlanticus}: Subacrosomal material (in
\emph{Discinisca}) but no subacrosomal space
\citep{Hodgson1994Ultrastructureof}.

\hypertarget{Phoronis-coding-81}{}
\emph{Phoronis}: The filament-like acrosome continues backwards as a
tube-like structure \citep[summarized in
\citet{Jamieson1991FishEvolution}]{Franzen1980Ultrastructureof}.

\hypertarget{Serpula-coding-81}{}
\emph{Serpula}: \citet{Gherardi2011}.

\hypertarget{Sipunculus-coding-81}{}
\emph{Sipunculus}: \citet{Rice1993}.

\hypertarget{Terebratulina-coding-81}{}
\emph{Terebratulina}: No subacrosomal material, let alone a subacrosomal
space \citep[e.g.][]{Hodgson1994Ultrastructureof}.

\subsection*{{[}82{]} Mid-piece}\label{mid-piece}
\addcontentsline{toc}{subsection}{{[}82{]} Mid-piece}

\includegraphics{Brachiopod_phylogeny_files/figure-latex/character-mapping-81.pdf}

\begin{quote}
\textbf{Character 82: Gametes: Spermatozoa: Mid-piece}

0: Multiple mitochondria\\
1: Single ring-shaped mitochondrion\\
Neomorphic character.
\end{quote}

Following Hodgson \& Reunov \citeyearpar{Hodgson1994Ultrastructureof}.

\hypertarget{Dentalium-coding-82}{}
\emph{Dentalium}: \citet{DufresneDube1983}.

\hypertarget{Flustra-coding-82}{}
\emph{Flustra}: Two mitochondrial derivatives in \emph{Flustra}
\citep{Franzen1981, Franzen2013}; four in \emph{Tubulipora}
\citep{Franzen1984}.

\hypertarget{Lingula-coding-82}{}
\emph{Lingula}, \emph{Terebratulina}: Following Hodgson \& Reunov
\citeyearpar{Hodgson1994Ultrastructureof}.

\hypertarget{Loxosomella-coding-82}{}
\emph{Loxosomella}: ``The midpiece consists of two long mitochondrial
rods connected with each other by a thin mitochondrial lamella''
\citep[in \emph{Loxosoma}]{Franzen2000}; these are essentially a single
organelle surrounding a central rod of electron-dense material.

\hypertarget{Novocrania-coding-82}{}
\emph{Novocrania}: Four mitochondria \citep{Afzelius1978Finestructure}.

\hypertarget{Pelagodiscus_atlanticus-coding-82}{}
\emph{Pelagodiscus atlanticus}: Following \emph{Discinisca}
\emph{tenuis}, described in Hodgson \& Reunov
\citeyearpar{Hodgson1994Ultrastructureof}.

\hypertarget{Phoronis-coding-82}{}
\emph{Phoronis}: The mitochondria fuse in the middle stage of
spermiogenesis to become a pair of mitochondria
\citep{Reunov2004Ultrastructuralstudy}.

\hypertarget{Serpula-coding-82}{}
\emph{Serpula}: Five mitochondria in ring \citep{Gherardi2011}.

\hypertarget{Sipunculus-coding-82}{}
\emph{Sipunculus}: Ring of five mitochondria around the central
centriole \citep{Rice1993}.

\subsection*{{[}83{]} Centrioles:
Orientation}\label{centrioles-orientation}
\addcontentsline{toc}{subsection}{{[}83{]} Centrioles: Orientation}

\includegraphics{Brachiopod_phylogeny_files/figure-latex/character-mapping-82.pdf}

\begin{quote}
\textbf{Character 83: Gametes: Spermatozoa: Centrioles: Orientation}

0: Orthogonal\\
1: Parallel\\
Neomorphic character.
\end{quote}

Following \citet{Hodgson1994Ultrastructureof}.

\hypertarget{Dentalium-coding-83}{}
\emph{Dentalium}: \citet{DufresneDube1983}.

\hypertarget{Flustra-coding-83}{}
\emph{Flustra}: \citep{Franzen1981}.

\hypertarget{Lingula-coding-83}{}
\emph{Lingula}, \emph{Terebratulina}: Following Hodgson \& Reunov
\citeyearpar{Hodgson1994Ultrastructureof}.

\hypertarget{Novocrania-coding-83}{}
\emph{Novocrania}: Two orthogonal centrioles
\citep{Afzelius1978Finestructure}.

\hypertarget{Pelagodiscus_atlanticus-coding-83}{}
\emph{Pelagodiscus atlanticus}: Following \emph{Discinisca}
\emph{tenuis}, described in Hodgson \& Reunov
\citeyearpar{Hodgson1994Ultrastructureof}.

\hypertarget{Phoronis-coding-83}{}
\emph{Phoronis}: Only one centriole in spermatzoon
\citep[p.~7]{Reunov2004Ultrastructuralstudy}, but centrioles are
perpendicularly oriented in spermatogonia (ibid., p.~2).

\hypertarget{Serpula-coding-83}{}
\emph{Serpula}: The proximal centriole is parallel to the flagellum
\citep{Gherardi2011}.

\subsection*{{[}84{]} Centrioles: Fusion}\label{centrioles-fusion}
\addcontentsline{toc}{subsection}{{[}84{]} Centrioles: Fusion}

\includegraphics{Brachiopod_phylogeny_files/figure-latex/character-mapping-83.pdf}

\begin{quote}
\textbf{Character 84: Gametes: Spermatozoa: Centrioles: Fusion}

0: Discrete\\
1: Fused\\
Neomorphic character.
\end{quote}

Following \citet{Smith2012}; see \citet{BucklandNicks2008}.

\hypertarget{Chiton-coding-84}{}
\emph{Chiton}: Proximal centriole fused lateral to distal centriole and
offset.

\hypertarget{Dentalium-coding-84}{}
\emph{Dentalium}: Proximal centriole fused anterior to distal centriole
\citep{DufresneDube1983}.

\hypertarget{Flustra-coding-84}{}
\emph{Flustra}, \emph{Sipunculus}: Proximal centriole fused anterior to
distal centriole.

\hypertarget{Lingula-coding-84}{}
\emph{Lingula}, \emph{Loxosomella}, \emph{Novocrania}, \emph{Phoronis}:
Basal body in deep nuclear fossa.

\subsection*{{[}85{]} Satellite fibre
complex}\label{satellite-fibre-complex}
\addcontentsline{toc}{subsection}{{[}85{]} Satellite fibre complex}

\includegraphics{Brachiopod_phylogeny_files/figure-latex/character-mapping-84.pdf}

\begin{quote}
\textbf{Character 85: Gametes: Spermatozoa: Satellite fibre complex}

0: Annulus not associated with satellite fibres\\
1: Annulus associated with satellite fibres\\
Neomorphic character.
\end{quote}

Following \citet{Smith2012}, after character 48 in \citet{Ponder1997}.

\subsection*{{[}86{]} Mitochondria: Shape}\label{mitochondria-shape}
\addcontentsline{toc}{subsection}{{[}86{]} Mitochondria: Shape}

\includegraphics{Brachiopod_phylogeny_files/figure-latex/character-mapping-85.pdf}

\begin{quote}
\textbf{Character 86: Gametes: Spermatozoa: Mitochondria: Shape}

1: Spherical to subspherical\\
2: Rods\\
3: Elongate, sac-like\\
Transformational character.
\end{quote}

After character 5 in \citet{BucklandNicks2008}; see also character 43 in
\citet{Ponder1997}.

\hypertarget{Chiton-coding-86}{}
\emph{Chiton}: See \citet{BucklandNicks1988}.

\hypertarget{Flustra-coding-86}{}
\emph{Flustra}: Rods \citep{Franzen1981}.

\hypertarget{Loxosomella-coding-86}{}
\emph{Loxosomella}: Elongate rods in \emph{Loxosoma}
\citep{Franzen2000}.

\subsection*{{[}87{]} Mitochondria: Cristae:
Configuration}\label{mitochondria-cristae-configuration}
\addcontentsline{toc}{subsection}{{[}87{]} Mitochondria: Cristae:
Configuration}

\includegraphics{Brachiopod_phylogeny_files/figure-latex/character-mapping-86.pdf}

\begin{quote}
\textbf{Character 87: Gametes: Spermatozoa: Mitochondria: Cristae:
Configuration}

0: Unmodified\\
1: Arranged in parallel plates\\
Neomorphic character.
\end{quote}

After character 44 in \citet{Ponder1997}. Cristae are internal
compartments formed by inner mitochondrial membranes.

\hypertarget{Flustra-coding-87}{}
\emph{Flustra}: ``Typical cristae''; ``Randomly oriented'' --
\citet{Franzen1984} (in \emph{Tubulipora}).

\hypertarget{Loxosomella-coding-87}{}
\emph{Loxosomella}: in \emph{Loxosoma} \citep{Franzen2000}.

\subsection*{{[}88{]} Mitochondria:
Midpiece}\label{mitochondria-midpiece}
\addcontentsline{toc}{subsection}{{[}88{]} Mitochondria: Midpiece}

\includegraphics{Brachiopod_phylogeny_files/figure-latex/character-mapping-87.pdf}

\begin{quote}
\textbf{Character 88: Gametes: Spermatozoa: Mitochondria: Midpiece}

1: Extremely short\\
2: Long\\
3: Forms continuous sheath\\
Transformational character.
\end{quote}

After \citet{Smith2012}; see also character 43 in \citet{Ponder1997};
character 164 in \citet{Giribet2002}.

\hypertarget{Flustra-coding-88}{}
\emph{Flustra}: Long \citep{Franzen1981}.

\hypertarget{Loxosomella-coding-88}{}
\emph{Loxosomella}: As long as the flagellum in \emph{Loxosoma}
\citep{Franzen2000}.

\hypertarget{Serpula-coding-88}{}
\emph{Serpula}: Five mitochondria surround the base of the flagellum in
short midpiece, comparable to that of \emph{Sipunculus} and
\emph{Dentalium} \citep{Gherardi2011}.

\hypertarget{Sipunculus-coding-88}{}
\emph{Sipunculus}: Short ring of five mitochondria around the central
centriole \citep{Rice1993}.

\section{Embryo: Cleavage}\label{embryo-cleavage}

\subsection*{{[}89{]} Equal}\label{equal}
\addcontentsline{toc}{subsection}{{[}89{]} Equal}

\includegraphics{Brachiopod_phylogeny_files/figure-latex/character-mapping-88.pdf}

\begin{quote}
\textbf{Character 89: Embryo: Cleavage: Equal}

1: Unequal\\
2: Equal\\
Transformational character.
\end{quote}

Following character 170 in \citet{Giribet2002}.

\subsection*{{[}90{]} Cross pattern}\label{cross-pattern}
\addcontentsline{toc}{subsection}{{[}90{]} Cross pattern}

\includegraphics{Brachiopod_phylogeny_files/figure-latex/character-mapping-89.pdf}

\begin{quote}
\textbf{Character 90: Embryo: Cleavage: Cross pattern}

0: Absent\\
1: Cross, whether ``molluscan'' or ``annelid''\\
Neomorphic character.
\end{quote}

The ``molluscan cross'' and ``annelid cross'' cannot be systematically
discriminated from one another, so are treated as a single state.\\
See characters 127 \& 128 in \citet{Rouse1999}; 1.49 in
\citet{SPS1996};\\
character 34 in \citet{Haszprunar1996}; 35 in \citet{Haszprunar2000};
172 in \citet{Giribet2002}.

\subsection*{{[}91{]} Polar lobe formation}\label{polar-lobe-formation}
\addcontentsline{toc}{subsection}{{[}91{]} Polar lobe formation}

\includegraphics{Brachiopod_phylogeny_files/figure-latex/character-mapping-90.pdf}

\begin{quote}
\textbf{Character 91: Embryo: Cleavage: Polar lobe formation}

1: Absent\\
2: Present\\
Transformational character.
\end{quote}

Following character 171 in \citet{Giribet2002}.

\subsection*{{[}92{]} Spiral}\label{spiral}
\addcontentsline{toc}{subsection}{{[}92{]} Spiral}

\includegraphics{Brachiopod_phylogeny_files/figure-latex/character-mapping-91.pdf}

\begin{quote}
\textbf{Character 92: Embryo: Cleavage: Spiral}

1: Absent\\
2: Present\\
Transformational character.
\end{quote}

See characters 32--33 in \citet{Haszprunar1996}; character 1.48 in
\citet{SPS1996}; character 29 in \citet{Glenner2004}.

\section{Embryo: Micromere size
{[}93{]}}\label{embryo-micromere-size-93}

\includegraphics{Brachiopod_phylogeny_files/figure-latex/character-mapping-92.pdf}

\begin{quote}
\textbf{Character 93: Embryo: Micromere size}

1: Small relative to macromeres\\
2: Similar to macomeres\\
Transformational character.
\end{quote}

\begin{enumerate}
\def\labelenumi{\arabic{enumi}.}
\item
\end{enumerate}

\subsection*{{[}94{]} Origin of mesoderm}\label{origin-of-mesoderm}
\addcontentsline{toc}{subsection}{{[}94{]} Origin of mesoderm}

\includegraphics{Brachiopod_phylogeny_files/figure-latex/character-mapping-93.pdf}

\begin{quote}
\textbf{Character 94: Embryo: Origin of mesoderm}

1: 4d cell, from the blastopore ridge, or as ectomesoderm\\
2: Archenteron\\
Transformational character.
\end{quote}

After characters 32 in \citet{Grobe2007} and 36--37 in
\citet{Glenner2004}.

\section{Larva}\label{larva}

\subsection*{{[}95{]} Metatroch}\label{metatroch}
\addcontentsline{toc}{subsection}{{[}95{]} Metatroch}

\includegraphics{Brachiopod_phylogeny_files/figure-latex/character-mapping-94.pdf}

\begin{quote}
\textbf{Character 95: Larva: Metatroch}

0: Absent\\
1: Present\\
Neomorphic character.
\end{quote}

See characters 129 and 131 in \citet{Rouse1999}; 40 in
\citet{Haszprunar1996}.\\
A prototroch is the defining character of a trochophore larva; a
metatroch is a secondary ciliary ring \citep{Rouse1999}.

\subsection*{{[}96{]} Telotroch}\label{telotroch}
\addcontentsline{toc}{subsection}{{[}96{]} Telotroch}

\includegraphics{Brachiopod_phylogeny_files/figure-latex/character-mapping-95.pdf}

\begin{quote}
\textbf{Character 96: Larva: Telotroch}

0: Absent\\
1: Present\\
Neomorphic character.
\end{quote}

A posterior ciliary band. Character 136 in \citet{Rouse1999}.

\hypertarget{Flustra-coding-96}{}
\emph{Flustra}: Absent \citep{Zimmer2013}.

\subsection*{{[}97{]} Feeding}\label{feeding}
\addcontentsline{toc}{subsection}{{[}97{]} Feeding}

\includegraphics{Brachiopod_phylogeny_files/figure-latex/character-mapping-96.pdf}

\begin{quote}
\textbf{Character 97: Larva: Feeding}

1: Lecithotrophic (or otherwise non-feeding)\\
2: Planktotrophic (or otherwise feeding)\\
Transformational character.
\end{quote}

Character 140 in \citet{Rouse1999}. See also \citet{Collin1997};
character 2.66 in \citet{SPS1996}; 153 in \citet{Giribet2002}.

\hypertarget{Flustra-coding-97}{}
\emph{Flustra}: Metamorphose almost immediately after release from
gonozooid \citep{Zimmer2013}.

\section{Larva: Cilia}\label{larva-cilia}

\subsection*{{[}98{]} Ciliated food groove}\label{ciliated-food-groove}
\addcontentsline{toc}{subsection}{{[}98{]} Ciliated food groove}

\includegraphics{Brachiopod_phylogeny_files/figure-latex/character-mapping-97.pdf}

\begin{quote}
\textbf{Character 98: Larva: Cilia: Ciliated food groove}

0: Absent\\
1: Present\\
Neomorphic character.
\end{quote}

Character 132 in \citet{Rouse1999}.

\hypertarget{Flustra-coding-98}{}
\emph{Flustra}: Cyclostomes are covered in cilia but not arranged in
food groove.

\subsection*{{[}99{]} Ciliary bands:
Downstream}\label{ciliary-bands-downstream}
\addcontentsline{toc}{subsection}{{[}99{]} Ciliary bands: Downstream}

\includegraphics{Brachiopod_phylogeny_files/figure-latex/character-mapping-98.pdf}

\begin{quote}
\textbf{Character 99: Larva: Cilia: Ciliary bands: Downstream}

0: Absent\\
1: Present\\
Neomorphic character.
\end{quote}

Downstream-collecting ciliary bands of compound cilia on multiciliated
cells. See character 32 in \citet{Glenner2004}.

\hypertarget{Serpula-coding-99}{}
\emph{Serpula}: ``Groups such as Sabellariidae {[}\ldots{}{]} have
evolved downstream-feeding without the aid of a metatroch'' --
\citep{Rouse2000}.

\hypertarget{Sipunculus-coding-99}{}
\emph{Sipunculus}: ``Taxa such as Sipuncula {[}\ldots{}{]} have a
metatroch and do not have downstream larval-feeding'' --
\citet{Rouse2000}.

\subsection*{{[}100{]} Ciliary bands:
Upstream}\label{ciliary-bands-upstream}
\addcontentsline{toc}{subsection}{{[}100{]} Ciliary bands: Upstream}

\includegraphics{Brachiopod_phylogeny_files/figure-latex/character-mapping-99.pdf}

\begin{quote}
\textbf{Character 100: Larva: Cilia: Ciliary bands: Upstream}

0: Absent\\
1: Present\\
Neomorphic character.
\end{quote}

Upstream-collecting ciliary bands with single cilia on monociliated
cells. See character 32 in \citet{Glenner2004}.

\subsection*{{[}101{]} Adoral ciliary band}\label{adoral-ciliary-band}
\addcontentsline{toc}{subsection}{{[}101{]} Adoral ciliary band}

\includegraphics{Brachiopod_phylogeny_files/figure-latex/character-mapping-100.pdf}

\begin{quote}
\textbf{Character 101: Larva: Cilia: Adoral ciliary band}

0: Absent\\
1: Present\\
Neomorphic character.
\end{quote}

Characters 1.50, 2.66 and 4.68 in \citet{SPS1996}; 2 in
\citet{Vinther2008}. See also characters 39 in \citet{Haszprunar1996}
and 153 in \citet{Giribet2002}.

\section{Larva: Nerve ring underlying ciliated larval swimming organ
{[}102{]}}\label{larva-nerve-ring-underlying-ciliated-larval-swimming-organ-102}

\includegraphics{Brachiopod_phylogeny_files/figure-latex/character-mapping-101.pdf}

\begin{quote}
\textbf{Character 102: Larva: Nerve ring underlying ciliated larval
swimming organ}

0: Absent\\
1: Present\\
Neomorphic character.
\end{quote}

Following \citet{Wanninger2009}.

\subsection*{{[}103{]} Paired dorsal setal
bundles}\label{paired-dorsal-setal-bundles}
\addcontentsline{toc}{subsection}{{[}103{]} Paired dorsal setal bundles}

\includegraphics{Brachiopod_phylogeny_files/figure-latex/character-mapping-102.pdf}

\begin{quote}
\textbf{Character 103: Larva: Paired dorsal setal bundles}

0: Absent\\
1: Present\\
Neomorphic character.
\end{quote}

Annelid chaetae are equivalent to the bundled setae expressed in certain
brachiopod larvae. See character 12 in \citet{Vinther2008}.

\hypertarget{Flustra-coding-103}{}
\emph{Flustra}: Absent \citep{Zimmer2013}.

\subsection*{{[}104{]} Apical organ with muscles extending to the
hyposphere}\label{apical-organ-with-muscles-extending-to-the-hyposphere}
\addcontentsline{toc}{subsection}{{[}104{]} Apical organ with muscles
extending to the hyposphere}

\includegraphics{Brachiopod_phylogeny_files/figure-latex/character-mapping-103.pdf}

\begin{quote}
\textbf{Character 104: Larva: Apical organ with muscles extending to the
hyposphere}

0: Absent\\
1: Present\\
Neomorphic character.
\end{quote}

Character 8 in \citet{Vinther2008}.

\hypertarget{Dentalium-coding-104}{}
\emph{Dentalium}: Apical organ has disappeared before musculature is set
in place \citep{Wanninger2002M}.

\subsection*{{[}105{]} Apical organ develops into adult
brain}\label{apical-organ-develops-into-adult-brain}
\addcontentsline{toc}{subsection}{{[}105{]} Apical organ develops into
adult brain}

\includegraphics{Brachiopod_phylogeny_files/figure-latex/character-mapping-104.pdf}

\begin{quote}
\textbf{Character 105: Larva: Apical organ develops into adult brain}

0: Brain has other origin\\
1: Adult brain derived from larval apical organ / apical pole\\
2:\\
Neomorphic character.
\end{quote}

Character 79 in \citet{Glenner2004}.

\subsection*{{[}106{]} Brain persists into
adulthood}\label{brain-persists-into-adulthood}
\addcontentsline{toc}{subsection}{{[}106{]} Brain persists into
adulthood}

\includegraphics{Brachiopod_phylogeny_files/figure-latex/character-mapping-105.pdf}

\begin{quote}
\textbf{Character 106: Larva: Brain persists into adulthood}

1: Brain lost\\
2: Brain retained to adulthood\\
Transformational character.
\end{quote}

After character 3 in \citet{Richter2010}.

\subsection*{{[}107{]} Serotonin-like immunoreactivity in apical
organ}\label{serotonin-like-immunoreactivity-in-apical-organ}
\addcontentsline{toc}{subsection}{{[}107{]} Serotonin-like
immunoreactivity in apical organ}

\includegraphics{Brachiopod_phylogeny_files/figure-latex/character-mapping-106.pdf}

\begin{quote}
\textbf{Character 107: Larva: Serotonin-like immunoreactivity in apical
organ}

1: Two lateral cells in the apical ganglion and the lateral serotonergic
projection to the prototroch;\\
2: Many serotonergic cells in the apical ganglion and a caudal
serotonergic projection to the ciliary band\\
Transformational character.
\end{quote}

(SLI) See \citet{Haszprunar2000}; \citet{Richter2010}\\
Absent (or faint) in \emph{Phascolion} only, so this character only
details the location in taxa in which SLR is readily evident.

\subsection*{{[}108{]} Origin of body
cavity}\label{origin-of-body-cavity}
\addcontentsline{toc}{subsection}{{[}108{]} Origin of body cavity}

\includegraphics{Brachiopod_phylogeny_files/figure-latex/character-mapping-107.pdf}

\begin{quote}
\textbf{Character 108: Larva: Origin of body cavity}

1: Mesenchyme\\
2: Coelom\\
Transformational character.
\end{quote}

Character 1.43 \href{mailto:in@SPS1996}{\nolinkurl{in@SPS1996}}.

\subsection*{{[}109{]} Formation of
coelomoducts}\label{formation-of-coelomoducts}
\addcontentsline{toc}{subsection}{{[}109{]} Formation of coelomoducts}

\includegraphics{Brachiopod_phylogeny_files/figure-latex/character-mapping-108.pdf}

\begin{quote}
\textbf{Character 109: Larva: Formation of coelomoducts}

1: Outgrowth\\
2: Ingrowth\\
Transformational character.
\end{quote}

Character 26 in \citet{Haszprunar2000}.

\hypertarget{Loxosomella-coding-109}{}
\emph{Loxosomella}: Coelomoducts absent \citep{Haszprunar2000}.

\section{Larva: Coelom}\label{larva-coelom}

\subsection*{{[}110{]} Paired}\label{paired}
\addcontentsline{toc}{subsection}{{[}110{]} Paired}

\includegraphics{Brachiopod_phylogeny_files/figure-latex/character-mapping-109.pdf}

\begin{quote}
\textbf{Character 110: Larva: Coelom: Paired}

0: Absent\\
1: Paired coelom originating from two teloblasts derived from 4d\\
Neomorphic character.
\end{quote}

Character 2.02 in \citet{Scheltema1993}.

\hypertarget{Flustra-coding-110}{}
\emph{Flustra}: Hypostegal coelom separated from principal (perigastric)
body cavity in cheilostomata -- but this is not clearly equivalent to
the paired coelom described here.

\subsection*{{[}111{]} Paried: Includes
pericardium}\label{paried-includes-pericardium}
\addcontentsline{toc}{subsection}{{[}111{]} Paried: Includes
pericardium}

\includegraphics{Brachiopod_phylogeny_files/figure-latex/character-mapping-110.pdf}

\begin{quote}
\textbf{Character 111: Larva: Coelom: Paried: Includes pericardium}

0: Paired coelom absent, or does not include pericardium\\
1: Paired coelom includes pericardium\\
Neomorphic character.
\end{quote}

Character 1.03 in \citet{Scheltema1993}.

\section{Larva}\label{larva-1}

\subsection*{{[}112{]} Foot}\label{foot}
\addcontentsline{toc}{subsection}{{[}112{]} Foot}

\includegraphics{Brachiopod_phylogeny_files/figure-latex/character-mapping-111.pdf}

\begin{quote}
\textbf{Character 112: Larva: Foot}

0: Absent\\
1: Present\\
Neomorphic character.
\end{quote}

Foot or neurotroch present in larval stage, whether or not it is also
present in mature individuals. Following \citet{Wingstrand1985}.

\section{Ciliary ultrastructure}\label{ciliary-ultrastructure}

\subsection*{{[}113{]} Accessory centriole}\label{accessory-centriole}
\addcontentsline{toc}{subsection}{{[}113{]} Accessory centriole}

\includegraphics{Brachiopod_phylogeny_files/figure-latex/character-mapping-112.pdf}

\begin{quote}
\textbf{Character 113: Ciliary ultrastructure: Accessory centriole}

0: Absent\\
1: Present\\
Neomorphic character.
\end{quote}

After \citet{Lundin2009}.

\hypertarget{Serpula-coding-113}{}
\emph{Serpula}: Present in certain annelids; not verified in
\emph{Serpula}.

\subsection*{{[}114{]} Aggregation of granules below basal
plate}\label{aggregation-of-granules-below-basal-plate}
\addcontentsline{toc}{subsection}{{[}114{]} Aggregation of granules
below basal plate}

\includegraphics{Brachiopod_phylogeny_files/figure-latex/character-mapping-113.pdf}

\begin{quote}
\textbf{Character 114: Ciliary ultrastructure: Aggregation of granules
below basal plate}

0: Absent\\
1: Present\\
Neomorphic character.
\end{quote}

After \citet{Lundin2009}.

\hypertarget{Serpula-coding-114}{}
\emph{Serpula}: Following \emph{Harmothoe} \citep{Holborow1969}.

\subsection*{{[}115{]} Radiating tubular
fibres}\label{radiating-tubular-fibres}
\addcontentsline{toc}{subsection}{{[}115{]} Radiating tubular fibres}

\includegraphics{Brachiopod_phylogeny_files/figure-latex/character-mapping-114.pdf}

\begin{quote}
\textbf{Character 115: Ciliary ultrastructure: Basal foot: Radiating
tubular fibres}

0: Absent\\
1: Present\\
Neomorphic character.
\end{quote}

After \citet{Lundin2009}. Fibres radiate from the distal end of the
basal foot of the cilia in certain taxa.

\hypertarget{Serpula-coding-115}{}
\emph{Serpula}: Basal foot in \emph{Magelona} is connected to
cytoplasmic microtubules \citep{Bartolomaeus1995}.

\section{Ciliary ultrastructure: Basal plate
{[}116{]}}\label{ciliary-ultrastructure-basal-plate-116}

\includegraphics{Brachiopod_phylogeny_files/figure-latex/character-mapping-115.pdf}

\begin{quote}
\textbf{Character 116: Ciliary ultrastructure: Basal plate}

1: Thin\\
2: Blurry\\
Transformational character.
\end{quote}

After \citet{Lundin2009}. Also termed ``dense plate''.

\hypertarget{Serpula-coding-116}{}
\emph{Serpula}: Broad and `blurry' in \emph{Magelona}
\citep{Bartolomaeus1995}.

\subsection*{{[}117{]} Brushborder of
microvilli}\label{brushborder-of-microvilli}
\addcontentsline{toc}{subsection}{{[}117{]} Brushborder of microvilli}

\includegraphics{Brachiopod_phylogeny_files/figure-latex/character-mapping-116.pdf}

\begin{quote}
\textbf{Character 117: Ciliary ultrastructure: Brushborder of
microvilli}

0: Absent\\
1: Present\\
Neomorphic character.
\end{quote}

After \citet{Lundin2009}; coded following \citet{Smith2012}.

\subsection*{{[}118{]} Centriolar triplet derivative in basal
body}\label{centriolar-triplet-derivative-in-basal-body}
\addcontentsline{toc}{subsection}{{[}118{]} Centriolar triplet
derivative in basal body}

\includegraphics{Brachiopod_phylogeny_files/figure-latex/character-mapping-117.pdf}

\begin{quote}
\textbf{Character 118: Ciliary ultrastructure: Centriolar triplet
derivative in basal body}

1: 9 + 2 pattern\\
2: 9 + 3 pattern\\
Transformational character.
\end{quote}

After \citet{Lundin2009}.

\hypertarget{Serpula-coding-118}{}
\emph{Serpula}: Following \emph{Enchytraeus} \citep{Reger1967},
\emph{Magelona} \citep{Bartolomaeus1995} and \emph{Harmothoe}
\citep{Holborow1969}.

\subsection*{{[}119{]} Ciliary necklace with connecting
strands}\label{ciliary-necklace-with-connecting-strands}
\addcontentsline{toc}{subsection}{{[}119{]} Ciliary necklace with
connecting strands}

\includegraphics{Brachiopod_phylogeny_files/figure-latex/character-mapping-118.pdf}

\begin{quote}
\textbf{Character 119: Ciliary ultrastructure: Ciliary necklace with
connecting strands}

0: Absent\\
1: Present\\
Neomorphic character.
\end{quote}

After \citet{Lundin2009}.\\
The ciliary necklace is defined by \citet{Gilula1972} as ``Well-defined
rows or strands of membrane particles that encircle the ciliary shaft''.
It occurs immediately below the basal plate, and comprises three beaded
circles of on the circumference of the cilia membrane.

\hypertarget{Serpula-coding-119}{}
\emph{Serpula}: Not evident in \emph{Enchytraeus} \citep{Reger1967},
\emph{Magelona} \citep{Bartolomaeus1995} or \emph{Harmothoe}
\citep{Holborow1969}.

\section{Ciliary ultrastructure: Compound
cilia}\label{ciliary-ultrastructure-compound-cilia}

\subsection*{{[}120{]} Presence}\label{presence}
\addcontentsline{toc}{subsection}{{[}120{]} Presence}

\includegraphics{Brachiopod_phylogeny_files/figure-latex/character-mapping-119.pdf}

\begin{quote}
\textbf{Character 120: Ciliary ultrastructure: Compound cilia: Presence}

0: Absent\\
1: Present\\
Neomorphic character.
\end{quote}

After \citet{Lundin2009}. Compound cilia are motile structures composed
of 10--100 regular cilia used in locomotion or feeding.

\hypertarget{Serpula-coding-120}{}
\emph{Serpula}: \citet{Nielsen1987}.

\subsection*{{[}121{]} Origin}\label{origin}
\addcontentsline{toc}{subsection}{{[}121{]} Origin}

\includegraphics{Brachiopod_phylogeny_files/figure-latex/character-mapping-120.pdf}

\begin{quote}
\textbf{Character 121: Ciliary ultrastructure: Compound cilia: Origin}

0: Several monociliate cells\\
1: On multiciliated cell\\
2:\\
Neomorphic character.
\end{quote}

Character 14 in \citet{Glenner2004}. Compound cilia can be produced by
the aggregation of cilia from multiple monociliate cells, or from a
single cell bearing multiple cilia \citep{Nielsen1987}.

\section{Ciliary ultrastructure: Glycocalyx ultrastructure
{[}122{]}}\label{ciliary-ultrastructure-glycocalyx-ultrastructure-122}

\includegraphics{Brachiopod_phylogeny_files/figure-latex/character-mapping-121.pdf}

\begin{quote}
\textbf{Character 122: Ciliary ultrastructure: Glycocalyx
ultrastructure}

1: Homogeneous\\
2: Layered\\
Transformational character.
\end{quote}

After \citet{Lundin2009}.

\subsection*{{[}123{]} Branched}\label{branched}
\addcontentsline{toc}{subsection}{{[}123{]} Branched}

\includegraphics{Brachiopod_phylogeny_files/figure-latex/character-mapping-122.pdf}

\begin{quote}
\textbf{Character 123: Ciliary ultrastructure: Microvilli on epidermal
surface: Branched}

0: Unbranched\\
1: Branched\\
Neomorphic character.
\end{quote}

After \citet{Lundin2009}.

\section{Ciliary ultrastructure: Vertical ciliary
rootlet}\label{ciliary-ultrastructure-vertical-ciliary-rootlet}

\subsection*{{[}124{]} Length}\label{length}
\addcontentsline{toc}{subsection}{{[}124{]} Length}

\includegraphics{Brachiopod_phylogeny_files/figure-latex/character-mapping-123.pdf}

\begin{quote}
\textbf{Character 124: Ciliary ultrastructure: Vertical ciliary rootlet:
Length}

1: Short\\
2: Long\\
Transformational character.
\end{quote}

After \citet{Lundin2009}. The vertical ciliary rootlet is also termed
the posterior rootlet.

\subsection*{{[}125{]} Shape}\label{shape}
\addcontentsline{toc}{subsection}{{[}125{]} Shape}

\includegraphics{Brachiopod_phylogeny_files/figure-latex/character-mapping-124.pdf}

\begin{quote}
\textbf{Character 125: Ciliary ultrastructure: Vertical ciliary rootlet:
Shape}

1: Conical\\
2: Flat\\
Transformational character.
\end{quote}

After \citet{Lundin2009}. The vertical ciliary rootlet is also termed
the posterior rootlet.

\hypertarget{Serpula-coding-125}{}
\emph{Serpula}: Conical in \emph{Enchytraeus} \citep{Reger1967} and
\emph{Magelona} \citep{Bartolomaeus1995}.

\section{Ciliary ultrastructure: Secondary ciliary
rootlet}\label{ciliary-ultrastructure-secondary-ciliary-rootlet}

\subsection*{{[}126{]} Presence}\label{presence-1}
\addcontentsline{toc}{subsection}{{[}126{]} Presence}

\includegraphics{Brachiopod_phylogeny_files/figure-latex/character-mapping-125.pdf}

\begin{quote}
\textbf{Character 126: Ciliary ultrastructure: Secondary ciliary
rootlet: Presence}

0: Absent\\
1: Present\\
Neomorphic character.
\end{quote}

After \citet{Lundin2009}. The secondary ciliary rootlet is also termed
the anterior ciliary rootlet.

\subsection*{{[}127{]} Length}\label{length-1}
\addcontentsline{toc}{subsection}{{[}127{]} Length}

\includegraphics{Brachiopod_phylogeny_files/figure-latex/character-mapping-126.pdf}

\begin{quote}
\textbf{Character 127: Ciliary ultrastructure: Secondary ciliary
rootlet: Length}

1: Short\\
2: Long\\
Transformational character.
\end{quote}

After \citet{Lundin2009}. The secondary ciliary rootlet is also termed
the anterior ciliary rootlet.

\hypertarget{Serpula-coding-127}{}
\emph{Serpula}: Short in \emph{Enchytraeus} \citep{Reger1967},
\emph{Magelona} \citep{Bartolomaeus1995} and \emph{Harmothoe}
\citep{Holborow1969}.

\subsection*{{[}128{]} Shape}\label{shape-1}
\addcontentsline{toc}{subsection}{{[}128{]} Shape}

\includegraphics{Brachiopod_phylogeny_files/figure-latex/character-mapping-127.pdf}

\begin{quote}
\textbf{Character 128: Ciliary ultrastructure: Secondary ciliary
rootlet: Shape}

1: Conical\\
2: Flat\\
Transformational character.
\end{quote}

After \citet{Lundin2009}. The secondary ciliary rootlet is also termed
the anterior ciliary rootlet.

\hypertarget{Serpula-coding-128}{}
\emph{Serpula}: Conical in \emph{Magelona} \citep{Bartolomaeus1995}.

\section{Nephridia}\label{nephridia}

\subsection*{{[}129{]} Podocytes}\label{podocytes}
\addcontentsline{toc}{subsection}{{[}129{]} Podocytes}

\includegraphics{Brachiopod_phylogeny_files/figure-latex/character-mapping-128.pdf}

\begin{quote}
\textbf{Character 129: Nephridia: Podocytes}

0: Absent\\
1: Present\\
Neomorphic character.
\end{quote}

See characters 21 and 28 in \citet{Haszprunar2000}; 1.12 in
\citet{Scheltema1993}.

\hypertarget{Serpula-coding-129}{}
\emph{Serpula}: Present in serpulids \citep{Bartolomaeus2005}.

\subsection*{{[}130{]} Rhogocytes}\label{rhogocytes}
\addcontentsline{toc}{subsection}{{[}130{]} Rhogocytes}

\includegraphics{Brachiopod_phylogeny_files/figure-latex/character-mapping-129.pdf}

\begin{quote}
\textbf{Character 130: Nephridia: Rhogocytes}

0: Absent\\
1: Present\\
Neomorphic character.
\end{quote}

Pore cells. Character 20 in \citet{Haszprunar2000}.

\subsection*{{[}131{]} Serve as excretory
organs}\label{serve-as-excretory-organs}
\addcontentsline{toc}{subsection}{{[}131{]} Serve as excretory organs}

\includegraphics{Brachiopod_phylogeny_files/figure-latex/character-mapping-130.pdf}

\begin{quote}
\textbf{Character 131: Nephridia: Serve as excretory organs}

0: No\\
1: Yes\\
Neomorphic character.
\end{quote}

See character 4.46 in \citet{SPS1996}.

\subsection*{{[}132{]} Protonephridia}\label{protonephridia}
\addcontentsline{toc}{subsection}{{[}132{]} Protonephridia}

\includegraphics{Brachiopod_phylogeny_files/figure-latex/character-mapping-131.pdf}

\begin{quote}
\textbf{Character 132: Nephridia: Protonephridia}

0: Absent\\
1: Present\\
Neomorphic character.
\end{quote}

Also termed cyrtocytes. Character 21 in \citet{Grobe2007}; 1.47 in
\citet{SPS1996}; 138 in \citet{Rouse1999}; 20 in \citet{Haszprunar1996};
90 in \citet{Glenner2004}.

\subsection*{{[}133{]} Metanephridia}\label{metanephridia}
\addcontentsline{toc}{subsection}{{[}133{]} Metanephridia}

\includegraphics{Brachiopod_phylogeny_files/figure-latex/character-mapping-132.pdf}

\begin{quote}
\textbf{Character 133: Nephridia: Metanephridia}

0: Absent\\
1: Present\\
Neomorphic character.
\end{quote}

See characters 35 in \citet{Rouse1999}; 28 in \citet{Haszprunar2000}; 93
in \citet{Glenner2004}; 1.47 in \citet{SPS1996}; 21 in
\citet{Grobe2007}; 138 in \citet{Rouse1999}; 20 in
\citet{Haszprunar1996}.

\section{Cuticle}\label{cuticle}

\subsection*{{[}134{]} Layers}\label{layers}
\addcontentsline{toc}{subsection}{{[}134{]} Layers}

\includegraphics{Brachiopod_phylogeny_files/figure-latex/character-mapping-133.pdf}

\begin{quote}
\textbf{Character 134: Cuticle: Layers}

0: Simple (i.e.~glycocalyx)\\
1: Distinct epicuticle and endocuticle\\
Neomorphic character.
\end{quote}

Character 1 in \citet{Haszprunar1996}.

\subsection*{{[}135{]} Resilience}\label{resilience}
\addcontentsline{toc}{subsection}{{[}135{]} Resilience}

\includegraphics{Brachiopod_phylogeny_files/figure-latex/character-mapping-134.pdf}

\begin{quote}
\textbf{Character 135: Cuticle: Resilience}

0: Labile\\
1: Robust\\
2:\\
Neomorphic character.
\end{quote}

Character 1 in \citet{Haszprunar2000}.

\section{Muscles}\label{muscles}

\subsection*{{[}136{]} Cytology}\label{cytology}
\addcontentsline{toc}{subsection}{{[}136{]} Cytology}

\includegraphics{Brachiopod_phylogeny_files/figure-latex/character-mapping-135.pdf}

\begin{quote}
\textbf{Character 136: Muscles: Cytology}

0: Smooth\\
1: Obliquely striated\\
2:\\
Neomorphic character.
\end{quote}

Character 19 in \citet{Haszprunar1996}; see also character 13 in
\citet{Haszprunar2000}.

\subsection*{{[}137{]} Histology}\label{histology}
\addcontentsline{toc}{subsection}{{[}137{]} Histology}

\includegraphics{Brachiopod_phylogeny_files/figure-latex/character-mapping-136.pdf}

\begin{quote}
\textbf{Character 137: Muscles: Histology}

0: Fibre-type\\
1: Epithelially organized\\
2:\\
Neomorphic character.
\end{quote}

See character 18 in \citet{Haszprunar1996}.

\section{Glands}\label{glands}

\subsection*{{[}138{]} Pedal gland}\label{pedal-gland}
\addcontentsline{toc}{subsection}{{[}138{]} Pedal gland}

\includegraphics{Brachiopod_phylogeny_files/figure-latex/character-mapping-137.pdf}

\begin{quote}
\textbf{Character 138: Glands: Pedal gland}

0: Absent\\
1: Present\\
Neomorphic character.
\end{quote}

Characters 1.13, 1.40 \& 2.08 in \citet{Scheltema1993}; 114 in
\citet{Giribet2002}; 1.53 in \citet{SPS1996}; 9 in
\citet{Haszprunar1996}.

\subsection*{{[}139{]} Paired pharyngeal
diverticulae}\label{paired-pharyngeal-diverticulae}
\addcontentsline{toc}{subsection}{{[}139{]} Paired pharyngeal
diverticulae}

\includegraphics{Brachiopod_phylogeny_files/figure-latex/character-mapping-138.pdf}

\begin{quote}
\textbf{Character 139: Glands: Paired pharyngeal diverticulae}

0: Absent\\
1: Present\\
Neomorphic character.
\end{quote}

\section{Brephic shell}\label{brephic-shell}

\subsection*{{[}140{]} Embryonic shell}\label{embryonic-shell}
\addcontentsline{toc}{subsection}{{[}140{]} Embryonic shell}

\includegraphics{Brachiopod_phylogeny_files/figure-latex/character-mapping-139.pdf}

\begin{quote}
\textbf{Character 140: Brephic shell: Embryonic shell}

0: Absent\\
1: Present\\
Neomorphic character.
\end{quote}

The embryonic shell or protegulum is secreted by the embryo immediately
before hatching.

\hypertarget{Chiton-coding-140}{}
\emph{Chiton}: The chiton \emph{Mopalia} has two distinct shell fields:
that anterior to the prototroch will develop into the first shell plate;
the one posterior to the prototroch becomes the subsequent plates
\citep{Wanninger2002C}. As the latter corresponds to the position of the
shell field in scaphopods \citep{Wanninger2001}, this field is treated
as the `embryonic shell' for the purposes of the present analysis.

\hypertarget{Clupeafumosus_socialis-coding-140}{}
\emph{Clupeafumosus socialis}: Described by Topper \emph{et al}.
\citeyearpar{Topper2013Reappraisalof}.

\hypertarget{Dentalium-coding-140}{}
\emph{Dentalium}: Codings for the embryonic shell follow the situation
in Antalis, which has been exquisitely described \citep{Wanninger2001}.

\hypertarget{Loxosomella-coding-140}{}
\emph{Loxosomella}: Absent, with no possible equivalent
\citep{Nielsen1966}.

\hypertarget{Namacalathus-coding-140}{}
\emph{Namacalathus}: Inapplicable insofar as reproduction occurs by
budding; there is no evidence for a free-living larval stage.
Nevertheless, the presence of a sexual reproductive phase in addition to
asexual reproduction cannot be discounted.

\hypertarget{Novocrania-coding-140}{}
\emph{Novocrania}: Shell not secreted until after metamorphosis
\citep{Popov2010Earliestontogeny}.

\subsection*{{[}141{]} Morphology}\label{morphology}
\addcontentsline{toc}{subsection}{{[}141{]} Morphology}

\includegraphics{Brachiopod_phylogeny_files/figure-latex/character-mapping-140.pdf}

\begin{quote}
\textbf{Character 141: Brephic shell: Morphology}

1: Flat, disc-like (cf. \emph{Micrina})\\
2: Three prominent lobes forming a Y (cf. \emph{Paterimitra})\\
3: Spherical\\
Transformational character.
\end{quote}

The brephic shell is the shell possessed by the young organism
\citep[see][ and references therein for discussion of
terminology]{Ushatinskaya2016Revisionof}.

\emph{Micrina} resembles linguliforms \citep{Holmer2011Firstrecord}: in
both, the brephic mitral shell has one pair of setal sacs enclosed by
lateral lobes, whereas the brephic ventral shell has two lateral setal
tubes.

\emph{Paterimitra} and \emph{Salanygolina} have ``identical'' ventral
brephic shells \citep{Holmer2011Firstrecord}, resembling the shape of a
ship's propeller.

\emph{Haplophrentis} is coded following typical hyoliths, which have a
spherical brephic shell; \emph{Pedunculotheca}'s, in contrast, is
seemingly cap-shaped.

\hypertarget{Askepasma_toddense-coding-141}{}
\emph{Askepasma toddense}: Renoid -- see fig. 4B3 in
\citet{Topper2013Theoldest}.

\hypertarget{Chiton-coding-141}{}
\emph{Chiton}: Disc-like, with ridge-like subdivisions
\citep{Wanninger2002C}.

\hypertarget{Clupeafumosus_socialis-coding-141}{}
\emph{Clupeafumosus socialis}: The flat larval shell of
\emph{Clupeafumosus} resembles that of \emph{Micrina} in outline
\citetext{\citealp{Topper2013Reappraisalof}; \citealp[cf.][]{Holmer2011Firstrecord}}.

\hypertarget{Coolinia_pecten-coding-141}{}
\emph{Coolinia pecten}: See fig. 3 in
\citet{Bassett2017Earliestontogeny}.

\hypertarget{Craniops-coding-141}{}
\emph{Craniops}: The embryonic shell is more or less circular in outline
-- see \citet{Freeman1999Changesin}, fig. 6A.

\hypertarget{Dentalium-coding-141}{}
\emph{Dentalium}: The shell field is initially disc-like, subsequently
expanding to fuse ventrally and produce the cylindircal protoconch
\citep{Wanninger2001}.

\hypertarget{Lingula-coding-141}{}
\emph{Lingula}: See fig. 159 in \citet{Williams1997Introduction}.

\hypertarget{Mickwitzia_muralensis-coding-141}{}
\emph{Mickwitzia muralensis}: Trifoliate appearance results from
prominent attachment rudiment and bunching of setal sacs
\citep{Balthasar2009Thebrachiopod}.

\hypertarget{Micromitra-coding-141}{}
\emph{Micromitra}: Subtriangular -- essentially round.

\hypertarget{Pelagodiscus_atlanticus-coding-141}{}
\emph{Pelagodiscus atlanticus}: See e.g.~fig 169 in Williams \emph{et
al}. \citeyearpar{Williams1997Introduction}.

\subsection*{{[}142{]} Embryonic shell extended in
larvae}\label{embryonic-shell-extended-in-larvae}
\addcontentsline{toc}{subsection}{{[}142{]} Embryonic shell extended in
larvae}

\includegraphics{Brachiopod_phylogeny_files/figure-latex/character-mapping-141.pdf}

\begin{quote}
\textbf{Character 142: Brephic shell: Embryonic shell extended in
larvae}

1: Not extended; embryonic shell contiguous with adult shell\\
2: Extended into larval shell, separated from adult shell by prominent
nick\\
Transformational character.
\end{quote}

Many taxa add to their embryonic shell (the protegulum possessed by the
embryo upon hatching) during the larval phase of their life cycle. The
shell that exists at metamorphosis, marked by a halo or nick point, is
variously termed the ``first formed shell'', ``metamorphic shell'' or
``larval shell'' \citep{Bassett2017Earliestontogeny}.

\hypertarget{Chiton-coding-142}{}
\emph{Chiton}: \citet{Wanninger2002C}.

\hypertarget{Clupeafumosus_socialis-coding-142}{}
\emph{Clupeafumosus socialis}: Described by Topper \emph{et al}.
\citeyearpar{Topper2013Reappraisalof}.

\hypertarget{Craniops-coding-142}{}
\emph{Craniops}: Prominent nick; see \citet{Freeman1999Changesin}, fig.
6A.

\hypertarget{Dentalium-coding-142}{}
\emph{Dentalium}: Prototroch and telotroch are clearly delineated in
post-metamorphic juveniles \citep{Wanninger2001}.

\hypertarget{Eoobolus-coding-142}{}
\emph{Eoobolus}: Nick point indicated by arrows in fig. 1 of Balthasar
\citeyearpar{Balthasar2009Thebrachiopod}.

\hypertarget{Pedunculotheca_diania-coding-142}{}
\emph{Pedunculotheca diania}: The flattened region at the umbo of the
ventral valve in smaller specimens conceivably represents an embryonic
shell, though it may alternatively represent a cicatrix or
colleplax-like structure.

\subsection*{{[}143{]} Surface ornament}\label{surface-ornament}
\addcontentsline{toc}{subsection}{{[}143{]} Surface ornament}

\includegraphics{Brachiopod_phylogeny_files/figure-latex/character-mapping-142.pdf}

\begin{quote}
\textbf{Character 143: Brephic shell: Surface ornament}

1: Smooth\\
2: Rounded pits\\
3: Polygonal impressions\\
4: Pustulose\\
Transformational character.
\end{quote}

Pitting of the larval shell characterises acrotretids and their
relatives. Pustules occur on Paterinidae. See Character 3 in Williams
\emph{et al}. \citeyearpar{Williams2000LinguliformeaCraniiformea} tables
5--6.

\hypertarget{Askepasma_toddense-coding-143}{}
\emph{Askepasma toddense}: Indented with hexagonal pits \citep[appendix
2]{Williams1998Thediversity}.

\hypertarget{Chiton-coding-143}{}
\emph{Chiton}: \citep{Wanninger2002C}.

\hypertarget{Eoobolus-coding-143}{}
\emph{Eoobolus}: Pitted \citep[table
8]{Williams2000LinguliformeaCraniiformea}.

\hypertarget{Lingula-coding-143}{}
\emph{Lingula}, \emph{Pelagodiscus atlanticus}: Smooth, following
family-level codings of \citet{Williams2000LinguliformeaCraniiformea},
table 6.

\hypertarget{Micrina-coding-143}{}
\emph{Micrina}: Smooth \citep{Holmer2011Firstrecord}.

\hypertarget{Micromitra-coding-143}{}
\emph{Micromitra}: Pustolose in Paterinidae \citep[table
6]{Williams2000LinguliformeaCraniiformea}.

\hypertarget{Paterimitra-coding-143}{}
\emph{Paterimitra}: Polygonal texture present
\citep{Holmer2011Firstrecord}, as in the adult shell.

\hypertarget{Salanygolina-coding-143}{}
\emph{Salanygolina}: Smooth \citep{Holmer2009Theenigmatic}.

\subsection*{{[}144{]} Larval attachment
structure}\label{larval-attachment-structure}
\addcontentsline{toc}{subsection}{{[}144{]} Larval attachment structure}

\includegraphics{Brachiopod_phylogeny_files/figure-latex/character-mapping-143.pdf}

\begin{quote}
\textbf{Character 144: Brephic shell: Larval attachment structure}

0: Without evidence of pedicle\\
1: With evidence of pedicle\\
Neomorphic character.
\end{quote}

Embryonic shells of \emph{Micrina} and certain linguliforms exhibit a
transversely folded posterior extension that speaks of the original
presence of a pedicle in the embryo.

This is independent of the presence of an adult pedicle, which may arise
after metamorphosis.

\hypertarget{Clupeafumosus_socialis-coding-144}{}
\emph{Clupeafumosus socialis}: The larval shell embraces the pedicle
foramen, suggesting a larval attachment. See fig. 4 of Topper \emph{et
al}. \citeyearpar{Topper2013Reappraisalof}.

\hypertarget{Eoobolus-coding-144}{}
\emph{Eoobolus}: Lobe related to the attachment rudiment \citep[fig.
2]{Balthasar2009Thebrachiopod}.

\hypertarget{Mickwitzia_muralensis-coding-144}{}
\emph{Mickwitzia muralensis}: Note the posterior lobe related to the
attachment rudiment in fig. 2 of \citet{Balthasar2009Thebrachiopod}.

\hypertarget{Siphonobolus_priscus-coding-144}{}
\emph{Siphonobolus priscus}: Interpreted as having planktotrophic (and
thus non-attached) larvae \citep{Popov2009Earlyontogeny}.

\subsection*{{[}145{]} Setulose}\label{setulose}
\addcontentsline{toc}{subsection}{{[}145{]} Setulose}

\includegraphics{Brachiopod_phylogeny_files/figure-latex/character-mapping-144.pdf}

\begin{quote}
\textbf{Character 145: Brephic shell: Setulose}

0: No evidence of setae in embryonic shell\\
1: Setae present\\
Neomorphic character.
\end{quote}

The protegulum of \emph{Micrina} is penetrated with canals that were
originally associated with setae, a character that it has in common with
linguliforms \citep{Holmer2011Firstrecord}.

\hypertarget{Botsfordia-coding-145}{}
\emph{Botsfordia}: ``One specimen shows fine capillae running laterally
from the posterior tubercles on the dorsal valve (Pl. 3, fig. 5b). This
is possibly the imprints of setae.'' --
\citet{Ushatinskaya2016Revisionof}.

\hypertarget{Clupeafumosus_socialis-coding-145}{}
\emph{Clupeafumosus socialis}: Setal bundles interpreted as present in
acrotretids by Ushatinskaya \citeyearpar{Ushatinskaya2016Protegulumand}.

\hypertarget{Lingulellotreta_malongensis-coding-145}{}
\emph{Lingulellotreta malongensis}: Familial character: larval shell
smooth (Williams \emph{et al}.,2000, p.72).

\hypertarget{Mickwitzia_muralensis-coding-145}{}
\emph{Mickwitzia muralensis}: Four setal sacs.

\section{Brephic shell: Setal sacs
{[}146{]}}\label{brephic-shell-setal-sacs-146}

\includegraphics{Brachiopod_phylogeny_files/figure-latex/character-mapping-145.pdf}

\begin{quote}
\textbf{Character 146: Brephic shell: Setal sacs}

0: Absent\\
1: Present\\
Neomorphic character.
\end{quote}

Setal sacs are recognizable as raised lumps on the juvenile shell
\citep[see][]{Bassett2017Earliestontogeny}.

\emph{Micrina} and linguliforms have setal sacs on their mitral/dorsal
embryonic shell, whereas these are absent in \emph{Paterimitra}
\citep{Holmer2011Firstrecord}.

\hypertarget{Botsfordia-coding-146}{}
\emph{Botsfordia}: A single pair of low tubercles are \citep[ state
``may be'']{Ushatinskaya2016Revisionof} located in the middle region of
the dorsal and the ventral brephic valve; these are interpreted as a
single pair of setal sacs, with the identity of the (dorsally unpaired)
tubercles uncertain.

\hypertarget{Clupeafumosus_socialis-coding-146}{}
\emph{Clupeafumosus socialis}: Setal bundles interpreted as present in
acrotretids by Ushatinskaya \citeyearpar{Ushatinskaya2016Protegulumand}.

\hypertarget{Lingula-coding-146}{}
\emph{Lingula}: Lingulids' larval setae are not arranged in bundles
\citep{Carlson1995Phylogeneticrelationships}.

\hypertarget{Lingulellotreta_malongensis-coding-146}{}
\emph{Lingulellotreta malongensis}: Familial character: larval shell
smooth (williams \emph{et al}.,2000, p.72).

\hypertarget{Novocrania-coding-146}{}
\emph{Novocrania}, \emph{Pelagodiscus atlanticus}: Three pairs
\citep{Carlson1995Phylogeneticrelationships}.

\subsection*{{[}147{]} Number}\label{number}
\addcontentsline{toc}{subsection}{{[}147{]} Number}

\includegraphics{Brachiopod_phylogeny_files/figure-latex/character-mapping-146.pdf}

\begin{quote}
\textbf{Character 147: Brephic shell: Setal sacs: Number}

1: One pair\\
2: Two pairs\\
3: Three pairs\\
Transformational character.
\end{quote}

Two pairs on e.g.~Coolina; one on e.g. \emph{Micrina}.

\hypertarget{Botsfordia-coding-147}{}
\emph{Botsfordia}: ``larval shell with one to three apical tubercles in
ventral valve and two in dorsal valve''
\citep{Williams2000LinguliformeaCraniiformea} -- if these correspond to
setal sacs, then we interpret this as equivalent to one pair.

\hypertarget{Clupeafumosus_socialis-coding-147}{}
\emph{Clupeafumosus socialis}: Two pairs identified in acrotretids by
Ushatinskaya \citeyearpar{Ushatinskaya2016Protegulumand}.

\hypertarget{Mickwitzia_muralensis-coding-147}{}
\emph{Mickwitzia muralensis}: See fig. 2 in
\citet{Balthasar2009Thebrachiopod}.

\hypertarget{Novocrania-coding-147}{}
\emph{Novocrania}, \emph{Pelagodiscus atlanticus}: Three pairs
\citep{Carlson1995Phylogeneticrelationships}.

\hypertarget{Siphonobolus_priscus-coding-147}{}
\emph{Siphonobolus priscus}: Two pairs of setal sacs
\citep{Popov2009Earlyontogeny}.

\section{Setae}\label{setae}

\subsection*{{[}148{]} Present in adults}\label{present-in-adults}
\addcontentsline{toc}{subsection}{{[}148{]} Present in adults}

\includegraphics{Brachiopod_phylogeny_files/figure-latex/character-mapping-147.pdf}

\begin{quote}
\textbf{Character 148: Setae: Present in adults}

0: Absent\\
1: Present\\
Neomorphic character.
\end{quote}

Although preservation of setae (in adults) is exceptional, their
presence can be inferred from shelly material
\citep[see][]{Holmer2006Aspinose}.

\hypertarget{Acanthotretella_spinosa-coding-148}{}
\emph{Acanthotretella spinosa}: Note that the setae do not obviously
emerge from tubes, leading Holmer and Caron to question their homology
with the setae of other taxa (\emph{Heliomedusa}, \emph{Mickwitzia}).

Both valves of \emph{Acanthotretella} were covered by long spine-like
and shell penetrating setae. The setae of \emph{A. decaius} are usually
preserved along anterior and anterolateral margins
\citep{Hu2010Softpart}.

\hypertarget{Chiton-coding-148}{}
\emph{Chiton}: The girdle elements of certain polyplacophorans are
chitinous and secreted by microvilli; it is therefore likely that they
are homologous with the setae of other lophotrochozoans.

\hypertarget{Clupeafumosus_socialis-coding-148}{}
\emph{Clupeafumosus socialis}: Setal bundles interpreted as present in
acrotretids by Ushatinskaya \citeyearpar{Ushatinskaya2016Protegulumand}.

\hypertarget{Flustra-coding-148}{}
\emph{Flustra}: The teeth of the Bryozoan gizzard have been homologized
with annelid setae \citep{Gordon1975}, but a gizzard is not present in
all bryozoan species and has not been reported in \emph{Flustra}.

\hypertarget{Novocrania-coding-148}{}
\emph{Novocrania}: ``Adult craniids are without setae (a feature shared
with the thecideides, the\\
shells of which are also cemented).'' -- \citet{Williams2007Supplement}.

\hypertarget{Siphonobolus_priscus-coding-148}{}
\emph{Siphonobolus priscus}: Phosphatised setae emerge from hollow
spines \citep{Popov2009Earlyontogeny}.

\hypertarget{Sipunculus-coding-148}{}
\emph{Sipunculus}: The absence of chitin or microvillar lineations in
sipunculan hooks argues against their interpretation as setae.

\hypertarget{Wiwaxia_corrugata-coding-148}{}
\emph{Wiwaxia corrugata}: Sclerites likely correspond with
lophotrochozoan setae \citep{Butterfield1990, Smith2014, Zhang2015}.

\subsection*{{[}149{]} Distribution}\label{distribution}
\addcontentsline{toc}{subsection}{{[}149{]} Distribution}

\includegraphics{Brachiopod_phylogeny_files/figure-latex/character-mapping-148.pdf}

\begin{quote}
\textbf{Character 149: Setae: Distribution}

1: Uniform\\
2: Only present at margins of shell\\
3: Serially repeated\\
Transformational character.
\end{quote}

Setae penetrate the valves of many brachiopods. In certain taxa, they
are apparent only at the margins of the valves, in association with the
commissure, being reduced or lost over the surface of the shell.

\hypertarget{Chiton-coding-149}{}
\emph{Chiton}: Uniformly distributed around girdle (though not within
shell) with no serial repetition.

\hypertarget{Eccentrotheca-coding-149}{}
\emph{Eccentrotheca}: Skovsted \emph{et al}.
\citeyearpar{Skovsted2011Scleritomeconstruction} assumed the setae may
have been present along the margin of the adapical opening, but there is
no fossil evidence.

\hypertarget{Heliomedusa_orienta-coding-149}{}
\emph{Heliomedusa orienta}: Throughout the shell -- see
\citet{Williams2007Supplement} -- causing the pustulose appearance
remarked upon by \citet{Chen2007Reinterpretationof}.

\section{Body organization}\label{body-organization}

\subsection*{{[}150{]} Serial repetition}\label{serial-repetition}
\addcontentsline{toc}{subsection}{{[}150{]} Serial repetition}

\includegraphics{Brachiopod_phylogeny_files/figure-latex/character-mapping-149.pdf}

\begin{quote}
\textbf{Character 150: Body organization: Serial repetition}

0: Absent\\
1: Present\\
Neomorphic character.
\end{quote}

Serial repetition in adult, whether expressed in valves, soft tissues or
exoskeletal elements. See character 13 in \citet{Rouse1999}; 19 in
\citet{Vinther2008}; 38 in \citet{Haszprunar1996}; 40--41 in
\citet{Sutton2012}; \citet{Wanninger2009}.

\hypertarget{Halkieria_evangelista-coding-150}{}
\emph{Halkieria evangelista}: Present in dorsal field only
\citep{Zhao2017}.

\subsection*{{[}151{]} Foot}\label{foot-1}
\addcontentsline{toc}{subsection}{{[}151{]} Foot}

\includegraphics{Brachiopod_phylogeny_files/figure-latex/character-mapping-150.pdf}

\begin{quote}
\textbf{Character 151: Body organization: Foot}

0: Absent\\
1: Present\\
Neomorphic character.
\end{quote}

See characters 8 in \citet{Haszprunar1996}; 4 in \citet{Vinther2008};
137 in \citet{Rouse1999}; 21 in \citet{BucklandNicks2008}; 37 in
\citet{Sutton2012}.\\
It is assumed that the adult foot is homologous with (and thus
contingent on) the larval foot.

\hypertarget{Cotyledion_tylodes-coding-151}{}
\emph{Cotyledion tylodes}: The stalk may conceivably be homologous with
the entoproct foot, but the evidence for homology is weak.

\hypertarget{Halkieria_evangelista-coding-151}{}
\emph{Halkieria evangelista}: The ventral surface of \emph{Halkieria} is
unarmoured, but its soft anatomy is unknown.

\hypertarget{Sipunculus-coding-151}{}
\emph{Sipunculus}: LISTED AS PRESENT IN \citet{Smith2012}: WHY?.

\subsection*{{[}152{]} Number}\label{number-1}
\addcontentsline{toc}{subsection}{{[}152{]} Number}

\includegraphics{Brachiopod_phylogeny_files/figure-latex/character-mapping-151.pdf}

\begin{quote}
\textbf{Character 152: Body organization: Coelomoducts: Number}

1: Single\\
2: Multiple\\
Transformational character.
\end{quote}

Character 27 in \citet{Haszprunar2000}.

\section{Body organization}\label{body-organization-1}

\subsection*{{[}153{]} Gills}\label{gills}
\addcontentsline{toc}{subsection}{{[}153{]} Gills}

\includegraphics{Brachiopod_phylogeny_files/figure-latex/character-mapping-152.pdf}

\begin{quote}
\textbf{Character 153: Body organization: Gills}

0: Absent\\
1: Present\\
Neomorphic character.
\end{quote}

Gills (or ctenidia) surround the molluscan foot.\\
Character 1.59--60, 2.09, 4.49 in \citet{SPS1996}; 10--11 in
\citet{Haszprunar2000}; 45 in \citet{Sutton2012}.

\section{Nervous system: Orthogonal
{[}154{]}}\label{nervous-system-orthogonal-154}

\includegraphics{Brachiopod_phylogeny_files/figure-latex/character-mapping-153.pdf}

\begin{quote}
\textbf{Character 154: Nervous system: Orthogonal}

0: Not orthogonal\\
1: Orthogonal\\
Neomorphic character.
\end{quote}

Character 14 in \citet{Haszprunar1996}. Paired longitudinal nerve cords
regularly interconnected by transversal commissures to form a
rectangular pattern.

\section{Body organization: Circulatory system
{[}155{]}}\label{body-organization-circulatory-system-155}

\includegraphics{Brachiopod_phylogeny_files/figure-latex/character-mapping-154.pdf}

\begin{quote}
\textbf{Character 155: Body organization: Circulatory system}

1: Epithelially lined\\
2: Poorly defined, involving sinuses and lacunae\\
3: Closed circulatory system\\
Transformational character.
\end{quote}

After character 23 in \citet{Haszprunar1996}; 24 in
\citet{Haszprunar2000}; 41 in \citet{Rouse1999}; 16 in
\citet{Scheltema1993}; 16 in \citet{Vinther2008}.

\hypertarget{Flustra-coding-155}{}
\emph{Flustra}: As Brachiopods, sipunculans and relatives
\citep{Ruppert1983}.

\hypertarget{Sipunculus-coding-155}{}
\emph{Sipunculus}: Open circulatory system.

\section{Nervous system}\label{nervous-system}

\subsection*{{[}156{]} Glial system}\label{glial-system}
\addcontentsline{toc}{subsection}{{[}156{]} Glial system}

\includegraphics{Brachiopod_phylogeny_files/figure-latex/character-mapping-155.pdf}

\begin{quote}
\textbf{Character 156: Nervous system: Glial system}

0: Absent\\
1: Present\\
Neomorphic character.
\end{quote}

Character 16 in \citet{Haszprunar1996}. The Gliointerstitial system
interconnects the nervous and muscle systems.

\subsection*{{[}157{]} Buccal nerve ring}\label{buccal-nerve-ring}
\addcontentsline{toc}{subsection}{{[}157{]} Buccal nerve ring}

\includegraphics{Brachiopod_phylogeny_files/figure-latex/character-mapping-156.pdf}

\begin{quote}
\textbf{Character 157: Nervous system: Buccal nerve ring}

0: Absent\\
1: Present\\
Neomorphic character.
\end{quote}

Character 7b in \citet{Haszprunar2008}.

\subsection*{{[}158{]} Formation of ganglia}\label{formation-of-ganglia}
\addcontentsline{toc}{subsection}{{[}158{]} Formation of ganglia}

\includegraphics{Brachiopod_phylogeny_files/figure-latex/character-mapping-157.pdf}

\begin{quote}
\textbf{Character 158: Nervous system: Formation of ganglia}

0: From cerebral region\\
1: In situ\\
2:\\
Neomorphic character.
\end{quote}

Character 1.22 in \citet{SPS1996}.

\subsection*{{[}159{]} Presence}\label{presence-2}
\addcontentsline{toc}{subsection}{{[}159{]} Presence}

\includegraphics{Brachiopod_phylogeny_files/figure-latex/character-mapping-158.pdf}

\begin{quote}
\textbf{Character 159: Nervous system: Cerebral ganglia: Presence}

0: Absent\\
1: Present\\
Neomorphic character.
\end{quote}

After character 13 in \citet{Haszprunar1996}.

\section{Nervous system}\label{nervous-system-1}

\subsection*{{[}160{]} Fused}\label{fused}
\addcontentsline{toc}{subsection}{{[}160{]} Fused}

\includegraphics{Brachiopod_phylogeny_files/figure-latex/character-mapping-159.pdf}

\begin{quote}
\textbf{Character 160: Nervous system: Cerebral gangila: Fused}

1: Pair of distinct ganglia\\
2: Single ganglion, or fused ganglia\\
Transformational character.
\end{quote}

After character 13 in \citet{Haszprunar1996}.

\section{Nervous system: Ventral longitudinal nerves
{[}161{]}}\label{nervous-system-ventral-longitudinal-nerves-161}

\includegraphics{Brachiopod_phylogeny_files/figure-latex/character-mapping-160.pdf}

\begin{quote}
\textbf{Character 161: Nervous system: Ventral longitudinal nerves}

0: Separate\\
1: Paired or secondarily fused\\
Neomorphic character.
\end{quote}

Character 80 in \citet{Glenner2004}.

\subsection*{{[}162{]} Location}\label{location}
\addcontentsline{toc}{subsection}{{[}162{]} Location}

\includegraphics{Brachiopod_phylogeny_files/figure-latex/character-mapping-161.pdf}

\begin{quote}
\textbf{Character 162: Nervous system: Location}

0: Not ventral\\
1: Ventrally localized\\
Neomorphic character.
\end{quote}

Character 6 in @Vinther 2008.

\hypertarget{Flustra-coding-162}{}
\emph{Flustra}: The nervous system is ``wedged in between the mouth and
anus'' \citep{Schwaha2015} and thus occupies a ventral position.

\section{Pedicle {[}163{]}}\label{pedicle-163}

\includegraphics{Brachiopod_phylogeny_files/figure-latex/character-mapping-162.pdf}

\begin{quote}
\textbf{Character 163: Pedicle}

0: Absent\\
1: Present\\
Neomorphic character.
\end{quote}

The brachiopod pedicle is a fleshy protuberance that emerges from the
posterior part of the body wall -- as denoted in fossil taxa by its
occurrence between the dorsal and ventral valves.

It is important to distinguish the pedicle from the ``pedicle sheath'',
a tubular extension of the umbo that grows by accretion from an isolated
portion of the ventral mantle. For discussion see
\citet{Holmer2018Theattachment} and \citet{Bassett2017Earliestontogeny}.

\hypertarget{Acanthotretella_spinosa-coding-163}{}
\emph{Acanthotretella spinosa}: The attachment structure of
\emph{Acanthotretella} originates at the margin of the dorsal and
ventral valves; although it emerges from the umbo of the ventral valve,
the presence of an internal pedicle tube betrays its identity as a
pedicle, rather than a pedicle sheath.

The pedicle of \emph{Acanthotretella} emerges from a short extension of
the umbo of the ventral valve. This extension is contiguous with the
valve and presumably grew by accretion; its position and continuity with
the valve suggest its interpretation as a pedicle sheath that is
superseded as an attachment structure. On the other hand, its continuity
with the internal pedicle tube suggests that is may represent an
independent organ.

\hypertarget{Botsfordia-coding-163}{}
\emph{Botsfordia}: Pedicle foramen was not necessarily occupied by a
pedicle (though it presumably was).

\hypertarget{Clupeafumosus_socialis-coding-163}{}
\emph{Clupeafumosus socialis}: A pedicle was presumably present, but
only the foramen is preserved.

\hypertarget{Cotyledion_tylodes-coding-163}{}
\emph{Cotyledion tylodes}: The stalk is conceivably homologous with the
brachiopod pedicle, but this possibility is impossible to test.

\hypertarget{Craniops-coding-163}{}
\emph{Craniops}: Attached apically by cementation.

\hypertarget{Flustra-coding-163}{}
\emph{Flustra}, \emph{Loxosomella}: Grows directly onto the substrate.

\hypertarget{Heliomedusa_orienta-coding-163}{}
\emph{Heliomedusa orienta}: ``It seems unlikely that \emph{H. orienta}
possessed a pedicle that attached it to\\
the soft seafloor, like most other Chengjiang brachiopods.'' \ldots{}\\
``The putative pedicle illustrated by Chen \emph{et al}.
\citeyearpar[Figs 4, 6, 7]{Chen2007Reinterpretationof} in fact is the
mold of a three-dimensionally preserved visceral cavity'' --
\citet{Zhang2009Architectureand}.

\hypertarget{Lingulosacculus-coding-163}{}
\emph{Lingulosacculus}: The absence of a pedicle is inferred from the
absence of an internal pedicle tube, and the absence of a pedicle at the
hinge.

\hypertarget{Mickwitzia_muralensis-coding-163}{}
\emph{Mickwitzia muralensis}: An attachment structure is inferred based
on the presence of an opening \citep{Balthasar2004Shellstructure}; this
is assumed to have been homologous with the brachiopod pedicle.

\hypertarget{Namacalathus-coding-163}{}
\emph{Namacalathus}: There is no obvious way to homologise the
attachment structure with the ventral pedicle of brachiopods.

\hypertarget{Nisusia_sulcata-coding-163}{}
\emph{Nisusia sulcata}: Has a pedicle, rather than a pedicle sheath as
in \emph{Kutorgina}
\citep{Holmer2018Evolutionarysignificance, Holmer2018Theattachment}.

\hypertarget{Paterimitra-coding-163}{}
\emph{Paterimitra}: ``\emph{Paterimitra} is interpreted to have attached
to hard substrates via a pedicle that emerged through the small
posterior opening'' -- \citet{Skovsted2009Thescleritome}.

\hypertarget{Siphonobolus_priscus-coding-163}{}
\emph{Siphonobolus priscus}: Presumed present, based on ventral foramen
with colleplax.

\hypertarget{Sipunculus-coding-163}{}
\emph{Sipunculus}: Absent; there is no clear basis to homologise the
larval attachment structure of certain sipunculans with a pedicle.

\subsection*{{[}164{]} Constitution}\label{constitution}
\addcontentsline{toc}{subsection}{{[}164{]} Constitution}

\includegraphics{Brachiopod_phylogeny_files/figure-latex/character-mapping-163.pdf}

\begin{quote}
\textbf{Character 164: Pedicle: Constitution}

1: Massive or uniform\\
2: Densely stacked tabular discs\\
Transformational character.
\end{quote}

The pedicle of certain chengjiang rhynchonelliforms comprises ``densely
stacked, three dimensionally preserved, tabular discs''
\citep{Holmer2018Evolutionarysignificance}.\\
This contrasts with the uniform (`massive') pedicles of living taxa.

\hypertarget{Terebratulina-coding-164}{}
\emph{Terebratulina}: Extant rhynconellid pedicles are massive,
consisting of a thick outer chitinous cuticle, a pedicle epithelium, and
a core composed of collagen fibres and cartilage-like connective tissue
\citep{Holmer2018Evolutionarysignificance}.

\subsection*{{[}165{]} Biomineralization}\label{biomineralization}
\addcontentsline{toc}{subsection}{{[}165{]} Biomineralization}

\includegraphics{Brachiopod_phylogeny_files/figure-latex/character-mapping-164.pdf}

\begin{quote}
\textbf{Character 165: Pedicle: Biomineralization}

1: Absent\\
2: Present\\
Transformational character.
\end{quote}

The pedicle of strophomenates such as \emph{Antigonambonites} is
biomineralized \citep{Holmer2018Evolutionarysignificance}.

\subsection*{{[}166{]} Bulb}\label{bulb}
\addcontentsline{toc}{subsection}{{[}166{]} Bulb}

\includegraphics{Brachiopod_phylogeny_files/figure-latex/character-mapping-165.pdf}

\begin{quote}
\textbf{Character 166: Pedicle: Bulb}

0: Absent\\
1: Present\\
Neomorphic character.
\end{quote}

A bulb is an expanded region of the distal pedicle, often embedded into
the sediment to improve anchorage.

\hypertarget{Acanthotretella_spinosa-coding-166}{}
\emph{Acanthotretella spinosa}: Holmer and Caron
\citeyearpar{Holmer2006Aspinose} interpret the presence of a bulb as
tentative; we score it as ambiguous.

\subsection*{{[}167{]} Distal rootlets}\label{distal-rootlets}
\addcontentsline{toc}{subsection}{{[}167{]} Distal rootlets}

\includegraphics{Brachiopod_phylogeny_files/figure-latex/character-mapping-166.pdf}

\begin{quote}
\textbf{Character 167: Pedicle: Distal rootlets}

0: Absent\\
1: Present\\
Neomorphic character.
\end{quote}

Observed in \emph{Pedunculotheca} and \emph{Bethia}
\citep{Sutton2005Silurianbrachiopods}.

\subsection*{{[}168{]} Tapering}\label{tapering}
\addcontentsline{toc}{subsection}{{[}168{]} Tapering}

\includegraphics{Brachiopod_phylogeny_files/figure-latex/character-mapping-167.pdf}

\begin{quote}
\textbf{Character 168: Pedicle: Tapering}

1: Uniform thickness\\
2: Tapering\\
Transformational character.
\end{quote}

Holmer \emph{et al}. \citeyearpar{Holmer2018Theattachment} remark that
the tapering aspect of the \emph{Nisusia} pedicle recalls that of
certain Chengjiang taxa (\emph{Alisina}, \emph{Longtancunella}) whilst
distinguishing it from many other taxa (\emph{Eichwaldia},
\emph{Bethia}) in which the pedicle is a constant thickness.

\hypertarget{Antigonambonites_planus-coding-168}{}
\emph{Antigonambonites planus}: Tapered pedicle sheath with holdfast.

\hypertarget{Pedunculotheca_diania-coding-168}{}
\emph{Pedunculotheca diania}: The pedicle thickness does not obviously
change between the apex of the shell and the holdfast.

\subsection*{{[}169{]} Coelomic region}\label{coelomic-region}
\addcontentsline{toc}{subsection}{{[}169{]} Coelomic region}

\includegraphics{Brachiopod_phylogeny_files/figure-latex/character-mapping-168.pdf}

\begin{quote}
\textbf{Character 169: Pedicle: Coelomic region}

1: Absent\\
2: Present\\
Transformational character.
\end{quote}

Certain brachiopods, such as \emph{Acanthotretella}, exhibit a coelomic
cavity within the pedicle or pedicle sheath.

Treated as transformational as it is not clear that either state is
necessarily ancestral.

\hypertarget{Nisusia_sulcata-coding-169}{}
\emph{Nisusia sulcata}: A coleomic canal is inferred based on the ease
with which the pedicle is deformed
\citep{Holmer2018Evolutionarysignificance}, but its presence is not
known for certain so is coded ambiguous.

\subsection*{{[}170{]} Surface ornament}\label{surface-ornament-1}
\addcontentsline{toc}{subsection}{{[}170{]} Surface ornament}

\includegraphics{Brachiopod_phylogeny_files/figure-latex/character-mapping-169.pdf}

\begin{quote}
\textbf{Character 170: Pedicle: Surface ornament}

1: Smooth\\
2: Irregular wrinkles\\
3: Regular annulations\\
Transformational character.
\end{quote}

Annulations are regular rings that surround the pedicle, and are
distinguished from wrinkles, which are irregular in magnitude and
spacing, and may branch or fail to entirely encircle the pedicle.

\hypertarget{Acanthotretella_spinosa-coding-170}{}
\emph{Acanthotretella spinosa}: ``The pedicle surface is ornamented with
pronounced annulated rings, disposed at intervals of about 0.2 mm''.

\hypertarget{Alisina-coding-170}{}
\emph{Alisina}: ``It appears that the pedicle lacks a coelomic space and
is distinctly annulated, with densely stacked tabular bodies'' --
\citet{Zhang2011Anobolellate}.

\hypertarget{Antigonambonites_planus-coding-170}{}
\emph{Antigonambonites planus}: ``The emerging pedicle has a consistent
shape in all the available specimens and is strongly annulated and
distally tapering'' -- \citet{Holmer2018Evolutionarysignificance}.

\hypertarget{Kutorgina_chengjiangensis-coding-170}{}
\emph{Kutorgina chengjiangensis}: ``Pronounced concentric annular discs
disposed at intervals of 0.6\textless{}96\textgreater{}1.0 mm''.

\hypertarget{Lingulellotreta_malongensis-coding-170}{}
\emph{Lingulellotreta malongensis}: Regularly annotated \citep[see fig.
14.9 in][]{Hou2017Brachiopoda}.

\hypertarget{Longtancunella_chengjiangensis-coding-170}{}
\emph{Longtancunella chengjiangensis}: ``The preserved pedicle has
condensed annulations'' -- \citet{Zhang2011Theexceptionally}.

\hypertarget{Nisusia_sulcata-coding-170}{}
\emph{Nisusia sulcata}: The ``strong annulations'' vary significantly in
transverse thickness \citep{Holmer2018Evolutionarysignificance}, so it
is not clear whether these represent true annulations or wrinkles.

\hypertarget{Yuganotheca_elegans-coding-170}{}
\emph{Yuganotheca elegans}: Annulations present in median collar.

\subsection*{{[}171{]} Nerve impression}\label{nerve-impression}
\addcontentsline{toc}{subsection}{{[}171{]} Nerve impression}

\includegraphics{Brachiopod_phylogeny_files/figure-latex/character-mapping-170.pdf}

\begin{quote}
\textbf{Character 171: Pedicle: Nerve impression}

0: Absent\\
1: Present\\
Neomorphic character.
\end{quote}

In certain taxa the impression of the pedicle nerve is evident in the
shell. See character 28 in Williams \emph{et al}.
\citeyearpar{Williams1998Thediversity} appendix 1. Care must be taken
not to code an impression as absent when the preservational quality is
insufficient to safely infer a genuine absence. Treated as neomorphic as
the presence of an innervation is considered a derived state.

\hypertarget{Alisina-coding-171}{}
\emph{Alisina}: Not described by
\citet{Williams2000LinguliformeaCraniiformea}.

\hypertarget{Askepasma_toddense-coding-171}{}
\emph{Askepasma toddense}, \emph{Glyptoria}, \emph{Kutorgina
chengjiangensis}, \emph{Micromitra}, \emph{Salanygolina}: Following
\citet{Williams1998Thediversity}, appendix 2.

\hypertarget{Botsfordia-coding-171}{}
\emph{Botsfordia}: Documented by \citet{Skovsted2017Depthrelated}.

\hypertarget{Clupeafumosus_socialis-coding-171}{}
\emph{Clupeafumosus socialis}: Coded as absent in Acrotretidae
\citep[table 6]{Williams2000LinguliformeaCraniiformea}.

\hypertarget{Lingula-coding-171}{}
\emph{Lingula}: Present in many lingulids
\citep{Williams2000LinguliformeaCraniiformea}, and coded as present in
Lingulidae \citep[table 6]{Williams2000LinguliformeaCraniiformea}.

\hypertarget{Lingulellotreta_malongensis-coding-171}{}
\emph{Lingulellotreta malongensis}: Coded as present in
Lingulellotretidae \citep[table
6]{Williams2000LinguliformeaCraniiformea}.

\hypertarget{Mummpikia_nuda-coding-171}{}
\emph{Mummpikia nuda}: Balthasar
\citeyearpar[p.~274]{Balthasar2008iMummpikia} identifies a canal as a
probable impression of a pedicle nerve.

\hypertarget{Nisusia_sulcata-coding-171}{}
\emph{Nisusia sulcata}, \emph{Orthis}: Not reported in
\citet{Williams2000LinguliformeaCraniiformea}.

\hypertarget{Pelagodiscus_atlanticus-coding-171}{}
\emph{Pelagodiscus atlanticus}: Coded as present in Discinidae
\citep[table 6]{Williams2000LinguliformeaCraniiformea}.

\hypertarget{Siphonobolus_priscus-coding-171}{}
\emph{Siphonobolus priscus}: Coded as absent in Siphonotretidae
\citep[table 6]{Williams2000LinguliformeaCraniiformea}.

\section{Head: Prostomial appendages
{[}172{]}}\label{head-prostomial-appendages-172}

\includegraphics{Brachiopod_phylogeny_files/figure-latex/character-mapping-171.pdf}

\begin{quote}
\textbf{Character 172: Head: Prostomial appendages}

0: Absent\\
1: Present\\
Neomorphic character.
\end{quote}

Annelid palps and sipunculan tentacles both develop from a pair of
protostomial initials and seem to be homologous \citep{Adrianov2005}.
The capatcula of scaphopods presumably have an independent origin but
have been coded as conceivably equivalent.

\section{Mantle canals}\label{mantle-canals}

\subsection*{{[}173{]} Presence}\label{presence-3}
\addcontentsline{toc}{subsection}{{[}173{]} Presence}

\includegraphics{Brachiopod_phylogeny_files/figure-latex/character-mapping-172.pdf}

\begin{quote}
\textbf{Character 173: Mantle canals: Presence}

0: Absent\\
1: Present\\
Neomorphic character.
\end{quote}

Whether impressed on a shell or expressed solely in soft tissue.

\subsection*{{[}174{]} Morphology}\label{morphology-1}
\addcontentsline{toc}{subsection}{{[}174{]} Morphology}

\includegraphics{Brachiopod_phylogeny_files/figure-latex/character-mapping-173.pdf}

\begin{quote}
\textbf{Character 174: Mantle canals: Morphology}

1: Pinnate (=lemniscate)\\
2: Bifurcate\\
3: Baculate\\
4: Saccate\\
Transformational character.
\end{quote}

The morphology of dorsal and ventral canals is identical in all included
taxa, so is assumed not to be independent -- hence the use of a single
character \citep[contra][]{Williams2000LinguliformeaCraniiformea}.

For a description of terms see Williams \emph{et al}.
\citeyearpar[2000]{Williams1997Introduction}.

Pinnate = ``rapidly branch into a number of subequal, radially disposed
canals''\\
Bifurcate = ``\emph{vascula} \emph{lateralia} in both valves divide
immediately after leaving the body cavity''\\
Baculate = ``extend forward without any major dichotomy or bifurcation''
\citep[ p.~418]{Williams1997Introduction}\\
Saccate = ``pouchlike sinuses lying wholly posterior to the arcuate
\emph{vascula} \emph{media}'' (ibid., p412).

\hypertarget{Acanthotretella_spinosa-coding-174}{}
\emph{Acanthotretella spinosa}: Following Table 6, for Siphonotretidae,
in Williams \emph{et al}.
\citeyearpar{Williams2000LinguliformeaCraniiformea}.

\hypertarget{Alisina-coding-174}{}
\emph{Alisina}, \emph{Nisusia sulcata}: Following Table 15 in Williams
\emph{et al}. \citeyearpar{Williams2000LinguliformeaCraniiformea}.

\hypertarget{Antigonambonites_planus-coding-174}{}
\emph{Antigonambonites planus}: Not reported in Treatise
\citep{Williams2000LinguliformeaCraniiformea}.

\hypertarget{Askepasma_toddense-coding-174}{}
\emph{Askepasma toddense}: Described as pinnate (at least in ventral
valve) by Williams \emph{et al}.
\citeyearpar[p.~250]{Williams1998Thediversity}.

\hypertarget{Botsfordia-coding-174}{}
\emph{Botsfordia}, \emph{Eoobolus}: Following
\citet{Williams1998Thediversity}, appendix 2, and Williams \emph{et al}.
\citeyearpar{Williams2000LinguliformeaCraniiformea}, table 8.

\hypertarget{Clupeafumosus_socialis-coding-174}{}
\emph{Clupeafumosus socialis}: Following Table 8 (for Acrotreta) in
Williams \emph{et al}.
\citeyearpar{Williams2000LinguliformeaCraniiformea}, and the general
pinnate condition for acrotretoids stated in Williams \emph{et al}.
\citeyearpar{Williams1997Introduction}, p.~420.

\hypertarget{Coolinia_pecten-coding-174}{}
\emph{Coolinia pecten}: Not reported in
\citet{Williams2000LinguliformeaCraniiformea}.

\hypertarget{Craniops-coding-174}{}
\emph{Craniops}: Not reported from fossil material.

\hypertarget{Gasconsia-coding-174}{}
\emph{Gasconsia}: Williams \emph{et al}. \citeyearpar[table
15]{Williams2000LinguliformeaCraniiformea} appear to use
Palaeotrimerella \citep[as drawn in][]{Williams1997Introduction} as a
model for \emph{Gasconsia}, which pre-supposes a close relationship. We
are not aware of any report of mantle canals from \emph{Gasconsia}
itself.

\hypertarget{Glyptoria-coding-174}{}
\emph{Glyptoria}: Following appendix 2 (char. 21) in Williams \emph{et
al}. \citeyearpar{Williams1998Thediversity}.

\hypertarget{Heliomedusa_orienta-coding-174}{}
\emph{Heliomedusa orienta}: Described as pinnate by Jin \& Wang
\citeyearpar{Jin1992Revisionof}.

\hypertarget{Kutorgina_chengjiangensis-coding-174}{}
\emph{Kutorgina chengjiangensis}, \emph{Novocrania}: Following table 15
in Williams \emph{et al}.
\citeyearpar{Williams2000LinguliformeaCraniiformea} (for
\emph{Neocrania}).

\hypertarget{Lingula-coding-174}{}
\emph{Lingula}, \emph{Lingulellotreta malongensis}: Following table 6 in
Williams \emph{et al}.
\citeyearpar{Williams2000LinguliformeaCraniiformea}.

\hypertarget{Lingulosacculus-coding-174}{}
\emph{Lingulosacculus}: Baculate \emph{vascula} \emph{media} --
Balthasar \& Butterfield \citeyearpar{Balthasar2009EarlyCambrian}.

\hypertarget{Longtancunella_chengjiangensis-coding-174}{}
\emph{Longtancunella chengjiangensis}: Reported by Zhang \emph{et al}.
\citeyearpar[2011T]{Zhang2007Agregarious} though the interpretation is
tentative.

\hypertarget{Micromitra-coding-174}{}
\emph{Micromitra}: Described as saccate by Williams \emph{et al}.
\citeyearpar{Williams1998Thediversity}.

\hypertarget{Mummpikia_nuda-coding-174}{}
\emph{Mummpikia nuda}: ``Poorly resolved'' --
\citet{Balthasar2008iMummpikia}.

\hypertarget{Orthis-coding-174}{}
\emph{Orthis}: Sacculate (sometimes digitate in dorsal valve)
\citep[p716]{Williams2000LinguliformeaCraniiformea}.

\hypertarget{Pelagodiscus_atlanticus-coding-174}{}
\emph{Pelagodiscus atlanticus}: Following table 6, for Discinidae, in
Williams \emph{et al}.
\citeyearpar{Williams2000LinguliformeaCraniiformea}.

\hypertarget{Salanygolina-coding-174}{}
\emph{Salanygolina}: Coded uncertain in appendix 2 in Williams \emph{et
al}. \citeyearpar{Williams1998Thediversity}.

\hypertarget{Siphonobolus_priscus-coding-174}{}
\emph{Siphonobolus priscus}: Interpreted as baculate, following
\citet{Havlicek1982LingulaceaPaterinacea}.

\hypertarget{Terebratulina-coding-174}{}
\emph{Terebratulina}: ``In modern terebratulides, the \emph{vascula}
\emph{media} are subordinate to the lemniscate or pinnate \emph{vascula}
\emph{genitalia}'' -- \citet{Williams1997Introduction}.

\hypertarget{Tomteluva_perturbata-coding-174}{}
\emph{Tomteluva perturbata}: Preservation not adequate to evaluate
\citep{Streng2016Anew}.

\subsection*{\texorpdfstring{{[}175{]} \emph{vascula}
\emph{lateralia}}{{[}175{]} vascula lateralia}}\label{vascula-lateralia}
\addcontentsline{toc}{subsection}{{[}175{]} \emph{vascula}
\emph{lateralia}}

\includegraphics{Brachiopod_phylogeny_files/figure-latex/character-mapping-174.pdf}

\begin{quote}
\textbf{Character 175: Mantle canals: \emph{vascula} \emph{lateralia}}

0: Absent\\
1: Present\\
Neomorphic character.
\end{quote}

We treat the \emph{vascula} \emph{lateralia} as equivalent to the
\emph{vascula} \emph{genitalia} of articulated brachiopods, allowing
phylogenetic analysis to test their proposed homology.

Williams \emph{et al}. \citeyearpar{Williams1997Introduction} write:
``The mantle canal system of most of the organophosphate-shelled species
consists of a single pair of main trunks in the ventral mantle
(\emph{vascula} \emph{lateralia}) and two pairs in the dorsal mantle,
one pair (\emph{vascula} \emph{lateralia}) occupying a similar position
to the single pair in the ventral mantle and a second pair projecting
from the body cavity near the midline of the valve. This latter pair may
be termed the \emph{vascula} \emph{media}, but whether they are strictly
homologous with the \emph{vascula} \emph{media} of articulated
brachiopods is a matter of opinion. It is also impossible to assert that
the \emph{vascula} \emph{lateralia} are the homologues of the
\emph{vascula} \emph{myaria} or \emph{genitalia} of articulated species,
although they are likely to be so as they arise in a comparable
position.''

``In inarticulated brachiopods, two main mantle canals (\emph{vascula}
\emph{lateralia}) emerge from the main body cavity through muscular
valves and bifurcate distally to produce an increasingly dense array of
blindly ending branches near the periphery of the mantle (fig.
71.1--71.2).''

\hypertarget{Acanthotretella_spinosa-coding-175}{}
\emph{Acanthotretella spinosa}: Following table 8 (which records
presence in Siphonotreta) in Williams \emph{et al}.
\citeyearpar{Williams2000LinguliformeaCraniiformea}.

\hypertarget{Alisina-coding-175}{}
\emph{Alisina}, \emph{Kutorgina chengjiangensis}, \emph{Nisusia
sulcata}: Following table 15 in Williams \emph{et al}.
\citeyearpar{Williams2000LinguliformeaCraniiformea}.

\hypertarget{Askepasma_toddense-coding-175}{}
\emph{Askepasma toddense}, \emph{Micromitra}: ``Laurie
\citeyearpar{Laurie1987Themusculature} has shown that arcuate
\emph{vascula} \emph{media} were present in the mantles of both valves
as were pouchlike \emph{vascula} \emph{genitalia}, especially in the
ventral valve'' -- \citet{Williams1997Introduction}.

\hypertarget{Botsfordia-coding-175}{}
\emph{Botsfordia}: Following Popov \citeyearpar{Popov1992TheCambrian}.

\hypertarget{Clupeafumosus_socialis-coding-175}{}
\emph{Clupeafumosus socialis}: Presence indicated in Table 8 (for
Acrotreta) in Williams \emph{et al}.
\citeyearpar{Williams2000LinguliformeaCraniiformea}.

\hypertarget{Gasconsia-coding-175}{}
\emph{Gasconsia}: Williams \emph{et al}. \citeyearpar[table
15]{Williams2000LinguliformeaCraniiformea} appear to use
Palaeotrimerella \citep[as drawn in][]{Williams1997Introduction} as a
model for \emph{Gasconsia}, which pre-supposes a close relationship. We
are not aware of any report of mantle canals from \emph{Gasconsia}
itself.

\hypertarget{Heliomedusa_orienta-coding-175}{}
\emph{Heliomedusa orienta}: Present: Williams \emph{et al}.
\citeyearpar{Williams2000LinguliformeaCraniiformea}; Jin \& Wang
\citeyearpar{Jin1992Revisionof}.

\hypertarget{Lingulellotreta_malongensis-coding-175}{}
\emph{Lingulellotreta malongensis}: Present
\citep{Williams2000LinguliformeaCraniiformea}.

\hypertarget{Longtancunella_chengjiangensis-coding-175}{}
\emph{Longtancunella chengjiangensis}: Presence is possible but requires
interpretation that is not unambiguous:

``In the dorsal valve, there can be seen two baculate grooves that arise
from the\\
anterior body wall at an antero-lateral position. These two grooves
(Figs 4H, 5D) could be taken to represent the \emph{vascula}
\emph{lateralia}'' -- \citet{Zhang2007Agregarious}.

\hypertarget{Novocrania-coding-175}{}
\emph{Novocrania}: Following table 15 in Williams \emph{et al}.
\citeyearpar{Williams2000LinguliformeaCraniiformea} (for
\emph{Neocrania}), who write that ``Holocene craniides have only a
single pair of main trunks in both valves, corresponding to the
\emph{vascula} \emph{lateralia}''. Williams \emph{et al}.
\citeyearpar{Williams2007Supplement} reiterate this position (p.~2875),
at least for the ventral valve.

\hypertarget{Orthis-coding-175}{}
\emph{Orthis}, \emph{Terebratulina}: = \emph{vascula} \emph{genitalia}.

\hypertarget{Pelagodiscus_atlanticus-coding-175}{}
\emph{Pelagodiscus atlanticus}: Following \emph{Lochkothele}
(Discinidae), Fig. 43.4a in Williams \emph{et al}.
\citeyearpar{Williams2000LinguliformeaCraniiformea}.

\hypertarget{Siphonobolus_priscus-coding-175}{}
\emph{Siphonobolus priscus}: Noted in \emph{Siphonobolus} by Williams
\emph{et al}. \citeyearpar{Williams2000LinguliformeaCraniiformea}, with
reference to Havlicek \citeyearpar{Havlicek1982LingulaceaPaterinacea}.

\hypertarget{Tomteluva_perturbata-coding-175}{}
\emph{Tomteluva perturbata}: Preservation not adequate to evaluate
\citep{Streng2016Anew}.

\hypertarget{Yuganotheca_elegans-coding-175}{}
\emph{Yuganotheca elegans}: Based on the figures and sketches in
\citet{Zhang2014Anearly} (and supplementary material), the mantle canals
are interpreted as lateral, with no clear \emph{vascula} \emph{media}
present.

\subsection*{\texorpdfstring{{[}176{]} \emph{vascula}
\emph{media}}{{[}176{]} vascula media}}\label{vascula-media}
\addcontentsline{toc}{subsection}{{[}176{]} \emph{vascula} \emph{media}}

\includegraphics{Brachiopod_phylogeny_files/figure-latex/character-mapping-175.pdf}

\begin{quote}
\textbf{Character 176: Mantle canals: \emph{vascula} \emph{media}}

0: Absent\\
1: Present (in dorsal valve)\\
Neomorphic character.
\end{quote}

Williams \emph{et al}. \citeyearpar{Williams1997Introduction} note that
in addition to the \emph{vascula} \emph{lateralia}, ``\emph{Discinisca}
has two additional mantle canals emanating from the body cavity into the
dorsal mantle (\emph{vascula} \emph{media}).''

These structures are only evident in the dorsal valve for the included
taxa, so only a single character is necessary.

\hypertarget{Acanthotretella_spinosa-coding-176}{}
\emph{Acanthotretella spinosa}: Following table 6 (for Siphonotretidae)
in Williams \emph{et al}.
\citeyearpar{Williams2000LinguliformeaCraniiformea}.

\hypertarget{Alisina-coding-176}{}
\emph{Alisina}, \emph{Kutorgina chengjiangensis}, \emph{Nisusia
sulcata}: Following table 15 in Williams \emph{et al}.
\citeyearpar{Williams2000LinguliformeaCraniiformea}.

\hypertarget{Askepasma_toddense-coding-176}{}
\emph{Askepasma toddense}: Following table 6 (for Paterinidae) in
Williams \emph{et al}.
\citeyearpar{Williams2000LinguliformeaCraniiformea}.

\hypertarget{Botsfordia-coding-176}{}
\emph{Botsfordia}: Following Popov \citeyearpar[fig.
2]{Popov1992TheCambrian}.

\hypertarget{Clupeafumosus_socialis-coding-176}{}
\emph{Clupeafumosus socialis}: Following \emph{Hadrotreta} schematic in
Williams \emph{et al}.
\citeyearpar{Williams2000LinguliformeaCraniiformea}.

\hypertarget{Eoobolus-coding-176}{}
\emph{Eoobolus}: Fig. 5 in \citet{Balthasar2009Thebrachiopod}.

\hypertarget{Gasconsia-coding-176}{}
\emph{Gasconsia}: Williams \emph{et al}. \citeyearpar[table
15]{Williams2000LinguliformeaCraniiformea} appear to use
Palaeotrimerella \citep[as drawn in][]{Williams1997Introduction} as a
model for \emph{Gasconsia}, which pre-supposes a close relationship. We
are not aware of any report of mantle canals from \emph{Gasconsia}
itself.

\hypertarget{Glyptoria-coding-176}{}
\emph{Glyptoria}: Present and divergent
\citep{Williams2000LinguliformeaCraniiformea}.

\hypertarget{Heliomedusa_orienta-coding-176}{}
\emph{Heliomedusa orienta}: Present: Williams \emph{et al}.
\citeyearpar{Williams2000LinguliformeaCraniiformea} p162, Jin \& Wang
\citeyearpar{Jin1992Revisionof}.

\hypertarget{Lingula-coding-176}{}
\emph{Lingula}, \emph{Lingulellotreta malongensis}: Following table 6 in
Williams \emph{et al}.
\citeyearpar{Williams2000LinguliformeaCraniiformea}.

\hypertarget{Longtancunella_chengjiangensis-coding-176}{}
\emph{Longtancunella chengjiangensis}: Reported by Zhang \emph{et al}.
\citeyearpar{Zhang2007Agregarious} though the interpretation is
tentative.

\hypertarget{Micromitra-coding-176}{}
\emph{Micromitra}: Reported by Williams \emph{et al}.
\citeyearpar{Williams1998Thediversity}.

\hypertarget{Novocrania-coding-176}{}
\emph{Novocrania}: Williams \emph{et al}.
\citeyearpar{Williams2000LinguliformeaCraniiformea} write ``Holocene
craniides have only a single pair of main trunks in both valves,
corresponding to the \emph{vascula} \emph{lateralia}'' -- an observation
reflected in their table 15 (for \emph{Neocrania}).\\
But in contrast, \citet{Williams2007Supplement}, p.~2875, identify the
dorsal valve's canals as a \emph{vascula} \emph{media} in living
cranidds (though both are \emph{lateralia} in Ordoviian craniides). This
character is therefore coded as ambiguous.

\hypertarget{Orthis-coding-176}{}
\emph{Orthis}: From idealised morphology in Williams \emph{et al}.
\citeyearpar{Williams2000LinguliformeaCraniiformea}.

\hypertarget{Pelagodiscus_atlanticus-coding-176}{}
\emph{Pelagodiscus atlanticus}: Following table 6 (for Discinidae) in
Williams \emph{et al}.
\citeyearpar{Williams2000LinguliformeaCraniiformea}.

\hypertarget{Siphonobolus_priscus-coding-176}{}
\emph{Siphonobolus priscus}: Noted in \emph{Siphonobolus} by Havlicek
\citeyearpar{Havlicek1982LingulaceaPaterinacea}.

\hypertarget{Terebratulina-coding-176}{}
\emph{Terebratulina}: ``In modern terebratulides, the \emph{vascula}
\emph{media} are subordinate to the lemniscate or pinnate \emph{vascula}
\emph{genitalia}'' -- \citet{Williams1997Introduction} p417.

\hypertarget{Tomteluva_perturbata-coding-176}{}
\emph{Tomteluva perturbata}: Preservation not adequate to evaluate
\citep{Streng2016Anew}.

\hypertarget{Yuganotheca_elegans-coding-176}{}
\emph{Yuganotheca elegans}: Based on the figures and sketches in
\citet{Zhang2014Anearly} (and supplementary material), the mantle canals
are interpreted as lateral, with no clear \emph{vascula} \emph{media}
present.

\subsection*{\texorpdfstring{{[}177{]} \emph{vascula}
\emph{terminalia}}{{[}177{]} vascula terminalia}}\label{vascula-terminalia}
\addcontentsline{toc}{subsection}{{[}177{]} \emph{vascula}
\emph{terminalia}}

\includegraphics{Brachiopod_phylogeny_files/figure-latex/character-mapping-176.pdf}

\begin{quote}
\textbf{Character 177: Mantle canals: \emph{vascula} \emph{terminalia}}

1: Exclusively marginal (peripheral)\\
2: Directed peripherally and (intero)medially\\
Transformational character.
\end{quote}

Presumed to be connected with setal follicles in life
\citep{Williams1998Thediversity}. See Williams \emph{et al}.
\citeyearpar{Williams2000LinguliformeaCraniiformea} for discussion.

\hypertarget{Acanthotretella_spinosa-coding-177}{}
\emph{Acanthotretella spinosa}: Preservation not clear enough to score
with certainty \citep{Holmer2006Aspinose}.

\hypertarget{Alisina-coding-177}{}
\emph{Alisina}: Interomedial \emph{vascula} \emph{terminalia} not
reported by Williams \emph{et al}.
\citeyearpar{Williams2000LinguliformeaCraniiformea}.

\hypertarget{Askepasma_toddense-coding-177}{}
\emph{Askepasma toddense}, \emph{Micromitra}: Peripheral only
\citep{Williams1998Thediversity, Williams2000LinguliformeaCraniiformea}.

\hypertarget{Botsfordia-coding-177}{}
\emph{Botsfordia}, \emph{Eoobolus}: Following
\citet{Williams1998Thediversity}, appendix 2.

\hypertarget{Glyptoria-coding-177}{}
\emph{Glyptoria}: Following appendix 2 in Williams \emph{et al}.
\citeyearpar{Williams1998Thediversity}.

\hypertarget{Heliomedusa_orienta-coding-177}{}
\emph{Heliomedusa orienta}: Inferred from Jin \& Wang
\citeyearpar{Jin1992Revisionof}.

\hypertarget{Kutorgina_chengjiangensis-coding-177}{}
\emph{Kutorgina chengjiangensis}, \emph{Salanygolina}: Coded uncertain
in appendix 2 in Williams \emph{et al}.
\citeyearpar{Williams1998Thediversity}.

\hypertarget{Lingula-coding-177}{}
\emph{Lingula}: Peripheral and medial for all Lingulata
\citep{Williams2000LinguliformeaCraniiformea}.

\hypertarget{Lingulellotreta_malongensis-coding-177}{}
\emph{Lingulellotreta malongensis}: Not described in Williams \emph{et
al}. \citeyearpar{Williams2000LinguliformeaCraniiformea}.

\hypertarget{Lingulosacculus-coding-177}{}
\emph{Lingulosacculus}: Strong indication of medially directed
\emph{vascula} \emph{terminalia} from \emph{vascula} \emph{lateralia};
see fig. 1.A1 in \citet{Balthasar2009EarlyCambrian}.

\hypertarget{Novocrania-coding-177}{}
\emph{Novocrania}: Peripheral only
\citep[p.158]{Williams2000LinguliformeaCraniiformea}.

\hypertarget{Orthis-coding-177}{}
\emph{Orthis}: See schematics in Williams \emph{et al}.
\citeyearpar{Williams2000LinguliformeaCraniiformea}.

\hypertarget{Pelagodiscus_atlanticus-coding-177}{}
\emph{Pelagodiscus atlanticus}: Following \emph{Lochkothele}
(Discinidae), fig. 43.4a in Williams \emph{et al}.
\citeyearpar{Williams2000LinguliformeaCraniiformea}.

\hypertarget{Siphonobolus_priscus-coding-177}{}
\emph{Siphonobolus priscus}: Not reported in
\citet{Havlicek1982LingulaceaPaterinacea} or
\citet{Williams2000LinguliformeaCraniiformea}.

\hypertarget{Terebratulina-coding-177}{}
\emph{Terebratulina}: Following idealised plectolophous terebratulid of
Emig \citeyearpar{Emig1992Functionaldisposition}.

\section{Lophophore}\label{lophophore}

\subsection*{{[}178{]} Presence}\label{presence-4}
\addcontentsline{toc}{subsection}{{[}178{]} Presence}

\includegraphics{Brachiopod_phylogeny_files/figure-latex/character-mapping-177.pdf}

\begin{quote}
\textbf{Character 178: Lophophore: Presence}

0: Absent\\
1: Present\\
Neomorphic character.
\end{quote}

The brachiopod lophophore arises from the second pair of coelomic sacs
in the larva \citep{Nielsen1991}, and thus cannot be homologous with the
protostomial tentacles of annelids and sipunculans.

\hypertarget{Cotyledion_tylodes-coding-178}{}
\emph{Cotyledion tylodes}: The tentacular crown \citep{Zhang2013} is
interpreted as a lophophore.

\subsection*{{[}179{]} Tentacle disposition}\label{tentacle-disposition}
\addcontentsline{toc}{subsection}{{[}179{]} Tentacle disposition}

\includegraphics{Brachiopod_phylogeny_files/figure-latex/character-mapping-178.pdf}

\begin{quote}
\textbf{Character 179: Lophophore: Tentacle disposition}

1: Single side\\
2: Both sides\\
Transformational character.
\end{quote}

Tentacles may occur along one or both sides of the axis of the
lophophore arm \citep{Carlson1995Phylogeneticrelationships}.

\hypertarget{Acanthotretella_spinosa-coding-179}{}
\emph{Acanthotretella spinosa}: Preservation insufficient to evaluate
\citep{Holmer2006Aspinose}.

\hypertarget{Alisina-coding-179}{}
\emph{Alisina}: Preservation inadequate.

\hypertarget{Cotyledion_tylodes-coding-179}{}
\emph{Cotyledion tylodes}: Tentacles seemingly occupy a single side of
the lophophore \citep{Zhang2013}.

\hypertarget{Flustra-coding-179}{}
\emph{Flustra}: Both sides \citep{Schwaha2015, Shunkina2015}.

\hypertarget{Heliomedusa_orienta-coding-179}{}
\emph{Heliomedusa orienta}: ``Each lophophoral arm bears a row of long,
slender flexible tentacles'' -- \citet{Zhang2009Architectureand}.

\hypertarget{Kutorgina_chengjiangensis-coding-179}{}
\emph{Kutorgina chengjiangensis}: Tentacles ``cannot be confidently
demonstrated in the available specimens.'' --
\citet{Zhang2007Rhynchonelliformeanbrachiopods}.

\hypertarget{Lingula-coding-179}{}
\emph{Lingula}, \emph{Novocrania}, \emph{Pelagodiscus atlanticus},
\emph{Phoronis}, \emph{Terebratulina}: Following coding for higher group
in \citet{Carlson1995Phylogeneticrelationships}, appendix 1, character
36.

\hypertarget{Lingulellotreta_malongensis-coding-179}{}
\emph{Lingulellotreta malongensis}: ``The tentacles are clearly visible,
and closely arranged in a single palisade'' -- \citet{Zhang2004Newdata}.

\hypertarget{Lingulosacculus-coding-179}{}
\emph{Lingulosacculus}: Preservation insufficient to evaluate.

\hypertarget{Longtancunella_chengjiangensis-coding-179}{}
\emph{Longtancunella chengjiangensis}: Inadequately preserved to
evaluate.

\hypertarget{Loxosomella-coding-179}{}
\emph{Loxosomella}: Single side \citep{Nielsen1966}.

\subsection*{{[}180{]} Tentacle rows per side in trocholophe
stage}\label{tentacle-rows-per-side-in-trocholophe-stage}
\addcontentsline{toc}{subsection}{{[}180{]} Tentacle rows per side in
trocholophe stage}

\includegraphics{Brachiopod_phylogeny_files/figure-latex/character-mapping-179.pdf}

\begin{quote}
\textbf{Character 180: Lophophore: Tentacle rows per side in trocholophe
stage}

0: No additional ablabial row\\
1: Adlabial and ablabial row\\
Neomorphic character.
\end{quote}

After Carlson \citeyearpar{Carlson1995Phylogeneticrelationships},
character 37. Lophophore tentacles are commonly arranged into an
ablabial and adlabial row, with ablabial tentacles sometimes added later
in development.

\hypertarget{Flustra-coding-180}{}
\emph{Flustra}, \emph{Loxosomella}: Inapplicable.

\hypertarget{Lingula-coding-180}{}
\emph{Lingula}, \emph{Pelagodiscus atlanticus}, \emph{Phoronis},
\emph{Terebratulina}: Following coding for higher taxon in Carlson
\citeyearpar{Carlson1995Phylogeneticrelationships}, appendix 1,
character 37.

\hypertarget{Novocrania-coding-180}{}
\emph{Novocrania}: Following coding for higher taxon in Carlson
\citeyearpar{Carlson1995Phylogeneticrelationships}, appendix 1,
character 37. Also states in
\citet{Williams2000LinguliformeaCraniiformea}, p.~158.

\subsection*{{[}181{]} Tentacle rows per side in post-trocholophe
stage}\label{tentacle-rows-per-side-in-post-trocholophe-stage}
\addcontentsline{toc}{subsection}{{[}181{]} Tentacle rows per side in
post-trocholophe stage}

\includegraphics{Brachiopod_phylogeny_files/figure-latex/character-mapping-180.pdf}

\begin{quote}
\textbf{Character 181: Lophophore: Tentacle rows per side in
post-trocholophe stage}

0: No additional ablabial row\\
1: Adlabial and ablabial row\\
Neomorphic character.
\end{quote}

After Carlson \citeyearpar{Carlson1995Phylogeneticrelationships},
character 37. Lophophore tentacles are commonly arranged into an
ablabial and adlabial row, with ablabial tentacles sometimes added later
in development (and thus interpreted as a neomorphic addition).

\hypertarget{Acanthotretella_spinosa-coding-181}{}
\emph{Acanthotretella spinosa}: Preservation insufficient to evaluate
\citep{Holmer2006Aspinose}.

\hypertarget{Cotyledion_tylodes-coding-181}{}
\emph{Cotyledion tylodes}: Additional row not evident \citep{Zhang2013}.

\hypertarget{Heliomedusa_orienta-coding-181}{}
\emph{Heliomedusa orienta}: ``The lophophoral arms bear laterofrontal
tentacles with a double row of cilia along their lateral edge, as in
extant lingulid brachiopods'' -- \citet{Zhang2009Architectureand}.

\hypertarget{Kutorgina_chengjiangensis-coding-181}{}
\emph{Kutorgina chengjiangensis}: Tentacles ``cannot be confidently
demonstrated in the available specimens.'' --
\citet{Zhang2007Rhynchonelliformeanbrachiopods}.

\hypertarget{Lingula-coding-181}{}
\emph{Lingula}, \emph{Novocrania}, \emph{Pelagodiscus atlanticus},
\emph{Phoronis}, \emph{Terebratulina}: Following coding for higher taxon
in Carlson \citeyearpar{Carlson1995Phylogeneticrelationships}, appendix
1, character 37.

\hypertarget{Lingulellotreta_malongensis-coding-181}{}
\emph{Lingulellotreta malongensis}: Single palisade
\citep{Zhang2004Newdata}.

\hypertarget{Lingulosacculus-coding-181}{}
\emph{Lingulosacculus}: Preservation insufficient to evaluate.

\hypertarget{Loxosomella-coding-181}{}
\emph{Loxosomella}: \citet{Nielsen1966}.

\hypertarget{Yuganotheca_elegans-coding-181}{}
\emph{Yuganotheca elegans}: ``helical lophophore fringed with a single
row of thick, widely spaced, parallel-sided and hollow tentacles'' --
\citet{Zhang2014Anearly}.

\subsection*{{[}182{]} Median tentacle in early
development}\label{median-tentacle-in-early-development}
\addcontentsline{toc}{subsection}{{[}182{]} Median tentacle in early
development}

\includegraphics{Brachiopod_phylogeny_files/figure-latex/character-mapping-181.pdf}

\begin{quote}
\textbf{Character 182: Lophophore: Median tentacle in early development}

0: Absent\\
1: Present\\
Neomorphic character.
\end{quote}

Following character 28 in \citet{Carlson1995Phylogeneticrelationships}.
Certain taxa exhibit a median tentacle early in development that is lost
at some point in ontogeny.

\hypertarget{Acanthotretella_spinosa-coding-182}{}
\emph{Acanthotretella spinosa}, \emph{Alisina}, \emph{Antigonambonites
planus}, \emph{Askepasma toddense}, \emph{Clupeafumosus socialis},
\emph{Coolinia pecten}, \emph{Dailyatia}, \emph{Eccentrotheca},
\emph{Gasconsia}, \emph{Glyptoria}, \emph{Haplophrentis carinatus},
\emph{Heliomedusa orienta}, \emph{Kutorgina chengjiangensis},
\emph{Lingulellotreta malongensis}, \emph{Lingulosacculus},
\emph{Longtancunella chengjiangensis}, \emph{Micrina},
\emph{Micromitra}, \emph{Mummpikia nuda}, \emph{Namacalathus},
\emph{Nisusia sulcata}, \emph{Orthis}, \emph{Paterimitra},
\emph{Pedunculotheca diania}, \emph{Salanygolina}, \emph{Tomteluva
perturbata}, \emph{Yuganotheca elegans}: Lophophore ontogeny presently
unknown.

\hypertarget{Loxosomella-coding-182}{}
\emph{Loxosomella}: \citet{Nielsen1966}.

\subsection*{{[}183{]} Forms closed loop}\label{forms-closed-loop}
\addcontentsline{toc}{subsection}{{[}183{]} Forms closed loop}

\includegraphics{Brachiopod_phylogeny_files/figure-latex/character-mapping-182.pdf}

\begin{quote}
\textbf{Character 183: Lophophore: Forms closed loop}

1: Diverging laterally\\
2: Closed loop\\
Transformational character.
\end{quote}

Whereas the lophophore of crown-group brachiopods typically forms a
closed loop, those of \emph{Haplophrentis} and \emph{Heliomedusa}
diverge laterally \citep{Moysiuk2017Hyolithsare}.

\hypertarget{Cotyledion_tylodes-coding-183}{}
\emph{Cotyledion tylodes}: Difficult to establish whether tentacles form
closed loop or a horseshoe. Coded as ambiguous.

\hypertarget{Flustra-coding-183}{}
\emph{Flustra}: Two arms of lophophore but tentacles surround mouth in
closed loop \citep{Schwaha2015}.

\hypertarget{Lingulosacculus-coding-183}{}
\emph{Lingulosacculus}: Two diverging arms of the lophophore are
preserved \citep{Balthasar2009EarlyCambrian}.

\hypertarget{Longtancunella_chengjiangensis-coding-183}{}
\emph{Longtancunella chengjiangensis}: Two distinct, diverging arms
reconstructed by \citet{Zhang2007Agregarious}.

\hypertarget{Loxosomella-coding-183}{}
\emph{Loxosomella}: \citet{Nielsen1966}.

\hypertarget{Namacalathus-coding-183}{}
\emph{Namacalathus}: The existence of a lophophore is speculative.

\hypertarget{Nisusia_sulcata-coding-183}{}
\emph{Nisusia sulcata}: No specimens of \emph{Nisusia} preserve the
lophophore.

\subsection*{{[}184{]} Coiling direction}\label{coiling-direction}
\addcontentsline{toc}{subsection}{{[}184{]} Coiling direction}

\includegraphics{Brachiopod_phylogeny_files/figure-latex/character-mapping-183.pdf}

\begin{quote}
\textbf{Character 184: Lophophore: Coiling direction}

1: Anteriad\\
2: Posteriad\\
Transformational character.
\end{quote}

The lophophore arms of \emph{Heliomedusa} and \emph{Haplophrentis} arch
posteriad, rather than anteriad as in lingulids. See
\citet{Zhang2009Architectureand}; \citet{Moysiuk2017Hyolithsare}.

\hypertarget{Acanthotretella_spinosa-coding-184}{}
\emph{Acanthotretella spinosa}, \emph{Lingulellotreta malongensis}: Arms
proceed anteriad before recurving.

\hypertarget{Cotyledion_tylodes-coding-184}{}
\emph{Cotyledion tylodes}: Cannot establish without distinguishing gut
from anus.

\hypertarget{Flustra-coding-184}{}
\emph{Flustra}: Arms reach in anal (i.e.~posterior) direction
\citep{Shunkina2015}.

\hypertarget{Loxosomella-coding-184}{}
\emph{Loxosomella}: Posterior (anal side) of lophophore has short
stretch lacking tentacles.

\hypertarget{Pelagodiscus_atlanticus-coding-184}{}
\emph{Pelagodiscus atlanticus}: ``converging anteriorly and coiling
anterior to the body cavity'' -- \citet{Zhang2009Architectureand}.

\hypertarget{Phoronis-coding-184}{}
\emph{Phoronis}: Coiling in direction of anus (i.e.~posteriad).

\subsection*{{[}185{]} Adjustor muscle}\label{adjustor-muscle}
\addcontentsline{toc}{subsection}{{[}185{]} Adjustor muscle}

\includegraphics{Brachiopod_phylogeny_files/figure-latex/character-mapping-184.pdf}

\begin{quote}
\textbf{Character 185: Lophophore: Adjustor muscle}

0: Absent\\
1: Present\\
Neomorphic character.
\end{quote}

Following character 55 in Carlson
\citeyearpar{Carlson1995Phylogeneticrelationships}. Not possible to code
in most fossil taxa.

\hypertarget{Acanthotretella_spinosa-coding-185}{}
\emph{Acanthotretella spinosa}, \emph{Alisina}, \emph{Antigonambonites
planus}, \emph{Askepasma toddense}, \emph{Clupeafumosus socialis},
\emph{Coolinia pecten}, \emph{Dailyatia}, \emph{Eccentrotheca},
\emph{Gasconsia}, \emph{Glyptoria}, \emph{Haplophrentis carinatus},
\emph{Heliomedusa orienta}, \emph{Kutorgina chengjiangensis},
\emph{Lingulellotreta malongensis}, \emph{Lingulosacculus},
\emph{Longtancunella chengjiangensis}, \emph{Micrina},
\emph{Micromitra}, \emph{Mummpikia nuda}, \emph{Namacalathus},
\emph{Nisusia sulcata}, \emph{Orthis}, \emph{Paterimitra},
\emph{Pedunculotheca diania}, \emph{Salanygolina}, \emph{Tomteluva
perturbata}, \emph{Yuganotheca elegans}: Preservation not adequate to
evaluate presence or absence of this muscle.

\hypertarget{Lingula-coding-185}{}
\emph{Lingula}, \emph{Novocrania}, \emph{Pelagodiscus atlanticus},
\emph{Phoronis}, \emph{Terebratulina}: Following coding for higher taxon
in Carlson \citeyearpar{Carlson1995Phylogeneticrelationships}, appendix
1, character 55.

\section{Digestive tract}\label{digestive-tract}

\subsection*{{[}186{]} Prominent pharynx}\label{prominent-pharynx}
\addcontentsline{toc}{subsection}{{[}186{]} Prominent pharynx}

\includegraphics{Brachiopod_phylogeny_files/figure-latex/character-mapping-185.pdf}

\begin{quote}
\textbf{Character 186: Digestive tract: Prominent pharynx}

0: Absent\\
1: Present\\
Neomorphic character.
\end{quote}

Hyoliths exhibit a prominent protrusible muscular pharynx at the base of
the lophophore \citep{Moysiuk2017Hyolithsare}. This is considered as
potentially equivalent to the anterior projection of the visceral cavity
in \emph{Heliomedusa}, and, by extension, in \emph{Lingulosacculus} and
Lingulotreta.

\hypertarget{Eoobolus-coding-186}{}
\emph{Eoobolus}: Prominent extension of dorsal visceral platform
\citep{Balthasar2009Thebrachiopod}.

\hypertarget{Heliomedusa_orienta-coding-186}{}
\emph{Heliomedusa orienta}: Corresponding to the ``neck'' of the
vase-shaped visceral cavity reported by
\citet{Zhang2009Architectureand}.

\hypertarget{Lingulellotreta_malongensis-coding-186}{}
\emph{Lingulellotreta malongensis}: An anterior projection of the
visceral area is noted by Williams \emph{et al}.
\citeyearpar{Williams2000LinguliformeaCraniiformea} and considered
equivalent to that observed in \emph{Lingulosacculus}
\citep{Balthasar2009EarlyCambrian}.

\hypertarget{Lingulosacculus-coding-186}{}
\emph{Lingulosacculus}: The prominent anterior extension of the visceral
area noted by Balthasar \& Butterfield
\citeyearpar{Balthasar2009EarlyCambrian} is considered as potentially
homologous with that of \emph{Heliomedusa}
\citep{Zhang2009Architectureand} and, by extension, \emph{Haplophrentis}
\citep{Moysiuk2017Hyolithsare}.

\hypertarget{Sipunculus-coding-186}{}
\emph{Sipunculus}: Eversible pharynx (introvert).

\hypertarget{Yuganotheca_elegans-coding-186}{}
\emph{Yuganotheca elegans}: Possibly present, following interpretation
of mouth \citep[see fig. 2c, d in][]{Zhang2014Anearly}.

\subsection*{{[}187{]} Radula}\label{radula}
\addcontentsline{toc}{subsection}{{[}187{]} Radula}

\includegraphics{Brachiopod_phylogeny_files/figure-latex/character-mapping-186.pdf}

\begin{quote}
\textbf{Character 187: Digestive tract: Radula}

0: Absent\\
1: Present\\
Neomorphic character.
\end{quote}

Any apparatus comprising multiple denticulate rows arranged serially in
the sagittal plane is treated as potentially homologous with the
molluscan radula.

\hypertarget{Wiwaxia_corrugata-coding-187}{}
\emph{Wiwaxia corrugata}: \citet{Smith2012M}.

\subsection*{{[}188{]} Oesophageal folds}\label{oesophageal-folds}
\addcontentsline{toc}{subsection}{{[}188{]} Oesophageal folds}

\includegraphics{Brachiopod_phylogeny_files/figure-latex/character-mapping-187.pdf}

\begin{quote}
\textbf{Character 188: Digestive tract: Oesophageal folds}

0: Absent\\
1: Present\\
Neomorphic character.
\end{quote}

Following character 73 in \citet{Ponder1997}; 86 in \citet{Giribet2002}.

\subsection*{{[}189{]} Oral sphincter}\label{oral-sphincter}
\addcontentsline{toc}{subsection}{{[}189{]} Oral sphincter}

\includegraphics{Brachiopod_phylogeny_files/figure-latex/character-mapping-188.pdf}

\begin{quote}
\textbf{Character 189: Digestive tract: Oral sphincter}

0: Absent\\
1: Present\\
Neomorphic character.
\end{quote}

Character 133 in \citet{Grobe2007}.

\hypertarget{Dentalium-coding-189}{}
\emph{Dentalium}: Present, but secondarily reduced.

\subsection*{{[}190{]} Locomotory cilia}\label{locomotory-cilia}
\addcontentsline{toc}{subsection}{{[}190{]} Locomotory cilia}

\includegraphics{Brachiopod_phylogeny_files/figure-latex/character-mapping-189.pdf}

\begin{quote}
\textbf{Character 190: Digestive tract: Foregut: Locomotory cilia}

0: Absent\\
1: Present\\
Neomorphic character.
\end{quote}

Character 66 in \citet{Haszprunar2000}.

\section{Digestive tract: Midgut}\label{digestive-tract-midgut}

\subsection*{{[}191{]} Subdivisions}\label{subdivisions}
\addcontentsline{toc}{subsection}{{[}191{]} Subdivisions}

\includegraphics{Brachiopod_phylogeny_files/figure-latex/character-mapping-190.pdf}

\begin{quote}
\textbf{Character 191: Digestive tract: Midgut: Subdivisions}

0: Not subdivided\\
1: Functional subdivisions\\
Neomorphic character.
\end{quote}

The molluscan midgut is functionally subdivided into a sorting area
(stomach), digestion area (midgut sac or gland), and transport tube
(intestine). Characters 42 in \citet{Haszprunar2000}, 1.38 in
\citet{SPS1996}.

\hypertarget{Wiwaxia_corrugata-coding-191}{}
\emph{Wiwaxia corrugata}: Subdivided, presumably functionally, but with
some ambiguity {[}\citet{Smith2012M};Smith2014{]}.

\subsection*{{[}192{]} Glands}\label{glands-1}
\addcontentsline{toc}{subsection}{{[}192{]} Glands}

\includegraphics{Brachiopod_phylogeny_files/figure-latex/character-mapping-191.pdf}

\begin{quote}
\textbf{Character 192: Digestive tract: Midgut: Glands}

0: Absent\\
1: Present\\
Neomorphic character.
\end{quote}

Characters 1.40, 2.30 and 4.59 in \citet{SPS1996}; 42 in
\citet{Haszprunar2000}.

\hypertarget{Wiwaxia_corrugata-coding-192}{}
\emph{Wiwaxia corrugata}: Annex to midgut interpreted as a gland
\citep{Smith2012M}.

\section{Digestive tract: Anus}\label{digestive-tract-anus}

\subsection*{{[}193{]} Presence}\label{presence-5}
\addcontentsline{toc}{subsection}{{[}193{]} Presence}

\includegraphics{Brachiopod_phylogeny_files/figure-latex/character-mapping-192.pdf}

\begin{quote}
\textbf{Character 193: Digestive tract: Anus: Presence}

1: Absent: digestive tract is blind sac\\
2: Present: through-gut\\
Transformational character.
\end{quote}

The digestive tract may either constitute a blind sac, or a through gut
with anus.

\hypertarget{Glyptoria-coding-193}{}
\emph{Glyptoria}: Scored according to familial level feature.

\hypertarget{Kutorgina_chengjiangensis-coding-193}{}
\emph{Kutorgina chengjiangensis}: Although ``the possibility of a blind
ending may not be completely eliminated {[}\ldots{}{]} the weight of
evidence {[}\ldots{}{]} leads us to reject the possibility of a
blind-ending intestine'' --
\citet{Zhang2007Rhynchonelliformeanbrachiopods}, p.~1399.

\subsection*{{[}194{]} Migration}\label{migration}
\addcontentsline{toc}{subsection}{{[}194{]} Migration}

\includegraphics{Brachiopod_phylogeny_files/figure-latex/character-mapping-193.pdf}

\begin{quote}
\textbf{Character 194: Digestive tract: Anus: Migration}

0: Not migrated: straight gut with posterior anus\\
1: Migrated: anus has migrated posteriad to create U-shaped gut\\
Neomorphic character.
\end{quote}

``The relative position of the mouth and anus in the larvae of
brachiopods and phoronids is similar: posterior anus and anterior
mouth'' -- \citet{Williams2007Supplement}, p.~2884.

\hypertarget{Kutorgina_chengjiangensis-coding-194}{}
\emph{Kutorgina chengjiangensis}: ``Five specimens have an exceptionally
preserved digestive tract, dorsally curved, with a putative
dorso-terminal anus located near the proximal end of a pedicle'' --
\citet{Zhang2007Rhynchonelliformeanbrachiopods}.

\hypertarget{Terebratulina-coding-194}{}
\emph{Terebratulina}: ``In rhynchonelliforms, the gut curves somewhat
into a C-shape and the (blind) anus becomes posteroventral in
position.'' -- \citet{Williams2007Supplement}, p.\\
2884.

\subsection*{{[}195{]} Migration: Within ring of
tentacles}\label{migration-within-ring-of-tentacles}
\addcontentsline{toc}{subsection}{{[}195{]} Migration: Within ring of
tentacles}

\includegraphics{Brachiopod_phylogeny_files/figure-latex/character-mapping-194.pdf}

\begin{quote}
\textbf{Character 195: Digestive tract: Anus: Migration: Within ring of
tentacles}

1: Not within ring of tentacles\\
2: Anterior - within ring of feeding tentacles\\
Transformational character.
\end{quote}

A migrated anus may be located laterally or within the lophophore ring
(as in entoprocts).

\hypertarget{Kutorgina_chengjiangensis-coding-195}{}
\emph{Kutorgina chengjiangensis}: ``Presumed to terminate in a
functional anus located near the proximal end of the pedicle.'' --
\citet{Zhang2007Rhynchonelliformeanbrachiopods}.

\subsection*{{[}196{]} Migration: Position}\label{migration-position}
\addcontentsline{toc}{subsection}{{[}196{]} Migration: Position}

\includegraphics{Brachiopod_phylogeny_files/figure-latex/character-mapping-195.pdf}

\begin{quote}
\textbf{Character 196: Digestive tract: Anus: Migration: Position}

1: Left\\
2: Right\\
3: Dorsally\\
4: Ventrally\\
Transformational character.
\end{quote}

If the anus is not within the ring of tentacles, in which direction is
it oriented?.

\hypertarget{Dentalium-coding-196}{}
\emph{Dentalium}: An alternative interpretation would be that the
posterior of the scaphopod has been extended to generate the relatively
anterior position of the originally ventral anus.

\hypertarget{Flustra-coding-196}{}
\emph{Flustra}: Anus remains on ventral surface. Arguably, rather than
the anus migrating, the dorsal surface of the animal has become
extended.

\hypertarget{Haplophrentis_carinatus-coding-196}{}
\emph{Haplophrentis carinatus}: Opening to the right -- see figures 1,
3, and extended data 5 in Moysiuk \emph{et al}.
\citeyearpar{Moysiuk2017Hyolithsare}. The text states in error that the
anus is to the left of the midline.

\hypertarget{Kutorgina_chengjiangensis-coding-196}{}
\emph{Kutorgina chengjiangensis}: ``Five specimens have an exceptionally
preserved digestive tract, dorsally curved, with a putative
dorso-terminal anus located near the proximal end of a pedicle'' --
\citet{Zhang2007Rhynchonelliformeanbrachiopods}.

\hypertarget{Lingula-coding-196}{}
\emph{Lingula}: ``In the lingulids, the {[}intestine{]} follows an
oblique course anteriorly to open at the anus on the right body wall.''
-- \citet{Williams1997Introduction}, p.~89.

\hypertarget{Lingulellotreta_malongensis-coding-196}{}
\emph{Lingulellotreta malongensis}: ``finally terminating in an anal
opening on the right anterior body wall'' \citep[p.66]{Zhang2007Noteon}.

\hypertarget{Lingulosacculus-coding-196}{}
\emph{Lingulosacculus}: ``This same arrangement occurs in \emph{L.
nuda}, with the looped dark line tracking the same course as the
exceptionally preserved guts of Chengjiang lingulellotretids, including
the median position of its posterior loop and the sharp right turn as it
exits the posterior extension of the ventral valve''
\citep[p.310]{Balthasar2009EarlyCambrian}.

\hypertarget{Longtancunella_chengjiangensis-coding-196}{}
\emph{Longtancunella chengjiangensis}: ``The intestine extends
posteriorly, and then turns right to continue as a tortuous strand,
finally terminating at the latero-median position of the anterior body
wall'' -- \citet{Zhang2007Agregarious}.

\hypertarget{Terebratulina-coding-196}{}
\emph{Terebratulina}: ``In rhynchonelliforms, the gut curves somewhat
into a C-shape and the (blind) anus becomes posteroventral in
position.'' -- \citet{Williams2007Supplement}, p.\\
2884.

\hypertarget{Yuganotheca_elegans-coding-196}{}
\emph{Yuganotheca elegans}: The identification of the ``very poorly
impressed possible anus at the lateral side of the anterior body wall''
is not yet confident, so this character is coded as not presently
available.

\hypertarget{fitch}{\chapter{Fitch parsimony}\label{fitch}}

Parsimony search was conducted in TNT v1.5 \citep{Goloboff2016} using
ratchet and tree drifting heuristics \citep{Goloboff1999, Nixon1999},
repeating the search until the optimal score had been hit by 1500
independent searches:

\begin{quote}
xmult:rat10 drift10 hits 1500 level 4 chklevel 5;
\end{quote}

Searches were conducted under equal weights and results saved to file:

\begin{quote}
piwe-; xmult; {/* Conduct search with equal weighting */}

tsav *TNT/ew.tre;sav;tsav/; {/* Save results to file */}

keep 0; hold 10000; {/* Clear trees from memory */}
\end{quote}

Further searches were conducted under extended implied weighting
\citep{Goloboff1997, Goloboff2014}, under the concavity constants 2, 3,
4.5 and 7:

\begin{quote}
xpiwe=; {/* Enable extended implied weighting */}

piwe=2; xmult; {/* Conduct analysis at k = 2 */}

tsav *TNT/xpiwe2.tre; sav; tsav/; {/* Save results to file */}

keep 0; hold 10000; {/* Clear trees from memory */}

piwe=3; xmult; {/* Conduct analysis at k = 3 */}

tsav *TNT/xpiwe3.tre; sav ;tsav/; {/* Save results to file */}
\end{quote}

We acknowledge the Willi Hennig Society for their sponsorship of the TNT
software.

\section{Results}\label{results-1}









\begin{figure}
\centering
\includegraphics{Brachiopod_phylogeny_files/figure-latex/tnt-iw-consensus-1.pdf}
\caption{\label{fig:tnt-iw-consensus}Strict consensus of all trees recovered by TNT
using Fitch parsimony with implied weighting at all values of \emph{k}, and at the individual
values \emph{k} = 2, 3 and 4.5.
The consensus of all implied weights runs is
not very well resolved, largely due to a few wildcard taxa, particularly
at \(k = 4.5\), which obscures a consistent set of relationships between
the remaining taxa.}
\end{figure}

\begin{figure}
\centering
\includegraphics{Brachiopod_phylogeny_files/figure-latex/tnt-iw-7-24-1.pdf}
\caption{\label{fig:tnt-iw-7-24}Strict consensus of all trees recovered by TNT
using Fitch parsimony with implied weighting, at \emph{k} = 7, NA, NA and NA.}
\end{figure}

\newpage










\begin{figure}
\centering
\includegraphics{Brachiopod_phylogeny_files/figure-latex/tnt-ew-consensus-1.pdf}
\caption{\label{fig:tnt-ew-consensus}Consensus of all trees obtained using equally weighted
Fitch parsimony in TNT. \emph{Mickwitzia} and \emph{Micrina} may equally
parsimoniously be reconstructed in the basal region of the linguliform
or rhynchonelliform lineages; as such, the inclusion of these taxa in
the consensus tree reduces resolution. These taxa were still included in
the analysis used to generate this tree, but were removed from each MPT
before the consensus was calculated in order that the relationships that
are present in each tree might be more easily observed.}
\end{figure}

\hypertarget{bayesian}{\chapter{Bayesian analysis}\label{bayesian}}

Bayesian search was conducted in MrBayes v3.2.6 \citep{Ronquist2012}
using the Mk model \citep{Lewis2001} with gamma-distributed rate
variation across characters:

\begin{quote}
lset coding=variable rates=gamma;
\end{quote}

Branch length was drawn from a dirichlet prior distribution, which is
less informative than an exponential model \citep{Rannala2012}, but
requires a prior mean tree length within about two orders of magnitude
of the true value \citep{Zhang2012}. To satisfy this latter criterion,
we specified the prior mean tree length to be equal to the length of the
most parsimonious tree under equal weights, using a Dirichlet prior with
\(\alpha_T = 1\), \(\beta_T = 1/(\)\emph{equal weights tree
length}\(/\)\emph{number of characters}\()\), \(\alpha = c = 1\):

\begin{quote}
prset brlenspr = unconstrained: gammadir(1, 0.36, 1, 1);
\end{quote}

Neomorphic and transformational characters
\citep[\emph{sensu}][]{Sereno2007} were allocated to two separate
partitions whose proportion of invariant characters and gamma shape
parameters were allowed to vary independently:

\begin{quote}
charset Neomorphic = 1 2 3 4 5 7 9 10 11 12 14 16 19 20 21 24 25 26 27
30 31 32 33 34 37 39 41 48 49 51 52 53 54 55 56 59 63 64 65 66 67 69 70
71 72 73 75 76 77 78 80 81 82 83 84 85 87 90 95 96 98 99 100 101 102 103
104 105 110 111 112 113 114 115 117 119 120 121 123 126 129 130 131 132
133 134 135 136 137 138 139 140 144 145 146 148 150 151 153 154 156 157
158 159 161 162 163 166 167 171 172 173 175 176 178 180 181 182 185 186
187 188 189 190 191 192 194;

charset Transformational = 6 8 13 15 17 18 22 23 28 29 35 36 38 40 42 43
44 45 46 47 50 57 58 60 61 62 68 74 79 86 88 89 91 92 93 94 97 106 107
108 109 116 118 122 124 125 127 128 141 142 143 147 149 152 155 160 164
165 168 169 170 174 177 179 183 184 193 195 196;

partition chartype = 2: Neomorphic, Transformational;

set partition = chartype;

unlink shape=(all) pinvar=(all);
\end{quote}

Neomorphic characters were not assumed to have a symmetrical transition
rate -- that is, the probability of the absent → present transition was
allowed to differ from that of the present → absent transition, being
drawn from a uniform prior:

\begin{quote}
prset applyto=(1) symdirihyperpr=fixed(1.0);
\end{quote}

The rate of variation in neomorphic characters was also allowed to vary
from that of transformational characters:

\begin{quote}
prset applyto=(1) ratepr=variable;
\end{quote}

\emph{Flustra} was selected as an outgroup:

\begin{quote}
outgroup Flustra;
\end{quote}

Four MrBayes runs were executed, each sampling eight chains for
5~000~000 generations, with samples taken every 500 generations. The
first 10\% of samples were discarded as burn-in.

\begin{quote}
mcmcp ngen=5000000 samplefreq=500 nruns=4 nchains=8 burninfrac=0.1;
\end{quote}

A posterior tree topology was derived from the combined posterior sample
of all runs. Convergence was indicated by PSRF = 1.00 and an estimated
sample size of \textgreater{} 200 for each parameter.

\section{Parameter estimates}\label{parameter-estimates}

\section{Results}\label{results-2}

\bibliography{References.bib}


\end{document}
