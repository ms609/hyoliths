\documentclass[]{book}
\usepackage{lmodern}
\usepackage{amssymb,amsmath}
\usepackage{ifxetex,ifluatex}
\usepackage{fixltx2e} % provides \textsubscript
\ifnum 0\ifxetex 1\fi\ifluatex 1\fi=0 % if pdftex
  \usepackage[T1]{fontenc}
  \usepackage[utf8]{inputenc}
\else % if luatex or xelatex
  \ifxetex
    \usepackage{mathspec}
  \else
    \usepackage{fontspec}
  \fi
  \defaultfontfeatures{Ligatures=TeX,Scale=MatchLowercase}
\fi
% use upquote if available, for straight quotes in verbatim environments
\IfFileExists{upquote.sty}{\usepackage{upquote}}{}
% use microtype if available
\IfFileExists{microtype.sty}{%
\usepackage{microtype}
\UseMicrotypeSet[protrusion]{basicmath} % disable protrusion for tt fonts
}{}
\usepackage[margin=1in]{geometry}
\usepackage{hyperref}
\hypersetup{unicode=true,
            pdftitle={Brachiopod origins - Supplementary material - Phylogenetic analysis},
            pdfauthor={Sun, Hai-Jing; Smith, Martin Ross; Zhu, Mao-Yan; Zeng, Han; Zhao, Fang-Chen},
            pdfborder={0 0 0},
            breaklinks=true}
\urlstyle{same}  % don't use monospace font for urls
\usepackage{natbib}
\bibliographystyle{plainnat}
\usepackage{color}
\usepackage{fancyvrb}
\newcommand{\VerbBar}{|}
\newcommand{\VERB}{\Verb[commandchars=\\\{\}]}
\DefineVerbatimEnvironment{Highlighting}{Verbatim}{commandchars=\\\{\}}
% Add ',fontsize=\small' for more characters per line
\usepackage{framed}
\definecolor{shadecolor}{RGB}{248,248,248}
\newenvironment{Shaded}{\begin{snugshade}}{\end{snugshade}}
\newcommand{\KeywordTok}[1]{\textcolor[rgb]{0.13,0.29,0.53}{\textbf{#1}}}
\newcommand{\DataTypeTok}[1]{\textcolor[rgb]{0.13,0.29,0.53}{#1}}
\newcommand{\DecValTok}[1]{\textcolor[rgb]{0.00,0.00,0.81}{#1}}
\newcommand{\BaseNTok}[1]{\textcolor[rgb]{0.00,0.00,0.81}{#1}}
\newcommand{\FloatTok}[1]{\textcolor[rgb]{0.00,0.00,0.81}{#1}}
\newcommand{\ConstantTok}[1]{\textcolor[rgb]{0.00,0.00,0.00}{#1}}
\newcommand{\CharTok}[1]{\textcolor[rgb]{0.31,0.60,0.02}{#1}}
\newcommand{\SpecialCharTok}[1]{\textcolor[rgb]{0.00,0.00,0.00}{#1}}
\newcommand{\StringTok}[1]{\textcolor[rgb]{0.31,0.60,0.02}{#1}}
\newcommand{\VerbatimStringTok}[1]{\textcolor[rgb]{0.31,0.60,0.02}{#1}}
\newcommand{\SpecialStringTok}[1]{\textcolor[rgb]{0.31,0.60,0.02}{#1}}
\newcommand{\ImportTok}[1]{#1}
\newcommand{\CommentTok}[1]{\textcolor[rgb]{0.56,0.35,0.01}{\textit{#1}}}
\newcommand{\DocumentationTok}[1]{\textcolor[rgb]{0.56,0.35,0.01}{\textbf{\textit{#1}}}}
\newcommand{\AnnotationTok}[1]{\textcolor[rgb]{0.56,0.35,0.01}{\textbf{\textit{#1}}}}
\newcommand{\CommentVarTok}[1]{\textcolor[rgb]{0.56,0.35,0.01}{\textbf{\textit{#1}}}}
\newcommand{\OtherTok}[1]{\textcolor[rgb]{0.56,0.35,0.01}{#1}}
\newcommand{\FunctionTok}[1]{\textcolor[rgb]{0.00,0.00,0.00}{#1}}
\newcommand{\VariableTok}[1]{\textcolor[rgb]{0.00,0.00,0.00}{#1}}
\newcommand{\ControlFlowTok}[1]{\textcolor[rgb]{0.13,0.29,0.53}{\textbf{#1}}}
\newcommand{\OperatorTok}[1]{\textcolor[rgb]{0.81,0.36,0.00}{\textbf{#1}}}
\newcommand{\BuiltInTok}[1]{#1}
\newcommand{\ExtensionTok}[1]{#1}
\newcommand{\PreprocessorTok}[1]{\textcolor[rgb]{0.56,0.35,0.01}{\textit{#1}}}
\newcommand{\AttributeTok}[1]{\textcolor[rgb]{0.77,0.63,0.00}{#1}}
\newcommand{\RegionMarkerTok}[1]{#1}
\newcommand{\InformationTok}[1]{\textcolor[rgb]{0.56,0.35,0.01}{\textbf{\textit{#1}}}}
\newcommand{\WarningTok}[1]{\textcolor[rgb]{0.56,0.35,0.01}{\textbf{\textit{#1}}}}
\newcommand{\AlertTok}[1]{\textcolor[rgb]{0.94,0.16,0.16}{#1}}
\newcommand{\ErrorTok}[1]{\textcolor[rgb]{0.64,0.00,0.00}{\textbf{#1}}}
\newcommand{\NormalTok}[1]{#1}
\usepackage{longtable,booktabs}
\usepackage{graphicx,grffile}
\makeatletter
\def\maxwidth{\ifdim\Gin@nat@width>\linewidth\linewidth\else\Gin@nat@width\fi}
\def\maxheight{\ifdim\Gin@nat@height>\textheight\textheight\else\Gin@nat@height\fi}
\makeatother
% Scale images if necessary, so that they will not overflow the page
% margins by default, and it is still possible to overwrite the defaults
% using explicit options in \includegraphics[width, height, ...]{}
\setkeys{Gin}{width=\maxwidth,height=\maxheight,keepaspectratio}
\IfFileExists{parskip.sty}{%
\usepackage{parskip}
}{% else
\setlength{\parindent}{0pt}
\setlength{\parskip}{6pt plus 2pt minus 1pt}
}
\setlength{\emergencystretch}{3em}  % prevent overfull lines
\providecommand{\tightlist}{%
  \setlength{\itemsep}{0pt}\setlength{\parskip}{0pt}}
\setcounter{secnumdepth}{5}
% Redefines (sub)paragraphs to behave more like sections
\ifx\paragraph\undefined\else
\let\oldparagraph\paragraph
\renewcommand{\paragraph}[1]{\oldparagraph{#1}\mbox{}}
\fi
\ifx\subparagraph\undefined\else
\let\oldsubparagraph\subparagraph
\renewcommand{\subparagraph}[1]{\oldsubparagraph{#1}\mbox{}}
\fi

%%% Use protect on footnotes to avoid problems with footnotes in titles
\let\rmarkdownfootnote\footnote%
\def\footnote{\protect\rmarkdownfootnote}

%%% Change title format to be more compact
\usepackage{titling}

% Create subtitle command for use in maketitle
\newcommand{\subtitle}[1]{
  \posttitle{
    \begin{center}\large#1\end{center}
    }
}

\setlength{\droptitle}{-2em}
  \title{Brachiopod origins - Supplementary material - Phylogenetic analysis}
  \pretitle{\vspace{\droptitle}\centering\huge}
  \posttitle{\par}
  \author{Sun, Hai-Jing; Smith, Martin Ross; Zhu, Mao-Yan; Zeng, Han; Zhao,
Fang-Chen}
  \preauthor{\centering\large\emph}
  \postauthor{\par}
  \predate{\centering\large\emph}
  \postdate{\par}
  \date{2018-03-14}

\usepackage{booktabs}

\usepackage{amsthm}
\newtheorem{theorem}{Theorem}[chapter]
\newtheorem{lemma}{Lemma}[chapter]
\theoremstyle{definition}
\newtheorem{definition}{Definition}[chapter]
\newtheorem{corollary}{Corollary}[chapter]
\newtheorem{proposition}{Proposition}[chapter]
\theoremstyle{definition}
\newtheorem{example}{Example}[chapter]
\theoremstyle{definition}
\newtheorem{exercise}{Exercise}[chapter]
\theoremstyle{remark}
\newtheorem*{remark}{Remark}
\newtheorem*{solution}{Solution}
\begin{document}
\maketitle

{
\setcounter{tocdepth}{1}
\tableofcontents
}
\hypertarget{brachiopod-origins}{%
\chapter*{Brachiopod origins}\label{brachiopod-origins}}
\addcontentsline{toc}{chapter}{Brachiopod origins}

This document provides a detailed discussion of analyses of the
\protect\hyperlink{dataset}{morphological dataset} constructed to
accompany Sun \emph{et al.} \citeyearpar{Sun2018}, and their results.

We first discuss the results presented in the main paper, which employ
the algorithm described by Brazeau, Guillerme and Smith
\citeyearpar{Brazeau2018} for correct handling of inapplicable data in a
parsimony setting, and explore how each character is
\protect\hyperlink{reconstructions}{reconstructed} on an optimal tree.

For completeness, we also document the results of
\protect\hyperlink{tnt}{standard Fitch parsimony} analysis, and the
results of \protect\hyperlink{MrBayes}{Bayesian analysis}, neither of
which treat inapplicable data in a logically consistent fashion.

\hypertarget{the-dataset}{%
\chapter{The dataset}\label{the-dataset}}

Analysis was performed on a new matrix of 35 early brachiozoan taxa,
including hyoliths, tommotiids and mickwitziids, which were coded for 95
morphological characters (42 neomorphic, 53 transformational).

The dataset can be viewed and downloaded at Morphobank
(\href{https://morphobank.org/permalink/?P2800}{project 2800}), where
each character is defined and its coding for each taxon discussed.

Characters are coded following the recommendations of Brazeau, Guillerme
and Smith \citep{Brazeau2018}. In brief, we have employed reductive
coding, using a distinct state to mark character inapplicability.
Character specifications follow the model of Sereno
\citeyearpar{Sereno2007}.

We have distinguished between neomorphic and transformational characters
\citep[sensu][]{Sereno2007} by reserving the token \texttt{0} to refer
to the absence of a neomorphic character. The states of transformational
characters are represented by the tokens \texttt{1}, \texttt{2},
\texttt{3}, \ldots{}

Following the recommendations of Brazeau, Guillerme and Smith
\citep[supplementary discussion]{Brazeau2018}, we code the absence of
neomorphic ontologically dependent characters \citep[sensu][]{Vogt2017}
as absence, rather than inapplicability.

\hypertarget{treesearch}{%
\chapter{Parsimony analysis}\label{treesearch}}

The phylogenetic dataset contains a high proportion of inapplicable
codings (404/ 3325 = 12\% of tokens), which are known to introduce error
and bias to phylogenetic reconstruction
(\citep{Maddison1993, Brazeau2018}). As such, phylogenetic search
employed a new algorithm that correctly handles inapplicable data
\citep{Brazeau2018}. This algorithm is implemented in the
\emph{MorphyLib} C library \citep{Brazeau2017Morphylib}, and
phylogenetic search was conducted using the \emph{R} package
\emph{TreeSearch} v0.0.8 \citep{Smith2018TreeSearch}.

\emph{Namacalathus} is included in the matrix but has been excluded from
analysis due to its potentially long branch, which is likely to mislead
analysis.

\hypertarget{search-parameters}{%
\section{Search parameters}\label{search-parameters}}

Heuristic searches were conducted using the parsimony ratchet
\citep{Nixon1999} under equal and implied weights \citep{Goloboff1997}
with a variety of concavity constants. The consensus tree presented in
the main manuscript represents a strict consensus of all trees that are
most parsimonious under one or more of the concavity constants
(\emph{k}) 2, 3, 4.5, 7, 10.5, 16 and 24, an approach that is known to
produce higher accuracy than equal weights at any fixed level of
precision \citep{Smith2017}.

\hypertarget{analysis}{%
\section{Analysis}\label{analysis}}

\hypertarget{load-data}{%
\subsection{Load data}\label{load-data}}

\hypertarget{generate-starting-tree}{%
\subsection{Generate starting tree}\label{generate-starting-tree}}

\emph{Dailyatia} has been selected as an outgroup as camenellans have
been interpreted as the earliest diverging members of the Brachiozoa
\citep{Skovsted2015, Zhao2017}.

\begin{Shaded}
\begin{Highlighting}[]
\NormalTok{nj.tree <-}\StringTok{ }\KeywordTok{NJTree}\NormalTok{(my_data)}
\NormalTok{rooted.tree <-}\StringTok{ }\KeywordTok{EnforceOutgroup}\NormalTok{(nj.tree, }\StringTok{'Dailyatia'}\NormalTok{)}
\NormalTok{start.tree <-}\StringTok{ }\KeywordTok{TreeSearch}\NormalTok{(}\DataTypeTok{tree=}\NormalTok{rooted.tree, }\DataTypeTok{dataset=}\NormalTok{my_data, }\DataTypeTok{maxIter=}\DecValTok{3000}\NormalTok{,}
                         \DataTypeTok{EdgeSwapper=}\NormalTok{RootedNNISwap, }\DataTypeTok{verbosity=}\DecValTok{0}\NormalTok{)}
\end{Highlighting}
\end{Shaded}

\hypertarget{implied-weights-analysis}{%
\subsection{Implied weights analysis}\label{implied-weights-analysis}}

\begin{Shaded}
\begin{Highlighting}[]
\ControlFlowTok{for}\NormalTok{ (k }\ControlFlowTok{in}\NormalTok{ kValues) \{}
\NormalTok{  iw.tree <-}\StringTok{ }\KeywordTok{IWRatchet}\NormalTok{(start.tree, iw_data, }\DataTypeTok{concavity=}\NormalTok{k,}
                       \DataTypeTok{ratchHits =} \DecValTok{60}\NormalTok{, }\DataTypeTok{searchHits=}\DecValTok{55}\NormalTok{,}
                       \DataTypeTok{swappers=}\KeywordTok{list}\NormalTok{(RootedTBRSwap, RootedSPRSwap, RootedNNISwap),}
                       \DataTypeTok{verbosity=}\DecValTok{0}\NormalTok{)}
\NormalTok{  score <-}\StringTok{ }\KeywordTok{IWScore}\NormalTok{(iw.tree, iw_data, }\DataTypeTok{concavity=}\NormalTok{k)}
  \CommentTok{# Write single best tree}
  \KeywordTok{write.nexus}\NormalTok{(iw.tree, }\DataTypeTok{file=}\KeywordTok{paste0}\NormalTok{(}\StringTok{"TreeSearch/hy_iw_k"}\NormalTok{, k, }\StringTok{"_"}\NormalTok{, }\KeywordTok{signif}\NormalTok{(score, }\DecValTok{5}\NormalTok{), }\StringTok{".nex"}\NormalTok{, }\DataTypeTok{collapse=}\StringTok{''}\NormalTok{))}

\NormalTok{  suboptFraction =}\StringTok{ }\FloatTok{0.02}
\NormalTok{  iw.consensus <-}\StringTok{ }\KeywordTok{IWRatchetConsensus}\NormalTok{(iw.tree, iw_data, }\DataTypeTok{concavity=}\NormalTok{k,}
                  \DataTypeTok{swappers=}\KeywordTok{list}\NormalTok{(RootedTBRSwap, RootedNNISwap),}
                  \DataTypeTok{searchHits=}\DecValTok{4}\NormalTok{,}
                  \DataTypeTok{suboptimal=}\NormalTok{score }\OperatorTok{*}\StringTok{ }\NormalTok{suboptFraction,}
                  \DataTypeTok{nSearch=}\DecValTok{150}\NormalTok{, }\DataTypeTok{verbosity=}\NormalTok{0L)}
  \KeywordTok{write.nexus}\NormalTok{(iw.consensus, }\DataTypeTok{file=}\KeywordTok{paste0}\NormalTok{(}\StringTok{"TreeSearch/hy_iw_k"}\NormalTok{, k, }\StringTok{"_"}\NormalTok{, }\KeywordTok{signif}\NormalTok{(}\KeywordTok{IWScore}\NormalTok{(iw.tree, iw_data, }\DataTypeTok{concavity=}\NormalTok{k), }\DecValTok{5}\NormalTok{), }\StringTok{".all.nex"}\NormalTok{, }\DataTypeTok{collapse=}\StringTok{''}\NormalTok{))}
\NormalTok{\}}
\end{Highlighting}
\end{Shaded}

\hypertarget{equal-weights-analysis}{%
\subsection{Equal weights analysis}\label{equal-weights-analysis}}

\begin{Shaded}
\begin{Highlighting}[]
\NormalTok{ew.tree <-}\StringTok{ }\KeywordTok{Ratchet}\NormalTok{(start.tree, my_data, }\DataTypeTok{verbosity=}\NormalTok{0L,}
                   \DataTypeTok{ratchHits =} \DecValTok{10}\NormalTok{, }\DataTypeTok{searchHits=}\DecValTok{55}\NormalTok{,}
                   \DataTypeTok{swappers=}\KeywordTok{list}\NormalTok{(RootedTBRSwap, RootedSPRSwap, RootedNNISwap))}
\KeywordTok{write.nexus}\NormalTok{(best.tree, }\DataTypeTok{file=}\KeywordTok{paste0}\NormalTok{(}\StringTok{"TreeSearch/hy_ew_"}\NormalTok{, }\KeywordTok{Fitch}\NormalTok{(ew.tree, my_data), }\StringTok{".nex"}\NormalTok{, }\DataTypeTok{collapse=}\StringTok{''}\NormalTok{))}

\NormalTok{ew.consensus <-}\StringTok{ }\KeywordTok{RatchetConsensus}\NormalTok{(ew.tree, my_data, }\DataTypeTok{nSearch=}\DecValTok{150}\NormalTok{,}
                                 \DataTypeTok{swappers=}\KeywordTok{list}\NormalTok{(RootedTBRSwap, RootedNNISwap),}
                                 \DataTypeTok{verbosity=}\NormalTok{0L)}
\KeywordTok{write.nexus}\NormalTok{(ew.consensus, }\DataTypeTok{file=}\KeywordTok{paste0}\NormalTok{(}\StringTok{"TreeSearch/hy_ew_"}\NormalTok{, }\KeywordTok{Fitch}\NormalTok{(ew.tree, my_data), }\StringTok{".nex"}\NormalTok{, }\DataTypeTok{collapse=}\StringTok{''}\NormalTok{))}
\end{Highlighting}
\end{Shaded}

\hypertarget{results}{%
\section{Results}\label{results}}

\hypertarget{implied-weights-results}{%
\subsection{Implied weights results}\label{implied-weights-results}}

\begin{Shaded}
\begin{Highlighting}[]
\CommentTok{# Read results from files}
\NormalTok{iw.trees <-}\StringTok{ }\KeywordTok{lapply}\NormalTok{(kValues, }\ControlFlowTok{function}\NormalTok{ (k) \{}
\NormalTok{  iw.best <-}\StringTok{ }\KeywordTok{list.files}\NormalTok{(}\StringTok{'TreeSearch'}\NormalTok{, }
                      \DataTypeTok{pattern=}\KeywordTok{paste0}\NormalTok{(}\StringTok{'hy_iw_k'}\NormalTok{,}
                                     \KeywordTok{gsub}\NormalTok{(}\StringTok{'}\CharTok{\textbackslash{}\textbackslash{}}\StringTok{.'}\NormalTok{, }\StringTok{'}\CharTok{\textbackslash{}\textbackslash{}\textbackslash{}\textbackslash{}}\StringTok{.'}\NormalTok{, k),}
                                     \StringTok{'_}\CharTok{\textbackslash{}\textbackslash{}}\StringTok{d+}\CharTok{\textbackslash{}\textbackslash{}}\StringTok{.?}\CharTok{\textbackslash{}\textbackslash{}}\StringTok{d*}\CharTok{\textbackslash{}\textbackslash{}}\StringTok{.all}\CharTok{\textbackslash{}\textbackslash{}}\StringTok{.nex'}\NormalTok{),}
                      \DataTypeTok{full.names=}\OtherTok{TRUE}\NormalTok{)}
  \CommentTok{# Return:}
  \ControlFlowTok{if}\NormalTok{ (}\KeywordTok{length}\NormalTok{(iw.best) }\OperatorTok{==}\StringTok{ }\DecValTok{0}\NormalTok{) \{}
    \KeywordTok{list}\NormalTok{()}
\NormalTok{  \} }\ControlFlowTok{else}\NormalTok{ \{}
    \KeywordTok{read.nexus}\NormalTok{(iw.best[}\KeywordTok{which.max}\NormalTok{(}\KeywordTok{file.mtime}\NormalTok{(iw.best))])}
\NormalTok{  \}}
\NormalTok{\})}
\end{Highlighting}
\end{Shaded}

\begin{Shaded}
\begin{Highlighting}[]
\NormalTok{omit=}\KeywordTok{c}\NormalTok{(lon, }\StringTok{'Clupeafumosus_socialis'}\NormalTok{,}
       \StringTok{'Heliomedusa_orienta'}\NormalTok{, }\StringTok{'Haplophrentis_carinatus'}\NormalTok{)}
\KeywordTok{ColPlot}\NormalTok{(}\KeywordTok{ConsensusWithout}\NormalTok{(}\KeywordTok{lapply}\NormalTok{(iw.trees, consensus), omit))}
\KeywordTok{ColMissing}\NormalTok{(omit)}
\end{Highlighting}
\end{Shaded}

\begin{figure}
\centering
\includegraphics{Brachiopod_phylogeny_files/figure-latex/unnamed-chunk-1-1.pdf}
\caption{\label{fig:unnamed-chunk-1}Consensus of implied weights analyses at
all values of k}
\end{figure}

\begin{Shaded}
\begin{Highlighting}[]
\CommentTok{# Plot consensus results}
\KeywordTok{par}\NormalTok{(}\DataTypeTok{mfrow=}\KeywordTok{c}\NormalTok{(}\DecValTok{4}\NormalTok{, }\DecValTok{2}\NormalTok{), }\DataTypeTok{mar=}\KeywordTok{rep}\NormalTok{(}\FloatTok{0.2}\NormalTok{, }\DecValTok{4}\NormalTok{))}

\KeywordTok{ColPlot}\NormalTok{(}\KeywordTok{consensus}\NormalTok{(}\KeywordTok{lapply}\NormalTok{(iw.trees, consensus)))}
\KeywordTok{text}\NormalTok{(}\OperatorTok{-}\FloatTok{0.5}\NormalTok{, }\DecValTok{1}\NormalTok{, }\DataTypeTok{pos=}\DecValTok{4}\NormalTok{, }\StringTok{"Consensus of all k values"}\NormalTok{, }\DataTypeTok{cex=}\FloatTok{0.8}\NormalTok{)}

\CommentTok{# Plot results for each value of k}
\ControlFlowTok{for}\NormalTok{ (i }\ControlFlowTok{in} \KeywordTok{seq_along}\NormalTok{(iw.trees)) \{}
  \KeywordTok{ColPlot}\NormalTok{(}\KeywordTok{consensus}\NormalTok{(iw.trees[[i]]))}
  \KeywordTok{text}\NormalTok{(}\DecValTok{1}\NormalTok{, }\DecValTok{1}\NormalTok{, }\KeywordTok{paste0}\NormalTok{(}\StringTok{'k = '}\NormalTok{, kValues[i]), }\DataTypeTok{pos=}\DecValTok{4}\NormalTok{)}
\NormalTok{\}}
\end{Highlighting}
\end{Shaded}

\begin{figure}
\centering
\includegraphics{Brachiopod_phylogeny_files/figure-latex/unnamed-chunk-2-1.pdf}
\caption{\label{fig:unnamed-chunk-2}Implied weights results}
\end{figure}

\hypertarget{equal-weights-results}{%
\subsection{Equal weights results}\label{equal-weights-results}}

\begin{Shaded}
\begin{Highlighting}[]
\NormalTok{ew.best <-}\StringTok{ }\KeywordTok{list.files}\NormalTok{(}\StringTok{'TreeSearch'}\NormalTok{, }\DataTypeTok{pattern=}\StringTok{'hy_ew_}\CharTok{\textbackslash{}\textbackslash{}}\StringTok{d*}\CharTok{\textbackslash{}\textbackslash{}}\StringTok{.nex'}\NormalTok{, }\DataTypeTok{full.names=}\OtherTok{TRUE}\NormalTok{)}
\NormalTok{ew.tree <-}\StringTok{ }\KeywordTok{read.nexus}\NormalTok{(}\DataTypeTok{file=}\NormalTok{ew.best[}\KeywordTok{which.max}\NormalTok{(}\KeywordTok{file.mtime}\NormalTok{(ew.best))])}
\KeywordTok{ColPlot}\NormalTok{(}\KeywordTok{consensus}\NormalTok{(ew.tree))}
\end{Highlighting}
\end{Shaded}

\begin{figure}
\centering
\includegraphics{Brachiopod_phylogeny_files/figure-latex/equal weights results in TreeSearch-1.pdf}
\caption{(\#fig:equal weights results in TreeSearch)Strict consensus of
equal weights results}
\end{figure}

\begin{Shaded}
\begin{Highlighting}[]
\NormalTok{omit <-}\StringTok{ }\KeywordTok{c}\NormalTok{(lon)}
\KeywordTok{ColPlot}\NormalTok{(}\KeywordTok{ConsensusWithout}\NormalTok{(ew.tree, omit))}
\KeywordTok{ColMissing}\NormalTok{(omit)}
\end{Highlighting}
\end{Shaded}

\begin{figure}
\centering
\includegraphics{Brachiopod_phylogeny_files/figure-latex/unnamed-chunk-3-1.pdf}
\caption{\label{fig:unnamed-chunk-3}Strict consensus of equal weights
results, taxa excluded}
\end{figure}

\hypertarget{reconstructions}{%
\chapter{Character reconstructions}\label{reconstructions}}

This page defines each characer and accounts for its coding in
particular taxa. Citations for all codings can be found by browsing the
\protect\hyperlink{dataset}{morphological dataset} on MorphoBank
(\href{https://morphobank.org/permalink/?P2800}{project 2800}).

The \emph{Inapp} \emph{R} package \citep{Brazeau2018} was used to map
each character onto one of the most parsimonious trees (obtained under
implied weighting, \(k = 4.5\)):

\hypertarget{sclerites-in-adult}{%
\section*{Sclerites in adult}\label{sclerites-in-adult}}
\addcontentsline{toc}{section}{Sclerites in adult}

\includegraphics{Brachiopod_phylogeny_files/figure-latex/unnamed-chunk-5-1.pdf}

\textbf{Character 1 : Sclerites in adult }

\begin{quote}
0: `Absent'\\
1: `Present'\\
Neomorphic character.
\end{quote}

Plate-like (wider than tall) skeletal elements, whether mineralized or
non-mineralized.\\
The definition deliberately excludes setae (which are taller than wide).

\hypertarget{sclerites-disposition}{%
\section*{Sclerites: disposition}\label{sclerites-disposition}}
\addcontentsline{toc}{section}{Sclerites: disposition}

\includegraphics{Brachiopod_phylogeny_files/figure-latex/unnamed-chunk-5-2.pdf}

\textbf{Character 2 : Sclerites: disposition }

\begin{quote}
1: `Single skeletal element'\\
2: `Bivalved: scleritome dominated by prominent dorsal and ventral
valve'\\
3: `Multiple skeletal elements with no differentiated pair'\\
Transformational character.
\end{quote}

Taxa in the bivalved condition may bear small additional elements, such
as the L-elements of \emph{Paterimitra} or the stegidial / deltidial
plates of certain brachiopods.

\emph{Paterimitra}: Sclerities of \emph{paterimitra} are opposing
symetrical and fused, not seperated like brachiopods

\emph{Yuganotheca elegans}: The conical element is taken to be
homologous to the colleplax and thus part of the ventral valve.

\hypertarget{sclerites-bivalved-hinge-line-shape}{%
\section*{Sclerites: Bivalved: Hinge line
shape}\label{sclerites-bivalved-hinge-line-shape}}
\addcontentsline{toc}{section}{Sclerites: Bivalved: Hinge line shape}

\includegraphics{Brachiopod_phylogeny_files/figure-latex/unnamed-chunk-5-3.pdf}

\textbf{Character 3 : Sclerites: Bivalved: Hinge line shape }

\begin{quote}
1: `Astrophic'\\
2: `Strophic'\\
Transformational character.
\end{quote}

\emph{Micrina}: See Holmer et al 2008.

\emph{Tomteluva perturbata}: ``Tomteluvid taxa all have a strongly
ventribiconvex, astrophic shell with a unisulcate commissure'' -- Streng
et al. 2016, p5

\emph{Kutorgina chengjiangensis}: Williams et al. (2000, p.~208)
consider the hinge of \emph{Kutorgina} to be stropic, whereas Bassett et
al (2001) argue for an astropic interpretation -- whilst noting that the
arrangement is prominently different from other astrophic taxa. We
therefore code this taxon as ambiguous.

\emph{Longtancunella chengjiangensis}: ``\emph{Longtancunella} has an
oval to subcircular shell with a very short strophic hinge line'' --
Zhang et al. 2011

\emph{Nisusia sulcata}: ``The strophic, articulated shells of the
Kutorginata rotated on simple hinge mechanisms that are different from
those of other rhynchonelliforms'' (Williams et al. p.~208)

\emph{Novocrania}: Craniides have a strophic posterior valve edge
(Williams et al. 2007, table 39 on p.~2853)

\emph{Yuganotheca elegans}: Not evident from fossil material; the
possibility of a short strophic hinge line (as in \emph{Longtancunella})
is difficult to discount.

\emph{Mickwitzia muralensis}: non-strophic

\hypertarget{sclerites-bivalved-apophyses}{%
\section*{Sclerites: Bivalved:
Apophyses}\label{sclerites-bivalved-apophyses}}
\addcontentsline{toc}{section}{Sclerites: Bivalved: Apophyses}

\includegraphics{Brachiopod_phylogeny_files/figure-latex/unnamed-chunk-5-4.pdf}

\textbf{Character 4 : Sclerites: Bivalved: Apophyses }

\begin{quote}
0: `Absent'\\
1: `Present'\\
Neomorphic character.
\end{quote}

Many brachiopods, in addition to \emph{Micrina} and others, bear
tooth-like structures or processes that articulate the two primary
valves.\\
Caution must be applied before taxa are coded as ``absent'', as teeth
can be subtle and may be overlooked.

Kutorginata don't have teeth or dental sockets, but their shells are
articulated by ``two triangular plates formed by dorsal interarea,
bearing oblique ridges on the inner sides'' (Williams et al 2000,
p.~211); this simple hinge mechanism is different from other
rhynchonelliforms (Williams et al. 2000, p.208), but serves an
equivalent purpose and is thus potentially homologous. We thus code
kutorginids as present, using a subsequent character to capture
difference in tooth morphology.

\emph{Tomteluva perturbata}: Tomteluvids {[}\ldots{}{]} lack
articulation structures such as teeth and sockets (Streng et al. 2016)

\emph{Mummpikia nuda}: No articulation structures are evident; instead,
the propareas are rotated inwards (Balthasar 2008). The definition of
Family Obolellidae in Willams et al. (2000) notes that articulation may
be lacking or vestigial in the group.

\emph{Kutorgina chengjiangensis}: ``Articulation characterized by two
triangular plates formed by dorsal interarea, bearing oblique ridges on
the inner sides'' -- Williams et al 2000, p.~211

\emph{Nisusia sulcata}: Pseudodont articulation: teeth formed by distal
lateral extensions from the ventral pseudodeltidium - Holmer et al 2018

\emph{Alisina}: ``Strophic articulation with paired, ventral denticles,
composed of secondary shell'' -- definition of family Trematobolidae in
Williams et al. 2000

\emph{Gasconsia}: ``Articulatory structure comprising ventral cardinal
socket and dorsal hinge plate {[}\ldots{}{]} The shape of the shell
probably correlates strongly with the unique type of articulation, which
consists of a dorsal hinge plate that fits tightly into a cardinal
socket in the ventral valve, with a concave homeodeltidium in the center
of the ventral interarea'' -- Williams et al. 2000, p.184, concerning
order Trimerellida

\emph{Clupeafumosus socialis}: No articulating processes evident or
reported by Topper et al. (2013).

\emph{Ussunia}: ``articulatory structures poorly developed'' -- Williams
et al. 2000, p.~192

\emph{Mickwitzia muralensis}: Not reported by or evident in Balthasar
(2004)

\hypertarget{sclerites-bivalved-apophyses-morphology}{%
\section*{Sclerites: Bivalved: Apophyses:
Morphology}\label{sclerites-bivalved-apophyses-morphology}}
\addcontentsline{toc}{section}{Sclerites: Bivalved: Apophyses:
Morphology}

\includegraphics{Brachiopod_phylogeny_files/figure-latex/unnamed-chunk-5-5.pdf}

\textbf{Character 5 : Sclerites: Bivalved: Apophyses: Morphology }

\begin{quote}
1: `Deltidiodont'\\
2: `Cyrtomatodont'\\
3: `Pseudodont'\\
Transformational character.
\end{quote}

Deltidiodont teeth are simple hinge teeth developed by the distal
accretion of secondary shell; Cyrtomatodont teeth are knoblike or
hook-shaped hinge teeth developed by differential secretion and
resorption of the secondary shell (fig. 322 in Williams et al 2000).

Kutorginata (here represented by \emph{Kutorgina} and \emph{Nisusia})
don't have teeth (apophyses) or dental sockets, but their shells are
articulated by ``two triangular plates formed by dorsal interarea,
bearing oblique ridges on the inner sides'' (Williams et al 2000,
p.~211); this simple hinge mechanism is different from other
rhynchonelliforms (Williams et al. 2000, p.208; table 13 character 30),
and is described as a ``pseudodont articulation'' (Holmer et al. 2018).

\emph{Micrina}: The simple knob-like teeth of \emph{Micrina} show no
evidence of resprobtion or the hook-like shape that characterises
Cyrtomatodont teeth.

\emph{Kutorgina chengjiangensis}: ``Articulation characterized by two
triangular plates formed by dorsal interarea, bearing oblique ridges on
the inner sides'' -- Williams et al 2000, p.~211

\emph{Nisusia sulcata}: The `'teeth'`are formed by the distal lateral
extensions from the ventral\\
pseudodeltidium fitting into the'`sockets'' on the inner side of the
dorsal interarea (Holmer et al 2018).

{[}Coded as ``deltidiodont teeth absent'' in Benedetto (2009).{]}

\emph{Terebratulina}: Cyrtomatodont - see fig. 322 in Williams et al
(2000)

\emph{Antigonambonites planus}: Coded as deltidiodont in Benedetto
(2009).

\emph{Orthis}: Coded as deltidiodont (in Eoorthis) in Benedetto (2009).

\emph{Glyptoria}: Coded as deltidiodont in Benedetto (2009).

\hypertarget{sclerites-bivalved-apophyses-dental-plates}{%
\section*{Sclerites: Bivalved: Apophyses: Dental
plates}\label{sclerites-bivalved-apophyses-dental-plates}}
\addcontentsline{toc}{section}{Sclerites: Bivalved: Apophyses: Dental
plates}

\includegraphics{Brachiopod_phylogeny_files/figure-latex/unnamed-chunk-5-6.pdf}

\textbf{Character 6 : Sclerites: Bivalved: Apophyses: Dental plates }

\begin{quote}
0: `Absent'\\
1: `Present'\\
Neomorphic character.
\end{quote}

Williams et al. 1997 (p362) write: ``Teeth {[}\ldots{}{]} are commonly
supported by a pair of variably disposed plates also built up
exclusively of secondary shell and known as dental plates (Fig. 323.1,
323.3).''

Dewing (2001) elaborates: ``Dental plates are near-vertical, narrow
sheets of shell tissue between the anteromedian edge of the teeth and
floor of the ventral valve. They are a composite structure, resulting
from the growth of teeth over the ridge that bounds the ventral-valve
muscle field.''

Williams et al. 2000 (p.201) write: ``The denticles lack supporting
structures in all Obolellida, but in Naukatida they are supported by an
arcuate plate below the\\
interarea, the anterise (Fig. 119.3a)''.

The anterise is conceivably homologous with the dental plates, thus the
presence of either is coded ``present'' for this character.

\emph{Coolinia pecten}: Coded as present following Dewing (2001), who
seems to use the term Strophomenoids to encompass \emph{Coolinia}, and
attests to the presence of dental plates.

\emph{Nisusia sulcata}: Coded as absent in Benedetto (2009)

\emph{Antigonambonites planus}: Coded as present (well developed) in
Benedetto (2009)

\emph{Orthis}: Coded as present (short and recessive, in Eoorthis) in
Benedetto (2009)

\emph{Gasconsia}: Coded ambiguous to reflect the possibility that the
hinge plate in trimerellids is homologous to the dental plates of other
taxa, and has replaced the teeth themselves as the primary articulatory
mechanism (see Williams et al. 2000, p.~184, for details of the
articulation)

\emph{Glyptoria}: Coded as absent in Benedetto (2009)

\hypertarget{sclerites-bivalved-sockets}{%
\section*{Sclerites: Bivalved:
Sockets}\label{sclerites-bivalved-sockets}}
\addcontentsline{toc}{section}{Sclerites: Bivalved: Sockets}

\includegraphics{Brachiopod_phylogeny_files/figure-latex/unnamed-chunk-5-7.pdf}

\textbf{Character 7 : Sclerites: Bivalved: Sockets }

\begin{quote}
0: `Absent'\\
1: `Present'\\
Neomorphic character.
\end{quote}

Simplified from Bassett et al. (2001) character 16.\\
This character is independent of apophyses, as several taxa bear sockets
without corresponding teeth; the function of these sockets is unknown.\\
See figs 323ff in Williams et al. (1997).

\emph{Tomteluva perturbata}: Tomteluvids {[}\ldots{}{]} lack
articulation structures such as teeth and sockets (Streng et al. 2016)

\emph{Nisusia sulcata}: Coded as absent in Benedetto (2009).

\emph{Antigonambonites planus}: Coded as present in Benedetto (2009)

\emph{Alisina}: ``bearing sockets, bounded by low ridges'' -- Williams
et al. 2000

\emph{Gasconsia}: ``Articulatory structure comprising ventral cardinal
socket and dorsal hinge plate'' -- Williams et al. 2000, p.~184

\emph{Glyptoria}: Coded as absent in Benedetto (2009)

\emph{Ussunia}: Following table 15 in Williams et al. 2000

\emph{Mickwitzia muralensis}: Not reported by or evident in Balthasar
(2004)

\hypertarget{sclerites-bivalved-socket-ridges}{%
\section*{Sclerites: Bivalved: Socket
ridges}\label{sclerites-bivalved-socket-ridges}}
\addcontentsline{toc}{section}{Sclerites: Bivalved: Socket ridges}

\includegraphics{Brachiopod_phylogeny_files/figure-latex/unnamed-chunk-5-8.pdf}

\textbf{Character 8 : Sclerites: Bivalved: Socket ridges }

\begin{quote}
0: `Absent'\\
1: `Present'\\
Neomorphic character.
\end{quote}

After Bassett et al. (2001) character 17. May be difficult to
distinguish from a brachiophore (see Fig 323 in Williams et al 1997), so
the two structures are not distinguished here.

\emph{Tomteluva perturbata}: Tomteluvids {[}\ldots{}{]} lack
articulation structures such as teeth and sockets (Streng et al. 2016)

\emph{Nisusia sulcata}: Coded as absent in Benedetto (2009)

\emph{Antigonambonites planus}: Coded as present in Benedetto (2009)

\emph{Alisina}: ``bearing sockets, bounded by low ridges'' -- Williams
et al. 2000

\emph{Glyptoria}: Coded as absent in Benedetto (2009)

\hypertarget{sclerites-bivalved-enclosing-filtration-chamber}{%
\section*{Sclerites: Bivalved: Enclosing filtration
chamber}\label{sclerites-bivalved-enclosing-filtration-chamber}}
\addcontentsline{toc}{section}{Sclerites: Bivalved: Enclosing filtration
chamber}

\includegraphics{Brachiopod_phylogeny_files/figure-latex/unnamed-chunk-5-9.pdf}

\textbf{Character 9 : Sclerites: Bivalved: Enclosing filtration chamber
}

\begin{quote}
0: `No filtration chamber, or open chamber'\\
1: `Shells close to form enclosed filtration chamber'\\
Neomorphic character.
\end{quote}

In crown-group brachiopods, the two primary shells close to form an
enclosed filtration chamber. Further down the stem, taxa such as
\emph{Micrina} do not.

\hypertarget{sclerites-bivalved-commissure}{%
\section*{Sclerites: Bivalved:
Commissure}\label{sclerites-bivalved-commissure}}
\addcontentsline{toc}{section}{Sclerites: Bivalved: Commissure}

\includegraphics{Brachiopod_phylogeny_files/figure-latex/unnamed-chunk-5-10.pdf}

\textbf{Character 10 : Sclerites: Bivalved: Commissure }

\begin{quote}
1: `Rectimarginate'\\
2: `Uniplicate'\\
3: `Sulcate'\\
Transformational character.
\end{quote}

The anterior commissure can be rectimarginate (i.e.~straight),
uniplicate (i.e.~median sulcus in ventral valve), or sulcate (with
median sulcus in dorsal valve).

\emph{Kutorgina chengjiangensis}: Following Appendix 2 in Williams et
al. (1998).

\emph{Salanygolina}: Following Appendix 2 in Williams et al. (1998).

\emph{Askepasma toddense}: ``ventral valve weakly to moderately
sulcate'' (Topper et al. 2013); a similar description is provided by
Williams et al. (2000).

\emph{Micromitra}: Following Appendix 2 in Williams et al. (1998).

\emph{Terebratulina}: ``Anterior commissure rectimarginate to
uniplicate'' -- uniplicate in fig. 1425.1c of Williams et al. (2006).

\emph{Glyptoria}: Following Appendix 2 in Williams et al. (1998).

\hypertarget{sclerites-bivalved-muscle-scars-ventral}{%
\section*{Sclerites: Bivalved: Muscle scars:
Ventral}\label{sclerites-bivalved-muscle-scars-ventral}}
\addcontentsline{toc}{section}{Sclerites: Bivalved: Muscle scars:
Ventral}

\includegraphics{Brachiopod_phylogeny_files/figure-latex/unnamed-chunk-5-11.pdf}

\textbf{Character 11 : Sclerites: Bivalved: Muscle scars: Ventral }

\begin{quote}
0: `Absent'\\
1: `Present'\\
Neomorphic character.
\end{quote}

After Bassett et al. (2001) character 6

\emph{Micrina}: Prominent ventral muscle scars - see e.g.~Holmer et al
2008, fig. 1f

\emph{Alisina}: Muscle scars scored based on \emph{Alisina} compleyensis
(Balthasar et al. 2001)

\emph{Mickwitzia muralensis}: Scars absent; instead, cones ornament
shell''s internal surface.

\hypertarget{sclerites-bivalved-muscle-scars-adjustor}{%
\section*{Sclerites: Bivalved: Muscle scars:
Adjustor}\label{sclerites-bivalved-muscle-scars-adjustor}}
\addcontentsline{toc}{section}{Sclerites: Bivalved: Muscle scars:
Adjustor}

\includegraphics{Brachiopod_phylogeny_files/figure-latex/unnamed-chunk-5-12.pdf}

\textbf{Character 12 : Sclerites: Bivalved: Muscle scars: Adjustor }

\begin{quote}
0: `Absent'\\
1: `Present'\\
Neomorphic character.
\end{quote}

After Bassett et al. (2001) character 7.\\
This character is contingent on the presence of a pedicle. Extreme
caution must be used in inferring an absent state, as adjustor scars can
be extremely difficult to distinguish from the adductor scars.

\emph{Alisina}: Muscle scars scored based on \emph{Alisina} comleyensis
(Bassett et al. 2001).

The presence of an adjustor is marked NPA as it is not clear that a
scar, if present, could be distinguished from the diminutive muscle
scars present.

\emph{Gasconsia}: No mention of an adjustor muscle in \emph{Gasconsia}
or Trimerellida more generally on pp.~184-185 of Williams et al. 2000,
nor in discussion in Williams et al. 2007 (p.~2850). Coded as absent.

\emph{Clupeafumosus socialis}: Not known in any acrotretid (Williams et
al. 2000); not evident in \emph{Clupeafumosus} (Topper et al. 2013).

\emph{Mickwitzia muralensis}: Scars absent; instead, cones ornament
shell''s internal surface.

\hypertarget{sclerites-bivalved-muscle-scars-dorsal-adductor}{%
\section*{Sclerites: Bivalved: Muscle scars: Dorsal
adductor}\label{sclerites-bivalved-muscle-scars-dorsal-adductor}}
\addcontentsline{toc}{section}{Sclerites: Bivalved: Muscle scars: Dorsal
adductor}

\includegraphics{Brachiopod_phylogeny_files/figure-latex/unnamed-chunk-5-13.pdf}

\textbf{Character 13 : Sclerites: Bivalved: Muscle scars: Dorsal
adductor }

\begin{quote}
1: `Dispersed'\\
2: `Radially arranged'\\
3: `Quadripartite'\\
Transformational character.
\end{quote}

After Bassett et al. (2001) character 8, and Williams et al. (1996,
character 35; 2000, p.~160, character 54)

In the dorsal valve, the anterior and posterior adductor scars of
articulated brachiopods form a single (quadripartite) muscle field
(Williams et al. 2000, p.~201)

In contrast, the anterior and posterior scars of e.g.~trimerellids have
prominently separate attachment points, with anterior and posterior
muscle fields clearly distinct, and coded as ``dispersed''.

In e.g.~kutorginates, adductor muscles are separated into left and right
fields; the same is the case in lingulids, where there are more separate
muscle groups and the left and right fields conspire to produce a radial
arrangement; both of these configurations are scored as ``radially
arranged''.

\emph{Haplophrentis carinatus}: Laterally dispersed, based on
interpretation of Moysiuk et al. (2017), and consistent with general
situation in hyoliths (see Dzik 1980).

\emph{Coolinia pecten}: ``radially arranged adductor scars'' -- Bassett
\& Popov 2017, p1

\emph{Salanygolina}: ``The dorsal valve of \emph{Salanygolina} has a
radial arrangement of adductor muscle scars and the scars of
posteromedially placed internal oblique muscles, which are also
characteristic of paterinates and chileates'' -- Holmer et al. (2009)

\emph{Heliomedusa orienta}: Distinct anterior and posterior fields (Chen
et al. 2007); coded as ``dispersed'' by Williams et al. (2000) in table
15.

\emph{Micromitra}: Williams et al. (1998) code as ``dispersed'', but
have a less divided scheme of character states and disagree with other
sources in some codings (e.g.~Bassett et al. 2001, in Kutorginates).
Williams et al. (2000) do not describe \emph{Micromitra} musculature and
we were unable to find any reliable description of the scars, so we code
as ``not presently available''.

\emph{Pelagodiscus atlanticus}: Discinids scored as ``open,
quadripartite'' by Williams et al. (1996)

\emph{Novocrania}: Craniids scored as ``open, quadripartite'' by
Williams et al. (1996)

\emph{Terebratulina}: Coded as ``grouped, quadripartite'' by Williams et
al. (1996)

\emph{Antigonambonites planus}: Treatise

\emph{Alisina}: Following Williams et al. (2000) table 15 (their
character 54)

\emph{Gasconsia}: Following the coding of Williams et al. (2000), table
15.

\emph{Glyptoria}: Scored as ``dispersed'' by Williams et al.
(1998)\ldots{} but then so is \emph{Kutorgina}, which Bassett et al
(2001) score as radial.

Williams et al. (2000) state, for superfamily Protorthida, ``dorsal
adductor scars probably linear'', which fits in the category of
``radial'' employed herein -- so that''s what we follow.

\emph{Clupeafumosus socialis}: Quadripartite, following reconstruction
of Williams (2000), fig. 51.

\emph{Ussunia}: Following table 15 in Williams et al. 2000

\emph{Mickwitzia muralensis}: Scars absent; instead, cones ornament
shell''s internal surface.

\hypertarget{sclerites-bivalved-muscle-scars-adductors-position}{%
\section*{Sclerites: Bivalved: Muscle scars: Adductors:
Position}\label{sclerites-bivalved-muscle-scars-adductors-position}}
\addcontentsline{toc}{section}{Sclerites: Bivalved: Muscle scars:
Adductors: Position}

\includegraphics{Brachiopod_phylogeny_files/figure-latex/unnamed-chunk-5-14.pdf}

\textbf{Character 14 : Sclerites: Bivalved: Muscle scars: Adductors:
Position }

\begin{quote}
1: `Oblique'\\
2: `At high angle'\\
Transformational character.
\end{quote}

Position of adductor muscles relative to commissural plane.\\
After Bassett et al. (2001) character 11.

\emph{Coolinia pecten}: Not reported by Williams et al. (2000), nor
Bassett \& Popov (2017), nor explicitly by Dewing (2001).

\emph{Pelagodiscus atlanticus}: Musculature considered essentially
equivalent to \emph{Lingula} by Williams et al 2000, so \emph{Lingula}
coding followed here.

\emph{Gasconsia}: See discussion under Trimerellida in Williams et al.
(2000).

\emph{Mickwitzia muralensis}: Scars absent; instead, cones ornament
shell''s internal surface.

\emph{Eoobolus}: ``\emph{Eoobolus} should have anterior and posterior
adductors and a variety of oblique muscles which were probably arranged
in criss-crossing pairs'' -- Balthasar 2009

\hypertarget{sclerites-bivalved-muscle-scars-dermal-muscles}{%
\section*{Sclerites: Bivalved: Muscle scars: Dermal
muscles}\label{sclerites-bivalved-muscle-scars-dermal-muscles}}
\addcontentsline{toc}{section}{Sclerites: Bivalved: Muscle scars: Dermal
muscles}

\includegraphics{Brachiopod_phylogeny_files/figure-latex/unnamed-chunk-5-15.pdf}

\textbf{Character 15 : Sclerites: Bivalved: Muscle scars: Dermal muscles
}

\begin{quote}
0: `Absent or weakly developed'\\
1: `Strongly developed'\\
Neomorphic character.
\end{quote}

Based on character 11 in Zhang et al. (2014).\\
Well developed dermal muscles present in the body wall of recent
lingulates, which are absent in all calcareous-shelled brachiopods.
These muscles are responsible for the hydraulic shell-opening mechanism,
and possibly present in all organophosphatic-shelled brachiopods, with
the possible exception of the paterinates (Williams et al., 2000, p.~32)

\emph{Tomteluva perturbata}: Though Williams et al. (2000, P.32) state
that these muscles are absent in all carbonate-shelled brachiopods,
their existence cannot be discounted with certainty in this taxon, which
is therefore coded not presently available.

\emph{Mummpikia nuda}: Though Williams et al. (2000, P.32) state that
these muscles are absent in all carbonate-shelled brachiopods, their
existence cannot be discounted with certainty in this taxon, which is
therefore coded not presently available.

\emph{Coolinia pecten}: According to the statement of Williams et al.
(2000, P.32) that these muscle are absent in all carbonate- shelled
brachiopods.

\emph{Kutorgina chengjiangensis}: According to the statement of Williams
et al. (2000, P.32) that these muscle are absent in all carbonate-
shelled brachiopods, and the coding for kutorginids in Zhang et al.
(2014).

\emph{Salanygolina}: According to the statement of Williams et al.
(2000, P.32) that these muscle are absent in all carbonate- shelled
brachiopods.

\emph{Askepasma toddense}: According to the statement of Williams et al.
(2000, P.32) that the presence of these muscles in paterinates is
uncertain

\emph{Micromitra}: Williams et al. (2000, P.32) are uncertain about the
presence of these muscles in the paterinates. Zhang et al. (2014) code
absence in Paterinida, but without specifying evidence; we follow their
coding here.

\emph{Nisusia sulcata}: According to the statement of Williams et al.
(2000, P.32) that these muscle are absent in all carbonate- shelled
brachiopods.

\emph{Pelagodiscus atlanticus}: Musculature considered essentially
equivalent to \emph{Lingula} by Williams et al 2000, so \emph{Lingula}
coding followed here.

\emph{Novocrania}: Following Zhang et al. (2014), and the statement of
Williams et al. (2000) that such muscles are absent in all
calcite-shelled brachiopods.

\emph{Terebratulina}: Williams et al. (2000, P.32) state that these
muscles are absent in all carbonate-shelled brachiopods.

\emph{Antigonambonites planus}: According to the statement of Williams
et al. (2000, P.32) that these muscle are absent in all carbonate-
shelled brachiopods.

\emph{Alisina}: According to the statement of Williams et al. (2000,
P.32) that these muscle are absent in all carbonate- shelled
brachiopods.

\emph{Orthis}: According to the statement of Williams et al. (2000,
P.32) that these muscle are absent in all carbonate- shelled
brachiopods.

\emph{Gasconsia}: According to the statement of Williams et al. (2000,
P.32) that these muscle are absent in all carbonate- shelled
brachiopods.

\emph{Glyptoria}: According to the statement of Williams et al. (2000,
P.32) that these muscle are absent in all carbonate- shelled
brachiopods.

\emph{Clupeafumosus socialis}: This character is coded based on the
score of Acrotreta in Zhang et al. (2014), and statement in Williams et
al. (2000, P.32).

\emph{Eoobolus}: Not remarked upon by Balthasar (2009).

\hypertarget{sclerites-bivalved-muscle-scars-unpaired-median-levator-ani}{%
\section*{Sclerites: Bivalved: Muscle scars: Unpaired median (levator
ani)}\label{sclerites-bivalved-muscle-scars-unpaired-median-levator-ani}}
\addcontentsline{toc}{section}{Sclerites: Bivalved: Muscle scars:
Unpaired median (levator ani)}

\includegraphics{Brachiopod_phylogeny_files/figure-latex/unnamed-chunk-5-16.pdf}

\textbf{Character 16 : Sclerites: Bivalved: Muscle scars: Unpaired
median (levator ani) }

\begin{quote}
0: `Absent'\\
1: `Present'\\
Neomorphic character.
\end{quote}

The levator ani is a diminutive unpaired medial muscle found in certain
calcitic brachiopods (Williams et al. 2000; see fig. 89, character 34 in
table 13)

\emph{Coolinia pecten}: Not reported in Dewing (2001)

\emph{Kutorgina chengjiangensis}: Following table 13 in Williams et al.
2000

\emph{Heliomedusa orienta}: Poor preservation of minor muscle scars
noted by Chen et al. (2007)

\emph{Nisusia sulcata}: Following table 13 in Williams et al. 2000

\emph{Pelagodiscus atlanticus}: Musculature considered essentially
equivalent to \emph{Lingula} by Williams et al 2000, so \emph{Lingula}
coding followed here.

\emph{Novocrania}: Following table 13 in Williams et al. 2000 (for
\emph{Novocrania})

\emph{Alisina}: Following table 13 in Williams et al. 2000

\emph{Gasconsia}: Following table 13 in Williams et al. 2000

\emph{Craniops}: See fig. 90 in Williams et al. 2000

\emph{Ussunia}: Following table 15 in Williams et al. 2000

\emph{Mickwitzia muralensis}: Scars absent; instead, cones ornament
shell''s internal surface.

\hypertarget{sclerites-bivalved-muscle-scars-dorsal-diductor}{%
\section*{Sclerites: Bivalved: Muscle scars: Dorsal
diductor}\label{sclerites-bivalved-muscle-scars-dorsal-diductor}}
\addcontentsline{toc}{section}{Sclerites: Bivalved: Muscle scars: Dorsal
diductor}

\includegraphics{Brachiopod_phylogeny_files/figure-latex/unnamed-chunk-5-17.pdf}

\textbf{Character 17 : Sclerites: Bivalved: Muscle scars: Dorsal
diductor }

\begin{quote}
0: `Absent'\\
1: `Present'\\
Neomorphic character.
\end{quote}

After Bassett et al. (2001) character 9

\emph{Acanthotretella}: Not observable in \emph{Acanthotretella} itself,
so coded as ambiguous -- though it is likely based on the anticipated
phylogenetic affinities of \emph{Acanthotretella} that the muscles are
absent.

\emph{Gasconsia}: Internal oblique muscles serve as diductors.

\emph{Clupeafumosus socialis}: Not reported by Topper et al. (2013), nor
reconstructed in generic Acrotretid by Williams et al. (2000)

\hypertarget{sclerites-bivalved-muscle-scars-dorsal-diductor-position}{%
\section*{Sclerites: Bivalved: Muscle scars: Dorsal diductor:
position}\label{sclerites-bivalved-muscle-scars-dorsal-diductor-position}}
\addcontentsline{toc}{section}{Sclerites: Bivalved: Muscle scars: Dorsal
diductor: position}

\includegraphics{Brachiopod_phylogeny_files/figure-latex/unnamed-chunk-5-18.pdf}

\textbf{Character 18 : Sclerites: Bivalved: Muscle scars: Dorsal
diductor: position }

\begin{quote}
1: `Close to commissural plane'\\
2: `Oblique to commissural plane'\\
3: `At high angle to commissural plane'\\
Transformational character.
\end{quote}

After Bassett et al. (2001) character 10.

\hypertarget{sclerites-dorsal-valve-growth-direction}{%
\section*{Sclerites: Dorsal valve: Growth
direction}\label{sclerites-dorsal-valve-growth-direction}}
\addcontentsline{toc}{section}{Sclerites: Dorsal valve: Growth
direction}

\includegraphics{Brachiopod_phylogeny_files/figure-latex/unnamed-chunk-5-19.pdf}

\textbf{Character 19 : Sclerites: Dorsal valve: Growth direction }

\begin{quote}
1: `Holoperipheral'\\
2: `Mixoperipheral'\\
3: `Hemiperipheral'\\
Transformational character.
\end{quote}

See Fig. 284 in Williams et al (1997).\\
The growth direction dictates the attitude of the cardinal area relative
to the hinge, which does not therefore represent an independent
character.\\
Crudely put, if, viewed from a dorsal position, the umbo falls within
the outer margin of the shell, growth is holoperipheral; if it falls
outside the margin, it is mixoperipheral; if it falls exactly on the
margin, it is hemiperipheral.

\emph{Micrina}: See Holmer et al. (2008)

\emph{Paterimitra}: S2 and L sclerites are clearly holoperipheral. See
Larsson et al. 2014, fig. 2.

\emph{Heliomedusa orienta}: ``holoperipheral growth in dorsal valve'' --
Williams et al. 2007.

The insinuation from Zhang et al. (2009) is that Chen et al. (2007)
misidentify the dorsal valve as the ventral valve.

\emph{Clupeafumosus socialis}: Appears hemiperipheral in fig. 3 in
Topper et al (2013), though bordering on holoperipheral, so scored as
ambiguous.

\emph{Craniops}: ``both valves with growth holoperipheral'' - Williams
et al. 2000 p164

\emph{Ussunia}: Following description of order in Williams et al. 2000.

\hypertarget{sclerites-dorsal-valve-posterior-surface-differentiated}{%
\section*{Sclerites: Dorsal valve: Posterior surface:
Differentiated}\label{sclerites-dorsal-valve-posterior-surface-differentiated}}
\addcontentsline{toc}{section}{Sclerites: Dorsal valve: Posterior
surface: Differentiated}

\includegraphics{Brachiopod_phylogeny_files/figure-latex/unnamed-chunk-5-20.pdf}

\textbf{Character 20 : Sclerites: Dorsal valve: Posterior surface:
Differentiated }

\begin{quote}
0: `Posterior shell not differentiated'\\
1: `Posterior shell forms distinct cardinal area or pseudointerarea'\\
Neomorphic character.
\end{quote}

In shells that grow by mixoperipheral growth, the triangular area
subtended between each apex and the posterior ends of the lateral
margins is termed the cardinal area. In shells with holoperipheral
growth, a flattened surface on the posterior margin of the valve is
termed a pseudointerarea (paraphrasing Williams et al. 1997).

In order for this character to be independent of a shell's growth
direction, we do not distinguish between a ``cardinal area'',
``interarea'' or ``pseudointerarea''.

\emph{Pedunculotheca diania}: Pseudointerarea.

\emph{Micrina}: = Sellate sclerite duplicature (Holmer et al. 2008)

\emph{Paterimitra}: Pseudointerarea.

\emph{Haplophrentis carinatus}: A very short pseudointerarea appears to
be present (Moysiuk et al. 2017)

\emph{Lingulosacculus}: Unclear from fossil material.

\emph{Tomteluva perturbata}: Cardinal area (interarea) present.

\emph{Mummpikia nuda}: Absent.

\emph{Coolinia pecten}: Cardinal area (interarea) present.

\emph{Kutorgina chengjiangensis}: Cardinal area (interarea) present.

\emph{Salanygolina}: Cardinal area (interarea) present.

\emph{Lingula}: Pseudointerarea present, following Williams et al.
(2000), table 6.

\emph{Acanthotretella}: Pseudointerarea present, following
Siphonotretidae coding in Williams et al. (2000), table 6.

\emph{Heliomedusa orienta}: Pseudointerea in ventral valve, but not
dorsal valve (Williams et al. 2000, 2007)

\emph{Longtancunella chengjiangensis}: Zhang et al. (2011) note that
``all evidence of a pseudointerarea is lacking'', but the
two-dimensional preservation style of Chengjiang material makes details
of dorsal valve difficult to distinguish, and the possibility of a
diminutive pseudointerarea cannot be excluded with total confidence.

\emph{Lingulellotreta malongensis}: Pseudointerarea present, following
Williams et al. (2000), table 6.

\emph{Askepasma toddense}: Well-defined pseudointerarea (Williams et al.
2000, p153)

\emph{Micromitra}: ``Dorsal pseudointerarea usually well defined, low,
anacline to catacline'' - Williams et al. 2000

\emph{Nisusia sulcata}: Cardinal area (interarea) present -- with
reference to Holmer et al. (2018).

\emph{Pelagodiscus atlanticus}: Absent, following entry for Discinidae
in Williams et al. (2000), table 6.

\emph{Novocrania}: Pseudointerarea

\emph{Terebratulina}: Interarea present

\emph{Antigonambonites planus}: Cardinal area (interarea) present.

\emph{Alisina}: Cardinal area (interarea) present.

\emph{Orthis}: Cardinal area (interarea) present.

\emph{Glyptoria}: Cardinal area (interarea) present.

\emph{Clupeafumosus socialis}: Pseudointerarea present; figured by
Topper et al. (2013), fig. 3j.

\emph{Yuganotheca elegans}: A differentiated region is not obvious in
fossil material or its reconstruction (Zhang et al. 2014), but the
two-dimensional preservation style of Chengjiang material makes details
of dorsal valve difficult to distinguish, and the possibility of a
diminutive pseudointerarea cannot be excluded with confidence.

\emph{Craniops}: ``Only some craniopsids (Lingulapholis, Pseudopholidops
{[}not \emph{Craniops}{]}) have well-developed pseudointerareas.'' --
Williams et al. 2000

\emph{Ussunia}: Following table 15 in Williams et al. 2000

\emph{Mickwitzia muralensis}: Shell flat.

\hypertarget{sclerites-dorsal-valve-differentiated-posterior-surface-morphology}{%
\section*{Sclerites: Dorsal valve: Differentiated posterior surface:
Morphology}\label{sclerites-dorsal-valve-differentiated-posterior-surface-morphology}}
\addcontentsline{toc}{section}{Sclerites: Dorsal valve: Differentiated
posterior surface: Morphology}

\includegraphics{Brachiopod_phylogeny_files/figure-latex/unnamed-chunk-5-21.pdf}

\textbf{Character 21 : Sclerites: Dorsal valve: Differentiated posterior
surface: Morphology }

\begin{quote}
0: `Curved lateral profile'\\
1: `Planar lateral profile'\\
Neomorphic character.
\end{quote}

It is possible for a cardinal area or pseudointerarea to be distinct
from the anterior part of the shell, yet to remain curved in lateral
profile.

Taking an undifferentiated posterior margin as primitive, the primitive
condition is curved -- flattening of the posterior margin represents an
additional modification that can only occur once the posterior margin is
differentiated.

\emph{Mummpikia nuda}: Posterior surface cannot be flat if it is not
differentiated.

\emph{Heliomedusa orienta}: Posterior surface cannot be flat if it is
not differentiated.

\emph{Micromitra}: Essentially straight; see fig. 3.7 in Ushatinskaya
2016

\emph{Pelagodiscus atlanticus}: Posterior surface cannot be flat if it
is not differentiated.

\emph{Gasconsia}: Posterior surface cannot be flat if it is not
differentiated.

\emph{Clupeafumosus socialis}: Truncated but essentially planar surface;
see e.g.~p196 of Topper et al. 2013

\emph{Ussunia}: Posterior surface cannot be flat if it is not
differentiated.

\emph{Mickwitzia muralensis}: Posterior surface cannot be flat if it is
not differentiated.

\emph{Eoobolus}: Essentially planar

\hypertarget{sclerites-dorsal-valve-posterior-surface-medial-groove}{%
\section*{Sclerites: Dorsal valve: Posterior surface: Medial
groove}\label{sclerites-dorsal-valve-posterior-surface-medial-groove}}
\addcontentsline{toc}{section}{Sclerites: Dorsal valve: Posterior
surface: Medial groove}

\includegraphics{Brachiopod_phylogeny_files/figure-latex/unnamed-chunk-5-22.pdf}

\textbf{Character 22 : Sclerites: Dorsal valve: Posterior surface:
Medial groove }

\begin{quote}
0: `Absent'\\
1: `Present'\\
Neomorphic character.
\end{quote}

Following character 29 in Williams et al. (2000), table 9 (which relates
to pseudointerarea)

\emph{Acanthotretella}: The dorsal pseudointerarea is poorly preserved,
but appears to have a median groove (Holmer \& Caron, 2006)

\emph{Heliomedusa orienta}: ``A posteriorly protruding dorsal
pseudointerarea with no median groove and no flexure lines'' - Chen et
al. 2007

\emph{Lingulellotreta malongensis}: Dorsal pseudointerarea with wide,
concave median groove and short propareas" - Williams et al 2000

\emph{Clupeafumosus socialis}: Present; figured by Topper et al. (2013),
fig. 3j.

\emph{Eoobolus}: Prominent medial groove

\hypertarget{sclerites-dorsal-valve-posterior-surface-notothyrium}{%
\section*{Sclerites: Dorsal valve: Posterior surface:
Notothyrium}\label{sclerites-dorsal-valve-posterior-surface-notothyrium}}
\addcontentsline{toc}{section}{Sclerites: Dorsal valve: Posterior
surface: Notothyrium}

\includegraphics{Brachiopod_phylogeny_files/figure-latex/unnamed-chunk-5-23.pdf}

\textbf{Character 23 : Sclerites: Dorsal valve: Posterior surface:
Notothyrium }

\begin{quote}
0: `Absent'\\
1: `Present'\\
Neomorphic character.
\end{quote}

A notothyrium is an opening in an interarea that accommodates the
pedicle, and may be filled with plates.

\emph{Longtancunella chengjiangensis}: No evidence or report of an
opening at the hinge line in fossil material in Zhang et al 2007, 2011

\hypertarget{sclerites-dorsal-valve-posterior-surface-notothyrium-shape}{%
\section*{Sclerites: Dorsal valve: Posterior surface: Notothyrium:
Shape}\label{sclerites-dorsal-valve-posterior-surface-notothyrium-shape}}
\addcontentsline{toc}{section}{Sclerites: Dorsal valve: Posterior
surface: Notothyrium: Shape}

\includegraphics{Brachiopod_phylogeny_files/figure-latex/unnamed-chunk-5-24.pdf}

\textbf{Character 24 : Sclerites: Dorsal valve: Posterior surface:
Notothyrium: Shape }

\begin{quote}
1: `Parallel-sided cleft'\\
2: `Triangular'\\
Transformational character.
\end{quote}

A notothyrium is an opening in an interarea that accommodates the
pedicle, and may be filled with plates.

A simplification of character 5 in Bassett et al. 2001.

\hypertarget{sclerites-dorsal-valve-posterior-surface-notothyrium-chilidial-plates}{%
\section*{Sclerites: Dorsal valve: Posterior surface: Notothyrium:
Chilidial
plates}\label{sclerites-dorsal-valve-posterior-surface-notothyrium-chilidial-plates}}
\addcontentsline{toc}{section}{Sclerites: Dorsal valve: Posterior
surface: Notothyrium: Chilidial plates}

\includegraphics{Brachiopod_phylogeny_files/figure-latex/unnamed-chunk-5-25.pdf}

\textbf{Character 25 : Sclerites: Dorsal valve: Posterior surface:
Notothyrium: Chilidial plates }

\begin{quote}
1: `Open'\\
2: `Covered by chilidial plates'\\
Transformational character.
\end{quote}

A notothyrium may be open or covered by a chilidium or two chilidial
plates.\\
No included taxa exhibit more than one chilidial plate.\\
Transformational as it is not self-evident whether the ancestral taxon
had an open or closed notothyrium.

\hypertarget{sclerites-dorsal-valve-notothyrial-platform}{%
\section*{Sclerites: Dorsal valve: Notothyrial
platform}\label{sclerites-dorsal-valve-notothyrial-platform}}
\addcontentsline{toc}{section}{Sclerites: Dorsal valve: Notothyrial
platform}

\includegraphics{Brachiopod_phylogeny_files/figure-latex/unnamed-chunk-5-26.pdf}

\textbf{Character 26 : Sclerites: Dorsal valve: Notothyrial platform }

\begin{quote}
0: `Absent'\\
1: `Present'\\
Neomorphic character.
\end{quote}

After Bassett et al. (2001) character 12.\\
The presence or absence of a notothyrial platform, which often serves as
an attachment point for the diductors in a similar fashion to the
cardinal processes, is independent of the presence of a notothyrium.

\emph{Coolinia pecten}: Referred to as the ``posterior platform'' in
Dewing (2001)

\emph{Kutorgina chengjiangensis}: ``Dorsal diductor scars impressed on
floor of notothyrial cavity'': Williams et al. 2000, regarding
Kutorginata.\\
Bassett et al. (2001) score as absent in Table 18.1.

\emph{Nisusia sulcata}: Bassett et al. (2001) score as absent in Table
18.1.\\
``Dorsal diductor scars impressed on floor of notothyrial cavity'':
Williams et al. 2000, regarding Kutorginata.

\emph{Alisina}: Bassett et al. (2001) score as present in Table 18.1.

\emph{Glyptoria}: Bassett et al. (2001) score as present in Table 18.1.

\emph{Ussunia}: ``Visceral platforms absent in both valves'' -- Williams
et al. 2000, p.~192

\hypertarget{sclerites-dorsal-valve-cardinal-processes}{%
\section*{Sclerites: Dorsal valve: Cardinal
processes}\label{sclerites-dorsal-valve-cardinal-processes}}
\addcontentsline{toc}{section}{Sclerites: Dorsal valve: Cardinal
processes}

\includegraphics{Brachiopod_phylogeny_files/figure-latex/unnamed-chunk-5-27.pdf}

\textbf{Character 27 : Sclerites: Dorsal valve: Cardinal processes }

\begin{quote}
0: `Absent'\\
1: `Present'\\
Neomorphic character.
\end{quote}

After Bassett et al. (2001) character 13.\\
Cardinal processes are unlikely to be homologous with the notothyrial
platform, even if their function is similar.

\emph{Longtancunella chengjiangensis}: Not evident, and ought arguably
to be discernable if present given the quality of preservation

\emph{Clupeafumosus socialis}: Not reported by Topper et al. (2013).

\hypertarget{sclerites-dorsal-valve-medial-septum}{%
\section*{Sclerites: Dorsal valve: Medial
septum}\label{sclerites-dorsal-valve-medial-septum}}
\addcontentsline{toc}{section}{Sclerites: Dorsal valve: Medial septum}

\includegraphics{Brachiopod_phylogeny_files/figure-latex/unnamed-chunk-5-28.pdf}

\textbf{Character 28 : Sclerites: Dorsal valve: Medial septum }

\begin{quote}
0: `Absent'\\
1: `Present'\\
Neomorphic character.
\end{quote}

The dorsal valve of many taxa is exhibits a septum or process (or
myophragm) along the medial line. See character 25 in Benedetto (2009).

\emph{Mummpikia nuda}: See pl. 2 panel 6 in Balthasar (2008).

\emph{Kutorgina chengjiangensis}: Absent - fig. 129.1f in Williams et
al. (2000)

\emph{Heliomedusa orienta}: Reported on `'ventral'`valve by Chen et al.
(2007); we consider their'`ventral'' valve to be the dorsal valve.

The structure is clearly figured, contra its coding as absent in
Williams et al. 2000 and its lack of mention in Williams et al. 2007 or
Zhang et al. 2009.

\emph{Lingulellotreta malongensis}: Very weakly developed but seemingly
present between muscle scars in \emph{Lingulellotreta}, more prominent
in Aboriginella (also Lingulellotretidae) (Williams et al. 2000, fig.
34).

\emph{Nisusia sulcata}: Fig. 125 in Williams et al. (2000)

\emph{Novocrania}: Median process evident: Williams et al. (2000) fig.
100.2a, d

\emph{Antigonambonites planus}: Weakly developed septum evident in
internal cast: Williams et al. 2000, fig. 508.2e

\emph{Orthis}: Short medial process (``low median ridge'', p.~724)
present in dorsal valve; see Fig. 523.3b in Williams et al. (2000)

\emph{Glyptoria}: Neither evident nor reported in Williams et al.
(2000).

\emph{Clupeafumosus socialis}: Prominent process evident (Topper et al.,
2013)

\emph{Ussunia}: Following char 42 in table 15 in Williams et al. 2000

\emph{Eoobolus}: A ``median projection'' is present (fig. 4g in
Balthasar 2009)

\hypertarget{sclerites-ventral-valve-relative-size}{%
\section*{Sclerites: Ventral valve: Relative
size}\label{sclerites-ventral-valve-relative-size}}
\addcontentsline{toc}{section}{Sclerites: Ventral valve: Relative size}

\includegraphics{Brachiopod_phylogeny_files/figure-latex/unnamed-chunk-5-29.pdf}

\textbf{Character 29 : Sclerites: Ventral valve: Relative size }

\begin{quote}
1: `Ventral valve markedly larger than dorsal valve (ventribiconvex)'\\
2: `Equivalve (subequally biconvex)'\\
3: `Dorsal valve markedly larger than ventral valve (dorsibiconvex)'\\
Transformational character.
\end{quote}

In many brachiopods, the valves are closely similar in size; in others,
the ventral valve is markedly larger than the dorsal, on account of
being more convex. Marginal cases are treated as ambiguous for the
relevant states.

\emph{Mummpikia nuda}: Aside from hinge, valves similar in convexity and
size (Balthasar 2008)

\emph{Kutorgina chengjiangensis}: Ventral valve larger (see Williams et
al. 2000, fig. 125.)

\emph{Heliomedusa orienta}: Ventral valve larger than the dorsal valve
(Zhang et al. 2009, p.~659)

\emph{Longtancunella chengjiangensis}: The ventral valve is somewhat,
but not markedly, larger than the dorsal; as such, this character is
coded ambiguous for equivalve/ventral valve larger

\emph{Nisusia sulcata}: Ventral valve larger (see Williams et al. 2000,
fig. 126.)

\emph{Antigonambonites planus}: Broadly equivalve - see Williams et al.
(2000) fig. 508.2c

\emph{Gasconsia}: Convexiplane (Williams et al. 2000, p.~187)

\emph{Yuganotheca elegans}: The ventral valve is somewhat, but not
markedly, larger than the dorsal; as such, this character is coded
ambiguous for equivalve/ventral valve larger

\emph{Craniops}: ``Shell subequally biconvex'' -- WIlliams et al. 2000

\emph{Ussunia}: Subequally biconvex (Williams et al. 2000, p.~192)

\emph{Eoobolus}: ``\emph{Eoobolus} is biconvex'', but in the amended
diagnosis,Balthasar descriped it as ``shell inequivalved,
dorsibiconvex'' .

\hypertarget{sclerites-ventral-valve-growth-direction}{%
\section*{Sclerites: Ventral valve: Growth
direction}\label{sclerites-ventral-valve-growth-direction}}
\addcontentsline{toc}{section}{Sclerites: Ventral valve: Growth
direction}

\includegraphics{Brachiopod_phylogeny_files/figure-latex/unnamed-chunk-5-30.pdf}

\textbf{Character 30 : Sclerites: Ventral valve: Growth direction }

\begin{quote}
1: `Holoperipheral'\\
2: `Mixoperipheral'\\
3: `Hemiperipheral'\\
Transformational character.
\end{quote}

See Fig. 284 in Williams et al. (1997) for depiction of terms.

The growth direction dictates the attitude of the cardinal area relative
to the hinge, which does not therefore represent an independent
character.\\
Crudely put, if, viewed from a dorsal position, the umbo falls within
the outer margin of the shell, growth is holoperipheral; if it falls
outside the margin, it is mixoperipheral; if it falls exactly on the
margin, it is hemiperipheral.

\emph{Paterimitra}: The apical flange notwithstanding, the umbo of the
S1 sclerite is posterior of the hinge line and the posterior edge of the
lateral plate -- see Larsson et al. 2014, fig. 2a, c.

\emph{Heliomedusa orienta}: Williams et al. (2000, 2007) reconstruct
mixoperipheral growth in the ventral valve (though Chen et al. (2007)
reconstruct the valves the other way round, i.e.~it is the ventral valve
that grows holoperipherally, and the dorsal mixoperipherally).

\emph{Clupeafumosus socialis}: Inferred from Topper et al. (2013).

\emph{Ussunia}: Following description of order in Williams et al. 2000

\hypertarget{sclerites-ventral-valve-posterior-surface-differentiated}{%
\section*{Sclerites: Ventral valve: Posterior surface:
Differentiated}\label{sclerites-ventral-valve-posterior-surface-differentiated}}
\addcontentsline{toc}{section}{Sclerites: Ventral valve: Posterior
surface: Differentiated}

\includegraphics{Brachiopod_phylogeny_files/figure-latex/unnamed-chunk-5-31.pdf}

\textbf{Character 31 : Sclerites: Ventral valve: Posterior surface:
Differentiated }

\begin{quote}
0: `Posterior surface of shell not differentiated'\\
1: `Posterior surface of shell forms distinct cardinal area or
pseudointerarea'\\
Neomorphic character.
\end{quote}

In shells that grow by mixoperipheral growth, the triangular area
subtended between each apex and the posterior ends of the lateral
margins is termed the cardinal area. In shells with holoperipheral
growth, a flattened surface on the posterior margin of the valve is
termed a pseudointerarea (paraphrasing Williams et al. 1997).

In order for this character to be independent of a shell's growth
direction, we do not distinguish between a ``cardinal area'',
``interarea'' or ``pseudointerarea''.

\emph{Paterimitra}: Triangular notch and subapical flange

\emph{Lingulosacculus}: The conical valve is interpreted as the ventral
valve with an extended pseudointerarea.

\emph{Tomteluva perturbata}: Interarea present.

\emph{Mummpikia nuda}: Balthasar (2008) interprets a pseudointerarea as
being present - e.g.~p273, ``Of particular interest is the vault that
bridges the most anterior portion of the ventral pseudointerarea and
raises it above the visceral platform.''; ``This pattern is reversed in
the ventral valves of M. \emph{nuda}, where the anterior projection of
the pedicle groove is raised above the valve floor whereas the lateral
parts of pseudointerarea are not''.

\emph{Coolinia pecten}: Interarea present.

\emph{Kutorgina chengjiangensis}: Interarea present.

\emph{Salanygolina}: Interarea present.

\emph{Heliomedusa orienta}: Zhang et al. (2009) report a moderate to
somewhat developed ventral pseudointerarea, confirmed by Williams et al
(2007).

\emph{Longtancunella chengjiangensis}: Though ``all evidence of a
pseudointerarea is lacking'' -- Zhang et al. 2011 -- the region of the
shell between the strophic hinge line and the colleplax (fig. 2 in Zhang
et al. 2011) is distinct from the rest of the shell; the ends of the
strophic hinge line are marked by prominent nicks in the shell margin.
\emph{Longtancunella} is therefore coded as having a differentiated
posterior surface.

\emph{Nisusia sulcata}: Interarea present.

\emph{Terebratulina}: Interarea.

\emph{Antigonambonites planus}: Interarea present.

\emph{Alisina}: Interarea present.

\emph{Orthis}: Interarea present.

\emph{Gasconsia}: The region corresponding to the ventral
(pseudo)interarea is described as a ``trimerellid ventral cardinal
area'' by Williams et al. (2000, p.162), who code both an interarea and
a pseudointerarea as absent in trimerellids.

\emph{Glyptoria}: Interarea present.

\emph{Clupeafumosus socialis}: Described by Topper et al. (2013).

\emph{Ussunia}: Following char 17 in table 15 in Williams et al. 2000

\emph{Mickwitzia muralensis}: What Balthasar (2004) terms an interarea
is, in the terminology employed herein, a pseudointerarea.

\hypertarget{sclerites-ventral-valve-posterior-margin-growth-direction}{%
\section*{Sclerites: Ventral valve: Posterior margin growth
direction}\label{sclerites-ventral-valve-posterior-margin-growth-direction}}
\addcontentsline{toc}{section}{Sclerites: Ventral valve: Posterior
margin growth direction}

\includegraphics{Brachiopod_phylogeny_files/figure-latex/unnamed-chunk-5-32.pdf}

\textbf{Character 32 : Sclerites: Ventral valve: Posterior margin growth
direction }

\begin{quote}
1: `Inward-growing'\\
2: `Outward-growing'\\
Transformational character.
\end{quote}

Balthasar (2008) notes an inward-growing posterior margin of the
pseudointerarea as potentially linking \emph{Mummpikia} with the
linguliform brachiopods.

The posterior margin can only grow inwards if it is differentiated from
the anterior margin; else the entire shell would grow in on itself.

\emph{Mummpikia nuda}: Balthasar (2008) interprets an inward-growing
posterior margin of the pseudointerarea - e.g.~p273, ``Of particular
interest is the vault that bridges the most anterior portion of the
ventral pseudointerarea and raises it above the visceral platform
{[}\ldots{}{]} An inward-growing posterior margin is otherwise known
only from the pseudointerareas of linguliform brachiopods''.

\emph{Lingulellotreta malongensis}: Transverse cross section of ventral
pseudointerarea concave.

\emph{Clupeafumosus socialis}: See Topper et al. (2013)

\hypertarget{sclerites-ventral-valve-posterior-surface-planar}{%
\section*{Sclerites: Ventral valve: Posterior surface:
Planar}\label{sclerites-ventral-valve-posterior-surface-planar}}
\addcontentsline{toc}{section}{Sclerites: Ventral valve: Posterior
surface: Planar}

\includegraphics{Brachiopod_phylogeny_files/figure-latex/unnamed-chunk-5-33.pdf}

\textbf{Character 33 : Sclerites: Ventral valve: Posterior surface:
Planar }

\begin{quote}
0: `Curved lateral profile'\\
1: `Planar lateral profile'\\
Neomorphic character.
\end{quote}

It is possible for a cardinal area or pseudointerarea to be distinct
from the anterior part of the shell, yet to remain curved in lateral
profile.

Taking an undifferentiated posterior margin as primitive, the primitive
condition is curved -- flattening of the posterior margin represents an
additional modification that can only occur once the posterior margin is
differentiated.

A flat and triangular interarea links \emph{Mummpikia} with the
Obolellidae (Balthasar 2008) -- but all included taxa have triangular
interareas, so this is not listed as a separate character.

\emph{Acanthotretella}: ventral pseudointerareas are most similar to
those found within the Order Siphonotretida

\emph{Longtancunella chengjiangensis}: Flattened, reflecting the
strophic hinge line

\emph{Lingulellotreta malongensis}: Transverse cross section of ventral
pseudointerarea concave.

\emph{Micromitra}: Essentially planar; see fig. 6 in Ushatinskaya 2016

\emph{Clupeafumosus socialis}: ``Ventral pseudointerarea is gently
procline and is flat in lateral profile''.

\emph{Eoobolus}: Some curvature retained

\hypertarget{sclerites-ventral-valve-posterior-surface-extent}{%
\section*{Sclerites: Ventral valve: Posterior surface:
Extent}\label{sclerites-ventral-valve-posterior-surface-extent}}
\addcontentsline{toc}{section}{Sclerites: Ventral valve: Posterior
surface: Extent}

\includegraphics{Brachiopod_phylogeny_files/figure-latex/unnamed-chunk-5-34.pdf}

\textbf{Character 34 : Sclerites: Ventral valve: Posterior surface:
Extent }

\begin{quote}
1: `Low'\\
2: `High'\\
Transformational character.
\end{quote}

Distinguishes taxa whose ventral valve is essentially flat from those
that are essentially conical

\emph{Coolinia pecten}: See fig. 485 in Williams et al. 2000

\emph{Salanygolina}: ``Ventral pseudointerarea well defined, high,
nearly flat, apsacline'' -- Williams et al. 2000, p.~156

\emph{Nisusia sulcata}: Scored High in data matrix of Benedetto (2009).

\emph{Antigonambonites planus}: Scored High in data matrix of Benedetto
(2009).

\emph{Orthis}: Scored `'Low'' for Eoorthis by Benedetto (2009); assumed
same in \emph{Orthis}.

\emph{Gasconsia}: ``ventral cardinal interarea low, apsacline, with
narrow, poorly defined homeodeltidium'' - Williams et al. 2000, p.~186

\emph{Clupeafumosus socialis}: Entire valve length -- see schematic in
Williams et al. (21997), fig. 286

\hypertarget{sclerites-ventral-valve-posterior-surface-delthyrium}{%
\section*{Sclerites: Ventral valve: Posterior surface:
Delthyrium}\label{sclerites-ventral-valve-posterior-surface-delthyrium}}
\addcontentsline{toc}{section}{Sclerites: Ventral valve: Posterior
surface: Delthyrium}

\includegraphics{Brachiopod_phylogeny_files/figure-latex/unnamed-chunk-5-35.pdf}

\textbf{Character 35 : Sclerites: Ventral valve: Posterior surface:
Delthyrium }

\begin{quote}
0: `Absent'\\
1: `Present'\\
Neomorphic character.
\end{quote}

A delthyrium is an opening in an interarea that accommodates the
pedicle, and may be filled with plates.

\emph{Micrina}: Opening inferred by Holmer et al. (2008)

\emph{Longtancunella chengjiangensis}: Unclear: a narrow ridge that may
correspond to a pseudodeltidium evident in fig 2a and sketched in fig.
2c is not discussed in the text of Zhang et al. 2011, so the delthyrial
region is coded as ambiguous.

\emph{Askepasma toddense}: Small notch present at base of parallel-sided
homeodeltidium -- See Williams et al. 2000

\emph{Glyptoria}: ``Delthyrium and notothyrium open, wide'' -- Cooper
1976

\emph{Yuganotheca elegans}: Details of the hinge region are unclear due
to the flattened and overprinted nature of fossil preservation.

\emph{Mickwitzia muralensis}: A delthyrium is present in young
individuals (Balthasar 2004).

\hypertarget{sclerites-ventral-valve-posterior-surface-delthyrium-shape}{%
\section*{Sclerites: Ventral valve: Posterior surface: Delthyrium:
Shape}\label{sclerites-ventral-valve-posterior-surface-delthyrium-shape}}
\addcontentsline{toc}{section}{Sclerites: Ventral valve: Posterior
surface: Delthyrium: Shape}

\includegraphics{Brachiopod_phylogeny_files/figure-latex/unnamed-chunk-5-36.pdf}

\textbf{Character 36 : Sclerites: Ventral valve: Posterior surface:
Delthyrium: Shape }

\begin{quote}
1: `Parallel sided'\\
2: `Triangular'\\
Transformational character.
\end{quote}

A parallel-sided delthyrium links \emph{Mummpikia} with the Obolellidae
(Balthasar 2008).

This character has the capacity for further resolution, but this is
unlikely to affect the results of the present study.

\hypertarget{sclerites-ventral-valve-posterior-surface-delthyrium-cover}{%
\section*{Sclerites: Ventral valve: Posterior surface: Delthyrium:
Cover}\label{sclerites-ventral-valve-posterior-surface-delthyrium-cover}}
\addcontentsline{toc}{section}{Sclerites: Ventral valve: Posterior
surface: Delthyrium: Cover}

\includegraphics{Brachiopod_phylogeny_files/figure-latex/unnamed-chunk-5-37.pdf}

\textbf{Character 37 : Sclerites: Ventral valve: Posterior surface:
Delthyrium: Cover }

\begin{quote}
1: `Open'\\
2: `Covered, at least in part'\\
Transformational character.
\end{quote}

An open delthyrium links \emph{Mummpikia} with the Obolellidae
(Balthasar 2008).

The delthyrial opening can be covered by one or more deltidial plates,
or a pseudodeltitium.

A convex pseudodeltidium completely covers the delthyrium in
\emph{Coolinia}.

\emph{Paterimitra}: Covered by subaical flange, in part.

\emph{Nisusia sulcata}: ``Covered only apically by a small convex
pseudodeltitium'' - Holmer et al. 2018

\emph{Glyptoria}: Coded as open by Williams et al. (1998)

\hypertarget{sclerites-ventral-valve-posterior-surface-delthyrium-cover-extent}{%
\section*{Sclerites: Ventral valve: Posterior surface: Delthyrium:
Cover:
Extent}\label{sclerites-ventral-valve-posterior-surface-delthyrium-cover-extent}}
\addcontentsline{toc}{section}{Sclerites: Ventral valve: Posterior
surface: Delthyrium: Cover: Extent}

\includegraphics{Brachiopod_phylogeny_files/figure-latex/unnamed-chunk-5-38.pdf}

\textbf{Character 38 : Sclerites: Ventral valve: Posterior surface:
Delthyrium: Cover: Extent }

\begin{quote}
1: `Covered only partially; partially open'\\
2: `Completely covered'\\
Transformational character.
\end{quote}

\emph{Micrina}: Remains somewhat open

\emph{Nisusia sulcata}: A well-defined pseudo-deltidium {[}\ldots{}{]}
closes only the apical part of\\
the delthyrium (Rowell \& Caruso 1985)

\hypertarget{sclerites-ventral-valve-posterior-surface-delthyrium-cover-identity}{%
\section*{Sclerites: Ventral valve: Posterior surface: Delthyrium:
Cover:
Identity}\label{sclerites-ventral-valve-posterior-surface-delthyrium-cover-identity}}
\addcontentsline{toc}{section}{Sclerites: Ventral valve: Posterior
surface: Delthyrium: Cover: Identity}

\includegraphics{Brachiopod_phylogeny_files/figure-latex/unnamed-chunk-5-39.pdf}

\textbf{Character 39 : Sclerites: Ventral valve: Posterior surface:
Delthyrium: Cover: Identity }

\begin{quote}
1: `Pseudodeltidium'\\
2: `Deltidial plate(s)'\\
Transformational character.
\end{quote}

Transformational.\\
This character has the capacity for further resolution (one or more
deltidial plates), but this is unlikely to affect the results of the
present study.

The pseudodelthyrium is also referred to as a homeodeltidium.

\emph{Micrina}: ``Ventral valve convex with apsacline interarea bearing
delthyrium, covered by a convex pseudodeltidium'' -- Homer et al. 2008

\emph{Lingulellotreta malongensis}: The subapical flange of the
\emph{Paterimitra} S1 sclerite has been homologised with the ventral
homeodeltidium of \emph{Micromitra} (Larsson et al 2014).

\emph{Alisina}: Stated as ``concave pseudodeltidium with median
plication'' -- Williams et al. 2000\\
Coded as ``Pseudodeltidium: Covered by concave plate'' by Bassett et al.
(2001)

\emph{Mickwitzia muralensis}: Termed a homoedeltidium by Balthasar
(2004)

\hypertarget{sclerites-ventral-valve-posterior-surface-delthyrium-pseudodeltidium-shape}{%
\section*{Sclerites: Ventral valve: Posterior surface: Delthyrium:
Pseudodeltidium:
Shape}\label{sclerites-ventral-valve-posterior-surface-delthyrium-pseudodeltidium-shape}}
\addcontentsline{toc}{section}{Sclerites: Ventral valve: Posterior
surface: Delthyrium: Pseudodeltidium: Shape}

\includegraphics{Brachiopod_phylogeny_files/figure-latex/unnamed-chunk-5-40.pdf}

\textbf{Character 40 : Sclerites: Ventral valve: Posterior surface:
Delthyrium: Pseudodeltidium: Shape }

\begin{quote}
1: `Convex'\\
2: `Concave'\\
Transformational character.
\end{quote}

A ridge-like (i.e.~convex) pseudodeltitium unites \emph{Salanygolina}
with \emph{Coolinia} and other Chileata (Holmer et al. 2009, p.~6).

\emph{Salanygolina}: ``The presence of {[}\ldots{}{]} a narrow
delthyrium covered by a convex pseudodeltidium, places Salanygolinidae
outside the Class Paterinata and strongly suggests affinity to the
Cambrian Chileida'' -- Holmer et al. 1009, p.~9

\emph{Alisina}: ``concave pseudodeltidium with median plication'' --
Williams et al. 2000\\
Coded as ``Pseudodeltidium: Covered by concave plate'' by Bassett et al.
(2001)

\emph{Gasconsia}: \emph{Gasconsia} possesses narrow concave
homeodeltidium, but absent pseudodeltidium

\hypertarget{sclerites-ventral-valve-posterior-surface-delthyrium-pseudodeltidium-hinge-furrows}{%
\section*{Sclerites: Ventral valve: Posterior surface: Delthyrium:
Pseudodeltidium: Hinge
furrows}\label{sclerites-ventral-valve-posterior-surface-delthyrium-pseudodeltidium-hinge-furrows}}
\addcontentsline{toc}{section}{Sclerites: Ventral valve: Posterior
surface: Delthyrium: Pseudodeltidium: Hinge furrows}

\includegraphics{Brachiopod_phylogeny_files/figure-latex/unnamed-chunk-5-41.pdf}

\textbf{Character 41 : Sclerites: Ventral valve: Posterior surface:
Delthyrium: Pseudodeltidium: Hinge furrows }

\begin{quote}
0: `Absent'\\
1: `Present'\\
Neomorphic character.
\end{quote}

After Bassett et al. (2001) character 18, ``Hinge furrows on lateral
sides of pseudodeltidium''.

\emph{Pedunculotheca diania}: Absent due to inapplicability of
neomorphic character.

\emph{Micrina}: Absent due to inapplicability of neomorphic character.

\emph{Paterimitra}: Absent due to inapplicability of neomorphic
character.

\emph{Eccentrotheca}: Absent due to inapplicability of neomorphic
character.

\emph{Haplophrentis carinatus}: Absent due to inapplicability of
neomorphic character.

\emph{Lingulosacculus}: Absent due to inapplicability of neomorphic
character.

\emph{Phoronis}: Absent due to inapplicability of neomorphic character.

\emph{Mummpikia nuda}: Absent due to inapplicability of neomorphic
character.

\emph{Salanygolina}: The presence of this feature is impossible to
determine based on the available material.

\emph{Dailyatia}: Absent due to inapplicability of neomorphic character.

\emph{Lingula}: Absent due to inapplicability of neomorphic character.

\emph{Acanthotretella}: Absent due to inapplicability of neomorphic
character.

\emph{Heliomedusa orienta}: Absent due to inapplicability of neomorphic
character.

\emph{Lingulellotreta malongensis}: Absent due to inapplicability of
neomorphic character.

\emph{Askepasma toddense}: Absent due to inapplicability of neomorphic
character.

\emph{Micromitra}: Absent due to inapplicability of neomorphic
character.

\emph{Pelagodiscus atlanticus}: Absent due to inapplicability of
neomorphic character.

\emph{Novocrania}: Absent due to inapplicability of neomorphic
character.

\emph{Terebratulina}: Absent due to inapplicability of neomorphic
character.

\emph{Orthis}: Absent due to inapplicability of neomorphic character.

\emph{Glyptoria}: Absent due to inapplicability of neomorphic character.

\emph{Clupeafumosus socialis}: Absent due to inapplicability of
neomorphic character.

\hypertarget{sclerites-ventral-valve-umbonal-perforation}{%
\section*{Sclerites: Ventral valve: Umbonal
perforation}\label{sclerites-ventral-valve-umbonal-perforation}}
\addcontentsline{toc}{section}{Sclerites: Ventral valve: Umbonal
perforation}

\includegraphics{Brachiopod_phylogeny_files/figure-latex/unnamed-chunk-5-42.pdf}

\textbf{Character 42 : Sclerites: Ventral valve: Umbonal perforation }

\begin{quote}
0: `Umbo imperforate (or neomorphic character inapplicable)'\\
1: `Umbonal perforation'\\
Neomorphic character.
\end{quote}

Certain taxa, particularly those with a colleplax, exhibit a perforation
at the umbo of the ventral valve. This opening is sometimes associated
with a pedicle sheath.

\emph{Eccentrotheca}: Inapplicable as scleritome is not bivalved.

The sclerites of \emph{Eccentrotheca} form a ring that surrounds the
inferred attachment structure; the attachment structure does not emerge
from an aperture within an individual sclerite. Thus no feature in
\emph{Eccentrotheca} is judged to be potentially homologous with the
apical perforation in bivalved brachiopods.

\emph{Lingulosacculus}: The apical termination of the fossil is unknown.

\emph{Dailyatia}: The B and C sclerites of \emph{Dailyatia} bear small
umbonal perforations (Skovsted et al 2015), but these are not considered
to be homologous with the ventral valve, so this character is coded as
inapplicable - though the possibility that the perforations are
equivalent is intriguing.

A1 sclerites typically have a pair of perforations, which are
conceivably equivalent to the setal tubes of \emph{Micrina} (Holmer et
al. 2011). The A1 sclerite of D. bacata has a structure that is arguably
similar to the `'colleplax'' of \emph{Paterimitra}. But the homology of
any of these structures to the umbonal aperture of brachiopods is
difficult to establish.

\emph{Heliomedusa orienta}:\\
``there is compelling evidence to demonstrate that the putative
pedicle\\
illustrated by Chen et al. (2007: Figs. 4, 6, 7) in fact is the mold of
a three-dimensionally preserved visceral cavity.''

\emph{Clupeafumosus socialis}: An umbonal perforation is distinctive of
Acrotretids, and is reported by Topper et al. (2013).

\emph{Mickwitzia muralensis}: The umbo itself is imperforate, though
note that an opening is incorporated at the base of the homeodeltidium
when the organism switches from early to late maturity (fig. 10 in
Balthasar 2004).

\hypertarget{sclerites-ventral-valve-umbonal-perforation-shape}{%
\section*{Sclerites: Ventral valve: Umbonal perforation:
Shape}\label{sclerites-ventral-valve-umbonal-perforation-shape}}
\addcontentsline{toc}{section}{Sclerites: Ventral valve: Umbonal
perforation: Shape}

\includegraphics{Brachiopod_phylogeny_files/figure-latex/unnamed-chunk-5-43.pdf}

\textbf{Character 43 : Sclerites: Ventral valve: Umbonal perforation:
Shape }

\begin{quote}
1: `Circular (or subcircular)'\\
2: `Rhombic to oval'\\
Transformational character.
\end{quote}

Chen et al. (2007) propose that an oval to rhombic foramen characterises
the discinids (and \emph{Heliomedusa}).

\emph{Coolinia pecten}: Bassett and Popov write ``a dominant feature of
the ventral umbo is a sub-oval perforation about 270 um long and 250 um
wide'': the width and height of this structure are almost identical, and
we score it as (sub) circular.

\emph{Kutorgina chengjiangensis}: The exact size and shape of the apical
perforation is obscured by the emerging pedicle

\emph{Salanygolina}: Essentially circular (Holmer et al. 2009, fig. 4)

\emph{Acanthotretella}: Too small to observe given quality of
preservation (Homer \& Caron 2006)

\emph{Heliomedusa orienta}: Rhombic to oval -- seen as evidence for a
discinid affinity (Chen et al. 2007)

\emph{Lingulellotreta malongensis}: Oval (Williams et al. 2000)

\emph{Nisusia sulcata}: " close to circular " (Holmer et al. 2018)

\emph{Antigonambonites planus}: Based on p.92, fig.4B.

\emph{Alisina}: Seemingly circular (Zhang et al. 2011).

\emph{Clupeafumosus socialis}: Taller than wide in some cases, but very
nearly circular in others; see Topper et alia (2013)

\hypertarget{sclerites-ventral-valve-colleplax-cicatrix-or-pedicle-sheath}{%
\section*{Sclerites: Ventral valve: Colleplax, cicatrix or pedicle
sheath}\label{sclerites-ventral-valve-colleplax-cicatrix-or-pedicle-sheath}}
\addcontentsline{toc}{section}{Sclerites: Ventral valve: Colleplax,
cicatrix or pedicle sheath}

\includegraphics{Brachiopod_phylogeny_files/figure-latex/unnamed-chunk-5-44.pdf}

\textbf{Character 44 : Sclerites: Ventral valve: Colleplax, cicatrix or
pedicle sheath }

\begin{quote}
0: `Absent'\\
1: `Present'\\
Neomorphic character.
\end{quote}

In certain taxa, the umbo of the ventral valve bears a colleplax,
cicatrix or pedicle sheath; Bassett et al. (2008) consider these
structures as homologous.

\emph{Pedunculotheca diania}: The flat apical termination of juvenile
individuals possibly functioned as colleplax for attachment, we treated
as potentially homologous.

\emph{Micrina}: Absent in \emph{Micrina}, though present in co-occurring
\emph{Paterimitra}

\emph{Heliomedusa orienta}: A cicatrix was reconstructed by Jin \& Wang
1992 (figs 6b, 7), but has not been reported by later authors; possibly,
as with the `'pedicle foramen'' of Chen et al. (2007), this structure
represents internal organs rather than a cicatrix proper (Zhang et al.
2009); as such it has been recorded as ambiguous.

\emph{Longtancunella chengjiangensis}: A ring-like structure surrounding
the pedicle sheath is interpreted as a colleplax (Zhang et al. 2011),
though the authors make no comparison with the pedicle capsule exhibited
by extant terebratulids (see Holmer et al. 2018).

\emph{Lingulellotreta malongensis}: The pedicle is identified as such
(rather than a pedicle sheath) by the internal pedicle tube.

\emph{Clupeafumosus socialis}: Not reported by Topper et al. (2013).

\emph{Yuganotheca elegans}: The median collar or conical tube is
conceivably homologous with the pedicle sheath.

\emph{Craniops}: Paracraniops is ``externally similar to
\emph{Craniops}, but lacking cicatrix'' -- indicating that
\emph{Craniops} bears a cicatrix. (Williams et al. 2000) Aso coded
present in their table 15.

\emph{Ussunia}: Following table 15 in Williams et al. 2000

\hypertarget{sclerites-ventral-valve-median-septum}{%
\section*{Sclerites: Ventral valve: Median
septum}\label{sclerites-ventral-valve-median-septum}}
\addcontentsline{toc}{section}{Sclerites: Ventral valve: Median septum}

\includegraphics{Brachiopod_phylogeny_files/figure-latex/unnamed-chunk-5-45.pdf}

\textbf{Character 45 : Sclerites: Ventral valve: Median septum }

\begin{quote}
0: `Absent'\\
1: `Present'\\
Neomorphic character.
\end{quote}

Chen et al. (2007) observe a median septum in the ventral valve of
\emph{Heliomedusa} and Discinisca, which they propose points to a close
relationship.

\emph{Haplophrentis carinatus}: The carina of H. \emph{carinatus} is an
angular elevation of the ventral valve surface, rather than a septum
growing inward on the interior of shell.

\emph{Mummpikia nuda}: ``Some specimens also reveal that the vault had a
slight median septum, which is now visible as a notch or a groove
dividing the right from the left part'' -- Balthasar 2008

\emph{Heliomedusa orienta}: Reported on `'ventral'`valve by Chen et al.
(2007); we consider the'`ventral'' valve to be the dorsal valve.

\emph{Micromitra}: Ventral ridge characteristic of \emph{Micromitra}
(Skovsted \& Peel 2010)

\emph{Pelagodiscus atlanticus}: Described as present in Discinisca by
Chen et al. 2007; assumed present also in \emph{Pelagodiscus}.

\emph{Novocrania}: Valve thin and often unmineralized.

\emph{Gasconsia}: Evident in moulds of ventral valve; see Watkins
(2002).

\emph{Glyptoria}: Neither evident nor reported in Williams et al.
(2000).

\emph{Clupeafumosus socialis}: A short medial ridge (septum) is present
in the ventral valve (Topper et al. 2013).

\emph{Ussunia}: Following char. 42 in table 15 in Williams et al. 2000

\emph{Eoobolus}: Prominent median septum (fig. 4d, e in Balthasar 2009)

\hypertarget{sclerites-concentric-ornament}{%
\section*{Sclerites: Concentric
ornament}\label{sclerites-concentric-ornament}}
\addcontentsline{toc}{section}{Sclerites: Concentric ornament}

\includegraphics{Brachiopod_phylogeny_files/figure-latex/unnamed-chunk-5-46.pdf}

\textbf{Character 46 : Sclerites: Concentric ornament }

\begin{quote}
0: `Smooth, or growth lines only'\\
1: `Concentric ornament present'\\
Neomorphic character.
\end{quote}

After character 11 in Williams et al. (1998).

\emph{Eccentrotheca}: More or less concentric ridges occur on
\emph{Eccentrotheca} sclerites (Skovsted et al. 2011)

\emph{Kutorgina chengjiangensis}: Following Appendix 2 in Williams et
al. (1998).

\emph{Salanygolina}: Following Appendix 2 in Williams et al. (1998).

\emph{Heliomedusa orienta}: The ornament on shell exterior is described
as concentric fila (Chen et al., 2007, P.43), and also scored as it in
Willaims et al. (2000, pp.160-163).

\emph{Askepasma toddense}: Following Appendix 2 in Williams et al.
(1998).

\emph{Micromitra}: Following Appendix 2 in Williams et al. (1998).

\emph{Pelagodiscus atlanticus}: Only growth lines evident (Williams et
al. 2000)

\emph{Novocrania}: Irregular ridges externally (Williams et al. 2000)

\emph{Terebratulina}: Single ridge evident in Williams et al. (2006)
fig. 1425.1a interpreted as interruption ot growth rather than inherent
feature, so coded as absent (i.e.~smooth).

\emph{Glyptoria}: Following Appendix 2 in Williams et al. (1998).

\emph{Mickwitzia muralensis}: Symmetric fila

\hypertarget{sclerites-concentric-ornament-symmetry}{%
\section*{Sclerites: Concentric ornament:
symmetry}\label{sclerites-concentric-ornament-symmetry}}
\addcontentsline{toc}{section}{Sclerites: Concentric ornament: symmetry}

\includegraphics{Brachiopod_phylogeny_files/figure-latex/unnamed-chunk-5-47.pdf}

\textbf{Character 47 : Sclerites: Concentric ornament: symmetry }

\begin{quote}
0: `Symmetric fila'\\
1: `Asymmetric fila, with outer faces'\\
Neomorphic character.
\end{quote}

After character 11 in Williams et al. (1998).

\emph{Micrina}: No obvious asymmetry, even if not obviously symmetric
either (Holmer et al. 2008). Coded as ambiguous.

\emph{Eccentrotheca}: Ornament, such as it is, is asymmetric, with
prominent outer faces (Skovsted et alia 2011).

\emph{Kutorgina chengjiangensis}: Following Appendix 2 in Williams et
al. (1998).

\emph{Salanygolina}: Following Appendix 2 in Williams et al. (1998).

\emph{Dailyatia}: Clear asymmetry (Skovsted et al. 2015).

\emph{Heliomedusa orienta}: See fig. 1715 in Williams et al. (2007)

\emph{Askepasma toddense}: Following Appendix 2 in Williams et al.
(1998).

\emph{Micromitra}: Following Appendix 2 in Williams et al. (1998).

\emph{Novocrania}: Clear outer faces (Williams et al. 2000, fig. 100.2b)

\emph{Alisina}: Seemingly asymmetric (Williams et al. 2000, fig. 122.3c;
Zhang et al. 2011, Fig. 1)

\emph{Glyptoria}: Following Appendix 2 in Williams et al. (1998).

\emph{Mickwitzia muralensis}: Symmetric fila (Balthasar 2004)

\hypertarget{sclerites-radial-ornament}{%
\section*{Sclerites: Radial ornament}\label{sclerites-radial-ornament}}
\addcontentsline{toc}{section}{Sclerites: Radial ornament}

\includegraphics{Brachiopod_phylogeny_files/figure-latex/unnamed-chunk-5-48.pdf}

\textbf{Character 48 : Sclerites: Radial ornament }

\begin{quote}
0: `Absent'\\
1: `Present'\\
Neomorphic character.
\end{quote}

Ridges radiating from umbo, i.e.~ribs

\emph{Heliomedusa orienta}: See fig. 1715 in Williams et al. (2007)

\emph{Askepasma toddense}: ``Ornament of irregularly developed,
concentric growth lamellae; microornament of irregularly arranged,
polygonal pits'' -- Williams et al. 2000, p153; figs on p.155

\emph{Gasconsia}: ``Ornament of indistinct low radial ribs'' -- Williams
et al. (2000, p167)

\emph{Glyptoria}: ``Coarsely costate'' - Williams et al. (2000, p710)

\emph{Ussunia}: Unornamented.

\emph{Eoobolus}: Very faint costellae in some specimens but coded
absent.

\hypertarget{sclerites-shell-penetrating-spines}{%
\section*{Sclerites: Shell-penetrating
spines}\label{sclerites-shell-penetrating-spines}}
\addcontentsline{toc}{section}{Sclerites: Shell-penetrating spines}

\includegraphics{Brachiopod_phylogeny_files/figure-latex/unnamed-chunk-5-49.pdf}

\textbf{Character 49 : Sclerites: Shell-penetrating spines }

\begin{quote}
0: `Absent'\\
1: `Present'\\
Neomorphic character.
\end{quote}

Mineralized or partly mineralized spines are observed in
\emph{Heliomedusa} and \emph{Acanthotretella}

\emph{Heliomedusa orienta}: The `'spines'' reported by Chen et al.
(2007) are pyritized spinelike\\
setae -- see pp.~2580-2590 in Williams et al. (2007).

\emph{Nisusia sulcata}: Bears numerous small, hollow spines (Williams et
al. 2000)

\emph{Glyptoria}: Neither evident nor reported in Williams et al.
(2000).

\hypertarget{sclerites-mineralogy}{%
\section*{Sclerites: Mineralogy}\label{sclerites-mineralogy}}
\addcontentsline{toc}{section}{Sclerites: Mineralogy}

\includegraphics{Brachiopod_phylogeny_files/figure-latex/unnamed-chunk-5-50.pdf}

\textbf{Character 50 : Sclerites: Mineralogy }

\begin{quote}
1: `Organic (non-mineralized)'\\
2: `Phosphatic'\\
3: `Calcitic'\\
4: `Aragonitic'\\
Transformational character.
\end{quote}

\emph{Mummpikia nuda}: Identified as calcareous by preservational
criteria, and description ``primary\\
calcitic shells of M. \emph{nuda}'' (Balthasar 2008).

\emph{Salanygolina}: Original mineralogy unknown, but known to be
mineralised and anticipated to be phosphatic (Holmer et al. 2009)

\emph{Acanthotretella}: Holmer \& Caron (2006) note the absence of
brittle breakage, interpreted as indicating the absence of a material
mineralized component to the shells.

\emph{Heliomedusa orienta}: ``Shell originally organophosphatic, but may
generally have been poorly mineralized'' -- Williams et al. 2007 --
cf.~ibid, p.~2889, " These strong similarities to discinoids in
soft-part anatomy imply that the \emph{Heliomedusa} shell was chitinous
or chitinophosphatic, not calcareous."

\emph{Longtancunella chengjiangensis}: ``The original composition of the
shell cannot be determined with certainty'', though it was ``most
probably entirely soft and organic'' -- Zhang et al. 2011

\emph{Novocrania}: Ventral valve uncalcified in extant forms or
sometimes thin (Williams et al., 2000), but coded as calcitic as
calcite-mineralizing pathways are present.

\emph{Gasconsia}: Confirmed in Trimerella by Balthasar et al. 2011

\emph{Clupeafumosus socialis}: Phosphatic - hence the conventional
placement within Linguliformea.

\emph{Craniops}: Shell calcitic

\emph{Mickwitzia muralensis}: Calcite and silica deemed diagenetic by
Balthasar (2004).

\emph{Eoobolus}: ``the original shell of \emph{Eoobolus} contained small
calcareous grains that were incorporated into organic-rich layers
alongside apatite'' (Balthasar 2007)

\hypertarget{sclerites-composition-of-cuticle-or-organic-matrix}{%
\section*{Sclerites: Composition of cuticle or organic
matrix}\label{sclerites-composition-of-cuticle-or-organic-matrix}}
\addcontentsline{toc}{section}{Sclerites: Composition of cuticle or
organic matrix}

\includegraphics{Brachiopod_phylogeny_files/figure-latex/unnamed-chunk-5-51.pdf}

\textbf{Character 51 : Sclerites: Composition of cuticle or organic
matrix }

\begin{quote}
1: `GAGs, chitin and collagen'\\
2: `Glycoprotein'\\
Transformational character.
\end{quote}

Williams et al. (1996) identify glycoprotein-based organic scaffolds as
distinct from those comprising glycosaminoglycans (GAGs), chitin and
collagen. This character can only be scored for extant taxa.

\emph{Phoronis}: ``The presence of sulphated glycosaminoglycans (GAGs)
in the chitinous cuticle of \emph{Phoronis} (Herrmann, 1997, p.~215)
would suggest a link with linguliforms, as GAGs are unknown in
rhynchonelliform shells (Fig. 1891, 1896)'' -- Williams et al. 2007,
p.~2830

\emph{Lingula}: Coded as GAGs, chitin and collagen in lingulids by
Williams et al (1996)

\emph{Pelagodiscus atlanticus}: Coded as GAGs, chitin and collagen in
discinids by Williams et al (1996)

\emph{Novocrania}: Coded as glycoprotein for craniids by Williams et al
(1996)

\emph{Terebratulina}: Coded as glycoprotein for terebratulids by
Williams et al (1996)

\hypertarget{sclerites-incorporation-of-sedimentary-particles}{%
\section*{Sclerites: incorporation of sedimentary
particles}\label{sclerites-incorporation-of-sedimentary-particles}}
\addcontentsline{toc}{section}{Sclerites: incorporation of sedimentary
particles}

\includegraphics{Brachiopod_phylogeny_files/figure-latex/unnamed-chunk-5-52.pdf}

\textbf{Character 52 : Sclerites: incorporation of sedimentary particles
}

\begin{quote}
0: `Absent'\\
1: `Present'\\
Neomorphic character.
\end{quote}

Phoronids and \emph{Yuganotheca} aggulutinate particles into their
sclerites.

\hypertarget{sclerites-periostracum-flexibility}{%
\section*{Sclerites: Periostracum:
Flexibility}\label{sclerites-periostracum-flexibility}}
\addcontentsline{toc}{section}{Sclerites: Periostracum: Flexibility}

\includegraphics{Brachiopod_phylogeny_files/figure-latex/unnamed-chunk-5-53.pdf}

\textbf{Character 53 : Sclerites: Periostracum: Flexibility }

\begin{quote}
1: `Flexible'\\
2: `Inflexible'\\
Transformational character.
\end{quote}

Following character 9 in Williams et al. (1998); see their p228-230 for
a discussion of how this might be inferred from fossil material.

\emph{Kutorgina chengjiangensis}: Following Appendix 2 in Williams et
al. (1998).

\emph{Salanygolina}: Following Appendix 2 in Williams et al. (1998).

\emph{Askepasma toddense}: Following Appendix 2 in Williams et al.
(1998).

\emph{Micromitra}: Following Appendix 2 in Williams et al. (1998).

\emph{Pelagodiscus atlanticus}: Flexible (Williams et al. 1998)

\emph{Glyptoria}: Following Appendix 2 in Williams et al. (1998).

\hypertarget{sclerites-microstructure-layers}{%
\section*{Sclerites: Microstructure:
Layers}\label{sclerites-microstructure-layers}}
\addcontentsline{toc}{section}{Sclerites: Microstructure: Layers}

\includegraphics{Brachiopod_phylogeny_files/figure-latex/unnamed-chunk-5-54.pdf}

\textbf{Character 54 : Sclerites: Microstructure: Layers }

\begin{quote}
1: `Single microstructural layer'\\
2: `Two microstructurally differentiated layers'\\
3: `Inner and outer laminae enclosing medial void'\\
4: `Three microstrurally differentiated layers'\\
Transformational character.
\end{quote}

Hyolith conchs comprise two mineralized layers of fibrous bundles.
Bundles are measure 5-15 um across; their constituent fibres are each
0.1-1.0 um wide. In the inner layer, the fibres are transverse; in the
outer layer, the bundles are inclined towards the umbo, becoming
longitudinal on the outermost margin.

Obolellids (like craniids) comprise a single laminated mineralogical
layer (Baltahsar 2008). Shell-penetrating canals are not considered as
contributing to the mineralogical microstructure and are coded
separately.

\emph{Namacalathus} exhibits three layers, none of which have any
obvious correspondence with those of brachiopods.

Coded as non-additive as there is no clear necessity to pass through the
brachiopod-like construction: the three layers could arise by the
addition of a void to a single preexisting layer, for example.

Inapplicable in taxa with a non-mineralized shell.

\emph{Micrina}: Identical to \emph{Mickwitzia} and more derived
linguliforms (Holmer et al 2011)

\emph{Haplophrentis carinatus}: Assumed to be equivalent to the hyoliths
described by Kouchinsky (2000)

\emph{Clupeafumosus socialis}: General acrotretid structure taken from
Zhang et al. (2016)

\emph{Mickwitzia muralensis}: ``the shell structure of \emph{Mickwitzia}
{[}\ldots{}{]} is closely similar to the columnar shell of linguliform
acrotretoid brachiopods as well as to the linguloid
\emph{Lingulellotreta}, in that it has slender columns in the laminar
succession'' -- Williams et al. 2007

\emph{Eoobolus}: ``\emph{Eoobolus} shells exhibit the general
characteristics of modern linguliform shells, i.e.~they were composed of
alternating sets of organic and apatite-rich layers that were separated
by thin sheets of recalcitrant organic layers.'' -- Balthasar 2007

\hypertarget{sclerites-microstructure-crystal-format}{%
\section*{Sclerites: Microstructure: Crystal
format}\label{sclerites-microstructure-crystal-format}}
\addcontentsline{toc}{section}{Sclerites: Microstructure: Crystal
format}

\includegraphics{Brachiopod_phylogeny_files/figure-latex/unnamed-chunk-5-55.pdf}

\textbf{Character 55 : Sclerites: Microstructure: Crystal format }

\begin{quote}
1: `Laminated'\\
2: `Fibrous bundles'\\
3: `Polygonal columns'\\
Transformational character.
\end{quote}

Hyolith conchs comprise two mineralized layers of fibrous bundles.
Bundles are measure 5-15 um across; their constituent fibres are each
0.1-1.0 um wide. In the inner layer, the fibres are transverse; in the
outer layer, the bundles are inclined towards the umbo, becoming
longitudinal on the outermost margin.

Obolellids (like craniids) comprise a single laminated mineralogical
layer (Baltahsar 2008). Shell-penetrating canals are not considered as
contributing to the mineralogical microstructure and are coded
separately.

The pervasive (not just superficial) polygonal structures in
\emph{Paterimitra} are distinct, and characterize \emph{Askepasma},
Sanalygolina, \emph{Eccentrotheca} and \emph{Paterimitra} (Larsson et
al. 2014)

The treatise (Williams et al 2000) identifies cross-bladed laminae as
diagnostic of Strophomenata, with the exception of some older groups
that contain fibres or laminar laths.

\emph{Namacalathus}: The inner and outer layer are foliated. The
columnar inflections lack canals, and as such we do not consider them to
bear any obvious homology with the hollow pillars of tommotiids and
certain brachiopods, their superficial similarity to strophomenid
pseudopunctae notwithstanding.

\emph{Antigonambonites planus}: Shell structure of this taxon is
laminated, rather than fibrous as previously considered.

\emph{Craniops}: ``with calcitic or possibly aragonitic inarticulated
shells with laminar (tabular) secondary layers'' (Williams et al. 2000)

\hypertarget{sclerites-microstructure-punctae}{%
\section*{Sclerites: Microstructure:
Punctae}\label{sclerites-microstructure-punctae}}
\addcontentsline{toc}{section}{Sclerites: Microstructure: Punctae}

\includegraphics{Brachiopod_phylogeny_files/figure-latex/unnamed-chunk-5-56.pdf}

\textbf{Character 56 : Sclerites: Microstructure: Punctae }

\begin{quote}
0: `Absent'\\
1: `Present'\\
Neomorphic character.
\end{quote}

Punctae are 10-20 um wide canals created by multicellular extensions of
the outer epithelium. They penetrate the full depth of the shell.

Balthasar (2008) writes:

``Vertical shell penetrating structures, such as punctae, pseudopunctae,
extropunctae and canals, are common in many groups of brachiopods and
are distinguished based on their geometry and size (Williams 1997).
Punctae are 10-20 um wide and represent multicellular extensions of the
outer epithelium (Owen and Williams 1969). Pseudopunctae and
extropunctae are similar in diameter but, instead of canals, are
vertical stacks of conical deflections of individual shell layers
(Williams and Brunton 1993). None of these three types of vertical shell
structure, all of which are confined to calcitic-shelled brachiopods,
compares with the much smaller canals (

\emph{Mummpikia nuda}: ``Vertical shell penetrating structures, such as
punctae, pseudopunctae, extropunctae and canals, are common in many
groups of brachiopods and are distinguished based on their geometry and
size (Williams 1997). Punctae are 10-20 um wide and represent
multicellular extensions of the outer epithelium (Owen and Williams
1969). {[}\ldots{}{]} None of these three types of vertical shell
structure, all of which are confined to calcitic-shelled brachiopods,
compares with the much smaller canals (

\emph{Heliomedusa orienta}: `'Identical'' to those in \emph{Mickwitzia}
-- see Williams et al. 2007

\emph{Terebratulina}: Endopunctae are relatively large canals, diameter
vary greatly from 5-20µm.

\emph{Craniops}: ``impunctate''

\emph{Mickwitzia muralensis}: Coded as present to reflect that the
chambers contained setae; following Carlson in Williams et al. 2007, the
punctae may or may not be homologous as punctae, but are likely
homologous as shell perforations; both these perforations and those of
\emph{Micrina} were associated with setae, even if their equivalence bay
be with juvenile vs adult setal structures in modern brachiopods
(Balthasar 2004, p, 397).

\hypertarget{sclerites-microstructure-canals}{%
\section*{Sclerites: Microstructure:
Canals}\label{sclerites-microstructure-canals}}
\addcontentsline{toc}{section}{Sclerites: Microstructure: Canals}

\includegraphics{Brachiopod_phylogeny_files/figure-latex/unnamed-chunk-5-57.pdf}

\textbf{Character 57 : Sclerites: Microstructure: Canals }

\begin{quote}
0: `Absent'\\
1: `Present'\\
Neomorphic character.
\end{quote}

A caniculate microstructure occurs in lingulids; canals are narrower (

\emph{Micrina}: Acrotretid laminae bear characteristic columns
(e.g.~Zhang et al. 2016); a similar fabric has been reported, and
assumed homologous, in \emph{Micrina} (Butler et al. 2012).

A similar columnar shell microstructure also occurs in the closely
related \emph{Mickwitzia} (Balthasar 2008).

\emph{Namacalathus}: Canal-like structures have been reported in
\emph{Namacalathus} (Zhuravlev et al. 2015), and interpreted as evidence
for a Lophophorate affinity. Though the structures are not necessarily
directly equivalent, the hypothesis of homology is followed here.

\emph{Haplophrentis carinatus}: The tubules within the centre of the
bundles of hyolith shells (Kouchinsky 2000) are c. 10 um wide, making
them an order of magnitude larger than the canals that characterize
lingulid valves. It is not clear that they are necessarily homologous.

\emph{Longtancunella chengjiangensis}: Preservational resolution not
sufficient to evaluate

\emph{Clupeafumosus socialis}: Acrotretid laminae bear characteristic
columns (e.g.~Zhang et al. 2016).

Balthasar (2008) considers these columns as homologous with tubules
within the columnar shell microstructure \emph{Mummpikia},
\emph{Mickwitzia} and lingulellotretids

\emph{Mickwitzia muralensis}: Coded as present to reflect similarity of
columnar microstructure remarked on by, among others, Balthasar (2008);
Williams et al. (2007); Skovsted \& Holmer (2003)

\hypertarget{sclerites-microstructure-pseudopunctae}{%
\section*{Sclerites: Microstructure:
Pseudopunctae}\label{sclerites-microstructure-pseudopunctae}}
\addcontentsline{toc}{section}{Sclerites: Microstructure: Pseudopunctae}

\includegraphics{Brachiopod_phylogeny_files/figure-latex/unnamed-chunk-5-58.pdf}

\textbf{Character 58 : Sclerites: Microstructure: Pseudopunctae }

\begin{quote}
0: `Absent'\\
1: `Present'\\
Neomorphic character.
\end{quote}

Pseudopunctae are not punctae, but deflections of shell laminae. They
characterise Strophomenata in particular.

\emph{Nisusia sulcata}: Scored absent in data matrix of Benedetto
(2009).

\emph{Antigonambonites planus}: Scored absent in data matrix of
Benedetto (2009).

\emph{Orthis}: Scored absent (in Eoorthis) in data matrix of Benedetto
(2009).

\emph{Glyptoria}: Scored absent in data matrix of Benedetto (2009).

\hypertarget{sclerites-microstructure-external-polygonal-ornament}{%
\section*{Sclerites: Microstructure: External polygonal
ornament}\label{sclerites-microstructure-external-polygonal-ornament}}
\addcontentsline{toc}{section}{Sclerites: Microstructure: External
polygonal ornament}

\includegraphics{Brachiopod_phylogeny_files/figure-latex/unnamed-chunk-5-59.pdf}

\textbf{Character 59 : Sclerites: Microstructure: External polygonal
ornament }

\begin{quote}
0: `Absent'\\
1: `Present'\\
Neomorphic character.
\end{quote}

Regular polygonal compartments, around 10 um in diameter, characterise
\emph{Paterimitra} and some other brachiopod groups. Walls between
compartments have the cross-section of an anvil. external polygonal
structure (possible imprints of epithelial tissue)occurs in
\emph{Dailyatia}, but it is surface pattern, which is different from the
polygonal prisms in the body wall of other paterinid like groups.

\emph{Clupeafumosus socialis}: The polygonal ornament reported in
acrotretids by Zhang et al. (2016) is on the internal surface of the
shell.

\hypertarget{egg-size}{%
\section*{Egg size}\label{egg-size}}
\addcontentsline{toc}{section}{Egg size}

\includegraphics{Brachiopod_phylogeny_files/figure-latex/unnamed-chunk-5-60.pdf}

\textbf{Character 60 : Egg size }

\begin{quote}
1: `Small: \textless{} 100 um, little yolk'\\
2: `Large: \textgreater{} 110 um, much yolk'\\
Transformational character.
\end{quote}

Following Carlson (1995), character 7. This character is only possible
to code in extant taxa. It is not considered independent of Carlson's
character 11, number of gametes released per spawning, as it is possible
to produce more small eggs than large eggs -- thus this latter character
is not reproduced in the present study. The same goes for Carlson's
character 12, gamete dispersal mode; brooders will tend to brood large
eggs.

\emph{Phoronis}: \emph{Phoronis} has planktotrophic larvae indicating a
small egg size -- Ruppert et al. 2004.

Carlson (1995) codes phoronids as polymorphic, as some members of the
phylum have eggs of each size.

\emph{Lingula}: Following coding for class in Carlson (1995) Appendix 1,
character 7.

\emph{Pelagodiscus atlanticus}: Following coding for class in Carlson
(1995) Appendix 1, character 7.

\emph{Novocrania}: Following coding for class in Carlson (1995) Appendix
1, character 7.

\emph{Terebratulina}: Following coding for class in Carlson (1995)
Appendix 1, character 7.

\hypertarget{site-of-gamete-maturation}{%
\section*{Site of gamete maturation}\label{site-of-gamete-maturation}}
\addcontentsline{toc}{section}{Site of gamete maturation}

\includegraphics{Brachiopod_phylogeny_files/figure-latex/unnamed-chunk-5-61.pdf}

\textbf{Character 61 : Site of gamete maturation }

\begin{quote}
0: `Body cavity'\\
1: `Mantle canals'\\
Neomorphic character.
\end{quote}

After Carlson (1995), character 9. Only possible to code in extant taxa.
Mantle canals is considered the derived state, as it represents a
migration from the body cavity, where gametes are produced.

\emph{Phoronis}: Following coding for class in Carlson (1995) Appendix
1, character 9.

\emph{Lingula}: Following coding for class in Carlson (1995) Appendix 1,
character 9.

\emph{Pelagodiscus atlanticus}: Following coding for class in Carlson
(1995) Appendix 1, character 9.

\emph{Novocrania}: Following coding for class in Carlson (1995) Appendix
1, character 9.

\emph{Terebratulina}: Following coding for class in Carlson (1995)
Appendix 1, character 9.

\hypertarget{embryonic-shell}{%
\section*{Embryonic shell}\label{embryonic-shell}}
\addcontentsline{toc}{section}{Embryonic shell}

\includegraphics{Brachiopod_phylogeny_files/figure-latex/unnamed-chunk-5-62.pdf}

\textbf{Character 62 : Embryonic shell }

\begin{quote}
0: `Absent'\\
1: `Present'\\
Neomorphic character.
\end{quote}

The embryonic shell or protegulum is secreted by the embryo immediately
before hatching.

\emph{Namacalathus}: Inapplicable insofar as reproduction occurs by
budding; there is no evidence for a free-living larval stage.
Nevertheless, the presence of a sexual reproductive phase in addition to
asexual reproduction cannot be discounted.

\emph{Clupeafumosus socialis}: Described by Topper et al. (2013).

\hypertarget{embryonic-shell-morphology}{%
\section*{Embryonic shell:
Morphology}\label{embryonic-shell-morphology}}
\addcontentsline{toc}{section}{Embryonic shell: Morphology}

\includegraphics{Brachiopod_phylogeny_files/figure-latex/unnamed-chunk-5-63.pdf}

\textbf{Character 63 : Embryonic shell: Morphology }

\begin{quote}
1: `Flat, disc-like (cf.~Micrina)'\\
2: `Three prominent lobes forming a Y (cf.~Paterimitra)'\\
3: `Spherical'\\
Transformational character.
\end{quote}

\emph{Micrina} resembles linguliforms (Holmer et al 2011): in both, the
mitral protegulum has one pair of setal sacs enclosed by lateral lobes,
whereas the ventral protegulum has two lateral setal tubes.

\emph{Paterimitra} and \emph{Salanygolina} have ``identical'' ventral
larval shells (Holmer et al 2011), resembling the shape of a ship's
propeller.

Hyoliths typically have a spherical larval shell;
\emph{Pedunculotheca}'s is seemingly cap-shaped.

\emph{Coolinia pecten}: See fig. 3 in Bassett et al. 2017

\emph{Lingula}: See fig. 159 in Williams et al. 1997

\emph{Askepasma toddense}: Renoid -- see fig. 4B3 in Topper et al. 2013

\emph{Micromitra}: Subtriangular -- essentially round

\emph{Pelagodiscus atlanticus}: See e.g.~fig 169 in Williams et al.
(1997)

\emph{Clupeafumosus socialis}: The flat larval shell of
\emph{Clupeafumosus} resembles that of \emph{Micrina} in outline (Topper
et al. 2013; cf.~Holmer et al. 2011).

\emph{Craniops}: The embryonic shell is more or less circular in outline
-- see Freeman \& Lundelius, 1999, fig. 6A

\emph{Mickwitzia muralensis}: Trifoliate appearance results from
prominent attachment rudiment and bunching of setal sacs (Balthasar
2009).

\hypertarget{embryonic-shell-extended-in-larvae}{%
\section*{Embryonic shell: Extended in
larvae}\label{embryonic-shell-extended-in-larvae}}
\addcontentsline{toc}{section}{Embryonic shell: Extended in larvae}

\includegraphics{Brachiopod_phylogeny_files/figure-latex/unnamed-chunk-5-64.pdf}

\textbf{Character 64 : Embryonic shell: Extended in larvae }

\begin{quote}
1: `Not extended; embryonic shell contiguous with adult shell'\\
2: `Extended into larval shell, separated from adult shell by prominent
nick'\\
Transformational character.
\end{quote}

Many taxa add to their embryonic shell during the larval phase of their
life cycle. The shell that exists at metamorphosis, marked by a halo or
nick point, is variously termed the ``first formed shell'',
``metamorphic shell'' or ``larval shell'' (Bassett \& Popov 2017).

\emph{Clupeafumosus socialis}: Described by Topper et al. (2013).

\emph{Craniops}: Prominent nick; see Freeman \& Lundelius 1999, fig. 6A

\emph{Eoobolus}: Nick point indicated by arrows in fig. 1 of Balthasar
(2009).

\hypertarget{larval-attachment-structure}{%
\section*{Larval attachment
structure}\label{larval-attachment-structure}}
\addcontentsline{toc}{section}{Larval attachment structure}

\includegraphics{Brachiopod_phylogeny_files/figure-latex/unnamed-chunk-5-65.pdf}

\textbf{Character 65 : Larval attachment structure }

\begin{quote}
0: `Without evidence of pedicle'\\
1: `With evidence of pedicle'\\
Neomorphic character.
\end{quote}

Embryonic shells of \emph{Micrina} and certain linguliforms exhibit a
transversely folded posterior extension that speaks of the original
presence of a pedicle in the embryo.

This is independent of the presence of an adult pedicle, which may arise
after metamorphosis.

\emph{Clupeafumosus socialis}: The larval shell embraces the pedicle
foramen, suggesting a larval attachment. See fig. 4 of Topper et al.
(2013).

\emph{Mickwitzia muralensis}: Note the posterior lobe related to the
attachment rudiment in fig. 2 of Balthasar 2009

\emph{Eoobolus}: Lobe related to the attachment rudiment (Balthasar
2009, fig. 2)

\hypertarget{setae-in-adults}{%
\section*{Setae in adults}\label{setae-in-adults}}
\addcontentsline{toc}{section}{Setae in adults}

\includegraphics{Brachiopod_phylogeny_files/figure-latex/unnamed-chunk-5-66.pdf}

\textbf{Character 66 : Setae in adults }

\begin{quote}
0: `Absent'\\
1: `Present'\\
Neomorphic character.
\end{quote}

Although preservation of setae (in adults) is exceptional, their
presence can be inferred from shelly material (see Holmer \& Caron
2006).

\emph{Acanthotretella}: Note that the setae do not obviously emerge from
tubes, leading Holmer and Caron to question their homology with the
setae of other taxa (\emph{Heliomedusa}, \emph{Mickwitzia}).

Both valves of \emph{Acanthotretella} were coveredd by long spine-like
and shell penetrating setae. The setae of A.decaius are usually
preserved along anterior and anterolateral margins (Hu et al. 2010).

\emph{Novocrania}: ``Adult craniids are without setae (a feature shared
with the thecideides, the\\
shells of which are also cemented).'' -- Williams et al. 2007

\emph{Clupeafumosus socialis}: Setal bundles interpreted as present in
acrotretids by Ushatinskaya (2016).

\hypertarget{setae-distribution}{%
\section*{Setae: Distribution}\label{setae-distribution}}
\addcontentsline{toc}{section}{Setae: Distribution}

\includegraphics{Brachiopod_phylogeny_files/figure-latex/unnamed-chunk-5-67.pdf}

\textbf{Character 67 : Setae: Distribution }

\begin{quote}
1: `Uniform'\\
2: `Only present at margins of shell'\\
Transformational character.
\end{quote}

Setae penetrate the valves of many brachiopods. In certain taxa, they
are apparent only at the margins of the valves, in association with the
commissure, being reduced or lost over the surface of the shell.

\emph{Eccentrotheca}: Skovsted et al (2011) assumed the setae may have
been present along the margin of the adapical opening, but there is no
fossil evidence.

\emph{Heliomedusa orienta}: Throughout the shell -- see Williams et al.
2007 -- causing the pustulose appearance remarked upon by Chen et al.
2007

\hypertarget{setae-present-in-larva}{%
\section*{Setae: present in larva}\label{setae-present-in-larva}}
\addcontentsline{toc}{section}{Setae: present in larva}

\includegraphics{Brachiopod_phylogeny_files/figure-latex/unnamed-chunk-5-68.pdf}

\textbf{Character 68 : Setae: present in larva }

\begin{quote}
0: `No evidence of setae in embryonic shell'\\
1: `Setae present'\\
Neomorphic character.
\end{quote}

The protegulum of \emph{Micrina} is penetrated with canals that were
originally associated with setae, a character that it has in common with
linguliforms (Holmer et al 2011).

\emph{Lingulellotreta malongensis}: Familial character: larval shell
smooth (williams et al.,2000, p.72).

\emph{Clupeafumosus socialis}: Setal bundles interpreted as present in
acrotretids by Ushatinskaya (2016).

\emph{Mickwitzia muralensis}: Four setal sacs

\hypertarget{setae-embryonic-setal-sacs}{%
\section*{Setae: Embryonic: Setal
sacs}\label{setae-embryonic-setal-sacs}}
\addcontentsline{toc}{section}{Setae: Embryonic: Setal sacs}

\includegraphics{Brachiopod_phylogeny_files/figure-latex/unnamed-chunk-5-69.pdf}

\textbf{Character 69 : Setae: Embryonic: Setal sacs }

\begin{quote}
0: `Absent'\\
1: `Present'\\
Neomorphic character.
\end{quote}

Setal sacs are recognizable as raised lumps on the juvenile shell (see
Bassett and Popov 2017).

\emph{Micrina} and linguliforms have setal sacs on their mitral/dorsal
embryonic shell, whereas these are absent in \emph{Paterimitra} (Holmer
et al 2011).

\emph{Lingula}: Lingulids'' larval setae are not arranged in bundles
(Carlson 1995)

\emph{Lingulellotreta malongensis}: Familial character: larval shell
smooth (williams et al.,2000, p.72).

\emph{Pelagodiscus atlanticus}: Three pairs (Carlson 1995)

\emph{Novocrania}: Three pairs (Carlson 1995)

\emph{Clupeafumosus socialis}: Setal bundles interpreted as present in
acrotretids by Ushatinskaya (2016).

\hypertarget{setae-embryonic-setal-sacs-number}{%
\section*{Setae: Embryonic: Setal sacs:
Number}\label{setae-embryonic-setal-sacs-number}}
\addcontentsline{toc}{section}{Setae: Embryonic: Setal sacs: Number}

\includegraphics{Brachiopod_phylogeny_files/figure-latex/unnamed-chunk-5-70.pdf}

\textbf{Character 70 : Setae: Embryonic: Setal sacs: Number }

\begin{quote}
1: `One pair'\\
2: `Two pairs'\\
3: `Three pairs'\\
Transformational character.
\end{quote}

Two pairs on e.g.~Coolina; one on e.g. \emph{Micrina}

\emph{Pelagodiscus atlanticus}: Three pairs (Carlson 1995)

\emph{Novocrania}: Three pairs (Carlson 1995)

\emph{Clupeafumosus socialis}: Two pairs identified in acrotretids by
Ushatinskaya (2016).

\emph{Mickwitzia muralensis}: See fig. 2 in Balthasar 2009

\hypertarget{pedicle}{%
\section*{Pedicle}\label{pedicle}}
\addcontentsline{toc}{section}{Pedicle}

\includegraphics{Brachiopod_phylogeny_files/figure-latex/unnamed-chunk-5-71.pdf}

\textbf{Character 71 : Pedicle }

\begin{quote}
0: `Absent'\\
1: `Present'\\
Neomorphic character.
\end{quote}

The brachiopod pedicle is a fleshy protuberance that emerges from the
posterior part of the body wall -- as denoted in fossil taxa by its
occurrence between the dorsal and ventral valves.

It is important to distinguish the pedicle from the ``pedicle sheath'',
a tubular extension of the umbo that grows by accretion from an isolated
portion of the ventral mantle. For discussion see Holmer et al. 2017 and
Bassett and Popov 2017.

\emph{Namacalathus}: There is no obvious way to homologise the
attachment structure with the ventral pedicle of brachiopods.

\emph{Lingulosacculus}: The absence of a pedicle is inferred from the
absence of an internal pedicle tube, and the absence of a pedicle at the
hinge.

\emph{Acanthotretella}: The attachment structure of
\emph{Acanthotretella} originates at the margin of the dorsal and
ventral valves; although it emerges from the umbo of the ventral valve,
the presence of an internal pedicle tube betrays its identity as a
pedicle, rather than a pedicle sheath.

The pedicle of \emph{Acanthotretella} emerges from a short extension of
the umbo of the ventral valve. This extension is contiguous with the
valve and presumably grew by accretion; its position and continuity with
the valve suggest its interpretation as a pedicle sheath that is
superseded as an attachment structure. On the other hand, its continuity
with the internal pedicle tube suggests that is may represent an
independent organ.

\emph{Heliomedusa orienta}: ``It seems unlikely that H. \emph{orienta}
possessed a pedicle that attached it to\\
the soft seafloor, like most other Chengjiang brachiopods.'' \ldots{}\\
``The putative pedicle illustrated by Chen et al. (2007: Figs. 4, 6, 7)
in fact is the mold of a three-dimensionally preserved visceral cavity''
-- Zhang et al. 2009

\emph{Nisusia sulcata}: Has a pedicle, rather than a pedicle sheath as
in \emph{Kutorgina} - see Holmer et al. 2017, 2018

\emph{Clupeafumosus socialis}: A pedicle was presumably present, but
only the foramen is preserved.

\emph{Craniops}: Attached apically by cementation

\emph{Mickwitzia muralensis}: An attachment structure is inferred based
on the presence of an opening (Balthasar 2004); this is assumed to have
been homologous with the brachiopod pedicle.

\hypertarget{pedicle-constitution}{%
\section*{Pedicle: Constitution}\label{pedicle-constitution}}
\addcontentsline{toc}{section}{Pedicle: Constitution}

\includegraphics{Brachiopod_phylogeny_files/figure-latex/unnamed-chunk-5-72.pdf}

\textbf{Character 72 : Pedicle: Constitution }

\begin{quote}
1: `Massive or uniform'\\
2: `Densely stacked tabular discs'\\
Transformational character.
\end{quote}

The pedicle of certain chengjiang rhynchonelliforms comprises ``densely
stacked, three dimensionally preserved, tabular discs'' (Holmer et al.
2018).\\
This contrasts with the uniform (`massive') pedicles of living taxa.

\emph{Terebratulina}: Extant rhynconellid pedicles are massive,
consisting of a thick outer chitinous cuticle, a pedicle epithelium, and
a core composed of collagen fibres and cartilage-like connective tissue
(Holmer et al. 2018).

\hypertarget{pedicle-biomineralization}{%
\section*{Pedicle: Biomineralization}\label{pedicle-biomineralization}}
\addcontentsline{toc}{section}{Pedicle: Biomineralization}

\includegraphics{Brachiopod_phylogeny_files/figure-latex/unnamed-chunk-5-73.pdf}

\textbf{Character 73 : Pedicle: Biomineralization }

\begin{quote}
1: `Absent'\\
2: `Present'\\
Transformational character.
\end{quote}

The pedicle of strophomenates such as \emph{Antigonambonites} is
biomineralized (Holmer et al. 2018)

\hypertarget{pedicle-bulb}{%
\section*{Pedicle: Bulb}\label{pedicle-bulb}}
\addcontentsline{toc}{section}{Pedicle: Bulb}

\includegraphics{Brachiopod_phylogeny_files/figure-latex/unnamed-chunk-5-74.pdf}

\textbf{Character 74 : Pedicle: Bulb }

\begin{quote}
0: `Absent'\\
1: `Present'\\
Neomorphic character.
\end{quote}

A bulb is an expanded region of the distal pedicle, often embedded into
the sediment to improve anchorage.

\emph{Acanthotretella}: Holmer and Caron interpret the presence of a
bulb as tentative; we score it as ambiguous.

\hypertarget{pedicle-distal-rootlets}{%
\section*{Pedicle: Distal rootlets}\label{pedicle-distal-rootlets}}
\addcontentsline{toc}{section}{Pedicle: Distal rootlets}

\includegraphics{Brachiopod_phylogeny_files/figure-latex/unnamed-chunk-5-75.pdf}

\textbf{Character 75 : Pedicle: Distal rootlets }

\begin{quote}
0: `Absent'\\
1: `Present'\\
Neomorphic character.
\end{quote}

Neomorphic.\\
Observed in \emph{Pedunculotheca} and Bethia (Sutton et al 2005).

\hypertarget{pedicle-tapering}{%
\section*{Pedicle: Tapering}\label{pedicle-tapering}}
\addcontentsline{toc}{section}{Pedicle: Tapering}

\includegraphics{Brachiopod_phylogeny_files/figure-latex/unnamed-chunk-5-76.pdf}

\textbf{Character 76 : Pedicle: Tapering }

\begin{quote}
1: `Uniform thickness'\\
2: `Tapering'\\
Transformational character.
\end{quote}

Holmer et al (2018) remark that the tapering aspect of the
\emph{Nisusia} pedicle recalls that of certain Chengjiang taxa
(\emph{Alisina}, \emph{Longtancunella}) whilst distinguishing it from
many other taxa (Eichwaldia, Bethia) in which the pedicle is a constant
thickness.

\emph{Pedunculotheca diania}: The pedicle thickness does not obviously
change between the apex of the shell and the holdfast.

\emph{Antigonambonites planus}: Tapered pedicle sheath with holdfast.

\hypertarget{pedicle-coelomic-region}{%
\section*{Pedicle: Coelomic region}\label{pedicle-coelomic-region}}
\addcontentsline{toc}{section}{Pedicle: Coelomic region}

\includegraphics{Brachiopod_phylogeny_files/figure-latex/unnamed-chunk-5-77.pdf}

\textbf{Character 77 : Pedicle: Coelomic region }

\begin{quote}
1: `Absent'\\
2: `Present'\\
Transformational character.
\end{quote}

Certain brachiopods, such as \emph{Acanthotretella}, exhibit a coelomic
cavity within the pedicle or pedicle sheath.

Treated as transformational as it is not clear that either state is
necessarily ancestral.

\emph{Nisusia sulcata}: A coleomic canal is inferred based on the ease
with which the pedicle is deformed (Holmer et al 2018), but its presence
is not known for certain so is coded ambiguous.

\hypertarget{pedicle-surface-ornament}{%
\section*{Pedicle: Surface ornament}\label{pedicle-surface-ornament}}
\addcontentsline{toc}{section}{Pedicle: Surface ornament}

\includegraphics{Brachiopod_phylogeny_files/figure-latex/unnamed-chunk-5-78.pdf}

\textbf{Character 78 : Pedicle: Surface ornament }

\begin{quote}
1: `Smooth'\\
2: `Irregular wrinkles'\\
3: `Regular annulations'\\
Transformational character.
\end{quote}

Annulations are regular rings that surround the pedicle, and are
distinguished from wrinkles, which are irregular in magnitude and
spacing, and may branch or fail to entirely encircle the pedicle.

\emph{Kutorgina chengjiangensis}: ``Pronounced concentric annular discs
disposed at intervals of 0.6--1.0 mm''

\emph{Acanthotretella}: ``The pedicle surface is ornamented with
pronounced annulated rings, disposed at intervals of about 0.2 mm''

\emph{Longtancunella chengjiangensis}: ``The preserved pedicle has
condensed annulations'' -- Zhang et al. 2011

\emph{Lingulellotreta malongensis}: Regularly annotated (see fig. 14.9
in Hou et al. 2017)

\emph{Nisusia sulcata}: The ``strong annulations'' vary significantly in
transverse thickness (Holmer et al. 2018), so it is not clear whether
these represent true annulations or wrinkles.

\emph{Antigonambonites planus}: ``The emerging pedicle has a consistent
shape in all the available specimens and is strongly annulated and
distally tapering'' -- Holmer et al. 2018

\emph{Alisina}: ``It appears that the pedicle lacks a coelomic space and
is distinctly annulated, with densely stacked tabular bodies'' -- Zhang
et al. 2011

\emph{Yuganotheca elegans}: Annulations present in median collar

\hypertarget{pedicle-position}{%
\section*{Pedicle: position}\label{pedicle-position}}
\addcontentsline{toc}{section}{Pedicle: position}

\includegraphics{Brachiopod_phylogeny_files/figure-latex/unnamed-chunk-5-79.pdf}

\textbf{Character 79 : Pedicle: position }

\begin{quote}
1: `At hinge'\\
2: `Ventral valve umbo via internal pedicle tube'\\
3: `Ventral valve umbo (pedicle sheath)'\\
Transformational character.
\end{quote}

The pedicle of certain brachiopods (e.g.~Ancanthotretella,
siphonotretides, acrotretides) travels from the hinge line to the umbo
(or in the case of Limgulellotreta manongensis, a foramen on the
pseudointerarea) within an internal pedicle tube. See Holmer \& Caron
(2006) for discussion.

The pedicle sheath, in contrast, emerges from the umbo of the ventral
valve without any indication of a relationship with the hinge.

\emph{Clupeafumosus socialis}: The presumed pedicle foramen is at the
ventral valve umbo, though no evidence of a pedicle sheath is present.

\hypertarget{mantle-canals-morphology}{%
\section*{Mantle canals: Morphology}\label{mantle-canals-morphology}}
\addcontentsline{toc}{section}{Mantle canals: Morphology}

\includegraphics{Brachiopod_phylogeny_files/figure-latex/unnamed-chunk-5-80.pdf}

\textbf{Character 80 : Mantle canals: Morphology }

\begin{quote}
0: `Pinnate (=lemniscate)'\\
1: `Bifurcate'\\
2: `Baculate'\\
3: `Saccate'\\
Neomorphic character.
\end{quote}

The morphology of dorsal and ventral canals is identical in all included
taxa, so is assumed not to be independent - hence the use of a single
character (contra Williams et al. 2000).

For a description of terms see Williams et al. (1997, 2000).

Pinnate = ``rapidly branch into a number of subequal, radially disposed
canals''\\
Bifurcate = ``vascula lateralia in both valves divide immediately after
leaving the body cavity''\\
Baculate = ``extend forward without any major dichotomy or bifurcation''
(Williams et al. 1997 p.~418)\\
Saccate = ``pouchlike sinuses lying wholly posterior to the arcuate
vascula media'' (ibid., p412)

\emph{Lingulosacculus}: Baculate vascula media - Balthasar \&
Butterfield (2009)

\emph{Tomteluva perturbata}: Preservation not adequate to evaluate
(Streng 2016).

\emph{Mummpikia nuda}: ``Poorly resolved'' -- Balthasar 2008

\emph{Coolinia pecten}: Not reported in Treatise (Williams et al. 2000).

\emph{Kutorgina chengjiangensis}: Following Table 15 in Williams et al.
(2000) (for Neocrania).

\emph{Salanygolina}: Coded uncertain in Appendix 2 in Williams et al.
(1998).

\emph{Lingula}: Following Table 6 in Williams et al. (2000).

\emph{Acanthotretella}: Following Table 6, for Siphonotretidae, in
Williams et al. (2000).

\emph{Heliomedusa orienta}: Described as pinnate by Jin \& Wang (1992)

\emph{Longtancunella chengjiangensis}: Reported by Zhang et al. (2007,
2011) though the interpretation is tentative.

\emph{Lingulellotreta malongensis}: Following Table 6 in Williams et al.
(2000).

\emph{Askepasma toddense}: Described as pinnate (at least in ventral
valve) by Williams et al. (1998, p.~250).

\emph{Micromitra}: Described as saccate by Williams et al. (1998).

\emph{Nisusia sulcata}: Following Table 15 in Williams et al. (2000).

\emph{Pelagodiscus atlanticus}: Following Table 6, for Discinidae, in
Williams et al. (2000).

\emph{Novocrania}: Following Table 15 in Williams et al. (2000) (for
Neocrania).

\emph{Terebratulina}: ``In modern terebratulides, the vascula media are
subordinate to the lemniscate or pinnate vascula genitalia'' -- Williams
et al. 1997

\emph{Antigonambonites planus}: Not reported in Treatise (Williams et
al. 2000).

\emph{Alisina}: Following Table 15 in Williams et al. (2000).

\emph{Orthis}: Sacculate (sometimes digitate in dorsal valve) (Williams
et al. 2000, p716)

\emph{Gasconsia}: Following Table 15 in Williams et al. (2000).

\emph{Glyptoria}: Following Appendix 2 (char. 21) in Williams et al.
(1998).

\emph{Clupeafumosus socialis}: Following Table 8 (for Acrotreta) in
Williams et al. (2000).

\emph{Craniops}: Not reported from fossil material

\hypertarget{mantle-canals-vascula-lateralia}{%
\section*{Mantle canals: vascula
lateralia}\label{mantle-canals-vascula-lateralia}}
\addcontentsline{toc}{section}{Mantle canals: vascula lateralia}

\includegraphics{Brachiopod_phylogeny_files/figure-latex/unnamed-chunk-5-81.pdf}

\textbf{Character 81 : Mantle canals: vascula lateralia }

\begin{quote}
0: `Absent'\\
1: `Present'\\
Neomorphic character.
\end{quote}

We treat the vascula lateralia as equivalent to the vascula genitalia of
articulated brachiopods, allowing phylogenetic analysis to test their
proposed homology.

Williams et al (1997) write:\\
``The mantle canal system of most of the organophosphate-shelled species
consists of a single pair of main trunks in the ventral mantle (vascula
lateralia) and two pairs in the dorsal mantle, one pair (vascula
lateralia) occupying a similar position to the single pair in the
ventral mantle and a second pair projecting from the body cavity near
the midline of the valve. This latter pair may be termed the vascula
media, but whether they are strictly homologous with the vascula media
of articulated brachiopods is a matter of\\
opinion. It is also impossible to assert that the vascula lateralia are
the homologues of the vascula myaria or genitalia of articulated
species, although they are likely to be so as they arise in a comparable
position.''

``In inarticulated brachiopods, two main mantle canals (vascula
lateralia) emerge from the main body cavity through muscular valves and
bifurcate distally to produce an increasingly dense array of blindly
ending branches near the periphery of the mantle (Fig. 71.1-71.2).''

\emph{Tomteluva perturbata}: Preservation not adequate to evaluate
(Streng 2016).

\emph{Kutorgina chengjiangensis}: Following Table 15 in Williams et al.
(2000).

\emph{Acanthotretella}: Following Table 8 (which records presence in
Siphonotreta) in Williams et al. (2000).

\emph{Heliomedusa orienta}: Present: Williams et al. (2000); Jin \& Wang
(1992).

\emph{Longtancunella chengjiangensis}: Presence is possible but requires
interpretation that is not unambiguous:

``In the dorsal valve, there can be seen two baculate grooves that arise
from the\\
anterior body wall at an antero-lateral position. These two grooves
(Figs 4H, 5D) could be taken to represent the vascula lateralia'' --
Zhang et al 2007

\emph{Lingulellotreta malongensis}: Present (Williams et al. 2000).

\emph{Askepasma toddense}: ``Laurie (1987) has shown that arcuate
vascula media were present in the mantles of both valves as were
pouchlike vascula genitalia, especially in the ventral valve'' --
Williams et al. 1997

\emph{Micromitra}: ``Laurie (1987) has shown that arcuate vascula media
were present in the mantles of both valves as were pouchlike vascula
genitalia, especially in the ventral valve'' -- Williams et al. 1997

\emph{Nisusia sulcata}: Following Table 15 in Williams et al. (2000).

\emph{Pelagodiscus atlanticus}: Wollowing Lochothele (Discinidae), Fig.
43.4a in Williams et al. (2000).

\emph{Novocrania}: Following Table 15 in Williams et al. (2000) (for
Neocrania).\\
Williams et al. (2000) write ``Holocene craniides have only a single
pair of main trunks in both valves, corresponding to the vascula
lateralia''\\
Williams et al. (2007) reiterate this position (p.~2875), at least for
the ventral valve.

\emph{Terebratulina}: = vascula genitalia

\emph{Alisina}: Following Table 15 in Williams et al. (2000).

\emph{Orthis}: = vascula genetalia

\emph{Gasconsia}: Following Table 15 in Williams et al. (2000).

\emph{Clupeafumosus socialis}: Presence indicated in Table 8 (for
Acrotreta) in Williams et al. (2000).

\emph{Yuganotheca elegans}: Based on the figures and sketches in Zhang
et al. 2014 (and supplementary material), the mantle canals are
interpreted as lateral, with no clear vascula media present.

\hypertarget{mantle-canals-vascula-media}{%
\section*{Mantle canals: vascula
media}\label{mantle-canals-vascula-media}}
\addcontentsline{toc}{section}{Mantle canals: vascula media}

\includegraphics{Brachiopod_phylogeny_files/figure-latex/unnamed-chunk-5-82.pdf}

\textbf{Character 82 : Mantle canals: vascula media }

\begin{quote}
0: `Absent'\\
1: `Present (in dorsal valve)'\\
Neomorphic character.
\end{quote}

Williams et al. (1997) note that in addition to the vascula lateralia,
``Discinisca has two additional mantle canals emanating from the body
cavity into the dorsal mantle (vascula media).''

These structures are only evident in the dorsal valve for the included
taxa, so only a single character is necessary.

\emph{Tomteluva perturbata}: Preservation not adequate to evaluate
(Streng 2016).

\emph{Kutorgina chengjiangensis}: Following Table 15 in Williams et al.
(2000).

\emph{Lingula}: Following Table 6 in Williams et al. (2000).

\emph{Acanthotretella}: Following Table 6 (for Siphonotretidae) in
Williams et al. (2000).

\emph{Heliomedusa orienta}: Present: Williams et al. (2000) p162, Jin \&
Wang (1992).

\emph{Longtancunella chengjiangensis}: Reported by Zhang et al. (2007)
though the interpretation is tentative.

\emph{Lingulellotreta malongensis}: Following Table 6 in Williams et al.
(2000).

\emph{Askepasma toddense}: Following Table 6 (for Paterinidae) in
Williams et al. (2000).

\emph{Micromitra}: Reported by Williams et al. (1998).

\emph{Nisusia sulcata}: Following Table 15 in Williams et al. (2000).

\emph{Pelagodiscus atlanticus}: Following Table 6 (for Discinidae) in
Williams et al. (2000).

\emph{Novocrania}: Williams et al. (2000) write ``Holocene craniides
have only a single pair of main trunks in both valves, corresponding to
the vascula lateralia'' -- an observation reflected in their table 15
(for Neocrania).\\
But in contrast, Williams et al. 2007, p.~2875, identify the dorsal
valve''s canals as a vascula media in living cranidds (though both are
lateralia in Ordoviian craniides). This character is therefore coded as
ambiguous.

\emph{Terebratulina}: ``In modern terebratulides, the vascula media are
subordinate to the lemniscate or pinnate vascula genitalia'' -- Williams
et al. 1997 p417

\emph{Alisina}: Following Table 15 in Williams et al. (2000).

\emph{Orthis}: From idealised morphology in Williams et al. (2000)

\emph{Gasconsia}: Following Table 15 in Williams et al. (2000).

\emph{Glyptoria}: Present and divergent (Williams et al. 2000).

\emph{Clupeafumosus socialis}: Following Hadrotreta schematic in
Williams et al. (2000).

\emph{Yuganotheca elegans}: Based on the figures and sketches in Zhang
et al. 2014 (and supplementary material), the mantle canals are
interpreted as lateral, with no clear vascula media present.

\emph{Eoobolus}: Fig. 5 in Balthasar 2009.

\hypertarget{mantle-canals-vascula-terminalia}{%
\section*{Mantle canals: vascula
terminalia}\label{mantle-canals-vascula-terminalia}}
\addcontentsline{toc}{section}{Mantle canals: vascula terminalia}

\includegraphics{Brachiopod_phylogeny_files/figure-latex/unnamed-chunk-5-83.pdf}

\textbf{Character 83 : Mantle canals: vascula terminalia }

\begin{quote}
0: `Exclusively marginal (peripheral)'\\
1: `Directed peripherally and (intero)medially'\\
Neomorphic character.
\end{quote}

Presumed to be connected with setal follicles in life (Williams et al.
1998). See Williams et al. (2000) for discussion.

\emph{Kutorgina chengjiangensis}: Coded uncertain in Appendix 2 in
Williams et al. (1998).

\emph{Salanygolina}: Coded uncertain in Appendix 2 in Williams et al.
(1998).

\emph{Lingula}: Peripheral and medial for all Lingulata (Williams et al.
2000).

\emph{Acanthotretella}: Preservation not clear enough to score with
certainty (Holmer \& Caron 2006)

\emph{Heliomedusa orienta}: Inferred from Jin \& Wang (1992).

\emph{Askepasma toddense}: Peripheral only (Williams et al. 1998, 2000).

\emph{Micromitra}: Peripheral only (Williams et al. 1998, 2000).

\emph{Pelagodiscus atlanticus}: Wollowing Lochothele (Discinidae), Fig.
43.4a in Williams et al. (2000).

\emph{Novocrania}: Peripheral only (Williams et al. 2000, p.158).

\emph{Terebratulina}: Following idealised plectolophous terebratulid of
Emig (1992).

\emph{Alisina}: Interomedial vascula terminalia not reported by Williams
et al. (2000).

\emph{Orthis}: See schematics in Williams et al. (2000)

\emph{Glyptoria}: Following Appendix 2 in Williams et al. (1998).

\hypertarget{lophophore-tentacle-disposition}{%
\section*{Lophophore: tentacle
disposition}\label{lophophore-tentacle-disposition}}
\addcontentsline{toc}{section}{Lophophore: tentacle disposition}

\includegraphics{Brachiopod_phylogeny_files/figure-latex/unnamed-chunk-5-84.pdf}

\textbf{Character 84 : Lophophore: tentacle disposition }

\begin{quote}
1: `Single side'\\
2: `Both sides'\\
Transformational character.
\end{quote}

Tentacles may occur along one or both sides of the axis of the
lophophore arm (Carlson 1995).

\emph{Lingulosacculus}: Preservation insufficient to evaluate

\emph{Phoronis}: Following coding for higher group in Carlson 1995,
Appendix 1, character 36.

\emph{Kutorgina chengjiangensis}: Tentacles ``cannot be confidently
demonstrated in the available specimens.'' -- Zhang et al. 2007

\emph{Lingula}: Following coding for higher group in Carlson 1995,
Appendix 1, character 36.

\emph{Acanthotretella}: Preservation insufficient to evaluate (Caron \&
Holmer 2006)

\emph{Heliomedusa orienta}: ``Each lophophoral arm bears a row of long,
slender flexible tentacles'' -- Zhang et al. 2009

\emph{Longtancunella chengjiangensis}: Inadequately preserved to
evaluate.

\emph{Lingulellotreta malongensis}: ``The tentacles are clearly visible,
and closely arranged in a single palisade'' -- Zhang et al. 2004

\emph{Pelagodiscus atlanticus}: Following coding for higher group in
Carlson 1995, Appendix 1, character 36.

\emph{Novocrania}: Following coding for higher group in Carlson 1995,
Appendix 1, character 36.

\emph{Terebratulina}: Following coding for higher group in Carlson 1995,
Appendix 1, character 36.

\emph{Alisina}: Preservation inadequate.

\hypertarget{lophophore-tentacle-rows-per-side-trocholophe-stage}{%
\section*{Lophophore: tentacle rows per side: trocholophe
stage}\label{lophophore-tentacle-rows-per-side-trocholophe-stage}}
\addcontentsline{toc}{section}{Lophophore: tentacle rows per side:
trocholophe stage}

\includegraphics{Brachiopod_phylogeny_files/figure-latex/unnamed-chunk-5-85.pdf}

\textbf{Character 85 : Lophophore: tentacle rows per side: trocholophe
stage }

\begin{quote}
0: `Single row'\\
1: `Ablabial and adlabial row'\\
Neomorphic character.
\end{quote}

After Carlson (1995), character 37. Lophophore tentacles are commonly
arranged into an ablabial and adlablial row, with ablabial tentacles
sometimes added later in development.

\emph{Phoronis}: Following coding for higher taxon in Carlson (1995),
appendix 1, character 37.

\emph{Lingula}: Following coding for higher taxon in Carlson (1995),
appendix 1, character 37.

\emph{Pelagodiscus atlanticus}: Following coding for higher taxon in
Carlson (1995), appendix 1, character 37.

\emph{Novocrania}: Following coding for higher taxon in Carlson (1995),
appendix 1, character 37. Also states in Williams et al. 2000, p.~158.

\emph{Terebratulina}: Following coding for higher taxon in Carlson
(1995), appendix 1, character 37.

\hypertarget{lophophore-tentacle-rows-per-side-post-trocholophe-stage}{%
\section*{Lophophore: tentacle rows per side: post-trocholophe
stage}\label{lophophore-tentacle-rows-per-side-post-trocholophe-stage}}
\addcontentsline{toc}{section}{Lophophore: tentacle rows per side:
post-trocholophe stage}

\includegraphics{Brachiopod_phylogeny_files/figure-latex/unnamed-chunk-5-86.pdf}

\textbf{Character 86 : Lophophore: tentacle rows per side:
post-trocholophe stage }

\begin{quote}
0: `Single row'\\
1: `Adbalial and ablabial row'\\
Neomorphic character.
\end{quote}

After Carlson (1995), character 37. Lophophore tentacles are commonly
arranged into an ablabial and adlablial row, with ablabial tentacles
sometimes added later in development.

\emph{Lingulosacculus}: Preservation insufficient to evaluate

\emph{Phoronis}: Following coding for higher taxon in Carlson (1995),
appendix 1, character 37.

\emph{Kutorgina chengjiangensis}: Tentacles ``cannot be confidently
demonstrated in the available specimens.'' -- Zhang et al. 2007

\emph{Lingula}: Following coding for higher taxon in Carlson (1995),
appendix 1, character 37.

\emph{Acanthotretella}: Preservation insufficient to evaluate (Caron \&
Holmer 2006)

\emph{Heliomedusa orienta}: ``The lophophoral arms bear laterofrontal
tentacles with a double row of cilia along their lateral edge, as in
extant lingulid brachiopods'' -- Zhang et al. 2009

\emph{Lingulellotreta malongensis}: Single palisade - Zhang et al. 2004

\emph{Pelagodiscus atlanticus}: Following coding for higher taxon in
Carlson (1995), appendix 1, character 37.

\emph{Novocrania}: Following coding for higher taxon in Carlson (1995),
appendix 1, character 37.

\emph{Terebratulina}: Following coding for higher taxon in Carlson
(1995), appendix 1, character 37.

\emph{Yuganotheca elegans}: ``helical lophophore fringed with a single
row of thick, widely spaced, parallel-sided and hollow tentacles'' --
Zhang et al. 2014

\hypertarget{lophophore-median-tentacle-in-early-development}{%
\section*{Lophophore: Median tentacle in early
development}\label{lophophore-median-tentacle-in-early-development}}
\addcontentsline{toc}{section}{Lophophore: Median tentacle in early
development}

\includegraphics{Brachiopod_phylogeny_files/figure-latex/unnamed-chunk-5-87.pdf}

\textbf{Character 87 : Lophophore: Median tentacle in early development
}

\begin{quote}
0: `Absent'\\
1: `Present'\\
Neomorphic character.
\end{quote}

Following character 28 in Carlson 1995. Certain taxa exhibit a median
tentacle early in development that is lost at some point in ontogeny.

\emph{Pedunculotheca diania}: Lophophore ontogeny presently unknown.

\emph{Micrina}: Lophophore ontogeny presently unknown.

\emph{Paterimitra}: Lophophore ontogeny presently unknown.

\emph{Namacalathus}: Lophophore ontogeny presently unknown.

\emph{Eccentrotheca}: Lophophore ontogeny presently unknown.

\emph{Haplophrentis carinatus}: Lophophore ontogeny presently unknown.

\emph{Lingulosacculus}: Lophophore ontogeny presently unknown.

\emph{Tomteluva perturbata}: Lophophore ontogeny presently unknown.

\emph{Mummpikia nuda}: Lophophore ontogeny presently unknown.

\emph{Coolinia pecten}: Lophophore ontogeny presently unknown.

\emph{Kutorgina chengjiangensis}: Lophophore ontogeny presently unknown.

\emph{Salanygolina}: Lophophore ontogeny presently unknown.

\emph{Dailyatia}: Lophophore ontogeny presently unknown.

\emph{Acanthotretella}: Lophophore ontogeny presently unknown.

\emph{Heliomedusa orienta}: Lophophore ontogeny presently unknown.

\emph{Longtancunella chengjiangensis}: Lophophore ontogeny presently
unknown.

\emph{Lingulellotreta malongensis}: Lophophore ontogeny presently
unknown.

\emph{Askepasma toddense}: Lophophore ontogeny presently unknown.

\emph{Micromitra}: Lophophore ontogeny presently unknown.

\emph{Nisusia sulcata}: Lophophore ontogeny presently unknown.

\emph{Antigonambonites planus}: Lophophore ontogeny presently unknown.

\emph{Alisina}: Lophophore ontogeny presently unknown.

\emph{Orthis}: Lophophore ontogeny presently unknown.

\emph{Gasconsia}: Lophophore ontogeny presently unknown.

\emph{Glyptoria}: Lophophore ontogeny presently unknown.

\emph{Clupeafumosus socialis}: Lophophore ontogeny presently unknown.

\emph{Yuganotheca elegans}: Lophophore ontogeny presently unknown.

\hypertarget{lophophore-forms-closed-loop}{%
\section*{Lophophore: forms closed
loop}\label{lophophore-forms-closed-loop}}
\addcontentsline{toc}{section}{Lophophore: forms closed loop}

\includegraphics{Brachiopod_phylogeny_files/figure-latex/unnamed-chunk-5-88.pdf}

\textbf{Character 88 : Lophophore: forms closed loop }

\begin{quote}
1: `Diverging laterally'\\
2: `Closed loop'\\
Transformational character.
\end{quote}

Whereas the lophophore of crown-group brachiopods typically forms a
closed loop, those of Haplophorentis and \emph{Heliomedusa} diverge
laterally (Moysiuk et al 2017).

\emph{Namacalathus}: The existence of a lophophore is speculative.

\emph{Lingulosacculus}: Two diverging arms of the lophophore are
preserved (Balthasar \& Butterfield 2009)

\emph{Longtancunella chengjiangensis}: Two distinct, diverging arms
reconstructed by Zhang et al. 2007

\emph{Nisusia sulcata}: No specimens of \emph{Nisusia} preserve the
lophophore.

\hypertarget{lophophore-coiling-direction}{%
\section*{Lophophore: coiling
direction}\label{lophophore-coiling-direction}}
\addcontentsline{toc}{section}{Lophophore: coiling direction}

\includegraphics{Brachiopod_phylogeny_files/figure-latex/unnamed-chunk-5-89.pdf}

\textbf{Character 89 : Lophophore: coiling direction }

\begin{quote}
1: `Anteriad'\\
2: `Posteriad'\\
Transformational character.
\end{quote}

The lophophore arms of \emph{Heliomedusa} and \emph{Haplophrentis} arch
posteriad, rather than anteriad as in lingulids. See Zhang et al. 2009;
Moysuik et al. 2017.

\emph{Phoronis}: Coiling in direction of anus (i.e.~posteriad)

\emph{Acanthotretella}: Arms proceed anteriad before recurving.

\emph{Lingulellotreta malongensis}: Arms proceed anteriad before
recurving.

\emph{Pelagodiscus atlanticus}: ``converging anteriorly and coiling
anterior to the body cavity'' -- Zhang et al. 2009

\hypertarget{lophophore-adjustor-muscle}{%
\section*{Lophophore: adjustor
muscle}\label{lophophore-adjustor-muscle}}
\addcontentsline{toc}{section}{Lophophore: adjustor muscle}

\includegraphics{Brachiopod_phylogeny_files/figure-latex/unnamed-chunk-5-90.pdf}

\textbf{Character 90 : Lophophore: adjustor muscle }

\begin{quote}
0: `Absent'\\
1: `Present'\\
Neomorphic character.
\end{quote}

Following character 55 in Carlson (1995). Not possible to code in most
fossil taxa.

\emph{Pedunculotheca diania}: Preservation not adequate to evaluate
presence or absence of this muscle.

\emph{Micrina}: Preservation not adequate to evaluate presence or
absence of this muscle.

\emph{Paterimitra}: Preservation not adequate to evaluate presence or
absence of this muscle.

\emph{Namacalathus}: Preservation not adequate to evaluate presence or
absence of this muscle.

\emph{Eccentrotheca}: Preservation not adequate to evaluate presence or
absence of this muscle.

\emph{Haplophrentis carinatus}: Preservation not adequate to evaluate
presence or absence of this muscle.

\emph{Lingulosacculus}: Preservation not adequate to evaluate presence
or absence of this muscle.

\emph{Phoronis}: Following coding for higher taxon in Carlson (1995),
appendix 1, character 55.

\emph{Tomteluva perturbata}: Preservation not adequate to evaluate
presence or absence of this muscle.

\emph{Mummpikia nuda}: Preservation not adequate to evaluate presence or
absence of this muscle.

\emph{Coolinia pecten}: Preservation not adequate to evaluate presence
or absence of this muscle.

\emph{Kutorgina chengjiangensis}: Preservation not adequate to evaluate
presence or absence of this muscle.

\emph{Salanygolina}: Preservation not adequate to evaluate presence or
absence of this muscle.

\emph{Dailyatia}: Preservation not adequate to evaluate presence or
absence of this muscle.

\emph{Lingula}: Following coding for higher taxon in Carlson (1995),
appendix 1, character 55.

\emph{Acanthotretella}: Preservation not adequate to evaluate presence
or absence of this muscle.

\emph{Heliomedusa orienta}: Preservation not adequate to evaluate
presence or absence of this muscle.

\emph{Longtancunella chengjiangensis}: Preservation not adequate to
evaluate presence or absence of this muscle.

\emph{Lingulellotreta malongensis}: Preservation not adequate to
evaluate presence or absence of this muscle.

\emph{Askepasma toddense}: Preservation not adequate to evaluate
presence or absence of this muscle.

\emph{Micromitra}: Preservation not adequate to evaluate presence or
absence of this muscle.

\emph{Nisusia sulcata}: Preservation not adequate to evaluate presence
or absence of this muscle.

\emph{Pelagodiscus atlanticus}: Following coding for higher taxon in
Carlson (1995), appendix 1, character 55.

\emph{Novocrania}: Following coding for higher taxon in Carlson (1995),
appendix 1, character 55.

\emph{Terebratulina}: Following coding for higher taxon in Carlson
(1995), appendix 1, character 55.

\emph{Antigonambonites planus}: Preservation not adequate to evaluate
presence or absence of this muscle.

\emph{Alisina}: Preservation not adequate to evaluate presence or
absence of this muscle.

\emph{Orthis}: Preservation not adequate to evaluate presence or absence
of this muscle.

\emph{Gasconsia}: Preservation not adequate to evaluate presence or
absence of this muscle.

\emph{Glyptoria}: Preservation not adequate to evaluate presence or
absence of this muscle.

\emph{Clupeafumosus socialis}: Preservation not adequate to evaluate
presence or absence of this muscle.

\emph{Yuganotheca elegans}: Preservation not adequate to evaluate
presence or absence of this muscle.

\hypertarget{prominent-pharynx}{%
\section*{Prominent pharynx}\label{prominent-pharynx}}
\addcontentsline{toc}{section}{Prominent pharynx}

\includegraphics{Brachiopod_phylogeny_files/figure-latex/unnamed-chunk-5-91.pdf}

\textbf{Character 91 : Prominent pharynx }

\begin{quote}
0: `Absent'\\
1: `Present'\\
Neomorphic character.
\end{quote}

Hyoliths exhibit a prominent protrusible muscular pharynx at the base of
the lophophore (Moysiuk et al. 2017); no equivalent feature is observed
in brachiopods.

\emph{Heliomedusa orienta}: Corresponding to the ``neck'' of the
vase-shaped visceral cavity reported by Zhang et al. 2009

\emph{Yuganotheca elegans}: Possibly present, following interpretation
of mouth (see fig. 2c, d in Zhang et al. 2014)

\hypertarget{anus}{%
\section*{Anus}\label{anus}}
\addcontentsline{toc}{section}{Anus}

\includegraphics{Brachiopod_phylogeny_files/figure-latex/unnamed-chunk-5-92.pdf}

\textbf{Character 92 : Anus }

\begin{quote}
1: `Absent: digestive tract is blind sac'\\
2: `Present: through-gut'\\
Transformational character.
\end{quote}

The digestive tract may either constitute a blind sac, or a through gut
with anus.

\emph{Kutorgina chengjiangensis}: Although ``the possibility of a blind
ending may not be completely eliminated {[}\ldots{}{]} the weight of
evidence {[}\ldots{}{]} leads us to reject the possibility of a
blind-ending intestine'' -- Zhang et al. 2007, p.~1399

\emph{Glyptoria}: Scored according to familial level feature.

\hypertarget{anus-migration}{%
\section*{Anus: migration}\label{anus-migration}}
\addcontentsline{toc}{section}{Anus: migration}

\includegraphics{Brachiopod_phylogeny_files/figure-latex/unnamed-chunk-5-93.pdf}

\textbf{Character 93 : Anus: migration }

\begin{quote}
0: `Not migrated: straight gut with posterior anus'\\
1: `Migrated: anus has migrated posteriad to create U-shaped gut'\\
Neomorphic character.
\end{quote}

``The relative position of the mouth and anus in the larvae of
brachiopods and phoronids is similar: posterior anus and anterior
mouth'' -- Williams et al. 2007, p.~2884

\emph{Kutorgina chengjiangensis}: ``Five specimens have an exceptionally
preserved digestive tract, dorsally curved, with a putative
dorso-terminal anus located near the proximal end of a pedicle'' --
Zhang et al. 2007

\emph{Terebratulina}: ``In rhynchonelliforms, the gut curves somewhat
into a C-shape and the (blind) anus becomes posteroventral in
position.'' -- Williams et al. 2007,\\
p.~2884

\hypertarget{anus-migration-within-ring-of-tentacles}{%
\section*{Anus: migration: within ring of
tentacles}\label{anus-migration-within-ring-of-tentacles}}
\addcontentsline{toc}{section}{Anus: migration: within ring of
tentacles}

\includegraphics{Brachiopod_phylogeny_files/figure-latex/unnamed-chunk-5-94.pdf}

\textbf{Character 94 : Anus: migration: within ring of tentacles }

\begin{quote}
1: `Not within ring of tentacles'\\
2: `Anterior - within ring of feeding tentacles'\\
Transformational character.
\end{quote}

A migrated anus may be located laterally or within the lophophore ring
(as in entoprocts).

\emph{Kutorgina chengjiangensis}: ``Presumed to terminate in a
functional anus located near the proximal end of the pedicle.'' -- Zhang
et al. 2007

\hypertarget{anus-migration-position}{%
\section*{Anus: migration: position}\label{anus-migration-position}}
\addcontentsline{toc}{section}{Anus: migration: position}

\includegraphics{Brachiopod_phylogeny_files/figure-latex/unnamed-chunk-5-95.pdf}

\textbf{Character 95 : Anus: migration: position }

\begin{quote}
1: `Left'\\
2: `Right'\\
3: `Dorsally'\\
4: `Ventrally'\\
Transformational character.
\end{quote}

If the anus is not within the ring of tentacles, in which direction is
it oriented?

\emph{Haplophrentis carinatus}: Opening to the right -- see figures 1,
3, and extended data 5 in Moysiuk et al. (2017). The text states in
error that the anus is to the left of the midline.

\emph{Lingulosacculus}: ``This same arrangement occurs in L.
\emph{nuda}, with the looped dark line tracking the same course as the
exceptionally preserved guts of Chengjiang lingulellotretids, including
the median position of its posterior loop and the sharp right turn as it
exits the posterior extension of the ventral valve'' (Balthasar et
al.,2009, p.310).

\emph{Kutorgina chengjiangensis}: ``Five specimens have an exceptionally
preserved digestive tract, dorsally curved, with a putative
dorso-terminal anus located near the proximal end of a pedicle'' --
Zhang et al. 2007

\emph{Lingula}: ``In the lingulids, the {[}intestine{]} follows an
oblique course anteriorly to open at the anus on the right body wall.''
-- Williams et al. 1997, p.~89

\emph{Longtancunella chengjiangensis}: ``The intestine extends
posteriorly, and then turns right to continue as a tortuous strand,
finally terminating at the latero-median position of the anterior body
wall'' -- Zhang et al. 2007

\emph{Lingulellotreta malongensis}: ``finally terminating in an anal
opening on the right anterior\\
body wall'' (Zhang et al., 2007, p.66).

\emph{Terebratulina}: ``In rhynchonelliforms, the gut curves somewhat
into a C-shape and the (blind) anus becomes posteroventral in
position.'' -- Williams et al. 2007,\\
p.~2884

\emph{Yuganotheca elegans}: The identification of the ``very poorly
impressed possible anus at the lateral side of the anterior body wall''
is not yet confident, so this character is coded as not presently
available.

\citep{Skovsted2008Thescleritome, Holmer2006Aspinose, Clarkson2013, Holmer2018Theattachment, Bassett2017Earliestontogeny, Balthasar2008iMummpikia, Williams1997PartH, Kouchinsky2000Skeletalmicrostructures, Holmer1989MiddleOrdovician, Cusack1999Chemicostructural, Balthasar2004Shellstructure, Holmer2011Firstrecord, Holmer2008EarlyCambrian, Larsson2014iPaterimitra, Holmer2009Theenigmatic, Zhang2011Theexceptionally, Streng2016Anew, Moysiuk2017Hyolithsare, Zhang2009Architectureand, Zhang2007Noteon, Balthasar2009EarlyCambrian, 0, Zhuravlev2015Ediacaranskeletal, Bassett2008Earlyontogeny, Holmer2018Evolutionarysignificance, Popov2007Earliestontogeny, Bassett2001Functionalmorphology, Zhang2011Anobolellate, Skovsted2015Theearly, Skovsted2011Scleritomeconstruction, Balthasar2009Homologousskeletal, Wright1996Areview, Madison2017Laminarshell, Dewing2001Hingemodifications, Zhang2007Rhynchonelliformeanbrachiopods, Rowell1985Theevolutionary, Zhang2014Anearly, Laurie1986Phosphaticfauna, Topper2013Theoldest, Hu2010Softpart, Williams1998Thediversity, Kruse1990, Ushatinskaya2016Protegulumand, Williams2000BrachiopodaLinguliformea, Robinson2014Themuscles, Williams2006Rhynchonelliformeapart, Ruppert1993, Chen2007Reinterpretationof, Holmer1997EarlyCambrian, Williams1996ASupra, Barczyk1973BraehiopodsTerebratulina, Ackerly1992Rapidshell, Conklin1902Theembryology, Chapman1914IVNotes, Williams2002Shellstructure, Harper2017Brachiopodsorigin, Hanken1985Thetaxonomy, Benedetto2009iChaniella, Topper2013Reappraisalof, Zhang2016Epithelialcell, Butler2012ConstructingCambrian, Zhang2007Agregarious, Jin1992Revisionof, Williams1998Chemicostructural, Watkins2002Newrecord, Emig1992Functionaldisposition, Elliott1939Noteon, Skovsted2010EarlyCambrian, Robson2001Cambrianand, Lee1986iNeocrania, Dzik1980Ontogenyof, Cooper1976LowerCambrian, Hou2017Brachiopoda, Carlson1995Phylogeneticrelationships, Ruppert2004, Zhang2004Newdata, Zhang2004Softtissue, Williams1989Biomineralizationin, Freeman1999Changesin, Williams2007PartH, Gorjansky1986Onthe, Balthasar2011Relicaragonite, Skovsted2003EarlyCambrian, Balthasar2009Thebrachiopod, Balthasar2007Anearly, Kuzmina2007Structureof}

\hypertarget{tnt}{%
\chapter{Fitch parsimony}\label{tnt}}

Parsimony search was conducted in TNT v1.5 \citep{Goloboff2016} using
sectorial and ratchet heuristics \citep{Goloboff1999, Nixon1999} under
equal and implied weights \citep{Goloboff1997}. We acknowledge the Willi
Hennig Society for their sponsorship of the TNT software.

\hypertarget{implied-weights}{%
\section{Implied weights}\label{implied-weights}}

The consensus of all implied weights runs is not very well resolved:

\includegraphics{Brachiopod_phylogeny_files/figure-latex/unnamed-chunk-7-1.pdf}
\includegraphics{Brachiopod_phylogeny_files/figure-latex/unnamed-chunk-8-1.pdf}

\includegraphics{Brachiopod_phylogeny_files/figure-latex/unnamed-chunk-9-1.pdf}

\includegraphics{Brachiopod_phylogeny_files/figure-latex/unnamed-chunk-10-1.pdf}

This lack of resolution is largely a product of a few wildcard taxa,
which obscure relationships that are nevertheless present in all most
parsimonious trees:

\hypertarget{paterinids-included}{%
\subsection{Paterinids included}\label{paterinids-included}}

\begin{figure}
\centering
\includegraphics{Brachiopod_phylogeny_files/figure-latex/unnamed-chunk-11-1.pdf}
\caption{\label{fig:unnamed-chunk-11}TNT implied weights consensus}
\end{figure}

\hypertarget{paterinids-excluded}{%
\subsection{Paterinids excluded}\label{paterinids-excluded}}

\begin{figure}
\centering
\includegraphics{Brachiopod_phylogeny_files/figure-latex/unnamed-chunk-12-1.pdf}
\caption{\label{fig:unnamed-chunk-12}TNT implied weights consensus}
\end{figure}

\hypertarget{equal-weights}{%
\section{Equal weights}\label{equal-weights}}

\begin{figure}
\centering
\includegraphics{Brachiopod_phylogeny_files/figure-latex/unnamed-chunk-13-1.pdf}
\caption{\label{fig:unnamed-chunk-13}TNT Equal weights consensus}
\end{figure}

\hypertarget{bayesian-analysis}{%
\chapter{Bayesian analysis}\label{bayesian-analysis}}

Bayesian search was conducted in MrBayes v3.2.6 \citep{Ronquist2012}
using the Mk model \citep{Lewis2001} with a gamma parameter:

\begin{quote}
lset coding=variable rates=gamma;
\end{quote}

Branch length was drawn from a dirichlet prior distribution, which is
less informative than an exponential model \citep{Rannala2012}, but
requires a prior mean tree length within about two orders of magnitude
of the true value \citep{Zhang2012}. To satisfy this latter criterion,
we specified the prior mean tree length to be equal to the length of the
most parsimonious tree under equal weights, usinga Dirichlet prior with
\(α_T = 1\), \(β_T = 1/\)(\emph{equal weights tree length} /
\emph{number of characters}), \(α = c = 1\):

\begin{quote}
prset brlenspr = unconstrained: gammadir(1, 0.33, 1, 1);
\end{quote}

Neomorphic and transformational characters \citep[sensu][]{Sereno2007}
were allocated to two separate partitions whose proportion of invariant
characters and gamma shape parameters were allowed to vary
independently:

\begin{quote}
charset Neomorphic = 2 3 5 10 13 14 18 19 24 25 29 30 32 34 36 37 38 39
40 43 50 51 53 54 55 60 63 64 67 70 72 73 76 77 78 79 84 88 89 92 94 95;

charset Transformational = 1 4 6 7 8 9 11 12 15 16 17 20 21 22 23 26 27
28 31 33 35 41 42 44 45 46 47 48 49 52 56 57 58 59 61 62 65 66 68 69 71
74 75 80 81 82 83 85 86 87 90 91 93;

partition chartype = 2: Neomorphic, Transformational;

set partition = chartype;

unlink shape=(all) pinvar=(all);
\end{quote}

Neomorphic characters were not assumed to have a symmetrical transition
rate -- that is, the probability of the absent → present transition was
allowed to differ from that of the present → absent transition, being
drawn from a uniform prior:

\begin{quote}
prset applyto=(1) symdirihyperpr=fixed(1.0);
\end{quote}

Four MrBayes runs were executed, each sampling eight chains for 1 000
000 generations, with samples taken every 500 generations:

\begin{quote}
mcmcp ngen=1000000 samplefreq=500 nruns=2 nchains=8;
\end{quote}

The first 10\% of samples were discarded as burn-in
(\texttt{burninfrac=0.1}), and a posterior tree topology was derived
from the combined posterior sample of both runs.

\begin{figure}
\centering
\includegraphics{Brachiopod_phylogeny_files/figure-latex/unnamed-chunk-15-1.pdf}
\caption{\label{fig:unnamed-chunk-15}Bayesian analysis, posterior
probability \textgreater{} 50\%}
\end{figure}

\begin{figure}
\centering
\includegraphics{Brachiopod_phylogeny_files/figure-latex/unnamed-chunk-16-1.pdf}
\caption{\label{fig:unnamed-chunk-16}Bayesian analysis, posterior
probability \textgreater{} 50\%}
\end{figure}

Convergence was indicated by PSRF = 1.00 and an average estimated sample
size of \textgreater{} 500 for each parameter:

\begin{table}

\caption{(\#tab:MrBayes parameter summary)MrBayes parameter estimates (.pstat file)}
\centering
\begin{tabular}[t]{l|r|r|r|r|r|r|r|r}
\hline
Parameter & Mean & Variance & Lower & Upper & Median & minESS & avgESS & PSRF\\
\hline
TL\{all\} & 6.708892 & 1.624940 & 4.8329690 & 9.249889 & 6.781895 & 9.742945 & 877.9326 & 1.022541\\
\hline
alpha\{1\} & 2.403829 & 1.536554 & 0.0001827 & 4.702963 & 2.194241 & 23.533840 & 1038.9600 & 1.005475\\
\hline
alpha\{2\} & 2.950148 & 2.036612 & 0.0022876 & 5.567708 & 2.770201 & 22.238850 & 1073.5860 & 1.004902\\
\hline
\end{tabular}
\end{table}

It's interesting to note that the clade of hyoliths + lingulellotretids
(+ relatives) is resolved as a grade under Bayesian analysis.

In parsimony analysis, these taxa are always resolved as a clade when
inapplicable data is correctly handled; they instead resolve as a grade
under certain conditions under the \protect\hyperlink{TNT}{standard
Fitch} algorithm (which mishandles inapplicable data).

We suggest that the failure of Bayesian analysis to recover this group
as a clade may reflect inappropriate handling of inapplicable data in
MrBayes, though at present (and until the algorithms used in a
likelihood context are improved) it is difficult to test this
hypothesis.

The same goes for the position of \emph{Gasconsia}, which is widely held
to have an affinity with the craniid brachiopods, and is recovered in
such a position using the inapplicable-safe parsimony algorithm (but not
always when the standard Fitch parsimony algorithm is used).

\bibliography{References.bib,MorphoBank.bib}


\end{document}
